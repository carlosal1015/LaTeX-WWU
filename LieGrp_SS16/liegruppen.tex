%!TEX TS-program = xelatex
%!TEX TS-options = -shell-escape
%!TEX root = ../LieGrp_SS16/liegruppen.tex
\RequirePackage{fix-cm} 
\documentclass[a4paper, twoside, headsepline, index=totoc,toc=listof,toc=bibliography,toc=index, fontsize=10pt, cleardoublepage=empty, headinclude, DIV=12, BCOR=5mm, titlepage,draft]{scrreprt}
%!TEX root = ../AnaTopGeo_SS14/ana_top_geo.tex
\usepackage{scrtime} % KOMA, Uhrzeit ermoeglicht

%--Pakete zum "Programmieren"
% ======================================================================================
\usepackage{etoolbox}
\usepackage{letltxmacro}
\usepackage{ifthen}
% ======================================================================================

%--Farbdefinitionen und Grafiken (muss vor tikz geladen werden)
% ======================================================================================
\usepackage[usenames, table, x11names]{xcolor}
\definecolor{dark_gray}{gray}{0.45}
\definecolor{light_gray}{gray}{0.6}
\definecolor{fb10_blue}{cmyk}{0.8,0.4,0.13,0.07}
\usepackage[final]{graphicx}
\usepackage{adjustbox}
\newcommand{\cfbox}[2]{% coloured frame box
	\ifmmode
	\mathchoice{\adjustbox{cfbox=#1}{$\displaystyle#2$}}{\adjustbox{cfbox=#1}{$\textstyle#2$}}{\adjustbox{cfbox=#1}{$\scriptstyle#2$}}{\adjustbox{cfbox=#1}{$\scriptscriptstyle#2$}}
	\else
	\adjustbox{cfbox=#1}{#2}
	\fi
}
% ======================================================================================

%--Zum Zeichnen/ TikZ-Kram (vor polyglossia bzw. babel geladen werden)
% ======================================================================================
\usepackage{tikz}
\usepackage{tikz-cd}
\usetikzlibrary{external}
\tikzset{>=latex}
\usetikzlibrary{%
	shapes,
	arrows.meta,
	intersections,
	calc,
	3d,
	decorations.pathreplacing,decorations.markings,decorations.pathmorphing,
	angles,
	quotes,
}
\tikzexternalize[prefix=tikz/,up to date check=diff]
\pgfkeys{/pgf/images/include external/.code=\includegraphics{#1}}
\tikzset{external/system call={lualatex \tikzexternalcheckshellescape -halt-on-error -interaction=batchmode --shell-escape -jobname "\image" "\texsource"}}
\AtBeginEnvironment{tikzcd}{\tikzexternaldisable} % tikzexternalize fuer tikzcd deaktivieren, da inkompatibel
\AtEndEnvironment{tikzcd}{\tikzexternalenable}
\tikzset{% um Inkompatibilitaeten von quotes und polyglossia bzw. babel zu vermeiden
  every picture/.append style={
    execute at begin picture={\shorthandoff{"}},
    execute at end picture={\shorthandon{"}}
  }
}
\usepackage{pgfplots}
\usepgfplotslibrary{colormaps}
\newcommand*\circled[1]{\tikzexternaldisable\tikz[baseline=(char.base)]{\node[shape=circle,draw,inner sep=2pt] (char) {#1};}\tikzexternalenable}
% ======================================================================================



%-- Mathepakete etc.
% ======================================================================================
\usepackage[T1]{fontenc}
\renewcommand{\rmdefault}{zpltlf}
\usepackage{mathtools} % beinhaltet amsmath
\mathtoolsset{showonlyrefs,centercolon,showmanualtags}
\newtagform{brackets}[\textbf]{[}{]}
\usetagform{brackets}
\usepackage{fix-cm}
\usepackage[bbgreekl]{mathbbol}
\usepackage{amssymb,marvosym} 
\usepackage{nicefrac} % schräge Brüche
\usepackage{faktor}
\newcommand{\Faktor}[1]{\faktor[\textstyle]{#1}}
\usepackage{xfrac}
\usepackage{cancel}
\usepackage{mathdots} % Verbesserung von Punkten wie zB \ldots
\usepackage[bb=px]{mathalfa} % \mathbb als px font
\usepackage{centernot}
\usepackage{stackrel}
\DeclareSymbolFont{bbold}{U}{bbold}{m}{n}
\DeclareSymbolFontAlphabet{\mathbbold}{bbold}
\newcommand{\ind}{\mathbbold{1}} % charakteristische-Funktion-Eins
\def\mathul#1#2{\color{#1}\underline{{\color{black}#2}}\color{black}} %farbiges Untersteichen im Mathe-Modus
\renewcommand{\le}{\leqslant}
\renewcommand{\ge}{\geqslant}
% ======================================================================================


%-- Von xfrac erzeuge font warnings ignorieren
% ======================================================================================
\usepackage{silence}
\WarningFilter{latexfont}{Size substitutions with differences}
\WarningFilter{latexfont}{Font shape `U/bbold/m/n' in size}
% ======================================================================================


%-- Typographie/Polyglossia
% ======================================================================================
\usepackage[euler-digits]{eulervm} % vor fontspec laden!
\usepackage[no-math]{fontspec}
\usepackage{polyglossia} % moderner babel-ersatz
\setmainlanguage[spelling=new,babelshorthands=true]{german}
\shorthandoff{"}
\setotherlanguage{english}
\defaultfontfeatures{Mapping=tex-text, WordSpace={1.2}, Ligatures={Required,Common,Contextual},Extension=.otf} %


\setmainfont{TeXGyrePagellaX}[UprightFont=*-Regular,BoldFont=*-Bold,ItalicFont=*-Italic,BoldItalicFont=*-BoldItalic,ItalicFeatures={Style=Historic},Ligatures={Required,Common,Contextual,Historic}]
\setsansfont{texgyreadventor}[Scale=MatchUppercase, UprightFont=*-regular, BoldFont=*-bold, ItalicFont=*-italic, BoldItalicFont=*-bolditalic]
\setmonofont{SourceCodePro}[Scale=0.9,UprightFont=*-Regular, BoldFont=*-Semibold, ItalicFont=*-Light]
\usepackage{xltxtra}
\usepackage{fontawesome}
\usepackage[final]{microtype}
\usepackage[draft=false]{scrlayer-scrpage} 
\flushbottom
% ======================================================================================


%-- Aufzählungen
% ======================================================================================
\usepackage[shortlabels,inline]{enumitem}
\setlist[itemize,1]{label=\faCaretRight}
\setlist[enumerate]{font=\bfseries}
\setlist[description]{font=\normalfont\bfseries}
\usepackage{multicol}
% ======================================================================================


%-- Floats/Figures/Tabellen
% ======================================================================================
\usepackage{wrapfig}
\usepackage{float}
\usepackage[margin=10pt, font=small, labelfont={sf, bf}, format=plain, indention=1em]{caption}
\captionsetup[wrapfigure]{name=Abb. }
\usepackage{booktabs}
% ======================================================================================


%-- korrekte Anführungszeichen und Zitierbefehle
% ======================================================================================
\usepackage[autostyle,german=quotes,english=british]{csquotes}
% ======================================================================================


%--Indexverarbeitung
% ======================================================================================
\usepackage{makeidx}
\newcommand{\bet}[1]{\textbf{\emph{#1}}}
\newcommand{\Index}[1]{\bet{#1}\index{#1}}
\makeindex
\setindexpreamble{{\noindent\sffamily\small Die \emph{Seitenzahlen} sind mit Hyperlinks versehen und somit anklickbar} \par \bigskip}
\renewcommand{\indexpagestyle}{scrheadings}
% ======================================================================================


%-- Marginnotes/Todonotes/Footnotes
% ======================================================================================
\deffootnote[1.5em]{1.5em}{1.5em}{\textsuperscript{\thefootnotemark}\ }
\usepackage[fulladjust]{marginnote}
\renewcommand*{\marginfont}{\itshape\footnotesize}
\usepackage[textsize=small]{todonotes}
\usepackage{ragged2e}
\renewcommand*{\raggedleftmarginnote}{\RaggedLeft}
\renewcommand*{\raggedrightmarginnote}{\RaggedRight}
\LetLtxMacro{\oldtodo}{\todo}
\renewcommand{\todo}[2][]{\tikzexternaldisable\oldtodo[#1]{#2}\tikzexternalenable}
\LetLtxMacro{\oldmissingfigure}{\missingfigure}
\renewcommand{\missingfigure}[2][]{\tikzexternaldisable\oldmissingfigure[{#1}]{#2}\tikzexternalenable}
% ======================================================================================


% -- BibLaTeX
% ======================================================================================
\usepackage[%
	backend=biber,
	sortlocale=auto,
	natbib,
	hyperref,
	backref,
	style=alphabetic
	]%
{biblatex}
\renewcommand*{\mkbibnamelast}[1]{%
  \ifmknamesc{\textsc{#1}}{#1}}
\renewcommand*{\mkbibnameprefix}[1]{%
  \ifboolexpr{ test {\ifmknamesc} and test {\ifuseprefix} }
    {\textsc{#1}}
    {#1}}
\def\ifmknamesc{%
  \ifboolexpr{ test {\ifcurrentname{labelname}}
               or test {\ifcurrentname{author}}
               or ( test {\ifnameundef{author}} and test {\ifcurrentname{editor}} ) }}
\addbibresource{../!config/quellen.bib}
% ======================================================================================

%--Konfiguration von Hyperref und Cleveref
% ======================================================================================
\usepackage[hidelinks, pdfpagelabels,  bookmarksopen=true, bookmarksnumbered=true, linkcolor=black, urlcolor=SkyBlue2, plainpages=false,pagebackref, citecolor=black, hypertexnames=true, pdfauthor={Jannes Bantje}, pdfborderstyle={/S/U}, linkbordercolor=SkyBlue2, colorlinks=false,final,backref=false]{hyperref}
\usepackage[nameinlink,noabbrev]{cleveref}
\newcommand{\appendLink}[1]{#1\,\faExternalLink}
\newcommand{\hrefsym}[2]{\href{#1}{\texttt{\appendLink{#2}}}}
\newcommand{\hrefsymX}[2]{\href{#1}{\appendLink{#2}}}
\newcommand{\hrefsymmail}[2]{\href{#1}{\texttt{\faEnvelopeO\,#2}}}
\renewcommand{\url}[1]{\hrefsym{#1}{\nolinkurl{#1}}}
% ======================================================================================


% -- QR-Codes (hinter hyperref laden!)
% ======================================================================================
\usepackage{qrcode}
% ======================================================================================

%--Römische Zahlen
% ======================================================================================
\newcommand{\RM}[1]{\MakeUppercase{\romannumeral #1{}}}
% ======================================================================================

%-- Definition von diversen Mathe-Befehlen
% ======================================================================================
%!TEX root = mitschrift_main.tex

% -- Zum Finetuning von Befehlen
% ======================================================================================
\makeatletter
\newcommand{\raisemath}[1]{\mathpalette{\raisem@th{#1}}}
\newcommand{\raisem@th}[3]{\raisebox{#1}{$#2#3$}}
\makeatother
\makeatletter
\newcommand{\killDescendersM}[1]{\mathpalette{\killD@scendersM{#1}}}
\newcommand{\killD@scendersM}[2]{\raisebox{0pt}[\height][0pt]{$#2#1$}}
\makeatother
\DeclareRobustCommand{\minwidthbox}[2]{%
  \ifmmode
    \expandafter\mathmakebox
  \else
    \expandafter\makebox
  \fi
  [\ifdim#2<\width\width\else#2\fi]{#1}%
}
% ======================================================================================


%-- Klammerbefehle
% ======================================================================================
\DeclarePairedDelimiter{\abs}{\lvert}{\rvert}
\DeclarePairedDelimiter{\floor}{\lfloor}{\rfloor}
\DeclarePairedDelimiter{\ceil}{\lceil}{\rceil}
\DeclarePairedDelimiter\norm{\Vert}{\Vert}
\DeclarePairedDelimiter\enbrace{(}{)}
\DeclarePairedDelimiter\benbrace{[}{]}
\DeclarePairedDelimiter\bbenbrace{[\![}{]\!]}
\DeclarePairedDelimiter\lenbrace{<}{>}
\DeclarePairedDelimiter\angbrace{\langle}{\rangle}
\newcommand{\ssbrace}[1]{{\scriptscriptstyle\enbrace{#1}}}
\newcommand{\ssbbrace}[1]{{\scriptscriptstyle\benbrace{#1}}}
% ======================================================================================

%-- Mengen
% ======================================================================================
\newcommand\SetSymbol[1][]{\nonscript\:#1\vert\allowbreak\nonscript\:\mathopen{}}
\providecommand\given{} % to make it exist
\DeclarePairedDelimiterX\set[1]\{\}{\renewcommand\given{\SetSymbol[\delimsize]}#1}
% ======================================================================================

%-- Skalarprodukt (3 Varianten) 
% ======================================================================================
\DeclarePairedDelimiterX\sprod[2]{\langle}{\rangle}{#1\,\delimsize\vert\,#2}
\DeclarePairedDelimiterX\skal[2]{\langle}{\rangle}{#1\,,\,#2}
\makeatletter
\DeclareFontFamily{OMX}{MnSymbolE}{}
\DeclareSymbolFont{MnLargeSymbols}{OMX}{MnSymbolE}{m}{n}
\SetSymbolFont{MnLargeSymbols}{bold}{OMX}{MnSymbolE}{b}{n}
\DeclareFontShape{OMX}{MnSymbolE}{m}{n}{
    <-6>  MnSymbolE5
   <6-7>  MnSymbolE6
   <7-8>  MnSymbolE7
   <8-9>  MnSymbolE8
   <9-10> MnSymbolE9
  <10-12> MnSymbolE10
  <12->   MnSymbolE12
}{}
\DeclareFontShape{OMX}{MnSymbolE}{b}{n}{
    <-6>  MnSymbolE-Bold5
   <6-7>  MnSymbolE-Bold6
   <7-8>  MnSymbolE-Bold7
   <8-9>  MnSymbolE-Bold8
   <9-10> MnSymbolE-Bold9
  <10-12> MnSymbolE-Bold10
  <12->   MnSymbolE-Bold12
}{}
\let\llangle\@undefined
\let\rrangle\@undefined
\DeclareMathDelimiter{\llangle}{\mathopen}%
                     {MnLargeSymbols}{'164}{MnLargeSymbols}{'164}
\DeclareMathDelimiter{\rrangle}{\mathclose}%
                     {MnLargeSymbols}{'171}{MnLargeSymbols}{'171}
\makeatother
\DeclarePairedDelimiterX\sskal[2]{\llangle}{\rrangle}{#1\,,\,#2}
% ======================================================================================

%-- Abbildungsdefinition
% ======================================================================================
\newcommand{\mapdef}[5]{%
	\[
		\begin{array}{rcl}
			\textstyle #1 &\xrightarrow{\minwidthbox{#5}{2em}} & \textstyle #2 \\[0.5ex]
			\textstyle #3 &\xmapsto{\minwidthbox{\mbox{ }}{2em}} & \textstyle #4
		\end{array}
	\]
}
% ======================================================================================

%-- modifiziertes Stackrel 
% ======================================================================================
\newcommand{\StackText}[2]{\stackrel{\mbox{\scriptsize #1}}{#2}}
\newcommand{\StackTextClap}[2]{\stackrel{\mathclap{\mbox{\scriptsize #1}}}{#2}}
% ======================================================================================

%-- Blitz
% ======================================================================================
\newcommand{\light}{\text{\raisebox{-.3ex}{\Large\Lightning}}}
% ======================================================================================


%-- Underbrace u.Ä. als Befehl in LaTeX-Syntax (und ohne Spacingprobleme mit nachfolgenden Operatoren...)
% ======================================================================================
\newcommand{\Underbrace}[2]{{\underbrace{#1}_{#2}}}
\newcommand{\Underbracket}[2]{{\underbracket[0.7pt][2pt]{#1}_{#2}}}
\newcommand{\Overbracket}[2]{{\overbracket[0.7pt][2pt]{#1}^{#2}}}
% ======================================================================================


%-- Deklaration weiterer Operatoren (allgemein)
% ======================================================================================
\DeclareMathOperator{\re}{Re} % Realteil
\let\Re\relax
\DeclareMathOperator{\Re}{Re} % Realteil
\DeclareMathOperator{\im}{im} % Bild
\let\Im\relax
\DeclareMathOperator{\Im}{Im} % Bild
\DeclareMathOperator{\id}{id} % identische Abbildung
\DeclareMathOperator{\conj}{conj} % Konjugation
\DeclareMathOperator{\sgn}{sgn} % Signum
\DeclareMathOperator{\End}{End} % Endomorphismen
\DeclareMathOperator{\Hom}{Hom} % Homomorphismen
\DeclareMathOperator{\Iso}{Iso} % Isomorphismen
\DeclareMathOperator{\Aut}{Aut} % Automorphismen
\DeclareMathOperator{\Span}{span} % Span
\DeclareMathOperator{\coker}{coker} % Kokern
\DeclareMathOperator{\Tr}{Tr} % Spur,Trace
\DeclareMathOperator{\pr}{pr} % Projektion
\DeclareMathOperator{\diag}{diag} % Diagonalmatrix
\DeclareMathOperator{\Rg}{Rg} % Rang
\DeclareMathOperator{\const}{const} % konstante Abbildung
\DeclareMathOperator{\Spur}{Spur} % Spur
\DeclareMathOperator{\Arg}{Arg} % Argument
\DeclareMathOperator{\dist}{dist} % Distanz
\DeclareMathOperator{\supp}{supp} % Träger
\DeclareMathOperator{\Char}{char} % Charakteristik
% ======================================================================================


%-- Deklaration weiterer Operatoren (Differentiale etc.)
% ======================================================================================
\DeclareMathOperator{\grad}{grad} % Gradient
\DeclareMathOperator{\dive}{div} % Gradient
\DeclareMathOperator{\rot}{rot} % Rotation
\newcommand{\D}{\ensuremath{\mathrm{D}\mkern-1.0mu}} % Differential
\newcommand{\mathd}{\ensuremath{\mathrm{d}\mkern-1.0mu}} % äußere Ableitung
\newcommand{\Tmap}{\ensuremath{\mathrm{T}\mkern-0.85mu}} % Tangentialraum
\let\Tang\Tmap
\DeclareMathOperator{\Diff}{Diff}
\newcommand{\diff}[2]{\ensuremath{\frac{{\partial #1}}{{\partial #2}} }}
\newcommand{\diffd}[2]{\ensuremath{\frac{\mathd #1}{\mathd #2} }}
\DeclareMathOperator{\rank}{rank}
% ======================================================================================


%-- Deklaration weiterer Operatoren (Topologie)
% ======================================================================================
\newcommand*\interior[1]{\overset{\smash{\raisebox{-0.18ex}{$\scriptstyle\circ$}}}{#1}}
\newcommand{\sing}{{\raisemath{1.1pt}{\scriptscriptstyle\mathrm{sing}}}}
\newcommand{\pt}{\mathrm{pt}}
\DeclareMathOperator{\Zyl}{Zyl}
\newcommand{\rZyl}{\widetilde{\Zyl}}
\DeclareMathOperator{\Tel}{Tel}
\newcommand{\op}{\mathrm{op}}
\DeclareMathOperator{\Sp}{Sp}
\DeclareMathOperator{\Keg}{Keg}
\newcommand{\slashedi}{i\hspace{-3.5pt}/}
\newcommand{\cupp}{\smallsmile}
\newcommand{\capp}{\smallfrown}
\DeclareMathOperator*{\colim}{colim}
\DeclareMathOperator{\PD}{PD}
\newcommand{\lf}{\mathrm{lf}}
\DeclareMathOperator{\sig}{sig}
\DeclareMathOperator{\Tor}{Tor}
\DeclareMathOperator{\Ext}{Ext}
\DeclareMathOperator{\AW}{AW}
\DeclareMathOperator{\Proj}{Proj}
\DeclareMathOperator{\Gr}{Gr}
\DeclareMathOperator{\res}{res}
\DeclareMathOperator{\Spec}{Spec}
\DeclareMathOperator{\co}{co}
\DeclareMathOperator{\ch}{ch}
\DeclareMathOperator{\wOp}{w}
\DeclareMathOperator{\Ar}{Ar}
\newcommand{\actson}{\mathrel{\curvearrowright}}
\let\acts\actson
\let\action\actson
\DeclareMathSymbol{\bbDelta}{\mathord}{bbold}{"01}
\newcommand{\DDelta}{\bbDelta}
\DeclareMathOperator{\Star}{Star}
\DeclareMathOperator{\Link}{Link}
\DeclareMathOperator{\EPK}{EPK}
\DeclareMathOperator{\Vol}{Vol}
\newcommand{\cell}{{\raisemath{1.1pt}{\scriptscriptstyle\mathrm{cell}}}}
\DeclarePairedDelimiter{\homologieklasse}{\llbracket}{\rrbracket}
\newcommand{\rand}[1]{\ensuremath{\partial^{\scriptscriptstyle #1}}}
\DeclareMathOperator{\ab}{ab}
\DeclareMathOperator{\CW}{CW}
% ======================================================================================


%-- Deklaration von Operatoren (Liegruppen)
% ======================================================================================
\DeclareMathOperator{\GL}{GL}
\DeclareMathOperator{\SO}{SO}
\DeclareMathOperator{\Ad}{Ad}
\DeclareMathOperator{\ad}{ad}
\DeclareMathOperator{\On}{O}
\DeclareMathOperator{\Un}{U}
\DeclareMathOperator{\SU}{SU}
\DeclareMathOperator{\Mat}{Mat}
\DeclareRobustCommand{\Der}{\mathop{\mathfrak{der}}}
\DeclareMathOperator{\SL}{SL}
\DeclareMathOperator{\Graph}{Graph}
\DeclareMathOperator{\Int}{Int}
\DeclareRobustCommand{\intAlg}{\mathop{\mathfrak{int}}}
\DeclareMathOperator{\aut}{aut}
\DeclareMathOperator{\Rad}{Rad}
\DeclareMathOperator{\Nil}{Nil}
\DeclareMathOperator{\rad}{rad}
\DeclareMathOperator{\nil}{nil}
\DeclareMathOperator{\Ric}{Ric}
\DeclareMathOperator{\ric}{ric}
\newcommand{\bi}{\mathrm{bi}}
\DeclareMathOperator{\Isom}{Isom}
\DeclareMathOperator{\Sym}{Sym}
\newcommand{\opL}{\ensuremath{\mathrm{L}\mkern-0.6mu}}
% ======================================================================================

%-- Deklaration von Operatoren (Funktionalanalysis)
% ======================================================================================
\DeclareMathOperator{\tr}{tr}
\newcommand{\w}{\mkern1mu\mathrm{w}}
\newcommand{\sa}{\mathrm{sa}}
\newcommand{\vb}{\mathrm{v\mkern-2.5mu.b\mkern-1.5mu.}} % vollständig beschränkt
\newcommand{\so}{\mathrm{\mkern.3mu s\mkern-1.4mu.\mkern-.6mu o\mkern-1.7mu.}} % \newcommand{\so}{\mathrm{s.o.}}
\newcommand{\solim}{\so\text{-}\mkern-0.8mu\lim}
\newcommand{\wo}{\mathrm{w\mkern-3mu.\mkern-.4mu o\mkern-1.7mu.}}
\newcommand{\Top}[1]{\mathcal{T}_{\mkern-2.3mu #1}}
\newcommand{\weakT}[1]{\ensuremath{\mathcal{T}_{#1}^{\mkern+1.0mu\text{\raisebox{0.4ex}{$\mathrm{w}$}}}}}
\newcommand{\weakTstar}[1]{\ensuremath{\mathcal{T}_{#1}^{\mkern+1.0mu\text{\raisebox{0.4ex}{$\mathrm{w}$}}^*}}}
\newcommand{\TWeakStar}{\Top{\w^*}}
\newcommand{\TWeakOp}{\Top{\wo}}
\newcommand{\Tso}{\Top{\so}}
\newcommand{\finSub}{\subset\mkern-0.7mu \subset}
\DeclareMathOperator{\Inv}{Inv}
\newcommand{\simm}{{\hspace{-1.6pt}\raisemath{0.5pt}{\sim}}}
\newcommand{\plus}{{\hspace{-1.6pt}+}}
\DeclareMathOperator{\ev}{ev}
\DeclareMathOperator{\Alg}{Alg}
\DeclareMathOperator{\her}{her}
\newcommand{\subher}{\subset_{\her}}
\newcommand{\grenzw}[1]{\xrightarrow{\minwidthbox{#1}{1.4em}}}
\newcommand{\grenzwl}[1]{\xleftarrow{\minwidthbox{#1}{1.4em}}}
\newcommand{\grenzwIn}[1]{\grenzw{\raisemath{-2pt}{#1}}}
\newcommand{\MyTo}[1]{\tikzexternaldisable\mathbin{\tikz[baseline] \draw[-to,line width=.4pt] (0ex,0.94ex) -- (#1,0.94ex);}\tikzexternalenable}
\newcommand{\dlim}{%
    \mathchoice
      {\lim\limits_{\MyTo{4.2ex}}}% \displaystyle
      {\lim\limits_{\MyTo{2.8ex}}}% \textstyle
      {\lim\limits_{\MyTo{2.3ex}}}% \scriptstyle
      {\lim\limits_{\MyTo{2.3ex}}}% \scriptscriptstyle
}
\newcommand{\Dlim}{\killDescendersM{\dlim}}
\DeclareMathOperator{\sep}{sep}
\DeclareMathOperator{\diam}{diam}
\DeclareMathOperator{\conv}{conv}
\DeclareMathOperator{\Prim}{Prim}
\DeclareMathOperator{\hull}{hull}
\DeclareMathOperator{\red}{red}
\DeclarePairedDelimiterX\bra[1]{\langle}{\rvert}{#1\,}
\DeclarePairedDelimiterX\ket[1]{\lvert}{\rangle}{\,#1}
\DeclarePairedDelimiterX\bracket[2]{\langle}{\rangle}{#1\,\delimsize\vert\,#2}
\newcommand{\tensormax}{\mathbin{\otimes_{\max}}}
\newcommand{\tensormin}{\mathbin{\otimes_{\min}}}
\DeclareMathOperator{\Ped}{Ped}
\newcommand{\alg}{\mathrm{alg}}
\DeclareMathOperator{\CPC}{CPC}
\DeclareMathOperator{\CP}{CP}
\DeclareMathOperator{\UPC}{UPC}
\newcommand{\DeltaOp}{\mathbin{\Delta}}
\newcommand{\kernedP}{\mathcal{P}\mkern-2mu}
\newcommand{\Pinfty}{\kernedP_{\infty}}
\DeclareMathOperator{\Groth}{Groth}
\DeclareMathOperator{\rk}{rk}
\newcommand{\MvN}{\mathrm{MvN}}
% ======================================================================================

%-- Kategorien
% ======================================================================================
\DeclareMathOperator{\Mor}{Mor}
\DeclareMathOperator{\mor}{mor}
\DeclareMathOperator{\Obj}{Obj}
\DeclareMathOperator{\Ob}{Ob}
\newcommand{\TOP}{\textsc{Top}}
\newcommand{\HTOP}{\textsc{HTop}}
\newcommand{\VR}{\textsc{VR}}
\newcommand{\MOD}{\textsc{Mod}}
\newcommand{\Mod}[1]{#1\text{-}\MOD}
\newcommand{\MONOIDE}{\textsc{Monoide}}
\newcommand{\SET}{\textsc{Set}}
\newcommand{\MAN}{\textsc{Man}}
\newcommand{\GRUPPEN}{\textsc{Gruppen}}
\newcommand{\ABELGRUPPEN}{\textsc{Abel.Gruppen}}
\newcommand{\ABEL}{\textsc{Abel}}
\newcommand{\KAT}{\textsc{Kat}}
\newcommand{\FUN}{\textsc{Fun}}
\newcommand{\SIMP}{\textsc{Simp}}
\newcommand{\VEKT}{\textsc{Vekt}}
\newcommand{\CH}{\textsc{Ch}}
\newcommand{\CSTARUN}{C^*\text{-}\textsc{Alg}^{\raisemath{-2.5pt}{1}}}
\newcommand{\CSTAR}{C^*\text{-}\textsc{Alg}}
\newcommand{\AB}{\textsc{Ab}}
% ======================================================================================
% ======================================================================================



% -- theorem packages
% ======================================================================================
\usepackage{amsthm}
\usepackage{thmtools,thm-restate}
\usepackage{mdframed}
\renewcommand{\listtheoremname}{Übersicht aller Aussagen}
\usepackage{bookmark}
\bookmarksetup{open,numbered}
\makeatletter
\newcommand*{\theorembookmark}{%
  \bookmark[
    dest=\@currentHref,
    rellevel=1,
    keeplevel,
  ]{%
    \thmt@thmname\space\csname the\thmt@envname\endcsname
    \ifx\thmt@shortoptarg\@empty
    \else
      \space(\thmt@shortoptarg)%
    \fi
  }%
}   
\makeatother
% ======================================================================================

% -- Definition der einzelnen Theorem-Umgebungen
% ======================================================================================
\declaretheoremstyle[%
	headfont=\sffamily\bfseries,
	notefont=\normalfont\sffamily\scshape,
	bodyfont=\normalfont,
	headformat=\NUMBER\ \NAME\NOTE,
	headpunct=.,
	postheadspace=1em,
	spaceabove=15pt,spacebelow=10pt,
	shaded={bgcolor=gray!20},
	postheadhook=\theorembookmark]%
{mainstyle}
\declaretheoremstyle[%
	headfont=\sffamily\bfseries,
	notefont=\normalfont\sffamily\scshape,
	bodyfont=\normalfont,
	headformat=\NUMBER\ \NAME\NOTE,
	headpunct=.,
	postheadspace=1em,
	spaceabove=15pt,spacebelow=10pt,
	shaded={bgcolor=fb10_blue!20},
	postheadhook=\theorembookmark]%
{mainstyle_blue}
\declaretheoremstyle[%
	headfont=\sffamily\bfseries,
	notefont=\normalfont\sffamily\scshape,
	bodyfont=\normalfont,
	headformat=\NUMBER\ \NAME\NOTE,
	headpunct=.,
	postheadspace=1em,
	spaceabove=15pt,spacebelow=10pt,
	postheadhook=\theorembookmark]%
{mainstyle_unshaded}
\declaretheoremstyle[%
	headfont=\sffamily\bfseries,
	notefont=\normalfont\sffamily\scshape,
	bodyfont=\normalfont,
	headformat=\NUMBER\NAME\NOTE,
	headpunct=.,
	postheadspace=1em,
	spaceabove=15pt,spacebelow=10pt,
	% shaded={bgcolor=gray!20},
	postheadhook=\theorembookmark]%
{mainstyle_unnumbered}
\declaretheoremstyle[%
	headfont=\sffamily\bfseries,
	notefont=\normalfont\sffamily\scshape,
	bodyfont=\normalfont,
	headformat=swapnumber,
	headpunct=.,
	postheadspace=1em,
	spaceabove=15pt,spacebelow=10pt,
	shaded={bgcolor=gray!20},
	postheadhook=\theorembookmark,
	qed=\qedsymbol]%
{mainstyleB}
\declaretheoremstyle[%
	headfont=\bfseries\scshape,
	bodyfont=\normalfont,
	headpunct=:,
	postheadspace=1em,
	spacebelow=12pt,spaceabove=2pt,
	qed=\qedsymbol]%
{beweise}
\declaretheoremstyle[%
	headfont=\bfseries\scshape,
	bodyfont=\normalfont,
	headpunct=:,
	postheadspace=1em,
	spacebelow=12pt,spaceabove=2pt]%
{beweisskizze}
\declaretheoremstyle[%
	headfont=\sffamily\bfseries,
	bodyfont=\normalfont,
	headpunct=:,
	postheadspace=1em,
	spacebelow=10pt,spaceabove=10pt]%
{bemerkungen}
\declaretheorem[name=Definition,parent=section,style=mainstyle_blue]{definition}
\declaretheorem[name=Definition \& Proposition,refname=Proposition,sharenumber=definition,style=mainstyle_blue]{definitionP}
\declaretheorem[name=Definition,numbered=no,style=mainstyle_unnumbered]{definition*}
\declaretheorem[name=Theorem,sharenumber=definition,style=mainstyle]{theorem}
\declaretheorem[name=Theorem,numbered=no,style=mainstyle_unnumbered]{theorem*}
\declaretheorem[name=Proposition,sharenumber=definition,style=mainstyle,refname=Proposition]{proposition}
\declaretheorem[name=Lemma,sharenumber=definition,style=mainstyle]{lemma}
\declaretheorem[name=Satz,sharenumber=definition,style=mainstyle,refname=Satz]{satz}
\declaretheorem[name=Satz,sharenumber=definition,style=mainstyle_unshaded]{satzUnshaded}
\declaretheorem[name=Definition,sharenumber=definition,style=mainstyle_unshaded]{definitionUnshaded}
\declaretheorem[name=Satz,numbered=no,style=mainstyle_unnumbered]{satz*}
\declaretheorem[name=Korollar,sharenumber=definition,style=mainstyle,refname=Korollar]{korollar}
\declaretheorem[name=Korollar,sharenumber=definition,style=mainstyleB,refname=Korollar]{korollarB}
\declaretheorem[name=Frage,numbered=no,style=mainstyle_unnumbered]{frage}
\declaretheorem[name=Frage,sharenumber=definition,style=mainstyle_unshaded]{frageA}
\declaretheorem[name=Erinnerung,sharenumber=definition,style=mainstyle_unshaded]{erinnerungA}
\declaretheorem[name=Ausblick,sharenumber=definition,style=mainstyle_unshaded]{ausblick}
\declaretheorem[name=Konvention,sharenumber=definition,style=mainstyle]{konvention}
\declaretheorem[name=Notation,sharenumber=definition,style=mainstyle_unshaded]{notation}
\declaretheorem[name=Bemerkung,sharenumber=definition,style=mainstyle_unshaded,refname=Bemerkung]{bemerkung}
\declaretheorem[name=Bemerkung,numbered=no,style=mainstyle_unnumbered]{bemerkung*}
\declaretheorem[name=Beispiel,sharenumber=definition,style=mainstyle_unshaded,refname=Beispiel]{beispiel}
\declaretheorem[name=Beispiel,numbered=no,style=mainstyle_unnumbered]{beispiel*}
\declaretheorem[name=Exkurs,numbered=no,style=mainstyle_unnumbered]{exkurs*}
\declaretheorem[name=Beweis,numbered=no,style=beweise]{beweis}
\declaretheorem[name=Übung,numbered=no,style=bemerkungen]{uebung}
\declaretheorem[name=Erinnerung,numbered=no,style=bemerkungen]{erinnerung}

% english versions
\declaretheorem[name=Remark,sharenumber=definition,style=mainstyle_unshaded]{remark}
\declaretheorem[name=Remark,numbered=no,style=mainstyle_unnumbered]{remark*}
\declaretheorem[name=Example,sharenumber=definition,style=mainstyle_unshaded]{example}
\declaretheorem[name=Corollary,sharenumber=definition,style=mainstyle]{corollary}
\let\proof\relax
\declaretheorem[name=Proof,numbered=no,style=beweise]{proof}
\declaretheorem[name=Sketch of Proof,numbered=no,style=beweisskizze]{sketch}
% ======================================================================================

%--Inhaltsverzeichnis
% ======================================================================================
\usepackage[tocindentauto]{tocstyle}
\usetocstyle{KOMAlike}
% ======================================================================================

%-- Dinge, die erst am Ende getan werden dürfen
% ======================================================================================
\shorthandon{"}
\usepackage{ellipsis}
% ======================================================================================


\newcommand{\fach}{Liegruppen}
\newcommand{\semester}{Sose 2016}
\newcommand{\homepage}{http://wwwmath.uni-muenster.de/42/arbeitsgruppen/ag-differentialgeometrie/prof-dr-christoph-boehm/vorlesung-liegruppen/}

\newcommand{\prof}{Prof.\ Dr.\ Christoph Böhm}
\publishers{\scalebox{10}{\Huge$\mathrm{Lie}$}}
\input{../!config/mitschrift_headings.tex}

\begin{document}
\pagenumbering{Roman}
\maketitle
\begin{abstract}
\section*{Aktuelle Version verfügbar bei}
\newcommand{\dieBreite}{11cm}
\begin{minipage}{4cm}
	\qrcode[height=3.3cm, version=6]{https://gitlab.com/JaMeZ-B/LaTeX-WWU}
\end{minipage}
\hfill
\begin{minipage}{\dieBreite}
	% \includegraphics[height=0.6cm, keepaspectratio]{../!config/Bilder/wm_no_bg.pdf}
	\includegraphics[height=0.8cm, keepaspectratio]{../!config/Bilder/wm_no_bg.pdf}\\
	\url{https://gitlab.com/JaMeZ-B/LaTeX-WWU} \smallskip\\
	Das zentrale Repository des \enquote{\LaTeX-WWU}-Projekts befindet sich auf der Plattform GitLab.com.
	Neben der Koordination aller Beteiligten werden über diesen Dienst auch die PDFs gebaut, die in der Readme verlinkt sind.
\end{minipage}\\[1cm]
\begin{minipage}{4cm}
	\qrcode[height=3.3cm, version=6]{https://github.com/JaMeZ-B/latex-wwu}
\end{minipage}
\hfill
\begin{minipage}{\dieBreite}
	\includegraphics[height=0.6cm, keepaspectratio]{../!config/Bilder/github_octo.pdf}
	\includegraphics[height=0.6cm, keepaspectratio]{../!config/Bilder/GitHub_Logo.pdf}\\
	\url{https://github.com/JaMeZ-B/latex-wwu} \smallskip\\
	Die Entwicklung des \enquote{\LaTeX-WWU}-Projekts hat ursprünglich auf GitHub stattgefunden, ist mittlerweile aber zu GitLab gewechselt.
	Das GitHub-Repository wird stündlich automatisch aktualisiert, Merge-Requests werden aber nicht mehr entgegengenommen.
\end{minipage}\\[1cm]
% \begin{minipage}{4cm}
% 	\qrcode[height=3.3cm, version=6]{https://uni-muenster.sciebo.de/public.php?service=files&t=965ae79080a473eb5b6d927d7d8b0462}
% \end{minipage}
% \hfill
% \begin{minipage}{\dieBreite}
% 	\raisebox{-2pt}{\includegraphics[height=0.6cm, keepaspectratio]{../!config/Bilder/sciebo_logo.pdf}}
% 	\resizebox{!}{0.5cm}{\large \sffamily\textbf{sciebo}} {\sffamily\large die Campuscloud} \\
% 	\resizebox{\dieBreite}{!}{\footnotesize\url{https://uni-muenster.sciebo.de/public.php?service=files&t=965ae79080a473eb5b6d927d7d8b0462}}\smallskip\\
% 	Sciebo ist ein Dropbox-Ersatz der Hochschulen in NRW, der von der Uni Münster in leitender Position auf Basis der OpenSource-Software Owncloud aufgebaut wurde.
% \end{minipage}\\[1cm]
\hrule \mbox{ }\\[0.7cm]
\begin{minipage}{4cm}
	\qrcode[height=3.3cm, version=6]{\homepage}
\end{minipage}
\hfill
\begin{minipage}{\dieBreite}
	\resizebox{!}{0.5cm}{\large\sffamily\textbf{Vorlesungshomepage}}\\
	\resizebox{\dieBreite}{!}{\footnotesize\url{\homepage}}\smallskip\\
	Hier ist ein Link zur offiziellen Vorlesungshomepage.
\end{minipage}
\newpage
\section*{Vorwort --- Mitarbeit am Skript}
Dieses Dokument ist eine Mitschrift aus der Vorlesung \enquote{\fach, \semester}, gelesen von \prof. 
Der Inhalt entspricht weitestgehend dem Tafelanschrieb. 
Für die Korrektheit des Inhalts übernehme ich keinerlei Garantie! 
Für Bemerkungen und Korrekturen -- und seien es nur Rechtschreibfehler -- bin ich sehr dankbar. 
Korrekturen lassen sich prinzipiell auf drei Wegen einreichen: 
\begin{itemize}
	\item Persönliches Ansprechen in der Uni, Mails an \hrefsymmail{mailto:\mail}{\mail} (gerne auch mit annotieren PDFs) oder Kommentare auf \url{https://gitlab.com/JaMeZ-B/LaTeX-WWU}.
	\item \emph{Direktes} Mitarbeiten am Skript: Den Quellcode poste ich auf GitLab (siehe oben), also stehen vielfältige Möglichkeiten der Zusammenarbeit zur Verfügung:
	Zum Beispiel durch Kommentare am Code über die Website und die Kombination Fork und Merge-Request. 
	Wer sich verdient macht oder ein Skript zu einer Vorlesung, die ich nicht besuche, beisteuern will, dem gewähre ich gerne auch Schreibzugriff.
	
	Beachten sollte man dabei, dass dazu ein Account bei \url{gitlab.com} notwendig ist, der allerdings ohne Angabe von persönlichen Daten angelegt werden kann. 
	Wer bei GitLab (bzw. dem zugrunde liegenden Open-Source-Programm \enquote{\texttt{git}}) -- verständlicherweise -- Hilfe beim Einstieg braucht, dem helfe ich gerne weiter. 
	Es gibt aber auch zahlreiche empfehlenswerte Tutorials im Internet.\footnote{zB. \url{https://try.github.io/levels/1/challenges/1}, ist auf Englisch, aber dafür interaktiv}
	\item \emph{Indirektes} Mitarbeiten: \TeX-Dateien per Mail verschicken. 
	
	Dies ist nur dann sinnvoll, wenn man einen ganzen Abschnitt ändern möchte (zB. einen alternativen Beweis geben), da ich die Änderungen dann per Hand einbauen muss! Ich freue mich aber auch über solche Beiträge!
\end{itemize}
\section*{Literatur}
\begin{itemize}
	\item \citetitle{Ziller} von Wolfgang \citeauthor{Ziller} \cite{Ziller}
	\item \citetitle{Berndt} von Jürgen \citeauthor{Berndt} \cite{Berndt} 
\end{itemize}
\end{abstract}

\tableofcontents
\cleardoubleoddemptypage

\pagenumbering{arabic}
\setcounter{page}{1}
\setcounter{footnote}{0}

\chapter{Grundlagen} % (fold)
\label{cha:grundlagen}

\section{Liegruppen und Liealgebren} % (fold)
\label{sec:1}
\begin{definition}[label=def:111,{name=[Liegruppe]}]
	Eine \Index{Liegruppe} ist eine abstrakte Gruppe, welche zusätzlich eine $n$-dimensionale, differenzierbare Mannigfaltigkeit ist, sodass die beiden Abbildungen 
	$m \colon G \times G \to G$, $(g,h) \mapsto g \cdot h$ und $^{-1} \colon G \to G$, $g \mapsto g^{-1}$ differenzierbar sind.
\end{definition}

\begin{beispiel*}[{name=[{Liegruppen}]}]
	Es gibt zahlreiche Liegruppen: 
	\begin{itemize}
		\item Die $1$-Sphäre $S^1 = \set*{e^{i \varphi} \in \mathbb{C} \given \varphi \in \mathbb{R}} = \set[\big]{z \in \mathbb{C} \given \abs*{z}=1} \simeq \set*{\begin{psmallmatrix*}[r]
			\cos \varphi & - \sin \varphi \\
			\sin \varphi & \cos \varphi
		\end{psmallmatrix*} \given \varphi \in \mathbb{R}} = \SO_\mathbb{R}(2)$
		\item Der $n$-Torus $T^n = S^1 \times \ldots \times S^1 $ ($n$-faches karthesisches Produkt)
		\item Sei $r \in \mathbb{R}$ und $S^1_r := \set*{\enbrace{e^{i \varphi}, e^{i r \varphi}} \in S^1 \times S^1 \given \varphi \in \mathbb{R}}$.
		Für $r \in \mathbb{Q}$ ist $S^1_r$ eine $1$-dimensionale Untermannigfaltigkeit des Torus $T^2= S^1 \times S^1$.
		Für $r \in \mathbb{R} \setminus \mathbb{Q}$ ist $S^1_r$ \emph{keine} eingebettete Untermannigfaltigkeit, denn es gilt 
		\[
			\overline{S^1_r} = T^2
		\]
		\emph{Diese Gruppe wird in Aufgabe 8 der Übungen genauer behandelt. Siehe \cref{sec:aufg8}.}
		\item Die $3$-Sphäre $S^3 = \set[\big]{q \in \mathbb{H} \given \abs*{q}=1}$, wobei $\mathbb{H}$ der Schiefkörper der \Index{Quaternionen}\footnote{siehe \url{https://de.wikipedia.org/wiki/Quaternion}} ist.
	
		Wir setzen $\mathbb{R}^3 := \Im \mathbb{H} = \Span_\mathbb{R} \set{i,j,k}$.
		Dann ist $\mathbb{R}^3$ invariant unter der adjungierten Darstellung von $S^3$, das heißt für alle $x \in \im \mathbb{H}$ und alle $q \in S^3$ gilt:
		\[
			\Ad(q)(x) := q \cdot x \cdot \Underbrace{q^{-1}}{=\overline{q}} \stackrel{!}{\in} \mathbb{R}^3 \qquad \text{ für } x \in \mathbb{R}^3 
		\]
		Man kann ferner zeigen, dass $\Ad(q)|_{\mathbb{R}^3} \subset SO(3)$.
		Es gilt sogar $\Ad(S^3) = SO(3)$ und $\pi \colon S^3 \to SO(3)$, $q \mapsto \Ad(q)|_{\mathbb{R}^3}$ ist eine $2$-fache Überlagerung.
		Dies folgt mit $\Ad(q)=\Ad(-q)$.
		
		\emph{In Übungsaufgabe 3 werden diese Behauptungen bewiesen. Siehe \cref{sec:aufg3}.}
		\item \Index{Orthogonale Gruppe} $\On(n) = \set*{A \in \Mat(n,\mathbb{R}) \given A A^T = E_n}$, $\SO(n) = \set[\big]{A \in \On(n) \given \det A=1}$
		\item \Index{Unitäre Gruppe} $\Un(n) = \set[\big]{A \in \Mat (n,\mathbb{C}) \given A A^* =E_n}$, $\SU(n) = \set[\big]{A \in \Un(n) \given \det A=1}$
		\item \Index{Allgemeine lineare Gruppe} $\GL(n,\mathbb{K}) = \set[\big]{A \in \Mat(n,\mathbb{K}) \given A \text{ invertierbar}}$ für $\mathbb{K} \in  \set*{\mathbb{R},\mathbb{C},\mathbb{H}}$ und die \Index{Spezielle lineare Gruppe}
		$\SL(n,\mathbb{K}) = \set[\big]{A \in \GL(n,\mathbb{K}) \given \det A =1}$
		\item Sind $G_1$ und $G_2$ Liegruppen, so ist auch $G_1 \times G_2$ eine Liegruppe.
		\item Ist $G$ eine Liegruppe, so ist\marginnote{die Abgeschlossenheit bzgl. der Gruppenoperationen folgt aus der Stetigkeit selbiger}
		\[
			G_0 :=  \set[\big]{g \in G \given \exists\, c \colon [0,1] \to G \text{ stetig mit } c(0)=e, c(1)=g}
		\]
		ein Liegruppe, die Zusammenhangskomponente der Eins.
		Beispiel: $\On(n)=\SO(n) \mathbin{\dot{\cup}} I \cdot \SO(n)$, wobei $I$ eine Spiegelung ist.
		Tatsächlich hat $\On(n)$ genau zwei Zusammenhangskomponenten; im Fall $n=2$:
		\[
			\On(2) = \SO(2) \mathbin{\dot{\cup}} \set*{\begin{pmatrix*}[r]
				\cos \varphi & \sin \varphi \\
				\sin \varphi & - \cos \varphi
			\end{pmatrix*} \given \varphi \in \mathbb{R}}
		\]
	\end{itemize}
\end{beispiel*}

\begin{definition}[{name=[Liealgebra und Lieklammer]}]
	Eine \Index{Liealgebra} über $\mathbb{R}$ oder $\mathbb{C}$ ist ein $\mathbb{K}$-Vektorraum $V$ zusammen mit einer bilinearen, schiefsymmetrischen Abbildung 
	\(
		[\cdot ,\cdot ] \colon V \times V \to V
	\),
	welche die \Index{Jacobi-Identität} 
	\[
		\benbrace[\big]{X, \benbrace*{Y,Z}} + \benbrace[\big]{Z, \benbrace*{X,Y}} + \benbrace[\big]{Y, \benbrace*{Z,X}} =0
	\]
	für alle $X,Y,Z \in V$ erfüllt.
	Man nennt $\benbrace*{\cdot ,\cdot }$ eine \Index{Lieklammer}. 
\end{definition}

Wir werden zeigen, dass für eine Liegruppe $G$ der Tangentialraum am neutralen Element $\Tmap_e G$ auf natürliche Art und Weise eine Liealgebra ist.
Dazu ein wenig Notation: Für $g \in G$ betrachten wir \bet{Links}- und \Index{Rechtsmultiplikation}\index{Linksmultiplikation}
\begin{align}
	\SwapAboveDisplaySkip
	L_g \colon G \longrightarrow G \quad , \qquad &h \longmapsto g \cdot h \\
	R_g \colon G \longrightarrow G \quad , \qquad &h \longmapsto h \cdot g
\end{align}
$L_g$ und $R_g$ sind nach Definition einer Liegruppe offensichtlich Diffeomorphismen.

\begin{definition}[{name=[linksinvariantes Vektorfeld]}]
	Ein Vektorfeld\index{Vektorfeld} $X$ auf $G$ nennt man \bet{linksinvariant}\index{Vektorfeld!linksinvariant}, falls für alle $g,h \in G$ gilt
	\[
		X(g \cdot h) = \mathd(L_g)_h \cdot X(h)
	\]
\end{definition}

Offenbar ist ein solches Vektorfeld $X$ schon eindeutig durch seinen Wert $X(e)$ bestimmt, denn 
\[
	X(g) = X(g \cdot e) = \mathd(L_g)_e \cdot X(e)
\]
Wir schreiben daher auch $X=X_v$ mit $X(e)=v \in \Tmap_e G$.

\begin{lemma}[label=lem:114,{name=[Linksinvariante Vektorfelder sind differenzierbar]}]
	Linksinvariante Vektorfelder sind differenzierbar.
\end{lemma}
\begin{beweis}
	Nach \autoref{def:111} ist die Multiplikation $m \colon G \times G \to G$ differenzierbar, somit auch $\mathd m \colon \Tmap G \times \Tmap G \to\Tmap G$ und damit für alle $v \in \Tmap_e G$ auch $\tilde{X}_v \colon G \to \Tmap G$, $g \mapsto (\mathd m)_{(g,e)}(0,v)$.
	Wir zeigen nun $\tilde{X}_v=X_v$.
	Sei dazu $\enbrace[\big]{\gamma(t)}_{t \in (-1,1)}$ eine differenzierbare Kurve in $G$ mit $\gamma(0)=e$ und $\gamma'(0)=v$.
	Dann gilt für $g \in G$ beliebig
	\[
		\tilde{X}_v(g)=(\mathd m)_{(g,e)}(0,v) = \diffd{}{t}\Big|_{t=0} \! m \enbrace[\big]{g,\gamma(t)} = \diffd{}{t}\Big|_{t=0}\! L_g\enbrace[\big]{\gamma(t)} = \mathd(L_g)_{\gamma(0)} \cdot \gamma'(0)
		 =\mathd(L_g)_e v = X_v(g) \qedhere
	\]
\end{beweis}

Wir bezeichnen mit $\mathfrak{g}$ (\enquote{g} in Frakturschrift) die Menge der linksinvarianten Vektorfelder auf $G$.

\begin{korollar}[{name=[{Isomorphismus linkinv. Vektorfelder und Tangentialraum an $e$}]}]
	Die Abbildung $L \colon \mathfrak{g} \to \Tmap_e G, X \mapsto X(e)$ ist ein linearer Isomorphismus.
\end{korollar}
\begin{beweis}
	Die Injektivität und Linearität sind klar.
	Für die Surjektivität sei $v \in \Tmap_e G$ ein Tangentialvektor.
	Nach \autoref{lem:114} ist $X_v$ ein glattes Vektorfeld auf $G$. 
	Es ist außerdem linksinvariant, da
	\[
		X_v(g \cdot h) = \mathd(L_{g \cdot h})_e v = \mathd(L_g)_h \cdot (\mathd L_h)_e \cdot v = \mathd (L_g)_h \cdot X_v(h)
	\]
	Damit ist die Surjektivität gezeigt.
\end{beweis}

Ein Vektorfeld $X$ ist linksinvariant genau dann, wenn $X$ für alle $g \in G$ \bet{$L_g$-verwandt}\index{Vektorfeld!verwandt} zu sich selbst ist, das heißt
\[
	X \enbrace[\big]{L_g(h)} = (\mathd L_g)_h X(h) \qquad  \forall h \in G
\]
Nach Differentialgeometrie \RM{1}. ist die Lieklammer von $L_g$-verwandten Vektorfeldern auch wieder $L_g$-verwandt, also ist die Lieklammer linksinvarianter Vektorfelder wieder ein linksinvariantes Vektorfeld.
Dies führt uns zu folgender Definition:

\begin{definition}[{name=[Liealgebra einer Liegruppe]}]
	Wir bezeichnen mit\marginnote{wir übertragen also die Liealgebra-Struktur von $\mathfrak{g}$ auf $\Tmap_e G$}
	\[
		\benbrace*{\cdot ,\cdot } \colon \Tmap_e G \times \Tmap_e G \to \Tmap_e G \quad ,\quad  (v,w) \longmapsto \benbrace*{X_v,X_w}(e)
	\] 
	die \Index{Lieklammer} von $\Tmap_e G = \mathfrak{g}$.
	Da die Lieklammer differenzierbarer Vektorfelder die Jacobi-Identität erfüllt, ist $\benbrace*{\cdot ,\cdot }$ eine Lieklammer auf $\Tmap_e G$.
\end{definition}

\begin{beispiel*}[{name=[{Lieklammer für allgemeine lineare Gruppe}]}]
	Wir wollen die Lieklammer der Liegruppe $G=\GL(n,\mathbb{K})$ mit $\mathbb{K}=\mathbb{R}$ ausrechnen (für $\mathbb{K}=\mathbb{C}$ analog):
	
	Da die Determinante $\det \colon G \to \mathbb{K}$ stetig ist, ist 
	\(
		{\det}^{-1} \enbrace[\big]{\mathbb{K}\setminus \set*{0}} = \GL(n,\mathbb{K}) 
	\)
	eine offene Teilmenge des euklidischen Vektorraums $\Mat(n,\mathbb{K})$ und somit eine differenzierbare Mannigfaltigkeit.
	Wir bezeichnen mit $x_{ij} \colon G \to \mathbb{K}$ die Abbildung $(a_{kl})_{kl} \mapsto a_{ij}$ für festes $(i,j)$ mit $1\le i,j\le n$.
	Da $x_{ij}$ linear ist, gilt für $A \in G$\marginnote{Linksmultiplikation ist in diesem Fall auch linear!}
	\[
		(X_v)(x_{ij})(A) = (\mathd x_{ij})_A X_v(A) = x_{ij}(A \cdot v) = (A v)_{ij}
	\]
	Somit ist
	\begin{align}
		\benbrace[\big]{X_v,X_w}(x_{ij})(e) = X_v(e) \enbrace[\big]{X_w(x_{ij})} - X_w(e) \enbrace[\big]{X_v(x_{ij})} &= v \enbrace[\Big]{\Underbrace{X_w(x_{ij})}{\mathclap{A \mapsto (A \cdot w)_{ij}}}} - w \enbrace[\big]{X_v(x_{ij})} \\
		&= \diffd{}{t} \enbrace[\big]{(ew + t \cdot vw)_{ij} - (ev + t \cdot wv)_{ij}} \Big|_{t=0} \\
		&= (v w- w v)_{ij}
	\end{align}
	Für ein beliebiges differenzierbares Vektorfeld $X$ und lokale Koordinaten $(x_1, \ldots ,x_n)=x$ gilt in einem Kartengebiet $U$
	\[
		X = \sum_{i=1}^{n} X(x_i) \cdot \diff{}{x_i}
	\]
	Somit gilt $\benbrace*{X_v,X_w} =X_{vw -wv}$ und daher $\benbrace*{v,w}_{\mathfrak{gl}(n,\mathbb{R})}=vw -wv$.
	Die Lieklammer auf der Liealgebra von $\GL_n(\mathbb{K})$ ist also durch den \Index{Matrix-Kommutator} gegeben.
\end{beispiel*}
% subsection 11 (end)

\section{Lieuntergruppen und Homomorphismen von Liegruppen} % (fold)
\label{sec:12}

\begin{definition}[{name=[Liegruppen- und Liealgebrenhomomorphismen]}]
	Seien $H,G$ Liegruppen.
	\begin{enumerate}[1)]
		\item Eine Abbildung $\Phi \colon H \to G$ nennt man \Index{Liegruppenhomomorphismus}, falls $\Phi$ differenzierbar und ein Gruppenhomomorphismus ist.
		\item Einen Liegruppenhomomorphismus $\Phi \colon H \to G$ nennt man \Index{Liegruppenisomorphismus}, falls $\Phi$ bijektiv und $\Phi^{-1}$ differenzierbar ist. ($\Phi^{-1}$ ist auch Gruppenhomomorphismus)
		\item \bet{Liealgebrenhomomorphismen}\index{Liealgebrenhomomorphismus} und Liealgebrenisomorphismen werden entsprechend definiert:  $L \colon \mathfrak{h} \to \mathfrak{g}$ linear heißt \Index{Liealgebrahomomorphismus}, falls für alle $v,w \in \mathfrak{h}$ gilt:
		\[
			L \benbrace[\big]{v,w}_{\mathfrak{h}} = \benbrace[\big]{L v, L w}_\mathfrak{g}
		\]
	\end{enumerate}
\end{definition}

Ist $\Phi \colon H \to G$ ein Liegruppenhomomorphismus, so kommutiert das folgende Diagramm
\[
	\begin{tikzcd}
		H \rar["\Phi"] \dar["L_h"] & G \dar["L_{\Phi(h)}"] \\
		H \rar["\Phi"] & G
	\end{tikzcd}
\]
denn es gilt
\begin{align}
	(\Phi \circ L_h)(\tilde{h}) = \Phi \enbrace*{L_h(\tilde{h})} = \Phi(h \cdot \tilde{h}) = \Phi(h) \cdot \Phi(\tilde{h}) = \enbrace*{L_{\Phi(h)} \circ \Phi}(\tilde{h})
\end{align}
Einen Homomorphismus $\Phi \colon G \to \GL(n,\mathbb{K})$ nennt man \bet{reelle}\index{Darstellung!reelle} bzw. \bet{komplexe Darstellung}\index{Darstellung!komplexe} von $G$.\marginnote{$\Phi_0(G)\equiv E_n$}
Ist $\Phi \colon G \to \GL(n,\mathbb{K})$ injektiv, so besitzt die Gruppe $G$ die \Index{Darstellung!treue} $\Phi(G)$ in $\GL(n,\mathbb{K})$.

\begin{lemma}[label=lem:122,{name=[{Differntial von Liegruppenhom. ist Liealgebrenhom.}]}]
	Sei $\Phi \colon H \to G$ ein Liegruppenhomomorphismus.\marginnote{es folgt, dass man die Liealgebra einer Liegruppe als Funktor auffassen kann}
	Dann ist das Differential am neutralen Element $(\mathd\Phi)_e \colon \Tmap_e H \to \Tmap_e G$ ein Liealgebrenhomomorphismus.
\end{lemma}
\begin{beweis}
	Seien $v_1,v_2 \in \Tmap_e H$ und $w_i := (\mathd \Phi)_e \, v_i$ für $i=1,2$.
	Zu zeigen
	\[
		(\mathd \Phi)_e \benbrace[\big]{X_{v_1}, X_{v_2}} (e) = \benbrace[\big]{X_{w_1},X_{w_2}}(e)
	\]
	Wir zeigen, dass $X_{v_1}$ und $X_{w_1}$ $\Phi$-verwandt sind: Für $h \in H$ gilt
	\begin{align}
		(\mathd \Phi)_h X_{v_1}(h) = \enbrace*{\mathd \Phi}_h (\mathd L_h)_e \cdot v_1 = \mathd \enbrace*{\Phi \circ L_h}_e \cdot v_1 
		= \mathd \enbrace*{L_{\Phi(h)} \circ \Phi}_e \cdot v_1 
		&= \mathd \enbrace*{L_{\Phi(h)}}_e \cdot \enbrace*{\mathd \Phi}_e \cdot v_1 \\
		&= \mathd \enbrace*{L_{\Phi(h)}}_e \cdot w_1 \\
		&= X_{w_1}\enbrace[\big]{\Phi(h)}
	\end{align}
	Somit sind $X_{v_1}$ und $X_{w_1}$ $\Phi$-verwandt und analog auch $X_{v_2}$ und $X_{w_2}$.
	Nach Differentialgeometrie \RM{1}. sind dann auch $\benbrace*{X_{v_1}, X_{v_2}}$ und $\benbrace*{X_{w_1},X_{w_2}}$ $\Phi$-verwandt, das heißt es gilt
	\[
		(\mathd \Phi)_h \benbrace[\big]{X_{v_1},X_{v_2}}(h) = \benbrace[\big]{X_{w_1}, X_{w_2}} \enbrace[\big]{\Phi(h)}
	\]
	Mit $h=e$ folgt nun die Behauptung.
\end{beweis}

\begin{beispiel*}[{name=[Konjugation und adjungierte Darstellung]}]
	Sei $G$ eine Liegruppe und $i_g \colon G \to G$ gegeben durch $h \mapsto g \cdot h \cdot g^{-1}$ die \Index{Konjugation} mit $g$.
	Dann ist $i_g$ ein Liegruppenisomorphismus:
	\begin{itemize}[itemsep=0pt]
		\item $i_g(h) = \enbrace*{l_g \circ R_{g^{-1}}}(h)$, also ist $i_g$ differenzierbar.
		\item Die Gruppenhomomorphismuseigenschaft ist klar. 
		\item Die inverese Abbildung ist offenbar gegeben durch $(i_g)^{-1}= i_{g^{-1}}$.
	\end{itemize}
	Somit ist $\Ad(g) := \enbrace*{\mathd i_g}_e \colon \mathfrak{g} \to \mathfrak{g}$ ein Liealgebrenisomorphismus aufgrund der Funktoreigenschaft aus \autoref{lem:122}.	
	Ist $G = \GL(n,\mathbb{K})$, so kann man $\Ad(g)$ einfach berechnen:
	Sei $(h(t))_{t \in (-\varepsilon,\varepsilon)}$ eine differenzierbare Kurve in $G$ mit $h(0)= E_n$ und $h'(0)=v$.
	Dann gilt
	\begin{align}
		\Ad(g)(v)= (\mathd i_g)_e \cdot h'(0) = \diffd{}{t}\Big|_{t=0} i_g \enbrace[\big]{h(t)} = \diffd{}{t}\Big|_{t=0} g \cdot h(t) \cdot g^{-1}
		&= g \cdot \enbrace*{\diffd{}{t}\Big|_{t=0} h(t)} \cdot g^{-1}\\
		&= g \cdot v \cdot g^{-1}
	\end{align}
	Ferner ist für eine beliebige Liegruppe $G$ die Abbildung $\Ad \colon G \to \GL\enbrace*{\Tmap_e G}$, $g \mapsto \Ad(g)$ ein Liegruppenhomomorphismus:
	\begin{align}
		\Ad(g_1 \cdot g_2) = (\mathd i_{g_1 \cdot g_2})(e) = (\mathd i_{g_1})_e \cdot (\mathd i_{g_2})_e = \Ad(g_1) \circ \Ad(g_2)
	\end{align}
	$\Ad \colon G \to \GL(\mathfrak{g})$ heißt die \Index{adjungierte Darstellung} von $G$.
\end{beispiel*}

\begin{definition}[{name=[{Lieuntergruppe und Lieunteralgebra}]}]
	Sei $G$ eine Liegruppe.
	\begin{enumerate}[1)]
		\item Mann nennt eine Teilmenge $H \subseteq G$ eine \Index{Lieuntergruppe}, falls 
		\begin{itemize}
			\item $H$ eine Untergruppe von $G$ ist,
			\item $H$ eine Liegruppe ist, wenn $H$ mit der von $G$ induzierten Multiplikation und Inversenbildung versehen wird und
			\item die Inklusionsabbildung $i \colon H \hookrightarrow G$ eine Immersion ist \marginnote{Immersion: injektives Differential}\\ 
			(nicht notwendigerweise eine Einbettung!).
		\end{itemize}
		\item Mann nennt einen Teilraum $\mathfrak{h} \subseteq \mathfrak{g}$ \Index{Lieunteralgebra}, falls $\benbrace*{X,Y}_\mathfrak{g} \in \mathfrak{h}$ für alle $X,Y \in \mathfrak{h}$ gilt.
	\end{enumerate}
\end{definition}

\begin{beispiel*}[{name=[Lieuntergruppen]}]
	\leavevmode
	\begin{itemize}
		\item Die Gruppe $S^1_r = \set*{\enbrace*{e^{i \varphi}, e^{i r \varphi}} \given \varphi \in \mathbb{R}}$ ist eine Lieuntergruppe von $T^2=S^1 \times S^1$.
		Für $r \in \mathbb{Q}$ ist $S^1_r$ eine (eingebettete) Untermannigfaltigkeit; für $r \in \mathbb{R}\setminus \mathbb{Q}$ ist $S^1_r$ \enquote{nur} eine immersierte Untermannigfaltigkeit, jedoch nicht eingebettet.
		
		\emph{Diese Gruppe wird in Aufgabe 8 der Übungen genauer behandelt. Siehe \cref{sec:aufg8}.}
		\item $\SO(n) \subset \GL(n,\mathbb{R})$ ist eine Lieuntergruppe mit Liealgebra 
		\[
			\mathfrak{so}(n) := \Tmap_e \SO(n) = \set*{X \in \Mat(n,\mathbb{R}) \given X^T=-X}
		\]
		Sei dazu $(A(t))_{t \in (-\varepsilon,\varepsilon)}$ eine differenzierbare Kurve in $\SO(n)$ mit $A(0)=E_n$ und $A'(0)=X$.
		Aus $A(t)^T \cdot A(t) \equiv E_n$ folgt durch Differenzieren $X^T + X =0$.
		Sei nun umgekehrt $X$ eine schiefsymmetrische Matrix.
		Setze dann 
		\[
			A(t) =\exp(t X) = \sum_{k=0}^{\infty} \frac{(tX)^k}{k!} 
		\]
		Behauptung: Es gilt $A(t) \in \SO(n)$.
		\begin{align}
			A(t)^T \cdot A(t) = \enbrace*{\sum_{k=0}^{\infty} \frac{(tX)^k}{k!} }^T \cdot \sum_{k=0}^{\infty} \frac{(tX)^k}{k!} \stackrel{X^T=-X}{=\joinrel=\joinrel=\joinrel=}\sum_{k=0}^{\infty} \frac{(-tX)^k}{k!}  \cdot \sum_{k=0}^{\infty} \frac{(tX)^k}{k!} 
			&= \exp(tX-tX) \\[-3pt]
			&= \exp(0) =E_n
		\end{align}
		Dies folgt mit $\exp(X+Y)= \exp(X) \exp(Y)$, falls $[X,Y]=0$.
	\end{itemize}
\end{beispiel*}

\begin{lemma}[{name=[Lieunteralgebra zu einer Lieuntergruppe]}]
	Sei $G$ eine Liegruppe und $H \subset G$ eine Lieuntergruppe.
	Dann ist $\mathfrak{h} = \Tmap_e H$ eine Lieunteralgebra von $\mathfrak{g} = \Tmap_e G$.
\end{lemma}
\begin{beweis}
	Sei $i \colon H \hookrightarrow G$ die Inklusionsabbildung.
	Dann ist $i$ ein Liegruppenhomomorphismus.
	Nach \autoref{lem:122} ist somit $(\mathd i)_e = \id_{\Tmap_e H} \colon \Tmap_e H \to \Tmap_e G$ ein Liealgebrenhomomorphismus, das heißt $\Tmap_e H = \mathfrak{h}$ ist eine Lieunteralgebra von $\mathfrak{g}$.
\end{beweis}

Wir erinnern daran, dass die Zusammenhangskomponente $G_0$ der \enquote{Eins} $e \in G$ einer Liegruppe wieder eine Liegruppe ist.
Zum Beispiel gilt für die orthogonale Gruppe: 
\[
	\On(n)= \SO(n) \mathbin{\dot{\cup}} \set*{ \begin{psmallmatrix}
		-1 & & &\\
		& 1 &  &\\
		& & \ddots & \\
		& & & 1
	\end{psmallmatrix} \cdot \SO(n)}
\]

\begin{satz}[label=satz:125,{name=[Lieuntergruppe zu Lieunteralgebra]}]
	Sei $G$ eine Liegruppe und $\mathfrak{h}$ eine Unteralgebra von $\mathfrak{g}$.
	Dann existiert genau eine zusammenhängende Lieuntergruppe $H$ von $G_0$ mit Liealgebra $\mathfrak{h}$. 
\end{satz}
\begin{beweis}
	Sei $g \in G$ und $\mathcal{F}_g := (\mathd L_g)_e \cdot \mathfrak{h} \subset \Tmap_g G$.
	Man nennt $\mathcal{F} = \set*{\mathcal{F}_g}_{g \in G}$ eine \Index{Distribution} von Unterräumen (gleicher Dimension!) auf $G$.
	Seien $v_1, \ldots ,v_r \in \Tmap_e G$ eine Basis von $\mathfrak{h}$.
	Wir bezeichnen mit $X_{v_1}, \ldots ,X_{v_r}$ die entsprechenden linksinvarianten Vektorfelder auf $G$.
	Es gilt
	\[
		\mathcal{F}_g = \Span_\mathbb{R} \enbrace[\big]{X_{v_1}(g), \ldots , X_{v_r}(g)}
	\]
	Die Distribution $\mathcal{F}$ ist \emph{involutiv}, das heißt es gilt $\benbrace*{\mathcal{F},\mathcal{F}} \subset \mathcal{F}$, denn nach Definition ist
	\[
		\benbrace[\big]{X_{v_i}, X_{v_j}} = X_{\benbrace*{v_i,v_j}_{\mathfrak{g}}} = X_w
	\]
	mit $w= \benbrace*{v_i,v_j} \in \mathfrak{h}$.
	Der \emph{Satz von Frobenius} \cite{LeeSmooth} besagt nun, dass durch jeden Punkt $g \in G$ eine zusammenhängende, maximale Integralmannigfaltigkeit $M^r_g$ geht, das heißt $\forall \tilde{g} \in M^r_g$ gilt $\Tmap_{\tilde{g}} M^r_g = \mathcal{F}_{\tilde{g}}$
	($M^r_g$ ist eine immersierte Untermannigfaltigkeit).
	Da die Distribution $\mathcal{F} = \set*{\mathcal{F}_g}_{g \in G}$ $G$-invariant ist, bildet der Diffeomorphismus $L_g \colon G \to G$ maximale Blätter (= Integralmannigfaltigkeiten) auf maximale Blätter ab.
	Wir setzen $H := M^r_e$ und wissen dann bereits, dass $H$ eine immersierte Untermannigfaltigkeit von $G$ ist.
	
	Zeige nun, dass $H$ eine Untergruppe ist:
	Da $H$ und $L_{h^{-1}}(H)$ beide $e$ enthalten und maximale Blätter sind, gilt $H=L_{h^{-1}}(H)$ für $h \in H$.
	Damit ist die Abgeschlossenheit unter Inversion gezeigt.
	Ähnlich zeigt man die Abgeschlossenheit unter Multiplikation.\marginnote{$h_1, h_2 \in H$, dann enthalten $L_{h_1}(H)$ und $H$ beide $e$, also $L_{h_1}(H)=H \Rightarrow h_1 h_2 \in H$}
	Da $m \colon G  \times G \to G$, $^{-1} \colon G \to G$ differenzierbare Abbildungen sind und die Einschränkung von differenzierbaren Abbildungen auf immersierte Untermannigfaltigkeiten wieder differenzierbare Abbildungen liefert, ist $H$ eine Lieuntergruppe von $G$.
	
	Es bleibt die Eindeutigkeit zu zeigen: 
	Sei also $\tilde{H} \subset L$ eine zusammenhängende Lieuntergruppe von $G$ mit $\Tmap_e \tilde{H} = \mathfrak{h} = \Tmap_e H$.
	Es gilt $\Tmap_{\tilde{h}} \tilde{H} = (\mathd L_{\tilde{h}})_e \cdot \Tmap_e \tilde{H}= \mathcal{F}_{\tilde{h}}$ für alle $\tilde{h} \in \tilde{H}$.
	Somit ist $\tilde{H}$ Integralmannigfaltigkeit von $\mathcal{F}$ mit $e \in \tilde{H}$.
	Da $H$ maximale Integralmannigfaltigkeit ist, folgt $\tilde{H} \subseteq H$.
	Es bleibt die Gleichheit zu zeigen:
	$\tilde{H}$ ist offen in $H$.
	Annahme: Es gibt ein $h \in H \setminus \tilde{H}$ und eine Folge $(\tilde{h}_i)_{i \in \mathbb{N}}$ in $\tilde{H}$ mit $\tilde{h}_i \to h$.
	Dann ist $h \in L_h(U)$, wobei $U$ eine offene Umgebung von $e$ in $\tilde{H}$ ist.
	Damit ist $h \in L_{h_i}(U)$ für $i \ge i_0$ für geeignetes $i_0 \in \mathbb{N}$.
	Dies ist ein Widerspruch, also folgt $\tilde{H}=H$.
\end{beweis}

\begin{korollar}[label=kor:126,{name=[gleiche induzierte Liealgebrenhom. implizieren Gleichheit der Liegruppenhom.]}]
	Seien $H,G$ zusammenhängende Liegruppen und $\Phi, \Psi \colon H \to G$ Liegruppenhomomorphismen.
	Gilt $(\mathd \Phi)_e = (\mathd \Psi)_e \colon \Tmap_e H \to \Tmap_e G$, so folgt $\Phi = \Psi$.
\end{korollar}
\begin{beweis}
	$H \times G$ ist eine Liegruppe mit Liealgebra $\mathfrak{h} \oplus \mathfrak{g}= \Tmap_e H \oplus \Tmap_e G$ (komponentenweise).
	Dann ist eine differenzierbare Abbildung $\Phi \colon H \to G$ genau dann ein Liegruppenhomomorphismus, wenn der Graph von $\Phi$, also $\set[\big]{(h,\Phi(h)) \given h \in H}$, eine Untergruppe von $H \times G $ ist (!).
	Ist dies der Fall, so ist 
	\[
		\Graph \enbrace[\big]{(\mathd \Phi)_e} = \set[\big]{ \enbrace*{v, (\mathd \Phi)_e \cdot v} \given v \in \mathfrak{h}}
	\]
	eine Unteralgebra von $\mathfrak{h} \oplus \mathfrak{g}$.
	Wegen $\Graph \enbrace*{(\mathd \Phi)_e} = \Graph \enbrace*{(\mathd \Psi)_e}$ sind diese beiden Unteralgebren gleich und somit auch die entsprechenden eindeutigen zusammenhängenden Lieuntergruppen nach \autoref{satz:125}.
	Somit ist $\Phi =\Psi$.
\end{beweis}

Wir bezeichnen mit $c_v \colon I_{\max} \to G$, $t \mapsto c_v(t)$ die \Index{maximale Integralkurve} von $X_v$ mit $c_v(0)=e$ und $c_v'(0)=v$.
Es gilt dann $c_v'(t) = X_v \enbrace*{c_v(t)}$ für alle $t$.
Wir wissen weiter, dass $I_{\max}$ offen in $\mathbb{R}$ ist.

\begin{beispiel*}[{name=[{maximale Integralkurve in der allgemeinen linearen Gruppe}]}]
	Betrachte $\GL(n,\mathbb{R})$ und $v \in \Tmap_e G = \Mat(n,\mathbb{R})$.
	Wie gehabt ist $X_v(A) = (\mathd L_A)_e \cdot v = A \cdot v$ und es gilt $c_v(t)= \exp(t \cdot v) = \sum_{k=0}^{\infty} \frac{(tv)^k}{k!}$, denn
	\begin{align}
		\diffd{}{t}\Big|_{t=t_0} c_v(t) = \diffd{}{t}\Big|_{t=t_0} \Underbrace{\exp(t v)}{=\exp(t_0 v) \cdot \exp((t-t_0) \cdot v)} 
		= \exp(t_0 v) \cdot \diffd{}{s}\Big|_{s=0} \exp(s v)
		&= \exp(t_0 v) \cdot v \\[-1em]
		&= c_v(t_0) \cdot v = X_v \enbrace[\big]{c_v(t_0)}
	\end{align}
\end{beispiel*}


\begin{lemma}[label=lem:127,{name=[Integralkurven sind Einparameteruntergruppen von $G$]}]
	Sei $G$ eine Liegruppe.
	Die Integralkurve $c_v$ ist auf ganz $\mathbb{R}$ definiert und es gilt 
	\[
		c_v(t+s) = c_v(t) \cdot c_v(s)
	\]
\end{lemma}
\begin{beweis}
	Setze $\tilde{c}(t) := c_v(t_0) \cdot c_v(t)$.
	Es gilt 
	\begin{align}
		\tilde{c}'(t) = \enbrace*{\mathd L_{c_v(t_0)}}_{c_v(t)}  \cdot \hspace{-1.5em}\Underbrace{\diffd{}{t} c_v(t)}{= X_v \enbrace[\big]{c_v(t)} = \enbrace*{\mathd L_{c_v(t)}} \cdot v} \hspace{-1.5em}= \enbrace*{\mathd L_{c_v(t_0) \cdot c_v(t)}}_e \cdot v = X_v \enbrace*{\tilde{c}(t)}
	\end{align}
	Somit ist $\tilde{c}(t)$ wieder eine Integralkurve von $X_v$ und beide Behauptungen folgen.\footnote{Für die Formel: Für Flüsse $\Phi$ gilt allgemein $\Phi\enbrace*{s, \Phi(t,p)} = \varphi(s+t,p)$, sofern die linke Seite definiert ist. Weiter ist wegen der Linksinvarianz von $X_v$ die Integralkurve an $g \in G$ gerade durch $L_g \circ c_v$ gegeben. Also
	\[
		c_v(t+s)= \Phi(s+t,e) = \Phi\enbrace*{s, \Phi(t,e)} = \Phi \enbrace*{s, \Phi(t,e) \cdot e} = \Phi(t,e) \cdot \Phi(s,e) = c_v(t) \cdot c_v(s)
	\]}
	Man nennt die Kurve $(c_v(t))_{t \in \mathbb{R}}$ auch \Index{Einparameteruntergruppe} von $G$.
\end{beweis}

\begin{definition}[label=def:exp,{name=[Exponentialabbildung]}]
	Sei $G$ eine Liegruppe.
	Dann nennt man 
	\[
		\exp \colon \mathfrak{g} \longrightarrow G ,\quad  v \longmapsto c_v(1)
	\]
	die \Index{Exponentialabbildung} von $G$. 
\end{definition}

\begin{lemma}[label=lem:129,{name=[{Differenzierbarkeit der Exponentialabbildung}]}]
	Sei $G$ eine Liegruppe.
	Dann ist die Exponentialabbildung $\exp \colon \mathfrak{g} \to G$ differenzierbar und es gilt 
	\[
		\enbrace*{\mathd \exp}_0 = \id_{\Tmap_e G}
	\]
\end{lemma}
\begin{beweis}
	Wir betrachten das Vektorfeld
	\mapdef{S \colon G \times \mathfrak{g}}{\Tmap (G \times \mathfrak{g})}{(g,v)}{\enbrace[\big]{X_v(g),0} \in \Tmap_g G \times \Tmap_v \mathfrak{g}}{}
	$S$ ist ein differenzierbares Vektorfeld mit globalem Fluss $\set*{\psi_t}_{t \in \mathbb{R}}$ gegeben durch 
	\[
		\psi_t(g,v) = \enbrace[\big]{g \cdot \exp(t \cdot v),v}
	\]
	denn\marginnote{man muss sich hier klarmachen, dass $c_{tv}(1)=c_{v}(t)$ gilt}
	\begin{align}
		\diffd{}{t}\Big|_{t=t_0} \psi_t (g,v) = \diffd{}{t}\Big|_{t=t_0} \enbrace[\big]{g \cdot \exp(tv),v} = \enbrace*{\diffd{}{t}\Big|_{t=t_0} g \cdot \exp(tv),0} &= (\mathd L_g)_{c_{t_0 v}(1)} \cdot \diffd{}{t}\Big|_{t=t_0} c_{tv}(1) \\[-11pt]
		&= \enbrace*{\mathd L_{g \cdot c_v(t_0)}}_e \cdot v \\
		&= X_v \enbrace[\big]{g \cdot c_v(t_0)} \\
		&= X_v\enbrace[\big]{g \cdot \exp(t_0 v)}
	\end{align}
	Somit ist $\psi_t \colon G \times \mathfrak{g} \to G \times \mathfrak{g}$ differenzierbar für alle $t \in \mathbb{R}$.
	Insbesondere ist $\psi_1$ differenzierbar.
	Mit $\psi_1(e,\cdot ) = \enbrace[\big]{\exp(\cdot ), \cdot }$ folgt, dass auch $\exp$ differenzierbar sein muss.
	Die zweite Behauptung ist nun sofort klar.
\end{beweis}

\begin{satz}[label=satz:1210,{name=[Abgeschlossenheit von Lieuntergruppen]}]
	Sei $G$ eine Liegruppe. Dann gilt:
	\begin{enumerate}[1)]
		\item Eine Lieuntergruppe $H$ von $G$ ist genau dann eine eingebettete Untermannigfaltigkeit von $G$, wenn $H$ abgeschlossen ist.
		\item Eine \enquote{abstrakte} Untergruppe $H$ von $G$ ist eine Lieuntergruppe, falls $H$ abgeschlossen ist.
	\end{enumerate}
\end{satz}
\begin{beweis}
	\begin{enumerate}[1)]
		\item Sei $H \subset G$ ein eingebettete Lieuntergruppe. 
		Zu zeigen ist, dass $H$ abgeschlossen ist.
		Angenommen es existiert eine Folge $(h_i)_{i \in \mathbb{N}} \subset H$ mit $h_i \to g \in G \setminus H$.
		Sei $U$ eine offene Umgebung von $e$, sodass $H \cap U$ eine zusammenhängende Untermannigfaltigkeit von $U$ ist.
		Eine solche Umgebung existiert, da $H$ eingebettet ist.
		Da $L_g$ ein Diffeomorphismus ist, ist $L_g(U)$ eine offene Umgebung von $g$.
		Somit existiert $i_0$ mit $h_i \in L_g(U)$ für alle $i \ge i_0$.
		Wegen $h_i \to g$ können wir $h_{i_0}^{-1} \cdot g \in U$ annehmen.
		Es gilt $h_{i_0}^{-1} \cdot g \notin H \cap U$, denn sonst folgt $g \in H$, da $H$ eine Untergruppe ist.
		Wegen $h_i \to g$ folgt
		\[
			\Underbracket{h_{i_0}^{-1}\cdot h_i}{\in H}  \grenzw{i \to \infty} h_{i_0}^{-1} \cdot g
		\]
		Dies ist -- in lokalen Koordinaten via einem Flachmacher -- ein Widerspruch und somit muss $H$ abgeschlossen sein.
		
		Sei nun $H$ eine abgeschlossene Lieuntergruppe von $G$.
		Zu zeigen ist, dass $H$ eine eingebettete Untermannigfaltigkeit ist, das heißt es existiert eine offene Umgebung $U$ von $e$ in $G$, sodass $H \cap U$ eine zusammenhängende Untermannigfaltigkeit von $U$ ist. 
		Annahme: Es existiert eine Umgebung $U$ von $e$, sowie eine Folge $(h_i)_{i \in \mathbb{N}}$ in $H$ mit
		\begin{itemize}
			\item $h_i \to e$ und
			\item $h_i \notin (H \cap U)_e$ (Zusammenhangskomponente der Eins)
		\end{itemize}
		Die Exponentialabbildung $\exp \colon \Tmap_e G \to G$ ist nahe bei $e$ ein lokaler Diffeomorphismus (\autoref{lem:129} und Umkehrsatz auf Mannigfaltigkeiten) mit 
		\[
			\enbrace*{\mathd \exp^{-1}}_e = \id_{\Tmap_e G}
		\]
		Für alle $i \in \mathbb{N}$ finden wir also $v_i \in \mathfrak{g}$ mit $h_i = \exp(v_i)$ und $v_i \to 0$ für $i \to \infty$.
		% Somit ist $h_i = \exp(v_i)$ für alle $i \in \mathbb{N}$ und $v_i \in \mathfrak{g}$ mit $v_i \to 0$ für $i \to \infty$.
		Wir wählen nun eine reelle Folge $(s_i)_{i \in \mathbb{N}}$ in $\mathbb{R}$ mit 
		\begin{equation}
			s_i \cdot v_i \grenzw{i \to \infty} v \label{eq:1210:1} \tag{\#}
		\end{equation}
		längst einer Teilfolge.
		(Da $\mathfrak{g}$ ein Vektorraum ist, können wir das Skalarprodukt auf $\mathfrak{g}$ betrachten und erhalten eine Norm, setze dann $s_i := \frac{1}{\norm*{v_i}} $.)
		Sei $t \in \mathbb{R}$ \emph{fest} gewählt.
		\eqref{eq:1210:1} impliziert dann
		\begin{equation}
			t \cdot s_i \cdot v_i \grenzw{i \to \infty} t \cdot v \label{eq:1210:2} \tag{\#\#}
		\end{equation}
		Eine einfache Überlegung zeigt, dass man $t \cdot s_i \in \mathbb{Z}$ annehmen kann ($s_i \leadsto \tilde{s}_i := s_i + \text{\enquote{$\sfrac{1}{t}$}}$).
		Beachte, dass \eqref{eq:1210:1} immernoch gilt.
		Aus $H \ni h_i = \exp(v_i)$ folgt 
		\[
			H \ni \exp(v_i)^{t \cdot s_i} = \exp \enbrace*{t \cdot s_i \cdot v_i} \xrightarrow[\text{\eqref{eq:1210:2}}]{i \to \infty} \exp(t \cdot v) \in H
		\]
		da $H$ abgeschlossen ist!
		Dies gilt für alle $t \in \mathbb{R}$ und somit ist $v \in \Tmap_e H$.
		Nächster Schritt: $\skal*{\cdot }{\cdot }$ sei ein Skalarprodukt auf $\Tmap_e G$.
		Dann gilt $\Tmap_e G = \mathfrak{h} \mathop{\oplus_{\bot}} \mathfrak{h}^{\bot}$.
		Mit dieser Zerlegung schreiben wir $v_i =x_i + y_i$.
		Die Abbildung 
		\mapdef{\Phi \colon \mathfrak{h} \mathop{\oplus_{\bot}} \mathfrak{h}^\bot}{G}{(x,y)}{\exp(x) \cdot \exp(y)}{}
		ist nahe $(0,0)$ ein lokaler Diffeomorphismus.
		Wir haben
		\[
			\Underbracket{\exp(v_i)}{\in H} = \Underbracket{\exp(\tilde{x}_i)}{\in H} \cdot \exp(\tilde{y}_i)
		\]
		Dann ist $e \longleftarrow  \exp(\tilde{y}_i) = \exp(- \tilde{x}_i) \cdot \exp(v_i) \in H \setminus (H \cap U)_e$.
		Wie eben bewiesen, können wir nun eine Grenzrichtung 
		\[
			\frac{y_i}{\norm*{y_i}} \grenzw{i \to \infty}  y
		\]
		in $\mathfrak{h}$ konstruieren (mittels Teilfolge).
		Die Eigenschaften $y \in H$, $y \in H^\bot$ und $\norm*{y}=1$ widersprechen sich!
		Also existiert eine solche Folge nicht und $H$ ist eine eingebettete Untermannigfaltigkeit wie behauptet.
		\item Sei $H$ ein abgeschlossene \enquote{abstrakte} Untergruppe von $G$.
		Wir zeigen zunächst, dass 
		\[
			\mathfrak{h} := \set[\big]{v \in \Tmap_e G \given \exp(t \cdot v) \in H \,\,\forall t \in \mathbb{R}} \ni 0
		\]
		ein Unterraum ist.
		Annahme: Es gibt eine Folge $(h_i)_{i \in \mathbb{N}}$ in $H \setminus \set*{e}$ mit $h_i \to e$ (falls nicht: $e$ ist isoliert und $H$ ist eine diskrete Untergruppe von $G$).
		Wie in 1) zeigt man, dass die \enquote{Grenzrichtungen} $\lim_{i \to \infty} \frac{v_i}{\norm*{v_i}}$ in $\mathfrak{h}$ liegen.
		\begin{itemize}[itemsep=0pt]
			\item $v \in \mathfrak{h} \implies t \cdot v \in \mathfrak{h}$ für alle $t \in \mathbb{R}$
			\item $v_1 + v_2 \in \mathfrak{h} \implies v_1 + v_2 \in \mathfrak{h}$
		\end{itemize}
		Sei $(t_n)_{n \in \mathbb{N}}$ eine Nullfolge in $\mathbb{R}$.
		Dann gilt $t_n v_1, t_n v_2 \to 0$ für $n \to \infty$.
		Folglich gilt 
		\[
			\exp(t_n v_1), \exp(t_n v_2) \grenzw{n \to \infty} e 
		\]
		für $n \to \infty$. Somit erhalten wir
		\[
			H \ni \exp(v_n) = \exp \enbrace*{t_n v_1} \exp(t_n v_2) \grenzw{n \to \infty} e
		\]
		mit $v_n \to 0$.
		Nun gilt $\exp(t_n v_1) \cdot \exp(t_n v_2) = \exp \enbrace[\big]{t_n(v_1 + v_2)} + \mathcal{O}(t_n^2)$.
		Somit
		\[
			\frac{v_n}{t_n} \grenzw{n \to \infty} v_1 +v_2 + \mathcal{O} \enbrace*{\frac{t_n^2}{t_n} \to 0 } 
		\]
		Damit ist $\mathfrak{h}$ ein Unterraum.
		Wie in 1) zeigt man nun, das $H$ ein Untermannigfaltigkeit ist.\qedhere
	\end{enumerate}
\end{beweis}

\begin{satz}[name={Ado},label=satz:1211]
	Jede endlichdimensionale Liealgebra $(V,\benbrace*{\cdot,\cdot})$ ist isomorph zu einer Unteralgebra von $\mathfrak{gl}(n,\mathbb{R}) = \Tmap_e \mkern-2mu\GL(n,\mathbb{R})$.
\end{satz}
\begin{beweis}
	\emph{Siehe zum Beispiel \cite[S. 199]{JacobsonLieAlg}.}
\end{beweis}

\begin{korollar}[{name=[{Zu jeder endlichdimensionalen Liealgebra existiert eine Liegruppe}]}]
	Zu jeder endlichdimensionalen Liealgebra $(V,\benbrace*{\cdot ,\cdot })$ existiert eine Liegruppe $G$ mit $V \cong \mathfrak{g}$
\end{korollar}
\begin{beweis}
	Folgt direkt aus \autoref{satz:1211} und \autoref{satz:125}.
\end{beweis}

\begin{bemerkung*}[{name=[{Existenz von Liegruppen, die keine Matrixgruppe ist}]}]
	Für $n \ge 3$ gilt $\pi_1 \enbrace[\big]{\SL(n,\mathbb{R})} = \mathbb{Z}_2$.
	Die universelle Überlagerung $\widetilde{\SL(n,\mathbb{R})}$ von $\SL(n,\mathbb{R})$ ist eine Liegruppe, welche in keine der Gruppen $\GL(N,\mathbb{R})$ eingebetten werden kann! Insbesondere ist sie keine Matrixgruppe! \emph{(siehe Aufgabe 17, \cref{sec:aufg17})}
\end{bemerkung*}

\begin{lemma}[label=lem:1213,{name=[{Zusammenhängede Liegruppe von Umgebung der Eins erzeugt}]}]
    Sei $G$ eine zusammenhängende Liegruppe und $U$ eine offene Umgebung von $e$ in $G$.
    Dann wird $G$ schon von $U$ erzeugt.
\end{lemma}
\begin{beweis}
    Sei $U \subset G$ offen mit $e \in U$.
    Dann existiert $V \subset U$ offen mit $e \in V$, sodass $V \cdot V^{-1} \subset U$, denn die Abbildungen $m \colon G \times G \to G$, $^{-1} \colon G \to G$ sind stetig.
    Wir setzen 
    \[
        H := \bigcup_{n \in \mathbb{Z}} V^n
    \]
    $H$ ist eine offene Untergruppe von $G$.
    Zu zeigen bleibt, dass $H$ abgeschlossen ist.
    Die Abgeschlossenheit folgt wie im Beweis von \autoref{satz:125}.
	
	Alternativ: Sei $x \in \overline{H}$ beliebig.
	Dann ist $x V$ eine Umgebung von $x$ und wir finden ein $h \in H$ mit $h \in xV$.
	Also muss wiederum ein $v \in V \subset H$ existieren mit $h = xv \iff x = hv^{-1}$. 
	Dann muss aber bereits $x \in H$ gelten.
\end{beweis}
% subsection 12 (end)

\section{Überlagerungen von Liegruppen} % (fold)
\label{sec:13}

\begin{definition}[{name=[Überlagerung]}]
    Sei $M^n$ eine $n$-dimensionale, zusammenhängende, differenzierbare Mannigfaltigkeit.
    Dann nennt man eine zusammenhängende, $n$-dimensionale, differenzierbare Mannigfaltigkeit $\widetilde{M}^n$ \Index{Überlagerung} von $M^n$, falls eine surjektive differenzierbare Abbildung $\pi \colon \widetilde{M} \to M$ existiert mit folgenden Eigenschaften:
    
    Für alle $p \in M$ existiert eine offene Umgebung $U$ von $p$ in $M^n$ mit folgenden Eigenschaften:
    \begin{enumerate}[1),itemsep=1pt]
        \item $\pi^{-1}(U) = \bigcup_{i \in I} \widetilde{U}_i$ mit $\widetilde{U}_i \subset \widetilde{M}$ offen und $\widetilde{U}_i \cap \widetilde{U}_j = \emptyset$ für $i \neq j$.\marginnote{Jeder Punkt wird \enquote{schlicht überlagert}}
        \item $\pi\big|_{\widetilde{U}_i} \colon \widetilde{U}_i \to U$ ist ein Diffeomorphismus für alle $i \in I$.
    \end{enumerate}
    Die Abbildung $\pi \colon \widetilde{M}^n \to M^n$ nennt man \Index{Überlagerungsabbildung} und $\abs*{I}$ die \Index{Blätterzahl}, dabei ist $\abs*{I}= \infty$ möglich.
\end{definition}


\begin{beispiel*}[{name=[Überlagerungen]}]
	\leavevmode
	\begin{itemize}[itemsep=2pt]
	    \item $M^n$ überlagert $M^n$ auf triviale Weise.
	    \item $S^3$ ist 2-fache Überlagerung von $\SO(3) = \mathbb{R}P^3$ (Antipoden miteinander identifizieren).
	    \item $S^n$ ist 2-fache Überlagerung von $\mathbb{R}P^n \coloneqq \set[\big]{\text{Menge der Geraden im }\mathbb{R}^{n+1}}$.
	    \item Ist $G$ eine Gruppe, die frei und eigentlich diskontinuierlich auf $\widetilde{M}^n$ operiert\footnote{siehe z.B. \url{https://de.wikipedia.org/wiki/G-Raum}}, so ist der Bahnenraum
	    \[
	        M^n \coloneqq \set[\big]{[p] \given p \in M^n} = \sfrac{\widetilde{M}^n}{G}
	    \]
	    eine $n$-dimensionale, differenzierbare Mannigfaltigkeit und $\pi \colon \widetilde{M}^n \to M^n$, $p \mapsto [p]$ eine Überlagerung.\marginnote{in \cite[Th.~9.19]{LeeSmooth} nur für diskrete Gruppen bewiesen!}
		Dabei gilt
	    \[
	        [p] = [\tilde{p}] :\Longleftrightarrow \exists g \in G : p = g . \tilde{p}
	    \]
	    Ist $G$ endlich, so ist frei und diskontinuierlich äquivalent zu: $g$ wirkt trivial auf einem Punkt $p$ genau dann, wenn $g$ auf allen $p$ trivial operiert.
	    \item Für jede zusammenhängende Mannigfaltigkeit $M^n$ existiert eine bis auf Diffeomorphie eindeutige, einfach zusammenhängende Mannigfaltigkeit $\widetilde{M}^n$, die $M^n$ überlagert, diese bezeichnet man als \Index{universelle Überlagerung}.\footnote{Existenz wie für topologische Räume: zusammenhängend, lokal wegzusammenhängend, semilokal einfachzusammenhängend. Universelle Eigenschaft: Überlagerung jeder weiteren Überlagerung. Überlagerungen von glatten Mannigfaltigkeiten haben stets eine glatte Struktur, sodass $\pi$ differenzierbar ist \cite[Prop.~2.12]{LeeSmooth}.}
	    Ferner gilt $M^n = \sfrac{\widetilde{M}^n}{\pi_1(M^n)}$.
	\end{itemize}
\end{beispiel*}


\begin{lemma}[label=lem:132,{name=[{Liegruppenstruktur der Überlagerung einer Liegruppe}]}]
    Sei $G$ eine zusammenhängende Liegruppe.
    \begin{enumerate}[1)]
        \item Ist $\widetilde{G}$ eine zusammenhängende Mannigfaltigkeit und $\pi \colon \widetilde{G} \to G$ eine Überlagerungsabbildung, so besitzt $\widetilde{G}$ genau eine Liegruppenstruktur, sodass $\pi$ ein Liegruppenhomomorphismus ist.
        \item Ein Liegruppenhomomorphismus $\Phi \colon \widetilde{G} \to G$ mit $\widetilde{G}$ zusammenhängend ist eine Überlagerung genau dann, wenn das Differential $(\mathd \Phi)_{\widetilde{e}} \colon \Tmap_{\widetilde{e}} \widetilde{G} \to \Tmap_e G$ ein Isomorphismus ist.
    \end{enumerate}
\end{lemma}
\begin{beweis}
    \begin{enumerate}[(1)]
        \item Sei $\widetilde{e} \in \pi^{-1}(e)$ fest gewählt.
        Überlagerungstheorie impliziert, dass die stetigen Funktionen $\begin{tikzcd}[cramped,sep=small]
        	\widetilde{G} \times \widetilde{G} \rar["\pi \times \pi"] & G \times G \rar["m"] & G
        \end{tikzcd}$ und $\begin{tikzcd}[cramped,sep=small]
        	\widetilde{G} \rar["\pi"] & G \rar["^{-1}"] & G
        \end{tikzcd}$
		entsprechende Lifts besitzen mit $\widetilde{m}(\widetilde{e},\widetilde{e})=\widetilde{e}$ bzw. $\widetilde{e}\widetilde{{}^{-1}} = \widetilde{e}$.\footnote{siehe zum Beispiel \textsc{Stöcker-Zischer} \emph{Algebraische Topologie} \cite[Abschnitt 6.2]{stockerAlgTop}}
        Die Eindeutigkeit ist durch die Wahl von $\widetilde{e}$ gegeben [Klar, falls $\widetilde{G}$ einfach zusammenhängend, sonst \enquote{kleines} Argument].
        
        Die Abbildungen $\widetilde{m},\widetilde{{ }^{-1}}$ sind differenzierbare Abbildungen, welche die Gruppenaxiome erfüllen: $\widetilde{m} (\widetilde{e}, \widetilde{g}) = \widetilde{g}$ für alle $\widetilde{g} \in \widetilde{G}$.
        Dies ergibt sich auch daraus, dass man Wege eindeutig liften kann.
        
        Die Abbildung $\pi \colon \widetilde{G} \to G$ ist somit ein Liegruppenhomomorphismus.
        Die Eindeutigkeit der Liegruppenstruktur auf $\widetilde{G}$, sodass $\pi \colon \widetilde{G} \to G$ ein Liegruppenhomomorphismus ist, folgt wieder aus der Eindeutigkeit der Lifts.
		Bei einer anderen Wahl von $\widetilde{e}$ erhält man eine isomorphe Liegruppenstruktur.
        \item Die erste Implikation ist klar, denn Überlagerungsabbildungen sind lokale Diffeomorphismen.
        
        Für die Umkehrung sei $(\mathd \Phi)_{\widetilde{e}} \colon \Tmap_{\widetilde{e}} \widetilde{G} \to \Tmap_e G$ ein Isomorphismus.
        Zu zeigen: Jedes $g \in G$ ist schlicht überlagert.
        Sei $\widetilde{U}$ eine offene Umgebung von $\widetilde{e}$ in $\widetilde{G}$, sodass $\Phi|_{\widetilde{U}} \colon \widetilde{U} \to \Phi(\widetilde{U})$ ein Diffeomorphismus ist (eine solche Umgebung existiert nach dem Umkehrsatz).
        Wir definieren nun $\widetilde{\Gamma} := \ker (\Phi)$.
        Dann gilt $\widetilde{\Gamma} \cap \widetilde{U}= \set{\widetilde{e}}$.
        Sei weiter $\widetilde{V}$ eine offene Umgebung von $\widetilde{e}$ mit $\widetilde{V} \cdot \widetilde{V}^{-1} \subset \widetilde{U}$.
        \begin{itemize}
            \item  Seien $\widetilde{\gamma}_1 \neq \widetilde{\gamma}_2 \in \widetilde{\Gamma}$ und $\widetilde{v}_1, \widetilde{v}_2 \in \widetilde{V}$.
            Annahme:
            \[
                \widetilde{\gamma}_1 \cdot \widetilde{v}_1 = \widetilde{\gamma}_2 \cdot \widetilde{v}_2 \iff \widetilde{\gamma}_2^{-1} \cdot \widetilde{\gamma}_1 = \widetilde{v}_2 \cdot \widetilde{v}_1^{-1} \in \widetilde{\Gamma} \cap \widetilde{U}
            \]
            Mit obiger Eigenschaft des Schnitts folgt $\widetilde{\gamma}_1 = \widetilde{\gamma}_2$ und da dies ein Widerspruch ist, muss $\widetilde{\gamma}_1 \widetilde{V} \cap \widetilde{\gamma}_2 \widetilde{V} = \emptyset$ gelten, also sind alle $\widetilde{\gamma} \widetilde{V}$ disjunkt.
            \item Es gilt $\Phi^{-1}(\Phi(\widetilde{V})) = \bigcup_{\widetilde{\gamma} \in \widetilde{\Gamma}} \widetilde{\gamma}\widetilde{V}$:
            Sei dazu $\widetilde{v} \in  \widetilde{V}$ und $\widetilde{g} \in \widetilde{G}$ mit $\Phi(\widetilde{v}) = \Phi(\widetilde{g})$.
            Dies ist äquivalent zu
            \[
                \Phi(\widetilde{g}^{-1} \widetilde{v}) = \Phi(\widetilde{g}^{-1}) \Phi(\widetilde{v}) = e
            \]
            Also ist $\widetilde{g}^{-1} \widetilde{v} = \widetilde{\gamma} \in \widetilde{\Gamma}$ und somit $\widetilde{v} = \widetilde{\gamma} \cdot \widetilde{g}$.
            \item Die Abbildung $\Phi|_{\widetilde{\gamma} \widetilde{V}} \colon \widetilde{\gamma}(\widetilde{V}) \to V$ ist ein Diffeomorphismus:
            
            \[
                \Phi(\widetilde{\gamma} \widetilde{v}) = \Phi(\widetilde{\gamma}) \cdot \Phi(\widetilde{v}) = \Phi \enbrace*{\widetilde{\gamma}^{-1} \widetilde{\gamma} \widetilde{v}} = \enbrace*{\Phi \circ L_{\widetilde{\gamma}^{-1}}} (\widetilde{\gamma} \widetilde{v})
            \]
        \end{itemize}
		Damit ist $e$ schlicht überlagert mittels $V$.
		Für $g \in G$ ist dann $g V$ eine entsprechende Umgebung, sodass $g$ schlicht überlagert ist.
		Die Abbildung $\Phi$ ist außerdem surjektiv nach \autoref{lem:1213}, denn eine offene Umgebung von $e$ liegt im Bild und $\Phi$ ist ein Homomorphismus.\qedhere
    \end{enumerate}
\end{beweis}

\begin{korollar}[label=lem:133,{name=[{Kern einer Überlagerung ist diskrete Untergruppe des Zentrums}]}]
    Es seien $\widetilde{G}$, $G$ zusammenhängende Liegruppen. 
    \begin{enumerate}[1)]
        \item Ist $\Phi \colon \widetilde{G} \to G$ eine Überlagerung von Liegruppen, so ist $\widetilde{\Gamma} = \ker \Phi$ eine diskrete Untergruppe von $Z(\widetilde{G})$, dem Zentrum von $\widetilde{G}$.
        \item Ist $\Gamma$ eine diskrete Untergruppe von $Z(G)$, so ist $\sfrac{G}{\Gamma}$ eine Liegruppe und $\Phi \colon G \to \sfrac{G}{\Gamma}$, $g \mapsto [g]$ eine Überlagerung.
    \end{enumerate}
\end{korollar}
\begin{beweis}
    \begin{enumerate}[1)]
        \item Sei $\widetilde{U}$ eine offene Umgebung von $\widetilde{e}$, sodass $\Phi|_{\widetilde{U}} \colon \widetilde{U} \to \Phi(\widetilde{U}) =: U$ ein Diffeomorphismus ist.
        Dann folgt $\widetilde{\Gamma} \cap \widetilde{U} = \set*{\widetilde{e}}$ und somit ist $\widetilde{\Gamma}$ eine diskrete Untergruppe von $\widetilde{G}$.
        Da $\widetilde{\Gamma}$ der Kern eines Homomorphismus ist, ist $\widetilde{\Gamma}$ außerdem ein Normalteiler in $\widetilde{G}$.
        Sei $\widetilde{g} \in \widetilde{G}$ und $\widetilde{g}(t) \colon [0,1] \to \widetilde{G}$ ein stetiger Weg mit $\widetilde{g}(0)=\widetilde{e}$, $\widetilde{g}(1)=\widetilde{g}$.
        Dann gilt für $\widetilde{\gamma} \in \widetilde{\Gamma}$
        \[
            \widetilde{g}(t) \,\widetilde{\gamma} \,\enbrace{\widetilde{g}(t)}^{-1} \in \widetilde{\Gamma}
        \]
        Da $\widetilde{\Gamma}$ diskret ist, folgt $\widetilde{g}(t)\, \widetilde{\gamma}\, (\widetilde{g}(t))^{-1} \equiv \widetilde{\gamma}$ und wir haben insgesamt
        \[
            \widetilde{g} \, \widetilde{\gamma}=  \widetilde{\gamma}\, \widetilde{g} \enspace \forall \widetilde{g} \in \widetilde{G} \iff \widetilde{\gamma} \in Z(\widetilde{G})
        \]
        \item Wir zeigen dazu zunächst, dass $\Gamma$ frei und \emph{eigentlich diskontinuierlich}\marginnote{einige Autoren sprechen lieber über  \emph{eigentliche Wirkungen}, da eigentlich diskontinuierliche Wirkungen kontinuierlich sind (!)} auf $G$ operiert.
		Dazu müssen wir zeigen:
		\begin{enumerate}[(i)]
			\item \label{enum:proper:1} Für jeden Punkt $g \in G$ existiert eine Umgebung $U$ von $g$, sodass $\gamma.U \cap U \neq \emptyset$ bereits $\gamma=e$ impliziert.
			\item \label{enum:proper:2} Für beliebige $g_1,g_2 \in G$, die nicht im selben Orbit liegen, existieren Umgebung $U$ von $g_1$ und $V$ von $g_2$, sodass $\gamma.U \cap V = \emptyset$ für alle $\gamma \in \Gamma$ gilt.
		\end{enumerate}
		\Cref{enum:proper:1} sorgt dafür, dass $\Phi \colon G \to \sfrac{G}{\Gamma}$ tatsächlich eine Überlagerung ist. 
		\Cref{enum:proper:2} garantiert, dass der Quotient wieder hausdorffsch ist und somit überhaupt die Chance hat, die Struktur einer Mannigfaltigkeit zu sein.
		
		Da $\Gamma$ diskret ist, finden wir eine Umgebung $U$ von $e$ mit $\Gamma \cap U = \set*{e}$.
		Weiter finden wir wieder $V$ mit $e \in V \subset U$ und $V \cdot V^{-1} \subset U$.
		Gilt nun $\gamma v_1 = v_2$ für $v_1, v_2 \in V$, so folgt $\gamma = v_2 \cdot v_1^{-1} \in U$, also $\gamma \in \Gamma \cap U = \set*{e}$.
		Dies zeigt \cref{enum:proper:1}.
		
		Wähle $g_1, g_2 \in G$ mit $g_1 \notin \Gamma g_2$.
		Seien $U$ und $V$ wie eben mit der zusätzlichen Forderungm dass $g_2^{-1} \Gamma g_1 \cap U = \emptyset$, was wir wegen $g_1 \notin \Gamma g_2$ annehmen dürfen.
		Wegen dieser Bedinung und
		\[
			\gamma g_1 v_1 = g_2 v_2 \iff g_2^{-1} \gamma g_1 = v_2 v_1^{-1}
		\]
		folgt, dass $g_1 V$ und $g_2 V$ die gesuchten Umgebungen für \cref{enum:proper:2} sind.
		$\Phi$ ist also tatsächlich eine Überlagerung und $\sfrac{G}{\Gamma}$ eine glatte Mannigfaltigkeit. 
		$\sfrac{G}{\Gamma}$ ist eine Gruppe, da $\Gamma$ im Zentrum liegt und eine Liegruppe, da $\Phi$ ein lokaler Diffeomorphismus ist, wodurch die Gruppenoperationen glatt werden.\qedhere
    \end{enumerate}
\end{beweis}

\begin{beispiel*}[{name=[Zentren der speziellen orthogonalen und unitären Gruppen]}]
	Das Zentrum der speziellen orthogonalen Gruppen ist wie folgt gegeben:
	\[
	    Z \enbrace[\big]{\SO(n)} = \begin{cases}
			\SO(2) &\text{ falls } n=2 \\ 
	        \set*{\id_n} &\text{ falls } n \text{ ungerade} \\
	        \set*{\pm \id_n} &\text{ falls $n$ gerade}
	    \end{cases}
	\]
	Für $n \ge 2$ ist das Zentrum der speziellen unitären Gruppe zyklisch:
	\[
	    Z \enbrace[\big]{\SU(n)} = \mathbb{Z}_n = \set*{e^{2 \pi i \sfrac{k}{n}} \cdot \id_n \given k=0,1,\ldots ,n-1 }
	\]
	Die Fundamentalgruppe ist $\pi_1(\SO(n))= \mathbb{Z}_2$ und die universelle Überlagerung ist $\mathrm{Spin}(n)$.
	Für $n\ge 2$ ist $\pi_1 \enbrace*{\SU(n)} = \set*{e}$ und es gilt $\SU(2) = S^3$.
\end{beispiel*}

\begin{lemma}[label=lem:134,{name=[{Liegruppenhomomorphismus aus Liealgebrenhomomorphismus wenn einfach zusammenhängend}]}]
	Seien $H,G$ zusammenhängende Liegruppen.
	Ist $H$ einfach zusammenhängend und $\psi \colon \Tmap_e H \to \Tmap_e G$ ein Liealgebrenhomomorphismus, so existiert ein eindeutiger Liegruppenhomomorphismus $\Phi \colon H \to G$ mit $(\mathd \Phi)_e = \psi$.
\end{lemma}
\begin{beweis}
	Da der Graph
	\[
		\Graph(\psi) = \set[\big]{\enbrace*{v,\psi(v)} \given v \in \Tmap_e H} \subset \Tmap_e H \oplus \Tmap_e G
	\]
	eine Unteralgebra von $\Tmap_e H \oplus \Tmap_e G$ ist, existiert nach \autoref{satz:125} eine eindeutig bestimmte Lieuntergruppe $A$ von $H \times G$ mit $\Tmap_e A = \Graph(\psi)$.
	Wir bezeichnen mit $\pi_1 \colon A \to H$ und $\pi_2 \colon A \to G$ die Projektionen auf den ersten bzw. zweiten Faktor von $H \times G$.
	Die Abbildungen $\pi_1$ und $\pi_2$ sind Liegruppenhomomorphismen.
	Ferner ist 
	\mapdef{(\mathd \pi_1)_e \colon \Tmap_e A}{\Tmap_e H}{(v,\psi(v))}{v}{}
	ein Isomorphismus.
	Nach \autoref{lem:132} 2) überlagert $A$ somit $H$.
	Da $H$ einfach zusammenhängend ist, folgt $A \cong H$.
	Wir erhalten somit einen Homomorphismus 
	\mapdef{\Phi \colon H = A}{G}{h}{\pi_2(h)}{}
	mit $(\mathd \Phi)_e = (\mathd \pi_2)_e = \psi$.
\end{beweis}

Ist in obiger Situation $H$ \emph{nicht} einfach zusammenhängend, so erhält man einen Homomorphismus $\tilde{\Phi} \colon \tilde{H} \to G$, wobei $\tilde{H}$ eine universelle Überlagerung von $H$ ist.
Dies liefert genau dann einen Homomorphismus $\Phi \colon H \to G$, wenn $\ker \pi \subset \ker \tilde{\Phi}$ ist, wobei $\pi \colon \tilde{H} \to H$ die Überlagerungsabbildung bezeichnet.

Dies ist aber nicht immer richtig:
So kann man zum Beispiel den durch die Identität gegeben Liealgebrenisomorphismus $\psi \colon \mathfrak{h} \to \mathfrak{h}$ betrachten.
Dann ist die induzierte Abbildung $\tilde{\Phi} \colon \tilde{H} \to \tilde{H}$ injektiv, aber $\pi \colon \tilde{H} \to H$ muss nicht injektiv sein.
Ein konkretes Beispiel wäre die universelle Überlagerung von $\SL(n,\mathbb{R})$.

\begin{satz}[{name=[{Liegruppen-Liealgebren-Zusammenhang im einfach zusammenhängenden Fall}]}]
	Es gilt
	\begin{enumerate}[1)]
		\item Zwei zusammenhängende, einfach zusammenhängende Liegruppen mit isomorphen Liealgebren sind isomorph.
		\item Für jede endlich dimensionale Liealgebra $(V, \benbrace*{\cdot,\cdot })$ existiert genau eine zusammenhängende, einfach zusammenhängende Liegruppe $\tilde{G}$ mit Liealgebra $\tilde{\mathfrak{g}}=V$.
		\item Jede zusammenhängende Liegruppe $G$ ist isomorph zu $\sfrac{\tilde{G}}{\tilde{\Gamma}}$, wobei $\tilde{G}$ eine zusammenhängende, einfach zusammenhängende Liegruppe und $\tilde{\Gamma} \subset Z(\tilde{G})$ eine diskrete Untergruppe ist.
	\end{enumerate}
\end{satz}
\begin{beweis}
	\begin{enumerate}[1)]
		\item Sei $\psi \colon \Tmap_e H \to \Tmap_e G$ ein Liealgebrenisomorphismus, $H,G$ zusammenhängende und einfach zusammenhängende Liegruppen.
		Nach \autoref{lem:134} existieren Homomorphismen $\Phi_1 \colon H \to G$ und $\Phi_2 \colon G \to H$ mit
		\[
			(\mathd \Phi_1)_e = \psi \qquad (\mathd \Phi_2)_e = \psi^{-1}
		\]
		Somit ist $\Phi_2 \circ  \Phi_1 \colon H \to G$ ein Liegruppenhomomorphismus mit $\enbrace*{\mathd (\Phi_2 \circ \Phi_1)}_e = \id_{\Tmap_e H}$.
		Nach \autoref{kor:126} folgt nun $\Phi_2 \circ \Phi_1= \id_H$ und somit $\Phi_2 = \Phi_1^{-1}$.
		\item Nach dem Satz von Ado existiert eine Liegruppe $G$ mit Liealgebra $\mathfrak{g}=V$.
		Die universelle Überlagerung $\tilde{G}$ von $G$ ist die gesuchte Liegruppe.
		Die Eindeutigkeit folgt aus 1).
		\item Mit $\widetilde{G}$ als universelle Überlagerung folgt dies direkt aus \autoref{lem:133}.\qedhere
	\end{enumerate}
\end{beweis}

\begin{beispiel*}[{name=[Überlagerung der speziellen linearen Gruppe]}]
	Wir betrachten die spezielle lineare Gruppe in Dimension $2$:\todo{hier könnte man noch die Übungen komplett einarbeiten \ldots}
	\[
		\SL(2,\mathbb{R}) = \set*{\begin{pmatrix}
			a & b \\ c & d
		\end{pmatrix} \given ad - bc =1}
	\]
	ist eine 3-dimensionale Untergruppe von $\GL(n,\mathbb{R})$.
	Es gilt $\SL(2,\mathbb{R}) = \SO(2) \times \mathbb{R}^2$.\marginnote{$\SL(2,\mathbb{R})$ ist homöomorph zum inneren eines soliden Torus, Iwasawa-Zerlegung}
	Somit ist $\pi_1 \enbrace*{\SL(2,\mathbb{R})} = \pi_1 (\SO(2)) = \mathbb{Z}$.
	Es sei $M^3 := D^2 \times \mathbb{R} \cong \mathbb{R}^3$ (diffeomorph) mit
	\[
		D^2 := \set[\big]{z \in \mathbb{C} \given \abs*{z} <1} = \set*{\begin{pmatrix}
			x \\ y
		\end{pmatrix} \in \mathbb{R}^2 \given x^2 +y^2 <1}
	\]
	Wir führen folgende Multiplikation auf $M^3$ ein:
	\[
		(\alpha,\varphi) \cdot (\beta, \psi) := \enbrace*{ \frac{\beta + \alpha \cdot e^{-i \psi}}{1 + \alpha \overline{\beta} e^{-i \psi}}, \varphi + \psi + 2 \cdot \Arg \enbrace*{1 + \alpha \overline{\beta} e^{-i \psi}} }
	\]
	Man prüft leicht nach, dass dies wohldefiniert ist.
	Es gilt zum Beispiel $(\alpha,\varphi) \cdot (0,0) = \enbrace*{\alpha,\varphi} = (0,0) \cdot (\alpha,\varphi)$.
	Das heißt das neutrale Element ist $e=(0,0)$.
	Ferner gilt
	\[
		(\alpha,\varphi) \cdot \enbrace*{- \alpha e^{i \varphi}, -\varphi} = (0,0) = e
	\]
	Somit ist $M^3$ eine einfach zusammenhängende Liegruppe (!).
	Man rechnet leicht nach, dass 
	\begin{align}
		E_1(\alpha,\varphi) &= \enbrace*{\frac{1}{2}  (1- \alpha^2), \Im \alpha} = \enbrace*{\frac{1}{2} (1 -x^2 +y^2), -xy,y} \\
		E_2(\alpha,\varphi) &= \enbrace*{\frac{1}{2} i (1+ \alpha^2), - \Re(\alpha) } = \enbrace*{-xy, \frac{1}{2} \enbrace*{1+x^2 -y^2}, -x} \\
		E_3(\alpha,\varphi) &= \enbrace*{-i \alpha,1} = (y,-x,1)
	\end{align}
	linksinvariante Vektorfelder auf $M^3$ sind mit
	\[
		E_1(0,0) = \enbrace*{\sfrac{1}{2},0,0 } \qquad E_2(0,0) = \enbrace*{0, \sfrac{1}{2}, 0 } \qquad E_3(0,0) = (0,0,1)
	\]
	Dies erhält man zum Beispiel durch
	\[
		E_1(\alpha,\varphi) = \mathd L_{(\alpha,\varphi)} \cdot \enbrace*{\sfrac{1}{2},0,0} = \diffd{}{t}\Big|_{t=0} (\alpha,\varphi) \cdot \enbrace*{\sfrac{1}{2}\cdot t,0,0} = \ldots 
	\]
	Für die Lieklammer gilt
	\[
		\benbrace*{E_1,E_2} = - E_3 \quad \benbrace*{E_3,E_1} = E_2 \quad \benbrace*{E_2,E_3} =E_1
	\]
	Man erhält nun einen Liealgebrenisomorphismus zur Liealgebra von $\SL(2,\mathbb{R})$ 
	\[
		\overline{E}_1 := \frac{1}{2} \begin{pmatrix}
			1 & 0 \\ 0 & -1
		\end{pmatrix} \quad 
		\overline{E}_2 := \frac{1}{2}  \begin{pmatrix}
			0 & 1 \\ 1 & 0
		\end{pmatrix} \quad 
		\overline{E}_3 := \frac{1}{2} \begin{pmatrix}
			0 & -1 \\ 1 & 0
		\end{pmatrix} 
	\]
\end{beispiel*}
% section 13 (end)

\section{Die Exponentialabbildung} % (fold)
\label{sec:14}

\begin{definition}[{name=[{Einparameteruntergruppe}]}]
	Sei $G$ eine Liegruppe.
	Dann nennt man einen Homomorphismus $\Phi \colon (\mathbb{R},+) \to G$ eine \Index{Einparameteruntergruppe} (von $G$).
\end{definition}

\begin{beispiel*}[{name=[Einparametergruppe der allgemeinen linearen Gruppe]}]
	$G = \GL(n;\mathbb{R})$, $v \in \Tmap_e \GL(n,\mathbb{R}) = \Mat(n,\mathbb{R})$.
	\[
		\Phi_v \colon \mathbb{R} \longrightarrow G \qquad t \longmapsto \exp(t \cdot v) = \sum_{k=0}^{\infty}  \frac{(tv)^k}{k!} 
	\]
	Es gilt $\Phi_v(s+t) = \Phi_v(s) \cdot \Phi_v(t) = \Phi_v(t) \cdot \Phi_v(s)$ für alle $s,t \in \mathbb{R}$, wie man sich leicht überlegt.
\end{beispiel*}

Zurück zum allgemeinen Fall: Sei $v \in \Tmap_e G \setminus \set*{0}$.
Wegen $\Tmap_0 \mathbb{R} \cong \mathbb{R}$ ist
\[
	\psi_v \colon \mathbb{R} \to \Tmap_e G \quad t \mapsto t \cdot v
\]
ein Liealgebrenhomomorphismus.
Da $\mathbb{R}$ einfach zusammenhängend ist, existiert nach \autoref{lem:134} ein eindeutig bestimmter Homomorphismus
\(
	\Phi_v \colon \mathbb{R} \to G
\)
mit $(\mathd \Phi_v)_0 = \psi_v$ und $\Phi_v'(0)= v$.

\begin{lemma}[label=lem:142,{name=[Verträglichkeit mit \cref{lem:127}]}]
	Sei $G$ eine Liegruppe mit Exponentialabbildung\index{Exponentialabbildung} $\exp \colon \Tmap_e G \to G$.
	Dann gilt $\Phi_v(1) = \exp(v)$ für alle $v \in \Tmap_e G$.
\end{lemma}
\begin{beweis}
	Wir hatten in \autoref{lem:127} gezeigt, dass $c_v(s +t) = c_v(s) \cdot c_v(t)$ für alle $t,s \in \mathbb{R}$ gilt, wobei $c_v$ die Integralkurve von $X_v$ ist mit $X_v(e)=v$.
	Somit gilt
	\[
		\exp \enbrace[\big]{(s+t) v} = c_{(s+t)v}(1) = c_v(s+t) = c_v(s) \cdot c_v(t) = c_{sv}(1) \cdot c_{tv}(1) = \exp(sv) \cdot \exp(tv)
	\]
	Damit ist $t \mapsto \exp(tv)$ eine Einparametergruppe mit $\diffd{}{t}\big|_{t=0} \exp(tv)=v$.
	Die Eindeutigkeit von $\Phi_v$ impliziert $\Phi_v(t)=\exp(tv)$.
\end{beweis}

\begin{lemma}[label=lem:143,{name=[Eigenschaften der Exponentialabbildung]}]
	Es gilt
	\begin{enumerate}[(1),itemsep=0pt]
		\item Für alle $v \in \Tmap_eG$ ist $\Phi_v(t)=\exp(tv)$ eine Einparamtergruppe von $G$ mit $\Phi'_v(0)=v$.
		\item Sei $g \in G$ und $v \in \Tmap_eG$.
		Dann ist $L_g(\Phi_v(t))$ Integralkurve von $X_v$.
		\item Ist $\Phi \colon H \to G$ ein Homomorphismus, so gilt für alle $w \in \mathfrak{h}$ für alle $t \in \mathbb{R}$\marginnote{$\exp$ ist natürlich!}
		\[
			\Phi \enbrace*{\exp_H(tw)} = \exp_G \enbrace[\big]{t \enbrace*{\mathd \Phi}_e \cdot w}
		\]
		\item Ist $H$ eine Lieuntergruppe von $G$, so gilt
		\[
			\Tmap_e H = \set*{v \in \Tmap_e G \given \exp_G(tv) \in H \enspace \forall t \in \mathbb{R}}
		\]
	\end{enumerate}
\end{lemma}
\begin{beweis}
	\begin{enumerate}[(1),itemsep=0pt]
		\item Bereits in \autoref{lem:142} gezeigt.
		\item Klar, denn $\exp(tv)=c_v(t)$.
		\item Homomorphismen bilden Einparametergruppen auf Einparametergruppen ab:
		\[
			\Phi \enbrace[\big]{h(s+t)} = \Phi\enbrace[\big]{h(s) \cdot h(t)} = \Phi \enbrace[\big]{h(s)} \cdot \Phi \enbrace[\big]{h(t)}
		\]
		Also definiert $\psi(t) = \Phi \enbrace*{\exp_H(tw)}$ eine Einparametergruppe mit $\psi'(0) = (\mathd \Phi)_e (w)$.
		Wegen der Eindeutigkeit muss somit $\psi(t) = \exp_G \enbrace*{t (\mathd \Phi)_e w}$ gelten.
		\item Sei $i \colon H \to G$ die Inklusion.
		Dann ist 
		\[
			\exp_H(tw) = i \enbrace*{\exp_H(tw)} \StackText{(3)}{=} \exp_G \enbrace[\big]{(\mathd i)_e \cdot tw} = \exp_G(tw)\qedhere
		\]
	\end{enumerate}
\end{beweis}

\begin{beispiel*}[{name=[Liealgebra von $\SO(n)$]}]
	Es gilt $\Tmap_e\SO(n) = \set*{V \in \Mat_n(\mathbb{R}) \given V^T =-V}$; die Abgeschlossenheit folgt mit
	\[
		\benbrace*{V_1,V_2}^T = \enbrace*{V_1 V_2 - V_2 V_1}^T = V_2 V_1 - V_1 V_2 = -\benbrace*{V_1,V_2}
	\]
	Aus (4) folgt nun $\exp(tv) \in \SO(n)$ für alle $t \in \mathbb{R}$.
\end{beispiel*}

\begin{bemerkung*}[{name=[weitere Eigenschaften der Exponentialabbildung]}]
	Weitere zentrale Eigenschaften der Exponentialabbildung:
	\begin{enumerate}[1)]
		\item Aus (4) folgt: Ist $H$ eine Lieuntergruppe von $G$, so gilt $\exp_H  = \exp_G\!\big|_{\Tmap_e H} \colon \Tmap_e H \to H$.
		\item Die Exponentialabbildung $\exp \colon \Tmap_e G \to G$ ist nicht immer surjektiv; dies ist allerdings richtig für kompakte zusammenhängende Liegruppen.
		Es gilt aber $\im \enbrace*{\exp (\Tmap_e G)} \supset U \supset \set*{e}$ für eine offene Umgebung von $e$ wegen dem Umkehrsatz und \cref{lem:129}.
		\item Es gilt $\exp(x)^{-1} = \exp(-x)$.
		\item Der Fluss eines linksinvarianten Vektorfeldes $X_v$ mit $X_v(g) = \enbrace*{\mathd L_g}_e \cdot v$ ist gegeben durch
		\[
			\enbrace*{R_{\exp(tv)}}_{t \in \mathbb{R}}
		\]
	\end{enumerate}
\end{bemerkung*}
% section 14 (end)

\section{Die adjungierte Darstellung} % (fold)
\label{sec:15}
Sei $G$ eine Liegruppe und $g \in G$ ein festes Gruppenelement. Betrachte die \Index{Konjugation}
\mapdef{i_g \colon G}{G}{\tilde{g}}{m \enbrace[\big]{m \enbrace*{g,\tilde{g}}, I(g)}= g \cdot \tilde{g} \cdot g^{-1}}{}
wobei bekanntermaßen $m(g,\tilde{g})= g \cdot \tilde{g}$ und $I(g)=g^{-1}$. 
Somit gilt $i_g = L_g \circ R_{g^{-1}} = R_{g^{-1}} \circ L_g$.
Wir hatten bereits gesehen, dass $i_g \colon G \to G$ ein Liegruppenhomomorphismus ist.
Folglich ist 
\[
	\Ad(g) := \enbrace*{\mathd i_g}_e \colon \Tmap_e G \longrightarrow \Tmap_eG = \mathfrak{g}
\]
ein Liealgebrenhomomorphismus.
Wir hatten auch gezeigt, dass aus $i(g_1 \cdot g_2) = i(g_1) \cdot i(g_2)$ die Identität 
\[
	\Ad(g_1 \cdot g_2) = \Ad(g_1) \circ \Ad(g_2)
\]
folgt (Beispiel nach \cref{lem:122}).

\begin{definition}[{name=[adjungierte Darstellung der Liegruppe]}]
	Sei $G$ eine Liegruppe.
	Dann nennt man 
	\[
		\Ad \colon G \to \GL(\mathfrak{g}) \qquad g \longmapsto \Ad(g)
	\]
	die \Index{adjungierte Darstellung} der Liegruppe $G$.
\end{definition}

Ist zum Beispiel $G$ abelsch, so ist $i_g = \id_G$ und es folgt ${\Ad(g)} = \id_{\mathfrak{g}}$ für alle $g \in G$.
Ist $G=S^3$, so wurde in den Übungen (Aufgabe 3, siehe \cref{sec:aufg3}) die adjungierte Darstellung von $G$ bestimmt:
\[
	\Tmap_e S^3 = \im \mathbb{H} = \Span_\mathbb{R} (i,j,k) =: \mathbb{R}^3
\]
Es gilt $\Ad(g)(x) = g \cdot x \cdot g^{-1} \in \mathbb{R}^3$.
Es wurde auch $\Im \enbrace*{\Ad|_{S^3}} = \SO(3)$ gezeigt.

\begin{definition}[{name=[adjungierte Darstellung der Liealgebra]}]
	Sei $G$ eine Liegruppe mit Liealgebra $\mathfrak{g}$.
	Dann bezeichnen wir mit 
	\[
		\ad_v \colon \mathfrak{g} \to \mathfrak{g} \qquad w \longmapsto \benbrace*{v,w}
	\]
	die \Index{adjungierte Darstellung} der Liealgebra $\mathfrak{g}$.
\end{definition}

Die Jacobi-Identität\index{Jacobi-Identität} 
\[
	\benbrace[\big]{X,\benbrace*{Y,Z}} + \benbrace[\big]{Z, \benbrace*{X,Y}} + \benbrace[\big]{Y, \benbrace*{Z,X}} =0
\]
für alle $X,Y,Z$ kann nun folgendermaßen beschrieben werden:
\[
	\ad_{\benbrace*{X,Y}} = \ad_X \circ \ad_Y - \ad_Y \circ \ad_X = \benbrace[\big]{\ad_X, \ad_Y}_0 \marginnote{Kommutator}
\]
Folglich ist $\ad \colon \enbrace*{\mathfrak{g}, \benbrace*{\cdot ,\cdot }} \to \enbrace[\big]{\mathfrak{gl}(\mathfrak{g}), \benbrace*{\cdot ,\cdot }_0}$
ein Liealgebrenhomomorphismus.

\begin{lemma}[{name=[Eigenschaften von $\Ad$ und $\ad$]},label=lem:153]
	Sei $G$ eine Liegruppe.
	Dann gilt
	\begin{enumerate}[(1)]
		\item $\enbrace*{\mathd \Ad}_{e} = \ad$
		\item $i_g \enbrace[\big]{\exp(v)} = \exp \enbrace[\big]{\Ad(g)v}$
		\item $\Ad \enbrace[\big]{\exp(v)} = e^{\ad_v} = \sum_{k=0}^{\infty} \frac{(\ad_v)^k}{k!} \colon \mathfrak{g} \to \mathfrak{g}$
		\item Ist $G$ zusammenhängend, so gilt $Z(G) = \ker (\Ad)$
	\end{enumerate}
\end{lemma}
\begin{beweis}
	\begin{enumerate}[(1)]
		\item Erinnerung: Ist $M^n$ eine differenzierbare Mannigfaltigkeit, $p \in M^n$ ein Punkt, $X,Y$ glatte Vektorfelder und $\enbrace*{\psi_t}_{\abs*{t}<\varepsilon}$ lokaler Fluss von $X$ in einer Umgebung von $p$.
		Dann gilt
		\[
			\benbrace*{X,Y}(p) = \diffd{}{t}\Big|_{t=0} \enbrace*{\mathd \psi_{-t}}_{\psi_t(p)} Y \enbrace[\big]{\psi_t(p)}
		\]
		Dies ist die \Index{Lie-Ableitung} von $Y$ entlang $X$. 
		Es folgt für alle $v,w \in \mathfrak{g}$
		\begin{align}
			\enbrace[\big]{\mathd \enbrace*{\Ad}_{e} \cdot v} (w) = \diffd{}{t}\Big|_{t=0} \!\Ad \enbrace[\big]{\exp(tv)}(w) &= \diffd{}{t}\Big|_{t=0} \enbrace*{\mathd R_{\exp(-tv)}}_{\exp{tv}} \cdot  \enbrace*{\mathd L_{\exp tv}}_{e} \cdot w \\
			&= \diffd{}{t}\Big|_{t=0} \enbrace*{\mathd R_{\exp(-tv)}}_{\exp{tv}} \cdot X_w \enbrace[\big]{\exp(tv)} \\
			&= \benbrace*{X_v,X_w}(e)  = \benbrace*{v,w}_{\mathfrak{g}} \\
			&= \ad_v(w)
		\end{align}
		Dabei haben wir benutzt, dass der Fluß von $X_v$ gegeben ist durch $R_{\exp(tv)}$.
		\item Folgt direkt aus \autoref{lem:143} (3).
		\item Folgt ebenfalls direkt aus \autoref{lem:143} (3).
		\item Zur Inklusion $Z(G) \subseteq ker (\Ad)$: Ist $g \in Z(g)$, so ist $i_g = {\id_G} $ und es folgt sofort $\Ad(g)=\id$.
		
		Es gelte andersrum $\Ad(g) = \id_{\mathfrak{g}}$.
		Dann sind $i_g, \id_G \colon G \to G$ Liegruppenhomomorphismen mit $\enbrace*{\mathd i_g}_e = (\mathd\id_G)_e$.
		Da $G$ zusammenhängend ist, folgt aus \autoref{kor:126} $i_g=\id_G$ und es folgt $g \in Z(G)$.\qedhere 
	\end{enumerate}
\end{beweis}

\begin{beispiel*}[{name=[Adjungierte Darstellung der allgemeinen linearen Gruppe]}]
	Betrachte $G= \GL(n,\mathbb{R})$.
	Dann gilt
	\[
		\begin{array}{ll}
			i_A(\tilde{A}) = A \cdot \tilde{A} \cdot A^{-1} \quad & \tilde{A} \in \GL(n,\mathbb{R}) \\
			\Ad(A)(B) = A \cdot B \cdot A^{-1} & B \in \Mat(n,\mathbb{R}) \\
			\ad_v(W) = V \cdot W + W (-V) = \benbrace*{V,W}_0 \quad 
		\end{array}
	\]
\end{beispiel*}

\begin{lemma}[label=lem:154,{name=[über abelsche Liegruppen, Normalteiler und das Zentrum]}]
	Sei $G$ eine zusammenhängende Liegruppe. Dann gilt
	\begin{enumerate}[1)]
		\item $G$ ist abelsch genau dann, wenn $\Tmap_e G = \mathfrak{g}$ abelsch ist.
		\item $G$ ist genau dann abelsch, wenn $G \cong T^k \times \mathbb{R}^{n-k}$.
		\item Ist $H$ eine zusammenhängende Untergruppe von $G$, so ist $H$ Normalteiler genau dann, wenn $\mathfrak{h} = \Tmap_e H$ Ideal von $\mathfrak{g}$ ist, das heißt wenn $\benbrace*{\mathfrak{g},\mathfrak{h}} \subset \mathfrak{h}$ gilt.
		\item Das Zentrum $Z(G)$ ist Lieuntergruppe von $G$ mit 
		\[
			\Tmap_e Z(G) = Z(\mathfrak{g}) = \set*{v \in \mathfrak{g} \given \ad_v =0}
		\]
		\item Ist $H$ eine Lieuntergruppe von $G$, so ist die Einschränkung der adjungierten Darstellung von $G$ auf $H$ gerade die adjungierte Darstellung von $H$.
	\end{enumerate}
\end{lemma}
\begin{beweis}
	\begin{enumerate}[1)]
		\item Sei $G$ abelsch. Dann gilt $i_g=\id_G$, also $\Ad(g) = \id_{\mathfrak{g}}$ für alle $g \in G$ und es folgt $\ad_v=0$. Damit ist auch die Lieklammer trivial und $\mathfrak{g}$ ist abelsch.
		
		Sei andersrum $\mathfrak{g}$ abelsch: Dann ist $\ad_v=0$ für alle $v \in \mathfrak{g}$. Es gilt $\id_{\mathfrak{g}} = e^{\ad_v} = \Ad \enbrace*{\exp(v)}$ und es folgt $\Ad(g) = \id_{\mathfrak{g}} $ für alle $g \in G$.
		\[
			i_g \enbrace[\big]{\exp(v)} =  g \exp(v) g^{-1} = \exp \enbrace[\big]{\Ad(g)v} = \exp(v)
		\]
		Somit ist $G$ abelsch.
		\item Da $G$ abelsch ist, ist $\mathfrak{g}$ abelsch und somit ist $\exp \colon (\Tmap_eG,+) \to (G,\cdot )$ ein Liegruppenhomomorphismus (wenn die Lieklammer verschwindet gilt die Funktionalgleichung, siehe \cref{sec:funktionalgleichung}).
		Somit ist $\exp \colon \Tmap_e G \to G$ eine Überlagerung und $\Gamma := \exp^{-1}(e)$ eine diskrete Untergruppe von $(\Tmap_e G,+)$.\marginnote{Gitter}
		Somit existieren $v_1, \ldots, v_k \in \mathfrak{g}$ mit
		\[
			\Gamma = \set*{\sum\nolimits_{i=1}^{k} \lambda_i v_i \given \lambda_i \in \mathbb{Z}} \cong \mathbb{Z}^k
		\]
		Sei $V := \Span_\mathbb{R}(v_1,\ldots ,v_k)$ und $V^\bot$ ein Komplement von $V$ in $\mathfrak{g}$.
		Es gilt also $\mathfrak{g} = V \oplus V^\bot$ (beide Grenzfälle sind möglich).
		Es ist nun \enquote{klar}, dass 
		\[
			\exp(V) = T^k
		\]
		und $\exp(V^\bot) = \mathbb{R}^{n-k}$ gilt. 
		Damit folgt die Behauptung.
		\item Sei $H$ ein zusammenhängender Normalteiler von $G$, das heißt es gilt $g H g^{-1} \subset H$ für alle $g \in G$.
		Es folgt $i_g(H) \subset H$ und damit auch $\Ad(g)(\Tmap_eH) \subset \Tmap_eH$ für alle $g \in G$.
		Damit gilt auch $\ad_v(\Tmap_e H) \subset \Tmap_e H$, also $\benbrace*{\mathfrak{g},\mathfrak{h}}\subset \mathfrak{h}$.
		
		Ist umgekehrt $\mathfrak{h} = \Tmap_e H$ ein Ideal von $\mathfrak{g}$, so gilt $\ad_v(\mathfrak{h}) \subset \mathfrak{h}$ für alle $v \in \mathfrak{g}$.
		Dann gilt auch $\Ad \enbrace*{\exp (v)}(\mathfrak{h}) = e^{\ad_v}(\mathfrak{h}) \subset \mathfrak{h}$.
		Damit gilt auch $\Ad(g)(\mathfrak{h}) \subset \mathfrak{h}$ für alle $g \in G$.
		\begin{align}
			H \ni \exp_G \enbrace[\big]{\Ad(g)(w)} \StackText{\ref{lem:153}}{=} i_g \enbrace[\big]{\exp(w)}= g \cdot \exp(w) \cdot g^{-1} 
		\end{align}
		für alle $g \in G$ und $w \in \mathfrak{h}$.
		Damit ist $H$ ein Normalteiler.
		\item Nach \autoref{satz:1210} genügt es zu zeigen, dass $Z(G)$ abgeschlossen ist, um zu beweisen, dass $Z(G)$ eine Lieuntergruppe ist.
		Betrachte $\varphi \colon G \to G$, $g \mapsto g h g^{-1}$ für ein festes $h \in G$. 
		Dann ist $Z(\set*{h}) = \varphi^{-1}(\set*{h})$, also sind Zentralisatoren\index{Zentralisator} als Schnitte von Mengen der Form $Z(\set*{h})$ abgeschlossen. 
		Insbesondere ist das Zentrum abgeschlossen.
		
		Bezüglich der Gleichung: $Z(\mathfrak{g}) = \set*{v \in \mathfrak{g} \given \ad_v =0}$ gilt per Definition.
		Nach \autoref{lem:153} (4) gilt $Z(G) = \ker \Ad$.
		Nach \autoref{lem:143} (4) enthält $\Tmap_e Z(G)$ genau die Elemente aus $\mathfrak{g}$ mit $\exp(tv) \in Z(G)$ für alle $t \in \mathbb{R}$.
		Damit folgt
		\[
			 \id_{\mathfrak{g}}=\Ad (\exp tv) =  e^{\ad_{tv}}
		\]
		was äquivalent zu $\ad_{tv}=0$ ist.
		\item Sei $\iota \colon H \hookrightarrow G$ die Inklusion.
		Für $h \in H$ und $v \in \mathfrak{h}$ gilt
		\[
			(\mathd \iota)_{e_H} \enbrace[\big]{\Ad_H(h)v} = \enbrace[\big]{\mathd(\iota \circ i_h)}_{e_H} v=\enbrace[\big]{\mathd(i_h \circ \iota)}_{e_H} = (\mathd i_h)_{e_G} v = \Ad_G(h) v
		\]
		Hier fließt ein, dass $\mathd \iota$ injektiv ist.\qedhere
	\end{enumerate}
\end{beweis}

\begin{frage}[{name=[Sind Normalteiler abgeschlossen?]}]
	Wann sind Normalteiler abgeschlossene Untergruppen?
	Beispielsweise ist für $r$ irrational
	\[
		S^1_r = \set*{\enbrace*{e^{i \varphi}, e^{i r \varphi}} \in T^2}
	\]
	ein Normalteiler, aber $\overline{S^1_r}=T^2$ und $S^1_r$ damit nicht abgeschlossen.
\end{frage}
% section 15 (end)

\section{Automorphismen von Liegruppen} % (fold)
\label{sec:16}
\begin{definition}[{name=[Automorphismen und Derivationen]}]
	Sei $\mathfrak{g}$ eine Liealgebra.
	\begin{enumerate}[(i)]
		\item Man nennt einen linearen Isomorphismus $A \colon \mathfrak{g} \to \mathfrak{g}$, der gleichzeitig ein Liealgebrenhomomorphismus ist, also
		\[
			A \benbrace*{x,y} = \benbrace*{Ay,Ay}
		\]
		für alle $x,y \in \mathfrak{g}$ erfüllt, einen \Index{Automorphismus} der Liealgebra $\mathfrak{g}$.
		Die Menge der Automorphismen bezeichnen wir mit $\Aut(\mathfrak{g}) \subset \GL(\mathfrak{g})$.
		\item Eine lineare Abbildung $A \colon \mathfrak{g} \to \mathfrak{g}$ nennt man \Index{Derivation}, falls 
		\[
			A \benbrace*{x,y} = \benbrace*{Ax,y} + \benbrace*{x,Ay}
		\] 
		für alle $x,y \in \mathfrak{g}$ gilt. Bezeichnung: $\Der(\mathfrak{g}) \subset \End(\mathfrak{g})$.
	\end{enumerate}
\end{definition}

\begin{proposition}[label=prop:162,{name=[Automorphismen sind Liegruppe, Derivationen die zugehörige Liealgebra]}]
	Sei $\mathfrak{g}$ eine Liealgebra.
	Dann ist $\Aut(\mathfrak{g})$ eine abgeschlossene Lieuntergruppe von $\GL(\mathfrak{g})$ mit Liealgebra $\Der(\mathfrak{g})$
\end{proposition}
\begin{beweis}
	Ist $A$ ein Automorphismus von $(\mathfrak{g},[\cdot ,\cdot ])$, so gilt für alle $x,y \in  \mathfrak{g}$
	\[
		A \benbrace[\big]{x,y} = \benbrace[\big]{A x, A y}
	\]
	Man überlegt sich nun leicht, dass $\Aut(\mathfrak{g})$ deshalb eine abgeschlossene Untergruppe und daher eine Lieuntergruppe von $\GL(\mathfrak{g})$ ist.
	Sei nun $A(t)\big|_{t \in (-\varepsilon,\varepsilon)}$ eine differenzierbare Kurve von Automorphismen mit $A(0) = {\id_{\mathfrak{g}}}$.
	Differenzieren der Identität
	\(
		A(t) \benbrace*{x,y} = \benbrace*{A(t) x, A(t) y}
	\)
	in $t=0$ liefert wegen den Produktregel
	\[
		A'(0) \benbrace[\big]{x,y} = \benbrace[\big]{A'(0) x,y} + \benbrace[\big]{x, A'(0)y}
	\]
	Also ist $A'(0) \in \Der(\mathfrak{g})$.
	Sei umgekehrt $V \in \Der(\mathfrak{g})$.
	Wir müssen zeigen, dass die Integralkurve $e^{t V}$, die a priori in $\GL(\mathfrak{g})$ liegt, tatsächlich in $\Aut(\mathfrak{g})$ liegt.
	Dann gilt\marginnote{$V \in \Der(\mathfrak{g}) \Rightarrow e^{tV} \in \GL(\mathfrak{g})$}
	\begin{align}
		\diffd{}{t}\Big|_{t=t_0} e^{-tV} \benbrace*{e^{tV}x, e^{tV}y} &= - e^{t_0 V} V \benbrace*{e^{t_0 V} x, e^{t_0 V} y} + e^{t_0 V} \benbrace*{V e^{t_0 V} x, e^{t_0 V} y} + e^{t_0 V} \benbrace*{e^{t_0 V} x , V e^{t_0 V} y} \\[-1em]
		\shortintertext{mit $z := e^{t_0 V} x$ und $w := e^{t_0 V}y$}
		&= - e^{t_0 V} \enbrace[\big]{V \benbrace*{z,w} - \benbrace*{Vz,w} - \benbrace*{z,Vw}} \stackrel{V \in \Der(\mathfrak{g})}{=\joinrel=\joinrel=} 0
	\end{align}
	Somit ist wegen $e^{tV}={\id}$ für $t_0=0$
	\[
		e^{-tV} \benbrace*{e^{tV}x, e^{tV}y} \equiv \benbrace[\big]{x,y}
	\]
	für alle $t \in (-\varepsilon,\varepsilon)$.
	Wir erhalten also $\benbrace*{e^{tV}x, e^{tV}y}= e^{tV}\benbrace[\big]{x,y}$ und damit $e^{tV} \in \Aut(\mathfrak{g})$.
\end{beweis}

Aus \autoref{prop:162} folgt, dass $\Ad \colon G  \to \Aut(\mathfrak{g})$, $g \mapsto \Ad(g)$ ein Liegruppenhomomorphismus ist. 
Damit ist $\ad \colon \mathfrak{g} \to \Der(\mathfrak{g})$, $v \mapsto \ad_v$ ein Liealgebrenhomomorphismus.
Insbesondere gilt sogar $\ad_v \in \Der(\mathfrak{g})$ für alle $v \in \mathfrak{g}$ aufgrund der Jacobi-Identität.

\begin{definition}[{name=[innere Derivationen und innere Automorphismen]}]
	Sei $\mathfrak{g}$ eine Liealgebra
	\begin{enumerate}[1)]
		\item Eine Derivation $A \in \Der(\mathfrak{g})$ nennt man \Index{innere Derivation}, falls $v \in \mathfrak{g}$ existiert mit $A = \ad_v$.
		Wir schreiben 
		\[
			\intAlg(\mathfrak{g}) \coloneqq \set[\big]{\ad_v \given v \in \mathfrak{g}}
		\] 
		\item Wir bezeichnen mit $\Int(\mathfrak{g}) \subset \Aut(\mathfrak{g})$ die zusammenhängende Lieuntergruppe\marginnote{nach \autoref{satz:125}} von $\Aut(\mathfrak{g})$ mit Liealgebra $\intAlg(\mathfrak{g})$.
		Elemente in $\Int(\mathfrak{g})$ nennt man \Index{innere Automorphismen}.
	\end{enumerate}
\end{definition}

Ist $G$ eine zusammenhängende Liegruppe, so gilt $\Ad(G) = \Int(\mathfrak{g})$, denn $\Ad \enbrace*{\exp v} = e^{\ad_{v}}$ impliziert, dass beide in einer Umgebung von $e$ übereinstimmen und somit die gleiche Liealgebra $\intAlg(\mathfrak{g})$ haben.
% sowohl $\Ad(G)$ also auch $\Int(\mathfrak{g})$ sind Lieuntergruppen von $\Aut(\mathfrak{g})$ mit Liealgebra $\intAlg(\mathfrak{g})$.

\begin{beispiel*}[{name=[Automorphismen einer abelsche Liealgebra]}]
	Ist $\mathfrak{t}$ eine abelsche Liealgebra, so ist jeder Isomorphismus $A \colon \mathfrak{t} \to \mathfrak{t}$ ein Automorphismus.
	Nichttriviale innere Automorphismen existieren in diesem auch nicht.
\end{beispiel*}

\begin{lemma}[{name=[Adjungierte Darstellung surjektiv auf innere Automorphismen (zusammenhängend)]},label=lem:164]
	Sie $G$ eine zusammenhängende Liegruppe mit Liealgebra $\mathfrak{g}$.
	Dann ist $\Ad \colon G \to \Int(\mathfrak{g})$ ein surjektiver Gruppenhomomorphismus mit Kern $Z(G)$.
	Es gilt also
	\[
		\Int(\mathfrak{g}) \cong \sfrac{G}{Z(G)} \quad \text{ und } \quad \intAlg(\mathfrak{g}) = \mathfrak{g} \ominus Z(\mathfrak{g})
	\]
\end{lemma}
\begin{beweis}
	Zur Surjektivität: Sei $V \in \Int(\mathfrak{g})$.
	Wir können $V= e^W$ für ein $W \in \intAlg(\mathfrak{g})$ annehmen.
	Weiter existiert nach Definition ein $w \in \mathfrak{g}$ mit $\ad_w = W$.
	Es folgt
	\[
		V = e^{\ad_w} \StackText{\ref{lem:153} (3)}{=\joinrel=\joinrel=} \Ad(\exp w)
	\]
	Weiter gilt $Z(G)= \ker \Ad$ bereits nach \autoref{lem:153} (4).
\end{beweis}

\begin{lemma}[label=lem:165,{name=[Menge der inneren Automorphismen ist Normalteiler]}]
	Sei $G$ eine zusammenhängende Liegruppe.
	Dann ist $\Int(\mathfrak{g})$ ein Normalteiler von $\Aut(\mathfrak{g})$.
\end{lemma}
\begin{beweis}
	Nach Definition ist $\Int(\mathfrak{g})$ zusammenhängend und nach \autoref{lem:154} (3) ist somit $\Int(\mathfrak{g})$ genau dann Normalteiler, wenn $\Tmap_e \Int(\mathfrak{g}) = \intAlg(\mathfrak{g})$ ein Ideal von $\Der(\mathfrak{g})= \aut(\mathfrak{g}) = \Tmap_e \Aut(\mathfrak{g})$ ist.
	Sei $L$ eine Derivation und $\ad_v \in \intAlg(\mathfrak{g})$.
	Dann gilt
	\begin{align}
		\benbrace[\big]{L,\ad_v}_0 (w) = \enbrace[\big]{L \circ \ad_v - \ad_v \circ L}(w) = L \benbrace*{v,w}- \benbrace*{v,Lw} \StackText{\ref{prop:162}}{=} \benbrace*{Lv, w} + \benbrace*{v,Lw} - \benbrace*{v,Lw} 
		= \ad_{Lv}(w)
	\end{align}
	Somit ist $\benbrace*{L, \ad_v} = \ad_{Lv} \in \intAlg(\mathfrak{g})$ für alle $L \in \Der(\mathfrak{g})$.
\end{beweis}

\begin{lemma}[{name=[zusammenhängender Normalteiler]}]
	Sei $G$ eine zusammenhängende, einfach zusammenhängende Liegruppe.
	Dann gilt: Ist $H$ ein zusammenhängender Normalteiler von $G$, so ist $H$ eine abgeschlossene Untergruppe.
\end{lemma}
\begin{beweis}
	Nach \autoref{lem:154} (3) ist $\mathfrak{h} \coloneqq \Tmap_e H$ ein Ideal von $\mathfrak{g}$.
	Wir bezeichnen mit $\mathfrak{k} \coloneqq \sfrac{\mathfrak{g}}{\mathfrak{h}}$ die Quotientenliealgebra(!) und mit $K$ die einfach zusammenhängende Liegruppe mit Liealgebra $\mathfrak{k}$.
	Der Liealgebrenhomomorphismus $\psi \colon \mathfrak{g} \to \mathfrak{k}$, $v \mapsto [v]$ kann zu einem Liegruppenhomomorphismus $\phi$ \enquote{fortgesetzt} werden (nach \autoref{lem:134}, da $G$ einfach zusammenhängend).
	Somit ist $\ker \phi$ eine abgeschlossene Untergruppe von $G$ mit Liealgebra $\ker \psi= \mathfrak{h}= \Tmap_e H$.
	Es folgt $H= \ker (\phi)$, das heißt $H$ ist abgeschlossen.
\end{beweis}

Achtung! $S^1_r \subset T^2 = S^1 \times S^1$ ist Normalteiler mit $\overline{S^1_r}= T^2$, die Eigenschaft einfach zusammenhängend zu sein, ist also von zentraler Wichtigkeit.

\begin{definition}[{name=[Killingform]}]
	Sei $\mathfrak{g}$ eine Liealgebra (über $\mathbb{R}$ oder $\mathbb{C}$).
	Dann nennt man 
	\mapdef{\mathcal{B} \colon \mathfrak{g} \times \mathfrak{g}}{\mathbb{K}}{(v,w)}{\tr_{\mathfrak{g}} \enbrace*{{\ad_v} \circ {\ad_w}}}{}
	die \Index{Killingform} von $\mathfrak{g}$.
\end{definition}

\begin{beispiel*}[{name=[spezielle orthogonale Gruppe]}]
	Wir betrachten $G=\SO(n)$.
	Dann ist 
	\[
		\mathfrak{so}(n) = \Tmap_e\SO(n) = \set*{V \in \Mat(n,\mathbb{R}) \given V^T=-V}
	\]
	Wir führen nun folgende Identifikation der Basisvektoren durch
	\[
		\begin{psmallmatrix}
			0 & & 1 \\
			& \ddots & \\
			-1 & & 0
		\end{psmallmatrix} \,\longleftrightarrow\, e_i \wedge e_j \in \Lambda^2(\mathbb{R}^n)
	\]
	wobei $i > j$ und $-1$ an der Stelle $(i,j)$ steht, $1$ an der Stelle $(j,i)$.
	Versieht man $\mathfrak{g} = \mathfrak{so}(n)$ mit dem Skalarprodukt
	\[
		\skal*{v}{w} = - \frac{1}{2}  \cdot \tr(v \cdot w) = \frac{1}{2} \tr(v^T \cdot w)
	\]
	und $\Lambda^2(\mathbb{R}^n)$ mit dem Standardskalarprodukt\marginnote{$(e_1,\ldots ,e_n)$ ONB von $\enbrace*{\mathbb{R}^n, \skal*{\cdot}{\cdot}_{\mathrm{std}}}$}, so ist
	\[
		e_1 \wedge e_2, e_1 \wedge e_3, \ldots , e_1 \wedge e_n, \quad  e_2 \wedge e_3, \ldots , e_{n-1} \wedge e_n
	\]
	ONB von $\enbrace*{\Lambda^2(\mathbb{R}^n), \skal*{\cdot}{\cdot}_{\mathrm{std}}}$, so ist obiger Isomorphismus eine Isometrie (!).
	Eine kurze Rechnung zeigt, dass\marginnote{jeweils wenn alle Indizes verschieden sind}
	\[
		\benbrace[\big]{e_i \wedge e_j , e_j \wedge e_k} = - e_i \wedge e_k \qquad \benbrace[\big]{e_i \wedge e_j, e_k \wedge e_l}= 0
	\]
	gilt.
	Wir erhalten für eine ONB $\set*{b_1, \ldots ,b_N}$ von $\mathfrak{so}(n)$, $N=\frac{1}{2}(n-1)n$
\end{beispiel*}


\todo{RevChap 1}
\begin{align}
	\mathcal{B}(v,w) = \tr \enbrace*{\ad_v \circ \ad_w} &= \sum_{i=1}^{N} \skal[\big]{(\ad_v \circ \ad_w)(b_i)}{b_i}_{\mathfrak{so}(n)} \\
	&\stackrel{!}{=} -\sum_{i<j} \skal*{ \benbrace*{w,e_i \wedge e_j}}{ \benbrace*{v, e_i \wedge e_j}} = - \ad_v
\end{align}
Somit ist $\mathcal{B}(e_k \wedge e_l, e_k \wedge e_k)=-2(n-2)$ und für $\set*{k,l} \neq \set*{\tilde{k},\tilde{l}}$
\[
	\mathcal{B} \enbrace*{e_k \wedge e_l, e_{\tilde{k}} \wedge e_{\tilde{l}}} =0
\]

\begin{beispiel*}[{name=[triviale Killingform trotz nicht abelscher Liealgebra]}]
	Die Killingform $\mathcal{B}$ einer Liealgebra $\mathfrak{g}$ kann aber auch Null sein, ohne dass $\mathfrak{g}$ abelsch ist: Betrachte
	\[
		H \coloneqq \set*{\begin{pmatrix}
			1 & 0 & 0 \\
			\alpha & 1 & 0 \\
			\beta & \gamma & 1
		\end{pmatrix} \given \alpha,\beta,\gamma \in \mathbb{R}}
	\]
	$H$ ist eine Untergruppe von $\GL(3,\mathbb{R})$ mit Liealgebra
	\[
		\mathfrak{h} = \set*{\begin{pmatrix}
			0 & 0 & 0 \\
			x & 0 & 0 \\
			y & z & 0
		\end{pmatrix} \given x,y,z \in \mathbb{R}}
	\]
	Wir setzen 
	\[
		e_1 \coloneqq \begin{pmatrix}
			0 & 0 & 0 \\
			1 & 0 & 0 \\
			0 & 0 & 0
		\end{pmatrix} \quad 
		e_2 \coloneqq \begin{pmatrix}
			0 & 0 & 0 \\
			0 & 0 & 0 \\
			0 & 1 & 0
		\end{pmatrix} \quad 
		e_3 \coloneqq \begin{pmatrix}
			0 & 0 & 0 \\
			0 & 0 & 0 \\
			1 & 0 & 0
		\end{pmatrix}
	\]
	Dann gilt $\benbrace*{e_i,e_3} = 0$ für $i=1,2,3$.
	Aber $\benbrace*{e_1,e_2}=-e_3$.
	Somit ist (bezüglich der Basis $(e_1,e_2,e_3)$)
	\[
		\ad_{e_1} = \begin{pmatrix}
			0 & 0 & 0 \\
			0 & 0 & 0 \\
			0 & -1 & 0
		\end{pmatrix} \quad 
		\ad_{e_2} = \begin{pmatrix}
			0 & 0 & 0 \\
			0 & 0 & 0 \\
			1 & 0 & 0 
		\end{pmatrix} \quad 
		\ad_{e_3} = \begin{pmatrix}
			0 & 0 & 0 \\
			0 & 0 & 0 \\
			0 & 0 & 0
		\end{pmatrix}
	\]
	Es gilt nun aber
	\[
		\ad(e_i) \cdot \ad(e_j) = \begin{pmatrix}
			0 & 0 & 0 \\
			0 & 0 & 0 \\
			0 & 0 & 0
		\end{pmatrix}
	\]
	für alle $1 \le i,j \le 3$ und somit ist $\mathcal{B}=0$.
	Die Gruppe $H$ heißt \Index{Heisenberggruppe}.
\end{beispiel*}

\begin{lemma}[label=lem:168,{name=[Verhalten Killingform mit Automorphismen]}]
	Sei $\mathfrak{g}$ eine Liealgebra mit Killingform $\mathcal{B}$.
	Dann gilt
	\begin{enumerate}[1)]
		\item $\mathcal{B}(Av,Aw) = \mathcal{B}(v,w)$ für alle $v,w \in \mathfrak{g}$ und jeden Automorphismus $A$ von $\mathfrak{g}$.
		\item $\mathcal{B}(Lv,w) + \mathcal{B}(v,Lw) =0$ für alle $v,w \in \mathfrak{g}$ und jede Derivation $L$ von $\mathfrak{g}$.
	\end{enumerate}
\end{lemma}
\begin{beweis}
	Beide Formeln lassen sich leicht nachrechnen:
	\begin{enumerate}[1)]
		\item Sei $A \in \Aut(\mathfrak{g})$.
		Dann gilt 
		\[
			\ad_{Av}(w) = \benbrace*{Av,w} = \benbrace*{Av,A (A^{-1}w)} = A \benbrace*{v,A^{-1} w} = \enbrace*{A \circ {\ad_v} \circ A^{-1}}(w)
		\]
		Somit ist
		\begin{align}
			\mathcal{B}(Av,Aw) =\tr \enbrace*{\ad_{Av} \circ \ad_{Aw}} &= \tr \enbrace*{A \circ {\ad_v} \circ A^{-1} \circ A \circ {\ad_w} \circ A^{-1}} \\
			&= \tr \enbrace*{{\ad_v} \circ {\ad_w}} = \mathcal{B}(v,w)
		\end{align}
		\item Ist $L$ eine Derivation von $\mathfrak{g}$, so ist $e^{tL} \in \Aut(\mathfrak{g})$ für alle $t \in \mathbb{R}$.
		Nach 1) erhalten wir für alle $v,w \in \mathfrak{g}$
		\[
			\mathcal{B}(v,w) \equiv \mathcal{B}\enbrace*{e^{tL}v, e^{tL}w} 
		\]
		Differenzieren in $t=0$ liefert $0= \mathcal{B} \enbrace*{Lv,w} + \mathcal{B}(v,Lw)$ wie gewünscht.\qedhere
	\end{enumerate}
\end{beweis}

\begin{lemma}[label=lem:169,{name=[Kern der Killingform ist Ideal]}]
	Sei $\mathfrak{g}$ eine Liealgebra mit Killingform $\mathcal{B}$.
	Dann ist 
	\[
		\ker(\mathcal{B}) \coloneqq \set[\big]{v \in \mathfrak{g} \given \mathcal{B}(v,w)=0 \,\,\forall w \in \mathfrak{g}}
	\]
	ein Ideal von $\mathfrak{g}$
\end{lemma}
\begin{beweis}
	Sei $v \in \ker \mathcal{B}$ und $w \in \mathfrak{g}$.
	Zu zeigen: $\benbrace*{v,w} \in \ker \mathcal{B}$.
	Sei $z \in \mathfrak{g}$ beliebig.
	Dann gilt 
	\[
		\mathcal{B} \enbrace[\big]{\benbrace*{v,w},z} = - \mathcal{B} \enbrace[\big]{\ad_w(v),z} \StackText{\ref{lem:168} 2)}{=\joinrel=} \mathcal{B} \enbrace[\big]{v, \ad_w(z)} \stackrel{v \in \ker \mathcal{B}}{=\joinrel=\joinrel=}0 
	\]
	Somit ist $\benbrace*{v,w} \in \ker \mathcal{B}$.
\end{beweis}

Wir werden später jeder Lie-Algebra $\mathfrak{g}$ zwei weitere Ideale zuordnen: Das \Index{Radikal} $\Rad(\mathfrak{g})$ und das \Index{Nilradikal} $\Nil(\mathfrak{g})$.
% section 16 (end)
% chapter 1 (end)

\chapter{Strukturresultate} % (fold)
\label{cha:2}
\section{Nilpotente und auflösbare Liegruppen} % (fold)
\label{sec:21}
Sei $\mathfrak{g}$ eine Liealgebra über $\mathbb{R}$ (oder $\mathbb{C}$)

\begin{definition}[{name=[aufsteigende Zentralreihe und absteigende Reihe]}]
	Sei $k \in \mathbb{N}$.
	Dann definieren wir
	\[
		\mathfrak{g}^k \coloneqq \benbrace*{\mathfrak{g},\mathfrak{g}^{k-1}} \qquad \quad \mathfrak{g}_k \coloneqq \benbrace[\big]{\mathfrak{g}_{k-1},\mathfrak{g}_{k-1}}
	\] 
	wobei $\mathfrak{g}^0 \coloneqq \mathfrak{g}$ und $\mathfrak{g}_0 \coloneqq \mathfrak{g}$.
	Für Unterräume $V,W \subset \mathfrak{g}$ ist dabei $\benbrace*{V,W}$ die kleinste Unteralgebra von $\mathfrak{g}$, welche $\Span \set[\big]{\benbrace*{v,w} \given v \in V, w \in W}$ enthält.
\end{definition}

Sind $V,W$ Ideale von $\mathfrak{g}$, so gilt
\[
	\benbrace*{V,W} = \Span_\mathbb{K} \set[\big]{\benbrace*{v,w} \given v \in V, w \in W}
\]
denn $\benbrace[\big]{\benbrace*{v_1,w_1}, \benbrace*{v_2,w_2}} \in \Span_\mathbb{K} \set[\big]{\benbrace*{v,w} \given v \in V, w \in W}$ für $v_1,v_2 \in V$ und $w_1,w_2 \in W$.
Sind $V,W$ Ideale, so auch $\benbrace*{V,W}$, denn für alle $z \in \mathfrak{g}$ gilt
\begin{align}
	\benbrace[\big]{z, \benbrace*{v,w}} = - \benbrace[\big]{w, \benbrace*{z,v}} - \benbrace[\big]{v,\benbrace*{w,z}} \in \benbrace*{V,W}
\end{align}
Damit sind auch $\mathfrak{g}^k$ und $\mathfrak{g}_k$ Ideale von $\mathfrak{g}$. Es gilt $\mathfrak{g}^1= \benbrace*{\mathfrak{g}, \mathfrak{g}} \subseteq \mathfrak{g}^0 = \mathfrak{g}$ und weiter $\mathfrak{g}^k = \benbrace*{\mathfrak{g},\mathfrak{g}^{k-1}} \subseteq \mathfrak{g}^{k-1}$ und somit erhalten wir eine Kette von Idealen
\[
	\mathfrak{g}^0 \supseteq \mathfrak{g}^1 \supseteq \mathfrak{g}^2 \supseteq \mathfrak{g}^3 \supseteq \ldots 
\]
Dies ist die \Index{absteigende Zentralreihe}. Analog erhält man die \Index{abgeleitete Reihe} (auch: \Index{derivierte Reihe})
\[
	\mathfrak{g}_0 \supseteq \mathfrak{g}_1 \supseteq \mathfrak{g}_2 \supseteq \mathfrak{g}_3 \supseteq \ldots 
\]

\begin{definition}[{name=[nilpotent und auflösbar]}]
	Eine Liealgebra $\mathfrak{g}$ nennt man
	\begin{itemize}
		\item \Index{nilpotent}, falls $\mathfrak{g}^k=0$ und $\mathfrak{g}^{k-1}\neq 0$ für ein $k \in \mathbb{N}$
		\item \Index{auflösbar}, falls $\mathfrak{g}_k=0$ und $\mathfrak{g}_{k-1} \neq 0$ für ein $k \in \mathbb{N}$  
	\end{itemize}
	Die obigen $k$ sind eindeutig bestimmt. 
	Man nennt $\mathfrak{g}$ dann $k$-Schritt nilpotent (auflösbar).
	Eine Liegruppe $G$ nennt man nilpotent (auflösbar), falls ihre Liealgebra $\Tmap_e G$ nilpotent (auflösbar) ist.
\end{definition}

\begin{beispiel*}[{name=[Heisenberggruppe ist nilpotent und auflösbar]}]
	Die Heisenberggruppe\index{Heisenberggruppe} ist $2$-Schritt nilpotent und auflösbar:
	\[
		\mathfrak{g}_1 = \mathfrak{g}^1 = \benbrace*{\mathfrak{g},\mathfrak{g}} = \set*{\begin{psmallmatrix}
			0 & 0 & 0 \\
			0 & 0 & 0 \\
			w & 0 & 0
		\end{psmallmatrix} \given w \in \mathbb{R}}
	\]
	Somit ist offenbar $\mathfrak{g}^2 =\mathfrak{g}_2=0$.
	Nilpotenz und Auflösbarkeit müssen aber nicht immer übereinstimmen:
	Sei 
	\[
		\mathfrak{S} \coloneqq \set*{\begin{psmallmatrix}
			w_{11} & 0 & 0 \\
			w_{21} & w_{22} & 0 \\
			w_{31} & w_{32} & w_{33}
		\end{psmallmatrix} \given w_{ij} \in \mathbb{R}}
	\]
	Dann gilt $\mathfrak{S}_1= \benbrace*{\mathfrak{S},\mathfrak{S}}\stackrel{!}{=} \mathfrak{g}$ und weiter
	$\mathfrak{S}_2 = \benbrace*{\mathfrak{S}_1,\mathfrak{S}_1} = \benbrace*{\mathfrak{g},\mathfrak{g}} = \mathfrak{g}^1$ und somit $\mathfrak{S}_3 = \benbrace*{\mathfrak{g}^1,\mathfrak{g}^1} =0$.
	Somit ist $\mathfrak{S}$ $3$-Schritt auflösbar.
	Man zeigt aber leicht, dass $\mathfrak{S}$ \emph{nicht} nilpotent ist.
\end{beispiel*}

\begin{lemma}[label=lem:213,{name=[Grundlegendes zu Nilpotenz und Auflösbarkeit]}]
	Sei $\mathfrak{g}$ eine Liealgebra nilpontent bzw. auflösbar in $k$ Schritten ist.
	Dann gilt:
	\begin{enumerate}[1),itemsep=0pt]
		\item Für alle $i \in \mathbb{N}$ gilt $\mathfrak{g}_i \subseteq \mathfrak{g}^i$, also impliziert nilpotent auflösbar.
		\item Für alle $i \in \mathbb{N}$ gilt: $\mathfrak{g}^i$ und $\mathfrak{g}_i$ sind Ideale.
		\item Ist $\mathfrak{g}$ nilpotent, so gilt $\mathfrak{g}^{k-1} \subset Z(\mathfrak{g})$.
		Ist $\mathfrak{g}$ auflösbar, so ist $\mathfrak{g}_{k-1}$ abelsch.
		\item Unteralgebren von nilpotenten (auflösbaren) Liealgebren sind nilpotent (auflösbar)
		\item Ist $\mathfrak{h}$ ein Ideal in $\mathfrak{g}$, so ist $\mathfrak{k} \coloneqq \sfrac{\mathfrak{g}}{\mathfrak{h}}$ nilpotent (auflösbar), falls $\mathfrak{g}$ nilpotent (auflösbar) ist.
		\item Sei 
		\[
			\begin{tikzcd}
				0 \rar & \mathfrak{h} \rar["\Psi"] & \mathfrak{k} \rar["\Phi"] & \mathfrak{g} \rar & 0
			\end{tikzcd}
		\]
		eine kurze exakte Sequenz von Liealgebren.
		Sind $\mathfrak{h}$ und $\mathfrak{g}$ auflösbar, so auch $\mathfrak{k}$.
		\item Sind $\mathfrak{h}, \mathfrak{k}$ nilpotente (auflösbare) Ideale von $\mathfrak{g}$, so ist $\mathfrak{h} + \mathfrak{k}$ ein nilpotentes (auflösbares) Ideal von $\mathfrak{g}$.
	\end{enumerate}
\end{lemma}
\begin{beweis}
	\begin{enumerate}[1),itemsep=0pt]
		\item $\mathfrak{g}_0 = \mathfrak{g} = \mathfrak{g}^0$.
		Annahme $\mathfrak{g}_{i-1} \subseteq \mathfrak{g}^{i-1}$. Induktionsschritt:
		\[
			\mathfrak{g}_i = \benbrace[\Big]{\Underbrace{\mathfrak{g}_{i-1}}{\subseteq \mathfrak{g}}, \Underbrace{\mathfrak{g}_{i-1}}{\subseteq \mathfrak{g}^{i-1}}} \subseteq \benbrace*{\mathfrak{g},\mathfrak{g}^{i-1}} = \mathfrak{g}^i
		\]
		\item Schon am Anfang dieses Abschnitts gezeigt.
		\item Es ist $0 = \mathfrak{g}^k = \benbrace*{\mathfrak{g}, \mathfrak{g}^{k-1}}$, also $\mathfrak{g}^{k-1} \subset Z(\mathfrak{g})$.
		Dass $\mathfrak{g}_{k-1}$ abelsch ist, folgt auch direkt aus $\benbrace[\big]{\mathfrak{g}_{k-1}, \mathfrak{g}_{k-1}}=0$.
		\item Auch klar: Da $\mathfrak{g}$ nilpotent ist, existiert $k$ mit $\mathfrak{g}^k =0$.
		Für ein Ideal $0 \neq \mathfrak{h}$ ist stets $\mathfrak{h}^i \subseteq \mathfrak{g}^i$ für alle $i$.
		Genauso ist $\benbrace*{\mathfrak{h}_{k-1}, \mathfrak{h}_{k-1}} \subseteq \benbrace*{\mathfrak{g}_{k-1}, \mathfrak{g}_{k-1}}$, also $\mathfrak{h}$ auch auflösbar, falls $\mathfrak{g}$ auflösbar ist.
		\item Auf der Quotientenliealgebra $\mathfrak{k} = \sfrac{\mathfrak{g}}{\mathfrak{h}}$ hatten wir die Lieklammer definiert durch
		\[
			\benbrace[\big]{v +\mathfrak{h}, w + \mathfrak{h}}_{\mathfrak{k}} \coloneqq \benbrace*{v,w}_{\mathfrak{g}} + \mathfrak{h} 
		\]
		Damit folgt sofort die Behauptung.
		\item Es seien $\mathfrak{h}$ und $\mathfrak{g}$ auflösbar. Per Induktion:
		Es gilt $\Phi(\mathfrak{k}) = \mathfrak{g}$ und\marginnote{da $\Phi$ surjektiv, ist $\Phi(\mathfrak{k}_i)$ ein Ideal} 
		\[
			\Phi(\mathfrak{k}_i) = \Phi \enbrace[\big]{\benbrace*{\mathfrak{k}_{i-1}, \mathfrak{k}_{i-1}}} = \benbrace[\big]{\Phi(\mathfrak{k}_{i-1}), \Phi(\mathfrak{k}_{i-1})}\StackText{I.V.}{\subseteq} \benbrace[\big]{\mathfrak{g}_{i-1}, \mathfrak{g}_{i-1}} = \mathfrak{g}_i
		\]
		Somit $\Phi(\mathfrak{k}_{i_0})=0$ für ein $i_0 \in \mathbb{N}$, da $\mathfrak{g}$ auflösbar ist.
		Damit ist $\mathfrak{k}_{i_0} \subset \ker \Phi = \im \Psi$. 
		Es gilt
		\[
			\mathfrak{k}_{i_0 +1} = \benbrace[\big]{\mathfrak{k}_{i_0}, \mathfrak{k}_{i_0}} \subseteq \benbrace[\big]{\Psi(\mathfrak{h}), \Psi(\mathfrak{h})} \subseteq \Psi \benbrace[\big]{\mathfrak{h},\mathfrak{h}} \subseteq \Psi(\mathfrak{h}_1)
		\]
		und 
		\[
			\mathfrak{k}_{i_0+2} = \benbrace[\big]{\mathfrak{k}_{i_0 +1}, \mathfrak{k}_{i_0 +1}} \subseteq \benbrace[\big]{\Psi(\mathfrak{h}_1), \Psi(\mathfrak{h}_1)} \subseteq \Psi \benbrace[\big]{\mathfrak{h}_1, \mathfrak{h}_1} = \Psi(\mathfrak{h}_2)
		\]
		Da $\mathfrak{h}$ auflösbar ist, folgt die Behauptung.
		\item Seien $\mathfrak{h}$, $\mathfrak{k}$ auflösbare Ideale.
		Die Sequenz
		\[
			\begin{tikzcd}
				0 \rar & \mathfrak{h} \rar & \mathfrak{h} + \mathfrak{k} \rar & \sfrac{\mathfrak{h} + \mathfrak{k}}{\mathfrak{h}} \rar & 0
			\end{tikzcd}
		\]
		ist exakt.
		Nach 6) reicht es zu zeigen, dass $\sfrac{\mathfrak{h} + \mathfrak{k}}{\mathfrak{h}}$ auflösbar ist: 
		Es ist $\sfrac{\mathfrak{h}+ \mathfrak{k}}{\mathfrak{h}} = \sfrac{\mathfrak{k}}{\mathfrak{h} \cap \mathfrak{k}}$ und $\mathfrak{h} \cap \mathfrak{k}$ ist ein Ideal.
		Da $\mathfrak{k}$ und $\mathfrak{h} \cap \mathfrak{k}$ auflösbare Ideale von $\mathfrak{g}$ sind, folgt aus 5), dass auch $\sfrac{\mathfrak{k}}{\mathfrak{h} \cap \mathfrak{k}}$ auflösbar ist.
		%  von $\mathfrak{g}$ und $\mathfrak{k}$.
		% Aus 5) folgt, dass $\sfrac{\mathfrak{k}}{\mathfrak{h} \cap \mathfrak{k}}$ auflösbar ist.
		
		Seien nun $\mathfrak{h}, \mathfrak{k}$ nilpotente Ideale von $\mathfrak{g}$.
		Dann sind auch $\mathfrak{h}^i$ und $\mathfrak{k}^i$ nilpotente Ideale von $\mathfrak{g}$.
		Es gilt
		\[
			\benbrace*{\mathfrak{g}, \mathfrak{h}^1} = \benbrace[\big]{\mathfrak{g}, \benbrace*{\mathfrak{h}, \mathfrak{h}}} \equiv \benbrace[\big]{\mathfrak{h}, \benbrace*{\mathfrak{g},\mathfrak{h}}} \subset \mathfrak{h}^1
		\]
		Nun gilt \marginnote{mit $\mathfrak{h}^0=\mathfrak{g}$ und $\mathfrak{k}^0=\mathfrak{g}$}
		\(
			(\mathfrak{h} + \mathfrak{k})^1 = \benbrace[\big]{\mathfrak{h} + \mathfrak{k}, \mathfrak{h} + \mathfrak{k}} \subseteq \benbrace[\big]{\mathfrak{g},\mathfrak{h} + \mathfrak{k}} \subseteq \mathfrak{h} + \mathfrak{k} \subseteq \sum_{i=0}^{1} \mathfrak{h}^i \cap \mathfrak{k}^{1-i}
		\).
		Weiter ist 
		\[
			(\mathfrak{h} + \mathfrak{k})^2 = \benbrace*{\mathfrak{h} + \mathfrak{k},(\mathfrak{h} + \mathfrak{k})^1} \subseteq \benbrace*{\mathfrak{h} + \mathfrak{k}, \sum_{i=0}^{1} \mathfrak{h}^1 \cap \mathfrak{k}^{1-i} } \subseteq \sum_{i=0}^{2} \mathfrak{h}^i \cap \mathfrak{k}^{2-i}
		\]
		Induktiv folgt
		\[
			(\mathfrak{h} + \mathfrak{k})^\ell \subset \sum_{i=0}^{\ell} \mathfrak{h}^i \cap \mathfrak{k}^{\ell-i}
		\]
		und somit auch die Behauptung für ausreichend großes $\ell$.\qedhere
	\end{enumerate}
\end{beweis}

\begin{beispiel*}[{name=[exakte Sequenzen und Auflösbarkeit]}]
	Hier ein paar weitere Beispiele zu Auflösbarkeit und Nilpotenz:
	\begin{itemize}[itemsep=1pt]
		\item Betrachte die Unteralgebra von $\mathfrak{gl}(n,\mathbb{K})$ der unteren Dreiecksmatrizen
		\[
			\mathfrak{0}(n,\mathbb{K}) = \set[\big]{A \in \Mat(n,\mathbb{K}) \given A_{ij}=0 \,\, \forall 1 \le i<j \le n}
		\]
		ist eine auflösbare Liealgebra mit $\mathfrak{0}(n,\mathbb{K})_n =0$, die \emph{nicht} nilpotent ist.
		\item Die Lieunteralgebra der strikten unteren Dreiecksmatrizen 
		\[
			\mathfrak{n}(n,\mathbb{K}) = \set[\big]{A \in \mathfrak{0}(n,\mathbb{K}) \given A_{ii}=0 \enspace i=1, \ldots ,n}
		\]
		ist eine nilpotente Liealgebra mit $\mathfrak{n}(n,\mathbb{K})_{n-1}=0$.
		\item Die Liealgebra 
		\[
			\mathfrak{g} = \set*{\begin{pmatrix}
				\alpha & \beta \\ 0 & 0
			\end{pmatrix} \given \alpha, \beta \in \mathbb{R}}
		\]
		ist auflösbar, aber nicht nilpotent, wie man leicht nachrechnet.
		Betrachte nun die exakte Sequenz
		\[
			\begin{tikzcd}[ampersand replacement=\&]
				0 \rar \& \lenbrace*{\begin{pmatrix}
					1 & 0 \\ 0 & 0
				\end{pmatrix}}_{\mathbb{R}} \rar \& \mathfrak{g} \rar \&
				\lenbrace*{\begin{pmatrix}
					0 & 1 \\ 0 & 0
				\end{pmatrix}}_{\mathbb{R}} \rar \& 0
			\end{tikzcd} 
		\]
		$1$-dimensionale Liealgebren sind abelsch und somit nilpotent, also auch die beiden \enquote{äußeren} Ideale.
		Damit ist \autoref{lem:213} 6) nicht richtig für nilpotente Algebren.
	\end{itemize}
\end{beispiel*}

\begin{definition}[{name=[Radikal und Nilradikal]},label=def:214]
	Sei $\mathfrak{g}$ eine Liealgebra.
	\begin{enumerate}[1),itemsep=0pt]
		\item Das \Index{Radikal} $\rad (\mathfrak{g})$ von $\mathfrak{g}$ ist das eindeutig bestimmte maximale auflösbare Ideal von $\mathfrak{g}$.\marginnote{für Existenz: Summen auflösbarer (nilpotenter) Ideale sind auflösbar (nilpotent)}
		\item Das \Index{Nilradikal} (oder \Index{Nilideal})  $\nil (\mathfrak{g})$ von $\mathfrak{g}$ ist das eindeutig bestimmte maximale nilpotente Ideal von $\mathfrak{g}$.
		\item Man nennt $\mathfrak{g}$ \bet{halbeinfach}\index{Liealgebra!halbeinfache}, falls $\rad(\mathfrak{g}) = \set*{0}$ ist.
		\item Man nennt $\mathfrak{g}$ \bet{reduktiv}\index{Liealgebra!reduktive}, falls $\nil(\mathfrak{g})= \set*{0}$ ist.
		\item Man nennt $\mathfrak{g}$ \bet{einfach}\index{Liealgebra!einfache}, falls $\set*{0}$ und $\mathfrak{g}$ die einzigen Ideale von $\mathfrak{g}$ sind und $\dim \mathfrak{g} > 1$ ist.
	\end{enumerate}
\end{definition}

\begin{bemerkung*}[{name=[Charakterisierung von halbeinfach]}]
	\begin{itemize}
		\item Es ist $\nil(\mathfrak{g}) \subseteq \rad(\mathfrak{g}) \subseteq \mathfrak{g}$, da nilpotent auflösbar impliziert.
		\item Eine Liealgebra $\mathfrak{g}$ ist genau dann halbeinfach, wenn $\mathfrak{g}$ keine abelschen Ideale besitzt:
		
		Wenn $0\neq\mathfrak{k}$ ein abelsches Ideal ist, so ist $\rad(\mathfrak{g}) \neq 0$.
		Ist andersrum $\rad(\mathfrak{g})\neq 0$, so ist $\rad(\mathfrak{g})_{k-1}$ ein abelsches Ideal, falls $k$ die kleinste Zahl mit $\rad(\mathfrak{g})_k=0$ ist.
	\end{itemize}
\end{bemerkung*}

\begin{proposition}[{name=[Quotient mit Radikal ist stets halbeinfach]}]
	Sei $\mathfrak{g}$ eine Liealgebra.
	Dann ist $\overline{\mathfrak{g}} \coloneqq \sfrac{\mathfrak{g}}{\rad(\mathfrak{g})}$ halbeinfach.
\end{proposition}
\begin{beweis}
	Die Quotientenabbildung $\pi \colon \mathfrak{g} \to \overline{\mathfrak{g}}$ ist ein Liealgebrenhomomorphismus.
	Somit ist für jedes Ideal $\overline{\mathfrak{i}} \subset \overline{\mathfrak{g}}$ das Urbild $\mathfrak{i} = \pi^{-1}(\overline{\mathfrak{i}})$ ebenfalls ein Ideal:
	\[
		\benbrace*{\pi^{-1}(\overline{\mathfrak{i}}), \mathfrak{g}} \subset \pi^{-1}(\overline{\mathfrak{i}}) \iff \pi \enbrace*{\benbrace*{\pi^{-1}(\overline{\mathfrak{i}}), \mathfrak{g}}} \subset \overline{\mathfrak{i}}
	\]
	Weiter ist $\benbrace*{\overline{\mathfrak{i}}, \overline{\mathfrak{g}}} \subset \overline{\mathfrak{i}}$ ein Ideal und somit ist die Sequenz
	\[
		\begin{tikzcd}
			0 \rar & \rad(\mathfrak{g}) \rar & \pi^{-1}(\overline{\mathfrak{i}}) \rar &\overline{\mathfrak{i}} \rar & 0
		\end{tikzcd}
	\] 
	exakt.
	Ist also $\overline{\mathfrak{i}}$ ein auflösbares Ideal von $\overline{\mathfrak{g}}$, so ist nach \autoref{lem:213} 6) auch $\pi^{-1}(\overline{\mathfrak{i}})$ ein auflösbares Ideal von $\mathfrak{g}$.
	Aus der Maximalität von $\rad(\mathfrak{g})$ folgt $\overline{\mathfrak{i}}=0$.
\end{beweis}

\begin{satz}[{name={Levi-Malcev}},label=satz:216]
	Sei $\mathfrak{g}$ eine Liealgebra.
	Dann existiert eine bis auf innere Automorphismen eindeutige halbeinfache Liealgebra $\mathfrak{h} \subset \mathfrak{g}$ mit\marginnote{Vektorraumsumme, semidirektes Produkt von Liealgebren}
	\[
		\mathfrak{g} = \mathfrak{h} \oplus  \rad(\mathfrak{g})
	\]
\end{satz}

\begin{lemma}[label=lem:217,{name=[Lieunteralgebra als nilpotenten Elementen]}]
	Sei $\mathfrak{g} \subset \mathfrak{gl}(V)$ eine Lieunteralgebra bestehend aus nilpotenten Elementen, das heißt für alle $A \in \mathfrak{g}$ existiert ein $k \in \mathbb{N}$ mit $A^k=0$.
	Dann existiert ein Vektor $v \in V \setminus \set*{0}$ mit $A \cdot v = 0$ für alle $A \in \mathfrak{g}$.
\end{lemma}
\begin{beweis}
	Zunächst eine kurze Vorüberlegung: Sei $X \in \mathfrak{g}$. Dann ist $\ad_X(Y) = X Y -Y X$ und es gilt
	\begin{align}
		(\ad_X)^2(Y) &= X \enbrace*{X Y -YX} - \enbrace*{XY -YX}X = X^2 Y - 2XYX+ YX^2 \\
		(\ad_X)^3(Y) &= X \enbrace*{X^2 Y - 2XYX+ YX^2}- \enbrace*{X^2 Y - 2XYX+ YX^2}X
		= X^3 y - 3 X^2YX+3 XYX^2 - YX^3
	\end{align}
	Induktiv zeigt man: Gilt $X^k=0$, so folgt $(\ad_X)^{2k}=0$ (!).
	Wir beweisen nun die eigentlich Behauptung per Induktion nach $n = \dim \mathfrak{g}$.
	% Nur zur eigentlichen Behauptung: Per Induktion nach $n=\dim \mathfrak{g}$. 
	
	Für $n=1$ sei $A \in \mathfrak{g} \setminus \set*{0}$.
	Ist $A$ invertierbar, so auch $A^k$ für alle $k \in \mathbb{N}$, was im Widerspruch zur Nilpotenz von $A$ steht.
	Für den Induktionsschritt sei $\dim \mathfrak{g} = n+1$.
	Sei $\mathfrak{h}$ eine Unteralgebra maximaler Dimension (d.h. $\le n$) von $\mathfrak{g}$\marginnote{1-dimensionlae Unterräume sind Unteralgebren}.
	Wir werden nun zeigen, dass $\dim \mathfrak{h}=n$ ist.
	Wegen $\ad_X(\mathfrak{h}) \subset \mathfrak{h}$ für alle $X \in \mathfrak{h}$ können wir die adjungierte Darstellung auf dem Quotienten betrachten.
	Damit ist 
	\[
		\mathfrak{h}' = \set[\big]{\ad_X|_{\sfrac{\mathfrak{g}}{\mathfrak{h}}} \colon \sfrac{\mathfrak{g}}{\mathfrak{h}} \to \sfrac{\mathfrak{g}}{\mathfrak{h}} \given X \in \mathfrak{h}} \subset \mathfrak{gl}\enbrace*{\sfrac{\mathfrak{g}}{\mathfrak{h}}}
	\]
	eine Unteralgebra.
	Nach obiger Vorüberlegung sind die Elemente von $\mathfrak{h}'$ nilpotent.
	Wegen $\dim \sfrac{\mathfrak{g}}{\mathfrak{h}} \le n$ existiert nach Induktionsvorraussetzung $\overline{A} \in \sfrac{\mathfrak{g}}{\mathfrak{h}}$ mit $(\ad_X)|_{\sfrac{\mathfrak{g}}{\mathfrak{h}}}(\overline{A})=0$ für alle $X \in \mathfrak{h}$.
	% Wegen $\ad_X(\mathfrak{h}) \subset \mathfrak{h}$ für alle $X \in \mathfrak{h}$ gilt nach der Zerlegung aus \autoref{satz:216}
	% \[
	% 	\End(\mathfrak{g}) \ni \ad_X = \begin{pmatrix}
	% 		A_{11} & A_{12} \\ 0 & A_{22}
	% 	\end{pmatrix}
	% \]
	% Beobachtung: Es gilt
	% \[
	% 	\begin{pmatrix}
	% 		A_{11} & A_{12} \\ 0 & A_{22}
	% 	\end{pmatrix}^2 = \begin{pmatrix}
	% 		A_{11}^2 & * \\ 0 & A_{22}^2
	% 	\end{pmatrix}
	% \]
	% und so weiter.
	% Damit ist $\ad_X$ nilpotent und somit auch $\ad_X|_{\sfrac{\mathfrak{g}}{\mathfrak{h}}}$, was wegen $\ad_X(\mathfrak{h}) \subset \mathfrak{h}$ wohldefiniert ist.
	% Ferner gilt
	% \[
	% 	\benbrace*{\begin{pmatrix}
	% 		A_{11} & A_{12} \\ 0 & A_{22}
	% 	\end{pmatrix}, \begin{pmatrix}
	% 		B_{11} & B_{12} \\ 0 & B_{22}
	% 	\end{pmatrix}} = \begin{pmatrix}
	% 		\benbrace*{A_{11},B_{11}} & * \\ 0 & \benbrace*{A_{22},B_{22}}
	% 	\end{pmatrix}
	% \]
	% Somit ist
	% \[
	% 	\ad(\mathfrak{g}) = \set*{(\ad_X)\big|_{\sfrac{\mathfrak{g}}{\mathfrak{h}}} \given X \in \mathfrak{h}}
	% \]
	% eine nilpotente Unteralgebra von $\mathfrak{gl}(\sfrac{\mathfrak{g}}{\mathfrak{h}})$.\bigskip
	Somit existiert $A \in \mathfrak{g}$ mit 
	\begin{equation}
		\ad_X(A) \subset \mathfrak{h} \label{eq:217} \tag{*}
	\end{equation}
	für alle $X \in \mathfrak{h}$.
	Somit ist $\mathfrak{h} \oplus \langle A \rangle_\mathbb{R}$ eine Unteralgebra von $\mathfrak{g}$ (mit \eqref{eq:217} folgt $\benbrace*{\mathfrak{h},A}\subset \mathfrak{h}$).
	Da $\dim \mathfrak{h}$ maximal war, folgt $h \oplus \langle A \rangle_\mathbb{R} = \mathfrak{g}$ und somit auch $\dim \mathfrak{h}=n$.
	Obige Behauptung folgt.
	
	Zurück zur eigentlichen Induktion:
	Wir setzen
	\[
		W \coloneqq \set[\big]{v \in V \given X(v)=0 \,\, \forall\, X \in \mathfrak{h}}
	\]
	mit $\mathfrak{h} \subsetneq \mathfrak{g}$, $\dim \mathfrak{h}=n$ Unteralgebra.
	Da $\mathfrak{h}$ als Teilmenge von $\mathfrak{g}$ aus nilpotenten Elementen besteht, gilt nach Induktionsvorraussetzung $W \neq \set*{0}$.
	Der Endomorphismus $A \in \mathfrak{g} \subset \mathfrak{gl}(V)$ lässt $W$ invariant:
	\[
		0 = \Underbracket{\benbrace*{X,A}}{\mathclap{\in \mathfrak{h} \text{ nach \eqref{eq:217}}} }(w) = X \enbrace*{A(w)} - A \enbrace*{X(w)} = X (A(w))
	\]
	für alle $X \in \mathfrak{h}$ und $w \in W$. Damit ist $A(W)\subset W \subset V$.
	Wir schreiben nun mit $V= W \oplus W^\bot$
	\[
		A = \begin{pmatrix}
			A|_W & * \\ 0 & A|_{W^\bot}
		\end{pmatrix}
	\]
	Da $A|_W$ nilpotent ist, existiert ein Vektor $w \in W$ mit $A(w)=0$ und nach Definition von $W$ gilt $X(w)=0$ für alle $X \in \mathfrak{h}$.
	Wegen $\mathfrak{g} = \mathfrak{h} \oplus \langle A \rangle_\mathbb{R}$ folgt die Behauptung.
\end{beweis}

\begin{satz}[name={Engel},label=satz:218]
	Sei $\mathfrak{g} \subset \mathfrak{gl}(V)$ eine Unteralgebra bestehend aus nilpotenten Elementen.
	Dann existiert eine Basis von $V$, sodass 
	\[
		\mathfrak{g} \subset \mathfrak{n}(n,\mathbb{K})  
	\]
	gilt mit $n = \dim V$.
\end{satz}
\begin{beweis}
	Sei $v \in V \setminus \set*{0}$ mit $X(v)=0$ für alle $X \in \mathfrak{g}$ nach \autoref{lem:217}.
	Sei $\overline{V} \coloneqq \sfrac{V}{\langle v \rangle_\mathbb{R}}$.
	Wir setzen $X(\overline{w}) \coloneqq X(w) + \langle v \rangle_\mathbb{R}$.
	Dies ist wohldefiniert, wie man sich leicht überlegt.
	Wir erhalten eine Liealgebra $\overline{\mathfrak{g}} \subset \mathfrak{gl}(\overline{V})$ von nilpotenten Elementen(!).
	Dies bildet den Induktionsanfang.
	
	Nach Induktionsvoraussetzung existiert nun eine Basis $\overline{v}_2, \ldots , \overline{v}_{n} \in \overline{V}$ sodass $\overline{\mathfrak{g}} \subset \mathfrak{n}(n-1,\mathbb{K})$.
	Wir setzen $v_1 \coloneqq v$ und wählen Urbilder $v_2, \ldots ,v_n \in V$ von $\overline{v}_1, \ldots ,\overline{v}_n$.
	Es gilt für alle $X \in \mathfrak{g}$
	\[
		X(v_1) =0 \quad \text{ und } \quad X(v_2) \in \langle v_1 \rangle_\mathbb{R}
	\]
	denn sonst $X(v_2) = \alpha \cdot v_1 + w$ und $X(\overline{v}_2)=\overline{w}\neq 0$.
	So entsteht induktiv die gesuchte Basis.
\end{beweis}

\begin{korollar}[label=korr:219,{name=[Charakterisierung von Nilpotenz einer Liealgebra]}]
	Eine Liealgebra $\mathfrak{g}$ ist genau dann nilpotent, wenn 
	\[
		\ad(\mathfrak{g}) = \set[\big]{\ad_X \given X \in \mathfrak{g}} \subset \End(\mathfrak{g})
	\]
	eine nilpotente Unteralgebra von $\End(\mathfrak{g})$ ist.
\end{korollar}
\begin{beweis}
	Sei zunächst $\mathfrak{g}$ nilpotent und $X \in \mathfrak{g}$.
	Dann gilt $\ad_X(\mathfrak{g}^k) \subset \mathfrak{g}^{k+1}$.
	Somit ist $\ad_X$ nilpotent.
	
	Ist umgekehrt $\ad(\mathfrak{g}) \subset \End(\mathfrak{g})$ nilpotent, so existiert nach \autoref{satz:218} eine Basis $v_1, \ldots ,v_n$ von $\mathfrak{g}=V$ mit $\ad_X(v_i) \in V_{i-1} = \langle v_1 ,\ldots ,v_{i-1}\rangle_\mathbb{R}$ für alle $i=1,\ldots ,n$ und alle $X \in \mathfrak{g}$, dabei ist $V_0 \coloneqq \set*{0}$.
	Für $X_1, \ldots ,X_s, \tilde{X} \in \mathfrak{g}$ erhalten wir 
	\[
		 \benbrace[\Big]{X_1,  \benbrace[\big]{\ldots ,\benbrace*{X_s,\tilde{X}}}} \in \enbrace*{{\ad_{X_1}} \circ \ldots \circ {\ad_{X_s}}}(\mathfrak{g}) \subseteq V_{n-s}
	\]
	Da $V_0=0$ ist, ist $\mathfrak{g}$ nilpotent.
\end{beweis}

Hier noch eine weitere Folgerung aus dem Satz von Engel (\ref{satz:218}):

\begin{korollar}[{name=[Nilpotente Liealgebren als Unteralgebra der strikt unteren Dreiecksmatrizen]}]
	Eine nilpotente Liealgebra $\mathfrak{g}$ ist isomorph zu einer Unteralgebra von $\mathfrak{n}(n,\mathbb{K})$.
\end{korollar}
\begin{beweis}
	Aus dem Satz von Ado folgt, dass wir $\mathfrak{g} \subset \mathfrak{gl}(n,\mathbb{K})$ annehmen können. 
	Mit \autoref{satz:218} folgt dann die Behauptung.
\end{beweis}

Wir kommen nun zu auflösbaren Liealgebren und notieren die folgenden Aussagen ohne Beweise; diese sind ähnlich wie im nilpotenten Fall und in \citetitle{procesiLie} von \textcite{procesiLie} nachzuschlagen.

\begin{satz}[{name={Lie}},label=satz:2111]
	Sei $\mathfrak{g}$ eine komplexe Lieunteralgebra von $\mathfrak{gl}(V,\mathbb{C})=\End(V,\mathbb{C})$.
	Ist $\mathfrak{g}$ auflösbar, so existiert eine 1-Form $\lambda \colon \mathfrak{g} \to \mathbb{C}$ und ein Vektor $v \in V\setminus \set*{0}$ mit $A(v)= \lambda(A) \cdot v$ für alle $A \in \mathfrak{g} \subset \mathfrak{gl}(V,\mathbb{C})$ ($\mathfrak{g}$ nilpotent $\implies \lambda=0$).
\end{satz}

\begin{bemerkung*}[{name=[Gültigkeit des Satzes von Lie]}]
	Die Aussage ist nicht richtig im reellen Fall!
\end{bemerkung*}

\begin{korollar}[label=kor:2112,{name=[komplexe auflösbare Unteralgebra in Dreiecksmatrizen]}]
	Sei $\mathfrak{g} \subset \mathfrak{gl}(V,\mathbb{C})$ eine komplexe, auflösbare Liealgebra.
	Dann gilt $\mathfrak{g} \subset \mathfrak{o}(V,\mathbb{C})$.
\end{korollar}

\autoref{satz:2111} und \autoref{kor:2112} werden in \cite[Sec.~4.6.3]{procesiLie} bewiesen.
Aus Ado's Satz folgt wieder:
\begin{korollar}[label=korr:2113,{name=[jede komplexe auflösbare Unteralgebra in Dreiecksmatrizen]}]
	Jede komplexe auflösbare Liealgebra ist Unteralgebra von $\mathfrak{o}(n,\mathbb{C})$ für ein $n \in \mathbb{N}$.
\end{korollar}

\begin{korollar}[label=korr:2114,{name=[Charakterisierung von Auflösbarkeit komplexer Liealgebren]}]
	Sei $\mathfrak{g}$ eine komplexe Liealgebra.
	Dann ist $\mathfrak{g}$ genau dann auflösbar, wenn $\benbrace*{\mathfrak{g},\mathfrak{g}}$ nilpotent ist.
\end{korollar}
\begin{beweis}
	Sei zunächst $\mathfrak{g}$ auflösbar.
	Nach \autoref{korr:2113} können wir $\mathfrak{g} \subset \mathfrak{o}(n,\mathbb{C})$ annehmen.
	Wegen $\benbrace[\big]{\mathfrak{o}(n,\mathbb{C}), \mathfrak{o}(n,\mathbb{C})} \subset \mathfrak{n}(n,\mathbb{C})$ folgt die erste Richtung der Behauptung.
	
	Ist andersrum $\benbrace*{\mathfrak{g},\mathfrak{g}}$ nilpotent, so ist nach \autoref{lem:213} 1) $\benbrace*{\mathfrak{g},\mathfrak{g}}$ auflösbar und somit auch $\mathfrak{g}$.\marginnote{mit einem \enquote{Schritt} mehr}
\end{beweis}

\begin{exkurs*}[name={Komplexifizierung}]
	Sei $\mathfrak{g}$ eine reelle Liealgebra, das heißt $\mathfrak{g}$ ist ein $\mathbb{R}$-Vektorraum versehen mit einer Lieklammer $\benbrace*{\cdot ,\cdot }_\mathbb{R} \colon \mathfrak{g} \times \mathfrak{g} \to \mathfrak{g}$.
	Dann ist 
	\[
		\mathfrak{g}_\mathbb{C} \coloneqq \mathfrak{g} \otimes_\mathbb{R} \mathbb{C} = \set[\big]{v \otimes 1 + \tilde{v} \otimes i \given v , \tilde{v} \in \mathfrak{g}}
	\]
	ein komplexer Vektorraum mit $\dim_\mathbb{R} \mathfrak{g} = \dim_\mathbb{C} \mathfrak{g}_\mathbb{C}$.
	Wir setzen die Lieklammer komplex linear auf $\mathfrak{g}_\mathbb{C}$ fort mittels
	\[
		\benbrace[\big]{v_1 \otimes \lambda_1, v_2 \otimes \lambda_2}_\mathbb{C} \coloneqq \benbrace*{v_1,v_2}_\mathbb{R} \otimes (\lambda_1 \lambda_2)
	\]
	Man nennt $\enbrace*{\mathfrak{g}_\mathbb{C}, \benbrace*{\cdot ,\cdot }_\mathbb{C}}$ \Index{Komplexifizierung} von $\enbrace*{\mathfrak{g}, \benbrace*{\cdot ,\cdot }}$.
	Man beachte, dass unterschiedliche reelle Liealgebren die gleiche Komplexifizierung besitzen können. 
	Man nennt alle diese reellen Liealgebren dann \Index{reelle Formen} von $\mathfrak{g}_\mathbb{C}$.
\end{exkurs*}

\begin{beispiel*}[{name=[Komplexifizierungen]}]
	\begin{itemize}
		\item Es gilt $\SL(2,\mathbb{R}) \otimes \mathbb{C} = \SU(2) \otimes \mathbb{C} = \SL(2,\mathbb{C})$.
		\item $\SU(2) = \set*{A \in \GL(2,\mathbb{C}) \given A^* A = \begin{psmallmatrix}
			1 & 0 \\ 0 & 1
		\end{psmallmatrix}, \det A = 1}$ ist kompakte Liegruppe.
		\item $\SL(2,\mathbb{R})$ ist nicht kompakte Liegruppe.
	\end{itemize}
\end{beispiel*}

Dazu ein kleines Lemma: Sei $\mathfrak{g}$ eine reelle Liealgebra.
Dann gilt
\begin{enumerate}[1)]
	\item $\mathfrak{h}$ ist Unteralgebra von $\mathfrak{g}$ genau dann, wenn $\mathfrak{h}_\mathbb{C}$ Unteralgebra von $\mathfrak{g}_\mathbb{C}$ ist.
	\item Es gilt für $i \in \mathbb{N}$
	\[
		\enbrace*{\mathfrak{g}^i}_\mathbb{C} = \enbrace*{\mathfrak{g}_\mathbb{C}}^i \qquad \enbrace*{\mathfrak{g}_i}_\mathbb{C} = \enbrace*{\mathfrak{g}_\mathbb{C}}_i
	\]
	\item Killingform: $\mathcal{B}_\mathbb{C} \enbrace*{v_1 \otimes \lambda_1, v_2 \otimes \lambda_2} = \mathcal{B}_\mathbb{R}(v_1,v_2) \cdot \lambda_1 \lambda_2 \in \mathbb{C}$
\end{enumerate}
Somit entsprechen sich nilpotente, auflösbare und halbeinfache reelle Liealgebren und ihre Komplexifizierung.

\begin{satz}[name={Cartan},label=satz:2115]
	Sei $\mathfrak{g}$ eine Liealgebra.\marginnote{Cartans Kriterium für Auflösbarkeit}
	Dann ist $\mathfrak{g}$ genau dann auflösbar, wenn die Killingform $\mathcal{B}$ von $\mathfrak{g}$ auf $\benbrace*{\mathfrak{g},\mathfrak{g}}$ Null ist.
\end{satz}
\begin{beweis}
	Sei zunächst $\mathfrak{g}$ auflösbar.
	Dann ist nach \autoref{korr:2114} $\benbrace*{\mathfrak{g},\mathfrak{g}}$ nilpotent.
	Somit ist $\mathcal{B}_{\benbrace{\mathfrak{g},\mathfrak{g}}}=0$ nach \autoref{korr:219} ($\benbrace*{\mathfrak{g},\mathfrak{g}} \subset \mathfrak{n}(n,\mathbb{\mathbb{K}})$).
	Da $\benbrace*{\mathfrak{g},\mathfrak{g}}$ ein Ideal von $\mathfrak{g}$ ist, gilt 
	\[
		\mathcal{B}_{\mathfrak{g}}\big|_{\benbrace*{\mathfrak{g},\mathfrak{g}}} = \mathcal{B}_{\benbrace*{\mathfrak{g},\mathfrak{g}}} =0
	\]
	nach \cref{sec:einschr_killing}.
	% (Komplexifiziere zuerst! zunächst ins Reelle).
	Die andere -- aufwendigere -- Implikation findet man in \cite[Sec.~4.6.4]{procesiLie}.
\end{beweis}

Nach \autoref{lem:169} ist $\ker \mathcal{B}$ ein Ideal.
Wir erhalten aus obigem Kriterium für Auflösbarkeit $\rad(\mathfrak{g}) \supseteq \ker \mathcal{B} \supseteq \nil(\mathfrak{g})$, dabei ist $\ker \mathcal{B}$ auflösbar.
Ist die Killingform $0$, so ist $\mathfrak{g}$ auflösbar, aber nicht notwendig nilpotent.\todo{Übungsaufgabe einarbeiten?}
% section 21 (end)

\section{Halbeinfache Liealgebren} % (fold)
\label{sec:22}

\begin{erinnerung}[{name=[halbeinfach]}]
	Eine Liealgebra $\mathfrak{g}$ ist halbeinfach, genau dann, wenn $\mathfrak{g}$ kein abelsches Ideal hat (\ref{def:214}).
\end{erinnerung}

\begin{satz}[{name={Cartan}}]
	Eine Liealgebra $\mathfrak{g}$ ist halbeinfach genau dann,\marginnote{Cartans Kriterium für Halbeinfachheit} wenn die Killingform $\mathcal{B}$ nicht degeneriert ist.
\end{satz}
\begin{beweis}
	Sei zunächst $\mathfrak{g}$ halbeinfach.
	Zu zeigen ist, dass $\ker \mathcal{B}= 0$ ist.
	Angenommen dies gilt nicht, so ist $\ker \mathcal{B}$ nach \autoref{satz:2115} ein auflösbares Ideal und es folgt $\rad(\mathfrak{g}) \neq 0$, was einen Widerspruch darstellt.
	
	Ist umgekehrt $\mathcal{B}$ nicht degeneriert, so nehmen wir an, dass $\mathfrak{a}$ ein abelsches Ideal von $\mathfrak{g}$ ist.
	Wir wählen eine Basis $a_1, \ldots ,a_k, b_1, \ldots ,b_s$ von $\mathfrak{g}$ mit $a = \langle a_1, \ldots , a_k\rangle_\mathbb{K}$.
	Dann gilt für $X \in \mathfrak{a}$ und $Y \in \mathfrak{g}$ 
	\[
		(\ad_Y \circ \ad_X)(a_i) = \benbrace[\big]{Y,\benbrace*{X,a_i}} = 0
	\]
	und weiter
	\[
		\enbrace*{\ad_Y \circ \ad_X}(b_i) = \benbrace[\big]{Y, \benbrace*{X,b_i}} \in  \mathfrak{a}
	\]
	Also ist für $X \in \mathfrak{a}$ und $Y \in \mathfrak{g}$ obige Verknüpfung eine strikte obere Dreiecksmatrix.
	Damit ist $\mathcal{B}(Y,X)= \tr (\ad_Y \circ \ad_X)=0$ und somit $\ker \mathcal{B} \neq 0$, ein Widerspruch!
\end{beweis}

Wir haben insbesondere gezeigt, dass $\ker \mathcal{B}$ für jede Liealgebra ein auflösbares Ideal ist und jedes abelsche Ideal enthält.

\begin{beispiel*}[{name=[halbeinfache Liealgebren/Liegruppen]}]
	Halbeinfache Liealgebren sind zum Beispiel 
	\[
		\mathfrak{sl}(n,\mathbb{R}), \mathfrak{sl}(n,\mathbb{C}), \mathfrak{so}(n), \mathfrak{su}(n), \mathfrak{sp}(n) \text{ und } \mathfrak{o}(n,\mathbb{K})
	\]
	Um dies zu sehen, kann man zeigen, dass, falls $\mathfrak{g}$ eine der obigen Liealgebren ist,\marginnote{Levi-Zerlegung: $\mathfrak{g} = \mathfrak{g}_{SS} \oplus \rad(\mathfrak{g})$}
	\[
		\mathcal{B}_{\mathfrak{g}} \enbrace*{X,Y} = \alpha_\mathfrak{g} \cdot \tr(X Y)
	\]
	gilt für eine von der Liealgebra abhängige Konstante $\alpha_\mathfrak{g} \neq 0$.\todo{Übungen einarbeiten?}
	In allen Beispielen rechnet man leicht nach, dass
	\[
		X \in \mathfrak{g} \implies X^T \in \mathfrak{g}
	\]
	Somit ist $\mathcal{B}(X,X^T) = \alpha_\mathfrak{g} \cdot \sum X_{ii}^2 \neq 0$ für $X\neq 0$ und damit nicht degeneriert.
\end{beispiel*}

% Folglich hat die Killingform auf $\mathfrak{sl}(n,\mathbb{R})$ positive und negative \enquote{Eigenwerte}.
Ist etwa $G$ kompakt, so ist $\mathcal{B}_\mathfrak{g}$ negativ definit!

\begin{satz}[label=satz:222,{name=[Eigenschaften halbeinfacher Liealgebren]}]
	Sei $\mathfrak{g}$ eine halbeinfache Liealgebra.
	Dann gilt
	\begin{enumerate}[1)]
		\item $\mathfrak{g} = \mathfrak{g}_1 \oplus  \ldots \oplus \mathfrak{g}_s$, wobei die $\mathfrak{g}_i$ einfache Ideale sind mit $\benbrace*{\mathfrak{g}_i,\mathfrak{g}_j}=0$ für $i \neq j$.
		\item Ist $\mathfrak{a} \subset \mathfrak{g}$ ein Ideal, so folgt $\mathfrak{a} = \mathfrak{g}_{i_1} \oplus  \ldots \oplus \mathfrak{g}_{i_k}$, insbesondere ist die Zerlegung eindeutig bis Umordnung.
		\item Es gilt $\benbrace*{\mathfrak{g},\mathfrak{g}}=\mathfrak{g}$.
		\item $\sfrac{\Aut(\mathfrak{g})}{\Int(\mathfrak{g})}$ ist diskret bzw. $\Der(\mathfrak{g}) = \intAlg(\mathfrak{g})$.
	\end{enumerate}
\end{satz}
\begin{beweis}
	Zunächst einige Vorüberlegungen:
	\begin{enumerate}[i),itemsep=0pt]
		\item \label{enum:222:1} Ist $\mathfrak{a}$ ein Ideal von $\mathfrak{g}$, so auch 
		\(
			\mathfrak{a}^\perp \coloneqq \set*{x \in \mathfrak{g} \given \mathcal{B}(x,y) =0 \,\forall y \in \mathfrak{a}}
		\), wie man sich leicht überlegt:
		Sei $x \in \mathfrak{a}^\perp$ und $y \in \mathfrak{g}$. Zu zeigen: $\benbrace*{x,y} \in \mathfrak{a}^\perp$.
		Für $z \in \mathfrak{a}$ gilt
		\[
			\mathcal{B} \enbrace[\big]{\benbrace*{x,y},z} = - \mathcal{B} \enbrace[\big]{\ad_y(x), z} = \mathcal{B} \enbrace[\big]{x, \ad_y(z)} =0
		\]
		\item \label{enum:222:2} Ist $\mathfrak{a}$ ein Ideal, so ist $\mathfrak{a} \cap \mathfrak{a}^\bot = \set*{0}$.
		Angenommen dies wäre nicht so, so wissen wir dass der Schnitt ein Ideal von $\mathfrak{g}$ ist, auf dem die Killingform verschwindet.
		Mit einer kleinen Rechnung folgt, dass der Schnitt dann auflösbar sein muss.
		Also ist $\rad(\mathfrak{g}) \neq 0$, was einen Widerspruch darstellt.
		\item \label{enum:222:3} Es gilt $\dim \mathfrak{g} = \dim \mathfrak{a} + \dim \mathfrak{a}^\perp$.
		
		Man wähle eine Basis $e_1,\ldots ,e_k$ von $\mathfrak{a}$.
		Dann sind die Gleichungen $\mathcal{B}(e_i,X) =0$ linear unabhängig, da $\mathcal{B}$ nicht degeneriert ist (Cartans Kriterium).
		Insbesondere hat der Lösungsraum Dimension $\dim \mathfrak{g} - \dim \mathfrak{a}$.
		
		%$\mathfrak{g} = \mathfrak{a} \oplus \mathfrak{a}^\perp$ (gilt viel allgemeiner!)
	\end{enumerate}
	Beginnen wir nun mit den eigentlichen Aussagen:
	\begin{enumerate}[1),itemsep=0pt]
		\item Sei $\mathfrak{a}$ ein nichttriviales Ideal von $\mathfrak{g}$.
		Die Vorüberlegungen \labelcref{enum:222:1,enum:222:2,enum:222:3} implizieren $\mathfrak{g} = \mathfrak{a} \oplus \mathfrak{a}^\perp$.
		Induktiv folgt dann die Behauptung.
		\item Sei $\mathfrak{a}$ ein \emph{einfaches} Ideal von $\mathfrak{g}$.
		Dann ist $\mathfrak{a} \cap \mathfrak{g}_1$ entweder $0$ oder $\mathfrak{g}_i$.
		Damit folgt direkt die Behauptung
		% \[
		% 	\mathfrak{a} \cap \mathfrak{g}_1 = \begin{cases}
		% 		0 \\
		% 		\mathfrak{g}_i
		% 	\end{cases}
		% \]
		% mit $\mathfrak{g}_1, \ldots ,\mathfrak{g}_s$ wie in 1).
		% Da $\mathfrak{g}$ kein Zentrum hat existiert $x \in \mathfrak{a}$ und $y \in \mathfrak{g}$ mit $\benbrace*{x,y} \neq 0$.
		% Wir schreiben $y = y_1 + \ldots + y_s$ bezüglich $\mathfrak{g} = \mathfrak{g}_1 \oplus  \ldots \oplus \mathfrak{g}_s$.
		% Dann gilt
		% \[
		% 	0 \neq\benbrace*{x,y} = \sum_{i=1}^{s} \benbrace*{x,y_i} \implies \benbrace*{x,y_{i_0}} \neq 0
		% \]
		% für ein $i_0 \in \set*{1,\ldots ,s}$. Aber $\benbrace*{x,y_{i_0}} \in \mathfrak{g}_{i_0} \cap \mathfrak{a}$.
		\item Ist $\mathfrak{g}$ einfach, so gilt $\benbrace*{\mathfrak{g},\mathfrak{g}}=0$ oder $\benbrace*{\mathfrak{g},\mathfrak{g}}=\mathfrak{g}$.
		Im ersten Fall würde dann $\mathfrak{g}$ abelsch sein, damit die Killingform trivial sein, was im Widerspruch zu Cartans Kriterium steht.
		Mit der Bilinearität von $\benbrace*{\cdot,\cdot}$ und 1) folgt der allgemeine Fall.
		% \[
		% 	\benbrace*{\mathfrak{g},\mathfrak{g}} = \begin{cases}
		% 		0 \\
		% 		\mathfrak{g}
		% 	\end{cases}
		% \]
		% Aus ersterem würde folgen, dass $\mathfrak{g}$ abelsch ist, also die Killlingform trivial ist, was ein Widerspruch ist.
		% Mit 1) folgt nun die Behauptung.
		\item Nach dem Beweis von \autoref{lem:165} ist $\intAlg(\mathfrak{g})$ ein Ideal von $\Der(\mathfrak{g}) = \Tmap_e\!\Aut(\mathfrak{g})$.
		Da $\mathfrak{g}$ halbeinfach ist, gilt $Z(\mathfrak{g}) =0$ und somit ist nach \autoref{lem:164}
		\[
			\intAlg(\mathfrak{g}) \cong \ad(\mathfrak{g}) \cong \mathfrak{g}
		\]
		Damit ist insbesondere $\intAlg(\mathfrak{g})$ halbeinfach.
		Wir bezeichnen nun mit $\mathcal{B}$ die Killingform von $\Der(\mathfrak{g})$.
		Dann gilt nach \cref{sec:einschr_killing}
		\[
			\mathcal{B}\big|_{\intAlg(\mathfrak{g})} = \mathcal{B}_{\intAlg(\mathfrak{g})}
		\]
		Die Vorüberlegungen \labelcref{enum:222:1,enum:222:2,enum:222:3} gelten auch, wenn nur $\mathcal{B}|_{\mathfrak{a}}$ nicht degeneriert ist.
		Somit folgt
		\[
			\Der(\mathfrak{g}) = \intAlg(\mathfrak{g}) \oplus \intAlg(\mathfrak{g})^\perp
		\]
		Sei nun $D \in \intAlg(\mathfrak{g})^\bot$.
		Dann gilt wieder nach \autoref{lem:165} $0= \benbrace*{D,\ad_x} = \ad_{D(x)}$ und somit $D(x) \in Z(\mathfrak{g}) = \set*{0}$.
		Da dies für alle $x \in \mathfrak{g}$ gilt, folgt $D=0$ und damit $\Der(\mathfrak{g}) = \intAlg(\mathfrak{g})$.\qedhere
	\end{enumerate}
\end{beweis}

\begin{bemerkung*}[{name=[Klassifikation der komplexen halbeinfachen Liealgebren]}]
	Halbeinfache komplexe Liealgebren sind von \textsc{Cartan} klassifiziert worden.
	Diese treten in vier unendlichen Familien $A_n$, $B_n$, $C_n$, $D_n$ und fünf sogenannten Ausnahmealgebren $G_2$, $F_4$, $E_6$, $E_7$, $E_8$ auf.
\end{bemerkung*}
% section 22 (end)

\section{Kompakte Liealgebren} % (fold)
\label{sec:23}

\begin{definition}[{name=[{kompakte Liealgebra}]}]
	Eine reelle Liealgebra $\mathfrak{g}$ nennt man \Index{kompakt}, falls es eine kompakte Liegruppe $G$ gibt $\Tmap_e G \cong \mathfrak{g}$. 
\end{definition}

\begin{lemma}[{name=[Existenz schöner Skalarprodukte und biinvarianter Metriken]},label=lem:232]
	Sei $\mathfrak{g}$ eine kompakte Liealgebra.
	Dann gilt:
	\begin{enumerate}[1)]
		\item Es existiert ein Skalarprodukt $\skal*{\cdot}{\cdot}$ auf $\mathfrak{g}$, sodass $\ad_X \colon \mathfrak{g} \to \mathfrak{g}$ \enquote{schiefsymmetrisch}\marginnote{$\skal*{\ad_X x}{y} = - \skal*{x}{\ad_Xy}$} bezüglich $\skal*{\cdot}{\cdot}$ ist für alle $X \in \mathfrak{g}$.
		\item $G$ besitzt eine \Index{biinvariante Riemannsche Metrik}\footnote{siehe \url{https://de.wikipedia.org/wiki/Riemannsche_Mannigfaltigkeit}} $g_\bi$, das heißt die Diffeomorphismen $L_g,R_g \colon G \to G$ sind alle für alle $g \in G$ Isometrien.
	\end{enumerate}
\end{lemma}
\begin{beweis}
	\begin{enumerate}[1)]
		\item Sei $\skal*{\cdot}{\cdot}_0$ \emph{irgendein} Skalarprodukt auf $\mathfrak{g}$.
		Wir setzen 
		\[
			\skal*{x}{y} \coloneqq \int_G \skal[\big]{\Ad(g)x}{\Ad(g)y}_0 \,\omega
		\]
		wobei $\omega$ eine \Index{biinvariante Volumenform} auf $G$ ist, das heißt es gilt $L^*_g \omega = R^*_g \omega= \omega$ (wir können $G$ als zusammenhängend annehmen). 
		Die Existenz von $\omega$ kann man sich wie folgt überlegen:
		Sei $\omega_0$ eine Volumenform auf $\mathfrak{g}$.
		Dann lässt sich diese mittels $\omega_g = L_g^* \omega_0$ auf $G$ fortsetzen.
		Man kann nun eine Funktion $f \colon G \to \mathbb{R}$ definieren mittels $R_g^* \omega = f(g) \omega$.
		Man rechnet nun leicht $f(gh) =f(g) f(h)$ nach, und somit muss $f(G)$ eine kompakte Untergruppe von $\mathbb{R}^*$ sein.
		Dann muss aber $f(G)=1$ gelten und damit ist $\omega$ auch rechtsinvariant.
		
		Sei nun $h \in G$.
		Dann gilt 
		\begin{align}
			\skal[\big]{\Ad(h) X}{\Ad(h)Y} &= \int_G \skal[\big]{\Ad(g) \Ad(h)(X)}{ \Ad(g) \Ad(h)(Y)}_0 \, \omega \\
			&= \int_G \skal[\big]{\Ad(gh)X}{\Ad(gh)Y}_0 \, R_h^* \enbrace*{R^*_{h^{-1}} \omega} \\
			&\StackTextClap{Trafo}{=} \int_G \skal[\big]{\Ad(g) X}{ \Ad(g)Y}_0 \, \Underbracket{R^*_{h^{-1}} \omega}{=\omega} = \skal*{X}{Y}
		\end{align}
		Somit ist $\skal*{\cdot}{\cdot}$ ein $\Ad(G)$-invariantes Skalarprodukt, das heißt $\Ad(G)\subset \On(\mathfrak{g},\skal*{\cdot}{\cdot})$, $G$ wirkt also durch Isometrien auf $\mathfrak{g}$.
		Da $\enbrace*{\mathd \Ad}_e = \ad$ folgt die Behauptung (die Liealgebra von $\On(n)$ ist die der schiefsymmetrischen Matrizen).
		\item Sei $\skal*{\cdot}{\cdot}$ ein $\Ad(G)$-invariantes Skalarprodukt auf $\mathfrak{g}$ wie eben konsturiert.
		Wir definieren $(g_\bi)_g \colon \Tmap_g G \times \Tmap_g G \to \mathbb{R}$ durch
		\[
			(g_\bi)_g(X,Y) \coloneqq \skal*{\enbrace*{\mathd L_{g^{-1}}}_g X}{ \enbrace*{\mathd L_{g^{-1}}}_g Y}
		\]
		Dies ist eine glatte(!) Riemannsche Metrik (Übung).
		Klar ist: Die Diffeomorphismen $L_h$ sind Isometrien!
		Zu zeigen $(\mathd R_h)_g \colon \Tmap_g G \to \Tmap_{R_h(g)} G $ ist eine Isometrie:
		Wegen $R_{gh} = R_h \circ R_g$, folgt mit der Kettenregel
		\[
			\enbrace*{\mathd R_{gh}}_e = \enbrace*{\mathd R_h}_g \cdot \enbrace*{\mathd R_g}_e
		\]
		Somit genügt es zu zeigen, dass $(\mathd R_g)_e \colon \Tmap_e G \to \Tmap_g G$ eine Isometrie ist für alle $g \in G$.
		Es seien $X,Y \in \mathfrak{g} = \Tmap_e G$.
		Dann gilt
		\begin{align}
			g_\bi \enbrace*{(\mathd R_g)_e \cdot X, \enbrace*{\mathd R_g}_e \cdot Y}_g = \skal*{ \enbrace*{\mathd L_{g^{-1}}}_g \enbrace*{\mathd R_g}_e \cdot X}{\cdots}
			&= \skal*{\Ad(g^{-1}) X}{ \Ad(g^{-1})Y} \\
			&= \skal*{X}{Y} = g_\bi(X,Y)_e
		\end{align}
		da $\skal*{\cdot }{\cdot}$ nach 1 $\Ad(G)$-invariant ist.
		Dies zeigt die Behauptung.\qedhere
	\end{enumerate}
\end{beweis}

\begin{lemma}[{name=[Eigenschaften kompakter Liealgebren]},label=lem:233]
	Sei $\mathfrak{g}$ eine reelle Liealgebra.
	Dann gilt 
	\begin{enumerate}[1)]
		\item $\mathcal{B} < 0 \iff \mathfrak{g}$ ist kompakt mit $Z(\mathfrak{g})=0$.
		\item Wenn $\mathfrak{g}$ kompakt ist, dann gilt 
		\[
			\mathfrak{g} = \benbrace*{\mathfrak{g},\mathfrak{g}} \oplus Z(\mathfrak{g})
		\]
		und $\benbrace*{\mathfrak{g},\mathfrak{g}}$ ist halbeinfach.
	\end{enumerate}
\end{lemma}
\begin{beweis}
	\begin{enumerate}[1)]
		\item Sei zunächst $\mathcal{B} < 0$.
		Dann ist $\mathfrak{g}$ halbeinfach, da nicht ausgeartet, und hat somit ein triviales Zentrum.
		Da nach \autoref{lem:168} die Killingform invariant unter Automorphismen ist, gilt $\Aut(\mathfrak{g}) \subset \On(\mathfrak{g},-\mathcal{B})$.\marginnote{$\On(n)$ ist kompakt, da abgeschlossene Teilmenge der Einheitskugel bzgl. Spektralnorm}
		Ferner ist nach \autoref{prop:162} $\Aut(\mathfrak{g})$ abgeschlossen in $\End(\mathfrak{g})$.
		Somit ist $\Aut(\mathfrak{g})$ kompakt.
		Nach \autoref{satz:222} 4) ist $\sfrac{\Aut(\mathfrak{g})}{\Int(\mathfrak{g})}$ diskret, also endlich.
		Somit ist
		\[
			\enbrace*{\Aut(\mathfrak{g})}_e = \Int(\mathfrak{g})
		\]
		$\Int(\mathfrak{g})$ ist also auch kompakt und wegen $\intAlg(\mathfrak{g}) \cong \mathfrak{g}$ folgt, dass $\mathfrak{g}$ eine kompakte Liealgebra mit $Z(\mathfrak{g})= \set*{0}$ ist.
		
		Sei nun $\mathfrak{g}$ eine kompakte Liealgebra mit trivialem Zentrum.
		Wähle ein $\Ad(G)$-invariantes Skalarprodukt $\skal*{\cdot}{\cdot}$ auf $\mathfrak{g}$.
		Dann gilt 
		\[
			- \mathcal{B}(X,X) = - \tr \enbrace*{\ad_X \circ \ad_X} = \tr \enbrace*{\ad_X \circ \ad_X^T} = \norm*{\ad_X}^2 \ge 0
		\]
		Somit ist $\mathcal{B} \le 0$ und $\mathcal{B}(X,X)=0 \iff \ad_X = 0 \iff X \in Z(\mathfrak{g})$.
		\item Wie im Beweis von \autoref{satz:222} 1) zeigt man, dass $\mathfrak{g} = Z(\mathfrak{g}) \oplus \mathfrak{b}$ für ein Ideal $\mathfrak{b}$.\footnote{$\skal*{\cdot}{\cdot}$ $\Ad(G)$-invariantes Skalarprodukt, $\mathfrak{b} \coloneqq Z(\mathfrak{g})^\perp \implies \ad_X$ schiefsymmetrisch}
		Wir wissen $\mathcal{B} \le 0$ und $\mathcal{B}(X,X) =0 \iff X \in Z(\mathfrak{g})$.
		Somit gilt für die Killingform $\mathcal{B}_\mathfrak{b} = \mathcal{B}_{\mathfrak{g}}|_{\mathfrak{b}} < 0$, nach Cartan ist also $\mathfrak{b}$ halbeinfach und damit folgt
		\[
			\benbrace*{\mathfrak{g},\mathfrak{g}} \stackrel{\mathfrak{g} = Z(\mathfrak{g}) \oplus \mathfrak{b}}{=\joinrel=\joinrel=\joinrel=\joinrel=\joinrel=}\benbrace*{\mathfrak{b},\mathfrak{b}} \StackText{\ref{satz:222} 3)}{=\joinrel=\joinrel=}\mathfrak{b}
		\]
		Dies zeigt $\mathfrak{g} = Z(\mathfrak{g}) \oplus \benbrace*{\mathfrak{g},\mathfrak{g}}$.\qedhere
	\end{enumerate}
\end{beweis}

\begin{korollarB}[{name=[kompakte Liegruppe mit endlichem Zentrum halbeinfach]}]
	Eine kompakte Liegruppe mit endlichem Zentrum ist halbeinfach.
\end{korollarB}

\begin{lemma}[{name={Weyl}},label=lem:235]
	Sei $G$ eine kompakte Liegruppe mit endlichem Zentrum.\marginnote{$\SO(n)$, $\SU(n)$, $\Sp(n)$ aber $\cancel{\Un(n)}$}
	Dann ist 
	\[
		\abs*{\pi_1(G)} < \infty
	\]
	Damit ist auch jede weitere Liegruppe mit Liealgebra $\mathfrak{g}$ kompakt.
\end{lemma}
\begin{beweis}
	\emph{Kommt noch. Wir werden zeigen, dass auf $G$ eine Riemannsche Metrik mit Ricci-Krümmung $\Ric >0$ existiert. Mit Bonnet-Myers folgt dann zusammen mit der Kompaktheit die Behauptung.}\todo{überprüfen, ob das wirklich nochmal drankommt \ldots}
\end{beweis}

\begin{korollar}[{name=[Isomorphie kompakte Liegruppe]}]
	Jede kompakte Liegruppe $G$ ist isomorph zu 
	\[
		\faktor{T^k \times G_1 \times \ldots \times G_s}{\Gamma}
	\]
	wobei $G_1, \ldots ,G_s$ kompakte, zusammenhängende, einfach zusammenhängende Liegruppen sind und $\Gamma \subset Z(T^k \times G_1 \times \ldots \times G_s)$ eine endliche Untergruppe ist.
\end{korollar}
\begin{beweis}
	Nach \autoref{lem:233} 2) und \autoref{satz:222} 1) ist $\mathfrak{g}$ isomorph zu $\mathbb{R}^n \times \mathfrak{g}_1 \times \ldots \times \mathfrak{g}_m$ mit $\mathfrak{g}_i$ einfach.
	Damit ist die zusammenhängende, einfach zusammenhängende Liegruppe mit Liealgebra $\mathfrak{g}$ isomorph zu $\widetilde{G} = \mathbb{R}^n \times G_1 \times \ldots \times G_m$, wobei die $G_i$ einfach zusammenhängend und einfach sind. 
	Insbesondere ist diese Liegruppe die universelle Überlagerung von $G$.
	Nach Weyl (\autoref{lem:235}) sind auch die $G_i$ kompakt.\todo{das verstehe ich noch nicht}
	Es ist nun $G = \sfrac{\widetilde{G}}{\widetilde{\Gamma}}$ für eine diskrete Untergruppe $\widetilde{\Gamma}$ des Zentrums von $\widetilde{G}$ (vgl. \cref{sec:13}).
	Die Zentren der $G_i$ sind nun aber endlich, womit die Projektion von $\widetilde{\Gamma}$ auf $G_1\times \ldots \times G_m$ als Homomorphismus endlichen Index hat.
	Außerdem ist ihr Kern $\overline{\Gamma}$ eine Untergruppe von $\mathbb{R}^n$.
	Da $G$ kompakt ist, muss auch $\sfrac{\mathbb{R}^n}{\overline{\Gamma}}$ kompakt sein und damit isomorph zu einem Torus $T^n$.
	
	Insgesamt wirkt die Gruppe $\Gamma = \sfrac{\widetilde{\Gamma}}{\overline{\Gamma}}$ auf $T^n \times G_1 \times \ldots \times G_m$ mit dem Quotienten $G$.
\end{beweis}

\begin{lemma}[{name=[Exponentialabbildung surjektiv für kompakte und zusammenhängend]},label=lem:237]
	Sei $G$ kompakte zusammenhängende Liegruppe.
	Dann ist die Exponentialabbildung $\exp\colon \mathfrak{g} = \Tmap_e G \to G$ surjektiv.
\end{lemma}
\begin{beweis}
	Wir werden zeigen, dass Einparametergruppen genau die Geodätischen von $(G,g_\bi)$ sind. 
	$G$ ist kompakt und damit vollständig bezüglich der von $g_\bi$ induzierten Metrik.
	Mit Hopf-Rinow\footnote{siehe \url{https://de.wikipedia.org/wiki/Satz_von_Hopf-Rinow}} folgt, dass die geometrische (=algebraische) Exponentialabbildung surjektiv ist.
\end{beweis}

\section{Maximale Tori und Weylgruppen} % (fold)
\label{sec:24}

\begin{definition}[{name=[maximal abelsche Unteralgebra]}]
	Sei $\mathfrak{g}$ eine Liealgebra.
	Dann nennt man eine Unteralgebra $\mathfrak{t} \subseteq \mathfrak{g}$ \Index{maximal abelsch}, falls $\mathfrak{t}$ abelsch ist und gilt: Ist $\mathfrak{t}' \subseteq \mathfrak{g}$ eine weitere abelsche Unteralgebra mit $\mathfrak{t} \subseteq \mathfrak{t}'$, so gilt bereits $\mathfrak{t} = \mathfrak{t}'$.
\end{definition}

\begin{beispiel*}[{name=[maximal abelsche Unteralgebren]}]
	\begin{itemize}
		\item Falls $\mathfrak{g}$ bereits abelsch ist, gilt natürlich $\mathfrak{t} = \mathfrak{g}$.
		\item Betrachte $G=\SO(2n)$.
		Dann ist 
		\[
			\mathfrak{t} = \set*{\begin{pmatrix}
				\alpha_1\begin{psmallmatrix}
					0 & -1 \\ 1 & 0
				\end{psmallmatrix} & & 0 \\
				& \ddots &  \\
				0 & & \alpha_n \begin{psmallmatrix}
					0 & -1 \\ 1 & 0
				\end{psmallmatrix}
			\end{pmatrix} \given \alpha_1,\ldots ,\alpha_n \in \mathbb{R}}
		\]
		die maximale abelsche Unteralgebra von $\mathfrak{so}(2n)$.
		\item Für $G = \SL(n,\mathbb{R})$ ist
		\[
			\mathfrak{t} = \set*{\begin{pmatrix}
				\alpha_1 & & 0 \\
				& \ddots & \\
				0 & & \alpha_n
			\end{pmatrix} \given \alpha_i \in \mathbb{R}, \sum_i \alpha_i = 0}
		\]
		die maximal abelsche Unteralgebra von $\mathfrak{sl}(n,\mathbb{R})$
	\end{itemize}
\end{beispiel*}

\begin{lemma}[{name=[Eigenschaften einer maximalen abelschen Unteralgebra]},label=lem:242]
	Sei $G$ eine kompakte Liegruppe und $\mathfrak{t} \subseteq \mathfrak{g}$ maximale abelsche Unteralgebra.
	Dann gilt
	\begin{enumerate}[(1)]
		\item Für alle $v \in \mathfrak{g}$ existiert ein $g \in G$ mit $\Ad(g) v \in \mathfrak{t}$.
		\item Sind $\mathfrak{t}_1$ und $\mathfrak{t}_2$ maximale abelsche Unteralgebren von $\mathfrak{g}$, so existiert ein $g \in G$ mit $\Ad(g)\mathfrak{t}_1 = \mathfrak{t}_2$.
	\end{enumerate}
\end{lemma}
\begin{beweis}
	Sei $T = \exp \mathfrak{t}$.
	Angenommen $T$ sei nicht abgeschlossen, so wäre $T' \coloneqq \overline{\exp \mathfrak{t}}$ eine zusammenhängende, abgeschlossene Untergruppe von $G$, also eine Lieuntergruppe.
	Da $T$ offenbar abelsch ist, ist auch $T'$ abelsch. 
	Wegen $T \subsetneq T'$ folgt $\mathfrak{t} \subsetneq \mathfrak{t}'$, was ein Widerspruch ist.
	Somit ist $T$ abgeschlossen, also kompakt, das heißt $T = S^1 \times \ldots \times S^1$.
	
	In den Übungen wird gezeigt, dass ein $\overline{v} \in \mathfrak{t}$ existiert mit $\overline{\set*{\exp(s \overline{v}) \given s \in \mathbb{R}}} = T$.\marginnote{siehe \cref{sec:dichte_teilmenge_torus}}
	Wir zeigen nun 
	\[
		\mathfrak{t} = \set[\big]{w \in \mathfrak{g} \given \benbrace*{\overline{v},w}=0}
	\]
	Die Inklusion \enquote{$\subseteq$} ist dabei klar.
	Sei also $w \in \mathfrak{g}$ mit $\benbrace*{\overline{v},w} =0$.
	Dann gilt
	\[
		w = \enbrace*{e^{s \ad_{\overline{v}}}}(w) \StackText{\ref{lem:153}}{=\joinrel=} \Ad \enbrace[\big]{\exp(s \overline{v})}(w)
	\]
	Mit der Dichtheit folgt $\Ad(T)(w)=w$ und Differenzieren liefert $\benbrace*{\mathfrak{t},w}=0$.
	Wegen der Maximalität von $\mathfrak{t}$ muss also $w \in \mathfrak{t}$ gelten.
	
	Sei nun $v \in \mathfrak{g}$ und $\skal*{\cdot}{\cdot}$ ein $\Ad(G)$-invariantes Skalarprodukt auf $\mathfrak{g}$.
	Wir betrachten die Funktion
	\mapdef{f \colon G}{\mathbb{R}}{g}{\skal*{\overline{v}}{\Ad(g)v}}{}
	Da $G$ kompakt ist, nimmt $f$ sein Minimum an, etwa in $g_0 \in G$.
	Es folgt
	\begin{align}
		0 = (\mathd f)_{g_0} (\mathd R_{g_0})_e z = \diffd{}{t}\Big|_{t=0} \skal[\Big]{\overline{v}}{\Ad \enbrace[\big]{\exp(t z)g_0} v} = \skal[\Big]{\overline{v}}{\ad(z) \enbrace[\big]{\Ad(g_0) v}}
		&= \skal[\Big]{\overline{v}}{\benbrace[\big]{z,\Ad(g_0) v}} \\
		&= -\skal[\Big]{\benbrace[\big]{\overline{v},\Ad(g_0)v}}{z}
	\end{align}
	für alle $z \in \mathfrak{g}$.
	Somit ist $\benbrace*{\overline{v}, \Ad(g_0)v}=0$, das heißt $\Ad(g_0) v \in \mathfrak{t}$.
	Damit ist (1) gezeigt.
	
	Zu (2): Sei $\overline{v}_i \in \mathfrak{t}_i$ mit $\overline{\set*{\exp(s \overline{v}_i) \given s \in \mathbb{R}}} = T_i$ für $i=1,2$ und die beiden maximalen abelschen Unteralgebren $\mathfrak{t}_1$ und $\mathfrak{t}_2$.
	Nach (1) existiert $g \in G$ mit $\Ad(g)(\overline{v}_1) \in \mathfrak{t}_2$.
	Da $\mathfrak{t}_2$ abelsch ist, gilt
	\[
		0 = \benbrace[\big]{\Ad(g) \overline{v}_1, t_2} = \benbrace*{\overline{v}_1, \Ad(g^{-1}) t_2}
	\]
	Dies zeigt $\Ad(g^{-1})(\mathfrak{t}_2) \subset \mathfrak{t}_1$.
	Vertauscht man die Rollen von $\mathfrak{t}_1$ und $\mathfrak{t}_2$, so folgt $\dim \mathfrak{t}_1 = \dim \mathfrak{t}_2$ und wir erhalten $\Ad(g^{-1})(\mathfrak{t}_2) = \mathfrak{t}_1$.
\end{beweis}

\begin{definition}[{name=[maximaler Torus einer Liegruppe]}]
	Sei $\mathfrak{t}$ eine maximale abelsche Unteralgebra von $\mathfrak{g}$.
	Dann nennt man $T= \exp(\mathfrak{t})$ einen \bet{maximalen Torus}\index{maximaler Torus} von $G$. 
\end{definition}

\begin{satz}[{name=[maximale Tori sind konjugiert]},label=satz:244]
	Sei $G$ eine zusammenhängende, kompakte Liegruppe.
	Dann gilt
	\begin{enumerate}[(1)]
		\item Sei $T$ ein maximaler Torus und $g_0 \in G$.
		Dann existiert ein $g \in G$ mit $g g_0 g^{-1} \in T$.
		\item Je zwei maximale Tori sind konjugiert.
	\end{enumerate}
\end{satz}
\begin{beweis}
	\begin{enumerate}[(1)]
		\item Sei $g_0 \in G$.
		Nach \autoref{lem:237} existiert ein $v_0 \in \mathfrak{g}$ mit $\exp(v_0)= g_0$.
		Nach \autoref{lem:242} existiert wiederum ein $g \in G$ mit $\Ad(g)(v_0) \in \mathfrak{t} = \Tmap_e T$.
		Es folgt nun
		\[
			g g_0 g^{-1} = g \exp(v_0) g^{-1} = \exp \enbrace*{\Ad(g) v_0} \in \exp(\mathfrak{t}) = T
		\]
		\item Man wähle $v_0 \in \mathfrak{t}_1$ mit $\overline{\set*{\exp(s v_0) \given s \in \mathbb{R}}} =T_1$ und $g \in G$ mit $\Ad(g)(v_0) \in \mathfrak{t}_2$, das heißt $\Ad(g)(s v_0) \in \mathfrak{t}_2$ für alle $s \in \mathbb{R}$.
		Die Behauptung folgt nun genauso wie eben.\qedhere
	\end{enumerate}
\end{beweis}

\begin{definition}[{name=[Rang einer Liegruppe]}]
	Sei $G$ eine kompakte Liegruppe.
	Dann nennt man die Dimension eines maximalen Torus in $G_e$ den \Index{Rang} von $G$.
\end{definition}

\begin{lemma}[label=lem:246,{name=[Zentralisator eines maximalen Torus]}]
	Sei $G$ eine zusammenhängende, kompakte Liegruppe.
	Dann gilt
	\begin{enumerate}[(1)]
		\item Sei $\tilde{T} \subseteq G$ ein Torus (d.h. eine kompakte abelsche Untergruppe) und $g \in G$ mit $g \tilde{t} = \tilde{t}g$ für alle $\tilde{t} \in \tilde{T}$.
		Dann existiert ein maximaler Torus $T$ von $G$ mit $g \in T$ und $\tilde{T} \subseteq T$.
		\item Der \Index{Zentralisator} $c(T) = \set*{g \in G \given g t =tg \,\forall t \in T}$ eines maximalen Torus $T$ von $G$ ist $T$ selbst.
	\end{enumerate}
\end{lemma}
\begin{beweis}
	Offensichtlich folgt (2) direkt aus (1).
	Für sei $A \subseteq G$ der Abschluss der von $g$ und $\tilde{T}$ erzeugten Untergruppe von $G$.
	$A$ ist kompakt und abelsch und somit ist $A_e$ ein Torus.
	Die Nebenklassen in $\sfrac{A}{A_e}$ werden wegen $\tilde{T} \subseteq A_e$ von $g A_e$ erzeugt und da $A$ kompakt ist, ist $\sfrac{A}{A_e}$ endlich.\footnote{also $A = A_e \mathbin{\dot{\cup}} g A_e \mathbin{\dot{\cup}} g^2 A_e \mathbin{\dot{\cup}} \ldots \mathbin{\dot{\cup}} g^{m-1}A_e$}
	Folglich ist $A \cong \sfrac{\mathbb{Z}}{m \mathbb{Z}} \times T^k$ für ein $k \ge 1$, $m \ge 1$ minimal mit $g^m \in A_e$.
	In den Übungen wird gezeigt,\todo{überprüfen} dass jeder Torus $T'$ \enquote{viele} Erzeuger $t'$ besitzt, das heißt $\overline{\set*{(t')^k \given k \in \mathbb{Z}}} = T'$ (Kroneckers Satz).
	Sei nun $\tilde{t}$ ein Erzeuger von $\tilde{T}$.
	Wir wählen $\tilde{t}' \in \tilde{T}$ mit $(\tilde{t}')^m = g^{-m} \tilde{t}$.
	Dann ist $g \tilde{t}'$ ein Erzeuger von $A$ (!).
	Da das Element $g \tilde{t}$ in einem maximalen Torus $T$ liegt, folgt $A \subseteq T$.
\end{beweis}

\begin{definition}[{name=[adjungierter Orbit]}]
	Sei $G$ eine Liegruppe mit Liealgebra $\mathfrak{g}$.
	Dann nennt man 
	\[
		\Ad(G)(v) = \set[\big]{\Ad(g)(v) \given g \in G}
	\]
	den \bet{adjungierten Orbit}\index{adjungierter Orbit} von $v \in \mathfrak{g}$. 
\end{definition}

Ist $G$ abelsch und zusammenhängend, so ist $\Ad(G)(v) = \set*{v}$ für alle $v \in \mathfrak{g}$.
Ist $\mathfrak{t} \subset \mathfrak{g}$ eine maximale abelsche Unteralgebra von $\mathfrak{g}$, dann folgt aus \autoref{lem:242} (1), dass für alle $v \in \mathfrak{g}$ der entsprechende adjungierte Orbit die maximal abelsche Unteralgebra $\mathfrak{t}$ schneidet.

\begin{definition}[{name=[Weyl-Gruppe]}]
	Sei $G$ eine zusammenhängende, kompakte Liegruppe und $T$ ein maximaler Torus von $G$.
	Dann nennt man\marginnote{$N_G(T)$ Normalisator von $T$}
	\[
		W = \sfrac{N_G(T)}{T}
	\]
	die \Index{Weyl-Gruppe} von $G$. 
\end{definition}

\begin{satz}[{name=[Weyl-Gruppe und adjungierte Orbiten]}]
	Sei $G$ eine kompakte, zusammenhängende Liegruppe mit maximalem Torus $T$ und Weyl-Gruppe $W$.
	Dann gilt
	\begin{enumerate}[1)]
		\item $W$ ist eine endliche Gruppe, die effektiv auf $\mathfrak{t} = \Tmap_e T$ operiert vermittels
		\[
			w.v \coloneqq \Ad(n)(v) \qquad \text{ falls } w = nT 
		\]
		\item Sei $v \in \mathfrak{t}$.
		Dann gilt 
		\[
			\Ad(G)(v) \cap \mathfrak{t} = W.v
		\]
		\item Adjungierte Orbiten schneiden maximale Tori orthogonal, wenn man $\mathfrak{g}$ mit einem $\Ad(G)$-invarianten Skalarprodukt versieht.
	\end{enumerate}
\end{satz}
\begin{beweis}
	Da $N_G(T)$ eine abgeschlossene Untergruppe von $G$ ist, ist $N_G(T)$ kompakt.
	Wir zeigen $(N_G(T))_e = T$.
	Dann ist $W=\sfrac{N_G(T)}{T}$ kompakt und diskret, also endlich.
	Für die Liealgebra $\mathfrak{n}(\mathfrak{t})$ von $N_G(T)$ gilt
	\[
		\mathfrak{n}(\mathfrak{t}) = \set[\big]{v \in \mathfrak{g} \given \benbrace*{v,\mathfrak{t}} \subseteq \mathfrak{t}}
	\]
	Wähle ein $\overline{v} \in \mathfrak{t}$, sodass $\set{\exp(s \overline{v}) \given s \in \mathbb{R}}$ dicht in $T$ ist.
	Sei nun $v \in \mathfrak{n}(\mathfrak{t})$.
	Dann gilt einerseits $\benbrace*{v,\overline{v}} \in \mathfrak{t}$, andererseits für $w \in \mathfrak{t}$
	\[
		\skal*{\benbrace*{v,\overline{v}}}{w} = \skal*{v}{ \benbrace*{\overline{v},w}} \stackrel[\mathfrak{t} \text{ abelsch} ]{\benbrace*{\overline{v},w}=0}{=\joinrel=\joinrel=\joinrel=} 0
	\]
	das heißt $\benbrace*{v, \overline{v}}=0$.
	Wir erhalten also $\benbrace*{\mathfrak{n}(\mathfrak{t}),\overline{v}}=0$.
	Im Beweis von \autoref{lem:242} haben wir gezeigt, dass $\mathfrak{t} = \set*{w \in \mathfrak{g} \given \benbrace*{w, \overline{v}}=0}$.
	Damit folgt also $\mathfrak{n}(\mathfrak{t}) \subseteq \mathfrak{t}$ und somit die Gleichheit.
	$W$ ist also endlich.
	
	Es bleibt zu zeigen, dass $W$ effektiv operiert.
	Da $T$ abelsch ist, gilt $\Ad(T)|_{\mathfrak{t}} = {\id_{\mathfrak{t}}}$.
	Somit ist die Wirkung von $w$ auf $\mathfrak{t}$ wohldefiniert.
	Sei nun $n \in N_G(T)$ mit $\Ad(n)|_{\mathfrak{t}} = {\id}$.
	Dann gilt für alle $v \in \mathfrak{t}$
	\[
		\exp(v) = \exp \enbrace[\big]{\Ad(n) v} = n \exp(v) n^{-1}
	\]
	Damit ist $n \in Z(T) =T$ nach \autoref{lem:246} (2), das heißt $n W = W$, die Wirkung ist also effektiv\marginnote{=treu}.
	
	Kommen wir nun zu 2).
	Es sein $v,\tilde{v} \in \mathfrak{t}$ und $g \in G$ mit $\Ad(g)(v) = \tilde{v}$.
	Zu zeigen ist, dass ein $n \in N_G(T)$ existiert mit $\Ad(n)(v) = \tilde{v}$.
	Wir setzen
	\[
		Z_v \coloneqq \set[\big]{\tilde{g} \in G \given \Ad(g) v = v}
	\]
	Dann gilt $T \subseteq Z_v$ und $g^{-1}Tg \subseteq Z_v$, denn
	\[
		\Ad \enbrace*{g^{-1} Tg}v = \Ad(g^{-1}) \Ad(T) \Ad(g) v = \Ad(g^{-1}) \Ad(T) \tilde{v} = \Ad(g^{-1}) \tilde{v} = v
	\]
	Nach \autoref{satz:244} (2) sind die maximalen Tori $T$ und $g^{-1} T g$ konjugiert in $(Z_v)_e$, das heißt es existiert ein $h \in (Z_v)_e$ mit $h T h^{-1} = g^{-1} T g$.
	Es folgt $gh \in N_G(T)$ und somit
	\[
		\Ad(gh)(v) = \Ad(g) \Ad(h) v = \Ad(g) v = \tilde{v}
	\]
	Mit $n \coloneqq gh$ folgt damit 2).
	
	Um 3) zu zeigen wählen wir $v \in \mathfrak{t}$ und $\tilde{v} \in \Ad(G) v \cap t$.
	Dann gilt für den Tangentialraum des Orbit $\Ad(G)v = \Ad(G)\tilde{v}$
	\[
		\Tmap_{\tilde{v}} \enbrace*{\Ad(G)v} = \Tmap_{\tilde{v}} \enbrace*{\Ad(G) \tilde{v}} = \set[\big]{\benbrace*{w,\tilde{v}} \given w \in \mathfrak{g}}
	\]
	Ferner gilt für $t \in \mathfrak{t}$
	\[
		\skal[\big]{\benbrace*{w,\tilde{v}}}{t} = \skal[\big]{w}{\benbrace*{\tilde{v}, t}} \stackrel[\mathfrak{t} \text{ abelsch} ]{\benbrace*{\overline{v},w}=0}{=\joinrel=\joinrel=\joinrel=}0
	\]
	Also ist $\Tmap_{\tilde{v}} \enbrace*{\Ad(G)v} \perp \mathfrak{t}$ wie gewünscht.
\end{beweis}

\begin{bemerkung*}[{name=[Klassifikation kompakter zusammenhängender einfach-zusammenhängender Liegruppen]}]
	Kompakte zusammenhängende, einfach zusammenhängende Liegruppen kann man klassifizieren: 
	\begin{itemize}
		\item $\operatorname{Spin}(n)$, $n \ge 3$, $\sfrac{\operatorname{Spin}}{\mathbb{Z}_2} = \SO(n)$
		\item $\SU(n)$, $n \ge 2$
		\item $\Sp(n)$, $n \ge 1$
		\item $G_2$, $F_4$, $E_6$, $E_7$, $E_8$ mit Dimension $14,52,78,133,248$.
	\end{itemize}
\end{bemerkung*}

\begin{beispiel*}[{name=[maximale Tori der speziellen unitären Gruppe]}]
	Die maximalen Tori in $\SU(n)$ sind 
	\[
		T = \set*{\begin{pmatrix}
			e^{i \varphi_1} & & \\
			& \ddots & \\
			& & e^{i \varphi_n}
		\end{pmatrix} \given \sum_{i=1}^{n} \varphi_i \in 2 \pi \mathbb{Z}}
	\]
	Die Weylgruppe ist $W = \operatorname{Sym}(n)$. Somit ist der Rang $n-1$.
	Es gilt 
	\[
		t = \Tmap_e T = \set*{ \begin{pmatrix}
			i \alpha_1 & & \\
			& \ddots & \\
			& & i \alpha_n
		\end{pmatrix} \given \sum_{i=1}^{n} \alpha_i = 0}
	\]
	Alle adjungierten Orbiten sind von der Form $\sfrac{\SU(n)}{\SU(n_1) \times \ldots \times \Un(n_k)}$ mit
	\[
		\sum_{j=1}^{1n} n_j = n \quad , \quad \alpha_1 = \ldots = \alpha_{n-1}, \alpha_{n_1 +1} = \ldots = \alpha_{n_1 +n_2}
	\]
	\todo[inline]{das Ganze könnte man mal nachrechnen}
\end{beispiel*}
% section 24 (end)





% section 23 (end)
% chapter 2 (end)

\chapter{Homogene Räume} % (fold)
\label{cha:3}

\section{Definition und Beispiele} % (fold)
\label{sec:31}

\begin{definition}[{name=[homogener Raum]}]
	Sei $G$ eine reelle Liegruppe und $H$ eine abgeschlossene Untergruppe von $G$.
	Dann nennt man den Raum $\sfrac{G}{H} \coloneqq \set*{g H \given g \in G}$ der Linksnebenklassen von $H$ auch \Index{homogener Raum}. 
	Wir bezeichnen mit $\pi \colon G \to \sfrac{G}{H}$, $g \mapsto gH$ die Projektionsabbildung und versehen $\sfrac{G}{H}$ mit der Quotiententopologie.
	Das heißt $U \subseteq \sfrac{G}{H}$ ist offen genau dann, wenn $\pi^{-1}(U)$ offen in $G$ ist.
\end{definition}

\begin{satz}[{name=[homogener Raum ist differenzierbare Mannigfaltigkeit]},label=satz:312]
	Sei $G$ eine Liegruppe und $H$ eine abgeschlossene Untergruppe von $G$.
	Dann ist der homogene Raum eine differenzierbare Mannigfaltigkeit der Dimension $n = \dim G - \dim H$.
	Ferner ist die Projektionsabbildung $\pi \colon G \to \sfrac{G}{H}$ eine differenzierbare Submersion, das heißt für alle $g \in G$ ist $(\mathd \pi)_g \colon \Tmap_g G \to \Tmap_{gH} \sfrac{G}{H}$ surjektiv.
\end{satz}
\begin{beweis}
	Der Beweis verläuft in mehreren Schritten.
	\begin{enumerate}[(i)]
		\item $\pi$ ist offen und $\sfrac{G}{H}$ erfüllt das zweite Abzählbarkeitsaxiom:
		
		Aus $U \subseteq G $ offen, folgt dass auch $\pi^{-1}(\pi(U)) = \bigcup_{h \in H} R_h(U)$ offen ist und somit $\pi(U)$ offen ist.
		Ist $(B_n)_n$ eine abzählbare Basis der Topologie auf $G$, so ist $\enbrace*{\pi(B_n)}_n$ eine Basis der Topologie auf $\sfrac{G}{H}$, da $\pi$ offen ist.
		\item $\sfrac{G}{H}$ ist ein $T_2$-Raum:
		
		Seien $g_1 H, g_2 H \in \sfrac{G}{H}$ mit $g_1 H \neq g_2 H$.
		Dann gilt $g_1 \notin g_2 H$.
		Da $H$ abgeschlossen ist und $L_{g_i}$ ein Diffeomorphismus ist, ist auch $g_i H$ abgeschlossen in $G$.
		Somit existiert eine offene Umgebung $U$ von $e$ in $G$ mit $U g_1 \cap g_2 H = \emptyset$.
		Da die Multiplikation $m \colon G \times G \to G$ stetig ist, existieren offene Mengen $A,B \subseteq G$ mit $e \in A,B$ und $A \times B \subseteq m^{-1}(U)$.
		Sei $V \coloneqq A \cap B$.
		Dann ist $V$ offen und es gilt $V^2 \subseteq U$ ist offen und $V^2 g_1 \cap g_2 H = \emptyset$.
		Damit folgt $V g_1 \cap V^{-1} g_2 H = \emptyset$, also auch $V g_1 H \cap V^{-1} g_2 H = \emptyset$.
		Insgesamt also $\pi(V g_1 H) \cap \pi(V g_1 H) = \emptyset$.
		\item Sei $\skal*{\cdot }{\cdot }$ ein Skalarprodukt auf $\mathfrak{g}$ und $\mathfrak{p} \coloneqq \mathfrak{h}^\bot$.
		Wir setzen 
		\[
			V_\varepsilon \coloneqq \set[\big]{v \in \mathfrak{p} \given \norm*{v} < \varepsilon} \subseteq \mathfrak{g} \quad \text{und} \quad D_\varepsilon \coloneqq \exp_G(V_\varepsilon) \subseteq G 
		\] 
		Behauptung: Für $\varepsilon>0$ klein genug ist 
		\mapdef{F\colon D_\varepsilon \times H}{G}{(g,h)}{g \cdot h}{}
		eine offene Einbettung. 
		Um dies zu zeigen, können wir annehmen, dass $\exp|_{V_\varepsilon} \colon V_\varepsilon \to D_\varepsilon$ ein Diffeomorphismus ist.
		Da $(\mathd F)_{e,e} \colon \Tmap_e D_\varepsilon \times \Tmap_e H \to \Tmap_e G$, $(v,w) \mapsto v+w$ invertierbar ist, existiert nach dem Umkehrsatz eine offene Umgebung $U_\varepsilon$ von $e$ in $H$, sodass 
		\begin{equation}
			F \colon D_\varepsilon \times U_\varepsilon \to D_\varepsilon H \label{ep:32:1}\tag{*}
		\end{equation}
		für kleine $\varepsilon>0$ ein lokaler Diffeomorphismus nahe $(e,e)$ ist.
		Wegen
		\begin{align}
			F\big|_{D_\varepsilon \times U_\varepsilon h_0} = R_{h_0} \circ F\big|_{D_\varepsilon \times U_\varepsilon} \circ R_{h_0^{-1}}
		\end{align}
		für $h_0 \in H$, ist $F$ ein lokaler Diffeomorphismus.
		Es bleibt zu zeigen, dass $F$ injektiv ist für kleine $\varepsilon$.
		Sei dazu $Z_e$ eine offene Umgebung von $e$ in $G$ mit $Z_e \cap H \subseteq U_\varepsilon$.
		Ferner sei $\tilde{Z}_e$ eine offene Umgebung von $e$ in $G$ mit $\tilde{Z}_e^{-1} \cdot \tilde{Z}_e \subseteq Z_e$.
		Wähle nun $\varepsilon' < \varepsilon$ so klein, dass $D_{\varepsilon'} \subseteq \tilde{Z}_e$.
		Seien nun $g_1, g_2 \in D_{\varepsilon'}$ und $h_1,h_2 \in H$ mit $g_1 h_1 = g_2 h_2$.
		Dann gilt $h \coloneqq g_1^{-1} g_2 = h_1 h_2^{-1} \in H \cap Z_e \subseteq U_\varepsilon$.
		Nach \eqref{ep:32:1} gilt $F(g_1,h) = F(g_2,e)$ und es folgt $g_1 =g_2$, $h=e$ und somit $h_1 =h_2$.
		Dies zeigt die Behauptung.
		\item $\sfrac{G}{H}$ ist eine differenzierbare Mannigfaltigkeit:
		
		Wir betrachten die offenen Mengen $U_g \coloneqq (g D_\varepsilon)H \subseteq G$.
		Dann ist $\sfrac{G}{H} = \bigcup_{g \in G} \sfrac{U_g}{H}$ eine offene Überdeckung von $\sfrac{G}{H}$.
		Die Karten $\varphi_g \colon \sfrac{U_g}{H} \to D_\varepsilon \cong V_\varepsilon \subseteq \mathbb{R}^n$ sind gegeben durch 
		\[
			\begin{tikzcd}
				\varphi_g^{-1} \colon D_\varepsilon = D_\varepsilon \times \set*{e}  \rar["F"] & D_\varepsilon H \rar["L_g"] & g D_\varepsilon H \rar["\pi"] & \sfrac{U_g}{H}
			\end{tikzcd}
		\]
		also $d \mapsto gdH$.
		Das heißt $\varphi_g^{-1} = \pi \circ L_g \circ F|_{D_\varepsilon \times \set*{e}}$ ist bijektiv, stetig und offen.
		Sei nun $\tilde{g} \in G$ mit
		\(
			\sfrac{U_{\tilde{g}}}{H} \cap \sfrac{U_g}{H} \neq \emptyset
		\),
		das heißt $U_{\tilde{g}} \cap U_{g} \neq \emptyset$.
		Dann gilt
		\mapdef{\varphi_g \circ \varphi_g^{-1} \coloneqq \varphi_g \enbrace*{\sfrac{U_{\tilde{g}}}{H} \cap \sfrac{U_g}{H}}}{\varphi_{\tilde{g}} \enbrace*{\sfrac{U_{\tilde{g}}}{H} \cap \sfrac{U_g}{H}}}{d}{\tilde{g}^{-1}g h_d}{}
		für ein $h_d \in H$, da $g d H = \tilde{g} \enbrace*{\tilde{g}^{-1} g d}H$.
		Da $g D_\varepsilon $ und $\tilde{g}D_\varepsilon$ von allen Linksnebenklassen $g' H$ transversal geschnitten werden, hängt $h_d$ differenzierbar von $d$ ab(!).\todo{genauer?}
		Damit sind die Kartenwechsel offenbar differenzierbar.
		\item $\pi$ ist eine differenzierbare Submersion:
		
		$\pi$ ist surjektiv und stetig. Ferner ist $\pi$ differenzierbar mit $(\mathd \pi)_g$ surjektiv für alle $g \in G$, denn folgendes Diagramm kommutiert
		\[
			\begin{tikzcd}
				U_g \dar["F^{-1} \circ L_{g^{-1}}"'] \rar["\pi"] & \sfrac{G}{H} \\
				D_\varepsilon \times H \rar["\pr_1"] & D_\varepsilon \uar["\varphi_g^{-1}"']
			\end{tikzcd}
		\]
		Dies schließt den Beweis ab.\qedhere
	\end{enumerate}
\end{beweis}

\begin{bemerkung*}[{name=[differenzierbare Struktur des Quotienten]}]
	\begin{enumerate}[1),itemsep=1pt]
		\item Die differenzierbare Struktur ist eindeutig bestimmt, wenn man $\pi \colon G \to \sfrac{G}{H}$ differenzierbar fordert.
		\item Man kann $G$ mit einer analytischen Struktur versehen, welche $\sfrac{G}{H}$ erbt.
		Somit ist jeder homogene Raum in natürlicher Weise sogar eine analytische Mannigfaltigkeit (Kartenwechsel sind analytische Diffeomorphismen).
		\item Ist $H$ eine nicht abgeschlossene Untergruppe, so ist $\sfrac{G}{H}$ \emph{nicht} hausdorffsch.
		Man kann aber immer noch lokale Quotienten betrachten.
		\item $\pi \colon G \to \sfrac{G}{H}$ ist ein differenzierbares Faserbündel\index{Faserbündel} mit Fasertyp $H \cong \pi^{-1}(gH) = gH$.
		Die Karten des Faserbündels sind gegeben durch
		\[
			\begin{tikzcd}
				\psi_g^{-1} \colon \sfrac{U_g}{H} \times H \rar["\varphi_g \times {\id}"] & D_\varepsilon \times H \rar["F"] & D_\varepsilon H \rar["L_g"] & g D_\varepsilon H = U_g
			\end{tikzcd}
		\]
	\end{enumerate}
\end{bemerkung*}

Sei nun $M^k$ eine differenzierbare Mannigfaltigkeit und 
\[
	C^\infty(G,M^k)^H \coloneqq \set[\big]{f \in C^\infty(G,M) \given \forall h \in H : f \circ R_h = f}
\]
Offenbar ist $f \in C^\infty(G,M)^H$ konstant auf den Linksnebenklassen $g H$.
Somit ist $\overline{f} \coloneqq f \circ \pi^{-1} \colon \sfrac{G}{H} \to M^k$, $gH \mapsto f(g)$ wohldefiniert.

\begin{lemma}[label=lem:313,{name=[Bijektion von Mengen glatter Abbildungen]}]
	Die Abbildung $\Phi \colon C^\infty(\sfrac{G}{H}, M^k) \to C^\infty(G,M^k)^H$ definiert durch $\overline{f} \mapsto f \coloneqq \overline{f} \circ \pi$ ist bijektiv. 
\end{lemma}
\begin{beweis}
	Die Injektivität ist klar: Seien $\overline{f},\overline{g} \in C^\infty(\sfrac{G}{H},M^k)$ mit $f=\overline{f} \circ \pi = \overline{g} \circ \pi = g$.
	Da $\pi$ surjektiv ist, folgt $\overline{f} = \overline{g}$.
	
	Zur Surjektivität: Sei $f \colon G \to M^k$ glatt mit $f \circ R_h = f$ für alle $h \in H$.
	Wir setzen $\overline{f} \colon \sfrac{G}{H} \to M^k$, $gH \mapsto f(g)$.
	Da $f$ konstant auf Linksnebenklassen ist, ist $\overline{f}$ ist wohldefiniert und $f \circ F$ konstant auf dem $H$-Faktor.
	Somit hängt $f \circ F$ nicht von der $H$-Variablen ab und die Funktion $D_\varepsilon \ni d \mapsto (f \circ L_g \circ F)(d) = \enbrace*{\overline{f} \circ \varphi_g^{-1}}(d)$ ist differenzierbar.
	Damit ist $\overline{f}$ das Urbild zu $f$.
\end{beweis}

\begin{korollar}[{name=[Linksmulitplikation Diffeomorphismus auf Quotienten]}]
	Sei $\sfrac{G}{H}$ ein homogener Raum.
	Dann ist für alle $g \in G$ die Abbildung $\overline{L}_g \colon \sfrac{G}{H} \to \sfrac{G}{H}$, $\overline{g}H \mapsto g \overline{g}H$ ein Diffeomorphismus.
\end{korollar}
\begin{beweis}
	Es gilt: $f \coloneqq \pi \circ L_g \colon G \to \sfrac{G}{H}$ ist differenzierbar mit $f \circ R_h = f$ für alle $h \in H$ und somit ist $\overline{f} \colon \sfrac{G}{H} \to \sfrac{G}{H}$, $\overline{g}H \mapsto f(g) = g \overline{g}H$ differenzierbar mit $\overline{f} = \overline{L}_g$ nach \autoref{lem:313}.
	Wegen $\enbrace{\overline{L}_g}^{-1} = \overline{L}_{g^{-1}}$  folgt die Behauptung.
\end{beweis}

\begin{definition}[{name=[Liegruppenwirkung und $G$-homogener Raum]}]
	Sei $G$ eine Liegruppe und $M$ eine differenzierbare Mannigfaltigkeit.
	Eine \Index{Liegruppenwirkung} von $G$ auf $M$ ist eine differenzierbare Gruppenwirkung $\Phi \colon G \times M \to M$, $(g,p) \mapsto g.p$.
	Insbesondere ist $\psi \colon G\to \operatorname{Diff}(M)$, $g \mapsto  \enbrace*{p \mapsto g.p}$ ein Homomorphismus.
	
	Die Mannigfaltigkeit $M$ nennt man \bet{$G$-homogenen Raum},\index{G-homogener Raum@$G$-homogener Raum} falls $M$ eine transitive Gruppenwirkung $\Phi$ besitzt, das heißt für alle $p,q \in M$ existiert ein $g \in G$ mit $g.p=q$.
\end{definition}

\begin{bemerkung*}[{name=[Vergleich mit homogenen Räumen]}]
	\begin{enumerate}[1)]
		\item Ein homogener Raum $\sfrac{G}{H}$ ist $G$-homogener Raum mittels
		\mapdef{\Phi \colon G \times \sfrac{G}{H}}{\sfrac{G}{H}}{(g,\overline{g}H)}{(g \cdot \overline{g})H}{}
		\item Ist $\Phi \colon G \times M^n \to M^n$ eine Liegruppenwirkung auf $M^n$, so ist $\ker \Phi \coloneqq \set*{g \in G \given \Phi(g) = {\id_{M^n}}}$ eine normale, abgeschlossene Untergruppe von $G$ und somit eine Lieuntergruppe.
		Der Quotient $\sfrac{G}{\ker \Phi}$ ist somit eine Liegruppe, die effektiv auf $M^n$ operiert, das heißt 
		\[
			\Phi(g)= \id_{M} \iff g=e
		\]
	\end{enumerate}
\end{bemerkung*}

\begin{lemma}[label=lem:316,{name=[$G$-äquivarianter Diffeo mit Standgruppe]}]
	Sei $\Phi \colon G \times M^n \to M^n$ eine transitive Liegruppenwirkung, $p \in M^n$ und 
	\[
		H \coloneqq G_p = \set[\big]{g \in G \given g.p=p}
	\]
	Dann ist die Abbildung $\varphi \colon \sfrac{G}{H} \to M^n$, $gH \mapsto g.p$ ein \bet{$G$-äquivarianter}\index{G-äquivariant@$G$-äquivariant} Diffeomorphismus, das heißt folgendes Diagramm kommutiert:
	\[
		\begin{tikzcd}
			\sfrac{G}{H} \dar["\overline{L}_g"] \rar["\varphi"] & M \dar["\Phi(g)"] \\
			\sfrac{G}{H} \rar["\varphi"] & M
		\end{tikzcd}
	\]
\end{lemma}
\begin{beweis}
	Wir müssen viele einzelne Dinge überprüfen:
	\begin{itemize}[itemsep=1pt]
		\item $\varphi$ ist wohldefiniert: $gH = \tilde{g}H \iff \tilde{g}^{-1}g \in H \iff g = \tilde{g}h \iff g.p = \tilde{g}.p$
		\item $\varphi$ ist injektiv: $\varphi(gH) = \varphi(\tilde{g}H) \iff g.p = \tilde{g}.p \iff \tilde{g}^{-1}g \in H \iff gH = \tilde{g}H$
		\item $\varphi$ ist surjektiv: Klar, da $\Phi$ transitiv ist.
		\item Äquivarianz: 
		\(
			(\varphi \circ \overline{L}_g)(\tilde{g}H) = \varphi \enbrace*{g \tilde{g}H} = g.\tilde{g}.p = \Phi(g) \enbrace*{\tilde{g}.p} = \enbrace*{\Phi(g) \circ \varphi}(\tilde{g}H)
		\)
		\item Differenzierbarkeit von $\varphi$ und $\varphi^{-1}$:
		Wir setzen $\hat{\varphi} \colon G \to M^n$, $g \mapsto g.p$.
		Nach Vorraussetzung ist $\hat{\varphi}$ differenzierbar mit 
		\[
			\begin{tikzcd}
				G \drar["\hat{\varphi}"] \dar["\pi"] & \\
				\sfrac{G}{H} \rar["\varphi"] & M^n
			\end{tikzcd}
		\]
		Dabei ist auch $\pi$ differenzierbar.
		Ferner gilt $\hat{\varphi} \circ R_h = \hat{\varphi}$ für alle $h \in H$.
		Somit ist nach \autoref{lem:313} die Abbildung $\varphi = \overline{\hat{\varphi}}$ differenzierbar.
		Es gilt $\ker (\mathd \pi)_e = \Tmap_e H$.
		Übung: Es gilt genauso $\ker (\mathd \hat{\varphi})_e = \ker (\mathd \pi)_e = \Tmap_e H$.
		
		Ferner ist $(\mathd \pi)_e \colon \Tmap_e G \to \Tmap_{eH} \sfrac{G}{H}$ surjektiv.
		Wegen $(\mathd \hat{\varphi})_e = (\mathd \varphi)_{eH} \circ  (\mathd \pi)_e$ ist somit $(\mathd \varphi)_{eH}$ injektiv.
		Aus der Äquivarianzeigenschaft, folgt 
		\[
			(\mathd \varphi)_{gH}\circ (\mathd\overline{L}_g)_H = \enbrace[\big]{\mathd\Phi(g)}_{p=\varphi(H)} \circ (\mathd \varphi)_H
		\]
		Da die Abbildungen direkt neben dem Gleichheitszeichen Isomorphismen sind und die rechte außerdem injektiv, muss auch $(\mathd \varphi)_{gH}$ injektiv sein.
		Somit ist $\varphi \colon \sfrac{G}{H} \to M$ eine injektive Immersion.
		Folglich ist $\dim \sfrac{G}{H} \le \dim M$.
		Mit der Surjektivität folgt direkt die Gleichheit der Dimensionen.\footnote{Angenommen es gelte $k \coloneqq \dim \sfrac{G}{H} < \dim M^n$. Da $\varphi$ eine Immersion ist, ist $\varphi \enbrace*{B_\varepsilon(gH)}$ eine eingebettete $k$-dimensionale Untermannigfaltigkeit von $M^n$ und hat somit Volumen 0.
		Damit hat auch das Bild $\varphi(\sfrac{G}{H})$ Volumen 0 und man hat einen Widerspruch zur Surjektivität}\qedhere
	\end{itemize}
\end{beweis}

\begin{bemerkung*}[{name=[Standgruppen sind konjugiert]}]
	Die Standgruppen sind konjugiert, das heißt es gilt $G_{g.p} = g \cdot G_p \cdot  g^{-1}$.
\end{bemerkung*}

\begin{beispiel*}[{name=[homogene Räume]},label=bsp:homSpaces]
	Es gibt viele homogene Räume:
	\begin{enumerate}[1)]
		\item Liegruppen selbst, da $\sfrac{G}{\set*{e}} =G$.
		\item Sphären $S^n \coloneqq \set*{x \in \mathbb{R}^{n+1} \given \norm*{x}_2=1}$, denn die Abbildung $\hat{\Phi} \colon \On(n+1) \times \mathbb{R}^{n+1} \to \mathbb{R}^{n+1}$, $(A,v) \mapsto A \cdot v$ ist eine differenzierbare Gruppenwirkung auf $\mathbb{R}^{n+1}$ und somit auch die Einschränkung $\varphi \colon \On(n+1) \times S^n \to S^n$.
		
		Diese Wirkung ist transitiv: Sei $v_0 \in S^n$ beliebig.
		Dann können wir $v_0$ zu einer Orthonormalbasis $v_0,v_1, \ldots ,v_n$ von $\mathbb{R}^{n+1}$ ergänzen und setzen $A_{v_0} \coloneqq (v_0,v_1, \ldots , v_n) \in \On(n+1)$.
		Es gilt $A_{v_0}(e_1) = v_0$.
		Die Standgruppe von $e_1$ erhält man durch
		\[
			A e_1 = e_1 \iff A = \begin{pmatrix}
				1 & 0 & \cdots & 0 \\
				0 & & & \\
				\vdots & & \tilde{A} & \\
				0 & & & 
			\end{pmatrix} \text{ mit } \tilde{A} \in \On(n)
		\]
		Nach \autoref{lem:316} ist somit $S^n = \sfrac{\On(n+1)}{\On(n)} = \sfrac{\SO(n+1)}{\SO(n)}$, wobei auch die Einbettung $\SO(n) \hookrightarrow \SO(n+1)$ wie oben erfolgt.
		\item Die unitäre Gruppe $\Un(n) = \SO(2n) \cap \Mat(n,\mathbb{C})$\todo{vlt Aufgabe 11 einbauen} operiert ebenfalls transitiv auf $S^{2n-1}$.
		Somit ist $S^{2n-1} = \sfrac{\Un(n)}{\Un(n-1)} \stackrel{!}{=} \sfrac{\SU(n)}{\SU(n-1)}$.
		\item Wie in 2) und 3) zeigt man, dass
		\(
			S^{4n-1} = \sfrac{\Sp(n)}{\Sp(n-1)}
		\) gilt.
		\item Der reell projektive Raum $\mathbb{R}P^n = \sfrac{S^n}{\set*{\pm \id}}$ ist ebenfalls ein homogener Raum: $\Phi \colon \SO(n+1) \times \mathbb{R}P^n \to \mathbb{R}P^n$, $(A,[v]) \mapsto [Av]$.
		
		Zur Standgruppe: 
		\[
			A [e_1] = [e_1] \iff A= \begin{pmatrix}
				\det \tilde{A} & 0 & \cdots & 0 \\
				0 & & & \\
				\vdots & & \tilde{A} & \\
				0 & & & 
			\end{pmatrix}\in \SO(n+1) , \tilde{A} \in \On(n)
		\]
		Somit ist die Standgruppe von $\mathbb{R}P^n = \sfrac{\SO(n+1)}{O(n)}$ \enquote{größer} als $\SO(n)$.
		\begin{itemize}
			\item Achtung: Nicht alle Unterlagerungen von homogenen Räumen sind homogene Räume!
			\item $\On(n)$ ist nicht zusammenhängend!
		\end{itemize}
		\item Die reellen Grassmannmannigfaltigkeiten:\index{Grassmannmannigfaltigkeit} 
		\begin{align}
			G_k (\mathbb{R}^n) &\coloneqq \set[\big]{\text{$k$-dimensionale reelle Unterräume von $\mathbb{R}^n$}} \\
			G_k^0(\mathbb{R}^n) &\coloneqq \set[\big]{(v,\sigma) \given v \in G_k(\mathbb{R}^n) \text{ und $\sigma$ Orientierung auf $V$}}
		\end{align}
		Die Abbildung $\pi \colon G_k^0(\mathbb{R}^n) \to G_k(\mathbb{R}^n)$, $(v,\sigma) \mapsto v$ ist eine 2-fache Überlagerung.
		Wie in 5) zeigt man, dass $\SO(n)$ transitiv operiert.
		Der Stabilisator von $G_k(\mathbb{R}^n)\ni V_0 = \Span_\mathbb{R}(e_1, \ldots ,e_k)$ berechnet sich wie folgt:
		\[
			\SO(n)_{V_0} = \set*{ \begin{pmatrix}
				A & 0 \\ 0 & B
			\end{pmatrix} \given A \in \On(k), B \in \On(n-k), \det A \cdot \det B = 1} = S \enbrace[\big]{\On(k) \times \On(n-k)}
		\]
		Somit ist $G_k(\mathbb{R}^n) = \sfrac{\SO(n)}{S \enbrace*{\On(k) \times \On(n-k)}} = \sfrac{\On(n)}{\On(k) \times \On(n-k)}$.
		Analog erhält man:
		\[
			G_k^0(\mathbb{R}^n) = \sfrac{\SO(n)}{\SO(k) \times \SO(n-k)}
		\]
		\item Komplexe Grassmannmannigfaltigkeiten: 
		\[
			G_k(\mathbb{C}^n) \coloneqq \set[\big]{\text{$k$-dimensionale kompakte Untermannigfaltigkeiten von $\mathbb{C}^n$} }
		\]
		Wie in 6) zeigt man, dass 
		\[
			G_k(\mathbb{C}^n) = \sfrac{\Un(n)}{\Un(k) \times \Un(n-k)} = \sfrac{\SU(n)}{S(\Un(k) \times \Un(n-k))}
		\]
		Im Fall $k=1$ nennt man $\mathbb{C}P^{n-1} \coloneqq G_1(\mathbb{C}^n)$ den \Index{komplex projektiven Raum}. 
		Es gilt $\mathbb{C}P^1 = S^2$.
		\item Quaternionale Grassmannmannigfaltigkeiten:
		\[
			G_k(\mathbb{H}^n) = \sfrac{\Sp(n)}{\Sp(k) \times \Sp(n-k)}
		\]
		Man nennt $\mathbb{H}P^{n-1} = G_1(\mathbb{H}^n)$ wieder quaternionalen projektiven Raum. 
		Es gilt $\mathbb{H}P^1 = S^4$.
		\item Stiefelmannigfaltigkeiten ($k$-Tupel orthonormaler Vektoren\footnote{siehe \url{https://de.wikipedia.org/wiki/Stiefel-Mannigfaltigkeit}})\index{Stiefelmannigfaltigkeit}
		\[
			V_k(\mathbb{R}^n) = \sfrac{\SO(n)}{\SO(n-k)} \quad V_k(\mathbb{C}^n) = \sfrac{\SU(n)}{\SU(n-k)} \quad V_k(\mathbb{H}^n) = \sfrac{\Sp(n)}{\Sp(n-k)}
		\]
		\item Fahnenmannigfaltigkeiten:\index{Fahnenmannigfaltigkeit}
		\[
			V_{\overline{n}}(\mathbb{R}^n) = \set*{V_{n_1} < \ldots < V_{n_k}}
		\]
		wobei die $V_{n_i}$ $n_i$-dimensionale reelle Unterräume von $\mathbb{R}^n$ sind, $\overline{n} = n_1$ und $ n_1, \ldots ,n_k \le n$.
		Es ist
		\[
			V_{\overline{n}}(\mathbb{R}^n) = \sfrac{\On(n)}{\On(n_1) \On(n_2 - n_1) \cdots \On(n-n_k)}
		\]
		mit $n_1 + n_2 - n_1 + n_3 - n_2 + \ldots + n - n_k =n$ und den obigen Kombinationen über $\ge 1$.
		\item Komplexe Strukturen auf $\mathbb{R}^{2n}$:\index{komplexe Struktur}
		Betrachte den Raum
		\[
			X = \set*{J \in  \On(2n) \given J^2 = - {\id}}
		\]
		Die Gruppe $\On(2n)$ operiert auf $X$ mittels Konjugation $(A,J) \mapsto A J A^T$, denn $(AJA^T)(AJA^T) = A J^2 A^T = - {\id}$.
		Behauptung: Es gilt 
		\[
			X = \set*{A \in \On(2n) \given A^T = -A}
		\]
		Für \enquote{$\subseteq$} sei $J \in X$, also $J^2 = - \id$ und folglich $J J J^T = - J^T$, also $J= -J^T$.
		Sei umgekehrt $A \in \On(2n)$ schiefsymmetrisch.
		Dann gilt
		\[
			A = A A A^T = - A^2 A\implies A^2 = - {\id}
		\]
		Jede schieffsymmetrische Matrix kann mittels Konjugation (durch $\On(2n)$) auf die Form 
		\[
			\begin{pmatrix}
				0 & - \alpha_1 & & & & \\
				\alpha_1 & 0 & & & & \\
				& & (\cdots) & & &  \\
				& & & \ddots & & \\
				& & & & 0 & -\alpha_n \\
				& & & & \alpha_n & 0
			\end{pmatrix}
		\]
		gebracht werden mit $\alpha_i \in \mathbb{R}$.
		Ist $J \in \On(2n)$, so ist $\abs*{\alpha_i}=1 $ für alle $i$.
		Beachte nun
		\[
			\begin{pmatrix}
				0 & 1 \\ 1 & 0
			\end{pmatrix} \cdot \begin{pmatrix}
				0 & -1 \\ 1 & 0
			\end{pmatrix} \cdot \begin{pmatrix}
				0 & 1 \\
				1 & 0
			\end{pmatrix} = \begin{pmatrix}
				0 & 1 \\ -1 & 0
			\end{pmatrix}
		\]
		Somit existiert für alle $J \in X$ ein $A \in \On(2n)$ mit 
		\[
			A J A^T = \begin{pmatrix}
				0 & -1 & & & \\
				1 & 0 & & & \\
				& & \ddots & & \\
				& & & 0 & -1 \\
				& & & 1 & 0
			\end{pmatrix} =: J_0 = i \cdot \diag(1, \ldots ,1) \in \Mat(n,\mathbb{C})
		\]
		Somit operiert $\On(2n)$ transitiv auf $X$.
		Isotropiegruppen von $J_0$:
		\[
			A J_0 A^T = J_0 \iff A J_0 = J_0 A \StackText{Übung}{\iff} A \in \Un(n) \subset \SO(2n)
		\]
		Somit gilt $X = \sfrac{\On(2n)}{\Un(n)}$.
 	\end{enumerate}
\end{beispiel*}

\begin{definition}[{name=[Riemannsche Mannigfaltigkeiten als homogene Räume]}]
	Eine vollständige Riemannsche Mannigfaltigkeit\footnote{$g= (g_p)_{p \in M^n}$ mit $g_p$ Skalarprodukt auf $\Tmap_p M^n$, welches differenzierbar vom Punkt $p$ abhängt.} $(M^n,g)$ nennt man \bet{$G$-homogenen Raum}\index{G-homogener Raum@$G$-homogener Raum!Riemannsche Mannigfaligkeit}, falls eine transitive und isometrische Liegruppenwirkung $\Phi \colon G\times M^n \to M^n$ existiert.
	Man nennt dann die Metrik $g$ \bet{$G$-homogen}\index{G-homogene Metrik@$G$-homogene Metrik} oder \bet{$G$-invariant}\index{G-invariante Metrik@$G$-invariante Metrik}.
\end{definition}

\begin{beispiel*}[{name=[zweidimensionale Untermannigfaltigkeit des dreidimensionalen Raumes]}]
	Sei $M^2 \subset \mathbb{R}^3$ eine Untermannigfaltigkeit. 
	Dann wird $M^2$ durch $g_p \coloneqq \skal*{\cdot }{\cdot }_{\mathrm{std}}^{\mathbb{R}^3}\big|_{\Tmap_p M^2 \times \Tmap_p M^2}$ zu einer Riemannschen Mannigfaltigkeit.
\end{beispiel*}

\begin{erinnerung}[{name=[Riemannsche Metrik]}]
	Sei $c \colon [0,1] \to M^n$ eine differenzierbare Kurve. 
	Wir definieren die \bet{Länge}\index{Länge einer glatten Kurve} von $c$ als 
	\[
		L(c) \coloneqq \int_0^1 \norm*{c'(t)}_{c(t)} \,\mathd t
	\]
	und den Abstand zweier Punkte, die \Index{Riemannsche Metrik}, durch 
	\[
		d_g(p,q) \coloneqq \inf \set[\big]{L(c) \given c \text{ diffb. Kurve von $p$ nach $q$}}
	\]
	Man muss an dieser Stelle natürlich noch zeigen, dass dies tatsächlich eine Metrik auf $M^n$ definiert.
	Mit dieser Metrik wird $(M^n,d_g)$ ein metrischer Raum.
	Die Eigenschaft \enquote{vollständig} bezieht sich auf eben diese Metrik.
\end{erinnerung}

\begin{beispiel*}[{name=[nicht vollständige Riemannsche Mannigfaltigkeit]}]
	Betrachte $\mathbb{R}^2 \setminus \set[\big]{\binom{0}{0}}$ mit der Standardmetrik. 
	Diese Riemannsche Mannigfaltigkeit ist \emph{nicht} vollständig.
\end{beispiel*}

\begin{bemerkung*}[{name=[Einbettungssatz von Nash]}]
	Nach dem Einbettungssatz von \textsc{Nash} kann jede vollständige riemannsche Mannigfaltigkeit $(M^n,g)$ isometrisch in $(\mathbb{R}^{N(n)},g_{\mathrm{std}})$ eingebettet werden.\footnote{siehe \url{https://de.wikipedia.org/wiki/Einbettungssatz_von_Nash}}
\end{bemerkung*}

\begin{satz}[{name={Myers-Steenrod}}]
	Sei $(M^n,g)$ eine vollständige Riemannsche Mannigfaltigkeit.
	Dann ist die Isometriegruppe 
	\[
		\Isom(M^n,g) = \set[\big]{f \colon (M^n,g) \to (M^n,g) \given \text{bijektive Isometrie}}
	\]
	eine Liegruppe, welche differenzierbar auf $M^n$ operiert mittels Auswertung.
\end{satz}
\begin{beweis}
	\emph{Siehe \textcite{MyersSteenrod}.}
\end{beweis}

Es gilt: Isometrien von $(M^n,g)$ sind Diffeomorphismen, welche die Riemannsche Metrik $g$ \enquote{respektieren}, das heißt für alle $p \in M^n$ und $v,w \in \Tmap_p M^n$ gilt
\[
	g_p(v,w) = g_{f(p)} \enbrace*{\mathd f_p \cdot v , \mathd f_p \cdot w}
\]
Wir kommen nun zur \bet{Riemannschen Exponentialabbildung}:\index{Riemannsche Exponentialabbildung} 
Sei $(M^n,g)$ eine vollständige Riemannsche Mannigfaltigkeit, $p \in M^n$.
Dann kann man jedem Tangentialvektor $v \in \Tmap_p M^n$ eine differenzierbare Kurve $\gamma_v \colon \mathbb{R} \to M^n$ zuordnen mit
\begin{enumerate}[1),itemsep=1pt]
	\item $\gamma_v(0)=p$, $\gamma'(0)=v$
	\item Für alle $t_0 \in \mathbb{R}$ existiert $\varepsilon_{t_0} >0$, sodass für alle $t_1,t_2 \in \benbrace*{t_0 - \varepsilon_{t_0}, t_0 + \varepsilon_{t_0}}$ gilt
	\[
		L \enbrace*{\gamma_v\big|_{\benbrace*{t_1,t_2}}} = d_g \enbrace[\big]{\gamma_v(t_1), \gamma_v(t_2)}
	\]
	Das heißt $\gamma_v$ ist die \emph{lokal Kürzeste}, $\gamma_v$ nennt man \Index{Geodätische}(\Index{Geodäte}).
	\item Es gilt $\norm*{\gamma_v'(t)} = \norm*{\gamma_v'(0)}$, das heißt $\gamma_v$ ist ein sog. \emph{reguläre Kurve}.
\end{enumerate}
Die \Index{Riemannsche Exponentialabbildung} $\exp_p = \exp_p^g \colon \Tmap_p M^n \to M^n$ ist -- wie auch die aus der Theorie der Liegruppen -- definiert durch $v \mapsto \gamma_v(1)$.
Es gilt:
\begin{itemize}[itemsep=1pt]
	\item $\exp_p$ ist differenzierbar,
	\item $\exp_p$ ist surjektiv nach Hopf-Rinow und
	\item $\mathd(\exp_p)_{0_p} \colon \Tmap_p M^n \to \Tmap_p M^n$ ist gleich $\id_{\Tmap_p M^n}$ und nach dem Umkehrsatz ein lokaler Diffeomorphismus.
\end{itemize} 

\begin{lemma}[{name=[Kompaktheit der Standgruppe]},label=lem:319]
	Sei $(M^n,g)$ ein effektiver, $G$-invarianter homogener Raum, $p \in M^n$ und 
	\[
		H = G_p = \set[\big]{g \in G \given \Phi(g) .p = p}
	\]
	Dann ist $H$ kompakt, falls $\Phi(G) \subset \Isom(M^n,g)$ abgeschlossen ist.
\end{lemma}
\begin{beweis}
	Sei $(h_i)_{i \in \mathbb{N}}$ eine Folge in $H$, das heißt nach Definition $\Phi(h_i)(p)=p$ für alle $i$.
	Somit ist $\mathd \enbrace*{\Phi(h_i)}_p \colon \Tmap_p M \to \Tmap_p M$ eine lineare Isometrie von $(\Tmap_p M, g_p)$ und somit gilt $\mathd  \enbrace*{\Phi(h_i)}_p \in \On \enbrace*{\Tmap_p M^n,g_p}$.
	Wegen Kompaktheit von $\On \enbrace*{\Tmap_p M^n,g_p}$ können wir also annehmen, dass\marginnote{sonst Übergang zu Teilfolge}
	\[
		\enbrace[\big]{\mathd\Phi(h_i)}_p \grenzw{i \to \infty} A_\infty \in \On \enbrace*{\Tmap_p M^n,g_p}
	\]
	Konstruiere nun eine Isometrie \enquote{$\Phi(h_\infty)$} mit $\Phi(h_\infty)(p)=p$ und $\enbrace*{\mathd\Phi(h_\infty)}_p = A_\infty$.
	Für $q \in M$ existiert ein $v_q \in \Tmap_p M^n$ mit $\exp_p(v_q) =q$.
	Wir definieren
	\[
		\Phi(h_\infty)(q) \coloneqq \exp_p \enbrace[\big]{A_\infty(v_q)}
	\]
	Nun gilt für alle $i \in \mathbb{N}$ 
	\[
		\Phi(h_i)(q) = \Phi(h_i) \enbrace[\big]{\exp_p(v_q)} \stackrel{!}{=} \exp_p \enbrace[\big]{ (\mathd \Phi(h_i))_p (v_q)} \grenzw{i \to \infty} \Phi(h_\infty)(q)
	\]
	Wobei \enquote{!} folgt, da Isometrien Geodätische auf Geodätische abbilden und Geodätische eindeutig bestimmt  sind durch $\gamma_v(0)$ und $\gamma_v'(0)$.
	Weiter gilt
	\begin{align}
		d_g \enbrace[\big]{\Phi(h_\infty)(q_1), \Phi(h_\infty)(q_2)} = d_g \enbrace*{\lim_{i \to \infty} \Phi(h_i)(q_1), \ldots (q_2)} &= \lim_{i \to \infty} d_g \enbrace[\big]{\Phi(h_i)(q_1), \Phi(h_i)(q_2)} \\
		&\equiv d_g(q_1,q_2)
	\end{align}
	Somit ist $\Phi(h_\infty)$ eine Isometrie von $(M^n,g)$ und damit ein Diffeomorphismus von $M^n$.
	Wir wissen, dass $\Phi(G) \subset \Isom(M^n,g)$ abgeschlossen ist.
	Wegen $\Phi(h_i) \in \Phi(G)$ für alle $i \in \mathbb{N}$ folgt $\Phi(h_\infty) \in \Phi(G)$.
	Da $\Phi$ effektiv ist, existiert genau ein $h_\infty \in G$ mit $\Phi(h_\infty) \mathrel{\text{\enquote{$=$}}} \Phi(h_\infty)$.
	Also muss $h_\infty \in H$ gelten.
\end{beweis}

\begin{beispiel*}[{name=[Gültigkeit von \autoref{lem:319}]}]
	Ist $G$ kompakt, so ist $\Phi(G)$ kompakt und der Satz gilt.
	
	Gegenbeispiel: $\enbrace*{S^3 \times S^3, g_{\mathrm{std}} \times g_{\mathrm{std}}}$. Dann operiert $G = S^3 \times S^3 \times \mathbb{R}$ symmetrisch auf $M^6=S^3 \times S^3$ wie folgt:\todo{hier fehlt irgendwie noch was}
	\[
		S^3 = \set[\big]{q \in \mathbb{H} \given \norm*{q}=1} \supset S^1 = \set[\big]{z \in \mathbb{C} \given \abs*{z}=1 }
	\]
	\emph{Übung!}
\end{beispiel*}

\begin{definition}[{name=[Isotropiedarstellung]}]
	Sei $(M^n,g)$ ein $G$-homogener Raum, $p \in M^n$ und $H= G_p$ die Isotropiegruppe eines Punktes $p \in M^n$.
	Dann nennt man
	\mapdef{\rho \colon H}{\On \enbrace*{\Tmap_p M^n, g_p}}{h}{\enbrace*{\mathd \Phi(h)}_p}{}
	% \[
	% 	\rho \colon H \to \On \enbrace*{\Tmap_p M^n, g_p} , h \mapsto \enbrace*{\mathd \Phi_h}_p
	% \]
	die \Index{Isotropiedarstellung} von $M^n$ 
\end{definition}

\begin{beispiel*}[{name=[Isotropiedarstellung]}]
	$M^n = S^n \subset \mathbb{R}^{n+1}$, $G = \SO(n+1)$, $p=e_1$, $H= G_p = \set*{\begin{psmallmatrix}
		1 & 0 \\ 0 & A
	\end{psmallmatrix} \given A \in \SO(n)}$.
	Die Gruppe $G = \SO(n+1)$ operiert linear auf $S^n$ mittels $(A,v) \mapsto A \cdot v = \Phi(A)(v)$.
	Somit auch $H$, das heißt $\mathd \enbrace*{\Phi_h}_p \colon \Tmap_p S^n \to \Tmap_p S^n$, $\binom{0}{x} \mapsto \binom{0}{Ax}$ mit $A=h \in \SO(n)$.
\end{beispiel*}

\begin{lemma}[{name=[Bijektion $G$-inv. Metriken und $\rho(H)$-inv. Skalarprodukte]},label=lem:3111]
	Sei $(M^n,g)$ ein $G$-homogener Raum, $p \in M^n$ und $H$ die Isotropiegruppe von $p$.
	Dann existiert eine Bijektion zwischen
	\begin{itemize}
		\item den $G$-invarianten Riemannschen Metriken auf $M^n$ und
		\item den $\rho(H)$-invarianten Skalarprodukten auf $\Tmap_p M^n$.
	\end{itemize}
\end{lemma}
\begin{beweis}
	Sei $\tilde{g}$ eine $G$-invariante Riemmannsche Metrik auf $M^n$.
	Dann gilt $\Phi(h) \colon M^n \to M^n$ ist eine Isometrie mit $\Phi(h)(p)=p$ für alle $h \in H$.
	Folglich ist $(\mathd \Phi(h))_p \colon \enbrace*{\Tmap_p M^n, g_p} \to \enbrace*{\Tmap_p M^n, g_p}$ eine Isometrie, das heißt für alle $v,w \in \Tmap_p M^n$ gilt
	\[
		g_p(v,w) = g_p \enbrace[\big]{\rho(h)v, \rho(h)w}
	\]
	für alle $ \in H$.
	
	Sei umgekehrt $\skal*{\cdot}{\cdot}$ ein $\rho(H)$-invariantes Skalarprodukt auf $\Tmap_p M^n$.
	Sei $q \in M^n$.
	Da $M^n$ ein $G$-homogener Raum ist, existiert $g \in G$ mit $q = g.p = \Phi(g)(p)$.
	Wir definieren $g_q \colon \Tmap_q M^n \times \Tmap_q M^n \to \mathbb{R}$ durch
	\[
		(x,y) \longmapsto \skal*{(\mathd\Phi(g^{-1}))_q \cdot x}{\enbrace*{\mathd\Phi(g^{-1})}_q \cdot y}
	\]
	Wir müssen Wohldefiniertheit zeigen: Sei $q = g' .p$.
	Somit $g'.p = g.p \iff g^{-1} g' .p =p \iff g^{-1} g \in H \iff g' = g \cdot h$ für ein $h \in H$. 
	Nun gilt:
	\begin{align}
		\skal[\Big]{\mathd \enbrace*{\Phi({g'}^{-1})}_q \cdot x}{\cdots} = \skal[\Big]{\mathd \enbrace*{\Phi(h^{-1}) \Phi(g^{-1})}x }{ \cdots} &= \skal*{ \enbrace[\big]{\mathd \Phi(h)}_p \enbrace[\big]{\mathd \Phi(g^{-1})}_q x}{ \cdots} \\
		&= \skal*{\enbrace*{\mathd \Phi (g^{-1})}_q x}{ \cdots} \tag*{$\rho(H)$-inv.}
	\end{align}
	Somit ist die Definition unabhängig von der Wahl von $g \in G$ mit $q=g.p$.
	In den Übungen wird gezeigt, dass $g=(g_q)_{q \in M^n}$ eine differenzierbare Metrik auf $M^n$ ist.
\end{beweis}

In \autoref{lem:316} hatten wir gezeigt, dass $\varphi \colon \sfrac{G}{H} \to M^n$, $g H \mapsto g .p$ ein $G$-äquivarianter Diffeomorphismus ist.
Weiter kommutiert offenbar das folgende Diagramm
\[
	\begin{tikzcd}
		\sfrac{G}{H} \dar["\overline{L}_g"'] \rar["\varphi"] & M^n \dar["\Phi(g)"] \\
		\sfrac{G}{H} \rar["\varphi"] & M^n
	\end{tikzcd}
\]
Somit können wir die Isotropiedarstellung $\rho \colon H \to \On \enbrace*{\Tmap_p M^n, g_p}$, $h \mapsto \enbrace*{\mathd\Phi(h)}_p$ identifizieren mit $\rho \colon H \to \On \enbrace*{\Tmap_{eH} \sfrac{G}{H},g_p}$, $h \mapsto \mathd (\overline{L}_h)_{eH}$.
Dazu noch ein Lemma:

\begin{lemma}[{name=[kommutierendes Diagramm mit Isotropiegruppe]},label=lem:3112]
	Das folgende Diagramm kommutiert
	\[
		\begin{tikzcd}[sep=large]
			\Tmap_e G \rar["\Ad(h)"] \dar["\mathd \pi_e"'] & \Tmap_e G \dar["\mathd \pi_e"] \\
			\Tmap_{eH} \sfrac{G}{H} \rar["(\mathd \overline{L}_h)_{eH}"] & \Tmap_{eH} \sfrac{G}{H}
		\end{tikzcd}
	\]
\end{lemma}
\begin{beweis}
	Für $h \in H$ bezeichneten wir mit $i_h \colon G \to G$, $g \mapsto h g h^{-1}$ und mit $\Ad(h) \colon \Tmap_e G \to \Tmap_e G$, $v \mapsto \mathd (i_h)_e \cdot v$ die Konjugation bzw. adjungierte Darstellung.
	Nun gilt
	\begin{align}
		\enbrace*{\overline{L}_h \circ \pi} (g) = h \cdot g \cdot H = h \cdot g \cdot h^{-1} \cdot H = i_h(g) \cdot H = (\pi \circ i_h)(g)
	\end{align}
	für alle $g \in G$, also $\overline{L}_h \circ \pi = \pi \circ i_h$.
	Differenzieren in $g=e$ liefert
	\[
		\enbrace*{\mathd \enbrace*{\overline{L}_h}}_{\pi(e)} \cdot (\mathd \pi)_e = (\mathd \pi)_e \cdot \enbrace*{\mathd i_h}_e = (\mathd \pi)_e \cdot \Ad(h) \qedhere
	\]
\end{beweis}

\begin{bemerkung*}[{name=[Differential der Projektion ist nicht injektiv]}]
	Es ist $\ker (\mathd \pi)_e = \Tmap_e H =\mathfrak{h}$.
	Somit ist $(\mathd \pi)_e$ \emph{kein} Isomorphismus, falls $\dim H > 0$.
\end{bemerkung*}

Wir nehmen nun an, dass auf $\Tmap_e G = \mathfrak{g}$ ein $\Ad(H)$-invariantes Skalarprodukt existiert oder äquivalent dazu $\Ad(H) \subset \Ad(G)$ kompakt ist. 
Dies ist klar, falls $H$ kompakt ist, aber es gibt auch Beispiele von $\sfrac{G}{H}$ mit $\Ad(H)$ kompakt und $H$ nicht kompakt!
Da $\Ad(H)$ kompakt ist, existiert eine $\Ad(H)$-invariantes Skalarprodukt auf $\Tmap_e G$ (Beweis analog wie \enquote{früher} in \cref{sec:23}).

Wir setzen nun $\mathfrak{m} \coloneqq \mathfrak{h}^\perp$.
Dann ist $\mathfrak{m}$ ebenfalls $\Ad(H)$-invariant ($\mathfrak{h} = \Tmap_e H$ ist natürlich $\Ad(H)$-invariant). 
Wegen $\mathfrak{h} = \ker (\mathd \pi)_e$ kommutiert folgendes Diagramm nach \autoref{lem:3112}
\[
	\begin{tikzcd}
		\mathfrak{m} \dar["\mathd \pi_e","\cong"'] \rar["\Ad(h)"] & \mathfrak{m} \dar["\mathd \pi_e","\cong"'] \\
		\Tmap_{eH} \sfrac{G}{H} \rar["\mathd \overline{L}_h"] & \Tmap_{eH} \sfrac{G}{H}
	\end{tikzcd}
\]

\begin{korollar}[{name=[Bijektion zw. Metriken auf Quotient und $\Ad(H)$-inv. Skalarprodukten auf Komplement]},label=kor:3113]
	Sei $\sfrac{G}{H}$ ein homogener Raum.
	Dann existiert eine Bijektion zwischen
	\begin{itemize}
		\item den $G$-invarianten Metriken auf $\sfrac{G}{H}$ und
		\item den $\Ad(H)$-invarianten Skalarprodukten auf einem $\Ad(H)$-invarianten Komplement $\mathfrak{m}$ von $\mathfrak{h}$ in $\mathfrak{g}$.
	\end{itemize}
\end{korollar}
\begin{beweis}[Rezept]
	$\sfrac{G}{H}$ gegeben, finde $\mathfrak{g} = \mathfrak{h} \oplus \mathfrak{m}$ mit $\mathfrak{m}$ $\Ad(H)$-invariant.
	$\Ad(H)$ operiert isometrisch bezüglich einem Skalarprodukt auf $\Tmap_e G$.
\end{beweis}

\begin{bemerkung*}[{name=[Eindeutigkeit des Komplements]}]
	Das Komplement $\mathfrak{m}$ ist im Allgemeinen \emph{nicht} eindeutig.
	In den Übungen wird $\sfrac{G}{H}$ ein \enquote{kanonisches} Komplement zugeordnet.\todo{einarbeiten: kanonisches Komplement}
\end{bemerkung*}

\begin{beispiel*}[{name=[Isotropiedarstellung der Sphäre]}]
	Es ist $S^n = \sfrac{\SO(n+1)}{\SO(n)}$ (siehe \cpageref{bsp:homSpaces}). 
	Es gilt weiter
	\[
		\mathfrak{so}(n+1) = \Tmap_e \SO(n+1) = \set*{V \in \Mat(n+1,\mathbb{R}) \given V^T = -V}
	\]
	Wir erhalten 
	\[
		\mathfrak{so}(n+1) = \mathfrak{so}(n) \oplus  \Underbracket{\set*{\begin{pmatrix}
			0 & x_1 & \cdots & x_n \\
			-x_1 & & & \\
			\vdots & & 0 & \\
			-x_n & & & 
		\end{pmatrix} \given x_i \in \mathbb{R}}}{=: \mathfrak{m}}
	\]
	Damit ist $\mathfrak{m}$ tatsächlich $\Ad(H)$-invariant.
	Sei $h \in H = \SO(n)$ und $\begin{psmallmatrix}
		0 & -x^T \\ x & 0
	\end{psmallmatrix} \in \mathfrak{m}$.
	Dann gilt 
	\begin{align}
			\Ad(h) \begin{pmatrix}
			0 & -x^T \\ x & 0
		\end{pmatrix} = \begin{pmatrix}
			1 & 0 \\ 0 & A
		\end{pmatrix} \cdot \begin{pmatrix}
			0 & -x^T \\ x & 0
		\end{pmatrix} \cdot \begin{pmatrix}
			1 & 0 \\
			0 & A^T
		\end{pmatrix} &= \begin{pmatrix}
			0 & - x^T \\ Ax & 0 
		\end{pmatrix} \cdot \begin{pmatrix}
			1 & 0 \\ 0 & A^T
		\end{pmatrix} \\
		&= \begin{pmatrix}
			0 & -(Ax)^T \\ Ax & 0
		\end{pmatrix} 
		\stackrel{!}{\in } \mathfrak{m}
	\end{align}
	Wir identifizieren $\mathfrak{m}$ mit $\mathbb{R}^n$ mittels $x \mapsto \begin{psmallmatrix}
		0 & -x^T \\ x & 0
	\end{psmallmatrix}$.
	Somit ist die Isotropiedarstellung von $S^n = \sfrac{\SO(n+1)}{\SO(n)}$ die Standarddarstellung von $\SO(n)$ auf $\mathbb{R}^n$ gegeben durch $(A,x) \mapsto A \cdot x$. \marginnote{gleich dem Differential, da Wirkung linear}
	Diese Darstellung ist \emph{irreduzibel}, das heißt dass $\mathfrak{m} = \mathfrak{m}_1 \oplus \mathfrak{m}_2$ schon $\mathfrak{m}_1= \set*{0}$ oder $\mathfrak{m}_2 =\set*{0}$ impliziert.
\end{beispiel*}

\begin{definition}[{name=[{isotropie-irreduzibel}]}]
	Einen homogenen Raum $\sfrac{G}{H}$ nennt man \Index{isotropie-irreduzibel}, falls die Isotropiedarstellung irreduzibel ist. 
\end{definition}

\begin{beispiel*}[{name=[Beispiel für nicht irreduzible Isotropiedarstellung]}]
	Betrachte $G= \SO(9)$ und $H = \SO(2) \SO(3) \SO(4)$.
	Dann ist $\mathfrak{g} = \mathfrak{h} \oplus \mathfrak{m}$ mit
	\[
		\mathfrak{m} = \set*{\begin{pmatrix}
			0 & x_1 & x_2 \\
			- x_1^T & 0 & x_3 \\
			-x_2^T & -x_3^T & 0
		\end{pmatrix} \given x_1 \in \Mat(2\times 3), x_2 \in \Mat(2 \times 4), x_3 \in \Mat(3\times 4)} 
	\]
	und $h \in H$ gegeben durch
	\[
		h = \begin{pmatrix}
			A_1 & 0 & 0 \\
			0 & A_2 & 0 \\
			0 & 0 & A_3
		\end{pmatrix}
	\]
	mit $A_i$ aus $\SO(i+1)$.
	Wir erhalten:
	\begin{align}
		\hspace{-1em}\begin{pmatrix}
			A_1 & 0 & 0 \\
			0 & A_2 & 0 \\
			0 & 0 & A_3
		\end{pmatrix}\!\!
		\begin{pmatrix}
			0 & x_1 & x_2 \\
			- x_1^T & 0 & x_3 \\
			-x_2^T & -x_3^T & 0
		\end{pmatrix}\!\!
		\begin{pmatrix}
			A_1^T & 0 & 0 \\
			0 & A_2^T & 0 \\
			0 & 0 & A_3^T
		\end{pmatrix} &=
		\begin{pmatrix}
			0 & A_1 x_1 & A_1 x_2 \\
			- A_2 x_1^T & 0 & A_2 x_3 \\
			- A_3 x_2^T & -A_3 x_3^T & 0
		\end{pmatrix}\!\!
		\begin{pmatrix}
			A_1^T & 0 & 0 \\
			0 & A_2^T & 0 \\
			0 & 0 & A_3^T
		\end{pmatrix}\\
		&= 
		\begin{pmatrix}
			0 & A_1 x_1 A_2^T & A_1 x_2 A_3^T \\
			(A_1 x_1 A_2^T)^T & 0 & A_2 x_3 A_3^T \\
			(A_1 x_2 A_3^T)^T & (A_2 x_3 A_3^T)^T & 0
		\end{pmatrix}
	\end{align}
	Damit ist $\mathfrak{m}$ wieder $\Ad(H)$-invariant.
	Wir schreiben $\mathfrak{m} = \mathfrak{m}_1 \oplus \mathfrak{m}_2 \oplus \mathfrak{m}_3$ mit 
	\[
		\mathfrak{m}_1 = \set*{\begin{pmatrix}
			0 & x_1 & 0\\
			-x_1^T & 0 & 0 \\
			0 & 0 & 0
		\end{pmatrix} \given x_1 \in \Mat(2 \times 3)}
	\]
	und $\mathfrak{m}_2$ und $\mathfrak{m}_3$ entsprechend.
	Man kann sich nun überlegen, dass $\mathfrak{m}_i$ $\Ad(H)$-irreduzibel sind.
\end{beispiel*}
% section 31 (end)

\section{Geometrie homogener Räume} % (fold)
\label{sec:32}

Wir haben nun bereits Riemannsche homogene Räume $(\sfrac{G}{H},g)$ kennengelernt mit $g$ $G$-invariant, das heißt die Diffeomorphismen $\overline{L}_g \colon \sfrac{G}{H} \to \sfrac{G}{H}$, $\tilde{g} H \mapsto g \tilde{g} H$ sind Isometrien von $(\sfrac{G}{H},g)$.
Weiter wissen wir schon dass die $g$-invarianten Metriken auf $\sfrac{G}{H}$ in 1:1-Beziehung stehen zu den $\overline{L}_h$-invarianten Skalarprodukten auf $\Tmap_e \sfrac{G}{H}$ für $h \in H = G_{eH}$ (siehe \autoref{lem:3111}).
Diese wiederum stehen auch 1:1-Beziehung zu den $\Ad(H)$-invarianten Sklarprodukten auf einem $\Ad(H)$-invarianten Komplement $\mathfrak{m}$ von $\mathfrak{h} = \Tmap_eH$ in $\mathfrak{g} = \Tmap_eG$ nach \autoref{kor:3113}.

Achtung: Zeige Existenz von $\mathfrak{m}$, falls $G$-invariante Metrik $g$ auf $\sfrac{G}{H}$ existiert.
Dies funktioniert gut, wenn $H$ kompakt ist, ist etwas schwieriger, wenn $H$ nicht kompakt ist (siehe Übungen).
\[
	\begin{tikzcd}
		\mathfrak{m} \rar["\Ad(h)"] \dar["\mathd \pi_e|_\mathfrak{m}","\cong"'] & \enbrace*{\mathfrak{m},\skal*{\cdot}{\cdot}} \dar["\mathd \pi_e|_\mathfrak{m}","\cong"'] \\
		\Tmap_{eH} \sfrac{G}{H} \rar["\mathd \overline{L}_h"] & \enbrace*{\Tmap_{eH} \sfrac{G}{H}, g_{eH}}
	\end{tikzcd}
\]
Somit ist $\mathd \pi_e|_\mathfrak{m} \colon \enbrace*{\mathfrak{m}, \skal*{\cdot }{\cdot }} \to \enbrace*{\Tmap_{eH} \sfrac{G}{H}, g_{eH}}$ eine Isometrie (Lineare Algebra).

\begin{erinnerung}[{name=[Levi-Civita-Ableitung]}]
	Sei $(M^n,g)$ eine vollständige Riemannsche Mannigfaltigkeit und $X,Y,Z \in \mathfrak{X}(M^n)$ glatte Vektorfelder.
	Die \Index{Levi-Civita-Ableitung} $\nabla_X Y = \nabla^g_X Y \in \mathfrak{X}(M^n)$ für $X,Y \in \mathfrak{X}(M^n)$ ist durch die \Index{Koszul-Formel} definiert (Dies kann man sich als Richtungsableitung von $Y$ in Richtung $X$ vorstellen):
	\[
		2 \cdot \skal*{\nabla_X Y}{Z} \coloneqq X \skal*{Y}{Z} - Z \skal*{X}{Y} + Y \skal*{Z}{X} + \skal*{\benbrace*{X,Y}}{Z} + \skal*{\benbrace*{Z,X}}{Y} - \skal*{\benbrace*{Y,Z}}{X}
	\]
	wobei $X(f)$ für $f \colon M^n \to \mathbb{R}$ glatt die Richtungsableitung von $f$ in Richtung $X$ ist, das heißt $X(f) = \mathd f \cdot X$ ($\benbrace*{X,Y} = \D Y \cdot X - \D X \cdot Y$).
	Die Levi-Cevita-Ableitung hat folgende Eigenschaften:
	\begin{itemize}
		\item $\nabla_{f X} Y = f \cdot \nabla_X Y$ für alle $f \in C^\infty(M^n,\mathbb{R})$ (tensoriell im ersten Argument)
		\item $\nabla_X(f \cdot Y) = X(f) \cdot Y + f \cdot \nabla_X Y $ für alle $f$ (derivativ im zweiten Argument)
		\item $\nabla$ ist metrisch:
		\[
			X \enbrace*{g(Y,Z)} = g \enbrace*{\nabla_X Y,Z} + g \enbrace*{Y,\nabla_X Z}
		\]
		\item $\nabla$ ist torsionsfrei: $\benbrace*{X,Y} = \nabla_X Y - \nabla_Y X$
	\end{itemize}
	Fakt: $\nabla=\nabla^g$ ist durch diese vier Eigenschaften eindeutig bestimmt.
\end{erinnerung}

\begin{beispiel*}[{name=[für Levi-Cevita-Ableitung]}]
	\begin{itemize}
		\item Betrachte $(\mathbb{R}^n,g_\mathrm{flach} = \skal*{\cdot}{\cdot}_\mathrm{std})$.
		Dann ist $\nabla^\mathrm{flach}_X Y = \D Y \cdot X$.
		\item Untermannigfaltigkeit: $M^2 \subset \mathbb{R}^3$ mit $g |_{\Tmap_p M^n \times \Tmap_p M^n} = \skal*{\cdot }{\cdot }_\mathrm{std} |_{\Tmap_p M^n \times \Tmap_p M^n}$.
		Dann ist $(\nabla_X^g Y)_p$ \enquote{$=$} $\pr_{\Tmap_p M^2}(\D \hat{Y} \cdot X) \in \Tmap_pM^2$\marginnote{$\hat{Y}$ Fortsetzung von $Y$ auf $\mathbb{R}^3$; Projektion ist orthogonal}
	\end{itemize}
\end{beispiel*}

\begin{erinnerung}[{name=[Killingvektorfeld]}]
	Ein Vektorfeld $X^*$ auf einer Riemannschen Mannigfaltigkeit $(M^n,g)$ nennt man \Index{Killingfeld}, falls $\nabla_\bullet X^* \colon \Tmap M^n \to \Tmap M^n$, $Y \mapsto \nabla_Y X^*$ \enquote{schiefsymmetrisch} ist, das heißt
	\[
		g \enbrace*{\nabla_Y X^*,Z} = - g \enbrace*{Y,\nabla_Z X^*}
	\]
	für alle $Y,Z$.
	\[
		0 = g \enbrace*{\nabla_{Z_1 + Z_2}X^*,Z_1 + Z_2} = g \enbrace*{\nabla_{Z_1} X^*,Z_2} + g \enbrace*{\nabla_{Z_2} X^*, Z_1} = g \enbrace*{Z_2, (\nabla X^*)^\perp (Z_1)}
	\]
	Dies ist äquivalent dazu, dass der Fluss $\set*{\Phi_t^*}_{t \in \mathbb{R}}$ von $X^*$ aus einer Einparameter-Gruppe von Isometrien von $(M^n,g)$ besteht.
\end{erinnerung}

\begin{beispiel*}[{name=[Killingvektorfelder]}]
	\begin{itemize}
		\item Man betrachte $S^2 \subset \mathbb{R}^3$.
		Dann sind beispielsweise die Vektorfelder, die einen Drehung um die $x$-, $y$- oder $z$-Achse erzeugen, Killingvektorfelder.
		% \missingfigure{Isometrie des $\mathbb{R}^3$, Drehung um $x$-Achse angewandt auf $S^2 \subset \mathbb{R}^3$, genau auch $y$- oder $z$-Achse}
		\item Sei $\enbrace*{\sfrac{G}{H},g}$ $G$-homogener Raum und $X \in \mathfrak{g}$. 
		Definiere damit $X^*(g H) \coloneqq \diffd{}{t}\big|_{t=0} \overline{L}_{\exp(tX)} \cdot gH$.
		Dies ist ein Killingfeld auf $(\sfrac{G}{H},g)$.
		Es gilt 
		\[
			X^*(gH) = \diffd{}{t}\Big|_{t=0} \Underbracket{\overline{L}_{\exp(tX)} \cdot gH}{= \exp(t X) \cdot g \cdot H = \pi \enbrace*{\exp(tX) \cdot g}} = (\mathd \pi)_g \cdot \Underbracket{\diffd{}{t}|_{t=0} \exp(tX) \cdot g }{= X^R(g)}
		\]
		mit $X^R(g) = (\mathd R_g)_e \cdot X$ rechtsinvariant.
	\end{itemize}
\end{beispiel*}

Betrachte nun wieder die Isometrie $(\mathd \pi_e)|_\mathfrak{m} \colon (\mathfrak{m},\skal*{\cdot }{\cdot }) \to \enbrace*{\Tmap_{eH} \sfrac{G}{H},g_{eH}}$ und eine beliebige Basis $Z_1, \ldots ,Z_n$ von $\mathfrak{m}$.
Dann erhalten wir eine Basis $Z_1^*, \ldots , Z_n^*$ für $\Tmap_{gH} \sfrac{G}{H}$ für $gH \in U$ Umgebung von $eH$.
Es folgt, dass man in jedem Punkt $g H$ eine Basis  von $\Tmap_{gH} \sfrac{G}{H}$ existiert, die aus Killingvektorfeldern besteht.

\begin{lemma}[label=lem:321,{name=[Eigenschaft von Killingfeldern]}]
	Sei $\enbrace*{\sfrac{G}{H},g}$ ein homogener Raum und seien $X^*,Y^*$ Killingvektorfelder.
	Dann gilt
	\begin{enumerate}[1)]
		\item $X^* \skal*{Y}{Z} = \skal*{\benbrace*{X^*,Y}}{Z} + \skal*{Y}{\benbrace*{X^*,Z}}$
		\item $2 \skal*{\nabla_{X^*} Y^*}{Z^*} = \skal*{\benbrace*{X^*,Y^*}}{Z^*} - \skal*{\benbrace*{Z^*,X^*}}{Y^*} + \skal*{\benbrace*{Y^*,Z^*}}{X^*}$
	\end{enumerate}
\end{lemma}
\begin{beweis}
	Beide Formeln lassen sich einfach nachrechnen:
	\begin{align}
		X^* \skal*{Y}{Z} \StackText{metr.}{= }\skal*{\nabla_{X^*}Y}{Z} + \skal*{Y}{\nabla_{X^*} Z} &= \skal[\big]{\nabla_Y X^* + \benbrace*{X^*,Y}}{Z } + \skal[\big]{Y}{\nabla_Z X^* + \benbrace*{X^*,Z}} \\
		&\stackrel[\mathclap{= - \nabla X^*}]{\mathclap{(\nabla X^*)^T}}{=}\hspace{1em} \skal[\big]{\benbrace*{X^*,Y}}{Z} + \skal[\big]{\benbrace*{X^*,Z}}{Y}
	\end{align}
	Die zweite Gleichung ergibt sich aus der Koszul-Formel und Formel 1)
	\begin{align}
		2 \skal*{\nabla_{X^*} Y^*}{Z^*} &= X^* \skal*{Y^*}{Z^*} - Z^* \skal*{X^*}{Y^*} + Y^* \skal*{X^*}{Z^*} \\
		&\hphantom{=} + \skal[\big]{\benbrace*{X^*,Y^*}}{Z^*} + \skal[\big]{\benbrace*{Z^*,X^*}}{Y^*} - \skal[\big]{\benbrace*{Y^*,Z^*}}{X^*} \\
		&\StackText{1)}{=} \skal[\big]{\benbrace*{X^*,Y^*}}{Z^*} + \skal[\big]{Y^*}{\benbrace*{X^*,Z^*}} + \skal[\big]{\benbrace*{X^*,Y^*}}{Z^*} \\
		&\hphantom{=} - \skal[\big]{\benbrace*{Z^*,X^*}}{Y^*} - \skal[\big]{X^*}{\benbrace*{Z^*,Y^*}} + \skal[\big]{\benbrace*{Z^*,X^*}}{Y^*} \\
		&\hphantom{=} + \skal[\big]{\benbrace*{Y^*,X^*}}{Z^*} + \skal[\big]{X^*}{\benbrace*{Y^*,Z^*}} - \skal[\big]{\benbrace*{Y^*,Z^*}}{X^*} \\
		&= \skal[\big]{\benbrace*{X^*,Y^*}}{Z^*} - \skal[\big]{\benbrace*{Z^*,X^*}}{Y^*} + \skal[\big]{\benbrace*{Y^*,Z^*}}{X^*} \qedhere
	\end{align}
\end{beweis}

\begin{lemma}[{name=[Verwandtschaft von Killingvektorfeldern]}]
	Sei $\enbrace*{\sfrac{G}{H},g}$ ein homogener Raum und $\Inv \colon G\to G$ die Inversion.
	Dann gilt für einen Tangentialvektor $X \in \mathfrak{g}$ 
	\begin{enumerate}[1)]
		\item $X^R$ ist $\Inv$-verwandt zu $-X^L$
		\item $X^R$ ist $\pi$-verwandt zu $X^*$
		\item $\benbrace*{X^*,Y^*} = - \benbrace*{X,Y}^*$
	\end{enumerate}
\end{lemma}
\begin{beweis}
	\begin{enumerate}[1)]
		\item Wir müssen zeigen, dass $(\mathd \Inv)_g X^R(g) = - X^L(g^{-1})$ gilt.\marginnote{in diesem Beweis steckt Aufgabe 12 drin}
		Dazu definieren wir ein weiteres Vektorfeld $\tilde{X}(g) = - (\mathd \Inv)_{g^{-1}} \enbrace*{X^L(g^{-1})}$ und zeigen, dass $\tilde{X}=X^R$ ist.
		Wir stellen zunächst fest, dass $\tilde{X}(e) = - (\mathd \Inv)_e X^L(e) = X^L(e)=X^R(e)$ gilt, es genügt also zu zeigen, dass $\tilde{X}$ auch rechtsinvariant ist.
		Sei $h \in G$ beliebig, dann gilt
		\begin{align}
			\enbrace*{\mathd R_h}_g \tilde{X}(g) = - \enbrace*{\mathd R_h}_G \enbrace*{\mathd \Inv}_{g^{-1}} X^L(g^{-1}) &= - \enbrace[\big]{\mathd \enbrace*{R_h \circ \Inv}}_{g^{-1}} X^L(g^{-1}) \\
			&= - \enbrace[\big]{\mathd \enbrace*{{\Inv} \circ L_{h^{-1}}}}_{g^{-1}} X^L(g^{-1}) \\
			&= - (\mathd \Inv)_{h^{-1}g^{-1}} X^L(h^{-1} g^{-1}) \\
			&= \tilde{X}(gh) = \tilde{X}\enbrace*{R_h(g)}
		\end{align}
		Also ist auch $\tilde{X}$ rechtsinvariant und es folgt $X^R = \tilde{X}$ und damit auch die gesuchte Verwandtschaft zwischen $X^R$ und $-X^L$.
		\item Betrachte $\pi \colon G \to \sfrac{G}{H}$, $g \mapsto gH$. 
		$\pi$-verwandt heißt per Definition $X^*(\pi(g)) = (\mathd \pi)_g \cdot X^R(g)$. 
		Dies haben wir weiter oben schon bewiesen.
		\item Unter Benutzung der vorigen zwei Punkte können wir dies nun einfach nachrechnen.
		Dabei benutzen wir außerdem, dass die Lieklammern $\benbrace*{X_1,Y_1}$ und $\benbrace*{X_2,Y_2}$ $\Phi$-verwandt sind, wenn $X_1$ und $X_2$ sowie $Y_1$ und $Y_2$ $\Phi$-verwandt sind.
		\begin{align}
			\benbrace*{X^*,Y^*}(gH) = (\mathd \pi)_g \benbrace*{X^R,Y^R}(g) &= (\mathd \pi)_g (\mathd \Inv)_{g^{-1}} \benbrace*{X^L,Y^L}(g^{-1}) \\
			&= (\mathd \pi)_g (\mathd \Inv)_{g^{-1}} (\mathd L_{g^{-1}})_e \benbrace*{X,Y}(e) \\
			&= (\mathd \pi)_g \mathd \enbrace*{R_g \circ \Inv}_e [X,Y](e) \\
			&= - (\mathd \pi)_g (\mathd R_g) \benbrace*{X,Y}(e) \\
			&= - (\mathd \pi)_g \benbrace*{X,Y}^R(g) \\
			&= - \benbrace*{X,Y}^*(gH) \qedhere
		\end{align}
	\end{enumerate}
\end{beweis}

\begin{korollar}[{name=[Killingfelder und die Levi-Cevita-Ableitung]}]
	Sei $\enbrace*{\sfrac{G}{H},g}$ ein homogener Raum und $\mathfrak{g} = \mathfrak{h} \oplus \mathfrak{m}$ eine $\Ad(H)$-invariante Zerlegung von $\mathfrak{g}$.
	Ferner sei $(\mathd \pi)_e|_{\mathfrak{m}}$ eine Isometrie.
	Dann gilt
	\begin{enumerate}[1)]
		\item $\skal*{\nabla_{X^*} Y^*}{Z^*}_{eH} = \skal[\big]{- \sfrac{1}{2} \cdot \benbrace*{X,Y}_{\mathfrak{m}} + u(X,Y)}{Z}$
		\item $\skal*{\nabla_{X^*} Y^*}{\nabla_{X^*} Y^*}_{eH} = \norm[\big]{- \sfrac{1}{2} \cdot \benbrace*{X,Y}_\mathfrak{m} + u(X,Y)}^2$
		\item $\skal*{\nabla_{X^*} X^*}{\nabla_{Y^*} Y^*}_{eH} = \skal[\big]{u(X,X)}{u(Y,Y)} $
	\end{enumerate}
	wobei $2 \cdot \skal*{u(X,Y)}{Z} = \skal*{\benbrace*{Z,X}_\mathfrak{m}}{Y} + \skal*{\benbrace*{Z,Y}_\mathfrak{m}}{X}$ für alle $X,Y,Z \in \mathfrak{m}$.
\end{korollar}
\begin{beweis}
	\begin{enumerate}[1)]
		\item Es gilt
		\begin{align}
			2 \cdot \skal*{\nabla_{X^*} Y^*}{Z^*}_{eH} &\StackTextClap{\ref{lem:321}}{=} \skal[\big]{\benbrace*{X^*,Y^*}}{Z^*} - \skal[\big]{\benbrace*{Z^*,X^*}}{Y^*} + \skal[\big]{\benbrace*{Y^*,Z^*}}{X^*} \\
			&= - \skal[\big]{\benbrace*{X,Y}^* }{Z^*}_{eH} + \skal[\big]{\benbrace*{Z,X}^*}{Y^*}_{eH} - \skal[\big]{\benbrace*{Y,Z}^*}{X^*}_{eH} \\
			&= - \skal[\big]{\benbrace*{X,Y}_\mathfrak{m}}{Z} + \skal[\big]{\benbrace*{Z,X}_\mathfrak{m}}{Y} + \skal[\big]{\benbrace*{Z,Y}_\mathfrak{m}}{X} \\
			&= - \skal[\big]{\benbrace*{X,Y}_\mathfrak{m}}{Z} + 2 \cdot \skal[\big]{u(X,Y)}{Z}
		\end{align}
		\item Klar mit 1).
		\item Sei $(Z_1,\ldots Z_n)$ ONB von $(\mathfrak{m},\skal*{\cdot }{\cdot })$.
		Dann ist $\enbrace[\big]{Z_1^*(eH), \ldots , Z_n^*(eH)}$ wiederum eine ONB von $\enbrace*{\Tmap_{eH}\sfrac{G}{H},g_{eH}}$ und somit gilt
		\begin{align}
			\skal[\big]{\nabla_{X^*}X^*}{\nabla_{Y^*} Y^*}_{eH} &= \sum_{i=1}^{n} \skal[\big]{\nabla_{X^*} X^*}{Z_i^*}_{eH} \cdot \skal[\big]{\nabla_{Y^*} Y^*}{Z_i^*}_{eH} \\
			&\StackText{1)}{=} \sum_{i=1}^{n} \skal[\big]{u(X,X)}{Z_i} \skal[\big]{u(Y,Y)}{Z_i} = \skal[\big]{u(X,X)}{u(Y,Y)} \qedhere
		\end{align}
	\end{enumerate}
\end{beweis}

\begin{erinnerung}[{name=[Krümmungstensor]}]
	Sei $(M^n,g)$ eine Riemannsche Mannigfaltigkeit.
	Der \Index{Krümmungstensor} $R=R^g$ ist definiert durch
	\[
		R_{XY}Z = \nabla_X \nabla_Y Z - \nabla_Y \nabla_X Z - \nabla_{\benbrace*{X,Y}} Z \in \mathfrak{X}(M^n)
	\] 
	für $X,Y,Z \in \mathfrak{X}(M^n)$.
	Fakt: Ist $X(p) = \tilde{X}(p)$, $X(p) = \tilde{Y}(p)$ und $Z(p) = \tilde{Z}(p)$ für ein $p \in M^n$, so gilt
	\[
		(R_{XY}Z)(p) \stackrel{!}{=} \enbrace*{R_{\tilde{X} \tilde{Y}} \tilde{Z}}(p)
	\]
	Dies ist sehr überraschend, denn $(\nabla_X Y)(p)$ hängt sicher \emph{nicht} nur von $Y(p)$ ab.
	Man setzt $R_{XYZW} \coloneqq \skal*{R_{XY} Z}{W} \in \mathbb{R}$. Es gilt
	\[
		R_{XYZW} = - R_{YXZW} = -R_{XYWZ} = R_{ZWXY}
	\]
\end{erinnerung}

\begin{satz}[label=satz:324,{name=[Formel für Krümmungstensor]}]
	Sei $\enbrace*{\sfrac{G}{H},g}$ ein homogener Raum und $\mathfrak{g} = \mathfrak{h} \oplus \mathfrak{m}$ eine $\Ad(H)$-invariante Zerlegung von $\mathfrak{g}$.
	Dann gilt für $X,Y \in \mathfrak{m}$ die folgende Identität:
	\begin{align}
		&\enbrace*{R_{X^*Y^*Y^*X^*}}(eH) \\
		&\hspace{3em} = -\frac{3}{4} \norm[\big]{\benbrace*{X,Y}_{\mathfrak{m}}}^2 - \frac{1}{2} \skal*{\benbrace[\big]{X, \benbrace*{X,Y}}_{\mathfrak{m}}}{Y} - \frac{1}{2} \skal*{\benbrace[\big]{Y, \benbrace*{Y,X}}_{\mathfrak{m}}}{X} + \norm[\big]{u(X,Y)}^2 - \skal[\big]{u(X,X)}{u(Y,Y)}
	\end{align}
	wobei $g_{eH}$ auf $\Tmap_{eH} \sfrac{G}{H}$ mit $\skal*{\cdot}{\cdot }$ auf $\mathfrak{m}$ identifiziert wird.
\end{satz}
\begin{beweis}
	Es gilt\todo{Rechnung detailliert durchgehen \ldots}
	\begin{align}
		R_{X^*Y^*Y^*X^*} &= - R_{X^*Y^*X^*Y^*} \\
		&= - \skal*{\nabla_{X^*} \nabla_{Y^*} X^* - \nabla_{Y^*} \nabla_{X^*} X^* - \nabla_{\benbrace*{X^*,Y^*}}X^*}{Y^*} \\
		&= - X^* \skal*{\nabla_{Y^*} X^*}{Y^*} + \skal*{\nabla_{Y^*} X^*}{\nabla_{X^*} Y^*} + Y^* \skal*{\nabla_{X^*} X^*}{Y^*} \\
		&\hphantom{=}\,\, - \skal*{\nabla_{X^*} X^*}{\nabla_{Y^*} Y^*} - \skal*{\benbrace*{X^*,Y^*}}{\nabla_{Y^*} X^*} \\
		&= - X^* \enbrace[\big]{\skal*{\benbrace*{Y^*,X^*}}{Y^*} - 0+ \skal*{\benbrace*{X^*,Y^*}}{Y^*}} \cdot \frac{1}{2}\\
		&\hspace{1.3em} {+} Y^* \enbrace[\big]{0- \skal*{\benbrace*{Y^*,X^*}}{X^*} + \skal*{\benbrace*{X^*,Y^*}}{X^*}}  \cdot \frac{1}{2} -\skal*{\nabla_{X^*} X^*}{\nabla_{Y^*} Y^*} + \norm*{\nabla_{Y^*} X^*}^2 \\
		&= \norm*{\nabla_{Y^*} X^*}^2 + Y^* \skal*{ \benbrace*{X^*,Y^*}}{X^*} - \skal*{\nabla_{X^*} Y^*}{\nabla_{Y^*} Y^*} \\
		&= \frac{1}{4} \norm[\big]{\benbrace*{X,Y}_{\mathfrak{m}}}^2 - \skal*{\benbrace*{Y,X}_{\mathfrak{m}}}{u(x,y)} + \norm*{u(X,Y)}^2 \\
		&\hspace{1em} + \skal*{\benbrace*{Y^*, \benbrace*{X^*,Y^*}}}{X^*} + \Underbracket{\skal*{\benbrace*{X^*,Y^*}}{\benbrace*{Y^*,X^*}}}{=\norm*{\benbrace*{X,Y}_\mathfrak{m}}^2} - \skal*{u(X,X)}{u(Y,Y)} \\
		&= - \frac{3}{4} \norm[\big]{\benbrace*{X,Y}_\mathfrak{m}}^2  + \norm[\big]{u(X,Y)}^2 - \skal*{u(X,X)}{u(Y,Y)} - \skal*{\benbrace*{Y,X}_\mathfrak{m}}{u(X,Y)} \\
		&\hspace{1em} + \Underbracket{\skal*{\benbrace*{Y^*, \benbrace*{X^*,Y^*}}}{X^*}}{= \skal*{\benbrace*{Y, \benbrace*{X,Y}}_\mathfrak{m}}{X}} 
	\end{align}
	Ferner gilt
	\begin{align}
		\skal[\big]{\benbrace*{X,Y}_\mathfrak{m}}{u(X,Y)} &= \enbrace[\Big]{\skal*{\benbrace[\big]{\benbrace*{X,Y}_\mathfrak{m},X}}{Y} + \skal*{\benbrace*{\benbrace*{X,Y}_\mathfrak{m},Y}}{X}} \cdot \frac{1}{2}  \\
		&= \frac{1}{2} \enbrace[\Big]{\skal*{\benbrace[\big]{\benbrace*{X,Y},X}}{Y} - \skal*{\benbrace[\big]{\benbrace*{X,Y}_\mathfrak{h},X}}{Y} + \skal*{\benbrace[\big]{\benbrace*{X,Y},Y}}{X} + \skal*{\benbrace[\big]{\benbrace*{X,Y}_\mathfrak{h},X}}{Y}}   \\
		&= \frac{1}{2} \enbrace[\Big]{\skal*{\benbrace[\big]{\benbrace*{X,Y},X}_\mathfrak{m}}{Y} + \skal*{\benbrace[\big]{\benbrace*{X,Y},Y}_\mathfrak{m}}{X}} \\
		&= -\frac{1}{2}  \skal*{\benbrace[\big]{X,\benbrace*{X,Y}_\mathfrak{m}}}{Y} + \frac{1}{2} \skal*{\benbrace[\big]{Y,\benbrace*{Y,X}}_\mathfrak{m}}{X} 
	\end{align}
	Die Behauptung folgt.
\end{beweis}

\begin{korollar}[{name=[Krümmungstensor einer kompakten Liegruppe]}]
	Sei $G$ eine kompakte Liegruppe, versehen mit einer biinvarianten Metrik $g_{e_i}$.
	Dann gilt
	\[
		R_{XYYX} = \frac{1}{4} \norm[\big]{\benbrace*{X,Y}}^2 
	\]
\end{korollar}
\begin{beweis}
	Da $G$ kompakt ist, besitzt $\mathfrak{g} = \Tmap_e G$ eine $\Ad(G)$-invariantes Skalarprodukt $\skal*{\cdot }{\cdot }$ nach \autoref{lem:232}.
	Wir bezeichnen mit $g_{b_i}$ die entsprechende linksinvariante Riemannsche Metrik auf $G$.
	Dann sind $L_g,R_g$ beides(!) Isometrien.
	Da $\skal*{\cdot}{\cdot }$ $\Ad(G)$-invariant ist, ist $\ad_X \colon \mathfrak{g} \to \mathfrak{g}$ schiefsymmetrisch bezüglich $\skal*{\cdot}{\cdot}$.
	Es folgt 
	\[
		2 \cdot \skal[\big]{u(X,Y)}{Z} = \skal*{ \benbrace*{Z,X}}{Y} + \skal*{\benbrace*{Z,Y}}{X} = -\skal*{X}{\benbrace*{Z,Y}} + \skal*{\benbrace*{Z,Y}}{X} =0
	\]
	Ferner gilt 
	\[
		\skal*{\benbrace[\big]{X,\benbrace*{X,Y}}}{Y} = - \skal[\big]{\benbrace*{X,Y}}{\benbrace*{X,Y}} = - \norm[\big]{\benbrace*{X,Y}}^2
	\]
	Die Behauptung folgt mit der Formel aus \autoref{satz:324}.
\end{beweis}

\begin{erinnerung}[{name=[Schnittkrümmung und Ricci-Krümmung]}]
	$(M^n,g)$ Riemannsche Mannigfaltigkeit, $p \in M^n$, $X_p,Y_p \in \Tmap_p M^n$ mit $\norm*{X_p} = \norm*{Y_p} = 1$ sowie $X_p \perp Y_p$.
	Dann nennt man $K(\Sigma_\alpha) \coloneqq R_{XYYX}$ die \Index{Schnittkrümmung der Ebene $\Sigma_\alpha$}, welche von $X,Y$ aufgespannt wird (dies ist wohldefiniert!).
	Mittelung über die Schnittkrümmung gibt die \Index{Ricci-Krümmung}
	\[
		\ric_p(X,Y) \coloneqq \sum_{i=1}^{n} R_{Xe_i e_i Y}
	\]
	wobei $(e_1, \ldots ,e_n)$ eine ONB von $\enbrace*{\Tmap_p M^n,g_p}$ ist.
	Dann gilt\todo{das ergibt hier noch keinen Sinn \ldots}
	\[
		\ric_p(e_1,e_2) = 0 + R_{1221} + R_{1331} + \ldots + R_{1nn1} = 0 + K(1,2) + K(1,3) + \ldots + K(1,n)
	\]
	Analog wie die Riemannsche Metrik $g$ selbst ist $\ric_g$ ebenfalls eine symmetrische Bilinearform auf $\Tmap M^n$.
\end{erinnerung}

\begin{lemma}
	Sei $\enbrace*{\sfrac{G}{H},g}$ ein Riemannscher homogener Raum mit $\Ad(H)$-invarianter Zerlegung $\mathfrak{g} = \mathfrak{h} \oplus \mathfrak{m}$ der Liealgebra.
	Sei $(X_1,\ldots X_n)$ eine ONB von $\enbrace*{\mathfrak{m},\skal*{\cdot }{\cdot}}$ und $Z \coloneqq \sum_{i=1}^{n} u(X_i,X_i)$.
	Dann gilt für alle $X \in \mathfrak{m}$:
	\[
		\operatorname{Spur} (\ad_X) = \skal*{Z}{X}
	\]
	Insbesondere ist $Z$ wohldefiniert.
\end{lemma}
\begin{beweis}
	Es gilt $\benbrace*{\mathfrak{h},\mathfrak{m}} \subset \mathfrak{m}$, da $\mathfrak{m}$ $\Ad(H)$-invariant ist und $\ad = (\mathd \Ad)_e$ ist.
	Somit gilt für $X \in \mathfrak{m}$\todo{RevChap 3.2}
	\[
		\ad_X = \begin{pmatrix}
			0 & A \\ B & C
		\end{pmatrix}
	\]
	und $\operatorname{Spur} (\ad(X)) = \operatorname{Spur}(\ad(X))|_\mathfrak{m}$.
	Wir erhalten 
	\begin{align}
		\skal*{Z}{X} = \skal*{\sum u(X_i,X_i)}{X} = \sum \skal[\big]{\benbrace*{X,X_i}|_\mathfrak{m}}{X_i} = \sum \skal[\big]{\ad_X(X_i)}{X_i} = \operatorname{Spur}(\ad_X|_\mathfrak{m})
	\end{align}
	Die Behauptung folgt.
\end{beweis}

Man nennt eine Liegruppe $G$ \Index{unimodular}, falls $Z=0$ ist.
Im Allgemeinen ist  $\Phi \colon \mathfrak{g} \to \mathbb{R}$, $X \mapsto \operatorname{Spur} (\ad(X))$  eine $\Ad(G)$-invariante Linearform, denn
\begin{align}
	\ad \enbrace*{\Ad(g)(X)}(Y) = \benbrace*{\Ad(g)(X),\Ad(g)(\Ad(g^{-1})(Y))} &= \Ad(g) \benbrace*{X, \Ad(g^{-1})(Y)} \\
	&= \enbrace*{\Ad(g) \circ \ad(X) \circ \Ad(g^{-1})}(Y)
\end{align}
Somit 
\[
	\Phi \enbrace*{\Ad(g) X} = \operatorname{Spur} \enbrace*{\ad \enbrace*{\Ad(g) X}} = \operatorname{Spur} \enbrace*{\Ad(g) \circ \ad(e) \circ \Ad(g)^{-1}} = \operatorname{Spur} \ad(X) = \Phi(X)
\]
Somit ist $\mathfrak{g}_\mathrm{uni} \coloneq \ker \Phi$ $\Ad(G)$-invariantes Ideal von $\mathfrak{g}$ mit Kodimension 1.
Die entsprechende Lieuntergruppe $G_\mathrm{uni} \subsetneq G$ ist also unimodular (und Normalteiler von $G$).

\begin{satz}
	Sei $\enbrace*{\sfrac{G}{H},g}$ ein Riemannscher homogener Raum und $\mathfrak{g} = \mathfrak{h} \oplus \mathfrak{m}$ eine $\Ad(H)$-invariante Zerlegung von $\mathfrak{g}$.
	Dann gilt für $X \in \mathfrak{m}$:
	\[
		\operatorname{ric}_g(X,X) = - \frac{1}{2} \mathcal{B}_\mathfrak{g}(X,X) + \frac{1}{4} \sum_{i,j=1}^n \skal*{\benbrace*{X_i,X_i}_\mathfrak{m}}{X}^2 - \frac{1}{2} \sum_{i=1}^{n} \norm*{\benbrace*{X,X_i}_\mathfrak{m}}^2 - \skal*{\ad(Z)(X)|_\mathfrak{m}}{X}
	\]
\end{satz}
\begin{beweis}
	Nach \autoref{satz:324} gilt 
	\begin{align}
		\ric(X,X) = \sum_{i=1}^{n} R_{XX_iY_iY} &= - \frac{3}{4} \norm*{\benbrace*{X,X_i}|_\mathfrak{m}}^2 - \frac{1}{2} \sum_{i} \skal*{\benbrace*{X,\benbrace*{X,X_i}}}{X} - \frac{1}{2} \sum_i \skal*{\benbrace*{X_i,\benbrace*{X_i,X}}}{X} \\
		& \quad + \sum_i \norm*{u(X_i,X_i)}^2 - \skal*{u(X,X)}{Z}
	\end{align}
	Erinnerung: Für $X \in \mathfrak{m}$ gilt $\ad(X) = \begin{psmallmatrix}
		0 & A \\ B & C
	\end{psmallmatrix}$.
	Somit
	\begin{align}
		\mathcal{B}_\mathfrak{g}(X,X) = \operatorname{Spur} \enbrace*{\ad(X) \circ \ad(X)} = \sum_{i=1}^{n} \skal*{\benbrace*{X,\benbrace*{X,X_i}}}{X_i} + \sum_{j=1}^{k} \skal*{\benbrace*{X, \benbrace*{X,k_j}}}{k_j}
	\end{align}
	für eine ONB $(Z_1, \ldots ,Z_k)$ von $\mathfrak{h}$, versehen mit einem $\Ad(H)$-invarianter Skalarprodukt $\skal*{\cdot }{\cdot }_\mathfrak{h}$ (Existenz wie in der Übung).
	Wegen 
	\begin{align}
		\sum_{j=1}^{k} \skal*{\enbrace*{\ad(X) \circ \ad(X)}(k_j)}{k_j}_\mathfrak{h} = \operatorname{Spur}(A \cdot B) = \operatorname{Spur}(B \cdot A) &= \sum_i \skal*{\benbrace*{X, \benbrace*{X,X_i}_\mathfrak{h}}}{X_i} \\
		&\StackText{$\ad(k_i)$ schiefsym.}{= }\sum_i \skal*{\benbrace*{X_i , \benbrace*{X_i,X}_\mathfrak{h}}}{X}
	\end{align}
	Somit 
	\begin{align}
		\ric_g(X,X) &= -\frac{3}{4}  \sum \norm*{\benbrace*{X_i,X_i}}^2 - \frac{1}{2} \mathcal{B}_\mathfrak{g}(X,X) - \frac{1}{2} \sum \skal*{\benbrace*{X_i , \benbrace*{X_i,X}_\mathfrak{m}}}{X} \\
		&\hspace{1em}+ \sum \norm*{u(X_i,X_i)}^2 - \skal*{\benbrace*{Z,X}_\mathfrak{m}}{X}
	\end{align}
	Ferner
	\begin{align}
		\sum_i \norm*{u(X,X_i)}^2 = \sum_{i,j} \skal*{u(X,X_i)}{X_i}^2 &= \frac{1}{4} \sum_{i,j} \enbrace*{\skal*{\benbrace*{X_i,X}}{X_i} + \skal*{\benbrace*{X_i,X_i}}{X}}^2 \\
		& \quad + \frac{1}{4} \sum_j \norm*{\benbrace*{X_j,X}_\mathfrak{m}}^2  + \frac{1}{2} \sum_j \skal*{\benbrace*{X_j, \sum_i X_i \skal*{\benbrace*{X_j,X}}{X_i}}}{X} \\
		& \quad + \frac{1}{4}  \sum_{i,j} \skal*{\benbrace*{X_i,X_j}}{X}^2 \qedhere 
	\end{align}
\end{beweis}
\todo[inline]{hier fehlt mal wieder eine Vorlesung}

\begin{satz}[label=satz:3210]
	Sei $\enbrace*{\sfrac{G}{H},g}$ ein homogener Raum mit $\Ad(H)$-invarianter Zerlegung $\mathfrak{g} = \mathfrak{h} \oplus \mathfrak{m}$.
	Dann gilt
	\[
		M_g = \set*{\exp(V) \given V \in \Sym(\mathfrak{m}), \Ad(H)|_\mathfrak{m} V = V \Ad(H)|_{\mathfrak{m}}}
	\]
\end{satz}
\begin{beweis}
	Sei $P \in M_G$.
	Dann gilt $\Ad(H)P = = \Ad(H)$.
	Wir schreiben nun $P = A \cdot \exp(D) \cdot A^T$, wobei $A \in \SO(n)$ und
	\[
		D = \begin{pmatrix}
			d_1 & & \\
			& \ddots & \\
			& & d_n
		\end{pmatrix} \Big|_{\mathfrak{m}}
	\]
	mit $d_i \in \mathbb{R}$ und $n = \dim_\mathbb{R} \mathfrak{m}$.
	Somit gilt $\Ad(H) A \exp(D) A^T = A \exp(D)A^T \Ad(H)$, woraus
	\[
		\enbrace*{A^T \Ad(H)|_\mathfrak{m} A} \exp(D) = \exp(D) \enbrace*{A^T \Ad(H)|_\mathfrak{m} A}
	\]
	folgt. Wir schreiben 
	\[
		D = \begin{pmatrix}
			d_1 I_{i_1} & & \\
			& \ddots & \\
			& & d_k I_{i_k}
		\end{pmatrix}
	\]
	mit $d_i \neq d_j$ für $i \neq j$.
	Mit einem kleinen Argument folgt
	\[
		\benbrace*{A^T \Ad(H)|_\mathfrak{m} A} = \begin{pmatrix}
			* & 0 & \cdots & 0 \\
			0 & \ddots & & \vdots \\
			\vdots & & \ddots & 0 \\
			0 & & *
		\end{pmatrix}
	\]
	indiziert über $i_1, \ldots , i_k$.
	Es folgt $\enbrace*{A^T \Ad(H)|_\mathfrak{m} A} D = D \enbrace*{A^T \Ad(H)|_\mathfrak{m} A}$ und somit
	\[
		\Ad(H)|_\mathfrak{m} A D A^T = A DA^T \Ad(H)|_\mathfrak{m}
	\]
	Also $\Ad(H)|_\mathfrak{m} \exp(t ADA^T) = \exp(t A D A^T) \Ad(H)|_{\mathfrak{m}}$.
	Folglich ist $\exp(t A DA^T) \in  M_G$.
	Somit ist $P = \exp(A D A^T)$ und $V= A DA^T$ ist $\Ad(H)|_\mathfrak{m}$-äquivariant.
\end{beweis}

\begin{korollar}[label=korr:3211]
	Sei $\enbrace*{\sfrac{G}{H}, g}$ ein Riemannscher homogener Raum und 
	\[
		M_G^1 = \set*{\exp(V) \given V \in \Sym(\mathfrak{m}), \operatorname{Spur} V =0, \Ad(H)|_\mathfrak{m} V = V \Ad(H)|_\mathfrak{m}}
	\]
	Das $\Ad(H)$-Modul $\mathfrak{m}$ kann in irreduzible $\Ad(H)$-Module zerlegt werden:
	\[
		\mathfrak{m} = \mathfrak{m}_1 \oplus \ldots \oplus \mathfrak{m}_r
	\]
	Diese Zerlegung ist im Allgemeinen \emph{nicht} eindeutig.
\end{korollar}

Man nennt $\Ad(H)$-Module $\mathfrak{m}_i, \mathfrak{m}_j$ äquivalent, falls es Isomorphismus $J \colon \mathfrak{m}_i \to \mathfrak{m}_j$ existiert mit 
\[
	\begin{tikzcd}
		\mathfrak{m}_i\rar["J"] \dar["\Ad(H)"] & \mathfrak{m}_j \dar["\Ad(H)"] \\
		\mathfrak{m}_i \rar["J"] & \mathfrak{m}_j
 	\end{tikzcd}
\]
Wir setzen $P_i \coloneqq \bigoplus_{j \in I_i} \mathfrak{m}_j$, wobei für $j, \overline{j}$ die Module $\mathfrak{m}_j, \mathfrak{m}_{\overline{j}}$ äquivalent sind. 
Es folgt $\mathfrak{m} = P_1 \oplus \ldots \oplus P_S$.
Die Zerlegung in die \Index{isotypischen} Summanden $P_1, \ldots , P_s$ ist eindeutig(!).
Im Allgemeinen sind die $P_i$ aber nicht mehr irreduzibel (Beispiel $\sfrac{G}{H} = \sfrac{G}{\set*{e}}$, $H = \set*{e}$).
Irreduzible Unterräume sind alle $1$-dimensional $\mathfrak{m} = \mathfrak{g} = P_0$.
Ein irreduzibler Modul $V$ einer linearen Darstellung einer kompakten Liegruppe $K$ (bei uns $K = \Ad(H)$ operiert auf dem Modul $\mathfrak{m}$) ist vom \bet{reellen}, \bet{komplexen}, \bet{quaternionalen} Typ, falls 
\[
	\Hom_K (V,V) = \begin{cases}
		\mathbb{R} &\\
		\mathbb{C} & \\
		\mathbb{H} &
	\end{cases}
\]
(siehe \cite{BrotD}).
\begin{beispiel*}
	\begin{itemize}
		\item $n \ge 3$, $V= \mathbb{R}^n$, $K= \SO(n)$, $(A,v) \mapsto A \cdot v$.
		Annahme: $J \colon \mathbb{R}^n \to \mathbb{R}^n$ existiert mit $J \cdot A = A \cdot J$ für alle $A \in \SO(n)$.
		Folglich ist $J = \lambda \cdot {\id}$. Also ist $\Hom_{\SO(n)} \enbrace*{\mathbb{R}^n,\mathbb{R}^n} \cong \mathbb{R}$.
		\item $n=2$, $V=\mathbb{R}^2 =\mathbb{C} $, $K=\SO(2)$.
		\[
			J_1 = {\id} \qquad J_2 = \begin{pmatrix}
				0 & -1 \\
				1 & 0
			\end{pmatrix}
		\]
		Wegen $J_2 \in  \SO(2) + \SO(2)$ abelsch gilt $J_2 \SO(2) = \SO(2) J_2$. Es folgt(!)
		\[
			\Hom_{\SO(2)} \enbrace*{\mathbb{R}^2,\mathbb{R}^2} \cong \mathbb{C}
		\]
	\end{itemize}
\end{beispiel*}

Es gilt $\mathfrak{m} = P_1 \oplus  \ldots \oplus P_s$. Schurs Lemma besagt nun, dass jedes $\Ad(H)$-invariante Skalarprodukt $\skal*{\cdot }{\cdot }$ diese Zerlegung respektiert, das heißt für $i\neq j$ gilt $\skal*{P_i}{P_j}=0$.

\begin{bemerkung*}
	Falsch für die $\mathfrak{m}_i$s!!!
	Somit hat jeder $\Ad(H)$-äquivarianter Endomorphismus von $\mathfrak{m}$ \enquote{Blockgestalt} bezüglich der Zerlegung in die $P_i$, das heißt $V(P_i) \subset P_i$ für alel $i=1,\ldots ,s$.
	Falls $P_i$ $\Ad(H)$-irreduzibel ist, gilt nach Schurs Lemma für jeden $\Ad(H)|_{P_i}$-äquivarianten Isomorphismus $Q_i \colon P_i \to P_i$ die Identität $Q_i = \alpha \cdot {\id_{P_i}}$ mit $\alpha \in \mathbb{R}\setminus \set*{0}$.
\end{bemerkung*}

Schurs Lemma: Sei $K$ eine kompakte Liegruppe und $\rho \colon K \to \GL(V)$ eine irreduzible Darstellung von $K$ (auf Vektorraum $V$).
Dann existiert bis auf Skalierung genau ein $\rho(K)$-invariantes Skalarprodukt auf $V$.
\begin{beweis}
	Existenz: $K$ ist kompakt also erhalten wir $\skal*{\cdot }{\cdot }_0$ $\rho(K)$-invariant.
	
	Sei $\skal*{\cdot }{\cdot }$ ein weiteres $\rho(K)$-invariantes Skalarprodukt auf $V$.
	Wir setzen 
	\[
		\skal*{v_1}{v_2} = \skal*{p \cdot v_1}{v_2}_0
	\]
	mit $P \colon V\to V$ positiv definit und $\rho(K)$-äquivariant.
	Wir zeigen $P = \alpha \cdot {\id_V}$.
	Falls dem nicht so ist, zerlegen wir $V = E_1 \oplus  \ldots E_\ell$ in Eigenräume von $P$ zu Eigenwerten $\alpha_1 < \lambda_2 < \ldots < \lambda_\ell$, $\ell \ge 2$.
	$P$ ist $\rho(K)$-invariant, also $P \rho(K) = \rho(K) P$, womit für $v_1 \in  E_1$ gilt
	\[
		P \cdot \enbrace*{\rho(K) v_1} = \rho(K) \cdot P \cdot v_1 = \rho(K) \cdot \lambda_1 \cdot v_1 = \lambda_1 \enbrace*{\rho(K) v_1}
	\]
	Somit ist $\rho(K)(v_1) \in E_1$, das heißt $E_1$ ist $\rho(K)$-invariant und damit auch $E_1^\perp$, folglich kann $V =E_1 \oplus E_1^\perp$ nicht irreduzibel sein.
\end{beweis}

\begin{korollar}
	Jedes $\Ad(H)$-invariante Skalarprodukt auf $\mathfrak{m} = P_1 \oplus \ldots \oplus P_s$ erfüllt $\skal*{P_i}{P_j}=0$ für $i\neq j$.
\end{korollar}

Sei nun $P_1$ ein isotypischer Summand vom reellen Typ.
Dann kann jeder $\Ad(H)$-invarianten Isomorphismus $P \colon P_1 \to P_1$ geschrieben werden als
\[
	P= \begin{pmatrix}
		\alpha_1 I_k & \cdots &  \alpha_{1,\abs*{i_1} }I_k \\
		\vdots & \ddots & \vdots \\
		\alpha_{1,\abs*{i_1}} I_k & \cdots & \alpha_{\abs{i_1}} I_k
	\end{pmatrix}
\]
wobei $P_1 = \mathfrak{m}_1 \oplus  \ldots \oplus \mathfrak{m}_{i_1}$ (Summanden äquivalent) mit $\dim_\mathbb{R} \mathfrak{m}_1 = k$.

\begin{beispiel*}
	$\mathfrak{m} = \mathfrak{m} \oplus \mathfrak{m}'$ $\dim \mathfrak{m} = k$.
	\[
		P = \begin{pmatrix}
			\alpha_1 {\id_k} & \alpha_{1,2} {\id_k} \\
			\alpha_{1,2} {\id_k} & \alpha_2 (\id_k)
		\end{pmatrix}
		\text{ \enquote{$\in$} } \Mat(2,\mathbb{R})
	\]
\end{beispiel*}

Linksinvariante Metriken auf Liegruppe $H=\set*{e}$. $\mathfrak{m} = \mathfrak{g} = P_0 = \mathfrak{m}_1 \oplus \ldots \oplus \mathfrak{m}_n$ (alle 1-dimensional)
\[
	\begin{pmatrix}
		\alpha_{1,1} & \cdots & \alpha_{1,m} \\
		\vdots & & \\
		\alpha_{1,n} & \cdots & \alpha_{n,n}
	\end{pmatrix} > 0
\]
Im komplexen Fal: $\mathfrak{m}_1$ vom komplexen Typ $\Rightarrow \dim_\mathbb{R} \mathfrak{m}_1 = k = 2 \cdot \ell$.
Es gilt: Diagonale wie oben ($= \alpha_i \cdot {\id_k}$ Schurs Lemma). 
Nebendiagonalelemente gegeben durch 
\[
	P_{ij} = \alpha_{i,j} \cdot \begin{pmatrix}
		Z_{ij} & & \\
		& \ddots & \\
		& & Z_{ij}
	\end{pmatrix}
\]
wobei
\[
	Z_{i,j} = \begin{pmatrix}
		a_{i,1}  & -b_{i,j} \\
		b_{i,1} & a_{i,j}
	\end{pmatrix} \longleftrightarrow a_{i,j} + i \cdot b_{i,j}
\]
Im quaternionalen Fall ($k=4 \ell$)
\[
	P_{ij} = \alpha_{ij} \cdot \begin{pmatrix}
		q_{i,j} & & \\
		& \ddots & \\
		& & q_{i,j}
	\end{pmatrix}
\]
\[
	q_{i,j} = \begin{pmatrix}
		\alpha & -\beta & - \gamma & - \delta \\
		\beta & \alpha & - \delta & + \gamma \\
		\gamma & \delta & \alpha & - \beta \\
		\delta & -\gamma & \beta & \alpha
	\end{pmatrix}
\]

Für den fall von linksinvarianter Metriken auf Liegruppen wollen wir nun die Formel für die Riccikrümmung noch besser verstehen.
Wir wählen eine Skalarprodukt $\skal*{\cdot }{\cdot }_0 \colon \mathfrak{g} \times \mathfrak{g} \to \mathbb{R}$.
Jedes weitere Skalarprodukt $\skal*{\cdot }{\cdot}$ auf $\mathfrak{g}$ können wir schreiben als
\[
	\skal*{\cdot }{\cdot } = \skal*{P \cdot }{\cdot }_0 = \skal*{h \cdot }{h \cdot }_0
\]
($h \in \GL(n,\mathbb{R})$, $\mathfrak{g} \cong \mathbb{R}^n$).
Die linksinvarianten Vektorfelder auf $G$ definieren Lieklammern $\mu_0 \coloneqq \benbrace*{\cdot,\cdot} \colon \mathfrak{g} \times \mathfrak{g} \to \mathfrak{g}$.
Es gilt 
\[
	\ric(g) (X,Y) = - \frac{1}{2} \mathcal{B}_\mathfrak{g}(X,Y) + \frac{1}{4} \sum_{i,j} \skal*{\benbrace*{X_i,X_j}}{X}^2 - \frac{1}{2} \sum_{i} \norm*{\benbrace*{X_i,X}}^2 - \skal*{\ad(Z)(X)}{X}
\]
$(X_1, \ldots ,X_n)$ $\skal*{\cdot }{\cdot }$-ONB.
Ist $(\overline{X}_1, \ldots ,\overline{X}_n)$ $\skal*{\cdot }{\cdot }_0$-ONB, so ist $\enbrace*{h^{-1} \overline{X}_1, \ldots , h^{-1} \overline{X}_n}$ eine $\skal*{\cdot }{\cdot }$-ONB, denn 
\[
	\skal*{h^{-1} \overline{X}_i}{h^{-1}\overline{X}_j} = \skal*{h \enbrace*{h^{-1} \overline{X}_i}}{h \enbrace*{h^{-1} \overline{X}_j}}_0 = \skal*{\overline{X}_i}{\overline{X}_j}_0 = \delta_{ij}
\]
Somit folgt für $X= h^{-1} \overline{X}$
\begin{align}
	\frac{1}{4}  \sum_{i,j} \skal*{\benbrace*{X_i,X_j}}{h^{-1} \overline{X}} - \frac{1}{2} \sum_i \norm*{\benbrace*{X_i, h^{-1} \overline{X}}}^2 &= \frac{1}{4} \sum_{i,j} \skal*{ \benbrace*{h^{-1} \overline{X_i}, h^{-1} \overline{X}_j}}{h^{-1} \overline{X}} - \frac{1}{2} \sum_i \skal*{\benbrace*{h^{-1} \overline{X}_i, h^{-1} \overline{X}}}{\benbrace*{h^{-1} \overline{X}_i, h^{-1} \overline{X}}}   \\
	&= \frac{1}{4} \sum_{i,j} \skal*{ h \benbrace*{h^{-1} \overline{X}_i, h^{-1} \overline{X}_j}}{\overline{X}}_0 - \frac{1}{2} \sum_i \skal*{h \benbrace*{h^{-1} \overline{X}_i, h^{-1} \overline{X}_j}}{\ldots }_0 
\end{align}
Für $h \in \GL(n,\mathbb{R})$ definieren wir eine \enquote{Lieklammer}
\mapdef{h . \mu_0 \colon \mathfrak{g} \times \mathfrak{g}}{\mathfrak{g}}{\enbrace*{\overline{X}, \overline{Y}}}{h \benbrace*{h^{-1} \overline{X}, h^{-1} \overline{Y}}}{}
wobei $\benbrace*{\cdot ,\cdot }= \mu_0$ \enquote{die} Lieklammer auf $\mathfrak{g}$ ist.
Beobachtung
\begin{itemize}
	\item $h . \mu_0 (\overline{X},\overline{Y}) = - (h \mu_0) \enbrace*{\overline{Y},\overline{X}}$
	\item Jacobi-Identität ist erfüllt.
	\item Für $X = h^{-1} \overline{X}$ und eine $\skal*{\cdot }{\cdot }_0$-ONB $\enbrace*{\overline{X}_1,\ldots ,\overline{X}_n}$ gilt 
	\begin{align}
		\frac{1}{4} \sum_{i,j} \skal*{\benbrace*{h^{-1} \overline{X}_i, h^{-1} \overline{X}_j}}{h^{-1} \overline{X}} &= \frac{1}{4} \sum \skal*{h \benbrace*{h^{-1} \overline{X}_i, h^{-1} \overline{X}_j}}{\overline{X}}_0 \\
		&= \frac{1}{4}\sum \skal*{(h.\mu_0) \enbrace*{\overline{X}_i, \overline{X}_j}}{\overline{X}}_0
	\end{align}
\end{itemize}
Wie früher definieren wir\marginnote{zum Vergleich $\ad(X)(Y)= \benbrace*{X,Y}$}
\mapdef{\ad_{h.\mu_0} \colon \mathbb{R}^n}{\End(\mathbb{R}^n)}{\overline{X}}{\enbrace*{\overline{Y} \mapsto (h.\mu_0)(\overline{X}, \overline{Y})}}{}
Nun gilt
\[
	\mathcal{B}_\mathfrak{g} \enbrace*{h^{-1} \overline{X}, h^{-1} \overline{X}} = \Spur \enbrace*{\ad(h^{-1} \overline{X}) \circ \ad(h^{-1} \overline{X})}
\]
Es gilt $\ad_{h.\mu_0}(\overline{X}) = h \circ \ad(h^{-1} \overline{X}) \circ h^{-1}$ und somit
\begin{align}
	\mathcal{B}_\mathfrak{g} \enbrace*{h^{-1} \overline{X}, h^{-1} \overline{X}} &= \Spur \enbrace*{h \circ  \ad(h^{-1}\overline{X}) \circ h^{-1} \circ h \circ \ad(h^{-1} \overline{X}) \circ h^{-1}} \\
	&= \Spur \enbrace*{\ad_{h.\mu_0}(\overline{X}) \circ \ad_{h.\mu_0}(\overline{X})} =: \mathcal{B}_{h.\mu_0}(\overline{X},\overline{X})
\end{align}
Weiter
\begin{align}
	\skal*{\benbrace*{Z, h^{-1} \overline{X}}}{h^{-1} \overline{X}} = \skal*{h \benbrace*{Z,h^{-1} \overline{X}}}{\overline{X}}_0 = \skal*{h \benbrace*{h^{-1}(hZ), h^{-1} \overline{X}}}{\overline{X}}_0 = \skal*{\ad_{h.\mu_0}(hZ)(X)}{\overline{X}}_0
\end{align}

\begin{satz}
	Sei $G$ eine Liegruppe mit $\Tmap_e G = \mathbb{R}^n$ und Skalarprodukt $\skal*{\cdot }{\cdot }_0$.
	Für $h \in \GL(n,\mathbb{R})$ sei $g = \skal*{\cdot }{\cdot } = \skal*{h \cdot }{h \cdot }_0$.
	Dann gilt
	\begin{align}
		\ric(g) \enbrace*{h^{-1} \overline{X}, h^{-1} \overline{X}} &= -\frac{1}{2} \mathcal{B}_{h.\mu_0} \enbrace*{\overline{X},\overline{X}} + \frac{1}{4} \sum_{i,j} \skal*{(h.\mu_0)(\overline{X}_i, \overline{X}_j)}{\overline{X}}_0^2 - \frac{1}{2} \sum_i \norm*{(h.\mu_0)(\overline{X}_i, X)}_0^2 \\
		&\hspace{1em} - \skal*{\ad_{h.\mu_0}(Z_{h.\mu_0})(\overline{X})}{\overline{X}} _0
	\end{align}
	wobei $\Spur \ad_{h.\mu_0}(\overline{X}) = \skal*{Z_{h.\mu_0}}{\overline{X}}_0$.
\end{satz}
\begin{beweis}
	Es genügt zu zeigen, dass $Z_{h.\mu_0} = h \cdot Z$.
	Dies gilt da
	\begin{align}
		\Spur \ad_{h.\mu_0} (\overline{X}) = \Spur h \circ \ad(h^{-1} \overline{X}) \circ h^{-1} = \skal*{Z}{h^{-1} \overline{X}} = \skal*{h Z}{\overline{X}}_0 
	\end{align}
	Also gilt $h \mathbb{Z} = Z_{h.\mu_0}$.
\end{beweis}

Wir bezeichnen mit $V(n) \coloneqq  \Lambda^2(\mathbb{R}^n)^* \otimes \mathbb{R}^n$ den Raum der bilinearen Abbildungen von $\mathbb{R}^n \times \mathbb{R}^n$ nach $\mathbb{R}^n$, die schiefsymmetrisch in den beiden ersten Argumenten sind.
\[
	\mu \in V(u) : \mu(x,y) = - \mu(y,x) \in \mathbb{R}^n
\]
Das \enquote{gegebene} Skalarprodukt $\skal*{\cdot }{\cdot }_0$ auf $\mathbb{R}^n$ induziert ein Skalarprodukt auf $V(n)$:
\[
	\skal*{\mu}{\lambda}_0 \coloneqq \sum_{i,j} \skal*{\mu(\overline{X}_i, \overline{X}_j)}{\lambda(\overline{X}_i, \overline{X}_j)}_0
\]
wobei $(\overline{X}_1, \ldots ,\overline{X}_n)$ eine $\skal*{\cdot }{\cdot }_0$-ONB ist. Dies ist wohldefiniert.
Ferner operiert die Gruppe $\GL(n)$ auf $V(n)$:
\[
	\Phi \colon \GL(n) \times V(n) \to V(n) \qquad \mu \mapsto h.\mu
\]
mit $(h.\mu)(\overline{X},\overline{Y}) \coloneqq h \mu \enbrace*{h^{-1} \overline{X}, h^{-1} \overline{Y}}$.
Man rechnet leicht nach, dass $(h_1 \cdot h_2).\mu = h_1.(h_2.\mu)$ ist.
Die Gruppe $\GL(n)$ operiert \emph{nicht} isometrisch auf $V(n)$, aber $\On(n) \subset \GL(n)$ schon!

\begin{bemerkung*}
	\begin{itemize}
		\item Die Orbiten sind nicht immer abgeschlossen!
		\item $\sfrac{V(n)}{\GL(n)}$ ist \emph{nicht} Hausdorff!
		\item Nichtkompaktheit von $\GL(n)$ ist \enquote{abelsch}:\marginnote{mit $T^n$ die Diagonalmatrizen}
		\[
			\GL(n) \StackText{diffeo}{=\joinrel=}\On(n) \cdot \sfrac{\GL(n)}{\On(n)} = \On(n) \cdot T^m \cdot \On(n)
		\]
		Dies führt zur sogenannten \emph{Geometrischen Invariantentheorie}.\footnote{siehe \url{https://en.wikipedia.org/wiki/Geometric_invariant_theory}}
	\end{itemize}
\end{bemerkung*}

Wir berechnen nun die Linearisierungen obiger Wirkung von $\GL(n)$ auf $V(n)$:
Sei $A \in \mathfrak{gl}(n,\mathbb{R}) = \Tmap_{\id} \GL(n) = \End(n)$ und $h(t)=\exp(t \cdot A)$.
Dann gilt $h(0)= {\id}$ und $h'(0) =A$ und es gilt
\[
	\diffd{}{t}\Big|_{t=0} (h_t.\mu)(\overline{X},\overline{Y}) = \diffd{}{t}\Big|_{t=0} h_t.\mu \enbrace*{h_t^{-1} \cdot \overline{X}, h_t^{-1} \overline{Y}}
	= A \mu(\overline{X},\overline{Y}) - \mu(A \overline{X}, \overline{Y}) - \mu(\overline{X}, A \overline{Y})
\]

\begin{definition}
	Wir bezeichnen mit $\pi \colon \End(n) \times V(n) \to V(n)$, $(A,\mu) \mapsto \pi(A)\cdot \mu$ mit 
	\[
		\pi(A)\mu(\cdot ,\cdot ) \coloneqq A \mu(\cdot ,\cdot ) - \mu (A \cdot ,\cdot ) - \mu(\cdot ,A \cdot )
	\]
\end{definition}

Die lineare Darstellung $\pi$ von $\End(n) = \mathfrak{gl}(n,\mathbb{R})$ ist eine Liealgebrendarstellung, das heißt
\[
	\pi \enbrace*{\benbrace*{A,B}} = \benbrace*{\pi(A), \pi(B)}
\]

\begin{lemma}
	Sei $A \in \mathfrak{so}(n) = \Tmap_e \SO(n) \subset \End(V(n))$.
	Dann gilt 
	\[
		\skal*{\pi(A) \mu}{\mu}_0 = 0 \enspace \forall \mu \in V(n)
	\]
	Damit ist $\pi(A)$ schiefsymmetrisch. Sei $S \in  \Sym(n) \subset \End(n)$.
	Dann gilt für alle $\mu,\lambda \in V(n)$
	\[
		\skal*{\pi(S) \mu}{\lambda}_0 = \skal*{\mu}{\pi(S)\lambda}_0
	\]
	Das heißt $\pi(S)$ ist symmetrisch.
\end{lemma}
\begin{beweis}
	Sei $A \in \mathfrak{so}(n)$.
	Dann gilt $\exp(t A) \in \SO(n)$.
	Somit gilt 
	\[
		\norm*{\mu}_0^2 \equiv \skal[\big]{\exp(t A). \mu}{ \exp(tA).\mu}_0
	\]
	Somit ist $0= \skal*{\pi(A).\mu}{\mu}_0$.
	Dies zeigt die erste Behauptung.
	Sei $S \in \Sym(n)$ und sei $\enbrace*{\overline{X}_1, \ldots , \overline{X}_n}$ eine Orthonormalbasis von Eigenvektoren bezüglich Eigenwerten $s_1, \ldots, s_n \in \mathbb{R}$.
	Dann gilt
	\begin{align}
		\skal*{\pi(S).\mu}{\lambda}_0 &= \sum_{i,j} \skal*{\enbrace*{\pi(S).\mu}(\overline{X}_i, \overline{X}_j)}{\lambda \enbrace*{\overline{X}_i, \overline{X}_j}}_0 \\
		&= \sum_{i,j,k} \skal*{S.\mu \enbrace*{\overline{X}_i, \overline{X}_j} - \mu \enbrace*{S \overline{X}_i, \overline{X}_j} - \mu \enbrace*{\overline{X}_i, S \overline{X}_j}}{\overline{X}_k}_0 \cdot \skal*{\lambda \enbrace*{\overline{X}_i, \overline{X}_j}}{\overline{X}_k}_0 \\
		&= \sum_{i,j,k} \enbrace*{s_k - s_i - s_j} \skal*{\mu \enbrace*{\overline{X}_i, \overline{X_j}}}{\overline{X}_k}_0 \cdot \skal*{\lambda \enbrace*{\overline{X}_i, \overline{X}_j}}{\overline{X}_k}_0 \\
		&= \skal*{\mu}{\pi(S) \lambda}_0 \qedhere
	\end{align}
\end{beweis}

\begin{definition}[{name=[Momentenabbildung]}]
	Die Abbildung $m \colon V(n) \setminus \set*{0} \to \Sym(n)$ implizit definiert mittels
	\[
		\skal*{m(\mu)}{S}_0 \coloneqq \sfrac{\skal*{\pi(S). \mu}{\mu}_0}{\norm*{\mu}^2}
	\]
	nennt man \Index{Momentenabbildung}. Wir setzen
	\[
		M_\mu \coloneqq \sfrac{1}{4} m(\mu) \cdot \norm*{\mu}^2
	\]
\end{definition}


\begin{lemma}
	Für $\overline{x}, \overline{y} \in \mathbb{R}^n$ gilt
	\[
		\skal*{M_\mu \overline{x}}{\overline{x}}_0 = \frac{1}{4} \sum_{i,j} \skal*{\mu(\overline{x}_i, \overline{x}_j)}{\overline{x}}_0^2 - \frac{1}{2} \sum_i \norm*{\mu(\overline{x}_i, \overline{x})}^2
	\]
	wobei $\enbrace*{\overline{x}_1, \ldots ,\overline{x}_n}$ eine Orthonormalbasis von $\mathbb{R}^n$ ist.
\end{lemma}
\begin{beweis}
	Sei $S 0 \diag(0, \ldots ,0, 1,0,\ldots ,0)$ mit $1$ an der Stelle $k$.
	Dann gilt
	\begin{align}
		\skal*{\pi(S) \mu}{\mu}_0 &= \sum_{i,j} \skal*{\enbrace*{\pi(S)\mu} \enbrace*{\overline{x}_i, \overline{x}_j}}{\mu \enbrace*{\overline{x}_i, \overline{x}_j}} \\
		&= \sum_{i,j} \skal*{S \mu \enbrace*{\overline{x}_i, \overline{x}_j} - \mu(S \overline{x}_i, \overline{x}_j) - \mu(\overline{x}_i, S \overline{x_j})}{\overline{x}_k}_0 \cdot \skal*{\mu(\overline{x}_i, \overline{x}_j)}{\overline{x}_k}_0 \\
		&= sum_{i,j} \skal*{\mu \enbrace*{\overline{x}_i, \overline{x}_j}}{x_k}_0^2 - 2 \sum_i \norm*{\mu(\overline{x}_i, \overline{x}_k)}_0^2
	\end{align}
	Für $S = \begin{psmallmatrix}
		& & \\
		& & 1 \\
		& 1 &
	\end{psmallmatrix}$ analog.
\end{beweis}

\begin{satz}
	Unter den Vorraussetzungen von Satz 3.2.10 gilt
	\begin{align}
		\ric(g)\enbrace*{h^{-1} \overline{x},h^{-1} \overline{x}} &= -\frac{1}{2} \mathcal{B}_{h.\mu_0} (\overline{x},\overline{x}) + \skal*{M_{h.\mu_0}(\overline{x})}{\overline{x}}_0 - \skal*{\ad_{h.\mu_0}(Z_{h.\mu_0})(\overline{x})}{\overline{x}}_0 \\
		&= \skal*{\enbrace*{- \frac{1}{2} \mathcal{B}_{h.\mu_0} + M_{h.\mu_0} - \ad_{h.\mu_0}(Z_h.\mu_0)}(\overline{x})}{\overline{x}}_0
	\end{align}
	wobei $\mathcal{B}_{h.\mu_0}(\cdot ,\cdot ) = \skal*{Z_{h.\mu_0} \cdot }{\cdot }_0$
\end{satz}

Bezeichnung $\Ric_\mu \coloneqq -\frac{1}{2} \mathcal{B}_\mu + M_\mu - \ad_\mu(Z_\mu)$ Ricci-Endomorphismus von $\mu \in V(n)$.

\begin{bemerkung*}
	$G$ nilpotent, also $\mathcal{B}_\mathfrak{g} =0$ und $Z_\mu =0$ und damit $\Ric_{h.\mu_0}=M_{h.\mu_0}$
\end{bemerkung*}
% section 32 (end)

\newpage
\section{Topologie homogener Räume} % (fold)
\label{sec:33}

\begin{definition}[{name=[Faserbündel]}]
	Seien $X,B,F$ differenzierbare Mannigfaltigkeiten und $\pi \colon X \to B$ differenzierbar.
	Dann nennt man $(X,B,F,\pi)$ ein \Index{Faserbündel} mit \Index{Faser}  $F$, \Index{Totalraum} $X$  und \bet{Basis}\index{Faserbündel!Basis} $B$, falls gilt 
	\begin{enumerate}[1)]
		\item $\pi$ ist surjektiv
		\item Für alle $b \in B$ existiert eine offene Umgebung $U \subseteq B$ von $b$, sowie ein Diffeomorphismus $\varphi_U \colon U \times F \to \pi^{-1}(U) \subseteq X$, sodass $\pr_1 = \pi \circ \varphi_U$ (\Index{lokale Trivialität}).
	\end{enumerate}
	Man schreibt 
	\(
		\begin{tikzcd}[cramped,sep=small]
			F^k \rar & X^n \rar & B^{n-k}
		\end{tikzcd}
	\)
\end{definition}

\begin{beispiel*}[{name=[einfache Faserbündel]}]
	\begin{enumerate}[(i)]
		\item \(
			\begin{tikzcd}[cramped,sep=small]
				F \rar & F \times B \rar & B
			\end{tikzcd} 
		\)
		ist das triviale Faserbündel.
		\item $K \le H \le G$ abgeschlossene Untergruppen von $G$, so ist 
		\(
			\begin{tikzcd}[cramped,sep=small]
				\sfrac{H}{K} \rar & \sfrac{G}{K} \rar & \sfrac{G}{H}
			\end{tikzcd}
		\)
		ein Faserbündel.
		\item Für $k=0$: Ein Faserbündel mit diskreter Faser ist eine \Index{Überlagerung}!
	\end{enumerate}
\end{beispiel*}

\begin{definition}[{name=[Prinzipalbündel]}]
	Ist $H$ eine Liegruppe, so ist ein $H$-\Index{Prinzipalbündel} ein differenzierbares Faserbündel \(
		\begin{tikzcd}[cramped,sep=small]
			H \rar & X \rar & B
		\end{tikzcd}
	\)
	zusammen mit einer freien transitiven differenzierbaren Rechtsliegruppenwirkung $\Phi \colon X \times H \to X$ mit $\pi(p.h) = \pi(p)$.
\end{definition}

\begin{beispiel*}[{name=[Prinzipalbündel]}]
	\begin{enumerate}[(i)]
		\item Ist $H$ eine abgeschlossene Lieuntergruppe von $G$, so ist \(
			\begin{tikzcd}[cramped,sep=small]
				H \rar & G \rar & \sfrac{G}{H}
			\end{tikzcd}
		\) ein $H$-Prinzipialbündel. 
		Die lokale Trivilität ist direkt durch Abbildung gegeben, die im Beweis von \autoref{satz:312} konstruiert wurde.
		\item \Index{Hopf-Faserung}: $\SU(n) \subseteq \Un(n) \subseteq \SU(n+1)$.
		\[
			\begin{tikzcd}
				\sfrac{\Un(n)}{\SU(n)} \dar[phantom,sloped,"\cong"] \rar & \sfrac{\SU(n+1)}{\SU(n)} \dar[phantom,sloped,"\cong"] \rar &  \sfrac{\SU(n+1)}{\Un(n)} \dar[phantom,sloped,"\cong"] \\[-1em]
				S^1 & S^{2n+1} & \mathbb{C}P^n
			\end{tikzcd}
		\] 
		Die Rechtsliegruppenwirkung $\Phi \colon S^{2n+1} \times S^1 \to S^{2n+1}$ ist gegeben durch
		\[
			\enbrace*{\begin{pmatrix}
				z_1 \\ \vdots \\ z_{n+1}
			\end{pmatrix}, e^{i \varphi}} \longmapsto
			\begin{pmatrix}
				z_1 \\ \vdots \\ z_{n+1} \cdot e^{i \varphi}
			\end{pmatrix}
		\]
	\end{enumerate}
\end{beispiel*}

\begin{definition}[{name=[Homotopiegruppen]}]
	Sei $X$ ein topologischer Raum, $p \in X$ und $v \in S^n$.
	Dann heißt 
	\[
		\pi_n(X,p) \coloneqq \benbrace[\big]{(S^n,v) , (X,p)} = \set[\big]{\benbrace*{f} \given f \colon S^n \to X \text{ stetig  mit } f(p)=v }
	\]
	\Index{n-te Homotopiegruppe von $X$} zum Basispunkt $p$ (punktierte Homotopieklassen).
	
	Analoge Definition: Betrachte den Einheitswürfel $I^n = \benbrace*{0,1}^n$. Dann ist $\pi_n(X,p) = \benbrace[\big]{(I^n,\partial I^n), (X,p)}$ und die Multiplikation gegeben durch $\benbrace*{f} \cdot \benbrace*{g} = \benbrace*{f g}$ mit
	\[
		f g (t ) = \begin{cases}
			f(2t_1, t_2, \ldots ,t_n) &\text{ falls }0 \le t_1 \le \sfrac{1}{2}\\
			g(2t_1 -1, t_2, \ldots , t_n) &\text{ falls } \sfrac{1}{2} \le t_1 \le 1
		\end{cases}
	\]
\end{definition}

Hier einige zentrale Eigenschaften, die hier ohne Beweis aufgeführt werden:
\begin{enumerate}[(i),itemsep=1pt]
	\item Wenn $X$ wegzusammenhängend ist, so ist $\pi_k(X,p) \cong \pi_k(X,\tilde{p})$ für alle $p, \tilde{p} \in X$. 
	Man schreibt dann $\pi_k(X)$.
	\item $\pi_k(X)$ ist eine Gruppe für $k \ge 1$ und abelsch für $k \ge 2$.
	\item $f \colon X \to Y$ stetig induziert einen Homomorphismus $f_* \colon \pi_n(X) \to \pi_n(Y)$ durch $\benbrace*{g} \mapsto \benbrace*{f \circ g}$.
	\item Sind $f, \tilde{f} \colon X \to Y$ homotop, so ist $f_* = \tilde{f}_*$ (Homotopieinvarianz).
	\item Für $f \colon X \to Y$, $g \colon Y \to Z$ gilt $(g \circ f)_* = g_* \circ f_*$ (Funktorialität).
\end{enumerate}

Ist $M$ eine differenzierbare Mannigfaltigkeit, so setzen wir 
\[
	C_k(M) \coloneqq \set*{\sum\nolimits_{i=1}^N r_i s_i \given N \in \mathbb{N}, r_i \in \mathbb{Z}, s_i \text{ $k$-Simplex in }M  }
\]
Dies liefert einen Kettenkomplex und damit lässt sich Homologie definieren.
Folgender Satz stellt einen Zusammenhang zwischen Homotopiegruppen und Homologie her:
\begin{satz}[name={Hurewicz}]
	Sei $M^n$ eine $m$-zusammenhängende, glatte Mannigfaltigkeit mit $m>0$, das heißt $\pi_k(M^n)= \set*{e}$ für $0 \le k \le m$.
	Dann gilt
	\begin{enumerate}[(i)]
		\item $H_k(M^n) =0$ für $1 \le k \le m$
		\item $\pi_{m+1}(M^n) \to H_{m+1}(M^n)$, $\benbrace*{\varphi} \mapsto \varphi_* \benbrace*{S^{m+1}}$ ist ein Isomorphismus, wobei $\benbrace*{S^{m+1}}$ ein Erzeuger von $H_{m+1}(S^{m+1}) \cong \mathbb{Z}$ ist.
	\end{enumerate}
\end{satz}
\begin{beweis}
	\emph{Siehe \textcite[Thm.~4.32]{Hatcher}.}
\end{beweis}

\begin{bemerkung*}[{name=[Homologie und Homotopiegruppen der Sphäre]}]
	Für $n \ge 1$ ist $S^n$ $(n-1)$-zusammenhängend mit $\pi_n(S^n) = \mathbb{Z} = H_n(S^n)$.
	Höhere Homotopiegruppen von $S^n$ zu bestimmen ist schwierig; höhere Homologiegruppen verschwinden.
\end{bemerkung*}

\begin{satz}[{name={Homotopiesequenz}}]
	Sei \(
		\begin{tikzcd}[cramped,sep=small]
			F \rar["\iota"] & X \rar["\pi"] & B
		\end{tikzcd}
	\)
	ein Faserbündel, $b \in B$, $x \in \pi^{-1}(b)=F$.
	Dann gibt es eine lange exakte Sequenz
	\[
		\begin{tikzcd}
			\ldots  \rar & \pi_n(F,x) \rar["\iota_*"] & \pi_n(X,x) \rar["\pi_*"] & \pi_n(B,b) \rar["\partial"] & \pi_{n-1}(F,x) \rar & \ldots \rar & \pi_0(B,b)
		\end{tikzcd}
	\]
\end{satz}
\begin{beweis}
	\emph{Siehe \textcite[Thm.~4.41]{Hatcher}.}
\end{beweis}

\begin{beispiel*}[{name=[Zusammenhangskomponenten wenn Quotient einfach zusammenhängend]}]
	Betrachte \(
		\begin{tikzcd}[cramped,sep=small]
			H \rar & G \rar & \sfrac{G}{H}
		\end{tikzcd}
	\) mit $\sfrac{G}{H}$ $1$-zusammenhängend.
	Dann ist die Sequenz 
	\[
		\begin{tikzcd}
			0 \rar & \pi_0(H) \rar & \pi_0(G) \rar &  0
		\end{tikzcd}
	\]
	exakt, das heißt $H$ und $G$ haben gleich viele Zusammenhangskomponenten ($\sfrac{G}{H} \cong \sfrac{G_e}{H \cap G_e}$ falls $\sfrac{G}{H}$ zusammenhängend).
\end{beispiel*}

Sei $H^*(M^n)$ die Kohomologiegruppe bezüglich der \Index{de-Rham-Kohomologie}.
Das $\cupp$-Produkt liefert zusätzlich eine Ringstruktur auf $H^*(M^n)$
\mapdef{\cupp \colon H^a(M^n) \times H^b(M^n)}{H^{a+b}(M^n)}{(\omega,\mu)}{\omega \wedge \mu}{}
Kohomologie kann man viel allgemeiner mit Koeffizienten in einem Ring $R$ betrachten ($R$ kommutativ mit Eins).

\begin{satz}[name={Serre-Sequenz, Spanier},label=satz:336]
	Sei \(
		\begin{tikzcd}[cramped,sep=small]
			F \rar["\iota"] & X \rar["\pi"] & B
		\end{tikzcd}
	\) ein differenzierbares Faserbündel, $F$ zusammenhängend, $B$ einfach-zusammenhängend, $k, l > 0$ mit
	\begin{enumerate}[1)]
		\item $H^i(F)=0$ für $0 < i < k$
		\item $H^j(B)=0$ für $0 < j< l$
	\end{enumerate}
	Dann ist die folgende Sequenz exakt
	\[
		\begin{tikzcd}[sep=small]
			0 \rar & H^1(X) \rar["\iota^*"] & H^1(F) \rar["\partial"] & H^2(X) \rar & \ldots \rar & H^{k+l-1}(B) \rar & H^{k+l-1}(X) \rar & H^{k+l-1} (F)
 		\end{tikzcd}
	\]
\end{satz}
\todo[inline]{Referenzen für diese Sätze raussuchen}
\begin{satz}[name={Leray-Hirsch},label=satz:337]
	Sei \(
		\begin{tikzcd}[cramped,sep=small]
			F \rar["i"] & X \rar["\pi"] & B
		\end{tikzcd}
	\) ein differenzierbares Faserbündel mit
	\begin{enumerate}[1)]
		\item $H^j(F,R)$ ist ein endlich erzeugter freier $R$-Modul für alle $j$
		\item Es existiert $C_j^\alpha \in H^j(X;R)$, sodass die Bilder $\iota^*(c^\alpha_j)^{k_j}_{\alpha=1}$ in $H^j(F;R)$ eine Basis von $H^j(F;R)$ bilden
	\end{enumerate}
	Dann ist $\varphi \colon H^*(B) \otimes_R H^*(F;R) \to H^*(X;R)$, $\sum_{k,j} b_k \otimes \iota^*(c_j) \mapsto \sum_{k,j} \pi^*(b_k) \cupp c_j$ ein Isomorphismus von $R$-Moduln.
\end{satz}

\begin{definition}[{name=[äußere Algebra]}]
	Seien $a_1, \ldots ,a_k$ formale Variablen.
	Dann ist die \Index{äußere Algebra} 
	\[
		\Lambda_R \benbrace*{a_1, \ldots ,a_k}
	\]
	definiert vermittels der Relationen 
	\begin{enumerate}[(i)]
		\item $a_i a_j = - a_j a_i$
		\item $a_1 \cdot \ldots \cdot a_k \neq 0$
	\end{enumerate}
	Es gilt $\Lambda_R \benbrace*{a_1, \ldots ,a_k} = \Span_R \set*{1, a_1, \ldots ,a_k, a_1 a_2, \ldots , a_1 \cdots a_k}$ mit $\Rg \enbrace*{\Lambda_R \benbrace*{a_1, \ldots ,a_k}} = 2^k$.
\end{definition}

Es gilt $\Lambda_R \benbrace*{a_1, \ldots ,a_{k+1}} = \Lambda_R \benbrace*{a_1, \ldots ,a_k} \mathbin{\hat{\otimes}_R} \Lambda \benbrace*{a_{k+1}}$ (graduiertes Tensorprodukt) mit 
\[
	a \mathop{\hat{\otimes}} b \cdot \benbrace*{c \mathop{\hat{\otimes}} d} = (-1)^{\abs*{c} \cdot \abs*{b}} (a \cdot c) \hat{\otimes} (b \cdot d)
\]
und $\abs*{\overline{a}} \in \set*{0,1} $ $\mathbb{Z}_2$-graduiert, das heißt $\abs*{a_i}=1$, $\abs*{a_i a_j}=0 $

\begin{satz}[{name=[Kohomologie der unitären Gruppe]}]
	Es gibt einen $R$-Algebraisomorphismus 
	\[
		H^*(\Un(n);R) \cong \Lambda_R \benbrace*{x_1, x_3, \ldots , x_{2n-1}}
	\]
	Unter diesem gilt $x_i \in H^i(\Un(n);R)$.
\end{satz}
\begin{beweis}
	Per Induktion: Für $n=1$ ist $\Un(1)=S^1$, das heißt $H^0(S^1;R) = R \cdot 1$ und $H^1(S^1;R) = R \cdot x_1$.
	Damit ist $H^*(\Un(1);R) = \Lambda_R[x_1]$.
	
	Für $n \ge 2$ können wir annehmen, dass $H^* \enbrace*{\Un(n-1);R} \cong \Lambda_r \benbrace*{x_1, \ldots , x_{2n-3}}$ gilt.
	Wir betrachten das Faserbündel
	\[
		\begin{tikzcd}
			\Un(n-1) \rar & \Un(n) \rar & \sfrac{\Un(n)}{\Un(n-1)} = S^{2n-1}
		\end{tikzcd}
	\]
	Es gibt $H^i \enbrace*{S^{2n-1};R} = 0$ für $0 < i < 2n-1$.
	Aus \autoref{satz:336} erhalten wir 
	\[
		\begin{tikzcd}[column sep=1.8em,row sep=1em]
			& 0 \rar & H^1 \enbrace[\big]{\Un(n)} \rar & H^1 \enbrace[\big]{\Un(n-1)} \rar & 0 = H^2\enbrace*{S^{2n-1}}  \\
			&\vdots & \vdots & \vdots & \vdots \\
			\cdots \rar & H^{2n-3}\enbrace*{S^{2n-1}} = 0  \rar & H^{2n-3} \enbrace[\big]{\Un(n)} \rar & H^{2n-3} \enbrace[\big]{\Un(n-1)} \rar & 0 = H^{2n-2} \enbrace*{S^{2n-1}}
		\end{tikzcd}
	\]
	Somit ist $i^* \colon H^k \enbrace*{\Un(n)} \to H^k \enbrace*{\Un(n-1)}$ ein Isomorphismus für alle $1\le k \le 2n-3$.
	Nach Vorraussetzung ist $H^* \enbrace*{\Un(n-1)}$ ein freier $R$-Modul, das heißt es gibt Klassen $c_1, \ldots ,c_{2n-3} \in H^*(\Un(n))$ mit $x_i = i^*(c_i)$\marginnote{$\Rightarrow c_{i_1} \cdot \ldots \cdot c_{i_\ell} \in H^{i_1 +\ldots + i_\ell}(\Un(n))$} und 
	\[
		\set[\big]{i^* \enbrace*{c_{i_1} \cdot \ldots \cdot  c_{i_\ell}} \given i_1 < \ldots < i_\ell}
	\]
	ist eine $R$-Basis von $H^* \enbrace*{\Un(n-1)}$.
	Aus \autoref{satz:337} folgt
	\[
		H^* \enbrace[\big]{\Un(n)} \cong H^* \enbrace[\big]{\Un(n-1)} \otimes_R H^* \enbrace*{S^{2n-1}} = \Lambda_r \benbrace*{x_1, x_3, \ldots ,x_{2n-3}} \otimes_R \enbrace*{R \oplus R \benbrace*{S^{2n-1}}}
	\]
	als $R$-Moduln.
	Das heißt $H^* \enbrace*{\Un(n)}$ wird erzeugt durch die Elemente 
	\[
		c_{i_1} \cdot \ldots \cdot c_{i_\ell} , c_{i_1} \cdot \ldots \cdot c_{i_\ell} \cdot \pi^* \benbrace*{S^{2n-1}}
	\]
	für $0 < i_1 < \ldots < i_\ell$.
	Da Cup-Produkte von ungeraden Kohomologieklassen antikommutieren (siehe \textcite[Thm.~3.11]{Hatcher}) und ferner $H^{n^2}(\Un(n)) \neq 0$ gilt ($\Un(n)$ ist Liegruppe, also orientierbar), folgt $c_1 \cdot \ldots \cdot c_{2n-3} \cdot \pi^* \benbrace*{S^{2n-1}} \neq 0$.\marginnote{$1+3+ \ldots + 2n-1 = n^2$}
	Somit ist 
	\[
		H^* \enbrace[\big]{\Un(n)} = \Lambda_R \benbrace*{c_1, \ldots , c_{2n-1}}
	\]
	als $R$-Algebra mit $c_{2n-1} = \pi^*\benbrace*{S^{2n-1}}$.
\end{beweis}

\begin{bemerkung*}[{name=[Kohomologie der speziellen unitären Gruppe]}]
	Ähnlich zeigt man ($n \ge 2$)
	\[
		H^* \enbrace[\big]{\SU(n);R} = \Lambda_R\benbrace*{x_3, x_5, \ldots , x_{2n-1}} \qquad H^* \enbrace[\big]{\Sp(n);R} = \Lambda_R \benbrace*{x_3, x_7, \ldots , x_{4n-1}}
	\]
\end{bemerkung*}
\begin{beweis}[name={Skizze}]
	Induktionsanfang mit $\SU(2) = \Sp(1) = S^3$. Weiter ist 
	\[
		3+5+ \ldots + 2n - 1 = n^2 -1 \qquad 3+7+ \ldots + 4n-1 = 2n^2 +n \qedhere
	\]
\end{beweis}
% section 33 (end)
% chapter 3 (end)

\cleardoubleoddemptypage
\pagenumbering{Alph}
\setcounter{page}{1}
\cleardoubleoddemptypage
\appendix

\chapter{Anhang} % (fold)
\label{sec:anhang}
%!TEX root = ana_top_geo.tex

\subsection{Ausführlicher Beweis zu \cref{lem:kpt-schnitte}} % (fold)
\label{sub:kpt-schnitte}
Sei $X$ ein Hausdorffraum. Dann ist $X$ genau dann kompakt, wenn gilt: Hat eine Familie $\mathcal{A}$ von abgeschlossenen Teilmengen von $X$ die endliche 
Durchschnittseigenschaft, so gilt 
\[
	\bigcap_{A \in \mathcal{A}} A \not= \emptyset.
\]
\begin{beweis}
	Für die erste Implikation sei $X$ kompakt und $\mathcal{A}$ eine Familie von abgeschlossenen Mengen mit der endlichen Durchschnittseigenschaft.
	Angenommen $\bigcap_{A \in \mathcal{A}} A = \emptyset$.
	Dann gilt
	\[
		X = X \setminus \bigcap_{A \in \mathcal{A}} A = \bigcup_{A \in \mathcal{A}} X \setminus A.
	\]
	Nun ist $\mathcal{U} \coloneqq \set*{X \setminus A \given A \in \mathcal{A}}$ eine offene Überdeckung von $X$ und da $X$ kompakt ist, existiert $\mathcal{A}_0 \subset \mathcal{A}$ endlich, sodass
	\[
		X = \bigcup_{A \in \mathcal{A}_0} X \setminus A = X \setminus \underbrace{\bigcap_{A \in \mathcal{A}_0 } A }_{\neq \emptyset} \quad \light
	\]
	Für die umgekehrte Implikation sei nun $\mathcal{U} = \set{U_i}_{i \in I}$ eine offene Überdeckung von $X$.
	Angenommen für jede endliche Teilmenge $J \subseteq I$ gilt $X \neq \bigcup_{i \in J} U_i$.
	Betrachte nun $\mathcal{A} =  \set{X \setminus U_i}_{i \in I}$. Dann gilt nach Annahme
	\[
		\bigcap_{i \in J} X \setminus U_i = X \setminus \bigcup_{i \in J} U_i \neq \emptyset.
	\]
	Also hat $\mathcal{A}$ die endliche Durchschnittseigenschaft. Nach Vorraussetzung gilt dann
	\[
		\emptyset \not= \bigcap_{i \in I} X \setminus U_i = X \setminus \underbrace{\bigcup_{i \in I} U_i}_{= X} \quad \light \qedhere
	\]
\end{beweis}


\subsection[Blatt3, Aufgabe 4: Hilfssatz für den Hauptsatz der Algebra]{Blatt 3, Aufgabe 4} % (fold)
\label{sub:B3A4}
\emph{Diese Übungsaufgabe ist zentral für den Beweis des Hauptsatzes der Algebra, \cref{satz:hauptsatz-algebra}.} 

Sei $p(x)= x^n + a_{n-1} x^{n-1} + \ldots + a_1 x + a_0$ mit $n \in \mathbb{N}_0$ ein Polynom mit Koeffizienten $a_i \in \mathbb{C}$, dass \emph{keine} Nullstelle in $\mathbb{C}$ besitzt. 
Sei $S^1= \set*{z \in \mathbb{C} \given \abs*{z}=1}$.
\begin{enumerate}[(a)]
	\item $f \colon S^1 \to S^1$ gegeben durch $f(z) = \frac{p(z)}{\abs*{p(z)} } $ ist wohldefiniert und homotop zu einer konstanten Abbildung.
	\item $f$ ist homotop zur Abbildung $g_n \colon S^1 \to S^1$ mit $g_n(z)= z^n$.
\end{enumerate}
\minisec{Beweis}
\begin{enumerate}[(a)]
	\item \begin{description}
		\item[Wohldefiniertheit:] Sei $z \in S^1$ beliebig. Dann gilt
		\[
			\abs*{\frac{p(z)}{\abs*{p(z)} } } = \frac{1}{\abs*{p(z)} } \cdot \abs*{p(z)} =1,
		\]
		also ist $f(z) \in S^1$.
		\item[Homotop zu einer konstanten Abbildung:] Definiere $f_t \colon S^1 \to S^1$ für $t \in [0,1]$ durch 
		\[
			f_t(z) = \frac{p(t \cdot z)}{\abs*{p(t \cdot z)} } 
		\]
		Dies ist mit der gleichen Begründung wie oben wohldefiniert. 
		Außerdem ist $f_0(z)= \frac{a_0}{\abs*{a_0} } \in S^1 $ konstant und $f_1(z)= \frac{p(z)}{\abs*{p(z)} }=f(z)$. 
		Definiere nun $H \colon S^1 \times [0,1] \to S^1$ durch $H(x,t) \coloneqq f_t(x)$. 
		Dann ist $H$ stetig, da Polynome und $\abs*{.} $, sowie Multiplikation stetig sind. 
		$H$ ist die gesuchte Homotopie.
	\end{description}
	\item Sei $h \colon S^1 \times [0,1] \to \mathbb{C}$ gegeben durch $h(z,t) = z^n + \sum_{k=0}^{n-1} a_k z^k t^{n-k}$. 
	Dann gilt $h(z,0)=z^n \not= 0$, da $z \in S^1$.
	Für $t \neq 0$ gilt nun
	\begin{align*}
		h(z,t) = 0 \iff \frac{h(z,t)}{t^n} = 0 \iff \frac{z^n}{t^n} + \sum_{k=0}^{n-1} a_k \frac{z^k}{t^k} = 0 \iff p \enbrace*{\frac{z}{t}} = 0
	\end{align*}
	Aber nach Vorraussetzung gilt $p \enbrace*{\frac{z}{t}} \neq 0$. 
	Also $h(z,t) \neq 0$ für alle $t \in [0,1]$. 
	Definiere nun $H \colon S^1 \times [0,1]\to S^1$ durch $H(z,t) = \frac{h(z,t)}{\abs*{h(z,t)}}$. 
	Wie eben gezeigt, ist dies wohldefiniert und offensichtlich stetig. Da
	\[
		H(z,0) = \frac{z^n}{\abs*{z^n} } = z^n \quad \text{ und } \quad H(z,1) = \frac{h(z,1)}{\abs*{h(z,1)} } = \frac{p(z)}{\abs*{p(z)} } =f(z)
	\]
	ist $H$ die gesuchte Homotopie. \qedhere
\end{enumerate}

\subsection{Blatt 10, Aufgabe 3} % (fold)
\label{sub:B10A3}
\emph{Diese Übungsaufgabe lieferte den Beweis zu \cref{prop:iso-covering}.} \smallskip \\
Sei $p \colon \overline{X} \to X$ eine Überlagerung. 
Seien $\overline{x}_0  \in \overline{X}$ und $x_0= p(\overline{x}_0 )$ Basispunkte. 
Dann ist die induzierte Abbildung $\pi_n (p) \colon \pi_n(\overline{X}, \overline{x}_0) \to \pi_n(X,x_0)$ ein Isomorphismus für alle $n \ge 2$.
\minisec{Beweis}
Als Überlagerung ist $p$ stetig, also ist $\pi_n(p)$ ein Gruppenhomomorphismus nach \hyperref[prop:eig-hom-gruppen:enum:4]{ \cref*{prop:eig-hom-gruppen} \ref*{prop:eig-hom-gruppen:enum:4}}.
\begin{description}
	\item[Surjektivität:] Sei $[\omega] \in \pi_n(X,x_0)$, also $\omega \colon I^n \to X$ mit $\omega(\partial I^n) = \set{x_0}$. Betrachte $\omega$ nun als Abbildung $I^{n-1} \times [0,1] \to X$:
	\[
		\begin{tikzcd}[column sep=4em]
			I^{n-1} \times \set{0} \dar[hook] \rar["\mathrm{const}_{\overline{x}_0}"] & \overline{X} \dar["p"]\\
			I^{n-1} \times I \rar["\omega"] & X  
		\end{tikzcd}
	\]
	$\mathrm{const}_{\overline{x}_0} \colon I^{n-1} \times \set{0}$ ist eine Hebung von $\omega\big|_{I^{n-1} \times \set{0}} \equiv x_0$. 
	Nach dem Homotopiehebungssatz (\ref{satz:hebung-homotopie}) existiert eine Hebung $\overline{\omega} \colon I^{n-1} \times I \to \overline{X}$ von $\omega$ mit $\overline{\omega}\big|_{I^{n-1} \times \set{0}} \equiv \overline{x}_0 $. 
	Also gilt
	\[
		p \circ \overline{\omega} \big|_{\partial I^n} = \omega \big|_{\partial I^n} \equiv x_0 \enspace \Longrightarrow \enspace \overline{\omega} \big|_{\partial I^n} 
		\in p ^{-1}( \set{x_0} ) .
	\]
	Da $p^{-1}(\set{x_0})$ diskret und $\partial I^n$ für $n \ge 2$ zusammenhängend ist, muss $\overline{\omega} \big|_{\partial I^n}$ konstant sein. 
	Da $\overline{\omega}\big|_{I^{n-1} \times \set{0}} \equiv \overline{x}_0 $ gilt, folgt somit $\overline{\omega}(\partial I^n) = \set{\overline{x}_0}$. 
	Also ist $[\overline{\omega}] \in \pi_n(\overline{X},\overline{x}_0)$ und weiter gilt
	\[
		\pi_n(p) \enbrace*{[\overline{\omega}]} = [p \circ \overline{\omega} ] = [\omega] \in \pi_n(X,x_0). 
	\]
	\item[Injektivität:] Sei $[\omega] \in \ker \pi_n(p)$, also $[p \circ \omega] = [c_{x_0}]$. 
	Es existiert also eine Homotopie $H$ relativ $\partial I^n$ zwischen $p \circ \omega$ und $c_{x_0}$. 
	Offensichtlich ist $\omega$ eine Hebung von $p \circ \omega$. 
	Mit dem Homotopiehebungssatz erhalten wir eine Hebung $\overline{H}$ von $H$ mit $\overline{H}(-,0) = \omega$. 
	Weiter wissen wir, dass
	\[
		\overline{H} \big|_{\partial I^n \times [0,1]} \in p ^{-1}(\set{x_0} ) \quad \text{ und }\quad  \overline{H} \big|_{ I^n \times \set{1}} \in p ^{-1}(\set{x_0} )
	\]
	gelten muss, da $H = p \circ \overline{H}$ und $H(-,1)= c_{x_0} \equiv x_0$. 
	Mit dem gleichen Argument wie oben folgt, dass $\overline{H} \big|_{\partial I^n \times [0,1]}$ und $\overline{H} \big|_{ I^n \times \set{1}}$ konstant sind. 
	Für $z \in \partial I^n$ gilt nun
	\[
		\overline{H}(z,0) = \omega(z) = \overline{x}_0
	\]
	Da $\partial I^n \times [0,1] \cap I^n \times \set{1} \not= \emptyset$, muss also auch $\overline{H}(-,1) \equiv \overline{x}_0$ gelten. 
	Damit folgt $[\omega] = [c_{x_0}]$.\qedhere
\end{description}
\printindex
\printbibliography
% \listoffigures
\listoftodos[To-do's und andere Baustellen]\todototoc
\end{document}
