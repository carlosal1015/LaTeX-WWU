%!TEX TS-program = xelatex
%!TEX TS-options = -shell-escape
%!TEX root = ../LieGrp_SS16/liegruppen.tex
\RequirePackage{fix-cm} 
\documentclass[a4paper, twoside, headsepline, index=totoc,toc=listof,toc=bibliography,toc=index, fontsize=10pt, cleardoublepage=empty, headinclude, DIV=12, BCOR=5mm, titlepage,draft]{scrreprt}
%!TEX root = ../AnaTopGeo_SS14/ana_top_geo.tex
\usepackage{scrtime} % KOMA, Uhrzeit ermoeglicht

%--Pakete zum "Programmieren"
% ======================================================================================
\usepackage{etoolbox}
\usepackage{letltxmacro}
\usepackage{ifthen}
% ======================================================================================

%--Farbdefinitionen und Grafiken (muss vor tikz geladen werden)
% ======================================================================================
\usepackage[usenames, table, x11names]{xcolor}
\definecolor{dark_gray}{gray}{0.45}
\definecolor{light_gray}{gray}{0.6}
\definecolor{fb10_blue}{cmyk}{0.8,0.4,0.13,0.07}
\usepackage[final]{graphicx}
\usepackage{adjustbox}
\newcommand{\cfbox}[2]{% coloured frame box
	\ifmmode
	\mathchoice{\adjustbox{cfbox=#1}{$\displaystyle#2$}}{\adjustbox{cfbox=#1}{$\textstyle#2$}}{\adjustbox{cfbox=#1}{$\scriptstyle#2$}}{\adjustbox{cfbox=#1}{$\scriptscriptstyle#2$}}
	\else
	\adjustbox{cfbox=#1}{#2}
	\fi
}
% ======================================================================================

%--Zum Zeichnen/ TikZ-Kram (vor polyglossia bzw. babel geladen werden)
% ======================================================================================
\usepackage{tikz}
\usepackage{tikz-cd}
\usetikzlibrary{external}
\tikzset{>=latex}
\usetikzlibrary{%
	shapes,
	arrows.meta,
	intersections,
	calc,
	3d,
	decorations.pathreplacing,decorations.markings,decorations.pathmorphing,
	angles,
	quotes,
}
\tikzexternalize[prefix=tikz/,up to date check=diff]
\pgfkeys{/pgf/images/include external/.code=\includegraphics{#1}}
\tikzset{external/system call={lualatex \tikzexternalcheckshellescape -halt-on-error -interaction=batchmode --shell-escape -jobname "\image" "\texsource"}}
\AtBeginEnvironment{tikzcd}{\tikzexternaldisable} % tikzexternalize fuer tikzcd deaktivieren, da inkompatibel
\AtEndEnvironment{tikzcd}{\tikzexternalenable}
\tikzset{% um Inkompatibilitaeten von quotes und polyglossia bzw. babel zu vermeiden
  every picture/.append style={
    execute at begin picture={\shorthandoff{"}},
    execute at end picture={\shorthandon{"}}
  }
}
\usepackage{pgfplots}
\usepgfplotslibrary{colormaps}
\newcommand*\circled[1]{\tikzexternaldisable\tikz[baseline=(char.base)]{\node[shape=circle,draw,inner sep=2pt] (char) {#1};}\tikzexternalenable}
% ======================================================================================



%-- Mathepakete etc.
% ======================================================================================
\usepackage[T1]{fontenc}
\renewcommand{\rmdefault}{zpltlf}
\usepackage{mathtools} % beinhaltet amsmath
\mathtoolsset{showonlyrefs,centercolon,showmanualtags}
\newtagform{brackets}[\textbf]{[}{]}
\usetagform{brackets}
\usepackage{fix-cm}
\usepackage[bbgreekl]{mathbbol}
\usepackage{amssymb,marvosym} 
\usepackage{nicefrac} % schräge Brüche
\usepackage{faktor}
\newcommand{\Faktor}[1]{\faktor[\textstyle]{#1}}
\usepackage{xfrac}
\usepackage{cancel}
\usepackage{mathdots} % Verbesserung von Punkten wie zB \ldots
\usepackage[bb=px]{mathalfa} % \mathbb als px font
\usepackage{centernot}
\usepackage{stackrel}
\DeclareSymbolFont{bbold}{U}{bbold}{m}{n}
\DeclareSymbolFontAlphabet{\mathbbold}{bbold}
\newcommand{\ind}{\mathbbold{1}} % charakteristische-Funktion-Eins
\def\mathul#1#2{\color{#1}\underline{{\color{black}#2}}\color{black}} %farbiges Untersteichen im Mathe-Modus
\renewcommand{\le}{\leqslant}
\renewcommand{\ge}{\geqslant}
% ======================================================================================


%-- Von xfrac erzeuge font warnings ignorieren
% ======================================================================================
\usepackage{silence}
\WarningFilter{latexfont}{Size substitutions with differences}
\WarningFilter{latexfont}{Font shape `U/bbold/m/n' in size}
% ======================================================================================


%-- Typographie/Polyglossia
% ======================================================================================
\usepackage[euler-digits]{eulervm} % vor fontspec laden!
\usepackage[no-math]{fontspec}
\usepackage{polyglossia} % moderner babel-ersatz
\setmainlanguage[spelling=new,babelshorthands=true]{german}
\shorthandoff{"}
\setotherlanguage{english}
\defaultfontfeatures{Mapping=tex-text, WordSpace={1.2}, Ligatures={Required,Common,Contextual},Extension=.otf} %


\setmainfont{TeXGyrePagellaX}[UprightFont=*-Regular,BoldFont=*-Bold,ItalicFont=*-Italic,BoldItalicFont=*-BoldItalic,ItalicFeatures={Style=Historic},Ligatures={Required,Common,Contextual,Historic}]
\setsansfont{texgyreadventor}[Scale=MatchUppercase, UprightFont=*-regular, BoldFont=*-bold, ItalicFont=*-italic, BoldItalicFont=*-bolditalic]
\setmonofont{SourceCodePro}[Scale=0.9,UprightFont=*-Regular, BoldFont=*-Semibold, ItalicFont=*-Light]
\usepackage{xltxtra}
\usepackage{fontawesome}
\usepackage[final]{microtype}
\usepackage[draft=false]{scrlayer-scrpage} 
\flushbottom
% ======================================================================================


%-- Aufzählungen
% ======================================================================================
\usepackage[shortlabels,inline]{enumitem}
\setlist[itemize,1]{label=\faCaretRight}
\setlist[enumerate]{font=\bfseries}
\setlist[description]{font=\normalfont\bfseries}
\usepackage{multicol}
% ======================================================================================


%-- Floats/Figures/Tabellen
% ======================================================================================
\usepackage{wrapfig}
\usepackage{float}
\usepackage[margin=10pt, font=small, labelfont={sf, bf}, format=plain, indention=1em]{caption}
\captionsetup[wrapfigure]{name=Abb. }
\usepackage{booktabs}
% ======================================================================================


%-- korrekte Anführungszeichen und Zitierbefehle
% ======================================================================================
\usepackage[autostyle,german=quotes,english=british]{csquotes}
% ======================================================================================


%--Indexverarbeitung
% ======================================================================================
\usepackage{makeidx}
\newcommand{\bet}[1]{\textbf{\emph{#1}}}
\newcommand{\Index}[1]{\bet{#1}\index{#1}}
\makeindex
\setindexpreamble{{\noindent\sffamily\small Die \emph{Seitenzahlen} sind mit Hyperlinks versehen und somit anklickbar} \par \bigskip}
\renewcommand{\indexpagestyle}{scrheadings}
% ======================================================================================


%-- Marginnotes/Todonotes/Footnotes
% ======================================================================================
\deffootnote[1.5em]{1.5em}{1.5em}{\textsuperscript{\thefootnotemark}\ }
\usepackage[fulladjust]{marginnote}
\renewcommand*{\marginfont}{\itshape\footnotesize}
\usepackage[textsize=small]{todonotes}
\usepackage{ragged2e}
\renewcommand*{\raggedleftmarginnote}{\RaggedLeft}
\renewcommand*{\raggedrightmarginnote}{\RaggedRight}
\LetLtxMacro{\oldtodo}{\todo}
\renewcommand{\todo}[2][]{\tikzexternaldisable\oldtodo[#1]{#2}\tikzexternalenable}
\LetLtxMacro{\oldmissingfigure}{\missingfigure}
\renewcommand{\missingfigure}[2][]{\tikzexternaldisable\oldmissingfigure[{#1}]{#2}\tikzexternalenable}
% ======================================================================================


% -- BibLaTeX
% ======================================================================================
\usepackage[%
	backend=biber,
	sortlocale=auto,
	natbib,
	hyperref,
	backref,
	style=alphabetic
	]%
{biblatex}
\renewcommand*{\mkbibnamelast}[1]{%
  \ifmknamesc{\textsc{#1}}{#1}}
\renewcommand*{\mkbibnameprefix}[1]{%
  \ifboolexpr{ test {\ifmknamesc} and test {\ifuseprefix} }
    {\textsc{#1}}
    {#1}}
\def\ifmknamesc{%
  \ifboolexpr{ test {\ifcurrentname{labelname}}
               or test {\ifcurrentname{author}}
               or ( test {\ifnameundef{author}} and test {\ifcurrentname{editor}} ) }}
\addbibresource{../!config/quellen.bib}
% ======================================================================================

%--Konfiguration von Hyperref und Cleveref
% ======================================================================================
\usepackage[hidelinks, pdfpagelabels,  bookmarksopen=true, bookmarksnumbered=true, linkcolor=black, urlcolor=SkyBlue2, plainpages=false,pagebackref, citecolor=black, hypertexnames=true, pdfauthor={Jannes Bantje}, pdfborderstyle={/S/U}, linkbordercolor=SkyBlue2, colorlinks=false,final,backref=false]{hyperref}
\usepackage[nameinlink,noabbrev]{cleveref}
\newcommand{\appendLink}[1]{#1\,\faExternalLink}
\newcommand{\hrefsym}[2]{\href{#1}{\texttt{\appendLink{#2}}}}
\newcommand{\hrefsymX}[2]{\href{#1}{\appendLink{#2}}}
\newcommand{\hrefsymmail}[2]{\href{#1}{\texttt{\faEnvelopeO\,#2}}}
\renewcommand{\url}[1]{\hrefsym{#1}{\nolinkurl{#1}}}
% ======================================================================================


% -- QR-Codes (hinter hyperref laden!)
% ======================================================================================
\usepackage{qrcode}
% ======================================================================================

%--Römische Zahlen
% ======================================================================================
\newcommand{\RM}[1]{\MakeUppercase{\romannumeral #1{}}}
% ======================================================================================

%-- Definition von diversen Mathe-Befehlen
% ======================================================================================
%!TEX root = mitschrift_main.tex

% -- Zum Finetuning von Befehlen
% ======================================================================================
\makeatletter
\newcommand{\raisemath}[1]{\mathpalette{\raisem@th{#1}}}
\newcommand{\raisem@th}[3]{\raisebox{#1}{$#2#3$}}
\makeatother
\makeatletter
\newcommand{\killDescendersM}[1]{\mathpalette{\killD@scendersM{#1}}}
\newcommand{\killD@scendersM}[2]{\raisebox{0pt}[\height][0pt]{$#2#1$}}
\makeatother
\DeclareRobustCommand{\minwidthbox}[2]{%
  \ifmmode
    \expandafter\mathmakebox
  \else
    \expandafter\makebox
  \fi
  [\ifdim#2<\width\width\else#2\fi]{#1}%
}
% ======================================================================================


%-- Klammerbefehle
% ======================================================================================
\DeclarePairedDelimiter{\abs}{\lvert}{\rvert}
\DeclarePairedDelimiter{\floor}{\lfloor}{\rfloor}
\DeclarePairedDelimiter{\ceil}{\lceil}{\rceil}
\DeclarePairedDelimiter\norm{\Vert}{\Vert}
\DeclarePairedDelimiter\enbrace{(}{)}
\DeclarePairedDelimiter\benbrace{[}{]}
\DeclarePairedDelimiter\bbenbrace{[\![}{]\!]}
\DeclarePairedDelimiter\lenbrace{<}{>}
\DeclarePairedDelimiter\angbrace{\langle}{\rangle}
\newcommand{\ssbrace}[1]{{\scriptscriptstyle\enbrace{#1}}}
\newcommand{\ssbbrace}[1]{{\scriptscriptstyle\benbrace{#1}}}
% ======================================================================================

%-- Mengen
% ======================================================================================
\newcommand\SetSymbol[1][]{\nonscript\:#1\vert\allowbreak\nonscript\:\mathopen{}}
\providecommand\given{} % to make it exist
\DeclarePairedDelimiterX\set[1]\{\}{\renewcommand\given{\SetSymbol[\delimsize]}#1}
% ======================================================================================

%-- Skalarprodukt (3 Varianten) 
% ======================================================================================
\DeclarePairedDelimiterX\sprod[2]{\langle}{\rangle}{#1\,\delimsize\vert\,#2}
\DeclarePairedDelimiterX\skal[2]{\langle}{\rangle}{#1\,,\,#2}
\makeatletter
\DeclareFontFamily{OMX}{MnSymbolE}{}
\DeclareSymbolFont{MnLargeSymbols}{OMX}{MnSymbolE}{m}{n}
\SetSymbolFont{MnLargeSymbols}{bold}{OMX}{MnSymbolE}{b}{n}
\DeclareFontShape{OMX}{MnSymbolE}{m}{n}{
    <-6>  MnSymbolE5
   <6-7>  MnSymbolE6
   <7-8>  MnSymbolE7
   <8-9>  MnSymbolE8
   <9-10> MnSymbolE9
  <10-12> MnSymbolE10
  <12->   MnSymbolE12
}{}
\DeclareFontShape{OMX}{MnSymbolE}{b}{n}{
    <-6>  MnSymbolE-Bold5
   <6-7>  MnSymbolE-Bold6
   <7-8>  MnSymbolE-Bold7
   <8-9>  MnSymbolE-Bold8
   <9-10> MnSymbolE-Bold9
  <10-12> MnSymbolE-Bold10
  <12->   MnSymbolE-Bold12
}{}
\let\llangle\@undefined
\let\rrangle\@undefined
\DeclareMathDelimiter{\llangle}{\mathopen}%
                     {MnLargeSymbols}{'164}{MnLargeSymbols}{'164}
\DeclareMathDelimiter{\rrangle}{\mathclose}%
                     {MnLargeSymbols}{'171}{MnLargeSymbols}{'171}
\makeatother
\DeclarePairedDelimiterX\sskal[2]{\llangle}{\rrangle}{#1\,,\,#2}
% ======================================================================================

%-- Abbildungsdefinition
% ======================================================================================
\newcommand{\mapdef}[5]{%
	\[
		\begin{array}{rcl}
			\textstyle #1 &\xrightarrow{\minwidthbox{#5}{2em}} & \textstyle #2 \\[0.5ex]
			\textstyle #3 &\xmapsto{\minwidthbox{\mbox{ }}{2em}} & \textstyle #4
		\end{array}
	\]
}
% ======================================================================================

%-- modifiziertes Stackrel 
% ======================================================================================
\newcommand{\StackText}[2]{\stackrel{\mbox{\scriptsize #1}}{#2}}
\newcommand{\StackTextClap}[2]{\stackrel{\mathclap{\mbox{\scriptsize #1}}}{#2}}
% ======================================================================================

%-- Blitz
% ======================================================================================
\newcommand{\light}{\text{\raisebox{-.3ex}{\Large\Lightning}}}
% ======================================================================================


%-- Underbrace u.Ä. als Befehl in LaTeX-Syntax (und ohne Spacingprobleme mit nachfolgenden Operatoren...)
% ======================================================================================
\newcommand{\Underbrace}[2]{{\underbrace{#1}_{#2}}}
\newcommand{\Underbracket}[2]{{\underbracket[0.7pt][2pt]{#1}_{#2}}}
\newcommand{\Overbracket}[2]{{\overbracket[0.7pt][2pt]{#1}^{#2}}}
% ======================================================================================


%-- Deklaration weiterer Operatoren (allgemein)
% ======================================================================================
\DeclareMathOperator{\re}{Re} % Realteil
\let\Re\relax
\DeclareMathOperator{\Re}{Re} % Realteil
\DeclareMathOperator{\im}{im} % Bild
\let\Im\relax
\DeclareMathOperator{\Im}{Im} % Bild
\DeclareMathOperator{\id}{id} % identische Abbildung
\DeclareMathOperator{\conj}{conj} % Konjugation
\DeclareMathOperator{\sgn}{sgn} % Signum
\DeclareMathOperator{\End}{End} % Endomorphismen
\DeclareMathOperator{\Hom}{Hom} % Homomorphismen
\DeclareMathOperator{\Iso}{Iso} % Isomorphismen
\DeclareMathOperator{\Aut}{Aut} % Automorphismen
\DeclareMathOperator{\Span}{span} % Span
\DeclareMathOperator{\coker}{coker} % Kokern
\DeclareMathOperator{\Tr}{Tr} % Spur,Trace
\DeclareMathOperator{\pr}{pr} % Projektion
\DeclareMathOperator{\diag}{diag} % Diagonalmatrix
\DeclareMathOperator{\Rg}{Rg} % Rang
\DeclareMathOperator{\const}{const} % konstante Abbildung
\DeclareMathOperator{\Spur}{Spur} % Spur
\DeclareMathOperator{\Arg}{Arg} % Argument
\DeclareMathOperator{\dist}{dist} % Distanz
\DeclareMathOperator{\supp}{supp} % Träger
\DeclareMathOperator{\Char}{char} % Charakteristik
% ======================================================================================


%-- Deklaration weiterer Operatoren (Differentiale etc.)
% ======================================================================================
\DeclareMathOperator{\grad}{grad} % Gradient
\DeclareMathOperator{\dive}{div} % Gradient
\DeclareMathOperator{\rot}{rot} % Rotation
\newcommand{\D}{\ensuremath{\mathrm{D}\mkern-1.0mu}} % Differential
\newcommand{\mathd}{\ensuremath{\mathrm{d}\mkern-1.0mu}} % äußere Ableitung
\newcommand{\Tmap}{\ensuremath{\mathrm{T}\mkern-0.85mu}} % Tangentialraum
\let\Tang\Tmap
\DeclareMathOperator{\Diff}{Diff}
\newcommand{\diff}[2]{\ensuremath{\frac{{\partial #1}}{{\partial #2}} }}
\newcommand{\diffd}[2]{\ensuremath{\frac{\mathd #1}{\mathd #2} }}
\DeclareMathOperator{\rank}{rank}
% ======================================================================================


%-- Deklaration weiterer Operatoren (Topologie)
% ======================================================================================
\newcommand*\interior[1]{\overset{\smash{\raisebox{-0.18ex}{$\scriptstyle\circ$}}}{#1}}
\newcommand{\sing}{{\raisemath{1.1pt}{\scriptscriptstyle\mathrm{sing}}}}
\newcommand{\pt}{\mathrm{pt}}
\DeclareMathOperator{\Zyl}{Zyl}
\newcommand{\rZyl}{\widetilde{\Zyl}}
\DeclareMathOperator{\Tel}{Tel}
\newcommand{\op}{\mathrm{op}}
\DeclareMathOperator{\Sp}{Sp}
\DeclareMathOperator{\Keg}{Keg}
\newcommand{\slashedi}{i\hspace{-3.5pt}/}
\newcommand{\cupp}{\smallsmile}
\newcommand{\capp}{\smallfrown}
\DeclareMathOperator*{\colim}{colim}
\DeclareMathOperator{\PD}{PD}
\newcommand{\lf}{\mathrm{lf}}
\DeclareMathOperator{\sig}{sig}
\DeclareMathOperator{\Tor}{Tor}
\DeclareMathOperator{\Ext}{Ext}
\DeclareMathOperator{\AW}{AW}
\DeclareMathOperator{\Proj}{Proj}
\DeclareMathOperator{\Gr}{Gr}
\DeclareMathOperator{\res}{res}
\DeclareMathOperator{\Spec}{Spec}
\DeclareMathOperator{\co}{co}
\DeclareMathOperator{\ch}{ch}
\DeclareMathOperator{\wOp}{w}
\DeclareMathOperator{\Ar}{Ar}
\newcommand{\actson}{\mathrel{\curvearrowright}}
\let\acts\actson
\let\action\actson
\DeclareMathSymbol{\bbDelta}{\mathord}{bbold}{"01}
\newcommand{\DDelta}{\bbDelta}
\DeclareMathOperator{\Star}{Star}
\DeclareMathOperator{\Link}{Link}
\DeclareMathOperator{\EPK}{EPK}
\DeclareMathOperator{\Vol}{Vol}
\newcommand{\cell}{{\raisemath{1.1pt}{\scriptscriptstyle\mathrm{cell}}}}
\DeclarePairedDelimiter{\homologieklasse}{\llbracket}{\rrbracket}
\newcommand{\rand}[1]{\ensuremath{\partial^{\scriptscriptstyle #1}}}
\DeclareMathOperator{\ab}{ab}
\DeclareMathOperator{\CW}{CW}
% ======================================================================================


%-- Deklaration von Operatoren (Liegruppen)
% ======================================================================================
\DeclareMathOperator{\GL}{GL}
\DeclareMathOperator{\SO}{SO}
\DeclareMathOperator{\Ad}{Ad}
\DeclareMathOperator{\ad}{ad}
\DeclareMathOperator{\On}{O}
\DeclareMathOperator{\Un}{U}
\DeclareMathOperator{\SU}{SU}
\DeclareMathOperator{\Mat}{Mat}
\DeclareRobustCommand{\Der}{\mathop{\mathfrak{der}}}
\DeclareMathOperator{\SL}{SL}
\DeclareMathOperator{\Graph}{Graph}
\DeclareMathOperator{\Int}{Int}
\DeclareRobustCommand{\intAlg}{\mathop{\mathfrak{int}}}
\DeclareMathOperator{\aut}{aut}
\DeclareMathOperator{\Rad}{Rad}
\DeclareMathOperator{\Nil}{Nil}
\DeclareMathOperator{\rad}{rad}
\DeclareMathOperator{\nil}{nil}
\DeclareMathOperator{\Ric}{Ric}
\DeclareMathOperator{\ric}{ric}
\newcommand{\bi}{\mathrm{bi}}
\DeclareMathOperator{\Isom}{Isom}
\DeclareMathOperator{\Sym}{Sym}
\newcommand{\opL}{\ensuremath{\mathrm{L}\mkern-0.6mu}}
% ======================================================================================

%-- Deklaration von Operatoren (Funktionalanalysis)
% ======================================================================================
\DeclareMathOperator{\tr}{tr}
\newcommand{\w}{\mkern1mu\mathrm{w}}
\newcommand{\sa}{\mathrm{sa}}
\newcommand{\vb}{\mathrm{v\mkern-2.5mu.b\mkern-1.5mu.}} % vollständig beschränkt
\newcommand{\so}{\mathrm{\mkern.3mu s\mkern-1.4mu.\mkern-.6mu o\mkern-1.7mu.}} % \newcommand{\so}{\mathrm{s.o.}}
\newcommand{\solim}{\so\text{-}\mkern-0.8mu\lim}
\newcommand{\wo}{\mathrm{w\mkern-3mu.\mkern-.4mu o\mkern-1.7mu.}}
\newcommand{\Top}[1]{\mathcal{T}_{\mkern-2.3mu #1}}
\newcommand{\weakT}[1]{\ensuremath{\mathcal{T}_{#1}^{\mkern+1.0mu\text{\raisebox{0.4ex}{$\mathrm{w}$}}}}}
\newcommand{\weakTstar}[1]{\ensuremath{\mathcal{T}_{#1}^{\mkern+1.0mu\text{\raisebox{0.4ex}{$\mathrm{w}$}}^*}}}
\newcommand{\TWeakStar}{\Top{\w^*}}
\newcommand{\TWeakOp}{\Top{\wo}}
\newcommand{\Tso}{\Top{\so}}
\newcommand{\finSub}{\subset\mkern-0.7mu \subset}
\DeclareMathOperator{\Inv}{Inv}
\newcommand{\simm}{{\hspace{-1.6pt}\raisemath{0.5pt}{\sim}}}
\newcommand{\plus}{{\hspace{-1.6pt}+}}
\DeclareMathOperator{\ev}{ev}
\DeclareMathOperator{\Alg}{Alg}
\DeclareMathOperator{\her}{her}
\newcommand{\subher}{\subset_{\her}}
\newcommand{\grenzw}[1]{\xrightarrow{\minwidthbox{#1}{1.4em}}}
\newcommand{\grenzwl}[1]{\xleftarrow{\minwidthbox{#1}{1.4em}}}
\newcommand{\grenzwIn}[1]{\grenzw{\raisemath{-2pt}{#1}}}
\newcommand{\MyTo}[1]{\tikzexternaldisable\mathbin{\tikz[baseline] \draw[-to,line width=.4pt] (0ex,0.94ex) -- (#1,0.94ex);}\tikzexternalenable}
\newcommand{\dlim}{%
    \mathchoice
      {\lim\limits_{\MyTo{4.2ex}}}% \displaystyle
      {\lim\limits_{\MyTo{2.8ex}}}% \textstyle
      {\lim\limits_{\MyTo{2.3ex}}}% \scriptstyle
      {\lim\limits_{\MyTo{2.3ex}}}% \scriptscriptstyle
}
\newcommand{\Dlim}{\killDescendersM{\dlim}}
\DeclareMathOperator{\sep}{sep}
\DeclareMathOperator{\diam}{diam}
\DeclareMathOperator{\conv}{conv}
\DeclareMathOperator{\Prim}{Prim}
\DeclareMathOperator{\hull}{hull}
\DeclareMathOperator{\red}{red}
\DeclarePairedDelimiterX\bra[1]{\langle}{\rvert}{#1\,}
\DeclarePairedDelimiterX\ket[1]{\lvert}{\rangle}{\,#1}
\DeclarePairedDelimiterX\bracket[2]{\langle}{\rangle}{#1\,\delimsize\vert\,#2}
\newcommand{\tensormax}{\mathbin{\otimes_{\max}}}
\newcommand{\tensormin}{\mathbin{\otimes_{\min}}}
\DeclareMathOperator{\Ped}{Ped}
\newcommand{\alg}{\mathrm{alg}}
\DeclareMathOperator{\CPC}{CPC}
\DeclareMathOperator{\CP}{CP}
\DeclareMathOperator{\UPC}{UPC}
\newcommand{\DeltaOp}{\mathbin{\Delta}}
\newcommand{\kernedP}{\mathcal{P}\mkern-2mu}
\newcommand{\Pinfty}{\kernedP_{\infty}}
\DeclareMathOperator{\Groth}{Groth}
\DeclareMathOperator{\rk}{rk}
\newcommand{\MvN}{\mathrm{MvN}}
% ======================================================================================

%-- Kategorien
% ======================================================================================
\DeclareMathOperator{\Mor}{Mor}
\DeclareMathOperator{\mor}{mor}
\DeclareMathOperator{\Obj}{Obj}
\DeclareMathOperator{\Ob}{Ob}
\newcommand{\TOP}{\textsc{Top}}
\newcommand{\HTOP}{\textsc{HTop}}
\newcommand{\VR}{\textsc{VR}}
\newcommand{\MOD}{\textsc{Mod}}
\newcommand{\Mod}[1]{#1\text{-}\MOD}
\newcommand{\MONOIDE}{\textsc{Monoide}}
\newcommand{\SET}{\textsc{Set}}
\newcommand{\MAN}{\textsc{Man}}
\newcommand{\GRUPPEN}{\textsc{Gruppen}}
\newcommand{\ABELGRUPPEN}{\textsc{Abel.Gruppen}}
\newcommand{\ABEL}{\textsc{Abel}}
\newcommand{\KAT}{\textsc{Kat}}
\newcommand{\FUN}{\textsc{Fun}}
\newcommand{\SIMP}{\textsc{Simp}}
\newcommand{\VEKT}{\textsc{Vekt}}
\newcommand{\CH}{\textsc{Ch}}
\newcommand{\CSTARUN}{C^*\text{-}\textsc{Alg}^{\raisemath{-2.5pt}{1}}}
\newcommand{\CSTAR}{C^*\text{-}\textsc{Alg}}
\newcommand{\AB}{\textsc{Ab}}
% ======================================================================================
% ======================================================================================



% -- theorem packages
% ======================================================================================
\usepackage{amsthm}
\usepackage{thmtools,thm-restate}
\usepackage{mdframed}
\renewcommand{\listtheoremname}{Übersicht aller Aussagen}
\usepackage{bookmark}
\bookmarksetup{open,numbered}
\makeatletter
\newcommand*{\theorembookmark}{%
  \bookmark[
    dest=\@currentHref,
    rellevel=1,
    keeplevel,
  ]{%
    \thmt@thmname\space\csname the\thmt@envname\endcsname
    \ifx\thmt@shortoptarg\@empty
    \else
      \space(\thmt@shortoptarg)%
    \fi
  }%
}   
\makeatother
% ======================================================================================

% -- Definition der einzelnen Theorem-Umgebungen
% ======================================================================================
\declaretheoremstyle[%
	headfont=\sffamily\bfseries,
	notefont=\normalfont\sffamily\scshape,
	bodyfont=\normalfont,
	headformat=\NUMBER\ \NAME\NOTE,
	headpunct=.,
	postheadspace=1em,
	spaceabove=15pt,spacebelow=10pt,
	shaded={bgcolor=gray!20},
	postheadhook=\theorembookmark]%
{mainstyle}
\declaretheoremstyle[%
	headfont=\sffamily\bfseries,
	notefont=\normalfont\sffamily\scshape,
	bodyfont=\normalfont,
	headformat=\NUMBER\ \NAME\NOTE,
	headpunct=.,
	postheadspace=1em,
	spaceabove=15pt,spacebelow=10pt,
	shaded={bgcolor=fb10_blue!20},
	postheadhook=\theorembookmark]%
{mainstyle_blue}
\declaretheoremstyle[%
	headfont=\sffamily\bfseries,
	notefont=\normalfont\sffamily\scshape,
	bodyfont=\normalfont,
	headformat=\NUMBER\ \NAME\NOTE,
	headpunct=.,
	postheadspace=1em,
	spaceabove=15pt,spacebelow=10pt,
	postheadhook=\theorembookmark]%
{mainstyle_unshaded}
\declaretheoremstyle[%
	headfont=\sffamily\bfseries,
	notefont=\normalfont\sffamily\scshape,
	bodyfont=\normalfont,
	headformat=\NUMBER\NAME\NOTE,
	headpunct=.,
	postheadspace=1em,
	spaceabove=15pt,spacebelow=10pt,
	% shaded={bgcolor=gray!20},
	postheadhook=\theorembookmark]%
{mainstyle_unnumbered}
\declaretheoremstyle[%
	headfont=\sffamily\bfseries,
	notefont=\normalfont\sffamily\scshape,
	bodyfont=\normalfont,
	headformat=swapnumber,
	headpunct=.,
	postheadspace=1em,
	spaceabove=15pt,spacebelow=10pt,
	shaded={bgcolor=gray!20},
	postheadhook=\theorembookmark,
	qed=\qedsymbol]%
{mainstyleB}
\declaretheoremstyle[%
	headfont=\bfseries\scshape,
	bodyfont=\normalfont,
	headpunct=:,
	postheadspace=1em,
	spacebelow=12pt,spaceabove=2pt,
	qed=\qedsymbol]%
{beweise}
\declaretheoremstyle[%
	headfont=\bfseries\scshape,
	bodyfont=\normalfont,
	headpunct=:,
	postheadspace=1em,
	spacebelow=12pt,spaceabove=2pt]%
{beweisskizze}
\declaretheoremstyle[%
	headfont=\sffamily\bfseries,
	bodyfont=\normalfont,
	headpunct=:,
	postheadspace=1em,
	spacebelow=10pt,spaceabove=10pt]%
{bemerkungen}
\declaretheorem[name=Definition,parent=section,style=mainstyle_blue]{definition}
\declaretheorem[name=Definition \& Proposition,refname=Proposition,sharenumber=definition,style=mainstyle_blue]{definitionP}
\declaretheorem[name=Definition,numbered=no,style=mainstyle_unnumbered]{definition*}
\declaretheorem[name=Theorem,sharenumber=definition,style=mainstyle]{theorem}
\declaretheorem[name=Theorem,numbered=no,style=mainstyle_unnumbered]{theorem*}
\declaretheorem[name=Proposition,sharenumber=definition,style=mainstyle,refname=Proposition]{proposition}
\declaretheorem[name=Lemma,sharenumber=definition,style=mainstyle]{lemma}
\declaretheorem[name=Satz,sharenumber=definition,style=mainstyle,refname=Satz]{satz}
\declaretheorem[name=Satz,sharenumber=definition,style=mainstyle_unshaded]{satzUnshaded}
\declaretheorem[name=Definition,sharenumber=definition,style=mainstyle_unshaded]{definitionUnshaded}
\declaretheorem[name=Satz,numbered=no,style=mainstyle_unnumbered]{satz*}
\declaretheorem[name=Korollar,sharenumber=definition,style=mainstyle,refname=Korollar]{korollar}
\declaretheorem[name=Korollar,sharenumber=definition,style=mainstyleB,refname=Korollar]{korollarB}
\declaretheorem[name=Frage,numbered=no,style=mainstyle_unnumbered]{frage}
\declaretheorem[name=Frage,sharenumber=definition,style=mainstyle_unshaded]{frageA}
\declaretheorem[name=Erinnerung,sharenumber=definition,style=mainstyle_unshaded]{erinnerungA}
\declaretheorem[name=Ausblick,sharenumber=definition,style=mainstyle_unshaded]{ausblick}
\declaretheorem[name=Konvention,sharenumber=definition,style=mainstyle]{konvention}
\declaretheorem[name=Notation,sharenumber=definition,style=mainstyle_unshaded]{notation}
\declaretheorem[name=Bemerkung,sharenumber=definition,style=mainstyle_unshaded,refname=Bemerkung]{bemerkung}
\declaretheorem[name=Bemerkung,numbered=no,style=mainstyle_unnumbered]{bemerkung*}
\declaretheorem[name=Beispiel,sharenumber=definition,style=mainstyle_unshaded,refname=Beispiel]{beispiel}
\declaretheorem[name=Beispiel,numbered=no,style=mainstyle_unnumbered]{beispiel*}
\declaretheorem[name=Exkurs,numbered=no,style=mainstyle_unnumbered]{exkurs*}
\declaretheorem[name=Beweis,numbered=no,style=beweise]{beweis}
\declaretheorem[name=Übung,numbered=no,style=bemerkungen]{uebung}
\declaretheorem[name=Erinnerung,numbered=no,style=bemerkungen]{erinnerung}

% english versions
\declaretheorem[name=Remark,sharenumber=definition,style=mainstyle_unshaded]{remark}
\declaretheorem[name=Remark,numbered=no,style=mainstyle_unnumbered]{remark*}
\declaretheorem[name=Example,sharenumber=definition,style=mainstyle_unshaded]{example}
\declaretheorem[name=Corollary,sharenumber=definition,style=mainstyle]{corollary}
\let\proof\relax
\declaretheorem[name=Proof,numbered=no,style=beweise]{proof}
\declaretheorem[name=Sketch of Proof,numbered=no,style=beweisskizze]{sketch}
% ======================================================================================

%--Inhaltsverzeichnis
% ======================================================================================
\usepackage[tocindentauto]{tocstyle}
\usetocstyle{KOMAlike}
% ======================================================================================

%-- Dinge, die erst am Ende getan werden dürfen
% ======================================================================================
\shorthandon{"}
\usepackage{ellipsis}
% ======================================================================================


\newcommand{\fach}{Liegruppen}
\newcommand{\semester}{Sose 2016}
\newcommand{\homepage}{http://wwwmath.uni-muenster.de/42/arbeitsgruppen/ag-differentialgeometrie/prof-dr-christoph-boehm/vorlesung-liegruppen/}

\newcommand{\prof}{Prof.\ Dr.\ Christoph Böhm}
\publishers{\scalebox{10}{\Huge$\mathrm{Lie}$}}
\input{../!config/mitschrift_headings.tex}

\begin{document}
\pagenumbering{Roman}
\maketitle
\begin{abstract}
\section*{Aktuelle Version verfügbar bei}
\newcommand{\dieBreite}{11cm}
\begin{minipage}{4cm}
	\qrcode[height=3.3cm, version=6]{https://gitlab.com/JaMeZ-B/LaTeX-WWU}
\end{minipage}
\hfill
\begin{minipage}{\dieBreite}
	% \includegraphics[height=0.6cm, keepaspectratio]{../!config/Bilder/wm_no_bg.pdf}
	\includegraphics[height=0.8cm, keepaspectratio]{../!config/Bilder/wm_no_bg.pdf}\\
	\url{https://gitlab.com/JaMeZ-B/LaTeX-WWU} \smallskip\\
	Das zentrale Repository des \enquote{\LaTeX-WWU}-Projekts befindet sich auf der Plattform GitLab.com.
	Neben der Koordination aller Beteiligten werden über diesen Dienst auch die PDFs gebaut, die in der Readme verlinkt sind.
\end{minipage}\\[1cm]
\begin{minipage}{4cm}
	\qrcode[height=3.3cm, version=6]{https://github.com/JaMeZ-B/latex-wwu}
\end{minipage}
\hfill
\begin{minipage}{\dieBreite}
	\includegraphics[height=0.6cm, keepaspectratio]{../!config/Bilder/github_octo.pdf}
	\includegraphics[height=0.6cm, keepaspectratio]{../!config/Bilder/GitHub_Logo.pdf}\\
	\url{https://github.com/JaMeZ-B/latex-wwu} \smallskip\\
	Die Entwicklung des \enquote{\LaTeX-WWU}-Projekts hat ursprünglich auf GitHub stattgefunden, ist mittlerweile aber zu GitLab gewechselt.
	Das GitHub-Repository wird stündlich automatisch aktualisiert, Merge-Requests werden aber nicht mehr entgegengenommen.
\end{minipage}\\[1cm]
% \begin{minipage}{4cm}
% 	\qrcode[height=3.3cm, version=6]{https://uni-muenster.sciebo.de/public.php?service=files&t=965ae79080a473eb5b6d927d7d8b0462}
% \end{minipage}
% \hfill
% \begin{minipage}{\dieBreite}
% 	\raisebox{-2pt}{\includegraphics[height=0.6cm, keepaspectratio]{../!config/Bilder/sciebo_logo.pdf}}
% 	\resizebox{!}{0.5cm}{\large \sffamily\textbf{sciebo}} {\sffamily\large die Campuscloud} \\
% 	\resizebox{\dieBreite}{!}{\footnotesize\url{https://uni-muenster.sciebo.de/public.php?service=files&t=965ae79080a473eb5b6d927d7d8b0462}}\smallskip\\
% 	Sciebo ist ein Dropbox-Ersatz der Hochschulen in NRW, der von der Uni Münster in leitender Position auf Basis der OpenSource-Software Owncloud aufgebaut wurde.
% \end{minipage}\\[1cm]
\hrule \mbox{ }\\[0.7cm]
\begin{minipage}{4cm}
	\qrcode[height=3.3cm, version=6]{\homepage}
\end{minipage}
\hfill
\begin{minipage}{\dieBreite}
	\resizebox{!}{0.5cm}{\large\sffamily\textbf{Vorlesungshomepage}}\\
	\resizebox{\dieBreite}{!}{\footnotesize\url{\homepage}}\smallskip\\
	Hier ist ein Link zur offiziellen Vorlesungshomepage.
\end{minipage}
\newpage
\section*{Vorwort --- Mitarbeit am Skript}
Dieses Dokument ist eine Mitschrift aus der Vorlesung \enquote{\fach, \semester}, gelesen von \prof. 
Der Inhalt entspricht weitestgehend dem Tafelanschrieb. 
Für die Korrektheit des Inhalts übernehme ich keinerlei Garantie! 
Für Bemerkungen und Korrekturen -- und seien es nur Rechtschreibfehler -- bin ich sehr dankbar. 
Korrekturen lassen sich prinzipiell auf drei Wegen einreichen: 
\begin{itemize}
	\item Persönliches Ansprechen in der Uni, Mails an \hrefsymmail{mailto:\mail}{\mail} (gerne auch mit annotieren PDFs) oder Kommentare auf \url{https://gitlab.com/JaMeZ-B/LaTeX-WWU}.
	\item \emph{Direktes} Mitarbeiten am Skript: Den Quellcode poste ich auf GitLab (siehe oben), also stehen vielfältige Möglichkeiten der Zusammenarbeit zur Verfügung:
	Zum Beispiel durch Kommentare am Code über die Website und die Kombination Fork und Merge-Request. 
	Wer sich verdient macht oder ein Skript zu einer Vorlesung, die ich nicht besuche, beisteuern will, dem gewähre ich gerne auch Schreibzugriff.
	
	Beachten sollte man dabei, dass dazu ein Account bei \url{gitlab.com} notwendig ist, der allerdings ohne Angabe von persönlichen Daten angelegt werden kann. 
	Wer bei GitLab (bzw. dem zugrunde liegenden Open-Source-Programm \enquote{\texttt{git}}) -- verständlicherweise -- Hilfe beim Einstieg braucht, dem helfe ich gerne weiter. 
	Es gibt aber auch zahlreiche empfehlenswerte Tutorials im Internet.\footnote{zB. \url{https://try.github.io/levels/1/challenges/1}, ist auf Englisch, aber dafür interaktiv}
	\item \emph{Indirektes} Mitarbeiten: \TeX-Dateien per Mail verschicken. 
	
	Dies ist nur dann sinnvoll, wenn man einen ganzen Abschnitt ändern möchte (zB. einen alternativen Beweis geben), da ich die Änderungen dann per Hand einbauen muss! Ich freue mich aber auch über solche Beiträge!
\end{itemize}
\section*{Literatur}
\begin{itemize}
	\item \citetitle{Ziller} von Wolfgang \citeauthor{Ziller} \cite{Ziller}
	\item \citetitle{Berndt} von Jürgen \citeauthor{Berndt} \cite{Berndt} 
\end{itemize}
\end{abstract}

\tableofcontents
\cleardoubleoddemptypage

\pagenumbering{arabic}
\setcounter{page}{1}
\setcounter{footnote}{0}

\chapter{Grundlagen} % (fold)
\label{cha:grundlagen}

\section{Liegruppen und Liealgebren} % (fold)
\label{sec:1}
\begin{definition}[label=def:111,{name=[Liegruppe]}]
	Eine \Index{Liegruppe} st eine abstrakte Gruppe, welche zusätzlich eine $n$-dimensionale, differenzierbare Mannigfaltigkeit ist, sodass die beiden Abbildungen 
	$m \colon G \times G \to G$, $(g,h) \mapsto g \cdot h$ und $^{-1} \colon G \to G$, $g \mapsto g^{-1}$ differenzierbar sind.
\end{definition}

\begin{beispiel*}[{name=[{Liegruppen}]}]
	Es gibt zahlreiche Liegruppen: 
	\begin{itemize}
		\item Die $1$-Sphäre $S^1 = \set*{e^{i \varphi} \in \mathbb{C} \given \varphi \in \mathbb{R}} = \set[\big]{z \in \mathbb{C} \given \abs*{z}=1} \simeq \set*{\begin{psmallmatrix*}[r]
			\cos \varphi & - \sin \varphi \\
			\sin \varphi & \cos \varphi
		\end{psmallmatrix*} \given \varphi \in \mathbb{R}} = \SO_\mathbb{R}(2)$
		\item Der $n$-Torus $T^n = S^1 \times \ldots \times S^1 $ ($n$-faches karthesisches Produkt)
		\item Sei $r \in \mathbb{R}$ und $S^1_r := \set*{\enbrace{e^{i \varphi}, e^{i r \varphi}} \in S^1 \times S^1 \given \varphi \in \mathbb{R}}$.
		Für $r \in \mathbb{Q}$ ist $S^1_r$ eine $1$-dimensionale Untermannigfaltigkeit des Torus $T^2= S^1 \times S^1$.
		Für $r \in \mathbb{R} \setminus \mathbb{Q}$ ist $S^1_r$ \emph{keine} eingebettete Untermannigfaltigkeit, denn es gilt 
		\[
			\overline{S^1_r} = T^2
		\]
		\emph{Diese Gruppe wird in Aufgabe 8 der Übungen genauer behandelt.}\todo{in den Anhang?}
		\item Die $3$-Sphäre $S^3 = \set[\big]{q \in \mathbb{H} \given \abs*{q}=1}$, wobei $\mathbb{H}$ der Schiefkörper der \Index{Quaternionen}\footnote{siehe \url{https://de.wikipedia.org/wiki/Quaternion}} ist.
	
		Wir setzen $\mathbb{R}^3 := \im \mathbb{H} = \Span_\mathbb{R} \set{i,j,k}$.
		Dann ist $\mathbb{R}^3$ invariant unter der adjungierten Darstellung von $S^3$, das heißt für alle $x \in \im \mathbb{H}$ und alle $q \in S^3$ gilt:
		\[
			\Ad(q)(x) := q \cdot x \cdot \Underbrace{q^{-1}}{=\overline{q}} \stackrel{!}{\in} \mathbb{R}^3 \qquad \text{ für } x \in \mathbb{R}^3 
		\]
		Man kann ferner zeigen, dass $\Ad(q)|_{\mathbb{R}^3} \subset SO(3)$.
		Es gilt sogar $\Ad(S^3) = SO(3)$ und $\pi \colon S^3 \to SO(3)$, $q \mapsto \Ad(q)|_{\mathbb{R}^3}$ ist eine $2$-fache Überlagerung.
		Dies folgt mit $\Ad(q)=\Ad(-q)$.
		
		\emph{In Übungsaufgabe 3 werden diese Behauptungen bewiesen.}\todo{in den Anhang?}
		\item \Index{Orthogonale Gruppe} $\On(n) = \set*{A \in \Mat(n,\mathbb{R}) \given A A^T = E_n}$, $\SO(n) = \set[\big]{A \in \On(n) \given \det A=1}$
		\item \Index{Unitäre Gruppe} $\Un(n) = \set[\big]{A \in \Mat (n,\mathbb{C}) \given A A^* =E_n}$, $\SU(n) = \set[\big]{A \in \Un(n) \given \det A=1}$
		\item \Index{Allgemeine lineare Gruppe} $\GL(n,\mathbb{K}) = \set[\big]{A \in \Mat(n,\mathbb{K}) \given A \text{ invertierbar}}$ für $\mathbb{K} \in  \set*{\mathbb{R},\mathbb{C},\mathbb{H}}$ und die \Index{Spezielle lineare Gruppe}
		$\SL(n,\mathbb{K}) = \set[\big]{A \in \GL(n,\mathbb{K}) \given \det A =1}$
		\item Sind $G_1$ und $G_2$ Liegruppen, so ist auch $G_1 \times G_2$ eine Liegruppe.
		\item Ist $G$ eine Liegruppe, so ist\marginnote{die Abgeschlossenheit bzgl. der Gruppenoperationen folgt aus der Stetigkeit selbiger}
		\[
			G_0 :=  \set[\big]{g \in G \given \exists c \colon [0,1] \to G \text{ stetig mit } c(0)=e, c(1)=g}
		\]
		ein Liegruppe, die Zusammenhangskomponente der Eins.
		Beispiel: $\On(n)=\SO(n) \mathbin{\dot{\cup}} I \cdot \SO(n)$, wobei $I$ eine Spiegelung ist.
		Tatsächlich hat $\On(n)$ genau zwei Zusammenhangskomponenten; im Fall $n=2$:
		\[
			\On(2) = \SO(2) \mathbin{\dot{\cup}} \set*{\begin{pmatrix*}[r]
				\cos \varphi & \sin \varphi \\
				\sin \varphi & - \cos \varphi
			\end{pmatrix*} \given \varphi \in \mathbb{R}}
		\]
	\end{itemize}
\end{beispiel*}

\begin{definition}[{name=[Liealgebra und Lieklammer]}]
	Eine \Index{Liealgebra} über $\mathbb{R}$ oder $\mathbb{C}$ ist ein $\mathbb{K}$-Vektorraum $V$ zusammen mit einer bilinearen, schiefsymmetrischen Abbildung 
	\(
		[\cdot ,\cdot ] \colon V \times V \to V
	\),
	welche die Jacobi-Identität 
	\[
		\benbrace[\big]{X, \benbrace*{Y,Z}} + \benbrace[\big]{Z, \benbrace*{X,Y}} + \benbrace[\big]{Y, \benbrace*{Z,X}} =0
	\]
	für alle $X,Y,Z \in V$ erfüllt.
	Man nennt $\benbrace*{\cdot ,\cdot }$ eine \Index{Lieklammer}. 
\end{definition}

Wir werden zeigen, dass für eine Liegruppe $G$ der Tangentialraum am neutralen Element $\Tmap_e G$ auf natürliche Art und Weise eine Liealgebra ist.
Dazu ein wenig Notation: Für $g \in G$ betrachten wir \bet{Links}- und \Index{Rechtsmultiplikation}\index{Linksmultiplikation}
\begin{align}
	\SwapAboveDisplaySkip
	L_g \colon G \longrightarrow G \quad , \qquad &h \longmapsto g \cdot h \\
	R_g \colon G \longrightarrow G \quad , \qquad &h \longmapsto h \cdot g
\end{align}
$L_g$ und $R_g$ sind nach Definition einer Liegruppe offensichtlich Diffeomorphismen.

\begin{definition}[{name=[linksinvariantes Vektorfeld]}]
	Ein Vektorfeld $X$ auf $G$ nennt man \Index{linksinvariant}, falls für alle $g,h \in G$ gilt
	\[
		X(g \cdot h) = \mathd(L_g)_h \cdot X(h)
	\]
\end{definition}

Offenbar ist ein solches Vektorfeld $X$ schon eindeutig durch seinen Wert $X(e)$ bestimmt, denn 
\[
	X(g) = X(g \cdot e) = \mathd(L_g)_e \cdot X(e)
\]
Wir schreiben daher auch $X=X_v$ mit $X(e)=v \in \Tmap_e G$.

\begin{lemma}[label=lem:114,{name=[Linksinvariante Vektorfelder sind differenzierbar]}]
	Linksinvariante Vektorfelder sind differenzierbar.
\end{lemma}
\begin{beweis}
	Nach \autoref{def:111} ist die Multiplikation $m \colon G \times G \to G$ differenzierbar, somit auch $\mathd m \colon \Tmap G \times \Tmap G \to\Tmap G$ und damit für alle $v \in \Tmap_e G$ auch $\tilde{X}_v \colon G \to \Tmap G$, $g \mapsto (\mathd m)_{(g,e)}(0,v)$.
	Wir zeigen nun $\tilde{X}_v=X_v$.
	Sei dazu $\enbrace[\big]{g(t)}_{t \in (-1,1)}$ eine differenzierbare Kurve in $G$ mit $g(0)=e$ und $g'(0)=v$.
	Dann gilt 
	\[
		\tilde{X}_v(g)=(\mathd m)_{(g,e)}(0,v) = \diffd{}{t}\Big|_{t=0} m \enbrace[\big]{g,g(t)} = \diffd{}{t}\Big|_{t=0} L_g\enbrace[\big]{g(t)} = \mathd(L_g)_{g(0)} \cdot g'(0)
		 =\mathd(L_g)_e v = X_v(g) \qedhere
	\]
\end{beweis}

Wir bezeichnen mit $\mathfrak{g}$ (\enquote{g} in Frakturschrift) die Menge der linksinvarianten Vektorfelder auf $G$.

\begin{korollar}[{name=[{Isomorphismus linkinv. Vektorfelder und Tangentialraum an $e$}]}]
	Die Abbildung $L \colon \mathfrak{g} \to \Tmap_e G, X \mapsto X(e)$ ist ein linearer Isomorphismus.
\end{korollar}
\begin{beweis}
	Die Injektivität und Linearität sind klar.
	Für die Surjektivität sei $v \in \Tmap_e G$ ein Tangentialvektor.
	Nach \autoref{lem:114} ist $X_v$ ein glattes Vektorfeld auf $G$. 
	Es ist außerdem linksinvariant, da
	\[
		X_v(g \cdot h) = \mathd(L_{g \cdot h})_e v = \mathd(L_g)_h \cdot (\mathd L_h)_e \cdot v = \mathd (L_g)_h \cdot X_v(h)
	\]
	Damit ist die Surjektivität gezeigt.
\end{beweis}

Ein Vektorfeld $X$ ist linksinvariant genau dann, wenn $X$ für alle $g \in G$ $L_g$-verwandt zu sich selbst ist, das heißt
\[
	X \enbrace[\big]{L_g(h)} = (\mathd L_g)_h X(h) \qquad  \forall h \in G
\]
Nach Differentialgeometrie \RM{1}. ist die Lieklammer von $L_g$-verwandten Vektorfeldern auch wieder $L_g$-verwandt, also ist die Lieklammer linksinvarianter Vektorfelder wieder ein linksinvariantes Vektorfeld.
Dies führt uns zu folgender Definition:

\begin{definition}[{name=[Liealgebra einer Liegruppe]}]
	Wir bezeichnen mit\marginnote{wir übertragen also die Liealgebra-Struktur von $\mathfrak{g}$ auf $\Tmap_e G$}
	\[
		\benbrace*{\cdot ,\cdot } \colon \Tmap_e G \times \Tmap_e G \to \Tmap_e G \quad ,\quad  (v,w) \longmapsto \benbrace*{X_v,X_w}(e)
	\] 
	die \Index{Lieklammer} von $\Tmap_e G = \mathfrak{g}$.
	Da die Lieklammer differenzierbarer Vektorfelder die Jacobi-Identität erfüllt, ist $\benbrace*{\cdot ,\cdot }$ eine Lieklammer auf $\Tmap_e G$.
\end{definition}

\begin{beispiel*}[{name=[{Lieklammer für allgemeine lineare Gruppe}]}]
	Wir wollen die Lieklammer der Liegruppe $G=\GL(n,\mathbb{K})$ mit $\mathbb{K}=\mathbb{R}$ ausrechnen (für $\mathbb{K}=\mathbb{C}$ analog):
	
	Da die Determinante $\det \colon G \to \mathbb{K}$ stetig ist, ist 
	\(
		{\det}^{-1} \enbrace[\big]{\mathbb{K}\setminus \set*{0}} = \GL(n,\mathbb{K}) 
	\)
	eine offene Teilmenge des euklidischen Vektorraums $\Mat(n,\mathbb{K})$ und somit eine differenzierbare Mannigfaltigkeit.
	Wir bezeichnen mit $x_{ij} \colon G \to \mathbb{K}$ die Abbildung $(a_{kl})_{k,l} \mapsto a_{ij}$ für festes $(i,j)$ mit $1\le i,j\le n$.
	Da $x_{ij}$ linear ist, gilt für $A \in G$
	\[
		(X_v)(x_{ij})(A) = (\mathd x_{ij})_A X_v(A) = x_{ij}(A \cdot v) = (A v)_{ij}
	\]
	Somit ist
	\begin{align}
		\benbrace[\big]{X_v,X_w}(x_{ij})(e) = X_v(e) \enbrace[\big]{X_w(x_{ij})} - X_w(e) \enbrace[\big]{X_v(x_{ij})} &= v \enbrace[\Big]{\Underbrace{X_w(x_{ij})}{\mathclap{A \mapsto (A \cdot w)_{ij}}}} - w \enbrace[\big]{X_v(x_{ij})} \\
		&= \diffd{}{t} \enbrace[\big]{(ew + t \cdot vw)_{ij} - (ev + t \cdot wv)_{ij}} \Big|_{t=0} \\
		&= (v w- w v)_{ij}
	\end{align}
	Für ein beliebiges differenzierbares Vektorfeld $X$ und lokale Koordinaten $(x_1, \ldots ,x_n)=x$ gilt in einem Kartengebiet $U$
	\[
		X = \sum_{i=1}^{n} X(x_i) \cdot \diff{}{x_i}
	\]
	Somit gilt $\benbrace*{X_v,X_w} =X(vw -wv)$ und daher $\benbrace*{v,w}_{\mathfrak{gl}(n,\mathbb{R})}=vw -wv$.
	Die Lieklammer auf der Liealgebra von $\GL_n(\mathbb{K})$ ist also durch den \Index{Matrix-Kommutator gegeben}.
\end{beispiel*}
% subsection 11 (end)

\section{Lieuntergruppen und Homomorphismen von Liegruppen} % (fold)
\label{sec:12}

\begin{definition}[{name=[Liegruppen- und Liealgebrenhomomorphismen]}]
	Seien $H,G$ Liegruppen.
	\begin{enumerate}[1)]
		\item Eine Abbildung $\Phi \colon H \to G$ nennt man \Index{Liegruppenhomomorphismus}, falls $\Phi$ differenzierbar und ein Gruppenhomomorphismus ist.
		\item Einen Liegruppenhomomorphismus $\Phi \colon H \to G$ nennt man \Index{Liegruppenisomorphismus}, falls $\Phi$ bijektiv und $\Phi^{-1}$ differenzierbar ist. ($\Phi^{-1}$ ist auch Gruppenhomomorphismus)
		\item \bet{Liealgebrenhomomorphismen}\index{Liealgebrenhomomorphismus} und Liealgebrenisomorphismen werden entsprechend definiert:  $L \colon \mathfrak{h} \to \mathfrak{g}$ linear heißt \Index{Liealgebrahomomorphismus}, falls für alle $v,w \in \mathfrak{h}$ gilt:
		\[
			L \benbrace[\big]{v,w}_{\mathfrak{h}} = \benbrace[\big]{L v, L w}_\mathfrak{g}
		\]
	\end{enumerate}
\end{definition}

Ist $\Phi \colon H \to G$ ein Liegruppenhomomorphismus, so kommutiert das folgende Diagramm
\[
	\begin{tikzcd}
		H \rar["\Phi"] \dar["L_h"] & G \dar["L_{\Phi(h)}"] \\
		H \rar["\Phi"] & G
	\end{tikzcd}
\]
denn es gilt
\begin{align}
	(\Phi \circ L_h)(\tilde{h}) = \Phi \enbrace*{L_h(\tilde{h})} = \Phi(h \cdot \tilde{h}) = \Phi(h) \cdot \Phi(\tilde{h}) = \enbrace*{L_{\Phi(h)} \circ \Phi}(\tilde{h})
\end{align}
Einen Homomorphismus $\Phi \colon G \to \GL(n,\mathbb{K})$ nennt man \bet{reelle}\index{Darstellung!reelle} bzw. \bet{komplexe Darstellung}\index{Darstellung!komplexe} von $G$.\marginnote{$\Phi_0(G)\equiv E_n$}
Ist $\Phi \colon G \to \GL(n,\mathbb{K})$ injektiv, so besitzt die Gruppe $G$ die \Index{treue Darstellung} $\Phi(G)$ in $\GL(n,\mathbb{K})$.

\begin{lemma}[label=lem:122,{name=[{Differntial von Liegruppenhom. ist Liealgebrenhom.}]}]
	Sei $\Phi \colon H \to G$ ein Liegruppenhomomorphismus.\marginnote{es folgt, dass man die Liealgebra einer Liegruppe als Funktor auffassen kann}
	Dann ist $(\mathd\Phi)_e \colon \Tmap_e H \to \Tmap_e G$ ein Liealgebrenhomomorphismus.
\end{lemma}
\begin{beweis}
	Seien $v_1,v_2 \in \Tmap_e H$ und $w_i := (\mathd \Phi)_e \, v_i$ für $i=1,2$.
	Zu zeigen
	\[
		(\mathd \Phi)_e \benbrace[\big]{X_{v_1}, X_{v_2}} (e) = \benbrace[\big]{X_{w_1},X_{w_2}}(e)
	\]
	Wir zeigen, dass $X_{v_1}$ und $X_{w_1}$ $\Phi$-verwandt sind:
	\begin{align}
		(\mathd \Phi)_h X_{v_1}(h) = \enbrace*{\mathd \Phi}_h (\mathd L_h)_e \cdot v_1 = \mathd \enbrace*{\Phi \circ L_h}_e \cdot v_1 
		&= \mathd \enbrace*{L_{\Phi(h)} \circ \Phi}_e \cdot v_1 \\
		&= \mathd \enbrace*{L_{\Phi(h)}}_e \cdot \enbrace*{\mathd \Phi}_e \cdot v_1 \\
		&= \mathd \enbrace*{L_{\Phi(h)}}_e \cdot w_1 \\
		&= X_{w_1}(\Phi(h))
	\end{align}
	Somit sind $X_{v_1}$ und $X_{w_1}$ $\Phi$-verwandt und analog auch $X_{v_2}$ und $X_{w_2}$.
	Nach Differentialgeometrie \RM{1}. sind dann auch $\benbrace*{X_{v_1}, X_{v_2}}$ und $\benbrace*{X_{w_1},X_{w_2}}$ $\Phi$-verwandt, das heißt es gilt
	\[
		(\mathd \Phi)_h \benbrace[\big]{X_{v_1},X_{v_2}}(h) = \benbrace[\big]{X_{w_1}, X_{w_2}} \enbrace*{\Phi(h)}
	\]
	Mit $h=e$ folgt nun die Behauptung.
\end{beweis}

\begin{beispiel*}[{name=[Konjugation und adjungierte Darstellung]}]
	Sei $G$ eine Liegruppe und $i_g \colon G \to G$ gegeben durch $h \mapsto g \cdot h \cdot g^{-1}$ die \Index{Konjugation} mit $g$.
	Dann ist $i_g$ ein Liegruppenisomorphismus:
	\begin{itemize}[itemsep=0pt]
		\item $i_g(h) = \enbrace*{l_g \circ R_{g^{-1}}}(h)$, also ist $i_g$ differenzierbar.
		\item Die Gruppenhomomorphismuseigenschaft ist klar. 
		\item Die inverese Abbildung ist offenbar gegeben durch $(i_g)^{-1}= i_{g^{-1}}$.
	\end{itemize}
	Somit ist $\Ad(g) := \enbrace*{\mathd i_g}_e \colon \mathfrak{g} \to \mathfrak{g}$ ein Liealgebrenisomorphismus.
	
	Ist $G = \GL(n,\mathbb{K})$, so kann man $\Ad(g)$ einfach berechnen:
	Sei $(h(t))_{t \in (-\varepsilon,\varepsilon)}$ eine differenzierbare Kurve in $G$ mit $h(0)= E_n$ und $h'(0)=v$.
	Dann gilt
	\begin{align}
		\Ad(g)(v)= (\mathd i_g)_e \cdot h'(0) = \diffd{}{t}\Big|_{t=0} i_g \enbrace[\big]{h(t)} = \diffd{}{t}\Big|_{t=0} g \cdot h(t) \cdot g^{-1}
		&= g \cdot \diffd{}{t}\Big|_{t=0} h(t) \cdot g^{-1}\\
		&= g \cdot v \cdot g^{-1}
	\end{align}
	Ferner ist für eine beliebige Liegruppe $G$ die Abbildung $\Ad \colon G \to \GL\enbrace*{\Tmap_e G}$, $g \mapsto \Ad(g)$ ein Liegruppenhomomorphismus:
	\begin{align}
		\Ad(g_1 \cdot g_2) = (\mathd i_{g_1 \cdot g_2})(e) = (\mathd i_{g_1})_e \cdot (\mathd i_{g_2})_e = \Ad(g_1) \circ \Ad(g_2)
	\end{align}
	$\Ad \colon G \to \GL(\mathfrak{g})$ heißt die \Index{adjungierte Darstellung} von $G$.
\end{beispiel*}

\begin{definition}[{name=[{Lieuntergruppe und Lieunteralgebra}]}]
	Sei $G$ eine Liegruppe.
	\begin{enumerate}[1)]
		\item Mann nennt eine Teilmenge $H \subseteq G$ eine \Index{Lieuntergruppe}, falls 
		\begin{itemize}
			\item $H$ eine Untergruppe von $G$ ist,
			\item $H$ eine Liegruppe ist, wenn $H$ mit der von $G$ induzierten Multiplikation und Inversenbildung versehen wird und
			\item die Inklusionsabbildung $i \colon H \hookrightarrow G$ eine Immersion ist \marginnote{Immersion: injektives Differential}\\ 
			(nicht notwendigerweise eine Einbettung!).
		\end{itemize}
		\item Mann nennt einen Teilraum $\mathfrak{h} \subseteq \mathfrak{g}$ \Index{Lieunteralgebra}, falls $\benbrace*{X,Y}_\mathfrak{g} \in \mathfrak{h}$ für alle $X,Y \in \mathfrak{h}$ gilt.
	\end{enumerate}
\end{definition}

\begin{beispiel*}[{name=[Lieuntergruppen]}]
	\leavevmode
	\begin{itemize}
		\item Die Gruppe $S^1_r = \set*{\enbrace*{e^{i \varphi}, e^{i r \varphi}} \given \varphi \in \mathbb{R}}$ ist eine Lieuntergruppe von $T^2=S^1 \times S^1$.
		Für $r \in \mathbb{Q}$ ist $S^1_r$ eine (eingebettete) Untermannigfaltigkeit; für $r \in \mathbb{R}\setminus \mathbb{Q}$ ist $S^1_r$ \enquote{nur} eine immersierte Untermannigfaltigkeit, jedoch nicht eingebettet.
		
		\emph{Diese Gruppe wird in Aufgabe 8 der Übungen genauer behandelt.}\todo{in den Anhang?}
		\item $\SO(n) \subset \GL(n,\mathbb{R})$ ist eine Lieuntergruppe mit Liealgebra 
		\[
			\mathfrak{so}(n) := \Tmap_e \SO(n) = \set*{X \in \Mat(n,\mathbb{R}) \given X^T=-X}
		\]
		Sei dazu $(A(t))_{t \in (-\varepsilon,\varepsilon)}$ eine differenzierbare Kurve in $\SO(n)$ mit $A(0)=E_n$ und $A'(0)=X$.
		Aus $A(t)^T \cdot A(t) \equiv E_n$ folgt durch Differenzieren $X^T + X =0$.
		Sei umgekehrt $X$ eine schiefsymmetrische Matrix.
		Setze dann 
		\[
			A(t) =\exp(t X) = \sum_{k=0}^{\infty} \frac{(tX)^k}{k!} 
		\]
		Behauptung: Es gilt $A(t) \in \SO(n)$.
		\begin{align}
			A(t)^T \cdot A(t) = \enbrace*{\sum_{k=0}^{\infty} \frac{(tX)^k}{k!} }^T \cdot \sum_{k=0}^{\infty} \frac{(tX)^k}{k!} \stackrel{X^T=-X}{=\joinrel=\joinrel=\joinrel=}\sum_{k=0}^{\infty} \frac{(-tX)^k}{k!}  \cdot \sum_{k=0}^{\infty} \frac{(tX)^k}{k!} 
			&= \exp(tX-tX) \\[-3pt]
			&= \exp(0) =E_n
		\end{align}
		Dies folgt mit $\exp(X+Y)= \exp(X) \exp(Y)$, falls $[X,Y]=0$.
	\end{itemize}
\end{beispiel*}

\begin{lemma}[{name=[Lieunteralgebra zu einer Lieuntergruppe]}]
	Sei $G$ eine Liegruppe und $H \subset G$ eine Lieuntergruppe.
	Dann ist $\mathfrak{h} = \Tmap_e H$ eine Lieunteralgebra von $\mathfrak{g} = \Tmap_e G$.
\end{lemma}
\begin{beweis}
	Sei $i \colon H \hookrightarrow G$ die Inklusionsabbildung.
	Dann ist $i$ ein Liegruppenhomomorphismus.
	Nach \autoref{lem:122} ist somit $(\mathd i)_e = \id_{\Tmap_e H} \colon \Tmap_e H \to \Tmap_e G$ ein Liealgebrenhomomorphismus, das heißt $\Tmap_e H = \mathfrak{h}$ ist eine Lieunteralgebra von $\mathfrak{g}$.
\end{beweis}

Wir erinnern daran, dass die Zusammenhangskomponente $G_0$ der \enquote{Eins} $e \in G$ einer Liegruppe wieder eine Liegruppe ist.
Beispiel: 
\[
	\On(n)= \SO(n) \mathbin{\dot{\cup}} \set*{ \begin{psmallmatrix}
		-1 & & &\\
		& 1 &  &\\
		& & \ddots & \\
		& & & 1
	\end{psmallmatrix} \cdot \SO(n)}
\]

\begin{satz}[label=satz:125,{name=[Lieuntergruppe zu Lieunteralgebra]}]
	Sei $G$ eine Liegruppe und $\mathfrak{h}$ eine Unteralgebra von $\mathfrak{g}$.
	Dann existiert genau eine zusammenhängende Lieuntergruppe $H$ von $G_0$ mit Liealgebra $\mathfrak{h}$. 
\end{satz}
\begin{beweis}
	Sei $g \in G$ und $\mathcal{F}_g := (\mathd L_g) \cdot \mathfrak{h} \subset \Tmap_g G$.
	Man nennt $\mathcal{F} = \set*{\mathcal{F}_g}_{g \in G}$ eine \Index{Distribution} von Unterräumen (gleicher Dimension!) auf $G$.
	Seien $v_1, \ldots ,v_r \in \Tmap_e G$ eine Basis von $\mathfrak{h}$.
	Wir bezeichnen mit $X_{v_1}, \ldots ,X_{v_r}$ die entsprechenden linksinvarianten Vektorfelder auf $G$.
	Es gilt
	\[
		\mathcal{F}_g = \Span_\mathbb{R} \enbrace[\big]{X_{v_1}(g), \ldots , X_{v_r}(g)}
	\]
	Die Distribution $\mathcal{F}$ ist \emph{involutiv}, das heißt es gilt $\benbrace*{\mathcal{F},\mathcal{F}} \subset \mathcal{F}$, denn nach Definition ist
	\[
		\benbrace[\big]{X_{v_i}, X_{v_j}} = X_{\benbrace*{v_i,v_j}_{\mathfrak{g}}} = X_w
	\]
	mit $w= \benbrace*{v_i,v_j} \in \mathfrak{h}$.
	Der \emph{Satz von Frobenius} \cite{LeeSmooth} besagt nun, dass durch jeden Punkt $g \in G$ eine zusammenhängende, maximale Integralmannigfaltigkeit $M^r_g$ geht, das heißt $\forall \tilde{g} \in M^r_g$ gilt $\Tmap_{\tilde{g}} M^r_g = \mathcal{F}_{\tilde{g}}$
	($M^r_g$ ist eine immersierte Untermannigfaltigkeit).
	Da die Distribution $\mathcal{F} = \set*{\mathcal{F}_g}_{g \in G}$ $G$-invariant ist, bildet der Diffeomorphismus $L_g \colon G \to G$ maximale Blätter (= Integralmannigfaltigkeiten) auf maximale Blätter ab.
	Wir setzen $H := M^r_e$ und wissen dann bereits, dass $H$ eine immersierte Untermannigfaltigkeit von $G$ ist.
	
	Zeige nun, dass $H$ eine Untergruppe ist:
	Da $H$ und $L_{h^{-1}}(H)$ beide $e$ enthalten und maximale Blätter sind, gilt $H=L_{h^{-1}}(H)$ für $h \in H$.
	Damit ist die Abgeschlossenheit unter Inversion gezeigt.
	Ähnlich zeigt man die Abgeschlossenheit unter Multiplikation.
	Da $m \colon G  \times G \to G$, $^{-1} \colon G \to G$ differenzierbare Abbildungen sind und die Einschränkung von differenzierbaren Abbildungen auf immersierte Untermannigfaltigkeiten wieder differenzierbare Abbildungen liefert, ist $H$ eine Lieuntergruppe von $G$.
	
	Es bleibt die Eindeutigkeit zu zeigen: 
	Sei also $\tilde{H} \subset L$ eine zusammenhängende Lieuntergruppe von $G$ mit $\Tmap_e \tilde{H} = \mathfrak{h} = \Tmap_e H$.
	Es gilt $\Tmap_{\tilde{h}} \tilde{H} = (\mathd L)_{\tilde{h}} \cdot \Tmap_e \tilde{H}= \mathcal{F}_{\tilde{h}}$ für alle $\tilde{h} \in \tilde{H}$.
	Somit ist $\tilde{H}$ Integralmannigfaltigkeit von $\mathcal{F}$ mit $e \in \tilde{H}$.
	Da $H$ maximale Integralmannigfaltigkeit ist, folgt $\tilde{H} \subseteq H$.
	Es bleibt die Gleichheit zu zeigen:
	$\tilde{H}$ ist offen in $H$.
	Annahme: Es gibt ein $h \in H \setminus \tilde{H}$ und eine Folge $(\tilde{h}_i)_{i \in \mathbb{N}}$ in $\tilde{H}$ mit $\tilde{h}_i \to h$.
	Dann ist $h \in L_h(U)$, wobei $U$ eine offene Umgebung von $e$ ist.
	Damit ist $h \in L_{h_i}(U)$ für $i \ge i_0$ für geeignetes $i_0 \in \mathbb{N}$.
	Dies ist ein Widerspruch, also folgt $\tilde{H}=H$.
\end{beweis}

\begin{korollar}[label=kor:126,{name=[gleiche induzierte Liealgebrenhom. implizieren Gleichheit der Liegruppenhom.]}]
	Seien $H,G$ zusammenhängende Liegruppen und $\Phi, \Psi \colon H \to G$ Gruppenhomomorphismen.
	Gilt $(\mathd \Phi)_e = (\mathd \Psi)_e \colon \Tmap_e H \to \Tmap_e G$, so folgt $\Phi = \Psi$.
\end{korollar}
\begin{beweis}
	$H \times G$ ist eine Liegruppe mit Liealgebra $\mathfrak{h} \oplus \mathfrak{g}= \Tmap_e H \oplus \Tmap_e G$ (komponentenweise).
	Dann ist eine differenzierbare Abbildung $\Phi \colon H \to G$ genau dann ein Liegruppenhomomorphismus, wenn der Graph von $\Phi$, also $\set[\big]{(h,\Phi(h)) \given h \in H}$ eine Untergruppe von $H \times G $ ist (!).
	Ist dies der Fall, so ist 
	\[
		\Graph \enbrace[\big]{(\mathd \Phi)_e} = \set[\big]{ \enbrace*{v, (\mathd \Phi)_e \cdot v} \given v \in \mathfrak{h}}
	\]
	eine Unteralgebra von $\mathfrak{h} \oplus \mathfrak{g}$.
	Wegen $\Graph \enbrace*{(\mathd \Phi)_e} = \Graph \enbrace*{(\mathd \Psi)_e}$ sind diese beiden Unteralgebren gleich und somit auch die entsprechenden eindeutigen zusammenhängenden Lieuntergruppen nach \autoref{satz:125}.
	Somit ist $\Phi =\Psi$.
\end{beweis}

Wir bezeichnen mit $c_v \colon I_{\max} \to G$, $t \mapsto c_v(t)$ die \Index{maximale Integralkurve} von $X_v$ mit $c_v(0)=e$ und $c_v'(0)=v$.
Es gilt dann $c_v'(t) = X_v \enbrace*{c_v(t)}$ für alle $t$.
Wir wissen weiter, dass $I_{\max}$ offen in $\mathbb{R}$ ist.

\begin{beispiel*}[{name=[{maximale Integralkurve in der allgemeinen linearen Gruppe}]}]
	Betrachte $\GL(n,\mathbb{R})$ und $v \in \Tmap_e G = \Mat(n,\mathbb{R})$.
	Wie gehabt ist $X_v(A) = (\mathd L_A)_e \cdot v = A \cdot v$ und es gilt $c_v(t)= \exp(t \cdot v) = \sum_{k=0}^{\infty} \frac{(tv)^k}{k!}$, denn
	\begin{align}
		\diffd{}{t}\Big|_{t=t_0} c_v(t) = \diffd{}{t}\Big|_{t=t_0} \Underbrace{\exp(t v)}{=\exp(t_0 v) \cdot \exp((t-t_0) \cdot v)} 
		&= \exp(t_0 v) \cdot \diffd{}{s}\Big|_{s=0} \exp(s v) \\[-1em]
		&= \exp(t_0 v) \cdot v \\
		&= c_v(t_0) \cdot v = X_v \enbrace[\big]{c_v(t_0)}
	\end{align}
\end{beispiel*}


\begin{lemma}[label=lem:127,{name=[Integralkurven sind Einparameteruntergruppen von $G$]}]
	Sei $G$ eine Liegruppe.
	Die Integralkurve $c_v$ ist auf ganz $\mathbb{R}$ definiert und es gilt 
	\[
		c_v(t+s) = c_v(t) \cdot c_v(s)
	\]
\end{lemma}
\begin{beweis}
	Setze $\tilde{c}(t) := c_v(t_0) \cdot c_v(t)$.
	Es gilt 
	\begin{align}
		\tilde{c}'(t) = \enbrace*{\mathd L_{c_v(t_0)}}_{c_v(t)}  \cdot \hspace{-1.5em}\Underbrace{\diffd{}{t} c_v(t)}{= X_v \enbrace[\big]{c_v(t)} = \enbrace*{\mathd L_{c_v(t)}} \cdot v} \hspace{-1.5em}= \enbrace*{\mathd L_{c_v(t_0) \cdot c_v(t)}}_e \cdot v = X_v \enbrace*{\tilde{c}(t)}
	\end{align}
	Somit ist $\tilde{c}(t)$ wieder eine Integralkurve von $X_v$ und beide Behauptungen folgen.
	Man nennt die Kurve $(c_v(t))_{t \in \mathbb{R}}$ auch \Index{Einparameteruntergruppe} von $G$.
\end{beweis}

\begin{definition}[{name=[Exponentialabbildung]}]
	Sei $G$ eine Liegruppe.
	Dann nennt man 
	\[
		\exp \colon \mathfrak{g} \longrightarrow G ,\quad  v \longmapsto c_v(1)
	\]
	die \Index{Exponentialabbildung} von $G$. 
\end{definition}

\begin{lemma}[{name=[{Differenzierbarkeit der Exponentialabbildung}]}]
	Sei $G$ eine Liegruppe.
	Dann ist die Exponentialabbildung $\exp \colon \mathfrak{g} \to G$ differenzierbar und es gilt 
	\[
		\enbrace*{\mathd \exp}_0 = \id_{\Tmap_e G}
	\]
\end{lemma}
\begin{beweis}
	Wir betrachten das Vektorfeld
	\mapdef{S \colon G \times \mathfrak{g}}{\Tmap (G \times \mathfrak{g})}{(g,v)}{\enbrace[\big]{X_v(g),0} \in \Tmap_g G \times \Tmap_v \mathfrak{g}}{}
	$S$ ist ein differenzierbares Vektorfeld mit globalem Fluss $\set*{\psi_t}_{t \in \mathbb{R}}$ gegeben durch 
	\[
		\psi_t(g,v) = \enbrace[\big]{g \cdot \exp(t \cdot v),v}
	\]
	denn\marginnote{man muss sich hier klarmachen, dass $c_{tv}(1)=c_{v}(t)$ gilt}
	\begin{align}
		\diffd{}{t}\Big|_{t=t_0} \psi_t (g,v) = \diffd{}{t}\Big|_{t=t_0} \enbrace[\big]{g \cdot \exp(tv),v} = \enbrace*{\diffd{}{t}\Big|_{t=t_0} g \cdot \exp(tv),0} &= (\mathd L_g)_{c_{t_0 v}(1)} \cdot \diffd{}{t}\Big|_{t=t_0} c_{tv}(1) \\[-9pt]
		&= \enbrace*{\mathd L_{g \cdot c_v(t_0)}}_e \cdot v \\
		&= X_v \enbrace[\big]{g \cdot c_v(t_0)} = \exp(t_0 v)
	\end{align}
	Somit ist $\psi_t \colon G \times \mathfrak{g} \to G \times \mathfrak{g}$ differenzierbar für alle $t \in \mathbb{R}$.
	Insbesondere ist $\psi_1$ differenzierbar.
	Mit $\psi_1(e,\cdot ) = \enbrace[\big]{\exp(\cdot ), \cdot }$ folgt, dass auch $\exp$ differenzierbar sein muss.
	Die zweite Behauptung ist nun sofort klar.
\end{beweis}

\begin{satz}[label=satz:1210,{name=[Abgeschlossenheit von Lieuntergruppen]}]
	Sei $G$ eine Liegruppe. Dann gilt:
	\begin{enumerate}[1)]
		\item Eine Lieuntergruppe $H$ von $G$ ist genau dann eine eingebettete Untermannigfaltigkeit von $G$, wenn $H$ abgeschlossen ist.
		\item Eine \enquote{abstrakte} Untergruppe $H$ von $G$ ist eine Lieuntergruppe, falls $H$ abgeschlossen ist.
	\end{enumerate}
\end{satz}
\begin{beweis}
	\begin{enumerate}[1)]
		\item Sei $H \subset G$ ein eingebettete Lieuntergruppe. 
		Zu zeigen ist, dass $H$ abgeschlossen ist.
		Angenommen es existiert eine Folge $(h_i)_{i \in \mathbb{N}} \subset H$ mit $h_i \to g \in G \setminus H$.
		Sei $U$ eine offene Umgebung von $e$, sodass $H \cap U$ eine zusammenhängende Untermannigfaltigkeit von $U$ ist.
		Eine solche Umgebung existiert, da $H$ eingebettet ist.
		Da $L_g$ ein Diffeomorphismus ist, ist $L_g(U)$ eine offene Umgebung von $g$.
		Somit existiert $i_0$ mit $h_i \in L_g(U)$ für alle $i \ge i_0$.
		Wegen $h_i \to g$ können wir $h_{i_0}^{-1} \cdot g \in U$ annehmen.
		Es gilt $h_{i_0}^{-1} \cdot g \notin H \cap U$, denn sonst folgt $g \in H$.
		Wegen $h_i \to g$ folgt
		\[
			\Underbracket{h_{i_0}^{-1}\cdot h_i}{\in H}  \grenzw{i \to \infty} h_{i_0}^{-1} \cdot g
		\]
		Dies ist -- in lokalen Koordinaten via einem Flachmacher -- ein Widerspruch und somit muss $H$ abgeschlossen sein.
		
		Sei nun $H$ eine abgeschlossene Lieuntergruppe von $G$.
		Zu zeigen ist, dass $H$ eine eingebettete Untermannigfaltigkeit ist, das heißt es existiert eine offene Umgebung $U$ von $e$ in $G$, sodass $H \cap U$ eine zusammenhängende Untermannigfaltigkeit von $U$ ist. 
		Annahme: Es existiert eine Umgebung $U$ von $e$, sowie eine Folge $(h_i)_{i \in \mathbb{N}}$ in $H$ mit
		\begin{itemize}
			\item $h_i \to e$
			\item $h_i \notin (H \cap U)_e$ (Zusammenhangskomponente der Eins)
		\end{itemize}
		Die Exponentialabbildung $\exp \colon \Tmap_e G \to G$ ist nahe bei $e$ ein lokaler Diffeomorphismus (Umkehrsatz auf Mannigfaltigkeiten) mit 
		\[
			\enbrace*{\mathd \exp^{-1}}_e = \id_{\Tmap_p M^n}
		\]
		Somit ist $h_i = \exp(v_i)$ für alle $i \in \mathbb{N}$ und $v_i \in \mathfrak{g}$ mit $v_i \to 0$ für $i \to \infty$.
		Wir wählen nun eine reelle Folge $(s_i)_{i \in \mathbb{N}}$ in $\mathbb{R}$ mit 
		\begin{equation}
			s_i \cdot v_i \grenzw{i \to \infty} v \label{eq:1210:1} \tag{\#}
		\end{equation}
		längst einer Teilfolge.
		(Da $\mathfrak{g}$ ein Vektorraum ist, können wir das Skalarprodukt auf $\mathfrak{g}$ betrachten und erhalten eine Norm, setze dann $s_i := \frac{1}{\norm*{v_i}} $.)
		Sei $t \in \mathbb{R}$ \emph{fest} gewählt.
		\eqref{eq:1210:1} impliziert dann
		\begin{equation}
			t \cdot s_i \cdot v_i \grenzw{i \to \infty} t \cdot v \label{eq:1210:2} \tag{\#\#}
		\end{equation}
		Eine einfache Überlegung zeigt, dass man $t \cdot s_i \in \mathbb{Z}$ annehmen kann ($s_i \leadsto \tilde{s}_i := s_i + \text{\enquote{$\sfrac{1}{t}$}}$).
		Beachte, dass \eqref{eq:1210:1} immernoch gilt.
		Aus $H \ni h_i = \exp(v_i)$ folgt 
		\[
			H \ni \exp(v_i)^{t \cdot s_i} = \exp \enbrace*{t \cdot s_i \cdot v_i} \xrightarrow[\text{\eqref{eq:1210:2}}]{i \to \infty} \exp(t \cdot v) \in H
		\]
		da $H$ abgeschlossen ist!
		Dies gilt für alle $t \in \mathbb{R}$ und somit ist $v \in \Tmap_e H$.
		Nächster Schritt: $\skal*{\cdot }{\cdot }$ sei ein Skalarprodukt auf $\Tmap_e G$.
		Dann gilt $\Tmap_e G = \mathfrak{h} \oplus_{\bot} \mathfrak{h}^{\bot}$.
		Mit dieser Zerlegung schreiben wir $v_i =x_i + y_i$.
		Die Abbildung 
		\mapdef{\Phi \colon \mathfrak{h} \oplus_{\bot} \mathfrak{h}^\bot}{G}{(x,y)}{\exp(x) \cdot \exp(y)}{}
		ist nahe $(0,0)$ ein lokaler Diffeomorphismus.
		Wir haben
		\[
			\Underbracket{\exp(v_i)}{\in H} = \Underbracket{\exp(\tilde{x}_i)}{\in H} \cdot \exp(\tilde{y}_i)
		\]
		Dann ist $e \longleftarrow  \exp(\tilde{y}_i) = \exp(- \tilde{x}_i) \cdot \exp(v_i) \in H \setminus (H \cap U)_e$.
		Wie eben bewiesen, können wir nun eine Grenzrichtung 
		\[
			\frac{y_i}{\norm*{y_i}} \grenzw{i \to \infty}  y
		\]
		in $\mathfrak{h}$ konstruieren (mittels Teilfolge).
		Die Eigenschaften $y \in H$, $y \in H^\bot$ und $\norm*{y}=1$ widersprechen sich!
		Also existiert eine solche Folge nicht und $H$ ist eine eingebettete Untermannigfaltigkeit wie behauptet.
		\item Sei $H$ ein abgeschlossene \enquote{abstrakte} Untergruppe von $G$.
		Wir zeigen zunächst, dass 
		\[
			\mathfrak{h} := \set[\big]{v \in \Tmap_e G \given \exp(t \cdot v) \in H \,\,\forall t \in \mathbb{R}} \ni 0
		\]
		ein Unterraum ist.
		Annahme: Es gibt eine Folge $(h_i)_{i \in \mathbb{N}}$ in $H \setminus \set*{e}$ mit $h_i \to e$ (falls nicht: $e$ ist isoliert und $H$ ist eine diskrete Untergruppe von $G$).
		Wie in 1) zeigt man, dass die \enquote{Grenzrichtungen} $\lim_{i \to \infty} \frac{v_i}{\norm*{v_i}}$ in $\mathfrak{h}$ liegen.
		\begin{itemize}[itemsep=0pt]
			\item $v \in \mathfrak{h} \implies t \cdot v \in \mathfrak{h}$ für alle $t \in \mathbb{R}$
			\item $v_1 + v_2 \in \mathfrak{h} \implies v_1 + v_2 \in \mathfrak{h}$
		\end{itemize}
		Sei $(t_n)_{n \in \mathbb{N}}$ eine Nullfolge in $\mathbb{R}$.
		Dann gilt $t_n v_1, t_n v_2 \to 0$ für $n \to \infty$.
		Folglich gilt 
		\[
			\exp(t_n v_1), \exp(t_n v_2) \grenzw{n \to \infty} e 
		\]
		für $n \to \infty$. Somit erhalten wir
		\[
			H \ni \exp(v_n) = \exp \enbrace*{t_n v_1} \exp(t_n v_2) \grenzw{n \to \infty} e
		\]
		mit $v_n \to 0$.
		Nun gilt $\exp(t_n v_1) \cdot \exp(t_n v_2) = \exp \enbrace[\big]{t_n(v_1 + v_2)} + \mathcal{O}(t_n^2)$.
		Somit
		\[
			\frac{v_n}{t_n} \grenzw{n \to \infty} v_1 +v_2 + \mathcal{O} \enbrace*{\frac{t_n^2}{t_n} \to 0 } 
		\]
		Damit ist $\mathfrak{h}$ ein Unterraum.
		Wie in 1) zeigt man nun, das $H$ ein Untermannigfaltigkeit ist.\qedhere
	\end{enumerate}
\end{beweis}

\begin{satz}[name={Ado},label=satz:1211]
	Jede endlichdimensionale Liealgebra $(V,\benbrace*{\cdot,\cdot})$ ist isomorph zu einer Unteralgebra von $\mathfrak{gl}(n,\mathbb{R}) = \Tmap_e \GL(n,\mathbb{R})$.
\end{satz}
\begin{beweis}
	\emph{Siehe zum Beispiel \cite[S. 199]{JacobsonLieAlg}.}
\end{beweis}

\begin{korollar}[{name=[{Zu jeder endlichdimensionalen Liealgebra existiert eine Liegruppe}]}]
	Zu jeder endlichdimensionalen Liealgebra $(V,\benbrace*{\cdot ,\cdot })$ existiert eine Liegruppe $G$ mit $V \cong \mathfrak{g}$
\end{korollar}
\begin{beweis}
	Folgt direkt aus \autoref{satz:1211} und \autoref{satz:125}.
\end{beweis}

\begin{bemerkung*}[{name=[{Existenz von Liegruppen, die keine Matrixgruppe ist}]}]
	Für $n \ge 3$ ist $\pi_1 \enbrace*{\SL(n,\mathbb{R})} = \mathbb{Z}_2$
	Die universelle Überlagerung $\widetilde{\SL(n,\mathbb{R})}$ von $\SL(n,\mathbb{R})$ ist eine Liegruppe, welche in keine der Gruppen $\GL(N,\mathbb{R})$ eingebetten werden kann! Insbesondere ist sie keine Matrixgruppe!
\end{bemerkung*}

\begin{lemma}[label=lem:1213,{name=[{Zusammenhängede Liegruppe von Umgebung der Eins erzeugt}]}]
    Sei $G$ eine zusammenhängende Liegruppe und $U$ eine offene Umgebung von $e$ in $G$.
    Dann wird $G$ schon von $U$ erzeugt.
\end{lemma}
\begin{beweis}
    Sei $U \subset G$ offen mit $e \in U$.
    Dann existiert $V \subset U$ offen mit $e \in V$, sodass $V \cdot V^{-1} \subset U$, denn die Abbildungen $m \colon G \times G \to G$, $^{-1} \colon G \to G$ sind stetig.
    Wir setzen 
    \[
        H := \bigcup_{n \in \mathbb{Z}} V^n
    \]
    $H$ ist eine offene Untergruppe von $G$.
    Zu zeigen bleibt, dass $H$ abgeschlossen ist.
    Die Abgeschlossenheit folgt wie im Beweis von \autoref{satz:125}.
\end{beweis}
% subsection 12 (end)

\section{Überlagerungen von Liegruppen} % (fold)
\label{sec:13}

\todo[inline]{RevChap 1}

\begin{definition}
    Sei $M^n$ eine $n$-dimensionale, zusammenhängende, differenzierbare Mannigfaltigkeit.
    Dann nennt man eine zusammenhängende, $n$-dimensionale, differenzierbare Mannigfaltigkeit $\tilde{M}^n$ \Index{Überlagerung} von $M^n$, falls eine surjektive differenzierbare Abbildung $\pi \colon \tilde{M} \to M$ existiert mit folgenden Eigenschaften:
    
    Für alle $p \in M$ existierte eine offene Umgebung $U$ von $p$ in $M^n$ mit folgenden Eigenschaften:
    \begin{enumerate}[1)]
        \item $\pi^{-1}(U) = \bigcup_{i \in I} \tilde{U}_i$ mit $\tilde{U}_i \subset \tilde{M}$ offen und $\tilde{U}_i \cap \tilde{U}_j = \emptyset$ für $i \neq j$.\marginnote{Jeder Punkt wird \enquote{schlicht überlagert}}
        \item $\pi\big|_{\tilde{U}_i} \colon \tilde{U}_i \to U$ ist ein Diffeomorphismus für alle $i \in I$.
    \end{enumerate}
    Die Abbildung $\pi \colon \tilde{M}^n \to M^n$ nennt man \Index{Überlagerungsabbildung} und $\abs*{I}$ die \Index{Blätterzahl} ($\abs*{I}= \infty$ ist möglich).
\end{definition}

Beispiele:
\begin{itemize}
    \item $M^n$ überlagert $M^n$.
    \item $S^3$ ist 2-fache Überlagerung von $\SO(3) = \mathbb{R}P^3$
    \item $S^n$ ist 2-fache Überlagerung von $\mathbb{R}P^n \coloneqq \set[\big]{\text{Menge der Geraden im }\mathbb{R}^{n+1}}$.
    \item Ist $G$ eine Gruppe, die frei und eigentlich diskontinuierlich auf $\tilde{M}^n$ operiert, so ist der Bahnenraum 
    \[
        M^n \coloneqq \set*{[p] \given p \in M^n} = \sfrac{\tilde{M}^n}{G}
    \]
    eine $n$-dimensionale, differenzierbare Mannigfaltigkeit und $\pi \colon \tilde{M}^n \to M^n$, $p \mapsto [p]$ eine Überlagerung
    \[
        [p] = [\tilde{p}] :\Longleftrightarrow \exists g \in G : p = g \cdot \tilde{p}
    \]
    Ist $G$ endlich: Dann ist $G$ frei und diskontinuierlich äquivalent zu: $g$ wirkt trivial auf einem Punkt $p$ genau dann, wenn $g$ auf allen $p$ trivial operiert.
    \item Für jede zusammenhängende Mannigfaltigkeit $M^n$ existiert eine bis auf Diffeomorphie einfach zusammenhängende Mannigfaltigkeit $\tilde{M}^n$, die $M^n$ überlagert (\Index{Universelle Überlagerung}).
    Ferner gilt $M^n = \sfrac{\tilde{M}^n}{\pi_1(M^n)}$.
\end{itemize}

\begin{lemma}[label=lem:132]
    Sei $G$ eine zusammenhängende Liegruppe.
    \begin{enumerate}[1)]
        \item Ist $\tilde{G}$ eine zusammenhängede Mannigfaltigkeit und $\pi \colon \tilde{G} \to G$ eine Überlagerungsabbildung, so besitzt $\tilde{G}$ genau eine Liegruppenstruktur, sodass $\pi$ ein Liegruppenhomomorphismus ist.
        \item Ein Liegruppenhomomorphismus $\Phi \colon \tilde{G} \to G$ mit $\tilde{G}$ zusammenhängend ist eine Überlagerung genau dann, wenn das Differential $(\mathd \Phi)_{\tilde{e}} \colon \Tmap_{\tilde{e}} \tilde{G} \to \Tmap_e G$ ein Isomorphismus ist.
    \end{enumerate}
\end{lemma}
\begin{beweis}
    \begin{enumerate}[(1)]
        \item Sei $\tilde{e} \in \pi^{-1}(e)$.
        Überlagerungstheorie impliziert (Stöcker-Zischer, Algebraische Topologie, Abschnitt 6.2)\todo{ref}, dass die stetigen Funktionen $m \colon G \times G \to G$, $^{-1} \colon G \to G$ entsprechende Lifts besitzen.
        Die Eindeutigkeit ist durch die Wahl von $\tilde{e}$ gegeben [Klar falls $\tilde{G}$ einfach zusammenhängend, sonst \enquote{kleines} Argument].
        
        Die Abbildungen $\tilde{m},\tilde{{ }^{-1}}$ sind differenzierbare Abbildungen, welche die Gruppenaxiome erfüllen: $\tilde{m} (\tilde{e}, \tilde{g}) = \tilde{g}$ für alle $\tilde{g} \in \tilde{G}$.
        Dies ergibt sich daraus, dass man Wege eindeutig liften kann.
        
        Die Abbildung $\pi \colon \tilde{G} \to G$ ist somit ein Liegruppenhomomorphismus.
        Die Eindeutigkeit der Liegruppenstruktur auf $\tilde{G}$, sodass $\pi \colon \tilde{G} \to G$ ein Liegruppenhomomorphismus, folgt wieder aus der Eindeutigkeit der Lifts.
        \item Die erste Implikation ist klar, denn Überlagerungsabbildungen sind lokale Diffeomorphismen.
        
        Für die Umkehrung sei $(\mathd \Phi)_{\tilde{e}} \colon \Tmap_{\tilde{e}} \tilde{G} \to \Tmap_e G$ ein Isomorphismus.
        Zu zeigen: Jedes $g \in G$ ist schlicht überlagert.
        Sei $\tilde{U}$ eine offene Umgebung von $\tilde{e}$ in $\tilde{G}$, sodass $\Phi|_{\tilde{U}} \colon \tilde{U} \to \Phi(\tilde{U})$ ein Diffeomorphismus ist (Umkehrsatz).
        Sei $\tilde{\Gamma} := \ker (\Phi)$.
        Dann gilt $\tilde{\Gamma} \cap \tilde{U}= \tilde{e}$.
        Sei nun $\tilde{V}$ eine offene Umgebung von $\tilde{e}$ mit $\tilde{V} \cdot \tilde{V}^{-1} \subset \tilde{U}$.
        \begin{itemize}
            \item  Seien $\tilde{\gamma}_1 \neq \tilde{\gamma}_2 \in \tilde{\Gamma}$ und $\tilde{v}_1, \tilde{v}_2 \in \tilde{V}$.
            Annahme:
            \[
                \tilde{\gamma}_1 \cdot \tilde{v}_1 = \tilde{\gamma}_2 \cdot \tilde{v}_2 \iff \tilde{\gamma}_2^{-1} \cdot \tilde{\gamma}_1 = \tilde{v}_2 \cdot \tilde{v}_1^{-1} \in \tilde{\Gamma} \cap \tilde{U}
            \]
            Mit obiger Eigenschaft des Schnitts folgt $\tilde{\gamma}_1 = \tilde{\gamma}_2$.
            Damit folgt $\tilde{\gamma}_1 \tilde{V} \cap \tilde{\gamma}_2 \tilde{V} = \emptyset$.
            \item Es gilt $\Phi^{-1}(V) = \bigcup_{\tilde{\gamma} \in \tilde{\Gamma}} \tilde{\gamma}(v)$:
            
            Sei $\tilde{v} \in  \tilde{V}$ und $\tilde{g} \in \tilde{G}$ mit $\Phi(\tilde{v}) = \Phi(\tilde{g})$.
            Dies ist äquivalent zu
            \[
                \Phi(\tilde{g}^{-1} \tilde{v}) = \Phi(\tilde{g}^{-1}) \Phi(\tilde{v}) = e
            \]
            Also ist $\tilde{g}^{-1} \tilde{v} = \tilde{\gamma} \in \tilde{\Gamma}$ und somit $\tilde{v} = \tilde{\gamma} \cdot \tilde{g}$.
            \item Die Abbildung $\Phi|_{\tilde{\gamma} \tilde{V}} \colon \tilde{\gamma}(\tilde{V}) \to V$ ist ein Diffeomorphismus:
            
            \[
                \Phi(\tilde{\gamma} \tilde{v}) = \Phi(\tilde{\gamma}) \cdot \Phi(\tilde{v}) = \Phi \enbrace*{\tilde{\gamma}^{-1} \tilde{\gamma} \tilde{v}} = \enbrace*{\Phi \circ L_{\tilde{\gamma}^{-1}}} (\tilde{\gamma} \tilde{v})
            \]
            \item Alle $g \in G$ sind schlicht überlagert.
            \item Die Abbildung $\Phi$ ist surjektiv nach \autoref{lem:1213}, denn eine offene Umgebung von $e$ liegt im Bild und $\Phi$ ist ein Homomorphismus.\qedhere
        \end{itemize}
    \end{enumerate}
\end{beweis}

\begin{korollar}[label=lem:133]
    Seien $\tilde{G}$, $G$ zusammenhängende Liegruppen. 
    \begin{enumerate}[1)]
        \item Ist $\Phi \colon \tilde{G} \to G$ eine Überlagerung von Liegruppen, so ist $\tilde{\Gamma} = \ker \Phi$ eine diskrete Untergruppe von $Z(G)$.\marginnote{Zentrum von $\tilde{G}$}
        \item Ist $\tilde{\Gamma}$ eine diskrete Untergruppe von $Z(\tilde{G})$, so ist $G \coloneqq \sfrac{\tilde{G}}{\tilde{\Gamma}}$ eine Liegruppe und $\Phi \colon \tilde{G} \to G$, $\tilde{g} \mapsto [\tilde{g}]$ eine Überlagerung.
    \end{enumerate}
\end{korollar}
\begin{beweis}
    \begin{enumerate}[1)]
        \item Sei $\tilde{U}$ eine offene Umgebung von $\tilde{e}$, sodass $\Phi|_{\tilde{U}} \colon \tilde{U} \to \Phi(\tilde{U}) =: U$ ein Diffeomorphismus ist.
        Dann folgt $\tilde{\Gamma} \cap \tilde{U} = \set*{\tilde{e}}$.
        Somit ist $\tilde{\Gamma}$ eine diskrete Untergruppe von $\tilde{G}$.
        Da $\tilde{\Gamma}$ der Kern eine Homomorphismus ist, ist $\tilde{\Gamma}$ ein Normalteiler in $\tilde{G}$.
        Sei $\tilde{g} \in \tilde{G}$ und $\tilde{g}(t) \colon [0,1] \to \tilde{G}$ ein stetiger Weg mit $\tilde{g}(0)=\tilde{e}$, $\tilde{g}(1)=\tilde{g}$.
        Dann gilt
        \[
            \tilde{g}(t) \tilde{\gamma} \tilde{g}^{-1}(t) \in \tilde{\Gamma}
        \]
        Es folgt -- da $\tilde{\Gamma}$ diskret ist -- $\tilde{g}(t) \tilde{\gamma} (\tilde{g}(t))^{-1} \equiv \tilde{\gamma}$.
        Es folgt 
        \[
            \tilde{g} \tilde{\gamma} \tilde{\gamma} \tilde{g} \forall \tilde{g} \in \tilde{G} \iff \tilde{\gamma} \in Z(\tilde{G})
        \]
        \item \emph{Übung!}
    \end{enumerate}
\end{beweis}

Ein kleines Beispiel:
\[
    Z \enbrace*{\SO(n)} = \begin{cases}
        \set*{\id_n} &\text{ falls } n \text{ ungerade} \\
        \set*{\pm \id_n} &\text{ falls $n$ gerade}
    \end{cases}
\]
Für $n \ge 2$
\[
    Z \enbrace*{\SU(n)} = \mathbb{Z}_n = \set*{e^{2 \pi i \sfrac{k}{n}} \cdot \id_n \given k=0,1,\ldots ,n-1 }
\]
Es gilt $\pi_1(\SO(n))= \mathbb{Z}_2$.
Die universelle Überlagerung ist $\mathrm{Spin}(n)$.
Für $n\ge 2$ $\pi \enbrace*{\SU(n)} = \set*{e}$ und $\SU(2) = S^3$.

\begin{lemma}[label=lem:134]
	Seien $H,G$ zusammenhängende Liegruppen.
	Ist $H$ einfach zusammenhängend und $\psi \colon \Tmap_e H \to \Tmap_e G$ ein Liealgebrenhomomorphismus, so existiert ein eindeutiger Liegruppenhomomorphismus $\Phi \colon H \to G$ mit $(\mathd \Phi)_e = \psi$.
\end{lemma}
\begin{beweis}
	Da der Graph
	\[
		\Graph(\psi) = \set[\big]{\enbrace*{v,\psi(v)} \given v \in \Tmap_e H} \subset \Tmap_e H \oplus \Tmap_e G
	\]
	eine Unteralgebra von $\Tmap_e H \oplus \Tmap_e G$ ist, existiert nach \autoref{satz:125} eine eindeutig bestimmte Lieuntergruppe $A$ von $H \times G$ mit $\Tmap_e A = \Graph(\psi)$.
	Wir bezeichnen mit $\pi_1 \colon A \to H$ und $\pi_2 \colon A \to G$ die Projektionen auf den ersten bzw. zweiten Faktor von $H \times G$.
	Die Abbildungen $\pi_1$ und $\pi_2$ sind Liegruppenhomomorphismen.
	Ferner ist 
	\mapdef{(\mathd \pi_1)_e \colon \Tmap_e A}{\Tmap_e H}{(v,\psi(v))}{v}{}
	ein Isomorphismus.
	Nach \autoref{lem:132} 2) überlagert $A$ somit $H$.
	Da $H$ einfach zusammenhängend ist, folgt $A \cong H$.
	Wir erhalten somit einen Homomorphismus 
	\mapdef{\Phi \colon H = A}{G}{h}{\pi_2(h)}{}
	mit $(\mathd \Phi)_e = (\mathd \pi_2)_e = \psi$.
\end{beweis}

Ist in obiger Situation $H$ \emph{nicht} einfach zusammenhängend, so erhält man einen Homomorphismus $\tilde{\Phi} \colon \tilde{H} \to G$, wobei $\tilde{H}$ eine universelle Überlagerung von $H$ ist.
Dies liefert genau dann einen Homomorphismus $\Phi \colon H \to G$, wenn $\ker \pi \subset \ker \tilde{\Phi}$ ist, wobei $\pi \colon \tilde{H} \to H$ die Überlagerungsabbildung ist.

Dies ist aber nicht immer richtig:
So kann man zum Beispiel den durch die Identität gegeben Liealgebrenisomorphismus $\psi \colon \mathfrak{h} \to \mathfrak{h}$ betrachten.
Dann ist die induzierte Abbildung $\tilde{\Phi} \colon \tilde{H} \to \tilde{H}$ injektiv, aber $\pi \tilde{H} \to H$ muss nicht injektiv sein.
Ein konkretes Beispiel wäre die universelle Überlagerung von $\SL(n,\mathbb{R})$.

\begin{satz}
	Es gilt
	\begin{enumerate}[1)]
		\item Zwei zusammenhängende, einfach zusammenhängenden Liegruppen mit isomorphen Liealgebren sind isomorph.
		\item Für jede endlich dimensionale Liealgebra $(V, \benbrace*{\cdot,\cdot })$ existiert genau eine zusammenhängende, einfach zusammenhängende Liegruppe $\tilde{G}$ mit Liealgebra $\tilde{\mathfrak{g}}=V$.
		\item Jede zusammenhängende Liegruppe $G$ ist isomorph zu $\sfrac{\tilde{G}}{\tilde{\Gamma}}$, wobei $\tilde{G}$ eine zusammenhängende, einfach zusammenhängende Liegruppe und $\tilde{\Gamma} \subset Z(\tilde{G})$ eine diskrete Untergruppe ist.
	\end{enumerate}
\end{satz}
\begin{beweis}
	\begin{enumerate}[1)]
		\item Sei $\psi \colon \Tmap_e H \to \Tmap_e G$ ein Liealgebrenisomorphismus, $H,G$ zusammenhängende und einfach zusammenhängende Liegruppen.
		Nach \autoref{lem:134} existieren Homomorphismen $\Phi_1 \colon H \to G$ und $\Phi_2 \colon G \to H$ mit
		\[
			(\mathd \Phi_1)_e = \psi \qquad (\mathd \Phi_2)_e = \psi^{-1}
		\]
		Somit ist $\Phi_2 \circ  \Phi_1 \colon H \to G$ ein Liegruppenhomomorphismus mit $(\mathd \Phi_2 \circ \Phi_1)_e = \id_{\Tmap_e H}$.
		Nach \autoref{kor:126} gilt $\Phi_2 \circ \Phi_1= \id_H$ und somit $\Phi_2 = \Phi_1^{-1}$.
		\item Nach dem Satz von Ado existiert eine Liegruppe $G$ mit Liealgebra $\mathfrak{g}=V$.
		Die universelle Überlagerung $\tilde{G}$ von $G$ ist die gesuchte Liegruppe.
		Eindeutigkeit folgt aus 1).
		\item \autoref{lem:133}\qedhere
	\end{enumerate}
\end{beweis}

Beispiel: 
\[
	\SL(2,\mathbb{R}) = \set*{\begin{pmatrix}
		a & b \\ c & d
	\end{pmatrix} \given ad - bc =1}
\]
ist eine 3-dimensionale Untergruppe von $\GL(n,\mathbb{R})$.
Es gilt $\SL(2,\mathbb{R}) = \SO(2) \times \mathbb{R}^2$.
Somit ist $\pi_1 \enbrace*{\SL(2,\mathbb{R})} = \pi_1 (\SO(2)) = \mathbb{Z}$.
Es sei $M^3 := D^2 \times \mathbb{R} \cong \mathbb{R}^3$ (diffeomorph) mit
\[
	D^2 := \set*{z \in \mathbb{C} \given \abs*{z} <1} = \set*{\begin{pmatrix}
		x \\ y
	\end{pmatrix} \in \mathbb{R}^2 \given x^2 +y^2 <1}
\]
Wir führen folgende Multiplikation auf $M^3$ ein:
\[
	(\alpha,\varphi) \cdot (\beta, \psi) := \enbrace*{ \frac{\beta + \alpha \cdot e^{-i \psi}}{1 + \alpha \tilde{\beta} e^{-i \psi}}, \varphi + \psi + 2 \mathrm{Arg} \enbrace*{1 + \alpha \overline{\beta} e^{-i \psi}} }
\]
Es gilt zum Beispiel $(\alpha,\varphi) \cdot (0,0) = \enbrace*{\alpha,\varphi} = (0,0) \cdot (\alpha,\varphi)$.
Das heißt das neutrale Element ist $e=(0,0)$.
Ferner gilt
\[
	(\alpha,\varphi) \cdot \enbrace*{- \alpha e^{i \varphi}, -\varphi} = (0,0) = e
\]
Somit ist $M^3$ eine einfach zusammenhängende Liegruppe (!).
Man rechnet leicht nach, dass 
\begin{align}
	E_1(\alpha,\varphi) &= \enbrace*{\frac{1}{2}  (1- \alpha^2), \Im \alpha} = \enbrace*{\frac{1}{2} (1 -x^2 +y^2), -xy,y} \\
	E_2(\alpha,\varphi) &= \enbrace*{\frac{1}{2} i (1+ \alpha^2), - \Re(\alpha) } = \enbrace*{-xy, \frac{1}{2} \enbrace*{1+x^2 -y^2}, -x} \\
	E_3(\alpha,\varphi) &= \enbrace*{-i \alpha,1} = (y,-x,1)
\end{align}
linksinvariante Vektorfelder mit
\[
	E_1(0,0) = \enbrace*{\sfrac{1}{2},0,0 } \qquad E_2(0,0) = \enbrace*{0, \sfrac{1}{2}, 0 } \qquad E_3(0,0) = (0,0,1)
\]
Dies erhält man zum Beispiel durch
\[
	E_1(\alpha,\varphi) = \mathd L_{(\alpha,\varphi)} \cdot \enbrace*{\sfrac{1}{2},0,0} = \diffd{}{t}\big|_{t=0} (\alpha,\varphi) \cdot \enbrace*{\sfrac{1}{2},0,0} = \ldots 
\]
Es gilt
\[
	\benbrace*{E_1,E_2} = - E_3 \quad \benbrace*{E_3,E_1} = E_2 \quad \benbrace*{E_2,E_3} =E_1
\]
Auf $\SL(2,\mathbb{R})$
\[
	\overline{E}_1 := \frac{1}{2} \begin{pmatrix}
		1 & 0 \\ 0 & -1
	\end{pmatrix} \quad 
	\overline{E}_2 := \frac{1}{2}  \begin{pmatrix}
		0 & 1 \\ 1 & 0
	\end{pmatrix} \quad 
	\overline{E}_3 := \frac{1}{2} \begin{pmatrix}
		0 & -1 \\ 1 & 0
	\end{pmatrix} 
\]
% section 13 (end)

\section{Die Exponentialabbildung} % (fold)
\label{sec:14}

\begin{definition}[{name=[{Einparameteruntergruppe}]}]
	Sei $G$ eine Liegruppe.
	Dann nennt man einen Homomorphismus $\Phi \colon (\mathbb{R},+) \to G$ eine \Index{Einparameteruntergruppe} (von $G$).
\end{definition}

Beispiel: $G = \GL(n;\mathbb{R})$, $v \in \Tmap_e \GL(n,\mathbb{R}) = \Mat(n,\mathbb{R})$.
\[
	\Phi_v \colon \mathbb{R} \longrightarrow G \qquad t \longmapsto \exp(t \cdot v) = \sum_{k=0}^{\infty}  \frac{(tv)^k}{k!} 
\]
Es gilt $\Phi_v(s+t) = \Phi_v(s) \cdot \Phi_v(t) = \Phi_v(t) \cdot \Phi_v(s)$ für alle $s,t \in \mathbb{R}$, wie man sich leicht überlegt.

Zurück zum allgemeinen Fall: Sei $v \in \Tmap_e G \setminus \set*{0}$.
Wegen $\Tmap_0 \mathbb{R} \cong \mathbb{R}$ ist
\[
	\psi_v \colon \mathbb{R} \to \Tmap_e G \quad t \mapsto t \cdot v
\]
ein Liealgebrenhomomorphismus. 
Da $\mathbb{R}$ einfach zusammenhängend ist, existiert nach \autoref{lem:134} ein eindeutig bestimmter Homomorphismus
\(
	\Phi_v \colon \mathbb{R} \to G
\)
mit $\Phi_v'(0)= (\mathd \Phi_v)_0 = \psi_v$.

\begin{lemma}[label=lem:142]
	Sei $G$ eine Liegruppe mit Exponentialabbildung $\exp \colon \Tmap_e G \to G$.
	Dann gilt $\Phi_v(1) = \exp(v)$ für alle $v \in \Tmap_e G$.
\end{lemma}
\begin{beweis}
	Wir hatten in \autoref{lem:127} gezeigt, dass $c_v(s +t) = c_v(s) \cdot c_v(t)$ für alle $t,s \in \mathbb{R}$ gilt, wobei $c_v$ die Integralkurve von $X_v$ ist mit $X_v(e)=v$.
	Somit gilt
	\[
		\exp \enbrace*{(s+t) v} = c_{(s+t)v}(1) = c_v(s+t) = c_v(s) \cdot c_v(t) = c_{sv}(1) \cdot c_{tv}(1) = \exp(sv) \cdot \exp(tv)
	\]
	Damit ist $t \mapsto \exp(tv)$ eine Einparametergruppe mit $\diffd{}{t}\big|_{t=0} \exp(tv)=v$.
	Die Eindeutigkeit impliziert $\Phi_v(t)=\exp(tv)$.
\end{beweis}

\begin{lemma}[label=lem:143]
	Es gilt
	\begin{enumerate}[(1),itemsep=0pt]
		\item Für alle $v \in \Tmap_eG$ ist $\Phi_v(t)=\exp(tv)$ eine Einparamtergruppe von $G$ mit $\Phi'_v(0)=v$.
		\item Sei $g \in G$ und $v \in \Tmap_eG$.
		Dann ist $L_g(\Phi_v(t))$ Integralkurve von $X_v$.
		\item Ist $\Phi \colon H \to G$ ein Homomorphismus, so gilt für alle $w \in \Tmap_eW$ für alle $t \in \mathbb{R}$
		\[
			\Phi \enbrace*{\exp_H(tw)} = \exp_G \enbrace*{t \enbrace*{\mathd \Phi}_e \cdot w}
		\]
		\item Ist $H$ eine Lieuntergruppe von $G$, so gilt
		\[
			\Tmap_e H = \set*{v \in \Tmap_e G \given \exp_G(tv) \in H \enspace \forall t \in \mathbb{R}}
		\]
	\end{enumerate}
\end{lemma}
\begin{beweis}
	\begin{enumerate}[(1),itemsep=0pt]
		\item siehe \autoref{lem:142}
		\item Klar, denn $\exp(tv)=c_v(t)$.
		\item Homomorphismen bilden Einparametergruppen auf Einparametergruppen ab:
		\[
			\Phi \enbrace*{h(s+t)} = \Phi \enbrace*{h(s) \cdot h(t)} = \Phi \enbrace*{h(s)} \Phi \enbrace*{h(t)}
		\]
		mit $\diffd{}{t}\big|_{t=0} (\Phi \circ h)(t) = \enbrace*{\mathd\Phi}_e \cdot h'(0)$.
		Aus der Eindeutigkeit folgt die Behauptung.
		\item Sei $i \colon H \to G$ die Inklusion.
		Dann ist $\exp_H(tw) = i \enbrace*{\exp_H(tw)} = \exp_G \enbrace*{(\mathd i)_e \cdot w \cdot t} = \exp_G(tw)$.\qedhere
	\end{enumerate}
\end{beweis}

Beispiel: Es gilt $\Tmap_e\SO(n) = \set*{V \in \Mat_n(\mathbb{R}) \given V^T =-V}$
\[
	\benbrace*{V_1,V_2}^T = \enbrace*{V_1 V_2 - V_2 V_1}^T = V_2 V_1 - V_1 V_2 = -\benbrace*{V_1,V_2}
\]
Aus (4) folgt $\exp(tv) \in \SO(n)$ für alle $t \in \mathbb{R}$. Anders: $\exp_H = \exp_G\big|_{\Tmap_e H} \colon \Tmap_e H \to H$.

Bemerkung: 
\begin{enumerate}[1)]
	\item ist $H$ eine Lieuntergruppe von $G$, so gilt $\exp_H  = \exp_B\big|_{\Tmap_e H} \colon \Tmap_e H \to H$
	\item Die Exponentialabbildung $\exp \colon \Tmap_e G \to G$ ist nicht immer surjektiv; richtig für kompakte zusammenhängende Liegruppen.
	Es gilt $\im \enbrace*{\exp (\Tmap_e G)} \supset U \supset \set*{e}$ für eine offene Umgebung von $e$.
	\item Es gilt $\exp(x)^{-1} = \exp(-x)$.
	\item Der Fluss eines linksinvarianten Vektorfeldes $X_v$ ist gegeben durch
	\[
		\enbrace*{R_{\exp(tv)}}_{t \in \mathbb{R}} \qquad X_v(g) = \enbrace*{\mathd L_g}_e \cdot v
	\]
\end{enumerate}
% section 14 (end)

\section{Die adjungierte Darstellung} % (fold)
\label{sec:15}
Sei $G$ eine Liegruppe und $g \in G$ ein festes Gruppenelement. Betrachte die Konjugation
\mapdef{i_g \colon G}{G}{\tilde{g}}{m \enbrace[\big]{m \enbrace*{g,\tilde{g}}, I(g)}= g \cdot \tilde{g} \cdot g^{-1}}{}
wobei bekanntermaßen $m(g,\tilde{g})= g \cdot \tilde{g}$ und $I(g)=g^{-1}$. 
Somit gilt
\[
	i_g = L_g \circ R_{g^{-1}} = R_{g^{-1}} \circ L_g
\]
Wir hatten bereits gesehen, dass $i_g \colon G \to G$ ein Liegruppenhomomorphismus ist.
Folglich ist 
\[
	\Ad(g) := \enbrace*{\mathd i_g}_e \colon \Tmap_e G \longrightarrow \Tmap_eG = \mathfrak{g}
\]
ein Liealgebrenhomomorphismus.
Wir hatten auch gezeigt, dass aus $i(g_1 \cdot g_2) = i(g_1) \cdot i(g_2)$ die Identität 
\[
	\Ad(g_1 \cdot g_2) = \Ad(g_1) \circ \Ad(g_2)
\]
folgt.

\begin{definition}[{name=[adjungierte Darstellung]}]
	Sei $G$ eine Liegruppe,
	Dann nennt man 
	\[
		\Ad \colon G \to \GL(\mathfrak{g}) \qquad g \longmapsto \Ad(g)
	\]
	die \Index{adjungierte Darstellung} von $G$.
\end{definition}

Ist zum Beispiel $G$ abelsch, so ist $i_g = \id_G$ und es folgt ${\Ad(g)} = \id_{\mathfrak{g}}$.
Ist $G=S^3$, so wurde in den Übungen die adjungierte Darstellung von $G$ bestimmt:
\[
	\Tmap_e S^3 = \im \mathbb{H} = \Span_\mathbb{R} (i,j,k) =: \mathbb{R}^3
\]
Es gilt $\Ad(g)(x) = g \cdot x \cdot g^{-1} \in \mathbb{R}^3$.
Es wurde auch $\Im \enbrace*{\Ad(S^3)} = \S0(3)$ gezeigt.

\begin{definition}
	Sei $G$ eine Liegruppe mit Liealgebra $\mathfrak{g}$.
	Dann bezeichnen wir mit 
	\[
		\ad_v \colon \mathfrak{g} \to \mathfrak{g} \qquad w \longmapsto \benbrace*{v,w}
	\]
	\emph{hier fehlt noch ein Satz irgendwie}
\end{definition}

Die Jacobi-Identität 
\[
	\benbrace*{X,\benbrace*{Y,Z}} + \benbrace*{Z, \benbrace*{X,Y}} + \benbrace*{Y, \benbrace*{Z,X}} =0
\]
für alle $X,Y,Z$ kann nun folgendermaßen beschrieben werden:\marginnote{Kommutator}
\[
	\ad_{\benbrace*{X,Y}} = \ad_X \circ \ad_Y - \ad_Y \circ \ad_X = \benbrace*{\ad_X, \ad_Y}_0
\]
Folglich ist $\ad \colon \enbrace*{\mathfrak{g}, \benbrace*{\cdot ,\cdot }} \to \enbrace*{\mathfrak{gl}(\mathfrak{g}), \benbrace*{\cdot ,\cdot }_0}$
ein Liealgebrenhomomorphismus.

\begin{lemma}
	Sei $G$ eine Liegruppe.
	Dann gilt
	\begin{enumerate}[(1)]
		\item $\enbrace*{\mathd \Ad}_{\id_{\mathfrak{g}}} = \ad$
		\item $i_g \enbrace*{\exp(v)} = \exp \enbrace*{\Ad(g)v}$
		\item $\Ad \enbrace*{\exp(v)} = e^{\ad_v} = \sum_{k=0}^{\infty} \frac{(\ad_v)^k}{k!} \colon \mathfrak{g} \to \mathfrak{g}$
		\item Ist $G$ zusammenhängend, so gilt $Z(G) = \ker (\Ad)$
	\end{enumerate}
\end{lemma}
\begin{beweis}
	\begin{enumerate}[(1)]
		\item Erinnerung: Ist $M^n$ eine differenzierbare Mannigfaltigkeit, $p \in M^n$ ein Punkt, $X,Y$ glatte Vektorfelder und $\enbrace*{\psi_t}_{\abs*{t}<\varepsilon}$ lokaler Fluss von $X$ in einer Umgebung von $p$.
		Dann gilt
		\[
			\benbrace*{X,Y}(p) = \diffd{}{t}\Big|_{t=0} \enbrace*{\mathd \psi_{-t}}_{\psi_t(p)} Y \enbrace*{\psi_t(p)}
		\]
		Es folgt für alle $w \in \mathfrak{g}$.
		\begin{align}
			\enbrace*{\mathd \enbrace*{\Ad(g)}_{\id_{\mathfrak{g}}} \cdot v} (w) = \diffd{}{t}\Big|_{t=0} \Ad \enbrace*{\exp(tv)}(w) &= \diffd{}{t}\Big|_{t=0} \enbrace*{\mathd R_{\exp(-tv)}}_{\exp{tv}} \cdot  \enbrace*{\mathd L_{\exp(tv)}}_{e} \cdot w \\
			&= \diffd{}{t}\Big|_{t=0} \enbrace*{\mathd\enbrace*{R_{\exp(-tv)}}}_{\exp{tv}} X_w \enbrace*{\exp(tv)} \\
			&= \benbrace*{X_v,X_w}(e)  = \benbrace*{v,w}_{\mathfrak{g}} \\
			&= \ad_v(w)
		\end{align}
		\item Folgt direkt aus \autoref{lem:143}.
		\item Folgt direkt aus \autoref{lem:143}.
		\item Zur Inklusion $Z(G) \subseteq ker (\Ad)$: Ist $g \in Z(g)$, so ist $i_g = {\id_G} $ und es folgt $\Ad(g)=\id$.
		
		Es gelte andersrum $\Ad(g) = \id_{\mathfrak{g}}$.
		Dann sind $i_g, \id_G \colon G \to G$ Liegruppenhomomorphismen mit $\enbrace*{\mathd i_g}_e = \mathd(\id_G)$.
		Da $G$ zusammenhängend ist, folgt aus \autoref{kor:126} $i_g=\id_G$ und es folgt $g \in Z(G)$.
	\end{enumerate}
\end{beweis}

Beispiel: Betrachte $G= \GL(n,\mathbb{R})$.
Dann gilt
\[
	\begin{array}{ll}
		i_A(\tilde{A}) = A \cdot \tilde{A} \cdot A^{-1} \quad & \tilde{A} \in \GL(n,\mathbb{R}) \\
		\Ad(A)(B) = A \cdot B \cdot A^{-1} & B \in \Mat(n,\mathbb{R}) \\
		\ad_v(W) = V \cdot W + W (-V) = \benbrace*{V,W}_0
	\end{array}
\]

\begin{lemma}[label=lem:154]
	Sei $G$ eine zusammenhängende Liegruppe. Dann gilt
	\begin{enumerate}[1)]
		\item $G$ ist abelsch genau dann, wenn $\Tmap_e G = \mathfrak{g}$ abelsch ist.
		\item $G$ ist genau dann abelsch, wenn $G \cong T^k \times \mathbb{R}^{n-k}$.
		\item Ist $H$ eine zusammenhängende Untergruppe von $G$, so ist $H$ Normalteiler genau dann, wenn $\mathfrak{h} = \Tmap_e H$ Ideal von $\mathfrak{g}$ ist, das heißt wenn $\benbrace*{\mathfrak{h},\mathfrak{h}} \subset \mathfrak{h}$ gilt.
		\item Das Zentrum $Z(G)$ ist Lieuntergruppe von $G$ mit 
		\[
			Z(\mathfrak{g}) = \Tmap_e Z(G) = \set*{v \in \mathfrak{g} \given \ad_v =0}
		\]
		\item Ist $H$ eine Lieuntergruppe von $G$, so ist die Einschränkung der adjungierten Darstellung von $G$ auf $H$ gerade die adjungierte Darstellung von $H$.
	\end{enumerate}
\end{lemma}
\begin{beweis}
	\begin{enumerate}[1)]
		\item Sei $G$ abelsch. Dann gilt $i_g=\id_G$, also $\Ad(g) = \id_{\mathfrak{g}}$ für alle $g \in G$ und es folgt $\ad_v=0$. Damit ist auch die Lieklammer trivial und $\mathfrak{g}$ ist abelsch.
		
		Sei andersrum $\mathfrak{g}$ abelsch: Dann ist $\ad_v=0$ für alle $v \in \mathfrak{g}$. Es gilt $\id_{\mathfrak{g}} = e^{\ad_v} = \Ad \enbrace*{\exp(v)}$
		Es folgt $\Ad(g) = \id_{\mathfrak{g}} $ für alle $g \in G$.
		\[
			i_g \enbrace*{\exp(v)} =  g \exp(v) g^{-1} = \exp \enbrace*{\Ad(g)v} = \exp(v)
		\]
		Somit ist $G$ abelsch.
		\item Da $G$ abelsch ist, ist $\mathfrak{g}$ abelsch und somit ist $\exp \colon (\Tmap_eG,+) \to (G,\cdot )$ ein Liegruppenhomomorphismus (wenn der Kommutator verschwindet gilt die Funktionalgleichung).
		Somit ist $\exp \colon \Tmap_e G \to G$ eine Überlagerung und $\Gamma := \exp^{-1}(e)$ eine diskrete Untergruppe von $(\Tmap_e G,+)$.\marginnote{Gitter}
		Somit existieren $v_1, \ldots, v_k \in \mathfrak{g}$ mit
		\[
			\Gamma = \set*{\sum\nolimits_{i=1}^{k} \lambda_i v_i \given \lambda_i \in \mathbb{Z}} \cong \mathbb{Z}^k
		\]
		Sei $V := \Span_\mathbb{R}(v_1,\ldots ,v_k)$ und $V^\bot$ ein Komplement von $V$ in $\mathfrak{g}$.
		Es gilt also $\mathfrak{g} = V \oplus V^\bot$ (beide Grenzfälle sind möglich).
		Es ist nun \enquote{klar}, dass 
		\[
			\im \enbrace*{\exp(V)} = T^k
		\]
		und $\im \enbrace*{\exp(V^\bot)} = \mathbb{R}^{n-k}$ gilt. 
		Die Behauptung folgt.
		\item Sei $H$ ein zusammenhängender Normalteiler von $G$, das heißt es gilt $g H g^{-1} \subset H$ für alle $g \in G$.
		Es folgt $i_g(H) \subset H$ und damit auch $\Ad(g)(\Tmap_eH) \subset \Tmap_eH$ für alle $g \in G$.
		Damit gilt auch $\ad_v(\Tmap_e H) \subset \Tmap_e H$, also $\benbrace*{\mathfrak{g},\mathfrak{h}}\subset \mathfrak{h}$.
		
		Ist umgekehrt $\mathfrak{h} = \Tmap_e H$ ein Ideal von $\mathfrak{g}$, so gilt $\ad_v(\mathfrak{h}) \subset \mathfrak{h}$ für alle $v \in \mathfrak{g}$.
		Dann gilt auch $\Ad \enbrace*{\exp (v)} = e^{\ad_v}(\mathfrak{h}) \subset \mathfrak{h}$.
		Damit gilt auch $\Ad(g)(\mathfrak{h}) \subset \mathfrak{h}$ für alle $g \in G$.
		\begin{align}
			H \ni \exp_G \enbrace*{\Ad(g)(w)} = i_g \enbrace*{\exp(w)}= g \cdot \exp(w) \cdot g^{-1} 
		\end{align}
		für alle $g \in G$ und $w \in \mathfrak{g}$.
		Damit ist $H$ ein Normalteiler.
		\item \emph{Übung!}
		\item \emph{Übung!}\qedhere
	\end{enumerate}
\end{beweis}

Frage: Wann sind Normalteiler abgeschlossene Untergruppen?
Für $r$ irrational ist
\[
	S^1_r = \set*{\enbrace*{e^{i \varphi}, e^{i r \varphi}} \in T^2}
\]
ist Normalteiler, aber $\overline{S^1_r}=T^2$.
% section 15 (end)

\section{Automorphismen von Liegruppen} % (fold)
\label{sec:16}

\begin{definition}
	Sei $G$ eine Liegruppe mit Liealgebra $\mathfrak{g}$.
	\begin{enumerate}[(i)]
		\item Dann nennt man einen linearen Isomorphismus $A \colon \mathfrak{g} \to \mathfrak{g}$ einen Liealgebra-Isomorphismus, falls 
	\[
		A \benbrace*{X,Y} = \benbrace*{AX,AY}
	\]
	für alle $X,Y \in \mathfrak{g}$ gilt.
	Diese Menge der Automorphismen bezeichnen wir mit $\Aut(\mathfrak{g}) \subset \GL(\mathfrak{g})$.
	\item Eine lineare Abbildung $A \colon \mathfrak{g} \to \mathfrak{g}$ nennt man \Index{Derivation}, falls 
	\[
		A \benbrace*{X,Y} = \benbrace*{AX,Y} + \benbrace*{X,AY}
	\] 
	für alle $X,Y \in \mathfrak{g}$ gilt. Bezeichnung: $\Der(\mathfrak{g}) \subset \End(\mathfrak{g})$.
	\end{enumerate}	
	Automorphismen sind Untergruppe von $\GL(\mathfrak{g})$
\end{definition}

\begin{proposition}[label=prop:162]
	Sei $\mathfrak{g}$ eine Liealgebra.
	Dann ist $\Aut(\mathfrak{g})$ eine abgeschlossene Untergruppe von $\GL(\mathfrak{g})$ mit Liealgebra $\Der(\mathfrak{g})$
\end{proposition}
\begin{beweis}
	Ist $A$ ein Automorphismus von $(\mathfrak{g},[\cdot ,\cdot ])$, so gilt für alle $x,y \in  \mathfrak{g}$
	\[
		A \benbrace*{x,y} = \benbrace*{A x, A y}
	\]
	somit ist $\Aut(\mathfrak{g})$ eine abgeschlossene Untergruppe und daher eine Lieuntergruppe von $\GL(\mathfrak{g})$.
	Sei $A(t)\big|_{t \in (-\varepsilon,\varepsilon)}$ eine differenzierbare Kurve von Automorphismen mit $A(0) = {\id_{\mathfrak{g}}}$.
	Differenzieren der Identität
	\[
		A(t) \benbrace*{x,y} = \benbrace*{A(t) x, A(t) y}
	\]
	 in $t=0$ liefert
	 \[
	 	A'(0) \benbrace*{x,y} = \benbrace*{A'(0) x , A'(0) y} + \benbrace*{x, A'(0)y}
	 \]
	 Also ist $A'(0) \in \Der(\mathfrak{g})$.
	 Sei umgekehrt $V \in \Der(\mathfrak{g})$.
	 Dann gilt\marginnote{$V \in \Der(\mathfrak{g}) \implies e^{tV} \in \End(\mathfrak{g})$}
	 \begin{align}
	 	\diffd{}{t}\Big|_{t=t_0} e^{-tV} \benbrace*{e^{tV}x, e^{tV}y} &= - e^{t_0 V} V \benbrace*{e^{t_0 V} x, e^{t_0 V} y} + e^{t_0 V} \benbrace*{V e^{t_0 V} x, e^{t_0 V} y} + e^{t_0 V} \benbrace*{e^{t_0 V} x , V e^{t_0 V} y} \\
		\intertext{mit $z := e^{t_0 V} x$ und $w := e^{t_0 V}y$}
		&= - e^{t_0 V} \enbrace[\big]{V \benbrace*{z,w} - \benbrace*{Vz,w} - \benbrace*{z,Vw}} =0
	 \end{align}
	 Somit ist wegen $e^{tV}={\id}$ für $t=0$
	 \[
	 	e^{tV} \benbrace*{e^{tV}x, e^{tV}y} \equiv \benbrace*{x,y}
	 \]
	 für alle $t \in (-\varepsilon,\varepsilon)$.
	 Wir erhalten also $\benbrace*{e^{tV}x, e^{tV}y}= e^{tV}\benbrace*{x,y}$.
\end{beweis}

Da $\Ad \colon G  \to \Aut(\mathfrak{g})$, $g \mapsto \Ad(g)$ ein Liegruppenhomomorphismus ist, ist $\ad \colon \mathfrak{g} \to \Der(\mathfrak{g})$, $v \mapsto \ad_v$ ein Liealgebrenhomomorphismus.
Insbesondere $\ad_v \in \Der(\mathfrak{g})$ für alle $v \in \mathfrak{g}$.

\begin{definition}
	Sei $\mathfrak{g}$ eine Liealgebra
	\begin{enumerate}[1)]
		\item Eine Derivation $A \in \Der(\mathfrak{g})$ nennt man \Index{innere Derivation}, falls $v \in \mathfrak{g}$ existiert mit $A = \ad_v$.
		Wir schreiben 
		\[
			\intAlg(\mathfrak{g}) \coloneqq \set*{\ad_v \given v \in \mathfrak{g}}
		\] 
		\item Wir bezeichnen mit $\Int(\mathfrak{g}) \subset \Aut(\mathfrak{g})$ die zusammenhängende Lieuntergruppe von $\Aut(\mathfrak{g})$ mit Liealgebra $\intAlg(\mathfrak{g})$.
		Elemente in $\Int(\mathfrak{g})$ nennt man \Index{innere Automorphismen}.
	\end{enumerate}
\end{definition}

Ist $G$ eine zusammenhängende Liegruppe, so gilt $\Ad(G) = \Int(\mathfrak{g})$, denn: Sowohl $\Ad(G)$ also auch $\Int(\mathfrak{g})$ sind Lieuntergruppen von $\Aut(\mathfrak{g})$ mit Liealgebra $\intAlg(\mathfrak{g})$.

Beispiel: Ist $\mathfrak{t}$ eine abelsche Liealgebra, so ist jeder Isomorphismus $A \colon \mathfrak{t} \to \mathfrak{t}$ ein Automorphismus.
Innere Automorphismen existieren in diesem auch nicht.

\begin{lemma}
	Sie $G$ eine zusammenhängende Liegruppe mit Liealgebra $\mathfrak{g}$.
	Dann ist $\Ad \colon G \to \Aut(\mathfrak{g})$ ein surjektiver Gruppenhomomorphismus mit Kern $Z(G)$.
	Es gilt also
	\[
		\Int(\mathfrak{g}) \cong \sfrac{G}{Z(G)} \quad \text{ und } \quad \intAlg(\mathfrak{g}) = \mathfrak(g) \ominus Z(\mathfrak{g})
	\]
\end{lemma}

\begin{lemma}[label=lem:165]
	Sei $G$ eine zusammenhängende Liegruppe.
	Dann ist $\Int(\mathfrak{g})$ ein Normalteiler von $\Aut(\mathfrak{g})_e$.
\end{lemma}
\begin{beweis}
	Nach Definition ist $\Int(\mathfrak{g})$ zusammenhängend und nach \autoref{lem:154} 3) ist somit $\Int(\mathfrak{g})$ genau dann Normalteiler, wenn $\Tmap_e \Int(\mathfrak{g}) = \intAlg{\mathfrak{g}}$ ein Ideal von $\aut(\mathfrak{g}) = \Tmap_e \Aut(\mathfrak{g})$ ist.
	Sei $L \in \aut(\mathfrak{g})$, $\ad_v \in \intAlg(\mathfrak{g})$.
	Dann gilt
	\begin{align}
		\benbrace*{L,\ad_v}_0 (w) = \enbrace*{L \circ \ad_v - \ad_v \circ L}(w) = L \benbrace*{v,w}- \benbrace*{v,Lw} \StackText{\ref{prop:162}}{=} \benbrace*{Lv, w} + \benbrace*{v,Lw} - \benbrace*{v,Lw} 
		= \ad_{Lv}(w)
	\end{align}
	Somit ist $\benbrace*{L, \ad_v} = \ad_{Lv} \in \intAlg(\mathfrak{g})$ für alle $L \in \Der(\mathfrak{g})$.
\end{beweis}

\begin{lemma}
	Sei $G$ eine zusammenhängende, einfach zusammenhängende Liegruppe.
	Dann gilt: Ist $H$ ein zusammenhängender Normalteiler von $G$, so ist $H$ eine abgeschlossene Untergruppe.
\end{lemma}
\begin{beweis}
	Nach \autoref{lem:154} 3) ist $\mathfrak{h} \coloneqq \Tmap_e H$ ein Ideal von $\mathfrak{g}$.
	Wir bezeichnen mit $\mathfrak{k} \coloneqq \sfrac{\mathfrak{g}}{\mathfrak{h}}$ die Quotientenliealgebra (!) und mit $K$ die einfach zusammenhängende Liegruppe mit Liealgebra $\mathfrak{k}$.
	Der Liealgebrenhomomorphismus $\psi \colon \mathfrak{g} \to \mathfrak{k}$, $v \mapsto [v]$ kann zu einem Liegruppenhomomorphismus $\phi$ \enquote{fortgesetzt} werden (da $G$ einfach zusammenhängend).
	Somit ist $\ker \phi$ eine abgeschlossene Untergruppe von $G$ mit Liealgebra $\ker \psi= \mathfrak{h}= \Tmap_e H$.
	Es folgt $H= \ker (\phi)_e$, das heißt $H$ ist abgeschlossen.
\end{beweis}

Achtung! $S^1_r \subset T^2 = S^1 \times S^1$ ist Normalteiler mit $\overline{S^1_r}= T^2$.

\begin{definition}[{name=[Killingform]}]
	Sei $\mathfrak{g}$ eine Liealgebra (über $\mathbb{R}$ oder $\mathbb{C}$).
	Dann nennt man 
	\mapdef{\mathcal{B} \colon \mathfrak{g} \times \mathfrak{g}}{\mathbb{K}}{(v,w)}{\tr_{\mathfrak{g}} \enbrace*{\ad_v \circ \ad_w}}{}
	die \Index{Killingform} von $\mathfrak{g}$.
\end{definition}

Beispiel: $G=\SO(n)$.
Dann ist 
\[
	\mathfrak{so}(n) = \Tmap_e\SO(n) = \set*{V \in \Mat(n,\mathbb{R}) \given V^T=-V}
\]
Wir identifizieren nun 
\[
	\begin{pmatrix}
		0 & & 1 \\
		& \ddots & \\
		-1 & & 0
	\end{pmatrix} \longleftrightarrow e_i \wedge e_j \in \Lambda^2(\mathbb{R}^n)
\]
wobei $i > j$ und $-1$ an der Stelle $(i,j)$ steht, $1$ an der Stelle $(j,i)$.
Versieht man $\mathfrak{g} = \mathfrak{so}(n)$ mit dem Skalarprodukt
\[
	\skal*{v}{w} = - \frac{1}{2}  \cdot \tr{V \cdot W} = \tr(V \cdot W^T)
\]
und $\Lambda^2(\mathbb{R}^n)$ mit dem Standardskalarprodukt ($(e_1,\ldots ,e_n)$ ONB von $\enbrace*{\mathbb{R}^n, \skal*{\cdot}{\cdot}_{\mathrm{std}}}$), so ist
\[
	e_1 \wedge e_2, e_1 \wedge e_3, \ldots , e_1 \wedge e_n, e_2 \wedge e_3, \ldots , e_{n-1} \wedge e_n
\]
ONB von $\benbrace*{\Lambda^2(\mathbb{R}^n), \skal*{\cdot}{\cdot}_{\mathrm{std}}}$, so ist obiger Isomorphismus eine Isometrie (!).
Es gilt\marginnote{jeweils wenn alle Indizes verschieden sind}
\[
	\benbrace*{e_i \wedge e_j , e_j \wedge e_k} = e_i \wedge e_k \qquad \benbrace*{e_i \wedge e_j, e_k \wedge e_l}= 0
\]
(nachrechnen)

Wir erhalten für eine ONB $\set*{b_1, \ldots ,b_N}$, $N=\frac{1}{2}(n-1)n$
\begin{align}
	\mathcal{B}(v,w) = \tr \enbrace*{\ad_v \circ \ad_w} &= \sum_{i=1}^{N} \skal*{(\ad_v \circ \ad_w)(b_i)}{b_i}_{\SO(n)} \\
	&\stackrel{!}{=} -\sum_{i<j} \skal*{ \benbrace*{w,e_i \wedge e_j}}{ \benbrace*{v, e_i \wedge e_j}} = - \ad_v
\end{align}
Somit ist $\mathcal{B}(e_k \wedge e_l, e_k \wedge e_k)=-2(n-2)$ und
\[
	\mathcal{B} \enbrace*{e_k \wedge e_l, e_{\tilde{k}} \wedge e_{\tilde{l}}} =0
\]
für $\set*{k,l} \neq \set*{\tilde{k},\tilde{l}}$.

Die Killingform $\mathcal{B}$ einer Liealgebra $\mathfrak{g}$ kann aber auch Null sein, ohne dass $\mathfrak{g}$ abelsch ist: 
\[
	H \coloneqq \set*{\begin{pmatrix}
		1 & 0 & 0 \\
		\alpha & 1 & 0 \\
		\beta & \gamma & 1
	\end{pmatrix} \given \alpha,\beta,\gamma \in \mathbb{R}}
\]
ist eine Untergruppe von $\GL(3,\mathbb{R})$ mit Liealgebra
\[
	\mathfrak{h} = \set*{\begin{pmatrix}
		0 & 0 & 0 \\
		x & 0 & 0 \\
		y & z & 0
	\end{pmatrix} \given x,y,z \in \mathbb{R}}
\]
Wir setzen 
\[
	e_1 \coloneqq \begin{pmatrix}
		0 & 0 & 0 \\
		1 & 0 & 0 \\
		0 & 0 & 0
	\end{pmatrix} \quad 
	e_2 \coloneqq \begin{pmatrix}
		0 & 0 & 0 \\
		0 & 0 & 0 \\
		0 & 1 & 0
	\end{pmatrix} \quad 
	e_3 \coloneqq \begin{pmatrix}
		0 & 0 & 0 \\
		0 & 0 & 0 \\
		1 & 0 & 0
	\end{pmatrix}
\]
Dann gilt $\benbrace*{e_i,e_3} = 0$ für $i=1,2,3$.
Aber $\benbrace*{e_1,e_2}=-e_3$.
Somit ist (bezüglich der Basis $(e_1,e_2,e_3)$)
\[
	\ad(e_1) = \begin{pmatrix}
		0 & 0 & 0 \\
		0 & 0 & 0 \\
		0 & 1 & 0
	\end{pmatrix} \quad 
	\ad(e_2) = \begin{pmatrix}
		0 & 0 & 0 \\
		0 & 0 & 0 \\
		1 & 0 & 0 
	\end{pmatrix} \quad 
	\ad(e_3) = \begin{pmatrix}
		0 & 0 & 0 \\
		0 & 0 & 0 \\
		0 & 0 & 0
	\end{pmatrix}
\]
Es gilt
\[
	\ad(e_i) \cdot \ad(e_j) = \begin{pmatrix}
		0 & 0 & 0 \\
		0 & 0 & 0 \\
		0 & 0 & 0
	\end{pmatrix}
\]
für alle $1 \le i,j \le 3$ und somit ist $\mathcal{B}=0$.
Die Gruppe $H$ heißt \Index{Heisenberggruppe}. 


\begin{lemma}[label=lem:168]
	Sei $\mathfrak{g}$ eine Liealgebra mit Killingform $\mathcal{B}$.
	Dann gilt
	\begin{enumerate}[1)]
		\item $\mathcal{B}(Av,Aw) = \mathcal{B}(v,w)$ für alle $v,w \in \mathfrak{g}$ und jeden Automorphismus $A$ von $\mathfrak{g}$.
		\item $\mathcal{B}(Lv,w) + \mathcal{B}(v,Lw) =0$ für alle $v,w \in \mathfrak{g}$ und jede Derivation $L$ von $\mathfrak{g}$.
	\end{enumerate}
\end{lemma}
\begin{beweis}
	\begin{enumerate}[1)]
		\item Sei $A \in \Aut(\mathfrak{g})$.
		Dann gilt 
		\[
			\ad_{Av}(w) = \benbrace*{Av,w} = \benbrace*{Av,A (A^{-1}w)} = A \benbrace*{v,A^{-1} w} = \enbrace*{A \circ \ad_v \circ A^{-1}}(w)
		\]
		Somit ist
		\begin{align}
			\tr \enbrace*{\ad_{Av} \circ \ad_{Aw}} &= \tr \enbrace*{A \circ \ad_v \circ A^{-1} \circ A \circ \ad_w \circ A^{-1}} \\
			&= \tr \enbrace*{\ad_v \circ \ad_w} = \mathcal{B}(v,w)
		\end{align}
		\item ist $L$ eine Derivation von $\mathfrak{g}$, so ist $e^{tL} \in \Aut(\mathfrak{g})$ für alle $t \in \mathbb{R}$.
		Nach 1) erhalten wir für alle $v,w \in \mathfrak{g}$
		\[
			\mathcal{B}(v,w) \equiv \mathcal{B}\enbrace*{e^{tL}v, e^{tL}w} 
		\]
		Differenzieren in $t=0$ liefert $0= \mathcal{B} \enbrace*{Lv,w} + \mathcal{B}(v,Lw)$.\qedhere
	\end{enumerate}
\end{beweis}

\begin{lemma}[label=lem:169]
	Sei $\mathfrak{g}$ eine Liealgebra mit Killingform $\mathcal{B}$.
	Dann ist 
	\[
		\ker(\mathcal{B}) \coloneqq \set*{v \in \mathfrak{g} \given \mathcal{B}(v,w)=0 \,\forall w \in \mathfrak{g}}
	\]
	ein Ideal von $\mathfrak{g}$
\end{lemma}
\begin{beweis}
	Sei $v \in \ker \mathcal{B}$ und $w \in \mathfrak{g}$.
	Zu zeigen: $\benbrace*{v,w} \in \ker \mathcal{B}$.
	Sei $z \in \mathfrak{g}$ beliebig.
	Dann gilt 
	\[
		\mathcal{B} \enbrace*{\benbrace*{v,w},z} = - \mathcal{B} \enbrace*{\ad_w(v),z} \StackText{\ref{lem:168}}{=} \mathcal{B} \enbrace*{v, \ad_w(z)} =0 
	\]
	Somit ist $\benbrace*{v,w} \in \ker \mathcal{B}$.
\end{beweis}

Wir werden später jeder Lie-Algebra zwei weitere Ideale zuordnen: Das \Index{Radikal} $\Rad(\mathfrak{g})$ und das \Index{Nilradikal} $\Nil(\mathfrak{g})$ 
% section 16 (end)
% chapter 1 (end)

\chapter{Strukturresultate} % (fold)
\label{cha:2}
\section{Nilpotente und auflösbare Liegruppen} % (fold)
\label{sec:21}
Sei $\mathfrak{g}$ eine Liealgebra über $\mathbb{R}$ (oder $\mathbb{C}$)

\begin{definition}
	Sei $k \in \mathbb{N}$.
	Dann definieren wir
	\[
		\mathfrak{g}^k \coloneqq \benbrace*{\mathfrak{g},\mathfrak{g}^{k-1}} \qquad \quad \mathfrak{g}_k \coloneqq \benbrace[\big]{\mathfrak{g}_{k-1},\mathfrak{g}_{k-1}}
	\] 
	wobei $\mathfrak{g}^0 \coloneqq \mathfrak{g}$ und $\mathfrak{g}_0 \coloneqq \mathfrak{g}$.
	Für Unterräume $V,W \subset \mathfrak{g}$ ist dabei $\benbrace*{V,W}$ die kleinste Unteralgebra von $\mathfrak{g}$, welche $\Span \set[\big]{\benbrace*{v,w} \given v \in V, w \in W}$ enthält.
\end{definition}

Sind $V,W$ Ideale von $\mathfrak{g}$, so gilt
\[
	\benbrace*{V,W} = \Span_\mathbb{K} \set[\big]{\benbrace*{v,w} \given v \in V, w \in W}
\]
denn $\benbrace*{\benbrace*{v_1,w_1}, \benbrace*{v_2,w_2}} \subset \Span_\mathbb{K} \set*{\benbrace*{v,w} \given v \in V, w \in W}$.

Sind $V,W$ Ideale, so auch $\benbrace*{V,W}$, denn für alle $z \in \mathfrak{g}$ gilt
\begin{align}
	\benbrace*{z, \benbrace*{v,w}} = - \benbrace*{w, \benbrace*{z,v}} - \benbrace*{v,\benbrace*{w,z}} \in \benbrace*{V,W}
\end{align}
Damit sind auch $\mathfrak{g}^k$ und $\mathfrak{g}_k$ Ideale von $\mathfrak{g}$. Es gilt $\mathfrak{g}^1= \benbrace*{\mathfrak{g}, \mathfrak{g}} \subseteq \mathfrak{g}^0 = \mathfrak{g}$ und weiter $\mathfrak{g}^k = \benbrace*{\mathfrak{g},\mathfrak{g}^{k-1}} \subseteq \mathfrak{g}^{k-1}$ und somit
\[
	\mathfrak{g}^0 \supseteq \mathfrak{g}^1 \supseteq \mathfrak{g}^2 \supseteq \mathfrak{g}^3 \supseteq \ldots 
\]
Dies ist die \Index{absteigende Zentralreihe}. Analog erhält man die \Index{abgeleitete Reihe} (auch: \Index{derivierte Reihe})
\[
	\mathfrak{g}_0 \supseteq \mathfrak{g}_1 \supseteq \mathfrak{g}_2 \supseteq \mathfrak{g}_3 \supseteq \ldots 
\]

\begin{definition}
	Eine Liealgebra $\mathfrak{g}$ nennt man
	\begin{enumerate}[1)]
		\item \Index{nilpotent}, falls $\mathfrak{g}^k=0$ und $\mathfrak{g}^{k-1}\neq 0$ für ein $k \in \mathbb{N}$
		\item \Index{auflösbar}, falls $\mathfrak{g}_k=0$ und $\mathfrak{g}_{k-1} \neq 0$ für ein $k \in \mathbb{N}$  
	\end{enumerate}
	Die obigen $k$ sind eindeutig bestimmt. 
	Man nennt $\mathfrak{g}$ dann $k$-Schritt nilpotent (auflösbar).
	Eine Liegruppe $G$ nennt man nilpotent (auflösbar), falls $\Tmap_e G$ nilpotent (auflösbar) ist.
\end{definition}

Beispiel: Die Heisenberggruppe ist $2$-Schritt nilpotent:
\[
	\mathfrak{g}_1 = \mathfrak{g}^1 = \benbrace*{\mathfrak{g},\mathfrak{g}} = \set*{\begin{pmatrix}
		0 & 0 & 0 \\
		0 & 0 & 0 \\
		w & 0 & 0
	\end{pmatrix} \given w \in \mathbb{R}}
\]
Somit ist offenbar $\mathfrak{g}^2=0$.

Sei 
\[
	S \coloneqq \set*{\begin{pmatrix}
		w_{11} & 0 & 0 \\
		w_{21} & w_{22} & 0 \\
		w_{31} & w_{32} & w_{33}
	\end{pmatrix} \given w_{ij} \in \mathbb{R}}
\]
Dann gilt $S_1= \benbrace*{S,S}\stackrel{!}{=} \mathfrak{g}$ und weiter
$S_2 = \benbrace*{S_1,S_1} = \benbrace*{\mathfrak{g},\mathfrak{g}} = \mathfrak{g}^1$ und somit $S_3 = \benbrace*{\mathfrak{g}^1,\mathfrak{g}^1} =0$.
Somit ist $S$ $3$-Schritt auflösbar.
Man zeigt aber leicht, dass $S$ \emph{nicht} nilpotent ist.

\begin{lemma}[label=lem:213]
	Sei $\mathfrak{g}$ eine Liealgebra mit entweder 
	\begin{itemize}
		\item $\mathfrak{g}^k=0$, $\mathfrak{g}^{k-1} \neq 0$ (nilpontent), oder
		\item $\mathfrak{g}_k=0$, $\mathfrak{g}_{k-1} \neq 0$ (auflösbar)
	\end{itemize}
	Dann gilt:
	\begin{enumerate}[1)]
		\item Für alle $i$ gilt $\mathfrak{g}_i \subseteq \mathfrak{g}^i$, also impliziert nilpotent auflösbar.
		\item Für alle $i$ gilt: $\mathfrak{g}^i$, $\mathfrak{g}_i$ sind Ideale
		\item Ist $\mathfrak{g}$ nilpotent, so gilt $\mathfrak{g}^{k-1} \subset Z(\mathfrak{g})$.
		Ist $\mathfrak{g}$ auflösbar, so ist $\mathfrak{g}_{k-1}$ abelsch.
		\item Unteralgebren von nilpotenten (auflösbaren) Liealgebren sind nilpotent (auflösbar)
		\item Ist $\mathfrak{h}$ ein Ideal in $\mathfrak{g}$, so ist $\mathfrak{k} \coloneqq \sfrac{\mathfrak{g}}{\mathfrak{h}}$ nilpotent (auflösbar), falls $\mathfrak{g}$ nilpotent (auflösbar) ist.
		\item Sei 
		\[
			\begin{tikzcd}
				0 \rar & \mathfrak{h} \rar["\Psi"] & \mathfrak{k} \rar["\Phi"] & \mathfrak{g} \rar & 0
			\end{tikzcd}
		\]
		eine kurze exakte Sequenz von Liealgebren.
		Sind $\mathfrak{h}$ und $\mathfrak{g}$ auflösbar, so auch $\mathfrak{k}$.
		\item Sind $\mathfrak{h}, \mathfrak{k}$ nilpotente (auflösbare) Ideale von $\mathfrak{g}$, so ist $\mathfrak{h} + \mathfrak{k}$ ein nilpotentes (auflösbares) Ideal von $\mathfrak{g}$.
	\end{enumerate}
\end{lemma}
\begin{beweis}
	\begin{enumerate}[1)]
		\item $\mathfrak{g}_0 = \mathfrak{g} = \mathfrak{g}^0$.
		Annahme $\mathfrak{g}_{i-1} \subseteq \mathfrak{g}^{i-1}$. Induktionsschritt:
		\[
			\mathfrak{g}_i = \benbrace[\Big]{\Underbrace{\mathfrak{g}_{i-1}}{\subseteq \mathfrak{g}}, \Underbrace{\mathfrak{g}_{i-1}}{\subseteq \mathfrak{g}^{i-1}}} \subset \benbrace*{\mathfrak{g},\mathfrak{g}^{i-1}} = \mathfrak{g}^i
		\]
		\item schon gezeigt
		\item Es ist $0 = \mathfrak{g}^k = \benbrace*{\mathfrak{g}, \mathfrak{g}^{k-1}}$, also $\mathfrak{g}^{k-1} \subset Z(\mathfrak{g})$.
		\item Auch klar: Da $\mathfrak{g}$ nilpotent ist, existiert $k$ mit $\mathfrak{g}^k =0$.
		Für ein Ideal $0 \neq \mathfrak{h}$ ist stets $\mathfrak{h}^i \subseteq \mathfrak{g}^i$ für alle $i$.
		\item Auf der Quotientenliealgebra $\mathfrak{k} = \sfrac{\mathfrak{g}}{\mathfrak{h}}$ hatten wir die Lieklammer definiert durch
		\[
			\benbrace[\big]{v +\mathfrak{h}, w + \mathfrak{h}}_{\mathfrak{k}} \coloneqq \benbrace*{v,w}_{\mathfrak{g}} + \mathfrak{h} 
		\]
		Damit folgt die Behauptung.
		\item Es seien $\mathfrak{h}$ und $\mathfrak{g}$ auflösbar. Per Idnuktion:
		Es gilt $\Phi(\mathfrak{k}) \subseteq \mathfrak{g}$ und\marginnote{da $\Phi$ surjektiv, ist $\Phi(\mathfrak{k}_i)$ ein Ideal} 
		\[
			\Phi(\mathfrak{k}_i) = \Phi \enbrace*{\benbrace*{\mathfrak{k}_{i-1}, \mathfrak{k}_{i-1}}} = \benbrace*{\Phi(\mathfrak{k}_{i-1}), \Phi(\mathfrak{k}_{i-1})}\StackText{I.V.}{\subseteq} \benbrace*{\mathfrak{g}_{i-1}, \mathfrak{g}_{i-1}} = \mathfrak{g}_i
		\]
		Somit $\Phi(\mathfrak{k}_{i_0})=0$ für ein $i_0 \in \mathbb{N}$ ($\mathfrak{g}$ auflösbar)
		Damit ist $\mathfrak{k}_{i_0} \subset \ker \Phi = \im \Psi$. 
		Es gilt
		\[
			\mathfrak{k}_{i_0 +1} = \benbrace*{\mathfrak{k}_{i_0}, \mathfrak{k}_{i_0}} \subseteq \benbrace*{\Psi(\mathfrak{h}), \Psi(\mathfrak{h})} \subseteq \Psi \benbrace*{\mathfrak{h},\mathfrak{h}} \subseteq \Psi(\mathfrak{h}_1)
		\]
		und 
		\[
			\mathfrak{k}_{i_0+2} = \benbrace*{\mathfrak{k}_{i_0}, \mathfrak{k}_{i_0}} \subseteq \benbrace*{\Psi(\mathfrak{h}_1), \Psi(\mathfrak{h}_1)} \subseteq \Psi \benbrace*{\mathfrak{h}_1, \mathfrak{h}_1} = \Psi(\mathfrak{h}_2)
		\]
		Da $\mathfrak{h}$ auflösbar ist, folgt die Behauptung.
		\item Seien $\mathfrak{h}$, $\mathfrak{k}$ auflösbare Ideale.
		Die Sequenz
		\[
			\begin{tikzcd}
				0 \rar & \mathfrak{h} \rar & \mathfrak{h} + \mathfrak{k} \rar & \sfrac{\mathfrak{h} + \mathfrak{k}}{\mathfrak{h}} \rar & 0
			\end{tikzcd}
		\]
		ist exakt.
		Nach 6) reicht es zu zeigen, dass $\sfrac{\mathfrak{h} + \mathfrak{k}}{\mathfrak{h}}$ auflösbar ist: 
		Es ist $\sfrac{\mathfrak{h}+ \mathfrak{k}}{\mathfrak{h}} = \sfrac{\mathfrak{k}}{\mathfrak{k} \cap \mathfrak{k}}$ und $\mathfrak{h} \cap \mathfrak{k}$ ist ein Ideal von $\mathfrak{g}$ und $\mathfrak{k}$.
		Aus 5) folgt, dass $\sfrac{\mathfrak{k}}{\mathfrak{h} \cap \mathfrak{k}}$ auflösbar ist.
		
		Seien nun $\mathfrak{h}, \mathfrak{k}$ nilpotente Ideale von $\mathfrak{g}$.
		Dann sind auch $\mathfrak{h}^i$ und $\mathfrak{k}^i$ nilpotente Ideale von $\mathfrak{g}$.
		Es gilt
		\[
			\benbrace*{\mathfrak{g}, \mathfrak{h}^1} = \benbrace*{\mathfrak{g}, \benbrace*{\mathfrak{h}, \mathfrak{h}}} \equiv \benbrace*{\mathfrak{h}, \benbrace*{\mathfrak{g},\mathfrak{h}}} \subset \mathfrak{h}^1
		\]
		Nun gilt \marginnote{mit $\mathfrak{h}^0=\mathfrak{g}$ und $\mathfrak{k}^0=\mathfrak{g}$}
		\[
			(\mathfrak{h} + \mathfrak{k})^1 = \benbrace*{\mathfrak{h} + \mathfrak{k}} \subseteq \benbrace*{\mathfrak{g},\mathfrak{h} + \mathfrak{k}} \subseteq \mathfrak{h} + \mathfrak{k} \subseteq \sum_{i=0}^{1} \mathfrak{h}^i \cap \mathfrak{k}^{1-i}
		\]
		Weiter ist 
		\begin{align}
			(\mathfrak{h} + \mathfrak{k})^2 \subseteq \benbrace*{\mathfrak{h} + \mathfrak{k},(\mathfrak{h} + \mathfrak{k})^1} &\subseteq \benbrace*{\mathfrak{h} + \mathfrak{k}, \sum_{i=0}^{1} \mathfrak{h}^1 \cap \mathfrak{k}^{1-i} } \\
			&\subseteq \sum_{i=0}^{2} \mathfrak{h}^i \cap \mathfrak{k}^{2-i}
		\end{align}
		Induktiv folgt
		\[
			(\mathfrak{h} + \mathfrak{k})^l \subset \sum_{i=0}^{l} \mathfrak{h}^i \cap \mathfrak{k}^{l-i}
		\]
		Die Behauptung folgt für ausreichend großes $l$.
	\end{enumerate}
\end{beweis}

\begin{beispiel*}
	\begin{itemize}
		\item \[
			\mathrm{o}(n,\mathbb{K}) = \set*{A \in \Mat(n,\mathbb{K}) \given A_{ij}=0 \, \forall 1 \le i<j \le n}
		\]
		ist eine auflösbare Liealgebra mit $\mathrm{o}(n,\mathbb{K})_n =0$, die \emph{nicht} nilpotent ist.
		\item $\mathrm{n}(n,\mathbb{K}) = \set*{A \in \mathrm{o}(n,\mathbb{K}) \given A_{ii}=0 \enspace i=1, \ldots ,n}$ ist eine nilpotente Liealgebra mit $\mathrm{n}(n,\mathbb{K})_{n-1}=0$
		\item Die Liealgebra 
		\[
			\mathfrak{g} = \set*{\begin{pmatrix}
				\alpha & \beta \\ 0 & 0
			\end{pmatrix} \given \alpha, \beta \in \mathbb{R}}
		\]
		ist auflösbar, aber nicht nilpotent(!).
		Ferner ist die Sequenz
		\[
			\begin{tikzcd}
				0 \rar & wird stressig
			\end{tikzcd}
		\]
		\todo{Sequenz fertig machen mit Matrizen in tikzcd}
		exakt.
		Ferner sind $1$-dimensionale Liealgebren abelsch, somit nilpotent.
		Somit ist \autoref{lem:213} 6) nicht richtig für nilpotente Algebren.
	\end{itemize}
\end{beispiel*}

\begin{definition}
	Sei $\mathfrak{g}$ eine Liealgebra.
	\begin{enumerate}[1),itemsep=0pt]
		\item Das \Index{Radikal} $\rad (\mathfrak{g})$ von $\mathfrak{g}$ ist das eindeutig bestimmte maximale auflösbare Ideal von $\mathfrak{g}$.
		\item Das \Index{Nilideal} $\nil (\mathfrak{g})$ von $\mathfrak{g}$ ist das eindeutig bestimmte maximale nilpotente Ideal von $\mathfrak{g}$.
		\item Man nennt $\mathfrak{g}$ \Index{halbeinfach}, falls $\rad(\mathfrak{g}) = \set*{0}$ ist.
		\item Man nennt $\mathfrak{g}$ reduktiv, falls $\nil(\mathfrak{g})= \set*{0}$ ist.
		\item Man nennt $\mathfrak{g}$ \Index{einfach}, falls $\set*{0}$ und $\mathfrak{g}$ die einzigen Ideale von $\mathfrak{g}$ sind und $\dim \gamma > 1$ ist.
	\end{enumerate}
\end{definition}

\begin{bemerkung*}
	\begin{itemize}
		\item Es ist $\nil(\mathfrak{g}) \subseteq \rad(\mathfrak{g}) \subseteq \mathfrak{g}$
		\item Eine Liealgebra $\mathfrak{g}$ ist genau dann halbeinfach, wenn $\mathfrak{g}$ keine abelschen Ideale besitzt:
		
		Wenn $0\neq\mathfrak{k}$ ein abelsches Ideal ist, so ist $\rad(\mathfrak{g}) \neq 0$.
		Ist andersrum $\rad(\mathfrak{g})\neq 0$, so ist $\rad(\mathfrak{g})_{k-1}$ ein abelsches Ideal.
	\end{itemize}
\end{bemerkung*}

\begin{proposition}
	Sei $\mathfrak{g}$ eine Liealgebra.
	Dann ist $\overline{\mathfrak{g}} \coloneqq \sfrac{\mathfrak{g}}{\rad(\mathfrak{g})}$ halbeinfach.
\end{proposition}
\begin{beweis}
	Die Quotientenabbildung $\pi \colon \mathfrak{g} \to \overline{\mathfrak{g}}$ ist ein Liealgebrenhomomorphismus.
	Somit ist für jedes Ideal $\overline{\mathfrak{i}} \subset \overline{\mathfrak{g}}$ das Urbild $\mathfrak{i} = \pi^{-1}(\overline{\mathfrak{i}})$ ebenfalls ein Ideal:
	\[
		\benbrace*{\pi^{-1}(\overline{\mathfrak{i}}), \mathfrak{g}} \subset \pi^{-1}(\overline{\mathfrak{i}}) \iff \pi \enbrace*{\benbrace*{\pi^{-1}(\overline{\mathfrak{i}}), \mathfrak{g}}} \subset \overline{\mathfrak{i}}
	\]
	Weiter ist $\benbrace*{\overline{\mathfrak{i}}, \overline{\mathfrak{g}}} \subset \overline{\mathfrak{i}}$ ein Ideal und somit ist die Sequenz
	\[
		\begin{tikzcd}
			0 \rar & \rad(\mathfrak{g}) \rar & \pi^{-1}(\overline{\mathfrak{i}}) \rar &\overline{\mathfrak{i}} \rar & 0
		\end{tikzcd}
	\] 
	exakt.
	Ist also $\overline{\mathfrak{i}}$ ein auflösbares Ideal von $\overline{\mathfrak{g}}$, so ist nach \autoref{lem:213} 6) auch $\pi^{-1}(\overline{\mathfrak{i}})$ ein auflösbares Ideal von $\mathfrak{g}$.
	Aus der Maximalität von $\rad(\mathfrak{g})$ folgt $\overline{\mathfrak{i}}=0$.
\end{beweis}

\begin{satz}[{name={Levi-Mallev}},label=satz:216]
	Sei $\mathfrak{g}$ eine Liealgebra.
	Dann existiert eine bis auf innere Automorphismen eindeutige halbeinfache Liealgebra $\mathfrak{h} \subset \mathfrak{g}$ mit 
	\[
		\mathfrak{g} = \mathfrak{h} \oplus  \rad(\mathfrak{g})
	\]
\end{satz}

\begin{lemma}[label=lem:217]
	Sei $\mathfrak{g} \subset \mathfrak{gl}(V)$ eine Lieunteralgebra bestehend aus nilpotenten Elementen, das heißt für alle $A \in \mathfrak{g}$ existiert ein $k \in \mathbb{N}$ mit $A^k=0$.
	Dann existiert ein Vektor $v \in V \setminus \set*{0}$ mit $A \cdot v = 0$ für alle $A \in \mathfrak{g}$.
\end{lemma}
\begin{beweis}
	\emph{Vorüberlegung:} Sei $X \in \mathfrak{g}$. Dann ist $\ad_X(Y) = X Y -Y X$ und es gilt
	\begin{align}
		(\ad_X)^2(Y) &= X \enbrace*{X Y -YX} - \enbrace*{XY -YX}X = X^2 Y - 2XYX+ YX^2 \\
		(\ad_X)^3(Y) &= X \enbrace*{X^2 Y - 2XYX+ YX^2}- \enbrace*{X^2 Y - 2XYX+ YX^2}X \\
		&= X^3 y - 3 X^2YX+3 XYX^2 - YX^3
	\end{align}
	Induktiv zeigt man: Gilt $X^k=0$, so folgt $(\ad_X)^{2k}=0$ (!).
	
	Nur zur eigentlichen Behauptung: Per Induktion nach $n=\dim \mathfrak{g}$. 
	Für $n=1$ sei $A \in \mathfrak{g} \setminus \set*{0}$.
	Ist $A$ invertierbar, so auch $A^k$ für alle $k \in \mathbb{N}$, was im Widerspruch zur Nilpotenz von $A$ steht.
	Für den Induktionsschritt sei $\dim \mathfrak{g} = n+1$.
	Sei $\mathfrak{h}$ eine Unteralgebra maximaler Dimension ($\le n$) von $\mathfrak{g}$ (1-dimensionlae Unterräume sind Unteralgebren).
	Wir werden zeigen, dass $\dim \mathfrak{h}=n$ ist.
	
	Wegen $\ad_X(\mathfrak{h}) \subset \mathfrak{h}$ für alle $X \in \mathfrak{h}$ gilt nach der Zerlegung aus \autoref{satz:216}
	\[
		\End(\mathfrak{g}) \ni \ad_X = \begin{pmatrix}
			A_{11} & A_{12} \\ 0 & A_{22}
		\end{pmatrix}
	\]
	Beobachtung:
	\[
		\begin{pmatrix}
			A_{11} & A_{12} \\ 0 & A_{22}
		\end{pmatrix}^2 = \begin{pmatrix}
			A_{11}^2 & * \\ 0 & A_{22}^2
		\end{pmatrix}
	\]
	und so weiter. 
	Damit ist $\ad_X$ nilpotent und somit auch $\ad_X|_{\sfrac{\mathfrak{g}}{\mathfrak{h}}}$.
	Ferner gilt
	\[
		\benbrace*{\begin{pmatrix}
			A_{11} & A_{12} \\ 0 & A_{22}
		\end{pmatrix}, \begin{pmatrix}
			B_{11} & B_{12} \\ 0 & B_{22}
		\end{pmatrix}} = \begin{pmatrix}
			\benbrace*{A_{11},B_{11}} & * \\ 0 & \benbrace*{A_{22},B_{22}}
		\end{pmatrix}
	\]
	Somit ist 
	\[
		\ad(\mathfrak{g}) = \set*{(\ad_X)\big|_{\sfrac{\mathfrak{g}}{\mathfrak{h}}} \given X \in \mathfrak{h}}
	\]
	eine nilpotente Unteralgebra von $\mathfrak{gl}(\sfrac{\mathfrak{g}}{\mathfrak{h}})$.
	Wegen $\dim \sfrac{\mathfrak{g}}{\mathfrak{h}} \le n$ existiert nach Induktionsvorraussetzung $\overline{A} \in \sfrac{\mathfrak{g}}{\mathfrak{h}}$ mit $(\ad_X)|_{\sfrac{\mathfrak{g}}{\mathfrak{h}}}(\overline{A})=0$ für alle $X \in \mathfrak{h}$.
	Somit existiert $A \in \mathfrak{g}$ mit 
	\begin{equation}
		\ad_X(A) \subset \mathfrak{h} \label{eq:217} \tag{*}
	\end{equation}
	für alle $X \in \mathfrak{h}$.
	Somit ist $\mathfrak{h} \oplus \langle A \rangle_\mathbb{R}$ eine Unteralgebra von $\mathfrak{g}$ (mit \eqref{eq:217} folgt $\benbrace*{\mathfrak{h},A}\subset \mathfrak{h}$).
	Da $\dim \mathfrak{h}$ maximal war, folgt $h \oplus \langle A \rangle_\mathbb{R} = \mathfrak{g}$ und somit auch $\dim \mathfrak{h}=n$.
	Obige Behauptung folgt.
	
	Zurück zur Induktion:
	Wir setzen
	\[
		W \coloneqq \set*{v \in V \given A(v)=0 \, \forall A \in \mathfrak{h}}
	\]
	mit $\mathfrak{h} \subsetneq \mathfrak{g}$, $\dim \mathfrak{h}=n$ Unteralgebra.
	Da $\mathfrak{h}$ als Teilmenge von $\mathfrak{g}$ aus nilpotenten Elementen besteht, gilt nach Induktionsvorraussetzung $W \neq \set*{0}$.
	Der Endomorphismus $A \in \mathfrak{g} \subset \mathfrak{gl}(V)$ lässt $W$ invariant:
	\[
		0 = \Underbracket{\benbrace*{X,A}}{\StackText{\eqref{eq:217}}{\in} \mathfrak{h}}(w) = X \enbrace*{A(w)} - A \enbrace*{X(w)} = X (A(w))
	\]
	für alle $X \in \mathfrak{h}$ und $w \in W$. Damit ist $A(W)\subset W \subset V$.
	Wir schreiben nun mit $V= W \oplus W^\bot$
	\[
		A = \begin{pmatrix}
			A|_W & * \\ 0 & A|_{W^\bot}
		\end{pmatrix}
	\]
	Da $A|_W$ nilpotent ist, existiert ein Vektor $w \in W$ mit $A(w)=0$ und nach Definition von $W$ gilt $X(w)=0$ für alle $X \in \mathfrak{h}$.
	Wegen $\mathfrak{g} = \mathfrak{h} \oplus \langle A \rangle_\mathbb{R}$ folgt die Behauptung.
\end{beweis}

\begin{satz}[name={Engel},label=satz:218]
	Sei $\mathfrak{g} \subset \mathfrak{gl}(V)$ eine Unteralgebra bestehend aus nilpotenten Elementen.
	Dann existiert eine Basis von $V$, sodass 
	\[
		\mathfrak{g} \subset N(n,\mathbb{K})  
	\]
	gilt mit $n = \dim V$.
\end{satz}
\begin{beweis}
	Sei $v \in V \setminus \set*{0}$ mit $X(v)=0$ für alle $X \in \mathfrak{g}$ nach \autoref{lem:217}.
	Sei $\overline{V} \coloneqq \sfrac{V}{\langle v \rangle_\mathbb{R}}$.
	Wir setzen $X(\overline{w}) \coloneqq X(w) + \langle v \rangle_\mathbb{R}$.
	Dies ist wohldefiniert, wie man sich leicht überlegt.
	Wir erhalten eine Liealgebra $\overline{\mathfrak{g}} \subset \mathfrak{gl}(V)$ von nilpotenten Elementen(!).
	
	Nach Induktionsvoraussetzung existieren Vektoren $\overline{v}_2, \ldots , \overline{v}_n \in \overline{V}$ mit $\overline{\mathfrak{g}} \subset \mathrm{n}(n-1,\mathbb{K})$.
	Wir setzen $v_1 \coloneqq v$ und wählen Urbilder $v_2, \ldots ,v_n \in V$ von $\overline{v}_1, \ldots ,\overline{v}_n$.
	Es gilt für alle $X \in \mathfrak{g}$
	\[
		X(v_1) =0 \quad \text{ und } \quad X(v_2) \in \langle v_1 \rangle_\mathbb{R}
	\]
	denn sonst $X(v_2) = \alpha \cdot v_1 + w$ und $X(\overline{v}_2)=\overline{w}\neq 0$. USW \ldots 
\end{beweis}

\begin{korollar}[label=korr:219]
	Eine Liealgebra $\mathfrak{g}$ ist genau dann nilpotent, wenn 
	\[
		\ad(\mathfrak{g}) = \set*{\ad_X \given X \in \mathfrak{g}} \subset \End(\mathfrak{g})
	\]
	eine nilpotente Unteralgebra von $\End(\mathfrak{g})$ ist.
\end{korollar}
\begin{beweis}
	Sei zunächst $\mathfrak{g}$ nilpotent und $X \in \mathfrak{g}$.
	Dann gilt $\ad_X(\mathfrak{g}^k) \subset \mathfrak{g}^{k+1}$.
	Somit ist $\ad_X$ nilpotent.
	
	Ist umgekehrt $\ad(\mathfrak{g}) \subset \End(\mathfrak{g})$ nilpotent, so existiert nach \autoref{satz:218} eine Basis $v_1, \ldots ,v_n$ von $\mathfrak{g}=V$ mit $\ad_X(v_i) \in V_{i-1} = \langle v_1 ,\ldots ,v_{i-1}\rangle_\mathbb{R}$ für alle $i=1,\ldots ,n$ und alle $X \in \mathfrak{g}$, $V_0 \coloneqq \set*{0}$.
	Für $X_1, \ldots ,X_s \in \mathfrak{g}$ erhalten wir 
	\[
		 \benbrace*{X_1, \benbrace*{X_2, \benbrace*{\ldots ,X_s}}}=\enbrace*{\ad_{X_1} \circ \ldots \circ \ad_{X_s}}(\mathfrak{g}) \in V_{n-2}
	\]
	Da $V_0=0$ ist, ist $\mathfrak{g}$ nilpotent.
\end{beweis}

Weitere Folgerung aus \autoref{satz:218}:

\begin{korollar}
	Eine nilpotente Liealgebra ist isomorph zu einer Unteralgebra von $\mathfrak{n}(n,\mathbb{K})$.
\end{korollar}
\begin{beweis}
	Aus dem Satz von Ado folgt, dass wir $\mathfrak{g} \subset \mathfrak{gl}(n,\mathbb{K})$ annehmen können. 
	Mit \autoref{satz:218} folgt dann die Behauptung.
\end{beweis}

Wir kommen nun zu auflösbaren Liealgebra und notieren die folgenden Aussagen ohene Beweise; diese sind ähnlich wie im nilpotenten Fall (Procesi Liegroups)\todo{Ref}.

\begin{satz}[{name={Lie}},label=satz:2111]
	Sei $\mathfrak{g}$ eine komplexe Lieunteralgebra von $\mathfrak{gl}(V,\mathbb{C})=\End(V,\mathbb{C})$.
	ist $\mathfrak{g}$ auflösbar, so existiert eine 1-Form $\lambda \colon \mathfrak{g} \to \mathbb{C}$ und ein Vektor $v \in V\setminus \set*{0}$ mit $A(v)= \lambda(A) \cdot v$ für alle $A \in \mathfrak{g} \subset \mathfrak{gl}(V,\mathbb{C})$ ($\mathfrak{g}$ nilpotent $\implies \lambda=0$).
\end{satz}

\begin{bemerkung*}
	Die Aussage ist nicht richtig im reellen Fall. 
\end{bemerkung*}

\begin{korollar}
	Sei $\mathfrak{g} \subset \mathfrak{gl}(V,\mathbb{C})$ eine komplexe, auflösbare Liealgebra.
	Dann gilt $\mathfrak{g} \subset \mathfrak{o}(V,\mathbb{C})$.
\end{korollar}

Aus Ado's Satz folgt wieder:
\begin{korollar}[label=korr:2113]
	Jede komplexe auflösbare Liealgebra ist Unteralgebra von $\mathfrak{o}(n,\mathbb{C})$ für ein $n \in \mathbb{N}$.
\end{korollar}

\begin{korollar}[label=korr:2114]
	Sei $\mathfrak{g}$ eine komplexe Liealgebra.
	Dann ist $\mathfrak{g}$ genau dann auflösbar, wenn $\benbrace*{\mathfrak{g},\mathfrak{g}}$ nilpotent ist.
\end{korollar}
\begin{beweis}
	Sei zunächst $\mathfrak{g}$ auflösbar.
	Nach \autoref{korr:2113} können wir $\mathfrak{g} \subset \mathfrak{o}(n,\mathbb{C})$ annehmen.
	Wegen $\benbrace*{\mathfrak{o}(n,\mathbb{C}), \mathfrak{o}(n,\mathbb{C})} \subset \mathfrak{n}(n,\mathbb{C})$ folgt die Behauptung.
	\[
		hier fehlt noch was
	\]
	Ist andersrum $\benbrace*{\mathfrak{g},\mathfrak{g}}$ nilpotent, so ist nach \autoref{korr:2113} $\benbrace*{\mathfrak{g},\mathfrak{g}}$ auflösbar und somit auch $\mathfrak{g}$.
\end{beweis}

Kleiner Exkurs: Komplexifizierung

Sei $\mathfrak{g}$ eine reelle Liealgebra, das heißt $\mathfrak{g}$ ist ein $\mathbb{R}$-Vektorraum versehen mit Lieklammer $\benbrace*{\cdot ,\cdot }_\mathbb{R} \colon \mathfrak{g} \times \mathfrak{g} \to \mathfrak{g}$.
Dann ist 
\[
	\mathfrak{g}_\mathbb{C} \coloneqq \mathfrak{g} \otimes_\mathbb{R} \mathbb{C} = \set[\big]{v \otimes 1 + \tilde{v} \otimes i \given v , \tilde{v} \in \mathfrak{g}}
\]
ein komplexer Vektorraum mit $\dim_\mathbb{R} \mathfrak{g} = \dim_\mathbb{C} \mathfrak{g}_\mathbb{C}$.
Wir setzen die Lieklammer komplex linear auf $\mathfrak{g}_\mathbb{C}$ fort mittels
\[
	\benbrace*{v_1 \otimes \lambda_1, v_2 \otimes \lambda_2}_\mathbb{C} \coloneqq \benbrace*{v_1,v_2}_\mathbb{R} \otimes (\lambda_1 \lambda_2)
\]
Man nennt $\enbrace*{\mathfrak{g}_\mathbb{C}, \benbrace*{\cdot ,\cdot }_\mathbb{C}}$ \Index{Komplexifizierung} von $\enbrace*{\mathfrak{g}, \benbrace*{\cdot ,\cdot }}$.
Man beachte, dass unterschiedliche reelle Liealgebren die gleiche Komplexifizierung besitzen können. 
Man nennt alle diese reellen Liealgebren dann \Index{reelle Formen} von $\mathfrak{g}_\mathbb{C}$.

\begin{beispiel*}
	\begin{itemize}
		\item Es gilt $\SL(2,\mathbb{R}) \otimes \mathbb{C} = \SU(2,\mathbb{C}) \otimes \mathbb{C} = \SL(2,\mathbb{C})$.
		\item $\SU(2) = \set*{A \in \GL(2,\mathbb{C}) \given A^* A = \begin{pmatrix}
			1 & 0 \\ 0 & 1
		\end{pmatrix}, \det A = 1}$ ist kompakte Liegruppe.
		\item $\SL(2,\mathbb{R})$ ist nicht kompakte Liegruppe.
	\end{itemize}
\end{beispiel*}

Dazu ein kleines Lemma: Sei $\mathfrak{g}$ eine reelle Liealgebra.
Dann gilt
\begin{enumerate}[1)]
	\item $\mathfrak{h}$ ist Unteralgebra von $\mathfrak{g}$ genau dann, wenn $\mathfrak{h}_\mathbb{C}$ Unteralgebra von $\mathfrak{g}_\mathbb{C}$ ist.
	\item Es gilt
	\[
		\enbrace*{\mathfrak{g}î}_\mathbb{C} = \enbrace*{\mathfrak{g}_\mathbb{C}}^i \qquad \enbrace*{\mathfrak{g}_i}_\mathbb{C} = \enbrace*{\mathfrak{g}_\mathbb{C}}_i
	\]
	\item Killingform: $\mathcal{B}_\mathbb{C} \enbrace*{v_1 \otimes \lambda_1, v_2 \otimes \lambda_2} = \mathcal{B}_\mathbb{R}(v_1,v_2) \cdot \lambda_1 \lambda_2 \in \mathbb{C}$
\end{enumerate}
Somit entsprechen sich nilpotente, auflösbare und halbeinfache reelle Liealgebren und ihre Komplexifizierung.

\begin{satz}[name={Cartan},label=satz:2115]
	Sei $\mathfrak{g}$ eine Liealgebra.
	Dann ist $\mathfrak{g}$ genau dann auflösbar, wenn die Killingform $\mathcal{B}$ von $\mathfrak{g}$ auf $\benbrace*{\mathfrak{g},\mathfrak{g}}$ Null ist.
\end{satz}
\begin{beweis}
	Sei zunächst $\mathfrak{g}$ auflösbar.
	Dann ist nach \autoref{korr:2114} $\benbrace*{\mathfrak{g},\mathfrak{g}}$ nilpotent.
	Somit ist $\mathcal{B}_{\mathfrak{g},\mathfrak{g}}=0$ nach \autoref{korr:219} ($\benbrace*{\mathfrak{g},\mathfrak{g}} \subset \mathfrak{n}(n,\mathbb{\mathbb{K}})$).
	Da $\benbrace*{\mathfrak{g},\mathfrak{g}}$ ein Ideal von $\mathfrak{g}$ ist, gilt 
	\[
		\mathcal{B}_{\mathfrak{g}}\big|_{\benbrace*{\mathfrak{g},\mathfrak{g}}} = \mathcal{B}_{\benbrace*{\mathfrak{g},\mathfrak{g}}} =0
	\]
	(Komplexifiziere zuerst! zunächst ins Reelle).
	
	Die andere Implikation findet man in Procesi S. 95-97.
\end{beweis}

Wir erhalten $\rad(\mathfrak{g}) \supseteq \ker \mathcal{B} \supseteq \nil(\mathfrak{g})$, dabei ist $\ker \mathcal{B}$ auflösbar.
Ist die Killingform $0$, so ist $\mathfrak{g}$ auflösbar, aber nicht notwendig nilpotent.
% section 21 (end)

\section{Halbeinfache Liealgebren} % (fold)
\label{sec:22}

\begin{erinnerung}
	Eine Liealgebra $\mathfrak{g}$ ist halbeinfach, genau dann, wenn $\mathfrak{g}$ kein abelsches Ideal hat.
\end{erinnerung}

\begin{satz}[{name={Cartan}}]
	Eine Liealgebra $\mathfrak{g}$ ist halbeinfach genau dann, wenn die Killingform $\mathcal{B}$ nicht degeneriert ist.
\end{satz}
\begin{beweis}
	Sei zunächst $\mathfrak{g}$ halbeinfach.
	Zu zeigen ist, dass $\ker \mathcal{B}= 0$ ist.
	Angenommen dies gilt nicht, so $\ker \mathcal{B}$ nach \autoref{satz:2115} ein auflösbares Ideal und es folgt $\rad(\mathfrak{g}) \neq 0$, was einen Widerspruch darstellt.
	
	Ist umgekehrt $\mathcal{B}$ nicht degeneriert, so nehmen wir an, dass $\mathfrak{a}$ ein abelsches Ideal von $\mathfrak{g}$ ist.
	Wir wählen eine Basis $a_1, \ldots ,a_k, b_1, \ldots ,b_s$ von $\mathfrak{g}$ mit $a = \langle a_1, \ldots , a_k\rangle_\mathbb{K}$.
	Dann gilt $(\ad_Y \circ \ad_X)(a_i) = \benbrace[\big]{Y,\benbrace*{X,a_i}} = 0$ und weiter
	\[
		\enbrace*{\ad_Y \circ \ad_X}(b_i) = \benbrace*{Y, \benbrace*{X,b_i}} \in  \mathfrak{a}
	\]
	Also ist für $X \in \mathfrak{a}$ und $Y \in \mathfrak{g}$ obige Verknüpfung eine strikte obere Dreiecksmatrix.
	Damit ist $\mathcal{B}(Y,X)= \Span (\ad_Y \circ \ad_X)=0$ und somit $\ker \mathcal{B} \neq 0$, ein Widerspruch!
\end{beweis}

Halbeinfache Liegruppen sind $\SL(n,\mathbb{R}), \SL(n,\mathbb{C}), \SO(n), \SU(n), \Sp(n), \ldots $\marginnote{$\mathfrak{g} = \mathfrak{g}_{SS} \oplus \rad(\mathfrak{g})$}

Man kann zeigen, dass 
\[
	\mathcal{B}_{\mathfrak{g}} \enbrace*{x,y} = \Underbracket{\alpha_\mathfrak{g}}{\neq 0} \cdot \tr(x \cdot y)
\]
Folglich hat die Killingform auf $\SL(n,\mathbb{R})$ positive und negative \enquote{Eigenwerte}.

In allen Beispielen gilt
\[
	X \in \mathfrak{g} \implies X^T \in \mathfrak{g}
\]
Somit ist $\mathcal{B}(X,X^T) = \alpha_\mathfrak{g} \cdot \sum X_{ii}^2 \neq 0$ für $x\neq 0$.
Ist etwa $G$ kompakt, so ist $\mathcal{B}_\mathfrak{g}$ negativ definit!

\begin{satz}[label=satz:222]
	Sei $\mathfrak{g}$ eine halbeinfache Liealgebra.
	Dann gilt
	\begin{enumerate}[1)]
		\item $\mathfrak{g} = \mathfrak{g}_1 \oplus  \ldots \oplus \mathfrak{g}_s$, wobei die $\mathfrak{g}_i$ einfache Ideale sind, $\benbrace*{\mathfrak{g}_i,\mathfrak{g}_j}=0$ für $i \neq j$
		\item Ist $\mathfrak{a} \subset \mathfrak{g}$ ein Ideal, so folgt $\mathfrak{a} = \mathfrak{g}_{i_1} \oplus  \ldots \oplus \mathfrak{g}_{i_k}$
		\item $\benbrace*{\mathfrak{g},\mathfrak{g}}=\mathfrak{g}$
		\item $\sfrac{\Aut(\mathfrak{g})}{\Int(\mathfrak{g})}$ ist diskret.
	\end{enumerate}
\end{satz}
\begin{beweis}
	Vorüberlegungen:
	\begin{enumerate}[i)]
		\item \label{enum:222:1} Ist $\mathfrak{a}$ ein Ideal von $\mathfrak{g}$, so auch 
		\[
			\mathfrak{a}^\perp \coloneqq \set[\big]{x \in \mathfrak{g} \given \mathcal{B}(x,y) =0 \,\forall y \in \mathfrak{a}}
		\]
		Sei $x \in \mathfrak{a}^\perp$ und $y \in \mathfrak{g}$. Zu zeigen: $\benbrace*{x,y} \in \mathfrak{a}^\perp$.
		Für $z \in \mathfrak{a}$ gilt
		\[
			\mathcal{B} \enbrace*{\benbrace*{x,y},z} = - \mathcal{B} \enbrace*{\ad_y(x), z} = \mathcal{B} \enbrace*{x, \ad_y(z)} =0
		\]
		\item \label{enum:222:2} Ist $\mathfrak{a}$ ein Ideal, so ist $\mathfrak{a} \cap \mathfrak{a}^\bot = \set*{0}$.
		Angenommen dies wäre nicht so, so wissen wir dass der Schnitt ein Ideal von $\mathfrak{g}$ ist, auf dem die Killingform verschwindet.
		Mit einer kleinen Rechnung folgt, dass der Schnitt dan auflösbar ist.
		Also ist $\rad(\mathfrak{g}) \neq 0$, was einen Widerspruch darstellt.
		\item \label{enum:222:3} $\mathfrak{g} = \mathfrak{a} \oplus \mathfrak{a}^\perp$ (gilt viel allgemeiner!)
	\end{enumerate}
	Beginnen wir nun mit den eigentlichen Aussagen:
	\begin{enumerate}[1)]
		\item Sei $\mathfrak{a}$ ein Ideal von $\mathfrak{g}$ mit $\mathfrak{a} \neq \set*{0}, \mathfrak{g}$.
		Nach \ref{enum:222:1}, \ref{enum:222:2} und \ref{enum:222:3} gilt $\mathfrak{g} = \mathfrak{a} \oplus \mathfrak{a}^\perp$.
		Induktiv folgt dann die Behauptung.
		\item Sei $\mathfrak{a}$ ein Ideal von $\mathfrak{g}$.
		Dann ist 
		\[
			\mathfrak{a} \cap \mathfrak{g}_1 = \begin{cases}
				0 \\
				\mathfrak{g}_i 
			\end{cases}
		\]
		mit $\mathfrak{g}_1, \ldots ,\mathfrak{g}_s$ wie in 1).
		Da $\mathfrak{g}$ kein Zentrum hat existiert $x \in \mathfrak{a}$ und $y \in \mathfrak{g}$ mit $\benbrace*{x,y} \neq 0$.
		Wir schreiben $y = y_1 + \ldots + y_s$ bezüglich $\mathfrak{g} = \mathfrak{g}_1 \oplus  \ldots \oplus \mathfrak{g}_s$.
		Dann gilt 
		\[
			0 \neq\benbrace*{x,y} = \sum_{i=1}^{s} \benbrace*{x,y_i} \implies \benbrace*{x,y_{i_0}} \neq 0
		\]
		für ein $i_0 \in \set*{1,\ldots ,s}$. Aber $\benbrace*{x,y_{i_0}} \in \mathfrak{g}_{i_0} \cap \mathfrak{a}$.
		\item Ist $\mathfrak{g}$ einfach, so gilt 
		\[
			\benbrace*{\mathfrak{g},\mathfrak{g}} = \begin{cases}
				0 \\
				\mathfrak{g}
			\end{cases}
		\]
		Aus ersterem würde folgen, dass $\mathfrak{g}$ abelsch ist, also die Killlingform trivial ist, was ein Widerspruch ist.
		Mit 1) folgt nun die Behauptung.
		\item Nach \autoref{lem:165} ist $\Int(\mathfrak{g})$ ein Ideal von $\Der(\mathfrak{g}) = \Tmap_e \Aut(\mathfrak{g})$.
		Da $\mathfrak{g}$ halbeinfach ist, gilt $Z(\mathfrak{g}) =0$ und somit ist
		\[
			\Int(\mathfrak{g}) \cong \ad(\mathfrak{g}) \cong \mathfrak{g}
		\]
		Damit ist $\int(\mathfrak{g})$ halbeinfach.
		Wir bezeichnen nun mit $\mathcal{B}$ die Killingform von $\Der(\mathfrak{g})$.
		Dann gilt
		\[
			\mathcal{B}\big|_{\Int(\mathfrak{g})} = \mathcal{B}_{\Int(\mathfrak{g})}
		\]
		Die Vorüberlegungen \ref{enum:222:1}, \ref{enum:222:2} und \ref{enum:222:3} gelten auch, wenn nur $\mathcal{B}|_{\mathfrak{a}}$ nicht degeneriert ist.
		Somit folgt
		\[
			\Der(\mathfrak{g}) = \Int(\mathfrak{g}) \oplus  \Int(\mathfrak{g})^\perp
		\]
		Sei nun $D \in \Int(\mathfrak{g})^\bot$.
		Dann gilt wieder nach \autoref{lem:165} $0= \benbrace*{D,\ad_X} = \ad_{D(x)}$ und somit $D(x) \in Z(\mathfrak{g}) = \set*{0}$.
		Da dies für alle $x \in \mathfrak{g}$ gilt, folgt $D=0$ und damit $\Der(\mathfrak{g}) = \Int(\mathfrak{g})$.\qedhere
	\end{enumerate}
\end{beweis}

\begin{bemerkung*}
	Einfache komplexe Liealgebren sind von Cartan klassifiziert worden.
	Diese treten in vir unendlichen Familien $A_n$, $B_n$, $C_n$, $D_n$ auf und 5 Ausnahmealgebren: $G_2$, $F_4$, $E_6$, $E_7$, $E_8$.
\end{bemerkung*}
% section 22 (end)

\section{Kompakte Liealgebren} % (fold)
\label{sec:23}

\begin{definition}[{name=[{kompakte Liealgebra}]}]
	Eine reelle Liealgebra $\mathfrak{g}$ nennt man \Index{kompakt}, falls es eine kompakte Liegruppe $G$ gibt $\Tmap_e G \cong \mathfrak{g}$. 
\end{definition}

\begin{lemma}
	Sei $\mathfrak{g}$ eine kompakte Liealgebra.
	Dann gilt:
	\begin{enumerate}[1)]
		\item Es existiert ein Skalarprodukt $\skal*{\cdot}{\cdot}$ auf $\mathfrak{g}$, sodass $\ad_X \colon \mathfrak{g} \to \mathfrak{g}$ \enquote{schiefsymmetrisch} bezüglich $\skal*{\cdot}{\cdot}$ ist für alle $x \in \mathfrak{g}$.\marginnote{$\skal*{Ax}{y} = - \skal*{x}{Ay}$}
		\item $G$ besitzt eine biinvariante Riemannsche Metrik\footnote{siehe \url{https://de.wikipedia.org/wiki/Riemannsche_Mannigfaltigkeit}} $g_\bi$, das heißt die Diffeomorphismen $L_g,R_g \colon G \to G$ sind alle für alle $g \in G$ Isometrien.
	\end{enumerate}
\end{lemma}
\begin{beweis}
	\begin{enumerate}[1)]
		\item Sei $\skal*{\cdot}{\cdot}_0$ \emph{ein} Skalarprodukt auf $\mathfrak{g}$.
		Wir setzen 
		\[
			\skal*{x}{y} \coloneqq \int_G \skal*{\Ad(g)x}{\Ad(g)y}_0 \,\omega
		\]
		wobei $\omega$ eine biinvariante Volumenform uf $G$ ist (wir können $G$ als zusammenhängend annehmen). 
		Die Existenz von $\omega$ zu beweisen, ist eine Übung.
		Sei $h \in G$.
		Dann gilt 
		\begin{align}
			\skal*{\Ad(h) x}{\Ad(h)y} = \int_G \skal*{\Ad(g) \Ad(h)(x)}{ \cdots}_0 \, \omega &= \int_G \skal*{\Ad(gh)x}{\Ad(gh)y} \, R_h^* \enbrace*{R^*_{h^{-1}} \omega} \\
			&\StackTextClap{Trafo}{=} \int_G \skal*{\Ad(g) x}{ \Ad(g)y}_0 \, \Underbracket{R^*_{h^{-1}} \omega}{=\omega} \\
			&= \skal*{x}{y}
		\end{align}
		Somit ist $\skal*{\cdot}{\cdot}$ $\Ad(G)$-invariantes Skalarprodukt, das heißt $\Ad(G) \subset \On(\mathfrak{g},\skal*{\cdot}{\cdot})$.
		Da $\enbrace*{\mathd \Ad}_e = \ad$ folgt die Behauptung.
		\item Sei $\skal*{\cdot}{\cdot}$ ein $\Ad(G)$-invariantes Skalarprodukt auf $\mathfrak{g}$.
		Wir definieren $(g_\bi)_g \colon \Tmap_g G \times \Tmap_g G \to \mathbb{R}$ durch
		\[
			(g_\bi)(x,y) \coloneqq \skal*{\enbrace*{\mathd L_{g^{-1}}}_g x}{ \enbrace*{\mathd L_{g^{-1}}}_gy}
		\]
		Dies ist eine glatte(!) Riemannsche Metrik (Übung).
		Klar ist: Die Diffeomorphismen $L_h$ sind Isometrien!
		Zu zeigen $(\mathd R_h)_g \colon \Tmap_g G \to \Tmap_{R_h(g)} G $ ist eine Isometrie:
		Wegen $R_{gh} = R_h \circ R_g$, folgt mit der Kettenregel
		\[
			\enbrace*{\mathd R_{gh}}_e = \enbrace*{\mathd R_h}_g \cdot \enbrace*{\mathd R_g}_e
		\]
		Somit genügt es zu zeigen, dass $(\mathd R_g)_e \colon \Tmap_e G \to \Tmap_g G$ eine Isometrie ist für alle $g \in G$.
		Es seien $x,y \in \mathfrak{g} = \Tmap_e G$.
		Dann gilt
		\begin{align}
			g_\bi \enbrace*{(\mathd R_g)_e \cdot x, \enbrace*{\mathd R_g}_e \cdot y}_g = \skal*{ \enbrace*{\mathd L_{g^{-1}}}_g \enbrace*{\mathd R_g}_e \cdot x}{\cdots}
			&= \skal*{\Ad(g^{-1}) x}{ \Ad(g^{-1})y} \\
			&= \skal*{x}{y} = g_\bi(x,y)_e
		\end{align}
		da $\skal*{\cdot }{\cdot}$ $\Ad(G)$-invariant ist.
		Dies zeigt die Behauptung.\qedhere
	\end{enumerate}
\end{beweis}

\begin{lemma}
	Sei $\mathfrak{g}$ eine reelle Liealgebra.
	Dann gilt 
	\begin{enumerate}[1)]
		\item $\mathcal{B} < 0 \iff \mathfrak{g}$ ist kompakt mit $Z(\mathfrak{g})=0$.
		\item Wenn $\mathfrak{g}$ kompakt ist, dann gilt 
		\[
			\mathfrak{g} = \benbrace*{\mathfrak{g},\mathfrak{g}} \oplus Z(\mathfrak{g})
		\]
		und $\benbrace*{\mathfrak{g},\mathfrak{g}}$ ist halbeinfach.
	\end{enumerate}
\end{lemma}
\begin{beweis}
	\begin{enumerate}[1)]
		\item Sei zunächst $\mathcal{B} < 0$.
		Dann ist $\mathfrak{g}$ halbeinfach mit trivialem Zentrum.
		Da die Killingform invariant unter Automorphismen ist, gilt $\Aut(\mathfrak{g}) \subset \On(\mathfrak{g},-\mathcal{B})$.
		Ferner ist nach \autoref{prop:162} $\Aut(g)$ abgeschlossen in $\End(\mathfrak{g})$.
		Somit ist $\Aut(\mathfrak{g})$ kompakt.
		Nach \autoref{satz:222} 4) ist $\sfrac{\Aut(\mathfrak{g})}{\Int(\mathfrak{g})}$ diskret, also endlich.
		Somit ist
		\[
			\enbrace*{\Aut(\mathfrak{g})}_e = \Int(\mathfrak{g})
		\]
		$\Int(\mathfrak{g})$ ist kompakt und wegen $\Int(\mathfrak{g}) \cong \mathfrak{g}$ folgt, dass $\mathfrak{g}$ eine kompakte Liealgebra mit $Z(\mathfrak{g})= \set*{0}$ ist.
		
		Sei nun $\mathfrak{g}$ eine kompakte Liealgebra mit trivialem Zentrum.
		Wähle ein $\Ad(G)$-invariantes Skalarprodukt $\skal*{\cdot}{\cdot}$ auf $\mathfrak{g}$.
		Dann gilt 
		\[
			- \mathcal{B}(x,x) = - \Sp \enbrace*{\ad_x \circ \ad_x} = \Sp \enbrace*{\ad_x \circ \ad_x^T} = \norm*{\ad_x}^2 \ge 0
		\]
		Somit ist $\mathcal{B} \le 0$ und $\mathcal{B}(x,x)=0 \iff x \in Z(\mathfrak{g})$.
		\item Wie im Beweis von \autoref{satz:222} 1) zeigt man, dass $\mathfrak{g} = Z(\mathfrak{g}) + \mathfrak{b}$ für ein Ideal $\mathfrak{b}$.\footnote{$\skal*{\cdot}{\cdot}$ $\Ad(G)$-invariantes Skalarprodukt, $\mathfrak{b} \coloneqq Z(\mathfrak{g})^\perp \implies \ad_x$ schiefsymmetrisch}
		Wir wissen $\mathcal{B} \le 0$ und $\mathcal{B}(x,x) =0 \iff x \in Z(\mathfrak{g})$.
		Somit $\mathcal{B}_\mathfrak{b} = \mathcal{B}_{\mathfrak{g}}|_{\mathfrak{b}} < 0$ und daher, da $\mathfrak{b}$ halbeinfach ist:
		\[
			\benbrace*{\mathfrak{g},\mathfrak{g}} = \benbrace*{\mathfrak{b},\mathfrak{b}} = \mathfrak{b}
		\]
		Dies zeigt $\mathfrak{g} = Z(\mathfrak{g}) \oplus \benbrace*{\mathfrak{g},\mathfrak{g}}$.\qedhere
	\end{enumerate}
\end{beweis}

\begin{korollarB}
	Eine kompakte Liegruppe mit endlichem Zentrum ist halbeinfach.
\end{korollarB}

\begin{lemma}[{name={Weyl}}]
	Sei $G$ eine kompakte Liegruppe mit endlichem Zentrum.\marginnote{$\SO(n)$, $\SU(n)$, $\Sp(n)$ aber $\cancel{\Un(n)}$}
	Dann ist 
	\[
		\abs*{\pi_1(G)} < \infty
	\]
\end{lemma}
\begin{beweis}
	\emph{Kommt noch. Wir werden zeigen, dass auf $G$ eine Riemannsche Metrik mit $\Ric >0$ existiert. Mit Bonnet-Myers folgt dann die Behauptung.}
\end{beweis}

\begin{korollar}
	Jede kompakte Liegruppe $G$ ist isomorph zu 
	\[
		\faktor{T^k \times G_1 \times \ldots \times G_s}{\Gamma}
	\]
	wobei $G_1, \ldots ,G_s$ kompakte, zusammenhängende, einfach zusammenhängende Liegruppen sind und $\Gamma \subset Z(T^k \times G_1 \times \ldots \times G_s)$ eine endliche Untergruppe ist.
\end{korollar}
\begin{beweis}
	\emph{Übung.}
\end{beweis}

\begin{lemma}
	Sei $G$ kompakte zusammenhängende Liegruppe.
	Dann ist $\exp\colon \mathfrak{g} = \Tmap_e G \to G$ surjektiv.
\end{lemma}
\begin{beweis}
	Wir werden zeigen, dass Einparametergruppen genau die Geodätischen von $(G,g_\bi)$ sind. 
	$G$ ist kompakt und damit vollständig bezüglich der von $g_\bi$ induzierten Metrik.
	Mit Hopf-Rinow folgt, dass die geometrische (=algebraische) Exponentialabbildung surjektiv ist.
\end{beweis}
% section 23 (end)
% chapter 2 (end)

\cleardoubleoddemptypage
\pagenumbering{Alph}
\setcounter{page}{1}
\cleardoubleoddemptypage
\appendix

\chapter{Anhang} % (fold)
\label{sec:anhang}
%!TEX root = ana_top_geo.tex

\subsection{Ausführlicher Beweis zu \cref{lem:kpt-schnitte}} % (fold)
\label{sub:kpt-schnitte}
Sei $X$ ein Hausdorffraum. Dann ist $X$ genau dann kompakt, wenn gilt: Hat eine Familie $\mathcal{A}$ von abgeschlossenen Teilmengen von $X$ die endliche 
Durchschnittseigenschaft, so gilt 
\[
	\bigcap_{A \in \mathcal{A}} A \not= \emptyset.
\]
\begin{beweis}
	Für die erste Implikation sei $X$ kompakt und $\mathcal{A}$ eine Familie von abgeschlossenen Mengen mit der endlichen Durchschnittseigenschaft.
	Angenommen $\bigcap_{A \in \mathcal{A}} A = \emptyset$.
	Dann gilt
	\[
		X = X \setminus \bigcap_{A \in \mathcal{A}} A = \bigcup_{A \in \mathcal{A}} X \setminus A.
	\]
	Nun ist $\mathcal{U} \coloneqq \set*{X \setminus A \given A \in \mathcal{A}}$ eine offene Überdeckung von $X$ und da $X$ kompakt ist, existiert $\mathcal{A}_0 \subset \mathcal{A}$ endlich, sodass
	\[
		X = \bigcup_{A \in \mathcal{A}_0} X \setminus A = X \setminus \underbrace{\bigcap_{A \in \mathcal{A}_0 } A }_{\neq \emptyset} \quad \light
	\]
	Für die umgekehrte Implikation sei nun $\mathcal{U} = \set{U_i}_{i \in I}$ eine offene Überdeckung von $X$.
	Angenommen für jede endliche Teilmenge $J \subseteq I$ gilt $X \neq \bigcup_{i \in J} U_i$.
	Betrachte nun $\mathcal{A} =  \set{X \setminus U_i}_{i \in I}$. Dann gilt nach Annahme
	\[
		\bigcap_{i \in J} X \setminus U_i = X \setminus \bigcup_{i \in J} U_i \neq \emptyset.
	\]
	Also hat $\mathcal{A}$ die endliche Durchschnittseigenschaft. Nach Vorraussetzung gilt dann
	\[
		\emptyset \not= \bigcap_{i \in I} X \setminus U_i = X \setminus \underbrace{\bigcup_{i \in I} U_i}_{= X} \quad \light \qedhere
	\]
\end{beweis}


\subsection[Blatt3, Aufgabe 4: Hilfssatz für den Hauptsatz der Algebra]{Blatt 3, Aufgabe 4} % (fold)
\label{sub:B3A4}
\emph{Diese Übungsaufgabe ist zentral für den Beweis des Hauptsatzes der Algebra, \cref{satz:hauptsatz-algebra}.} 

Sei $p(x)= x^n + a_{n-1} x^{n-1} + \ldots + a_1 x + a_0$ mit $n \in \mathbb{N}_0$ ein Polynom mit Koeffizienten $a_i \in \mathbb{C}$, dass \emph{keine} Nullstelle in $\mathbb{C}$ besitzt. 
Sei $S^1= \set*{z \in \mathbb{C} \given \abs*{z}=1}$.
\begin{enumerate}[(a)]
	\item $f \colon S^1 \to S^1$ gegeben durch $f(z) = \frac{p(z)}{\abs*{p(z)} } $ ist wohldefiniert und homotop zu einer konstanten Abbildung.
	\item $f$ ist homotop zur Abbildung $g_n \colon S^1 \to S^1$ mit $g_n(z)= z^n$.
\end{enumerate}
\minisec{Beweis}
\begin{enumerate}[(a)]
	\item \begin{description}
		\item[Wohldefiniertheit:] Sei $z \in S^1$ beliebig. Dann gilt
		\[
			\abs*{\frac{p(z)}{\abs*{p(z)} } } = \frac{1}{\abs*{p(z)} } \cdot \abs*{p(z)} =1,
		\]
		also ist $f(z) \in S^1$.
		\item[Homotop zu einer konstanten Abbildung:] Definiere $f_t \colon S^1 \to S^1$ für $t \in [0,1]$ durch 
		\[
			f_t(z) = \frac{p(t \cdot z)}{\abs*{p(t \cdot z)} } 
		\]
		Dies ist mit der gleichen Begründung wie oben wohldefiniert. 
		Außerdem ist $f_0(z)= \frac{a_0}{\abs*{a_0} } \in S^1 $ konstant und $f_1(z)= \frac{p(z)}{\abs*{p(z)} }=f(z)$. 
		Definiere nun $H \colon S^1 \times [0,1] \to S^1$ durch $H(x,t) \coloneqq f_t(x)$. 
		Dann ist $H$ stetig, da Polynome und $\abs*{.} $, sowie Multiplikation stetig sind. 
		$H$ ist die gesuchte Homotopie.
	\end{description}
	\item Sei $h \colon S^1 \times [0,1] \to \mathbb{C}$ gegeben durch $h(z,t) = z^n + \sum_{k=0}^{n-1} a_k z^k t^{n-k}$. 
	Dann gilt $h(z,0)=z^n \not= 0$, da $z \in S^1$.
	Für $t \neq 0$ gilt nun
	\begin{align*}
		h(z,t) = 0 \iff \frac{h(z,t)}{t^n} = 0 \iff \frac{z^n}{t^n} + \sum_{k=0}^{n-1} a_k \frac{z^k}{t^k} = 0 \iff p \enbrace*{\frac{z}{t}} = 0
	\end{align*}
	Aber nach Vorraussetzung gilt $p \enbrace*{\frac{z}{t}} \neq 0$. 
	Also $h(z,t) \neq 0$ für alle $t \in [0,1]$. 
	Definiere nun $H \colon S^1 \times [0,1]\to S^1$ durch $H(z,t) = \frac{h(z,t)}{\abs*{h(z,t)}}$. 
	Wie eben gezeigt, ist dies wohldefiniert und offensichtlich stetig. Da
	\[
		H(z,0) = \frac{z^n}{\abs*{z^n} } = z^n \quad \text{ und } \quad H(z,1) = \frac{h(z,1)}{\abs*{h(z,1)} } = \frac{p(z)}{\abs*{p(z)} } =f(z)
	\]
	ist $H$ die gesuchte Homotopie. \qedhere
\end{enumerate}

\subsection{Blatt 10, Aufgabe 3} % (fold)
\label{sub:B10A3}
\emph{Diese Übungsaufgabe lieferte den Beweis zu \cref{prop:iso-covering}.} \smallskip \\
Sei $p \colon \overline{X} \to X$ eine Überlagerung. 
Seien $\overline{x}_0  \in \overline{X}$ und $x_0= p(\overline{x}_0 )$ Basispunkte. 
Dann ist die induzierte Abbildung $\pi_n (p) \colon \pi_n(\overline{X}, \overline{x}_0) \to \pi_n(X,x_0)$ ein Isomorphismus für alle $n \ge 2$.
\minisec{Beweis}
Als Überlagerung ist $p$ stetig, also ist $\pi_n(p)$ ein Gruppenhomomorphismus nach \hyperref[prop:eig-hom-gruppen:enum:4]{ \cref*{prop:eig-hom-gruppen} \ref*{prop:eig-hom-gruppen:enum:4}}.
\begin{description}
	\item[Surjektivität:] Sei $[\omega] \in \pi_n(X,x_0)$, also $\omega \colon I^n \to X$ mit $\omega(\partial I^n) = \set{x_0}$. Betrachte $\omega$ nun als Abbildung $I^{n-1} \times [0,1] \to X$:
	\[
		\begin{tikzcd}[column sep=4em]
			I^{n-1} \times \set{0} \dar[hook] \rar["\mathrm{const}_{\overline{x}_0}"] & \overline{X} \dar["p"]\\
			I^{n-1} \times I \rar["\omega"] & X  
		\end{tikzcd}
	\]
	$\mathrm{const}_{\overline{x}_0} \colon I^{n-1} \times \set{0}$ ist eine Hebung von $\omega\big|_{I^{n-1} \times \set{0}} \equiv x_0$. 
	Nach dem Homotopiehebungssatz (\ref{satz:hebung-homotopie}) existiert eine Hebung $\overline{\omega} \colon I^{n-1} \times I \to \overline{X}$ von $\omega$ mit $\overline{\omega}\big|_{I^{n-1} \times \set{0}} \equiv \overline{x}_0 $. 
	Also gilt
	\[
		p \circ \overline{\omega} \big|_{\partial I^n} = \omega \big|_{\partial I^n} \equiv x_0 \enspace \Longrightarrow \enspace \overline{\omega} \big|_{\partial I^n} 
		\in p ^{-1}( \set{x_0} ) .
	\]
	Da $p^{-1}(\set{x_0})$ diskret und $\partial I^n$ für $n \ge 2$ zusammenhängend ist, muss $\overline{\omega} \big|_{\partial I^n}$ konstant sein. 
	Da $\overline{\omega}\big|_{I^{n-1} \times \set{0}} \equiv \overline{x}_0 $ gilt, folgt somit $\overline{\omega}(\partial I^n) = \set{\overline{x}_0}$. 
	Also ist $[\overline{\omega}] \in \pi_n(\overline{X},\overline{x}_0)$ und weiter gilt
	\[
		\pi_n(p) \enbrace*{[\overline{\omega}]} = [p \circ \overline{\omega} ] = [\omega] \in \pi_n(X,x_0). 
	\]
	\item[Injektivität:] Sei $[\omega] \in \ker \pi_n(p)$, also $[p \circ \omega] = [c_{x_0}]$. 
	Es existiert also eine Homotopie $H$ relativ $\partial I^n$ zwischen $p \circ \omega$ und $c_{x_0}$. 
	Offensichtlich ist $\omega$ eine Hebung von $p \circ \omega$. 
	Mit dem Homotopiehebungssatz erhalten wir eine Hebung $\overline{H}$ von $H$ mit $\overline{H}(-,0) = \omega$. 
	Weiter wissen wir, dass
	\[
		\overline{H} \big|_{\partial I^n \times [0,1]} \in p ^{-1}(\set{x_0} ) \quad \text{ und }\quad  \overline{H} \big|_{ I^n \times \set{1}} \in p ^{-1}(\set{x_0} )
	\]
	gelten muss, da $H = p \circ \overline{H}$ und $H(-,1)= c_{x_0} \equiv x_0$. 
	Mit dem gleichen Argument wie oben folgt, dass $\overline{H} \big|_{\partial I^n \times [0,1]}$ und $\overline{H} \big|_{ I^n \times \set{1}}$ konstant sind. 
	Für $z \in \partial I^n$ gilt nun
	\[
		\overline{H}(z,0) = \omega(z) = \overline{x}_0
	\]
	Da $\partial I^n \times [0,1] \cap I^n \times \set{1} \not= \emptyset$, muss also auch $\overline{H}(-,1) \equiv \overline{x}_0$ gelten. 
	Damit folgt $[\omega] = [c_{x_0}]$.\qedhere
\end{description}
\printindex
\printbibliography
\listoffigures
\todototoc
\listoftodos[To-do's und andere Baustellen]
\end{document}
