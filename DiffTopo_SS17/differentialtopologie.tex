%!TEX TS-program = xelatex
%!TEX TS-options = -shell-escape
%!TEX root = ../LieGrp_SS16/liegruppen.tex
\RequirePackage{fix-cm} 
\documentclass[a4paper, twoside, headsepline, index=totoc,toc=listof,toc=bibliography,toc=index, fontsize=10pt, cleardoublepage=empty, headinclude, DIV=12, BCOR=5mm, titlepage,draft]{scrreprt}
%!TEX root = ../AnaTopGeo_SS14/ana_top_geo.tex
\usepackage{scrtime} % KOMA, Uhrzeit ermoeglicht

%--Pakete zum "Programmieren"
% ======================================================================================
\usepackage{etoolbox}
\usepackage{letltxmacro}
\usepackage{ifthen}
% ======================================================================================

%--Farbdefinitionen und Grafiken (muss vor tikz geladen werden)
% ======================================================================================
\usepackage[usenames, table, x11names]{xcolor}
\definecolor{dark_gray}{gray}{0.45}
\definecolor{light_gray}{gray}{0.6}
\definecolor{fb10_blue}{cmyk}{0.8,0.4,0.13,0.07}
\usepackage[final]{graphicx}
\usepackage{adjustbox}
\newcommand{\cfbox}[2]{% coloured frame box
	\ifmmode
	\mathchoice{\adjustbox{cfbox=#1}{$\displaystyle#2$}}{\adjustbox{cfbox=#1}{$\textstyle#2$}}{\adjustbox{cfbox=#1}{$\scriptstyle#2$}}{\adjustbox{cfbox=#1}{$\scriptscriptstyle#2$}}
	\else
	\adjustbox{cfbox=#1}{#2}
	\fi
}
% ======================================================================================

%--Zum Zeichnen/ TikZ-Kram (vor polyglossia bzw. babel geladen werden)
% ======================================================================================
\usepackage{tikz}
\usepackage{tikz-cd}
\usetikzlibrary{external}
\tikzset{>=latex}
\usetikzlibrary{%
	shapes,
	arrows.meta,
	intersections,
	calc,
	3d,
	decorations.pathreplacing,decorations.markings,decorations.pathmorphing,
	angles,
	quotes,
}
\tikzexternalize[prefix=tikz/,up to date check=diff]
\pgfkeys{/pgf/images/include external/.code=\includegraphics{#1}}
\tikzset{external/system call={lualatex \tikzexternalcheckshellescape -halt-on-error -interaction=batchmode --shell-escape -jobname "\image" "\texsource"}}
\AtBeginEnvironment{tikzcd}{\tikzexternaldisable} % tikzexternalize fuer tikzcd deaktivieren, da inkompatibel
\AtEndEnvironment{tikzcd}{\tikzexternalenable}
\tikzset{% um Inkompatibilitaeten von quotes und polyglossia bzw. babel zu vermeiden
  every picture/.append style={
    execute at begin picture={\shorthandoff{"}},
    execute at end picture={\shorthandon{"}}
  }
}
\usepackage{pgfplots}
\usepgfplotslibrary{colormaps}
\newcommand*\circled[1]{\tikzexternaldisable\tikz[baseline=(char.base)]{\node[shape=circle,draw,inner sep=2pt] (char) {#1};}\tikzexternalenable}
% ======================================================================================



%-- Mathepakete etc.
% ======================================================================================
\usepackage[T1]{fontenc}
\renewcommand{\rmdefault}{zpltlf}
\usepackage{mathtools} % beinhaltet amsmath
\mathtoolsset{showonlyrefs,centercolon,showmanualtags}
\newtagform{brackets}[\textbf]{[}{]}
\usetagform{brackets}
\usepackage{fix-cm}
\usepackage[bbgreekl]{mathbbol}
\usepackage{amssymb,marvosym} 
\usepackage{nicefrac} % schräge Brüche
\usepackage{faktor}
\newcommand{\Faktor}[1]{\faktor[\textstyle]{#1}}
\usepackage{xfrac}
\usepackage{cancel}
\usepackage{mathdots} % Verbesserung von Punkten wie zB \ldots
\usepackage[bb=px]{mathalfa} % \mathbb als px font
\usepackage{centernot}
\usepackage{stackrel}
\DeclareSymbolFont{bbold}{U}{bbold}{m}{n}
\DeclareSymbolFontAlphabet{\mathbbold}{bbold}
\newcommand{\ind}{\mathbbold{1}} % charakteristische-Funktion-Eins
\def\mathul#1#2{\color{#1}\underline{{\color{black}#2}}\color{black}} %farbiges Untersteichen im Mathe-Modus
\renewcommand{\le}{\leqslant}
\renewcommand{\ge}{\geqslant}
% ======================================================================================


%-- Von xfrac erzeuge font warnings ignorieren
% ======================================================================================
\usepackage{silence}
\WarningFilter{latexfont}{Size substitutions with differences}
\WarningFilter{latexfont}{Font shape `U/bbold/m/n' in size}
% ======================================================================================


%-- Typographie/Polyglossia
% ======================================================================================
\usepackage[euler-digits]{eulervm} % vor fontspec laden!
\usepackage[no-math]{fontspec}
\usepackage{polyglossia} % moderner babel-ersatz
\setmainlanguage[spelling=new,babelshorthands=true]{german}
\shorthandoff{"}
\setotherlanguage{english}
\defaultfontfeatures{Mapping=tex-text, WordSpace={1.2}, Ligatures={Required,Common,Contextual},Extension=.otf} %


\setmainfont{TeXGyrePagellaX}[UprightFont=*-Regular,BoldFont=*-Bold,ItalicFont=*-Italic,BoldItalicFont=*-BoldItalic,ItalicFeatures={Style=Historic},Ligatures={Required,Common,Contextual,Historic}]
\setsansfont{texgyreadventor}[Scale=MatchUppercase, UprightFont=*-regular, BoldFont=*-bold, ItalicFont=*-italic, BoldItalicFont=*-bolditalic]
\setmonofont{SourceCodePro}[Scale=0.9,UprightFont=*-Regular, BoldFont=*-Semibold, ItalicFont=*-Light]
\usepackage{xltxtra}
\usepackage{fontawesome}
\usepackage[final]{microtype}
\usepackage[draft=false]{scrlayer-scrpage} 
\flushbottom
% ======================================================================================


%-- Aufzählungen
% ======================================================================================
\usepackage[shortlabels,inline]{enumitem}
\setlist[itemize,1]{label=\faCaretRight}
\setlist[enumerate]{font=\bfseries}
\setlist[description]{font=\normalfont\bfseries}
\usepackage{multicol}
% ======================================================================================


%-- Floats/Figures/Tabellen
% ======================================================================================
\usepackage{wrapfig}
\usepackage{float}
\usepackage[margin=10pt, font=small, labelfont={sf, bf}, format=plain, indention=1em]{caption}
\captionsetup[wrapfigure]{name=Abb. }
\usepackage{booktabs}
% ======================================================================================


%-- korrekte Anführungszeichen und Zitierbefehle
% ======================================================================================
\usepackage[autostyle,german=quotes,english=british]{csquotes}
% ======================================================================================


%--Indexverarbeitung
% ======================================================================================
\usepackage{makeidx}
\newcommand{\bet}[1]{\textbf{\emph{#1}}}
\newcommand{\Index}[1]{\bet{#1}\index{#1}}
\makeindex
\setindexpreamble{{\noindent\sffamily\small Die \emph{Seitenzahlen} sind mit Hyperlinks versehen und somit anklickbar} \par \bigskip}
\renewcommand{\indexpagestyle}{scrheadings}
% ======================================================================================


%-- Marginnotes/Todonotes/Footnotes
% ======================================================================================
\deffootnote[1.5em]{1.5em}{1.5em}{\textsuperscript{\thefootnotemark}\ }
\usepackage[fulladjust]{marginnote}
\renewcommand*{\marginfont}{\itshape\footnotesize}
\usepackage[textsize=small]{todonotes}
\usepackage{ragged2e}
\renewcommand*{\raggedleftmarginnote}{\RaggedLeft}
\renewcommand*{\raggedrightmarginnote}{\RaggedRight}
\LetLtxMacro{\oldtodo}{\todo}
\renewcommand{\todo}[2][]{\tikzexternaldisable\oldtodo[#1]{#2}\tikzexternalenable}
\LetLtxMacro{\oldmissingfigure}{\missingfigure}
\renewcommand{\missingfigure}[2][]{\tikzexternaldisable\oldmissingfigure[{#1}]{#2}\tikzexternalenable}
% ======================================================================================


% -- BibLaTeX
% ======================================================================================
\usepackage[%
	backend=biber,
	sortlocale=auto,
	natbib,
	hyperref,
	backref,
	style=alphabetic
	]%
{biblatex}
\renewcommand*{\mkbibnamelast}[1]{%
  \ifmknamesc{\textsc{#1}}{#1}}
\renewcommand*{\mkbibnameprefix}[1]{%
  \ifboolexpr{ test {\ifmknamesc} and test {\ifuseprefix} }
    {\textsc{#1}}
    {#1}}
\def\ifmknamesc{%
  \ifboolexpr{ test {\ifcurrentname{labelname}}
               or test {\ifcurrentname{author}}
               or ( test {\ifnameundef{author}} and test {\ifcurrentname{editor}} ) }}
\addbibresource{../!config/quellen.bib}
% ======================================================================================

%--Konfiguration von Hyperref und Cleveref
% ======================================================================================
\usepackage[hidelinks, pdfpagelabels,  bookmarksopen=true, bookmarksnumbered=true, linkcolor=black, urlcolor=SkyBlue2, plainpages=false,pagebackref, citecolor=black, hypertexnames=true, pdfauthor={Jannes Bantje}, pdfborderstyle={/S/U}, linkbordercolor=SkyBlue2, colorlinks=false,final,backref=false]{hyperref}
\usepackage[nameinlink,noabbrev]{cleveref}
\newcommand{\appendLink}[1]{#1\,\faExternalLink}
\newcommand{\hrefsym}[2]{\href{#1}{\texttt{\appendLink{#2}}}}
\newcommand{\hrefsymX}[2]{\href{#1}{\appendLink{#2}}}
\newcommand{\hrefsymmail}[2]{\href{#1}{\texttt{\faEnvelopeO\,#2}}}
\renewcommand{\url}[1]{\hrefsym{#1}{\nolinkurl{#1}}}
% ======================================================================================


% -- QR-Codes (hinter hyperref laden!)
% ======================================================================================
\usepackage{qrcode}
% ======================================================================================

%--Römische Zahlen
% ======================================================================================
\newcommand{\RM}[1]{\MakeUppercase{\romannumeral #1{}}}
% ======================================================================================

%-- Definition von diversen Mathe-Befehlen
% ======================================================================================
%!TEX root = mitschrift_main.tex

% -- Zum Finetuning von Befehlen
% ======================================================================================
\makeatletter
\newcommand{\raisemath}[1]{\mathpalette{\raisem@th{#1}}}
\newcommand{\raisem@th}[3]{\raisebox{#1}{$#2#3$}}
\makeatother
\makeatletter
\newcommand{\killDescendersM}[1]{\mathpalette{\killD@scendersM{#1}}}
\newcommand{\killD@scendersM}[2]{\raisebox{0pt}[\height][0pt]{$#2#1$}}
\makeatother
\DeclareRobustCommand{\minwidthbox}[2]{%
  \ifmmode
    \expandafter\mathmakebox
  \else
    \expandafter\makebox
  \fi
  [\ifdim#2<\width\width\else#2\fi]{#1}%
}
% ======================================================================================


%-- Klammerbefehle
% ======================================================================================
\DeclarePairedDelimiter{\abs}{\lvert}{\rvert}
\DeclarePairedDelimiter{\floor}{\lfloor}{\rfloor}
\DeclarePairedDelimiter{\ceil}{\lceil}{\rceil}
\DeclarePairedDelimiter\norm{\Vert}{\Vert}
\DeclarePairedDelimiter\enbrace{(}{)}
\DeclarePairedDelimiter\benbrace{[}{]}
\DeclarePairedDelimiter\bbenbrace{[\![}{]\!]}
\DeclarePairedDelimiter\lenbrace{<}{>}
\DeclarePairedDelimiter\angbrace{\langle}{\rangle}
\newcommand{\ssbrace}[1]{{\scriptscriptstyle\enbrace{#1}}}
\newcommand{\ssbbrace}[1]{{\scriptscriptstyle\benbrace{#1}}}
% ======================================================================================

%-- Mengen
% ======================================================================================
\newcommand\SetSymbol[1][]{\nonscript\:#1\vert\allowbreak\nonscript\:\mathopen{}}
\providecommand\given{} % to make it exist
\DeclarePairedDelimiterX\set[1]\{\}{\renewcommand\given{\SetSymbol[\delimsize]}#1}
% ======================================================================================

%-- Skalarprodukt (3 Varianten) 
% ======================================================================================
\DeclarePairedDelimiterX\sprod[2]{\langle}{\rangle}{#1\,\delimsize\vert\,#2}
\DeclarePairedDelimiterX\skal[2]{\langle}{\rangle}{#1\,,\,#2}
\makeatletter
\DeclareFontFamily{OMX}{MnSymbolE}{}
\DeclareSymbolFont{MnLargeSymbols}{OMX}{MnSymbolE}{m}{n}
\SetSymbolFont{MnLargeSymbols}{bold}{OMX}{MnSymbolE}{b}{n}
\DeclareFontShape{OMX}{MnSymbolE}{m}{n}{
    <-6>  MnSymbolE5
   <6-7>  MnSymbolE6
   <7-8>  MnSymbolE7
   <8-9>  MnSymbolE8
   <9-10> MnSymbolE9
  <10-12> MnSymbolE10
  <12->   MnSymbolE12
}{}
\DeclareFontShape{OMX}{MnSymbolE}{b}{n}{
    <-6>  MnSymbolE-Bold5
   <6-7>  MnSymbolE-Bold6
   <7-8>  MnSymbolE-Bold7
   <8-9>  MnSymbolE-Bold8
   <9-10> MnSymbolE-Bold9
  <10-12> MnSymbolE-Bold10
  <12->   MnSymbolE-Bold12
}{}
\let\llangle\@undefined
\let\rrangle\@undefined
\DeclareMathDelimiter{\llangle}{\mathopen}%
                     {MnLargeSymbols}{'164}{MnLargeSymbols}{'164}
\DeclareMathDelimiter{\rrangle}{\mathclose}%
                     {MnLargeSymbols}{'171}{MnLargeSymbols}{'171}
\makeatother
\DeclarePairedDelimiterX\sskal[2]{\llangle}{\rrangle}{#1\,,\,#2}
% ======================================================================================

%-- Abbildungsdefinition
% ======================================================================================
\newcommand{\mapdef}[5]{%
	\[
		\begin{array}{rcl}
			\textstyle #1 &\xrightarrow{\minwidthbox{#5}{2em}} & \textstyle #2 \\[0.5ex]
			\textstyle #3 &\xmapsto{\minwidthbox{\mbox{ }}{2em}} & \textstyle #4
		\end{array}
	\]
}
% ======================================================================================

%-- modifiziertes Stackrel 
% ======================================================================================
\newcommand{\StackText}[2]{\stackrel{\mbox{\scriptsize #1}}{#2}}
\newcommand{\StackTextClap}[2]{\stackrel{\mathclap{\mbox{\scriptsize #1}}}{#2}}
% ======================================================================================

%-- Blitz
% ======================================================================================
\newcommand{\light}{\text{\raisebox{-.3ex}{\Large\Lightning}}}
% ======================================================================================


%-- Underbrace u.Ä. als Befehl in LaTeX-Syntax (und ohne Spacingprobleme mit nachfolgenden Operatoren...)
% ======================================================================================
\newcommand{\Underbrace}[2]{{\underbrace{#1}_{#2}}}
\newcommand{\Underbracket}[2]{{\underbracket[0.7pt][2pt]{#1}_{#2}}}
\newcommand{\Overbracket}[2]{{\overbracket[0.7pt][2pt]{#1}^{#2}}}
% ======================================================================================


%-- Deklaration weiterer Operatoren (allgemein)
% ======================================================================================
\DeclareMathOperator{\re}{Re} % Realteil
\let\Re\relax
\DeclareMathOperator{\Re}{Re} % Realteil
\DeclareMathOperator{\im}{im} % Bild
\let\Im\relax
\DeclareMathOperator{\Im}{Im} % Bild
\DeclareMathOperator{\id}{id} % identische Abbildung
\DeclareMathOperator{\conj}{conj} % Konjugation
\DeclareMathOperator{\sgn}{sgn} % Signum
\DeclareMathOperator{\End}{End} % Endomorphismen
\DeclareMathOperator{\Hom}{Hom} % Homomorphismen
\DeclareMathOperator{\Iso}{Iso} % Isomorphismen
\DeclareMathOperator{\Aut}{Aut} % Automorphismen
\DeclareMathOperator{\Span}{span} % Span
\DeclareMathOperator{\coker}{coker} % Kokern
\DeclareMathOperator{\Tr}{Tr} % Spur,Trace
\DeclareMathOperator{\pr}{pr} % Projektion
\DeclareMathOperator{\diag}{diag} % Diagonalmatrix
\DeclareMathOperator{\Rg}{Rg} % Rang
\DeclareMathOperator{\const}{const} % konstante Abbildung
\DeclareMathOperator{\Spur}{Spur} % Spur
\DeclareMathOperator{\Arg}{Arg} % Argument
\DeclareMathOperator{\dist}{dist} % Distanz
\DeclareMathOperator{\supp}{supp} % Träger
\DeclareMathOperator{\Char}{char} % Charakteristik
% ======================================================================================


%-- Deklaration weiterer Operatoren (Differentiale etc.)
% ======================================================================================
\DeclareMathOperator{\grad}{grad} % Gradient
\DeclareMathOperator{\dive}{div} % Gradient
\DeclareMathOperator{\rot}{rot} % Rotation
\newcommand{\D}{\ensuremath{\mathrm{D}\mkern-1.0mu}} % Differential
\newcommand{\mathd}{\ensuremath{\mathrm{d}\mkern-1.0mu}} % äußere Ableitung
\newcommand{\Tmap}{\ensuremath{\mathrm{T}\mkern-0.85mu}} % Tangentialraum
\let\Tang\Tmap
\DeclareMathOperator{\Diff}{Diff}
\newcommand{\diff}[2]{\ensuremath{\frac{{\partial #1}}{{\partial #2}} }}
\newcommand{\diffd}[2]{\ensuremath{\frac{\mathd #1}{\mathd #2} }}
\DeclareMathOperator{\rank}{rank}
% ======================================================================================


%-- Deklaration weiterer Operatoren (Topologie)
% ======================================================================================
\newcommand*\interior[1]{\overset{\smash{\raisebox{-0.18ex}{$\scriptstyle\circ$}}}{#1}}
\newcommand{\sing}{{\raisemath{1.1pt}{\scriptscriptstyle\mathrm{sing}}}}
\newcommand{\pt}{\mathrm{pt}}
\DeclareMathOperator{\Zyl}{Zyl}
\newcommand{\rZyl}{\widetilde{\Zyl}}
\DeclareMathOperator{\Tel}{Tel}
\newcommand{\op}{\mathrm{op}}
\DeclareMathOperator{\Sp}{Sp}
\DeclareMathOperator{\Keg}{Keg}
\newcommand{\slashedi}{i\hspace{-3.5pt}/}
\newcommand{\cupp}{\smallsmile}
\newcommand{\capp}{\smallfrown}
\DeclareMathOperator*{\colim}{colim}
\DeclareMathOperator{\PD}{PD}
\newcommand{\lf}{\mathrm{lf}}
\DeclareMathOperator{\sig}{sig}
\DeclareMathOperator{\Tor}{Tor}
\DeclareMathOperator{\Ext}{Ext}
\DeclareMathOperator{\AW}{AW}
\DeclareMathOperator{\Proj}{Proj}
\DeclareMathOperator{\Gr}{Gr}
\DeclareMathOperator{\res}{res}
\DeclareMathOperator{\Spec}{Spec}
\DeclareMathOperator{\co}{co}
\DeclareMathOperator{\ch}{ch}
\DeclareMathOperator{\wOp}{w}
\DeclareMathOperator{\Ar}{Ar}
\newcommand{\actson}{\mathrel{\curvearrowright}}
\let\acts\actson
\let\action\actson
\DeclareMathSymbol{\bbDelta}{\mathord}{bbold}{"01}
\newcommand{\DDelta}{\bbDelta}
\DeclareMathOperator{\Star}{Star}
\DeclareMathOperator{\Link}{Link}
\DeclareMathOperator{\EPK}{EPK}
\DeclareMathOperator{\Vol}{Vol}
\newcommand{\cell}{{\raisemath{1.1pt}{\scriptscriptstyle\mathrm{cell}}}}
\DeclarePairedDelimiter{\homologieklasse}{\llbracket}{\rrbracket}
\newcommand{\rand}[1]{\ensuremath{\partial^{\scriptscriptstyle #1}}}
\DeclareMathOperator{\ab}{ab}
\DeclareMathOperator{\CW}{CW}
% ======================================================================================


%-- Deklaration von Operatoren (Liegruppen)
% ======================================================================================
\DeclareMathOperator{\GL}{GL}
\DeclareMathOperator{\SO}{SO}
\DeclareMathOperator{\Ad}{Ad}
\DeclareMathOperator{\ad}{ad}
\DeclareMathOperator{\On}{O}
\DeclareMathOperator{\Un}{U}
\DeclareMathOperator{\SU}{SU}
\DeclareMathOperator{\Mat}{Mat}
\DeclareRobustCommand{\Der}{\mathop{\mathfrak{der}}}
\DeclareMathOperator{\SL}{SL}
\DeclareMathOperator{\Graph}{Graph}
\DeclareMathOperator{\Int}{Int}
\DeclareRobustCommand{\intAlg}{\mathop{\mathfrak{int}}}
\DeclareMathOperator{\aut}{aut}
\DeclareMathOperator{\Rad}{Rad}
\DeclareMathOperator{\Nil}{Nil}
\DeclareMathOperator{\rad}{rad}
\DeclareMathOperator{\nil}{nil}
\DeclareMathOperator{\Ric}{Ric}
\DeclareMathOperator{\ric}{ric}
\newcommand{\bi}{\mathrm{bi}}
\DeclareMathOperator{\Isom}{Isom}
\DeclareMathOperator{\Sym}{Sym}
\newcommand{\opL}{\ensuremath{\mathrm{L}\mkern-0.6mu}}
% ======================================================================================

%-- Deklaration von Operatoren (Funktionalanalysis)
% ======================================================================================
\DeclareMathOperator{\tr}{tr}
\newcommand{\w}{\mkern1mu\mathrm{w}}
\newcommand{\sa}{\mathrm{sa}}
\newcommand{\vb}{\mathrm{v\mkern-2.5mu.b\mkern-1.5mu.}} % vollständig beschränkt
\newcommand{\so}{\mathrm{\mkern.3mu s\mkern-1.4mu.\mkern-.6mu o\mkern-1.7mu.}} % \newcommand{\so}{\mathrm{s.o.}}
\newcommand{\solim}{\so\text{-}\mkern-0.8mu\lim}
\newcommand{\wo}{\mathrm{w\mkern-3mu.\mkern-.4mu o\mkern-1.7mu.}}
\newcommand{\Top}[1]{\mathcal{T}_{\mkern-2.3mu #1}}
\newcommand{\weakT}[1]{\ensuremath{\mathcal{T}_{#1}^{\mkern+1.0mu\text{\raisebox{0.4ex}{$\mathrm{w}$}}}}}
\newcommand{\weakTstar}[1]{\ensuremath{\mathcal{T}_{#1}^{\mkern+1.0mu\text{\raisebox{0.4ex}{$\mathrm{w}$}}^*}}}
\newcommand{\TWeakStar}{\Top{\w^*}}
\newcommand{\TWeakOp}{\Top{\wo}}
\newcommand{\Tso}{\Top{\so}}
\newcommand{\finSub}{\subset\mkern-0.7mu \subset}
\DeclareMathOperator{\Inv}{Inv}
\newcommand{\simm}{{\hspace{-1.6pt}\raisemath{0.5pt}{\sim}}}
\newcommand{\plus}{{\hspace{-1.6pt}+}}
\DeclareMathOperator{\ev}{ev}
\DeclareMathOperator{\Alg}{Alg}
\DeclareMathOperator{\her}{her}
\newcommand{\subher}{\subset_{\her}}
\newcommand{\grenzw}[1]{\xrightarrow{\minwidthbox{#1}{1.4em}}}
\newcommand{\grenzwl}[1]{\xleftarrow{\minwidthbox{#1}{1.4em}}}
\newcommand{\grenzwIn}[1]{\grenzw{\raisemath{-2pt}{#1}}}
\newcommand{\MyTo}[1]{\tikzexternaldisable\mathbin{\tikz[baseline] \draw[-to,line width=.4pt] (0ex,0.94ex) -- (#1,0.94ex);}\tikzexternalenable}
\newcommand{\dlim}{%
    \mathchoice
      {\lim\limits_{\MyTo{4.2ex}}}% \displaystyle
      {\lim\limits_{\MyTo{2.8ex}}}% \textstyle
      {\lim\limits_{\MyTo{2.3ex}}}% \scriptstyle
      {\lim\limits_{\MyTo{2.3ex}}}% \scriptscriptstyle
}
\newcommand{\Dlim}{\killDescendersM{\dlim}}
\DeclareMathOperator{\sep}{sep}
\DeclareMathOperator{\diam}{diam}
\DeclareMathOperator{\conv}{conv}
\DeclareMathOperator{\Prim}{Prim}
\DeclareMathOperator{\hull}{hull}
\DeclareMathOperator{\red}{red}
\DeclarePairedDelimiterX\bra[1]{\langle}{\rvert}{#1\,}
\DeclarePairedDelimiterX\ket[1]{\lvert}{\rangle}{\,#1}
\DeclarePairedDelimiterX\bracket[2]{\langle}{\rangle}{#1\,\delimsize\vert\,#2}
\newcommand{\tensormax}{\mathbin{\otimes_{\max}}}
\newcommand{\tensormin}{\mathbin{\otimes_{\min}}}
\DeclareMathOperator{\Ped}{Ped}
\newcommand{\alg}{\mathrm{alg}}
\DeclareMathOperator{\CPC}{CPC}
\DeclareMathOperator{\CP}{CP}
\DeclareMathOperator{\UPC}{UPC}
\newcommand{\DeltaOp}{\mathbin{\Delta}}
\newcommand{\kernedP}{\mathcal{P}\mkern-2mu}
\newcommand{\Pinfty}{\kernedP_{\infty}}
\DeclareMathOperator{\Groth}{Groth}
\DeclareMathOperator{\rk}{rk}
\newcommand{\MvN}{\mathrm{MvN}}
% ======================================================================================

%-- Kategorien
% ======================================================================================
\DeclareMathOperator{\Mor}{Mor}
\DeclareMathOperator{\mor}{mor}
\DeclareMathOperator{\Obj}{Obj}
\DeclareMathOperator{\Ob}{Ob}
\newcommand{\TOP}{\textsc{Top}}
\newcommand{\HTOP}{\textsc{HTop}}
\newcommand{\VR}{\textsc{VR}}
\newcommand{\MOD}{\textsc{Mod}}
\newcommand{\Mod}[1]{#1\text{-}\MOD}
\newcommand{\MONOIDE}{\textsc{Monoide}}
\newcommand{\SET}{\textsc{Set}}
\newcommand{\MAN}{\textsc{Man}}
\newcommand{\GRUPPEN}{\textsc{Gruppen}}
\newcommand{\ABELGRUPPEN}{\textsc{Abel.Gruppen}}
\newcommand{\ABEL}{\textsc{Abel}}
\newcommand{\KAT}{\textsc{Kat}}
\newcommand{\FUN}{\textsc{Fun}}
\newcommand{\SIMP}{\textsc{Simp}}
\newcommand{\VEKT}{\textsc{Vekt}}
\newcommand{\CH}{\textsc{Ch}}
\newcommand{\CSTARUN}{C^*\text{-}\textsc{Alg}^{\raisemath{-2.5pt}{1}}}
\newcommand{\CSTAR}{C^*\text{-}\textsc{Alg}}
\newcommand{\AB}{\textsc{Ab}}
% ======================================================================================
% ======================================================================================



% -- theorem packages
% ======================================================================================
\usepackage{amsthm}
\usepackage{thmtools,thm-restate}
\usepackage{mdframed}
\renewcommand{\listtheoremname}{Übersicht aller Aussagen}
\usepackage{bookmark}
\bookmarksetup{open,numbered}
\makeatletter
\newcommand*{\theorembookmark}{%
  \bookmark[
    dest=\@currentHref,
    rellevel=1,
    keeplevel,
  ]{%
    \thmt@thmname\space\csname the\thmt@envname\endcsname
    \ifx\thmt@shortoptarg\@empty
    \else
      \space(\thmt@shortoptarg)%
    \fi
  }%
}   
\makeatother
% ======================================================================================

% -- Definition der einzelnen Theorem-Umgebungen
% ======================================================================================
\declaretheoremstyle[%
	headfont=\sffamily\bfseries,
	notefont=\normalfont\sffamily\scshape,
	bodyfont=\normalfont,
	headformat=\NUMBER\ \NAME\NOTE,
	headpunct=.,
	postheadspace=1em,
	spaceabove=15pt,spacebelow=10pt,
	shaded={bgcolor=gray!20},
	postheadhook=\theorembookmark]%
{mainstyle}
\declaretheoremstyle[%
	headfont=\sffamily\bfseries,
	notefont=\normalfont\sffamily\scshape,
	bodyfont=\normalfont,
	headformat=\NUMBER\ \NAME\NOTE,
	headpunct=.,
	postheadspace=1em,
	spaceabove=15pt,spacebelow=10pt,
	shaded={bgcolor=fb10_blue!20},
	postheadhook=\theorembookmark]%
{mainstyle_blue}
\declaretheoremstyle[%
	headfont=\sffamily\bfseries,
	notefont=\normalfont\sffamily\scshape,
	bodyfont=\normalfont,
	headformat=\NUMBER\ \NAME\NOTE,
	headpunct=.,
	postheadspace=1em,
	spaceabove=15pt,spacebelow=10pt,
	postheadhook=\theorembookmark]%
{mainstyle_unshaded}
\declaretheoremstyle[%
	headfont=\sffamily\bfseries,
	notefont=\normalfont\sffamily\scshape,
	bodyfont=\normalfont,
	headformat=\NUMBER\NAME\NOTE,
	headpunct=.,
	postheadspace=1em,
	spaceabove=15pt,spacebelow=10pt,
	% shaded={bgcolor=gray!20},
	postheadhook=\theorembookmark]%
{mainstyle_unnumbered}
\declaretheoremstyle[%
	headfont=\sffamily\bfseries,
	notefont=\normalfont\sffamily\scshape,
	bodyfont=\normalfont,
	headformat=swapnumber,
	headpunct=.,
	postheadspace=1em,
	spaceabove=15pt,spacebelow=10pt,
	shaded={bgcolor=gray!20},
	postheadhook=\theorembookmark,
	qed=\qedsymbol]%
{mainstyleB}
\declaretheoremstyle[%
	headfont=\bfseries\scshape,
	bodyfont=\normalfont,
	headpunct=:,
	postheadspace=1em,
	spacebelow=12pt,spaceabove=2pt,
	qed=\qedsymbol]%
{beweise}
\declaretheoremstyle[%
	headfont=\bfseries\scshape,
	bodyfont=\normalfont,
	headpunct=:,
	postheadspace=1em,
	spacebelow=12pt,spaceabove=2pt]%
{beweisskizze}
\declaretheoremstyle[%
	headfont=\sffamily\bfseries,
	bodyfont=\normalfont,
	headpunct=:,
	postheadspace=1em,
	spacebelow=10pt,spaceabove=10pt]%
{bemerkungen}
\declaretheorem[name=Definition,parent=section,style=mainstyle_blue]{definition}
\declaretheorem[name=Definition \& Proposition,refname=Proposition,sharenumber=definition,style=mainstyle_blue]{definitionP}
\declaretheorem[name=Definition,numbered=no,style=mainstyle_unnumbered]{definition*}
\declaretheorem[name=Theorem,sharenumber=definition,style=mainstyle]{theorem}
\declaretheorem[name=Theorem,numbered=no,style=mainstyle_unnumbered]{theorem*}
\declaretheorem[name=Proposition,sharenumber=definition,style=mainstyle,refname=Proposition]{proposition}
\declaretheorem[name=Lemma,sharenumber=definition,style=mainstyle]{lemma}
\declaretheorem[name=Satz,sharenumber=definition,style=mainstyle,refname=Satz]{satz}
\declaretheorem[name=Satz,sharenumber=definition,style=mainstyle_unshaded]{satzUnshaded}
\declaretheorem[name=Definition,sharenumber=definition,style=mainstyle_unshaded]{definitionUnshaded}
\declaretheorem[name=Satz,numbered=no,style=mainstyle_unnumbered]{satz*}
\declaretheorem[name=Korollar,sharenumber=definition,style=mainstyle,refname=Korollar]{korollar}
\declaretheorem[name=Korollar,sharenumber=definition,style=mainstyleB,refname=Korollar]{korollarB}
\declaretheorem[name=Frage,numbered=no,style=mainstyle_unnumbered]{frage}
\declaretheorem[name=Frage,sharenumber=definition,style=mainstyle_unshaded]{frageA}
\declaretheorem[name=Erinnerung,sharenumber=definition,style=mainstyle_unshaded]{erinnerungA}
\declaretheorem[name=Ausblick,sharenumber=definition,style=mainstyle_unshaded]{ausblick}
\declaretheorem[name=Konvention,sharenumber=definition,style=mainstyle]{konvention}
\declaretheorem[name=Notation,sharenumber=definition,style=mainstyle_unshaded]{notation}
\declaretheorem[name=Bemerkung,sharenumber=definition,style=mainstyle_unshaded,refname=Bemerkung]{bemerkung}
\declaretheorem[name=Bemerkung,numbered=no,style=mainstyle_unnumbered]{bemerkung*}
\declaretheorem[name=Beispiel,sharenumber=definition,style=mainstyle_unshaded,refname=Beispiel]{beispiel}
\declaretheorem[name=Beispiel,numbered=no,style=mainstyle_unnumbered]{beispiel*}
\declaretheorem[name=Exkurs,numbered=no,style=mainstyle_unnumbered]{exkurs*}
\declaretheorem[name=Beweis,numbered=no,style=beweise]{beweis}
\declaretheorem[name=Übung,numbered=no,style=bemerkungen]{uebung}
\declaretheorem[name=Erinnerung,numbered=no,style=bemerkungen]{erinnerung}

% english versions
\declaretheorem[name=Remark,sharenumber=definition,style=mainstyle_unshaded]{remark}
\declaretheorem[name=Remark,numbered=no,style=mainstyle_unnumbered]{remark*}
\declaretheorem[name=Example,sharenumber=definition,style=mainstyle_unshaded]{example}
\declaretheorem[name=Corollary,sharenumber=definition,style=mainstyle]{corollary}
\let\proof\relax
\declaretheorem[name=Proof,numbered=no,style=beweise]{proof}
\declaretheorem[name=Sketch of Proof,numbered=no,style=beweisskizze]{sketch}
% ======================================================================================

%--Inhaltsverzeichnis
% ======================================================================================
\usepackage[tocindentauto]{tocstyle}
\usetocstyle{KOMAlike}
% ======================================================================================

%-- Dinge, die erst am Ende getan werden dürfen
% ======================================================================================
\shorthandon{"}
\usepackage{ellipsis}
% ======================================================================================


\newcommand{\fach}{Differentialtopologie}
\newcommand{\semester}{SoSe 2017}
\newcommand{\homepage}{https://ivv5hpp.uni-muenster.de/u/jeber_02/sommer17/differentialtopologie.html}

\newcommand{\prof}{Prof.\ Dr.\ Johannes Ebert}
\publishers{\scalebox{10}{\Huge$C^{\scriptscriptstyle\infty}$}}

\setlist{itemsep=1pt}
\input{../!config/mitschrift_headings.tex}

\begin{document}
\pagenumbering{Roman}
\maketitle
\begin{abstract}
\section*{Aktuelle Version verfügbar bei}
\newcommand{\dieBreite}{11cm}
\begin{minipage}{4cm}
	\qrcode[height=3.3cm, version=6]{https://gitlab.com/JaMeZ-B/LaTeX-WWU}
\end{minipage}
\hfill
\begin{minipage}{\dieBreite}
	% \includegraphics[height=0.6cm, keepaspectratio]{../!config/Bilder/wm_no_bg.pdf}
	\includegraphics[height=0.8cm, keepaspectratio]{../!config/Bilder/wm_no_bg.pdf}\\
	\url{https://gitlab.com/JaMeZ-B/LaTeX-WWU} \smallskip\\
	Das zentrale Repository des \enquote{\LaTeX-WWU}-Projekts befindet sich auf der Plattform GitLab.com.
	Neben der Koordination aller Beteiligten werden über diesen Dienst auch die PDFs gebaut, die in der Readme verlinkt sind.
\end{minipage}\\[1cm]
\begin{minipage}{4cm}
	\qrcode[height=3.3cm, version=6]{https://github.com/JaMeZ-B/latex-wwu}
\end{minipage}
\hfill
\begin{minipage}{\dieBreite}
	\includegraphics[height=0.6cm, keepaspectratio]{../!config/Bilder/github_octo.pdf}
	\includegraphics[height=0.6cm, keepaspectratio]{../!config/Bilder/GitHub_Logo.pdf}\\
	\url{https://github.com/JaMeZ-B/latex-wwu} \smallskip\\
	Die Entwicklung des \enquote{\LaTeX-WWU}-Projekts hat ursprünglich auf GitHub stattgefunden, ist mittlerweile aber zu GitLab gewechselt.
	Das GitHub-Repository wird stündlich automatisch aktualisiert, Merge-Requests werden aber nicht mehr entgegengenommen.
\end{minipage}\\[1cm]
% \begin{minipage}{4cm}
% 	\qrcode[height=3.3cm, version=6]{https://uni-muenster.sciebo.de/public.php?service=files&t=965ae79080a473eb5b6d927d7d8b0462}
% \end{minipage}
% \hfill
% \begin{minipage}{\dieBreite}
% 	\raisebox{-2pt}{\includegraphics[height=0.6cm, keepaspectratio]{../!config/Bilder/sciebo_logo.pdf}}
% 	\resizebox{!}{0.5cm}{\large \sffamily\textbf{sciebo}} {\sffamily\large die Campuscloud} \\
% 	\resizebox{\dieBreite}{!}{\footnotesize\url{https://uni-muenster.sciebo.de/public.php?service=files&t=965ae79080a473eb5b6d927d7d8b0462}}\smallskip\\
% 	Sciebo ist ein Dropbox-Ersatz der Hochschulen in NRW, der von der Uni Münster in leitender Position auf Basis der OpenSource-Software Owncloud aufgebaut wurde.
% \end{minipage}\\[1cm]
\hrule \mbox{ }\\[0.7cm]
\begin{minipage}{4cm}
	\qrcode[height=3.3cm, version=6]{\homepage}
\end{minipage}
\hfill
\begin{minipage}{\dieBreite}
	\resizebox{!}{0.5cm}{\large\sffamily\textbf{Vorlesungshomepage}}\\
	\resizebox{\dieBreite}{!}{\footnotesize\url{\homepage}}\smallskip\\
	Hier ist ein Link zur offiziellen Vorlesungshomepage.
\end{minipage}
\newpage
\section*{Vorwort --- Mitarbeit am Skript}
Dieses Dokument ist eine Mitschrift aus der Vorlesung \enquote{\fach, \semester}, gelesen von \prof. 
Der Inhalt entspricht weitestgehend dem Tafelanschrieb. 
Für die Korrektheit des Inhalts übernehme ich keinerlei Garantie! 
Für Bemerkungen und Korrekturen -- und seien es nur Rechtschreibfehler -- bin ich sehr dankbar. 
Korrekturen lassen sich prinzipiell auf drei Wegen einreichen: 
\begin{itemize}
	\item Persönliches Ansprechen in der Uni, Mails an \hrefsymmail{mailto:\mail}{\mail} (gerne auch mit annotieren PDFs) oder Kommentare auf \url{https://gitlab.com/JaMeZ-B/LaTeX-WWU}.
	\item \emph{Direktes} Mitarbeiten am Skript: Den Quellcode poste ich auf GitLab (siehe oben), also stehen vielfältige Möglichkeiten der Zusammenarbeit zur Verfügung:
	Zum Beispiel durch Kommentare am Code über die Website und die Kombination Fork und Merge-Request. 
	Wer sich verdient macht oder ein Skript zu einer Vorlesung, die ich nicht besuche, beisteuern will, dem gewähre ich gerne auch Schreibzugriff.
	
	Beachten sollte man dabei, dass dazu ein Account bei \url{gitlab.com} notwendig ist, der allerdings ohne Angabe von persönlichen Daten angelegt werden kann. 
	Wer bei GitLab (bzw. dem zugrunde liegenden Open-Source-Programm \enquote{\texttt{git}}) -- verständlicherweise -- Hilfe beim Einstieg braucht, dem helfe ich gerne weiter. 
	Es gibt aber auch zahlreiche empfehlenswerte Tutorials im Internet.\footnote{zB. \url{https://try.github.io/levels/1/challenges/1}, ist auf Englisch, aber dafür interaktiv}
	\item \emph{Indirektes} Mitarbeiten: \TeX-Dateien per Mail verschicken. 
	
	Dies ist nur dann sinnvoll, wenn man einen ganzen Abschnitt ändern möchte (zB. einen alternativen Beweis geben), da ich die Änderungen dann per Hand einbauen muss! Ich freue mich aber auch über solche Beiträge!
\end{itemize}
\section*{Literatur}
\begin{itemize}
	\item \citetitle{Miln} \textcite{Miln}
\end{itemize}
\end{abstract}

\tableofcontents
\cleardoubleoddemptypage

\pagenumbering{arabic}
\setcounter{page}{1}
\setcounter{footnote}{0}

\chapter{Mannigfaltigkeiten} % (fold)
\label{cha:mannigfaltigkeiten}

\section{Definitionen} % (fold)
\label{sec:definitionen}

\begin{definition}[{name=[topologische Mannigfaltigkeit]}]
	Eine \Index{topologische Mannigfaltigkeit} der Dimension $n$ ist ein topologischer Raum $M$ mit den folgenden Eigenschaften
	\begin{enumerate}[1)]
		\item $M$ ist \Index{lokal euklidisch}, das heißt für alle $p \in M$ gibt es eine offene Umgebung $U \subseteq M$ von $p$ und einen Homöomorphismus $h \colon U \to h(U) \subseteq \mathbb{R}^n$, wobei $h(U)$ offen ist.
		\item $M$ ist ein Hausdorffraum.
		\item $M$ erfüllt das zweite Abzählbarkeitsaxiom, das heißt es gibt eine abzählbare Basis der Topologie.
	\end{enumerate}
\end{definition}

\begin{definition}[{name=[{Karte, Atlas, Kartenwechsel}]}]
	Sei $M$ eine topologische Mannigfaltigkeit.
	Ein Paar $(U,h)$, sodass $U \subseteq M$ offen und $h \colon U \to h(U) \subseteq \mathbb{R}^n$ ein Homöomorphismus ist, heißt \Index{Karte}.
	
	Ein \Index{Atlas} von $M$ ist eine Familie $(U_i,h_i)_{i \in I}$ von Karten, mit $\bigcup_{i \in I} U_i = M$.
	Sind $(U,h)$ und $(V,k)$ Karten, so heißt der Homöomorphismus 
	\[
		k \circ h^{-1} \colon h(U \cap V) \longrightarrow k(U \cap V)
	\]
	\Index{Übergangsfunktion} oder \Index{Kartenwechsel}.  
\end{definition}

\begin{figure}
	\Centering
	\begin{tikzpicture}[scale=2.2, every pin edge/.style={black!80, very thin}]
		% \draw[help lines] (-3,-2) grid (7,2);
		\path (-2,-1) coordinate (h0x) ++ (0.2,0.3) coordinate (endh0);
		\path (-0.2,0.1) coordinate (x) ++ (-0.3,0) coordinate (starth0);
		\path (2,-1) coordinate (h1y) ++ (-0.2,0.3) coordinate (endh1);
		\path (0.1,0.15) coordinate (y) ++ (0.3,0) coordinate (starth1);
	
		% sphere
		\shade[ball color=white,opacity=.7] (0,0) circle (1cm);
		\draw (-1,0) arc (180:360:1cm and 0.5cm);
		\draw[dashed, black!60] (-1,0) arc (180:0:1cm and 0.5cm);
		\draw (0,1) arc (90:270:0.5cm and 1cm);
		\draw[dashed, black!60] (0,1) arc (90:-90:0.5cm and 1cm);
		\draw (0,0) circle (1cm);

		% U_0
		\begin{scope}
			\clip (x) circle[radius=0.25];
		  		  	\foreach \y in {0.5,0.45,...,-0.3}%
		  		    	\draw[IndianRed3!80,line width=0.25mm, opacity=0.5] (-0.5,\y) .. controls +(-35:0.4) and +(205:0.4) .. (0.5,\y);
		\end{scope}
		\draw[IndianRed3!80] (x) circle[radius=0.25] ++ (-95:0.43) node {$U$};
	
		% h_0
		\draw[IndianRed3, -to, thick] (starth0) to[bend right] (endh0) ;
		\node[above=6pt,IndianRed3] at (endh0) {$h$};
	
		% V_0
		\begin{scope}
			\clip (h0x) circle[radius=0.3];
		  		  	\foreach \x in {0,-0.1,...,-1}%
		  		    	\draw[IndianRed3!70,line width=0.25mm] (-2.5,\x)--(-1.5,\x-1);
		\end{scope}
		\draw[IndianRed3, thick] (h0x) circle[radius=0.3] ++ (0.4,-0.4) node[below] {$h(U) \supseteq h(U \cap V)$};
		
	
		% U_1
		\begin{scope}
			\clip (y) circle[radius=0.27];
		  		  	\foreach \y in {0.5,0.45,...,-0.3}%
		  		    	\draw[SeaGreen4!80,line width=0.25mm, opacity=0.5] (-0.6,\y) .. controls +(-42:0.5) and +(215:0.5) .. ($(0.6,\y)+(0,.2)$);
		\end{scope}
		\draw[SeaGreen4!80] (y) circle[radius=0.27] ++ (-80:0.43) node {$V$};
	
		% h_1
		\draw[SeaGreen4, -to, thick] (starth1) to[bend left] (endh1) ;
		\node[above=6pt,SeaGreen4] at (endh1) {$k$};
	
		% V_1
		\begin{scope}
			\clip (h1y) circle[radius=0.3];
		  		  	\foreach \x in {0,-0.1,...,-1}%
		  		    	\draw[SeaGreen4!70,line width=0.25mm] (2.5,\x)--(1.5,\x-1);
		\end{scope}
		\draw[SeaGreen4, thick] (h1y) circle[radius=0.3] ++ (-0.4,-0.4) node[below] {$k(U \cap V) \subseteq k(V)$};
		\draw[thick,-To] ($(h0x) + (1.2,-.57)$) -- ++ (1.6,0) node[above,midway]{$k \circ h^{-1}$};
	\end{tikzpicture}
	\caption{Kartenwechsel am Beispiel der 2-Sphäre.}
\end{figure}

Wie treffen die folgende Vereinbarung für diese Vorlesung: Eine Abbildung heißt $f \colon F \to \mathbb{R}^m$ \Index{differenzierbar}, falls $f$ ein $C^\infty$-Abbildung also unendlich oft differenzierbar ist.

\begin{definition}[{name=[differenzierbare Struktur]}]
	Sei $M$ eine topologische Mannigfaltigkeit und $\mathcal{A} = (U_i,h_i)_{i \in I}$ ein Atlas von $M$.
	$\mathcal{A}$ heißt \Index{differenzierbar}, falls alle Kartenwechsel $h_i \circ h_j^{-1} \colon h_j(U_i \cap U_j) \to h_i(U_i \cap U_j)$  $C^\infty$-Diffeomorphismen sind.
	
	Eine \Index{differenzierbare Struktur} auf $M$ ist ein maximaler differenzierbarer Atlas $\mathcal{A}$, das heißt wenn $(U,h)$ eine Karte von $M$ ist, sodass $(U,h) \notin \set*{(U_i,h_i) \given i \in I}$, so ist $\mathcal{A} \cup \set*{(U,h)}$ nicht differenzierbar.
	
	Eine \Index{differenzierbare Mannigfaltigkeit} ist ein Paar $(M,\mathcal{A})$, wobei $M$ eine topologische Mannigfaltigkeit ist und $\mathcal{A}$ eine differenzierbare Struktur auf $M$.
\end{definition}

\begin{bemerkung}[{name=[jeder Atlas ist in einem maximalen enthalten]}]
	Falls $\mathcal{A}= (U_i,h_i)_{i \in I}$ ein differenzierbarer Atlas von $M$ ist, so ist $\mathcal{A}$ in genau einem maximalen differenzierbaren Atlas enthalten, nämlich 
	\[
		\mathcal{A}' = \set*{(U,h) \given \text{Karten von }M : \forall i \in I  \text{ ist } h \circ  h_i^{-1} \text{ ein Diffeomorphismus}}
	\]
\end{bemerkung}

\begin{beispiel}[{name=[Differenzierbare Struktur auf offener Teilmenge]}]
	Sei $U \subseteq \mathbb{R}^n$ offen und $\mathcal{A} = \set*{(U, {\id})}$ ein differenzierbarer Atlas.
	$\mathcal{A}'$ wie oben ist dann eine differenzierbare Struktur.
	Schreibe $U = U_0 \cup U_1$ und betrachte $\overline{\mathcal{A}} = \set*{(U_0,{\id}), (U_1,{\id})}$.
	Es gilt dann $\overline{\mathcal{A}}' = \mathcal{A}'$; dies ist der Zweck der obigen Definition einer differenzierbaren Struktur.
\end{beispiel}

\begin{definition}[{name=[differenzierbare Abbildung]}]
	Seien $M^m, N^n$ differenzierbare Mannigfaltigkeiten.
	Eine Abbildung $f \colon M \to N$ heißt \Index{differenzierbar}, wenn gilt:
	\begin{enumerate}[1)]
		\item $f$ ist stetig
		\item Für jedes Paar $(U,h), (V,k)$ von differenzierbaren Karten von $M$ bzw. $N$ mit $f(U) \subseteq V$ ist die Abbildung
		\[
			\begin{tikzcd}[column sep=large]
				U \dar["h"] \rar["f"] & V \dar["k"]\\
				h(U) \rar[dashed,"k \circ f \circ h^{-1}"] & k(V)
			\end{tikzcd}
		\]
		eine $C^\infty$-Abbildung zwischen offenen Teilmengen des $\mathbb{R}^m$ bzw. $\mathbb{R}^n$.
	\end{enumerate} 
\end{definition}

\begin{bemerkung}[{Überprüfen von Differenzierbarkeit}]
	Um nachzuweisen, dass $f$ differenzierbar ist, reicht es zu zeigen, dass $f$ stetig ist und: Für alle $p \in M$ existieren Karten $(U,h)$ von $M$ mit $p \in U$, $(V,k)$ von $N$ mit $f(U) \subseteq V$ und $k \circ f \circ h^{-1}$ ist $C^\infty$-differenzierbar. 
\end{bemerkung}

\begin{definition}[{name=[Diffeomorphismus]}]
	Seien $M$ und $N$ differenzierbare Mannigfaltigkeiten.
	Ein \Index{Diffeomorphismus} $f \colon M \to N$ ist eine bijektive differenzierbare Abbildung, deren Umkehrfunktion $f^{-1} \colon N \to M$ ebenfalls differenzierbar ist.
\end{definition}

\begin{definition}[{name=[Untermannigfaltigkeit]}]
	Sei $M^m$ eine differenzierbare Mannigfaltigkeit.
	Eine $n$-dimensionale differenzierbare \Index{Untermannigfaltigkeit} $N$ von $M$ ist eine Teilmenge $N \subseteq M$, sodass gilt: Zu jedem $p \in N$ gibt es eine differenzierbare Karte $(U,h)$ von $M$ mit $p \in U$, sodass 
	\[
		h(U \cap N) = (\mathbb{R}^n \times \set*{0}) \cap h(U) \subseteq \mathbb{R}^m
	\]
\end{definition}
% section definitionen (end)

\section*{Konstruktion differenzierbarer funktionen} % (fold)
\label{sec:konstruktion_differenzierbarer_funktionen}
\setcounter{definition}{14}
\begin{bemerkung}
	Sei $f \colon \mathbb{R}\to \mathbb{R}$ definiert durch 
	\[
		f(x) \coloneqq \begin{cases}
			e^{-1/x} &\text{ falls }x>0\\
			0 &\text{ falls } x \le 0
		\end{cases}
	\]
	ist eine $C^\infty$-Funktion.
	Für $a < b$ ist $g_{a,b}(x) = \frac{f(b-x)}{f(x-a) + f(b-x)}$ eine $C^\infty$-Funktion.
	Es ist 
	\[
		g_{a,b}(x)= \begin{cases}
			0 &\text{ falls }x \ge b\\
			1 &\text{ falls } x \le a
		\end{cases}
	\]
	Sei nun $M^n$ eine differenzierbare Mannigfaltigkeit, $p \in U \subseteq M$ offen und $(V,h)$ eine Karte von $M$ mit $p \in V \subseteq U$ und $h(p)=0$.
	Finde nun ein $\delta >0$, sodass $B_\delta(0) \subseteq h(V)$.
	Definiere nun $\varphi \colon M \to [0,1]$ durch
	\[
		\varphi(x) \coloneqq \begin{cases}
			g_{\delta/3,2\delta/3} \enbrace[\big]{\norm*{h(x)}} &\text{ falls }x \in V\\
			0 &\text{ falls } x \notin V
		\end{cases}
	\]
	Dann ist $\varphi$ differenzierbar (Übung; benutzt, dass $M$ hausdorffsch ist).
	$\varphi \equiv 1$ nah bei $p$ und es gilt $\supp(\varphi) \coloneqq \overline{\set*{x \in M \given \varphi(x)\neq 0}} \subseteq U$.
	Der \Index{Träger} $\supp(\varphi)$ ist kompakt.
\end{bemerkung}

\todo[inline]{Das soll jetzt schon 1.17 sein?}
\begin{definition}
	Sei $M$ eie differenzierbare Mannigfaltigkeit, $\mathcal{U} = (U_i)_{i \in I}$ eine offene Überdeckung von $M$.
	eine \Index{untergeordnete Teilung er Eins} ist eine Familie $(\mu_j)_{j \in J}$ von differenzierbaren Funktionen $\mu_j \colon M \to [0,1]$, sodass gilt:
	\begin{enumerate}[(i)]
		\item $\forall j \in J : \exists i \in I : \supp(\mu_j) \subseteq U_i$
		\item $\forall p \in M : \exists \text{ Umgebung } U$ mit $p \in U$, sodass $\set*{j \in J \given \supp(\mu_j) \cap U \neq \emptyset}$ endlich ist.
		\item $\sum_{j \in J} \mu_j(p) = 1$ für alle $p \in M$.
	\end{enumerate}
\end{definition}

\begin{satz}[{name=[Existenz einer untergeordneten Teilung der Eins]},label=satz:ex_teilung_eins]
	Sei $M$ eine differenzierbare Mannigfaltigkeit, $(U_i)_{i \in I}$ eine offene Überdeckung von $M$.
	Dann gibt es eine dieser Überdeckung untergeordnete Teilung der Eins.
\end{satz}

\begin{lemma}[{name=[Existenz kompakter Ausschöpfungen]},label=lem:ex_kpt_ausschoepfung]
	Sei $M$ eine Mannigfaltigkeit.
	Dann gibt es kompakte Teilmengen $K_1 \subseteq K_2 \subseteq \ldots \subseteq M$, sodass $\bigcup_{n=1}^\infty K_n = M$ und sodass $K_n \subseteq \interior{K}_{n+1}$.
\end{lemma}
\begin{beweis}
	\emph{Siehe ATG oder Topologie I.}
\end{beweis}
\begin{beweis}[{name={von \ref{satz:ex_teilung_eins}}}]
	Wähle eine kompakte Ausschöpfung $K_1 \subseteq K_2 \subseteq \ldots $ von $M$ (möglich nach \cref{lem:ex_kpt_ausschoepfung}).
	$K_n \setminus \interior{K}_{n-1}$ ist kompakt und $\interior{K}_{n+1} \setminus K_{n-2}$ ist eine offene Umgebung davon.
	Für $p \in K_n \setminus \interior{K}_{n-1}$ gibt es Funktionen $\varphi \colon M \to [0,1]$ mit $\varphi(p)=1$ und $\supp(\varphi) \subseteq U_i \cap \enbrace*{\interior{K}_{n+1} \setminus K_{n-2}}$ (wie eben).
	Dies impliziert die Existenz von $C^\infty$-Funktionen $\mu_{n,1}, \ldots , \mu_{n,r_n} \colon M \to [0,1]$ mit
	\[
		\supp(\mu_{n,j}) \subset \enbrace*{\interior{K}_{n+1} \setminus K_{n-2}} \cap U_i
	\]
	für ein $i \in I$.
	Es gilt $\mu_{n,1}+ \ldots +  \mu_{n,r_n} > 0$ auf $K_n \setminus \interior{K}_{n-1}$.
	Setze $J \coloneqq \set*{(n,j) \given n \in \mathbb{N}, i \le j \le r_n}$.
	Dann erfüllt $\enbrace*{\mu_{n,j}}_{(n,j) \in J}$ die Punkte 1 und 2.
	Weiter haben wir bereits $\sum_{(n,j) \in J} \mu_{n,j}(p) >0$.
	Setze nun
	\[
		\lambda_{n,j} \coloneqq \frac{\mu_{n,j}}{\sum_{(n,j) \in J} \mu_{n,j}} 
	\]
	Dann ist $(\lambda_{n,j})_{(n,j) \in J}$ die gesuchte Teilung der Eins.
\end{beweis}

% section konstruktion_differenzierbarer_funktionen (end)

\section{Lokale Geometrie differenzierbarer Abbildungen} % (fold)
\label{sec:lokale_geometrie_differenzierbarer_abbildungen}

\begin{definition}[{name=[Rang einer differenzierbaren Abbildung]}]
	Seien $M^m,N^n$ differenzierbare Mannigfaltigkeiten und $f \colon M \to N$ differenzierbar, $p \in M$.
	Wähle Karten $(U,h)$ und $(V,k)$ von $M$ bzw. $N$ mit $p \in U$ und $f(U) \subseteq V$.
	Betrachte 
	\[
		\begin{tikzcd}[column sep=large]
			U \dar["h","\cong"'] \rar["f"] & V \dar["k","\cong"']\\
			h(U) \rar["k \circ f \circ h^{-1}"] & k(V)
		\end{tikzcd}
	\]
	Der Rang von $\D(k \circ f \circ h^{-1})(h(p)) \in \Mat_{n,m}(\mathbb{R})$ hängt nicht von der Wahl von $h,k$ ab und heißt
	\[
		\rank_p(f) \in \mathbb{N}_0
	\]
\end{definition}

\begin{bemerkung}
	Sei $r \in \mathbb{N}_0$.
	Dann ist die Menge $\set*{p \in M \given \rank_p(f) \ge r} \subseteq M$ offen und die Menge $\set*{p \in M \given \rank_p (f) \le r} \subseteq M$ abgeschlossen.
	Der Grund dafür ist: 
	\[
		\set*{A \in \Mat_{n,m}(\mathbb{R}) \given \rank (A) \ge r} \subseteq \Mat_{n,m}(\mathbb{R}) \text{ offen} 
	\]
\end{bemerkung}

\begin{definition}[{name=[{reguläre Punkte/Werte, singuläre Punkte, kritische Werte}]}]
	Sei $f \colon M^m \to N^n$ differenzierbar und $p \in M$.
	\begin{enumerate}[(i)]
		\item $p$ heißt \Index{regulärer Punkt} von $f$, wenn gilt $\rank_p(f) = n$.
		\item $p$ heißt \Index{singulärer Punkt} von $f$, wenn gilt $\rank_p(f) < n$.
		\item $q \in N$ heißt \Index{kritischer Wert} von $f$, wenn gilt $\exists p \in M  : f(p)=q$ mit $p$ singulär.
		\item $q \in N$ heißt \Index{regulärer Wert} von $f$, wenn $q$ kein kritischer Wert ist.
		\item $f$ heißt \Index{Immersion}, wenn für alle $p \in M$ gilt: $\rank_p (f) = m$.
		\item $f$ heißt \Index{Submersion}, wenn für alle $p \in M$ gilt: $\rank_p(f)=n$.  
	\end{enumerate}
\end{definition}

\begin{satz}[{name={Rangsatz}},label=satz:rangsatz]
	Seien $M^m,N^n$ Mannigfaltigkeiten und $f \colon M \to N$ differenzierbar.
	$f$ habe konstanten Rang (d.h. $\rank_p(f) = r$ für alle $p \in M$).
	Dann gibt es zu jedem $p \in M$ Karten $(U,h)$ von $M$, $(V,k)$ von $N$ mit $p \in U$, $h(p)=0$, $f(U) \subseteq V$, $k(f(p))=0$, sodass die Abbildung $k \circ f \circ h^{-1} \colon h(U) \to k(V)$ von der Form 
	\[
		(x_1, \ldots ,x_m) \longmapsto (x_1,\ldots ,x_r,0, \ldots ,0) 
	\]
	ist.
\end{satz}
\begin{beweis}
	Ohne Beschränkung der Allgemeinheit, können wir $M=U \subseteq \mathbb{R}^m$ offen, $p=0$, $f \colon U \to \mathbb{R}^n$ und $f(p)=0$ annehmen.
	Da $\D f(0)$ den Rang $r$ hat, existieren nach Linearer Algebra Matrizen $S \in \GL_m(\mathbb{R})$, $T \in \GL_n(\mathbb{R})$ mit
	\[
		S \cdot \D f(0) \cdot T = 
		\enbrace*{\begin{array}{c|c}
			\scalebox{.8}{$\begin{array}{ccc}
				1 & & \\
				& \ddots & \\
				& & 1
			\end{array}$} & 0 \\ \hline
			0 & 0
		\end{array}}
	\]
	Ersetze $f$ durch $S \circ f \circ T$.
	Dann können wir annehmen, dass $\D f(0)$ bereits die obere Matrixgestalt hat.
	Sei $h \colon U \to \mathbb{R}^m$ gegeben durch 
	\[
		h(x_1, \ldots ,x_m) = \enbrace[\big]{f_1(x), \ldots , f_r(x), x_{r+1}, \ldots , x_m}
	\]
	Dann gilt $h(0)=0$ und $\D h(0) = \ind_m$.
	Nach dem Umkehrsatz aus Analysis II. gibt es $0 \in U_0 \subseteq U$, sodass $h|_{U_0} \colon U_0 \to h(U_0)$ ein Diffeomorphismus ist.
	\[
		\begin{tikzcd}
			U_0 \rar["f"] \dar["h"',"\cong"] & \mathbb{R}^n & & (x_1, \ldots ,x_m) \rar[mapsto] \dar[mapsto,"h"'] & \enbrace*{f_1(x), \ldots , f_n(x)} \\
			h(U_0) \urar["g=f \circ h^{-1}"'] & & &  \enbrace[\big]{f_1(x), \ldots , f_r(x), x_{r+1}, \ldots ,x_m} \urar[mapsto]
		\end{tikzcd}
	\]
	Damit ist $g$ von der Form $g(z_1, \ldots ,z_m) = \enbrace*{z_1, \ldots ,z_r, g_{r+1}(z), \ldots , g_n(z)}$.
	Also folgt
	\[
		\D g(z) = 
		% \begin{pmatrix}
		% 	1 & & &  \\
		% 	& \ddots & & 0 \\
		% 	& & 1 &  \\
		% 	 & * & & A(z)
		% \end{pmatrix}
		\enbrace*{\begin{array}{c|c}
			\scalebox{.8}{$\begin{array}{ccc}
				1 & & \\
				& \ddots & \\
				& & 1
			\end{array}$} & 0 \\ \hline
			* & A(z)
		\end{array}}
	\]
	mit $A(z) \in \Mat_{n-r,m-r}$.
	Für alle $z$ ist $\rank \D g(z) = r$ und somit ist $A(z) = 0$ für alle $z$.
	Damit ist $\diff{g_i(z)}{z_j} = 0$ für alle $r +1 \le i \le n$ und $r+1 \le j \le m$.
	Für genügend kleine $z_{r+1}, \ldots z_m$ gilt also $g_i(z_1, \ldots ,z_m) = g_i \enbrace*{z_1, \ldots , z_r,0, \ldots ,0}$ für alle $r +1 \le i \le n$.
	
	Betrachte nun für eine geeignete Umgebung $V_0$ von 0 die Abbildung
	\mapdef{k \colon V_0}{\mathbb{R}^n}{(y_1, \ldots ,y_n)}{\scriptstyle\enbrace[\big]{y_1, \ldots ,y_r, y_{r+1} - g_{r+1}(y_1, \ldots ,y_r, 0, \ldots ,0), \ldots , y_n - g_n(y_1, \ldots ,y_r, 0, \ldots ,0)}}{}
	Es gilt dann
	\[
		\D k (0) = 
		% \begin{pmatrix}
		% 	1 & & & & & \\
		% 	& \ddots & & & 0 & \\
		% 	& & 1 & & &  \\
		% 	& & & 1 & & \\
		% 	& * & & & \ddots & \\
		% 	& & & & & 1
		% \end{pmatrix}
		\enbrace*{\begin{array}{c|c}
			\scalebox{.8}{$\begin{array}{ccc}
				1 & & \\
				& \ddots & \\
				& & 1
			\end{array}$} & 0 \\ \hline
			* & \scalebox{.8}{$\begin{array}{ccc}
				1 & & \\
				& \ddots & \\
				& & 1
			\end{array}$}
		\end{array}}
	\]
	Diese Matrix ist invertierbar und somit folgt wieder aus dem Umkehrsatz, dass $0 \in V_1 \subseteq V_0$ existiert, sodass $k \colon V_1 \to k(V_1)$ ein Diffeomorphismus ist.
	Setze nun $U_1 \coloneqq f^{-1}(V_1)$.
	Dann gilt für $k \circ f \circ h^{-1} \colon h(U_1) \to k(V_1)$ die Identität $k \circ f \circ h^{-1} = k \circ g$
	\begin{align}
		(z_1, \ldots ,z_m) &\xmapsto{g} \enbrace[\big]{z_1, \ldots ,z_r, g_{r+1}(z), \ldots , g_m(z)}  \\
		&\xmapsto{k} \enbrace[\Big]{z_1, \ldots ,z_r, \Underbrace{g_{r+1}(z) - g_{r+1}(z_1, \ldots ,z_r, 0, \ldots ,0)}{=0 \text{ in kleiner Umgb.}}, \ldots , \Underbrace{g_n(z) - g_n(z_1, \ldots ,z_r,0,\ldots ,0)}{=0 \text{ in kleiner Umgb.}}}
	\end{align}
	Dies war zu zeigen.
\end{beweis}

\begin{satz}[name={vom regulären Wert}]
	Sei $f \colon M^m \to N^n$ differenzierbar.
	Sei $q \in N$ ein regulärer Wert von $f$.
	Dann ist $f^{-1}(q) \subseteq M$ eine Untermannigfaltigkeit der Dimension $m-n$.
\end{satz}
\begin{beweis}
	Sei $p \in f^{-1}(q)$.
	Dann ist $\rank_p(f) = n$ maximal, also gibt es eine Umgebung $U$ von $p$, sodass $\rank_{p'}(f) = n$ für alle $p' \in U$.
	Damit hat $f|_U \colon U \to N$ konstanten Rang.
	Nach dem Rangsatz \ref{satz:rangsatz} existieren Karten $(U_0,h_0)$ von $M$, $U_0 \subseteq U$ mit $h_0(p)=0$ und $(V,k)$ von $N$ mit $f(U_0) \subseteq V$ und $k(q)=0$, sodass
	\[
		\begin{tikzcd}
			U_0 \rar["f"] \dar["h_0"] & V \dar["k"] & q \dar[mapsto] \\
			h(U_0) \rar["k \circ f \circ h_0^{-1}"] & k(V) &  0 \\
			(x_1, \ldots ,x_m) \rar[mapsto] & (x_1, \ldots ,x_n, 0, \ldots ,0)
		\end{tikzcd}
	\]
	Damit ist $h_0(f^{-1}(q) \cap U_0) = \set*{x \given x_1, \ldots ,x_n=0} = h(U_0) \cap \enbrace*{\set*{0} \times \mathbb{R}^{m-n}}$.
\end{beweis}

\begin{beispiel}[{name=[Mannigfaltigkeiten als Urbilder regulärer Werte]}]
	\begin{enumerate}[1)]
		\item Sei $f \colon \mathbb{R}^{n+1} \to \mathbb{R}$ gegeben durch $f(x) = \skal*{x}{x}$.
		Dann gilt $\D f(x) v = 2 \skal*{x}{v}$ und somit ist für alle $x\neq 0$, dass $\rank_x f = 1$ ist.
		Folglich ist $S^n \coloneqq f^{-1}(1)$ eine $n$-dimensionale Untermannigfaltigkeit des $\mathbb{R}^{n+1}$.
		\item $\mathrm{St}_{k,n} = \set*{(v_1, \ldots,v_k) \given v_1i \in \mathbb{R}^n, \skal*{v_i}{v_j} = \delta_{ij}}$ heißt \Index{Stiefel-Mannigfaltigkeit}.
		Es gilt $\mathrm{St}_{1,n}=S^{n-1}$ und $\mathrm{St}_{n,n}$ ist die Menge aller Orthonormalbasen des $\mathbb{R}^n$, also $\On(n)$, die orthogonale Gruppe.
		Wir haben
		\[
			\mathrm{St}_{k,n} = \set*{A \in \Mat_{n,k}(\mathbb{R}) \given A^T A = \ind_k}
		\]
		Für die Abbildung $f \colon \Mat_{n,k}(\mathbb{R}) \to \Sym_{k}(\mathbb{R})$ gegeben durch $A \mapsto A^T A$ gilt $\mathrm{St}_{k,n} = f^{-1}(\ind_k)$.
		Um zu zeigen, dass $\mathrm{St}_{k,n}$ eine Mannigfaltigkeit der Dimension $nk - \sfrac{1}{2}k(k+1)$ ist, müssen wir zeigen, dass $\ind_k$ ein regulärer Wert von $f$ ist:
		Für $A,X \in \Mat_{n,k}(\mathbb{R})$ gilt
		\begin{align}
			\D f(A) \cdot X = \diffd{}{t}\Big|_{t=0} \enbrace*{A + tX}^T \enbrace*{A + tX} = A^T X+ X^T A
		\end{align}
		Wir müssen zeigen, dass falls $A^T A = \ind_k$ gilt, $X \mapsto A^T X + X^T A$ eine surjektive Abbildung $\Mat_{_n,k}(\mathbb{R}) \to \Sym_k(\mathbb{R})$ ist.
		Für $C \in \Sym_k(\mathbb{R})$ ist $\frac{1}{2} AC$ ein Urbild.
		
		\emph{Bemerkung:} $\On(n)$ ist eine Untermannigfaltigkeit von $\Mat_{n,n}(\mathbb{R})$.
		Sei $\mu \colon \On(n) \times \On(n) \to \On(n)$ gegeben durch Matrixmultiplikation und $\iota \colon \On(n) \to \On(n)$ die Inversion.
		Beide Abbildungen sind differenzierbar, wodurch $\On(n)$ zu einer \Index{Lie-Gruppe} wird.
		\item $\GL_n(\mathbb{R}) \subseteq \Mat_{n,n}(\mathbb{R})$ ist offen.
		Die Determinante $\det \colon \GL_n(\mathbb{R}) \to \mathbb{R}$ ist differenzierbar.
		Es gilt
		\[
			\D \det (A) \cdot X = \det (A) \cdot \tr(A^{-1} \cdot X)
		\]
		Dazu rechnet man nach, dass 
		\[
			\diffd{}{t}\Big|_{t=0} \det \enbrace*{\ind + t E_{ij}} = \begin{cases}
				0 &\text{ falls }i \neq j\\
				1 &\text{ falls } i =j
			\end{cases}
		\]
		Also stimmt die Formel für $A=\ind$.
		Nun gilt
		\[
			\D \det(A) \cdot X = \diff{}{t}\Big|_{t=0} \det (A + tX) = \det A \cdot \diffd{}{t}\Big|_{t=0} \det \enbrace*{1 + tA^{-1} X}
		\]
		Also ist $1 \in \mathbb{R}$ ein regulärer Wert von $\det$.
		Damit ist $\SL_n(\mathbb{R}) = \det^{-1}(1)$ eine Mannigfaltigkeit.
	\end{enumerate}
\end{beispiel}

Noch eine Anmerkung zum Satz vom regulären Wert: Wenn $n >m$, so ist $q$ ein regulärer Wert von $f$ genau dann, wenn $f^{-1}(q)=\emptyset$.
Es gilt folgende Konvention: $\emptyset$ ist eine Mannigfaltigkeit beliebiger, auch negativer Dimension.

\todo[inline]{soll 2.8 sein \ldots }
\begin{satz}[{name={Sard}},label=satz:sard]
	Sei $U \subset \mathbb{R}^m$ offen, $f \colon U \to \mathbb{R}^n$ sei $C^\infty$.
	Setze
	\begin{align}
		S(f) &= \set*{x \in U \given \D f(x) \text{ nicht surjektiv}} \subseteq U \\
		C(f) &= f\enbrace*{S(f)}
	\end{align}
	Dann hat $C(f)$ das Lebesgue-Maß $0$.
\end{satz}
\begin{beweis}
	Siehe \textcite[S.~113~ff]{brocker1992analysis}.
\end{beweis}

Diese Nullmenge muss nicht notwendigerweise abgeschlossen sein und kann sogar dicht sein!
Im Fall $n>m$ gilt $S(f)=U$ und somit $C(f)=f(U)\subseteq \mathbb{R}^n$.
Dann folgt aus \cref{satz:sard}, dass $f(U)$ eine Nullmenge ist.

\begin{definition}[{name=[Nullmenge auf einer Mannigfaltigkeit]}]
	$M$ sei eine Mannigfaltigkeit.
	$S \subseteq M$ heißt \Index{Nullmenge}, falls es einen Atlas $(U_i,h_i)_{i \in I}$ von $M$ gibt, sodass $h_i(S \cap U_i) \subseteq \mathbb{R}^m$ Lebesgue-Maß $0$ hat.
\end{definition}

\begin{satz}[{name={aus Analysis III}}]
	Sei $U \subseteq \mathbb{R}^n$ offen und $f \colon U \to \mathbb{R}^n$ differenzierbar.
	Wenn $S \subseteq U$ eine Nullmenge ist, so ist $f(S)$ auch eine Nullmenge.
\end{satz}

\begin{lemma}[label=lem:mfkt_nullmengen]
	Sei $M^m$ eine Mannigfaltigkeit und $S \subseteq M$ eine Nullmenge.
	\begin{enumerate}[1)]
		\item Für jede Karte $(U,h)$ von $M$ ist $h(U \cap S) \subseteq \mathbb{R}^m$ eine Nullmenge.
		\item $M\setminus S$ liegt dicht in $M$.
		\item Wenn $S_1, S_2, \ldots $ Nullmengen sind, so ist $\bigcup_{i=1}^\infty S_i$ eine Nullmenge
	\end{enumerate}
\end{lemma}
\begin{beweis}
	Für die erste Aussage überdecke $U$ durch abzählbar viele $U_i$.
	Sei nun $h_i \colon U_i \to \mathbb{R}^n$ eine Karte.
	Dann ist $h_i(U_i \cap S)$ eine Nullmenge und es folgt
	\[
		h(S \cap U) = \bigcup_{i=1}^\infty h \enbrace*{S \cap U \cap U_i} = \bigcup_{i=1}^\infty \Underbrace{h \circ h_i^{-1} \enbrace[\Big]{\Underbrace{h_i \enbrace*{S \cap U_i \cap U}}{\text{Nullmenge}}}}{\text{Nullmenge, da $h \circ h_i^{-1}$ Diffeo}}
	\]
	Damit ist $h(S \cap U)$ also eine Nullmenge.
	
	Zur zweiten Aussage: Angenommen $M \setminus S$ wäre nicht dicht.
	Dann existiert ein Punkt $p \in M$ mit $p \in V \subseteq S$ mit $V \subseteq M$ offen.
	Sei $h \colon V \to \mathbb{R}^n$ eine Karte.
	Da $h(V)$ offen in $\mathbb{R}^n$ ist und $h(V) \neq \emptyset$ gilt, folgt $\mu \enbrace*{h(V)}>0$.
	Also ist $S$ keine Nullmenge nach 1). Widerspruch!
	
	Die dritte Aussage folgt aus der ersten mit der $\sigma$-Additivität des Lebesgue-Maßes.
\end{beweis}

\begin{satz}[{name={Sard, globale Version}}]
	Sei $f \colon M \to N$ differenzierbar.
	Dann ist die Menge der singulären Werte von $f$ eine Nullmenge.
	
	Insbesondere ist die Menge der regulären Werte von $f$ dicht in $N$.
\end{satz}
\begin{beweis}
	Es gibt abzählbare Atlanten $(U_i,h_i)_{i \in \mathbb{N}}$, $(V_i,k_i)_{i \in \mathbb{N}}$ mit $f(U_i) \subseteq V_i$.
	Setze $f_i \coloneqq k_i \circ f \circ h_i^{-1}$ und 
	\[
		C(f) = f \enbrace[\big]{S(f)} = \bigcup_{i=1}^\infty k_i^{-1} \enbrace[\big]{C(f_i)}
	\]
	Nach \cref{satz:sard} sind die $C(f_i)$ Nullmengen in $\mathbb{R}^n$.
	Damit sind auch die $k_i^{-1}\enbrace[\big]{C(f_i)}$ Nullmengen in $N$ nach \cref{lem:mfkt_nullmengen} 1) und mit \cref{lem:mfkt_nullmengen} 3) ist $C(f)$ also eine Nullmenge.
\end{beweis}

\section*{Immersionen und Einbettungen}

\begin{satz}[label=satz:immersion_bild]
	Sei $f \colon M^m \to N^n$ eine Immersion (also $\rank_p f =m$ für alle $p \in M$).
	Dann gibt es zu jedem $p \in M$ eine Umgebung $p \in U \subseteq M$, sodass $f(U) \subseteq N$ eine $m$-dimensionale Untermannigfaltigkeit ist.
\end{satz}
\begin{beweis}
	Sei $p \in M$.
	Der Rangsatz liefert uns Karten $(U,h)$ von $M$ und $(V,k)$ von $N$ mit $p \in U$, $h(p)=0$, $f(p) \in V$, $k(f(p))=0$ und $f(U) \subseteq V$, sodass
	\[
		\begin{tikzcd}
			U \rar["f"] \dar["h","\cong"'] & V \dar["\cong","k"'] \\
			h(U) \rar & k(V) \\[-2em]
			(x_1, \ldots ,x_m) \rar[mapsto] & (x_1, \ldots ,x_m, 0, \ldots ,0)
		\end{tikzcd}
	\]
	Damit folgt direkt die Behauptung.
\end{beweis}

Ist eine entsprechende globale Aussage möglich? Betrachte die folgende Abbildung $f \colon \mathbb{R} \to \mathbb{R}^2$.
Problem: $f$ ist nicht injektiv.
\missingfigure{Schleifenbild}

Problem 2: Auch eine injektive Immersion muss kein Homöomorphismus auf dem Bild sein.
\missingfigure{zweite Schleife}

Auch beim Torus $T^2= S^1 \times S^1$ kann man eine Schleife mit Steigung $\theta \notin \pi \cdot \mathbb{Q}$ betrachten.
Dann ist das Bild keine Untermannigfaltigkeit!

\begin{definition}[{name=[Einbettung]}]
	Eine differenzierbare Abbildung $f \colon M^m \to N^n$ heißt \Index{Einbettung},\marginnote{$f$ ist eine injektive Immersion} wenn $f(M) \subseteq N$ eine Untermannigfaltigkeit ist und $f \colon M \to f(M)$ ein Diffeomorphismus ist.
\end{definition}

\begin{definition}[{name=[eigentliche Abbildung]}]
	Eine stetige Abbildung $f\colon X \to Y$ von topologischen Räumen heißt \Index{eigentlich}, falls für jedes kompakte $K \subseteq Y$ das Urbild $f^{-1}(K) \subseteq X$ kompakt ist.
\end{definition}

Unser Ziel ist  es nun zu zeigen, dass eine eigentliche Immersion eine Einbettung ist.

Falls $X,Y$ metrische Räume sind, dann ist $f \colon X \to Y$ eigentlich genau dann, wenn aus $f(x_n) \to y$ stets folgt, dass $(x_n)_n$ ein konvergente Teilfolge hat.

\begin{lemma}[label=lem:bild_abgeschl]
	Wenn $X,Y$ hausdorffsch und lokalkompakt sind und $f \colon X \to Y$ eigentlich.
	Dann ist $f$ abgeschlossen, das heißt falls $A \subseteq X$ abgeschlossen ist, ist auch $f(A) \subseteq Y$ abgeschlossen.
\end{lemma}
\begin{beweis}
	Betrachte die Einpunktkompaktifizierung $X^+ \coloneqq X \cup \set*{\infty}$.
	$X^+$ ist kompakt und hausdorffsch.
	Definiere $f^+ \colon X^+ \to Y^+$ durch Fortsetzen von $f$ mittels $f(\infty)=\infty$.
	Da $f$ eigentlich ist, ist $f$ stetig (warum?).
	
	Sei nun $A \subseteq X$ abgeschlossen.
	Dann ist $A \cup \set*{\infty} \subseteq X^+$ abgeschlossen und kompakt.
	Mit der Stetigkeit von $f^+$ folgt die Kompaktheit von $f^+(A \cup \set*{\infty}) = f(A) \cup \set*{\infty}$.
	Da $Y^+$ hausdorffsch ist, ist $f(A) \cup \set*{\infty} \subseteq Y^+$ abgeschlossen.
	Damit ist $f(A) = (f(A) \cup \set*{\infty}) \cap Y \subseteq Y$ abgeschlossen.
\end{beweis}

\begin{korollar}[label=kor:homeo_bild]
	Seien $X,Y$ lokalkompakt, hausdorffsch und $f \colon X \to Y$ eigentlich und injektiv.
	Dann ist $f(X) \subseteq Y$ abgeschlossen und $f \colon X \to f(X)$ ein Homöomorphismus.
\end{korollar}
\begin{beweis}
	Die erste Aussage folgt aus \cref{lem:bild_abgeschl}.
	Die zweite sie $A \subseteq X$ abgeschlossen.
	Aus dem \cref{lem:bild_abgeschl} folgt, dass $f(A) \subseteq Y$ abgeschlossen ist und somit ist auch $f(A) \subseteq f(X)$ abgeschlossen.
	Also ist $A \subseteq X$ genau dann abgeschlossen, wenn $f(A) \subseteq f(X)$ abgeschlossen ist.
	Da $f \colon X \to f(X)$ bijektiv ist, ist $f \colon X \to f(X)$ ein Homöomorphismus. 
\end{beweis}

\begin{satz}
	Eine bijektive, eigentliche Immersion $f \colon M^m \to N^n$ ist eine Einbettung.
\end{satz}
\begin{beweis}
	$p \in M$ und $q =f(p) \in N$.
	Wie im Beweis von \cref{satz:immersion_bild} erhalten wir Karten $(U,h), (V,k)$ mit
	\[
		\begin{tikzcd}
			U \rar["f"] \dar["h","\cong"'] & V \dar["\cong","k"'] \\
			h(U) \rar & k(V) \\[-2em]
			(x_1, \ldots ,x_m) \rar[mapsto] & (x_1, \ldots ,x_m, 0, \ldots ,0)
		\end{tikzcd}
	\]
	Nach \cref{kor:homeo_bild} ist $f \colon M \to f(M)$ ein Homöomorphismus.
	Damit gibt es eine offene Umgebung $W$ von $q$ in $N$ mit $U =f^{-1}(W)$ (warum?).
	Betrachte $k' \coloneqq k|_{V \cap W}$.
	Es gilt 
	\[
		k'(f(M) \cap V \cap W) = (\mathbb{R}^m \times \set*{0}) \cap k'(V \cap W)
	\]
	Damit ist $f(M)$ eine Untermannigfaltigkeit mit $\dim f(M) =m$, aber $\rank_p(f)=m$.
	Also ist $f \colon M \to f(M)$ ein Homöomorphismus und insbesondere $f^{-1} \colon f(M) \to M$ stetig und sogar differenzierbar nach dem Umkehrsatz. 
\end{beweis}

% section lokale_geometrie_differenzierbarer_abbildungen (end)
% chapter mannigfaltigkeiten (end)


\cleardoubleoddemptypage
\pagenumbering{Alph}
\setcounter{page}{1}
\cleardoubleoddemptypage
\appendix

\chapter{Anhang} % (fold)
\label{sec:anhang}
%!TEX root = ana_top_geo.tex

\subsection{Ausführlicher Beweis zu \cref{lem:kpt-schnitte}} % (fold)
\label{sub:kpt-schnitte}
Sei $X$ ein Hausdorffraum. Dann ist $X$ genau dann kompakt, wenn gilt: Hat eine Familie $\mathcal{A}$ von abgeschlossenen Teilmengen von $X$ die endliche 
Durchschnittseigenschaft, so gilt 
\[
	\bigcap_{A \in \mathcal{A}} A \not= \emptyset.
\]
\begin{beweis}
	Für die erste Implikation sei $X$ kompakt und $\mathcal{A}$ eine Familie von abgeschlossenen Mengen mit der endlichen Durchschnittseigenschaft.
	Angenommen $\bigcap_{A \in \mathcal{A}} A = \emptyset$.
	Dann gilt
	\[
		X = X \setminus \bigcap_{A \in \mathcal{A}} A = \bigcup_{A \in \mathcal{A}} X \setminus A.
	\]
	Nun ist $\mathcal{U} \coloneqq \set*{X \setminus A \given A \in \mathcal{A}}$ eine offene Überdeckung von $X$ und da $X$ kompakt ist, existiert $\mathcal{A}_0 \subset \mathcal{A}$ endlich, sodass
	\[
		X = \bigcup_{A \in \mathcal{A}_0} X \setminus A = X \setminus \underbrace{\bigcap_{A \in \mathcal{A}_0 } A }_{\neq \emptyset} \quad \light
	\]
	Für die umgekehrte Implikation sei nun $\mathcal{U} = \set{U_i}_{i \in I}$ eine offene Überdeckung von $X$.
	Angenommen für jede endliche Teilmenge $J \subseteq I$ gilt $X \neq \bigcup_{i \in J} U_i$.
	Betrachte nun $\mathcal{A} =  \set{X \setminus U_i}_{i \in I}$. Dann gilt nach Annahme
	\[
		\bigcap_{i \in J} X \setminus U_i = X \setminus \bigcup_{i \in J} U_i \neq \emptyset.
	\]
	Also hat $\mathcal{A}$ die endliche Durchschnittseigenschaft. Nach Vorraussetzung gilt dann
	\[
		\emptyset \not= \bigcap_{i \in I} X \setminus U_i = X \setminus \underbrace{\bigcup_{i \in I} U_i}_{= X} \quad \light \qedhere
	\]
\end{beweis}


\subsection[Blatt3, Aufgabe 4: Hilfssatz für den Hauptsatz der Algebra]{Blatt 3, Aufgabe 4} % (fold)
\label{sub:B3A4}
\emph{Diese Übungsaufgabe ist zentral für den Beweis des Hauptsatzes der Algebra, \cref{satz:hauptsatz-algebra}.} 

Sei $p(x)= x^n + a_{n-1} x^{n-1} + \ldots + a_1 x + a_0$ mit $n \in \mathbb{N}_0$ ein Polynom mit Koeffizienten $a_i \in \mathbb{C}$, dass \emph{keine} Nullstelle in $\mathbb{C}$ besitzt. 
Sei $S^1= \set*{z \in \mathbb{C} \given \abs*{z}=1}$.
\begin{enumerate}[(a)]
	\item $f \colon S^1 \to S^1$ gegeben durch $f(z) = \frac{p(z)}{\abs*{p(z)} } $ ist wohldefiniert und homotop zu einer konstanten Abbildung.
	\item $f$ ist homotop zur Abbildung $g_n \colon S^1 \to S^1$ mit $g_n(z)= z^n$.
\end{enumerate}
\minisec{Beweis}
\begin{enumerate}[(a)]
	\item \begin{description}
		\item[Wohldefiniertheit:] Sei $z \in S^1$ beliebig. Dann gilt
		\[
			\abs*{\frac{p(z)}{\abs*{p(z)} } } = \frac{1}{\abs*{p(z)} } \cdot \abs*{p(z)} =1,
		\]
		also ist $f(z) \in S^1$.
		\item[Homotop zu einer konstanten Abbildung:] Definiere $f_t \colon S^1 \to S^1$ für $t \in [0,1]$ durch 
		\[
			f_t(z) = \frac{p(t \cdot z)}{\abs*{p(t \cdot z)} } 
		\]
		Dies ist mit der gleichen Begründung wie oben wohldefiniert. 
		Außerdem ist $f_0(z)= \frac{a_0}{\abs*{a_0} } \in S^1 $ konstant und $f_1(z)= \frac{p(z)}{\abs*{p(z)} }=f(z)$. 
		Definiere nun $H \colon S^1 \times [0,1] \to S^1$ durch $H(x,t) \coloneqq f_t(x)$. 
		Dann ist $H$ stetig, da Polynome und $\abs*{.} $, sowie Multiplikation stetig sind. 
		$H$ ist die gesuchte Homotopie.
	\end{description}
	\item Sei $h \colon S^1 \times [0,1] \to \mathbb{C}$ gegeben durch $h(z,t) = z^n + \sum_{k=0}^{n-1} a_k z^k t^{n-k}$. 
	Dann gilt $h(z,0)=z^n \not= 0$, da $z \in S^1$.
	Für $t \neq 0$ gilt nun
	\begin{align*}
		h(z,t) = 0 \iff \frac{h(z,t)}{t^n} = 0 \iff \frac{z^n}{t^n} + \sum_{k=0}^{n-1} a_k \frac{z^k}{t^k} = 0 \iff p \enbrace*{\frac{z}{t}} = 0
	\end{align*}
	Aber nach Vorraussetzung gilt $p \enbrace*{\frac{z}{t}} \neq 0$. 
	Also $h(z,t) \neq 0$ für alle $t \in [0,1]$. 
	Definiere nun $H \colon S^1 \times [0,1]\to S^1$ durch $H(z,t) = \frac{h(z,t)}{\abs*{h(z,t)}}$. 
	Wie eben gezeigt, ist dies wohldefiniert und offensichtlich stetig. Da
	\[
		H(z,0) = \frac{z^n}{\abs*{z^n} } = z^n \quad \text{ und } \quad H(z,1) = \frac{h(z,1)}{\abs*{h(z,1)} } = \frac{p(z)}{\abs*{p(z)} } =f(z)
	\]
	ist $H$ die gesuchte Homotopie. \qedhere
\end{enumerate}

\subsection{Blatt 10, Aufgabe 3} % (fold)
\label{sub:B10A3}
\emph{Diese Übungsaufgabe lieferte den Beweis zu \cref{prop:iso-covering}.} \smallskip \\
Sei $p \colon \overline{X} \to X$ eine Überlagerung. 
Seien $\overline{x}_0  \in \overline{X}$ und $x_0= p(\overline{x}_0 )$ Basispunkte. 
Dann ist die induzierte Abbildung $\pi_n (p) \colon \pi_n(\overline{X}, \overline{x}_0) \to \pi_n(X,x_0)$ ein Isomorphismus für alle $n \ge 2$.
\minisec{Beweis}
Als Überlagerung ist $p$ stetig, also ist $\pi_n(p)$ ein Gruppenhomomorphismus nach \hyperref[prop:eig-hom-gruppen:enum:4]{ \cref*{prop:eig-hom-gruppen} \ref*{prop:eig-hom-gruppen:enum:4}}.
\begin{description}
	\item[Surjektivität:] Sei $[\omega] \in \pi_n(X,x_0)$, also $\omega \colon I^n \to X$ mit $\omega(\partial I^n) = \set{x_0}$. Betrachte $\omega$ nun als Abbildung $I^{n-1} \times [0,1] \to X$:
	\[
		\begin{tikzcd}[column sep=4em]
			I^{n-1} \times \set{0} \dar[hook] \rar["\mathrm{const}_{\overline{x}_0}"] & \overline{X} \dar["p"]\\
			I^{n-1} \times I \rar["\omega"] & X  
		\end{tikzcd}
	\]
	$\mathrm{const}_{\overline{x}_0} \colon I^{n-1} \times \set{0}$ ist eine Hebung von $\omega\big|_{I^{n-1} \times \set{0}} \equiv x_0$. 
	Nach dem Homotopiehebungssatz (\ref{satz:hebung-homotopie}) existiert eine Hebung $\overline{\omega} \colon I^{n-1} \times I \to \overline{X}$ von $\omega$ mit $\overline{\omega}\big|_{I^{n-1} \times \set{0}} \equiv \overline{x}_0 $. 
	Also gilt
	\[
		p \circ \overline{\omega} \big|_{\partial I^n} = \omega \big|_{\partial I^n} \equiv x_0 \enspace \Longrightarrow \enspace \overline{\omega} \big|_{\partial I^n} 
		\in p ^{-1}( \set{x_0} ) .
	\]
	Da $p^{-1}(\set{x_0})$ diskret und $\partial I^n$ für $n \ge 2$ zusammenhängend ist, muss $\overline{\omega} \big|_{\partial I^n}$ konstant sein. 
	Da $\overline{\omega}\big|_{I^{n-1} \times \set{0}} \equiv \overline{x}_0 $ gilt, folgt somit $\overline{\omega}(\partial I^n) = \set{\overline{x}_0}$. 
	Also ist $[\overline{\omega}] \in \pi_n(\overline{X},\overline{x}_0)$ und weiter gilt
	\[
		\pi_n(p) \enbrace*{[\overline{\omega}]} = [p \circ \overline{\omega} ] = [\omega] \in \pi_n(X,x_0). 
	\]
	\item[Injektivität:] Sei $[\omega] \in \ker \pi_n(p)$, also $[p \circ \omega] = [c_{x_0}]$. 
	Es existiert also eine Homotopie $H$ relativ $\partial I^n$ zwischen $p \circ \omega$ und $c_{x_0}$. 
	Offensichtlich ist $\omega$ eine Hebung von $p \circ \omega$. 
	Mit dem Homotopiehebungssatz erhalten wir eine Hebung $\overline{H}$ von $H$ mit $\overline{H}(-,0) = \omega$. 
	Weiter wissen wir, dass
	\[
		\overline{H} \big|_{\partial I^n \times [0,1]} \in p ^{-1}(\set{x_0} ) \quad \text{ und }\quad  \overline{H} \big|_{ I^n \times \set{1}} \in p ^{-1}(\set{x_0} )
	\]
	gelten muss, da $H = p \circ \overline{H}$ und $H(-,1)= c_{x_0} \equiv x_0$. 
	Mit dem gleichen Argument wie oben folgt, dass $\overline{H} \big|_{\partial I^n \times [0,1]}$ und $\overline{H} \big|_{ I^n \times \set{1}}$ konstant sind. 
	Für $z \in \partial I^n$ gilt nun
	\[
		\overline{H}(z,0) = \omega(z) = \overline{x}_0
	\]
	Da $\partial I^n \times [0,1] \cap I^n \times \set{1} \not= \emptyset$, muss also auch $\overline{H}(-,1) \equiv \overline{x}_0$ gelten. 
	Damit folgt $[\omega] = [c_{x_0}]$.\qedhere
\end{description}
\printindex
\printbibliography
\listoffigures
\listoftodos[To-do's und andere Baustellen]
\todototoc
\end{document}
