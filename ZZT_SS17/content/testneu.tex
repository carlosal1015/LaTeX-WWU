%!TEX root = ../ZZT_SS17.tex
\section{Yeah!}
Huhu! \todo{Bah!}

\[
	\int_{a}^{b} \sqrt{\frac{1}{\sqrt{5}} f(x)} dx
\]
Was \enquote{geht}? \marginnote{Iiih!}

\begin{enumerate}[(i)]
	\item Das
	\item ist 
	\item eine
	\item Aufzählung.
\end{enumerate}

\begin{description}
	\item[Das] ist
	\item[eine] Beschreibung.
\end{description}

\begin{definition}[Fibonacci-Folge]
	\label{def:fibonacci}
	Wir definieren für $n \in \NN_0$ die \Index{Fibonacci-Folge} durch:
	\[
		F_n := \begin{cases}
			0, & \text{falls } n = 0, \\
			1, & \text{falls } n = 1, \\
			F_{n-1} + F_{n-2}, & \text{sonst.}
		\end{cases}
	\]
\end{definition}

\begin{beispiel}
	Für $n = 4$ ist
	\[
		F_4 = F_2 + F_3 = F_0 + F_1 + F_1 + F_2 = 0 + 1 + 1 + F_0 + F_1 = 2 + 0 + 1 = 3.
	\]
\end{beispiel}

\begin{satz}[Satz von de Moivre-Binet]
	Für $n \in \NN_0$ und $F_n$ wie in \thref{def:fibonacci} gilt:
	\[
		F_n = \frac{1}{\sqrt{5}} \benb{ \enb{\frac{1+\sqrt{5}}{2}}^n - \enb{\frac{1-\sqrt{5}}{2}}^n}
	\]
\end{satz}

\begin{beweis}
	Per Induktion.
\end{beweis}

\begin{korollar}[Fibonacci-Folge und Goldener Schnitt]
	Für $\Phi = \frac{1+\sqrt{5}}{2}$ gilt:
	\begin{enumerate}[(i)]
		\item \[
			\lim\limits_{n \rightarrow \infty} \frac{F_{n+1}}{F_n} = \Phi.
		\]
		\item \[
			\Phi = 1 + \cfrac{1}{1+ \cfrac{1}{1 + \cfrac{1}{1 + \cfrac{1}{1 + \cfrac{1}{1 + \cdots}}}}}
		\]
	\end{enumerate}
\end{korollar}
\lipsum

\section{Alles gut bis hierhin.}
\begin{definition}[Test]
	Eine Testdefinition.
\end{definition}