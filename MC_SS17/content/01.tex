%!TEX root = ../MC_SS17.tex
\section{Einführung}
\label{sec:para1}
\nextlecture
\subsection*{Transformationelle Programme vs. reaktive Systeme}
\subsubsection*{Transformationelle Programme}
\begin{itemize}
	\item Aus einer Eingabe wird eine Ausgabe berechnet:
	\item[] \todo{Grafik einfügen}
	\item wird beispielsweise in der Algorithmik studiert
	\item typische Anforderung: Das Programm soll terminieren
	\item Klassische Programmverifikationsmethoden wie die Hoare-Logig oder die Methode von Floyd werden in den Vorlesungen \enquote{Theorie der Programmierung} und \enquote{Formale Methoden der Software-Entwicklung} behandelt.
\end{itemize}

\subsection*{Reaktive Systeme}
\begin{itemize}
	\item[] \todo{Grafik einfügen}
	\item Programm steht in ständiger Interaktion mit seiner Umgebung
	\item Beispiele: Betriebssysteme, Netzwerkkontext, Kommunikationsprotokolle, Eingebettete Systeme: \underline{Ampel}, \underline{Flugzeug}, \underline{ABS}, Fernseher, Waschmaschine, \underline{Medizinische Geräte}, \dots
	\item[] \textunderscore: oft sicherheitskritisch $\Rightarrow$ Hohe Anforderung an Korrektheit
	\item Typisch für reaktive Systeme: Nicht terminierend
\end{itemize}

\subsection*{Model Checking}

\cleardoubleoddemptypage