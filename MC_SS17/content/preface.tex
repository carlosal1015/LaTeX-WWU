%!TEX root = ../MC_SS17.tex
\begin{abstract}
	\section*{Vorwort}
	\label{sec:preface}
	Der vorliegende Text ist eine Mitschrift der Vorlesung \textit{Model Checking}, gelesen von Prof. Dr. Markus Müller-Olm an der WWU Münster im Sommersemester 2017. Der Inhalt entspricht weitestgehend dem Tafelanschrieb. Dieses Werk ist daher keine Eigenleistung des Autors und wird nicht von der Dozentin der Veranstaltung korrekturgelesen. Für die Korrektheit des Inhalts wird keinerlei Garantie übernommen. Bemerkungen, Korrekturen und Ergänzungen kann man folgenderweise loswerden:
	\begin{itemize}
		\item persönlich durch Überreichen von Notizen oder per E-Mail
		\item durch Abändern der entsprechenden \TeX-Dateien und Versand per E-Mail an mich
		\item direktes Mitarbeiten via GitLab. Dieses Skript befindet sich im \texttt{LaTeX-WWU}-Repository von Jannes Bantje:
		\begin{center}
			\url{https://gitlab.com/JaMeZ-B/latex-wwu}
		\end{center}
	\end{itemize}
	
	\subsection*{Literatur}
	\label{sub:lit}
	\begin{itemize}
		\item C. Baier und J.-P. Katoen, Principles of Model Checking, MIT Press, 2008.
		\item B. Berard, M. Bidoit, A. Finkel, F. Larroussinie, A. Petit, L. Petrucci, Ph. Schnoebelen, Systems and Software Verification, Springer-Verlag, 2001.
	\item E. M. Clarke, O. Grumberg, D. A. Peled, Model Checking, MIT Press, 1999
		
	\end{itemize}
	
	\subsection*{Kommentar des Dozenten}
	Eine zunehmend auch von der Industrie eingesetzte Technik zur Aufdeckung versteckter Fehler in Hard- und Softwaresystemen ist das sogenannte Model-Checking. Ein Model Checker überprüft automatisch, ob ein Modell eines Systems eine gewünschte, typischerweise in einer temporalen Logik spezifizierte Eigenschaft besitzt. Die Vorlesung behandelt die verschiedenen Ansätze zur Konstruktion von Model-Checkern sowie theoretische und praktische Fragestellungen im Umfeld dieser Methode.
	\subsection*{Vorlesungswebsite}
	\label{sub:link}
	\begin{center}
		\url{https://www.uni-muenster.de/Informatik.AGMueller-Olm/teaching/ss17/mc.shtml}
	\end{center}
	
	\vfill
	\begin{flushright}
		Florian Küpper \\
		f.k@wwu.de
	\end{flushright}
	\newpage
\end{abstract}
\cleardoubleemptypage