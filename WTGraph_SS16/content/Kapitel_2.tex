%!TEX root = ./WTG.tex

Heute wollen wir eine weitere Größe kennenlernen, die bei der Beantwortung von Fragen über Irrfahrten hilft

\section{Energie}
Sei zunächst $G = \enb{V,E,c}$ ein endliches Netzwerk.
\begin{definition}
	Sei $l^2\enb{V}$ die Menge aller Funktionen $F: V \to Real$. $l^2(V)$ sei versehen mit dem üblichen Skalarprodukt 
	\begin{gather}
		<f,g> := \sum\limits_{v \in V} f(v)g(v), ||f||^2 = <f,f>
	\end{gather}	
\end{definition}
\begin{definition}
	Sei $l^2(E)$ die Menge aller antisymmetrischen Funktionen 
	\begin{gather}
		\Theta = E \to \Real \text{mit } \Theta(-e) = -\Theta(e)
	\end{gather}
	mit Skalarprodukt 
	\begin{gather}
		<\Theta,\Theta'> = \frac{1}{2}\sum\limits_{e \in E} \Theta(e) \Theta'(e) = \sum\limits_{e \in E_{1/2}} \Theta(e)\Theta'(e)
	\end{gather}
	wobei $E_{1/2}$ eine Teilmenge von $E$, die von jedem Paar $(e,-e), e \in E$ genau einen Vertreter enthält.
\end{definition}

\begin{definition}
	Sei $d:l^2(V) \to l^2(E)$ der \emph{Co-Rand-Operator}, definiert als
	\begin{gather}
		(df)(e) =  f(e^-) - f(e^+) \text{wobei } e= (e^-,e^+)
	\end{gather}
	Weiter sei $d^*: l^2(E) \to l^2(V)$ der Rand Operator, definiert als
	\begin{gather}
		(d^*\Theta)(x) = \sum\limits_{e:e^-=x} \Theta(e)
	\end{gather}
\end{definition}
\begin{uebung}
	$\forall f \in l^2(V), \forall \Theta \in l^2_-(E)gilt$
	\begin{gather}
		<df,\Theta>_{l^2_-E} = <f,d^*\Theta>_{l^2(V)}
	\end{gather}
	Man kann Ohm und Kirchhoff mittels $d$ und $d^*$ elegant aufschreiben.
	\begin{itemize}
		\item Ohm: $dv = ir$, d.h. $\forall \ c \in E$ gilt $(dv)(e) = i(e) \cdot r(e)$.
		\item Kirchhoff: Für alle $x$, die nmicht an einer Spannungsquelle liegen, gilt $(d^*i)(x)= 0$.
	\end{itemize}
\end{uebung}
Wir wollen nun Flüsse auf Netzwerken studieren. Ein Fluss $\Theta$ auf $G = (V,E,c)$ sei ein Element auf $l^2_-(E)$. Ein Fluss $\Theta$ heißt Fluss von $A$ nach $Z$, $A,Z \subseteq V, A \cap Z = \setzero$, falls $g'H (d^*\Theta)(a) > 0 \forall a\in A , (d^*\Theta)(z)<0 \forall z \in Z$. Für alle $x \notin A\cup Z$ gilt $(d^*\Theta)(x) = 0$. Die \emph{Stärke} von $\Theta$ sei 
\begin{gather}
	S(\Theta) :=\sum\limits_{a \in A}(d^*\Theta)(a)
\end{gather}
$A$ heißt \emph{Quelle} und $Z$ heißt \emph{Senke}. Es gibt quellefreie  Flüsse. Vernünftige Flüsse wollen nichts verlieren:
\begin{lemma}
	Sei $G$ endlich, $A,Z \subseteq V, A \cap Z = \setzero$. Dann gilt $S(\Theta) = -\sum\limits_{z \in Z}(d^*\Theta)(z)$ 
\end{lemma}\todo{korrekt?} 
\begin{beweis}
	\begin{gather}
		\sum\limits_{a \in A} (d^*\Theta)(a) + \sum\limits_{z \in Z} (d^*\Theta)(z) = \sum\limits_{x \in V}(d^*\Theta)(x) \cdot 1 = <d^*\Theta,\mathds{1}> = <\Theta,d \mathds{1}> = 0.
	\end{gather}
\end{beweis}

\begin{lemma}
	Sei $G$ endlich, $A,Z \subseteq V, A\cap Z = \setzero$ und $\Theta$ ein Fluss von $A$ nahc $Z$. Sei $f \in l^2(V)$ mit $f|_A \equiv \alpha, f|_Z = \zeta$. Dann gilt
	\begin{gather}
		<\Theta,df> = (\alpha - \zeta) S(\Theta).
	\end{gather}
\end{lemma}
\begin{beweis}
	\begin{align}
		<\Theta,df> &= <d^*\Theta,f> = \sum\limits_{x \in V} (d*\Theta)(x)f(x) \\
			&= \alpha\sum\limits_{x\in A} (d^* \Theta)(x) + \sum\limits_{x \notin A \cup Z} (d^*\Theta)(x)f(x) + \zeta\sum\limits_{x\in Z} (d^*\Theta)(x) \\
			&= \alpha S(\Theta) + 0 - \zeta S(\Theta)  = (\alpha - \zeta) S(\Theta)
	\end{align}
\end{beweis}
Idee aus der Physik: An einem Widerstand der Stärke $v$ wird eine Arbeit der Stärke $s^2v$ geleistet. Wir übertragen das auf Flüsse auf Netzwerken:
\begin{definition}
	Für einen beliebigen Vektrorraum $H$ mit Skalarprodukt $<\cdot,\cdot>$ definiere für $f,g,h \in H$
	\begin{align}
		<f,g>_h = <fh,g> = <f,gh> && ||f||^2_v = <f,f>.
	\end{align}
\end{definition}
Für $\Theta \in l^2_-(E)$ sei $\mathcal{E}(\Theta) = ||\Theta||^2_v$ die Energie von $\Theta$. Insbesondere ist für einen Stromfluss $i$. $\mathcal{E}(i) = ||i||^2_v = <i,i\cdot v>  =<i, dv>$ nach Ohm.

\begin{uebung}
	Für $A\cap Z = ??$ definiere $C(A \leftrightarrow Z)$ als $C(a \leftrightarrow Z)$, wenn man alkke $x \in A$ zu $a$ identifiziert nun entsprechend $R(A \leftrightarrow Z) = \frac{1}{C(A \leftrightarrow Z)}$. Man zeige, dass dann für \todo{const?}
	\begin{align} 
		v|_A \equiv const = v_A && v|_Z \equiv const = v_Z \\ 
	\end{align}
	\begin{align}
		v_A - v_Z = \mathcal{I}_{AZ}R(A \leftrightarrow Z), \text{wobei } \mathcal{I}_{AZ} = \sum\limits_{??}\sum\limits_{??}??
	\end{align}
	
	
	Damit lässt sich die Energie für einen Einheitsraum mit 
\end{uebung}
-------------------------------------------------------------------------------------------

Sei $X^l = \mathds{1}_{\set{e}} - \mathds{1}_{\set{-e}}$. Dann gilt für alle $\Theta \in l^2_-(E):$
\begin{gather}
	<\Theta,\chi^e>_r = \Theta(e)r(e)
\end{gather}
Somit 
\begin{gather}
	<\sum\limits_{e \cdot e^ = x} c(e)\chi^e,\Theta> = \sum\limits_{e \cdot e^-}<c(e\chi^e,\Theta(e))> = \sum\limits_{e^-=x} \Theta(e) = (d^*\Theta)(x).
\end{gather}
Wenn an $x$ keine Spannung anliegt, sagt Kirchhoff
\begin{gather}
	<\sum\limits_{e \cdot e^- = x} c(e)\chi^e,\Theta> = 0
\end{gather}
Die kirchhoffsche Maschenregel lässt sich schreiben als 
\begin{gather}
	<\sum\limits_{k=1}^{n}\chi^{e_k},i> = 0 \text{für einen gerichteten Kreis } e_1, \dots, r_n
\end{gather}
Definiere
\begin{align}
	\star &= SPAN \set{\Theta \in l^2_- \in E \given \Theta = \sum\limits_{e \cdot e^- = x}c(e)\chi^e} \\
	\mathcal{O} &= SPAN \set{\Theta \in l^2_-(E) \given \Theta = \sum\limits_{k=1}^{e_n}\chi^{e_k}} \text{für einen gerichteten Kreis} e_1, \dots e_n
\end{align}

\begin{uebung}
	STERN und O sund bezüglich $<\cdot,\cdot>_r$ auf $E$ orthogonal
	\begin{gather}
		STERN DIREKTE SUMME O = l^2_-(E)
	\end{gather}
\end{uebung}
Es folgt aus der orthogonalen Zerlegung von $\Theta$ in $i$ und $\Theta - i$, dass $||\Theta||^2_r = ||i||^2_r$. Dies ergibt sofort
\begin{satz}[Thompsonsches Prinzip]
	Sei $G$ endlich, $A,Z \subseteq V, A \cap Z = \setzero.$ Sei $\Theta$ ein Fluss von $A$ nach $Z$ und $i$ in der Stromfluss von $A$ nach $Z$, mit $(d^*\Theta) = (d^2i)$. Dann gilt $\mathcal{E}(\Theta) \geq \mathcal(i)$
\end{satz}
\begin{beweis}
	Folgt, da $\mathcal{E}(\Theta) = ||\Theta||^2_r$
\end{beweis}

\begin{satz}[Bel... Prinzip]
	bla
\end{satz}
\begin{beweis}
	Es gilt
	\begin{gather}
		\mathcal{E}(i_c) =  R_c(A \leftrightarrow Z) = \frac{1}{\mathcal{C}(A \leftrightarrow Z)}
	\end{gather}
	und 
	\begin{gather}
		\mathcal{E}_c(i_c) = ||i_c||^2_r = ||i_c||^2_c =\geq ||i_c||^2_{1/c'} \geq \mathcal{E}_{c'}(i_c) \geq \mathcal{E}_{c'}(i_{c'}).
	\end{gather}
	Kehrwert Bilden ergibt $\mathcal{_c}(A \leftrightarrow) \leq \mathcal{C}_{c'} (A \leftrightarrow Z)$
\end{beweis}


\section{Transienz \& Rekkurenz}

Wir haben gesehen: Begriffe wie \enquote{effektive Kapazität}, \enquote{effektiver Widerstand}, \enquote{Energie}, \enquote{Spannung} helfen, um das Verhalten von Irrfahrten auf Graphen besser zu verstehen. Zum Beispiel entsprechen Spannungen Treffwahrscheinlichkeitenm effektive Widerstände bzw. Kapazitäten entsprechen Durchgangswahrscheinlichkeiten, Energie ist so etwas wie effektive Widerstände.

Bislang haben wir einiges erst auf endlichen Graphen definiert, dieses Setup ist für Fragen der Rekkurenz und Transienz jedoch zu klein. Sei nun $G$ unendlich. Wir definieren wieder

\begin{align}
	l^2(V) = \set{f_V \to \Real: \sum\limits_{v\in V} f^2(v)<\infty } && \text{mit Skalarprodukt } <f,g> = \sum\limits_{v} f(v)g(v) \text{und } ||f||^2 = <f,f>
\end{align}
und
\begin{gather}
	l^2_-(E) = \set{\Theta:F \to \Real: \Theta(-e) = -\Theta(e), \sum \Theta^2(e)r(e) < \infty} \\
	 \text{mit Skalarprodukt } <\Theta, \Theta'>_r = \sum\limits_{e \in E_{1/2}} \Theta(e)\Theta'(e)r(e) \text{ ,und Energie } \mathcal{E}(\Theta) = ||\Theta||^2_r
\end{gather}

Man kann dann den Co-Randoperator $(df)(e)= d(e^-) - f(e^+)$ definieren, den Randoperator $(d^*\Theta)(x) = \sum\limits_{x = c^-}\Theta(e)$ und man erhält wieder deren Dualität. Man kann sich vergewissern, dass sich so das gesamte Kapiel 1.4 \todo{Referenz} af die unendliche Situation verallgemeinern lässt. Wir wollen nun Flüsse aus einem $a \in V$ nach undendlich betrachten, d.h. $\Theta$ ist von der Form $d^*\Theta(x) := \alpha EINSFKT_{a}(x)$. 

Wir wollen die Frage der Transienz in Beziehung zur Existenz solcher Flüsse setzen.
\todo[inline]{Missing part}

\begin{satz}[Lyans]
	Wenn $G$ ein abzählbares, zusammenhängendes Netzwerk ist, dann ist die Irrfahrt uf $G$ genau dann transient, wenn es einen Fluss endlicher Energie von $a \in V$ nach $\infty$ gibt.
\end{satz}
\begin{beweis}
	Sei $G_n  \subseteq G$ endlich, $G_n \uparrow G$. Die Punkte in $V \backslash V_n$ kleben wir wieder summen zu $z_n$, behalten Doppelkanten und werfen Schlaufen weg. Das resultierende Nezuwerk sei $G^w_n$ o.E. $a \in G_n$. Es gilt $R(a \leftrightarrow \infty) = \lim R(a \leftrightarrow z_n)$.
	\todo[inline]{Layout}
	Nun ist einerseits $R(a \leftrightarrow z_n) = \frac{1}{\weight{a \leftrightarrow z_n}} \sim \frac{1}{\prop{\tau^+_a < \tau_{z_n}}}$, d.h. 
	\begin{equation}
		lim R(a \leftrightarrow z_n) < \infty \Leftrightarrow \lim\limits_{n \to \infty} \prop{a \to z_n} < 0. \marginnote{Transienz}
	\end{equation}
	Auf der anderen Seite ist $\lim R(a \leftrightarrow z_n) = \lim \mathcal{E}(in)$ für einen Einheitsstrom von $a$ nach $z_n$. Das heißt Transienz der Irrfahrt ist äquivalent zur Endlichkeit von $\lim \mathcal{E}(in)$. Sei nun $\Theta$ irgendein Einheitsfliuss endlicher Energie von $a$ nach $\infty$. Sei $in$ der Einheitsstrom-fluss von $a$ nach $z_n$, $\Theta_n = \Theta|_{G^W_n}$. Dann gilt nach Thompson \todo{verweis}
	\begin{align}
		\mathcal{E}(i_n) \leq \mathcal{E}(\Theta_n) = \mathcal{E}(\Theta) < \infty \text{für alle } n
	\end{align}
	zu $Rightarrow:$ $\lim \mathcal{E}(i_n) < \infty \Rightarrow$ Die Irrfahrt ist transient. \\
	
	Sei nun die Irrfahrt transient. Wir wollen nun einen Fluss von $a$ nach $\infty$ endlicher Energie basteln, Kandidat: $\lim i_n.$ Nach der Vorüberlegung gilt;
	\begin{align}
		\exists N > 0: \lim \mathcal{E}(i_n) \leq N < \infty, (\mathcal{E}(i_n)\leq N \forall n)
	\end{align}
	Sei nun 
	\begin{align}
		Y_n(x) &= \text{Anzahl der Besuche der Irrfahrt bei Start in $a$ in $x$ vor Verlassen von } G_n \\
		Y(x) &= \text{Anzahl der Besuche der Irrfahrt bei Start in $a$ in $x$} 
	\end{align}
	Dann gilt mit der monotonen Konvergenz:
	\begin{align}
		\EWE{Y(x)} = \EWE{\lim Y_n(x)} = \lim\limits_{n \to \infty} \EWE{Y_n(x)} = \lim\limits_{n \to \infty} \pi(x) v_n(x) = \pi(x)v(x) \marginnote{$v_n(x)$ Spannung von $x$ in $G_n$}
	\end{align}
	Da die Irrfahrt transient ist, ist die linke Seite endlich, also existiert $\lim v_n(x) = v(x)$. Also ist
	\begin{align}
		\Gamma := c (dv) = \lim\limits_{n\to\infty} c (dv_n) = \lim\limits_{n\to\infty} i_n
	\end{align}
	der Strom auf $G$.

	Die ist ein Stromfluss endlicher Energie nach der folgenden Übung.
\end{beweis}

\begin{uebung}
	$G$ abzählbar, $(\Theta_n)_n \in l^2_e(E): \mathcal{E} \leq M < \infty$ für ein $M$ und $\Theta_n(e) \to \Theta(e)$. Dann ist $\Theta$ antisymmetrisch und $\mathcal{E}(\Theta) \leq \liminf \mathcal{E}(\Theta_n), \forall x: (d^*\Theta_n)(x) \to (d^*\Theta)(x)$
	\todo[inline]{missing parts}
\end{uebung}

Allgemeiner gilt (ohne Beweis)

	\begin{satz}
		Sei $G$ transient und zusammenhängend, $G_n \uparrow G$, $G_n$ endlich. $G_n^W$ kostruiert wie immer und $i_n$ der Einheitsstrom in $G^W_n$ von einem $a$ nach $z_n$ Dann:
		
		$i_n$ konvergiert punkt-(d.h. Kanten-) weise gegen ein $i$, der Einheitsstromfluss von $a$ nach $\infty$.

	Sind $v_n$ die zugehörigen Spannungen in $G^W_n$ mit $v(z_n) = 0,$ gilt $v  = \lim v_n$ existiert.

	Es gilt:
	\begin{align}
		dv = ir && v(0) = \mathcal{E} = R(a \leftrightarrow \infty) && \forall x: \frac{v(x)}{v(0)} = \prop{\tau_a < \infty}
	\end{align}
	Weiter sei $\mathcal{G}(a,x) := \pi(x)v(x) = \EW{\text{Treffer von }x}.$ Schließlich 
	\begin{equation}
		i(l) = \EW{\text{Anzahl signierte Überquerungen von}e} 
	\end{equation}

\end{satz}
Wie findet man nun heraus ob es einen Fluss endlicher Energie von einem $a\in V$ nach $\infty$ gibt? Für die erste Methode benötigen wir die folgende Definition
\begin{definition}
	$\pi \subseteq E$ heißt \emph{Cut-Set} zwischen $a$ und $\infty$, wenn jeder Pfad $\gamma$ von $a$ nach $\infty$ die Ungleichheit $\gamma \cap \pi \neq \setzero$ erfüllt.
\end{definition}

\begin{satz}[Nash-Williams]
	Sei $(\pi_n)$ eine Folge paarweise disjunkter Cut-Sets, die $a$ von $\infty$ trennt. Dann gilt 
	\begin{align}
		R(a \leftrightarrow \infty) \geq \sum\limits_{n = 1}^{\infty}\enb{\sum\limits_{e \in \pi_n} c(e)}^{-1}
	\end{align}
\end{satz}

\begin{bemerkung}
	\begin{enumerate}[a)]
		\item Ist für eine Folge von $\Pi_n$ wie Satz 1.10\todo{Ref} die rechte Seite unendlich, so ist die Irrfahrt rekurrent
		\item Intuitiv sollte die Aussage des Satzes \todo{Ref} 1.10 klar sein, denn jedes $Pi_n$ hat einen Widerstand $ \geq \enb{\sum\limits_{l\in \Pi_n} c(e)}^{-1}$
	\end{enumerate}
\end{bemerkung}
\begin{beweis}
	$\Theta$ sei ein Einheitsfluss von $a$ nach $\infty$. Es gilt:
	\begin{align}
		\sum\limits_{e \in \Pi_n} \Theta^2(e) r(e) \sum\limits_{e \in \Pi_n} c(e) \geq \enb{\sum\limits_{e \in \Pi_n}\abs{\Theta(e)} \sqrt{r(e)}\sqrt{c(e)}}^2 = \enb{\sum\limits_{e \in \Pi_n}\abs{\Theta(e)}}^2 \geq \enb{\sum\limits_{e \in F_n}\abs{\Theta(e)} }^2
	\end{align}
	wobei $k_n$ (kommt noch) die Menge aller Kanten ist, die \underline{nicht} von $a$ getrennt werden durch $\Pi_n$, und $F_n$ die Menge aller Kanten $c = (e^-,e^+)$ mit $e^- \in k_n$, aber $e^+ \notin k_n$. Es ist klar, dass $F_n \leq \Pi_n$.
	
	Nun ist 
	\begin{gather}
		\sum\limits_{r \in F_n} \abs{\Theta_i(e)} \geq \sum\limits_{e \in F_n} \Theta(e) = \sum\limits_{x \in k_n} (d^*\Theta)(x) = (d^*\Theta)(a) = 1 \\
		\Rightarrow \sum\limits_{e \in \Pi_n} \Theta^2(e)r(e) \geq \frac{1}{\sum\limits_{c \in \Pi_n}c(e)}.
	\end{gather}
	Also 
	\begin{align}
		R (a \leftrightarrow \infty) = \mathcal{E}(i) \geq \sum\limits_{n \geq 1}\sum\limits_{e\in \Pi_n} \Theta^2(e)r(e) \geq \sum\limits_{n \geq 1} \enb{\sum\limits_{e \in\Pi} c(e)}
	\end{align}
	
	
\end{beweis}

Für einen Transienz--Beweis hilft Nash-Williams zunächst nicht. Hier versuchen wir die zufällige Pfadwahl. Dazu sei $\p$ eine Wahrscheinlichkeit auf der Menge der Pfade von $a$ nach $\infty$. So ein Pfad $(e_n)$ entspricht einem Einheitsfluss
\begin{gather}
	\gamma^{(e_n)} (e) := \sum\limits_{n \geq 1} \chi^{e_n} = \sum\limits_{n \geq 1} \mathds{1}_{\set{e_n}}- \mathds{1}_{\set{-e_n}}
\end{gather}
Da alle $(e_n)$ in $a$ starten, ist die Erwartung dieses Einheitsfluss von $a \to \infty$  
\begin{align}
	\Theta = \EW{\gamma} = \sum\limits_{n \geq 1}\prop{e_n} - \prop{-e_n}
\end{align}
Von einem solchen $\Theta$ möchte man die Energie beschränken. 

\marginnote{Beginn Vorlesung vom 28.04}
Wählt man $\p$ so, dass Kanten hoher Leitfähigkeit oft benutzt, und Kanten niedriger Leitfähigkeit wenig, kann man hoffen einen Fluss endlicher Energie von $a \to \infty$ zu konstruieren. Ein Beispiel ist der folgende Satz

\begin{satz}[Polyò]
	Die Irrfahrt auf $\ZZ^d$ ist rekurrent für $d\leq 2$ und transient für $d \geq 3$.
\end{satz}
\begin{beweis}
	Für $d=1,2$ verwenden wir das Kriterium von Nash-Williams. \todo{ref} \todo[inline]{missing figure}
	Sind das die $\Pi_n$, so gilt 
	\begin{gather}
		(a \leftrightarrow \infty) \leq \sum\limits_{n \geq 1} \enb{\sum c(e) }^{-1} = \sum\limits_{n \geq 1 } \frac{1}{2} = \infty
	\end{gather}
	
	Sei $n = 2$, dann
	
	\begin{gather}
		\Pi_n = \set{e = (x,y) \given x \in Q_n, y \in Q_{n+1}}
	\end{gather}
	$Q_n$ sei der Quader der Kantenlänge $n$ \todo[inline]{missing figure}
	$\abs{\Pi_n} = 8n + 4$. Damit erhält man 
	\begin{gather}
		R(a\leftrightarrow \infty) \geq \sum\limits_{n \geq 1} \enb{\sum\limits_{c \in \Pi} 1}^{-1} = \sum\limits_{n \geq 1} \frac{1}{8n + 4} = \infty
	\end{gather}
	Nach Rayleigh \todo{ref} genügt es für die Transienz den Fall $d=3$ zu betrachten. 
	
	\underline{Idee:} Wähle alle Pfade, die \enquote{direkt} von $0$ nach $\infty$ laufen \enquote{mit gleicher Wahrscheinlichkeit}.
	
	Wir wählen einen Punkt $s \in S^2$ nach dem Haarschen Maß. Konstruiere eine Gerade durch $\overrightarrow{0s}$. Wähle zu jeden $\overrightarrow{0s}$ einen Pfad in $\ZZ^3$, der nach $\infty$ läuft und \enquote{direkt} an $\overrightarrow{0s}$ liegt (Es gibt einen solchen Pfad, der Abstand $\leq 4$ hat). \todo{figuuuuure}
	Das gibt eine Menge von Pfaden $\set{(e_n)}$ und hierauf habe ich ein Wahrscheinlichkeit die durch das Haar--Maß induziert wird. Wir wollen für das wie in $(*)$ \todo{ref} konstruierte $\Theta$ $\mathcal{E}(\Theta)$ berechnen.
	\begin{align}
		\mathcal{E}(\Theta) &= \sum\limits_{e} \Theta^2(e) r(e) = \sum\limits_{e}\Theta^2(e) \leq \sum\limits_{e} \prop{e_n=e}[2] \\
							&= \sum\limits_{n=1}^{\infty} \sum\limits_{e\colon d(0,e^+) = n}\prop{e_n = e}[2] \\
							&= \sum\limits_{n = 1}^{\infty}Cn^2\frac{c'}{n^4} = const \sum\limits_{n \geq} \frac{1}{n^2} < \infty \marginnote{Es gibt $Cn^2$ viele Kanten mit $d(0,e^+) = n$} 
	\end{align}
	
	Die Wahrscheinlichkeit eine Kante zu benutzen ist nach oben beschränkt durch $\frac{c'}{n^2}$, weil durch die Wahl von $s$ alle kanten im Bastand \enquote{ungefähr gleich} häufig getroffen werden.
	
\end{beweis}

\section{Spezielle Netzwerke}

\subsection{Flüsse, Cutsets und Ford........}
\todo{subsections? Titel?}
Gegeben sei irgendein Netzwerk $G = (V,R,c).$ Sei $a,(z) \in V$ oder $z = \infty$. Gibt es einen Fluss $\Theta$ von $a$ nach $z$, dergestalt
\begin{align}
	\abs{\Theta(e)} \leq c(e) && \text{für alle } e
\end{align}
Zum Beispiel gilt für den Einheitsstromfluss von $a$ ncah $z$ mit Spannungen $v$, dass 
\begin{align}
	\abs{i(e)} = \abs{c(e) dv(e)} \leq c(e)v(a)
\end{align}
Somit ist $\frac{i}{v(a)}$ ein Fluss von $a$ nach $z$, mit $\frac{\abs{i(e)}}{v(a)} \leq c(e)$. \todo{v(a) oder v(0)??} Solche Flüsse wollen wir zulässig nennen.
\todo[inline]{missing figure. böum}
Die erste Frage ist: Was ist die maximale Flussstärke eines zulässigen Flusses von $A$ nach $Z$ mit $A \cap Z = \setzero$?
\todo[inline]{uuuuuund noch eine Zeichnung}
Es ist intuitiv zu erwarten, dass ein Fluss nicht stärker sein kann, als die Summe der Leitfähigkeiten der Gewichte in einem Cutset. \marginnote{\enquote{Katze!}} Das erstaunliche ist, dass man auf diese Weise tatsächlich die maximale Flussstärke ermitteln kann. Es gilt

\begin{satz}[Ford Fulkersen, Max--Float--Min--Cut--Theorem]
	Sei $G = (V,E,c)$ ein ungerichteter, endlicher Graph. Sei $A,Z \subset V, A \cap Z = \setzero$. Dann gilt für zulässige Flüsse $\Theta$ von $A$ nach $Z$
	\begin{align}
		\max\limits_{\Theta \text{ zulässig}} S(\Theta) = \min \set{\sum\limits_{e \in \Pi}c(e) \given \Pi \text{ist ein $A$--$Z$ Cut--Set}}
	\end{align}
\end{satz}
Wir beweisen allgemeiner eine Version für gerichtete Graphen
\begin{beweis}
	Dazu sei $G = (V,E,c), E \subseteq V \times V$. Dabei ist $c(e) = c(-e)$ nicht mehr notwendigerweise erfüllt. Sei
	\begin{align}
		\Phi (x,e) = \mathds{1}_{\set{x = e^-}}- \mathds{1}_{\set{x = e^+}}
	\end{align}
	die Kanten--Knoten--Inzidenz-Funktion.

\end{beweis}
\todo{blabla weil bweweis gferichteter netze?}
\begin{definition}
	$\Theta\colon E \to \Real^+$ heißt Fluss von $A$ nach $Z$ mit $A \cap Z = \setzero$, wenn gilt
	\begin{align}
		x \in A &\Rightarrow \sum\limits_{e} \Phi(x,e) \Theta(e) > 0 \\
		x \in Z &\Rightarrow \sum\limits_{e} \Phi(x,e) \Theta(e) < 0 \\
		x \notin A \cup Z &\Rightarrow \sum\limits_{e} \Phi(x,e) \Theta(e) = 0
	\end{align}
	$A$ heißt \emph{Quelle}, $Z$ heißt \emph{Senke} von $\Theta$. $\Theta$ heißt \emph{zulässig}, wenn $\Theta(e) \leq c(e)$ für alle $e \in E$. Die \emph{Flussstärke} $S$ ist definiert durch
	\begin{align}
		S(\Theta):= \sum\limits_{x \in A} \sum\limits_{e \in E} \Phi(x,e) \Theta(e)
	\end{align}
\end{definition}
	
\begin{satz}
	$G = (V,E,c)$ gerichtetes, endliches Netzwerk. Dann gilt für zulässiger Flüsse $\Theta$ von $A$ nach $Z$ mit $\abs{\Theta(e)}  \leq c(e), \forall e$:
	\begin{align}
		\max\limits_{\Theta \text{ zulässig}}\set{S(\Theta)}  = \min\set{\sum\limits_{e \in \Pi}c(e) \given \Pi \text{ ist ein $A$--$Z$ Cutset}}
	\end{align}
\end{satz}
\begin{beweis}
	Das Maximum existiert und wird angenommen. Die Menge aller Flüsse $T := \set{\Theta \colon E \to \RR^+ \given \Theta \text{ ist zulässiger $A$--$Z$--Fluss}}\subseteq \RR_+^{\abs{E}}$. Aus der Zulässigkeitsbedingung folgt $\Theta(e) \leq c(e), \forall e$. Damit ist die Menge aller zulässigen Flüsse abgeschlossen und beschränkt in $\Real^{\abs{E}}$, also kompakt. $S$ ist stetig in $\Theta$, somit wird das Maximum sowie das Minimum auf $T$ angenommen. 
	
	Sei $\Theta$ maximierend, sei $\Pi$ ein $A$--$Z$--Cutset. Dann ist $A'$ die Menge der Vertizes, die nicht durch \todo{malen malen malen} $\Pi$ von $A$ getrennt werden. Insbesondere ist $A \subseteq A'$. Dann gilt
	\begin{align}
		S(\Theta) = \sum\limits_{x \in A}\sum\limits_{c \in E} \Phi(x,e) \Theta(e) = \sum\limits_{x \in A'} \sum\limits_{c \in E} \Phi(x,e) \Theta(e)
	\end{align}
	da $A' \supseteq A$ und $\sum\limits_{e} \Phi (x,e) \Theta(e) = 0$ für $x \in A' \backslash A.$
	\begin{align}
		\dots = \sum\limits_{c \in E}\Theta(e)\sum\limits_{x\in A'}\Phi(x,e) \leq \sum\limits_{e \in \Pi} \Theta(e) \leq \sum\limits_{e \in \Pi} c(e)
	\end{align}
	Nun ist
	\begin{align}
		\sum\limits_{x \in A'} = 
			\begin{cases}
			0, & \text{wenn } e = (a',a'') \text{ mit } a',a'' \in A \text{ oder } a',a'' \notin A' \\
			1, & \text{wenn } c = (a',y), a' \in A', y\notin A'\\
			-1, & \text{wenn } e= (y,a'), a' \in A', y \notin A'
			\end{cases}
	\end{align}
	Für die andere Richtung definieren wir für $\Theta$, dass $x_0,x_1, \dots , x_k$ ein vergrößerbarer Pfad ist, wenn $x_0 \in A$ und für alle $i = 1,\dots,k$ gilt, dass entweder
	\begin{align}
		e = (x_{i-1}x_i) \in E \text{ und } \Theta(e) < c(e) \\
		e' = (x_i,x_{i-1}) \in E \text{ und } \Theta'(e) > 0.
	\end{align}
	
	Sei $B= \set{y \in V \given \exists \text{ vergrößerbarer Pfad von einem $x \in A$ nach $y$}}$, mit Nebenbedingung $A\subseteq B$.
	
	\underline{Behauptung:} $B \cap Z = \setzero$. Angenommen $z \in B \cap Z$. Dann gibt es $\epsilon > 0$
	\begin{align}
		\epsilon = \min\Big\{\min\set{c(e) -\Theta(e), e = (x_{i-1},x_i)}, \min\set{\Theta(e'), e' = (x_i,x_{i-1})}\Big\}
	\end{align}
	
	Der Fluss \begin{align}
		\hat{\Theta(e)} = 
			\begin{cases}
				\Theta(e) + \epsilon, & \text{für } e = (x_{i-1},x_i)\\
 				\Theta(e) - \epsilon, & \text{für } e = (x_i,x_{i-1})
			\end{cases}
	\end{align}
	vergrößert dann den Fluss $\Theta(e)$ entlang des Pfades $x_0, \dots, x_e$. Das ist im Widerspruch zur Maximalität von $\Theta$. Also ist $B \subseteq Z^c$. Definiere
	\begin{align}
		\Pi = B \times B^c%\set{(b,b') \given b \in B, b' \in B^c \lor b \in B^c, b' \in B}
	\end{align} 
	Dann gilt $\Theta(e) = c(e)$ für $e \in B \times B^c$ und $\Theta(e) = 0$ für ein $e \in B^c \times B$. Mit der gleichen Rechnung wie eben erhalten wir 
	\begin{align} 
		S(\Theta) &= \sum\limits_{x\in A} \sum\limits_{e \in E} \Phi(x,e) \Theta(e)\\ \marginnote{(*) \text{mit $A'$ wie eben}}
				&\overset{(*)}{=} \sum\limits_{x \in A'} \sum\limits_{e \in E} \Phi(x,e) \Theta(e) \\
				&= \sum\limits_{e \in Pi}\Theta(e) = \sum\limits_{e \in \Pi}c(e) 
	\end{align}
\end{beweis}
Nun die Verallgemeinerung auf abzählbare Netzwerke. Dazu sei $G= (V,E,c)$ ein abzählbares Netzwerk, d.h. $V,E$ abzählbar. Weiter sei für alle $x$ 
\begin{align}
	\sum\limits_{e ^- = x} c(e) < \infty
\end{align}

dann gilt die folgende Verallgemeinerung von \todo{ref} vorherigem Satz. 

\begin{satz}
	Sei $G$ ein zusammenhängendes, abzählbares Netzwerk wie oben. Dann gilt für $a$ (die Quelle)
	\begin{align}
		\max\set{S(\Theta) \given \Theta \text{ ist ein zulässiger Fluss von $a$ nach $\infty$}} \\
		=\inf\set{\sum\limits_{e\in\Pi}c(e) \given \Pi \text{ ist ein $a$--$\infty$ -- Cutset} }.
	\end{align}
\end{satz}

\begin{bemerkung}
	Tatsächlich ist die linke Seite ein Maximum, nicht nur ein Supremum. Denn ist $(\Theta_n)$ eine maximierende Folge auf endlichen $G_n \uparrow G$. Dann konvergiert $\Theta_n$ kantenweise und der Limes ist ein zulässiger $a \to \infty$ Fluss auf $G$.
\end{bemerkung}

Im ersten Schritt reduzieren wir unser Netzwerk, d.h. es gibt $\forall \epsilon > 0$ eine Menge an Kanten $D$ sodass $(V,E\backslash D,c)$ lokal--endlich ist \marginnote{lokal--endlich: Nur endlich viele Nachbarn an jedem Knoten} und $\sum\limits_{c \in D}c(e) < \epsilon$, denn sei $x_1,x_2, \dots$ eine Aufzählung von $V$, dann gibt es in jedem $x$ eine endliche Menge von Kanten $\mathcal{E}_{x_i} = \set{e,e^- = x_i}$ und 
\begin{align}
	\sum\limits_{e^- = x_i, e \notin \mathcal{E}_{x_i}} c(e) < \epsilon \cdot 2^{-i},
\end{align}
wobei wir nur die Kanten behalten welche in einem $\mathcal{E}_{x_i}$ liegen.

Ein Pfad heißt einfach, wenn er keinen Vertex mehrfach seit.
Auf $P$ kann man eine Metrik definieren:

\begin{align}
	d(\gamma,\gamma') = \inf\set{\frac{1}{n+1} \given \text{$\gamma$ und $\gamma'$ stimmen in den ersten n Schritten überein}}
\end{align}	
\begin{uebung}
	Das ist eine Metrik.
\end{uebung}
Bezüglich $d$ ist $P$ kompakt. Zeige: $p$ ist folgen--kompakt.

Sei $\Big(\enb{e^m_n}_n \Big)^m$ eine Folge von Pfaden in $P$. Alle starten in $a$ in $G'$ ist lokal--endlich. Es gibt also eine Teilfolge $\Big(\enb{e^{m_1}_n}_n \Big)^{m_1}$ sodass alle Pfade dieser Teilfolge die gleiche 1. Kante haben. 
Nun gibt es eine Teilfolge $\Big(\enb{e^{m_2}_n}_n \Big)^{m_2}$ von $\Big(\enb{e^{m_1}_n}_n \Big)^{m_1}$, sodass die Pfade dieser Teilfolge in den ersten zwei Kanten übereinstimmen (mit dem gleichen Argument), und so weiter. Ein Diagonalfolgenargument zeigt dann, dass es eine konvergente Teilfolge (in $d$) der gesamten Folge gibt. Also ist $(P,d)$ kompakt. 

Nun sei für $e \in E \backslash D$ 
\begin{align}
	\Gamma_e = \set{\gamma \in P, e \in \gamma}.
\end{align}
Die $\Gamma_e$ sind offen, denn wenn $d_V(a,e^+) = n$, dann gehört mit $\gamma \in \Gamma_e$ auch $\gamma' \in \Gamma_e$, wenn $d_P(\gamma,\gamma') \leq \frac{1}{n+2}$. Ist nun $\Pi$ ein Cutset, dann ist $(\Gamma_e)_{e \in \Pi}$ eine offene Überdeckung von $P$. Also gibt es davon eine endliche Teilüberdekcung $(\Gamma^*_e)$, welche die Gestalt $(\Gamma^*_e)_{e\in\Pi^*}, \Pi^*$ endlich hat. Da $G'$ lokal--endlich ist und $\Pi^*$ endlich, ist die Menge $A' := \set{x \given x \text{ wird von $\Pi^*$ von $\infty$ getrennt}}$ endlich.

\underline{Daher:}
\begin{align}
	S(\Theta) = \sum\limits_{e \in E} \Phi(a,e) \Theta(e)  = \sum\limits_{x \in A'}\sum\limits_{e \in E\backslash D} \Phi(x,e) \Theta(e) \leq \sum\limits_{e \in \Pi} \Theta(e) + \epsilon \leq \sum\limits_{e \in \Pi} c(e) + \epsilon
\end{align}
\todo{letzte zeile ggfls nachtragen}





