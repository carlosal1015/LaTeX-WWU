%!TEX root = ./WTG.tex

\chapter{Spezielle Netzwerke}
\label{chap:SpezielleNetzwerke}
\section{Flüsse, Cutsets und Ford Fulkerson}
Gegeben sei irgendein Netzwerk $G = (V,R,c).$ Sei $a,(z) \in V$ oder $z = \infty$. Gibt es einen Fluss $\Theta$ von $a$ nach $z$, dergestalt
\begin{gather}
	\abs{\Theta(e)} \leq c(e)  \text{ für alle } e?
\end{gather}
Zum Beispiel gilt für den Einheitsstromfluss von $a$ nach $z$ mit Spannungen $v$, dass 
\begin{align}
	\abs{i(e)} = \abs{c(e) dv(e)} \leq c(e)v(a)
\end{align}
Somit ist $\frac{i}{v(a)}$ ein Fluss von $a$ nach $z$, mit $\frac{\abs{i(e)}}{v(a)} \leq c(e)$. Solche Flüsse wollen wir zulässig nennen.
\todo[inline]{missing figure. böum}
Die erste Frage ist: Was ist die maximale Flussstärke eines zulässigen Flusses von $A$ nach $Z$ mit $A \cap Z = \emptyset$?
\todo[inline]{uuuuuund noch eine Zeichnung}
Es ist intuitiv zu erwarten, dass ein Fluss nicht stärker sein kann, als die Summe der Leitfähigkeiten der Gewichte in einem Cutset. \marginnote{\enquote{Katze!}} Das erstaunliche ist, dass man auf diese Weise tatsächlich die maximale Flussstärke ermitteln kann. Es gilt

\begin{satz}[Ford Fulkersen, Max--Flow--Min--Cut--Theorem]
	Sei $G = (V,E,c)$ ein ungerichteter, endlicher Graph. Sei $A,Z \subset V, A \cap Z = \emptyset$. Dann gilt für zulässige Flüsse $\Theta$ von $A$ nach $Z$
	\begin{align}
		\max\limits_{\Theta \text{ zulässig}} S(\Theta) = \min \set{\sum\limits_{e \in \Pi}c(e) \given \Pi \text{ ist ein $A$--$Z$ Cut--Set}}
	\end{align}
\end{satz}
Wir beweisen allgemeiner eine Version für gerichtete Graphen
\begin{beweis}
	Dazu sei $G = (V,E,c), E \subseteq V \times V$. Dabei ist $c(e) = c(-e)$ nicht mehr notwendigerweise erfüllt. Sei
	\begin{align}
		\phi (x,e) = \mathds{1}_{\set{x = e^-}}- \mathds{1}_{\set{x = e^+}}
	\end{align}
	die Kanten-Knoten-Inzidenz-Funktion.

\end{beweis}

\begin{definition}
	$\Theta\colon E \to \RR^+$ heißt Fluss von $A$ nach $Z$ mit $A \cap Z = \emptyset$, wenn gilt
	\begin{align}
		x \in A &\Rightarrow \sum\limits_{e} \phi(x,e) \Theta(e) > 0 \\
		x \in Z &\Rightarrow \sum\limits_{e} \phi(x,e) \Theta(e) < 0 \\
		x \notin A \cup Z &\Rightarrow \sum\limits_{e} \phi(x,e) \Theta(e) = 0
	\end{align}
	$A$ heißt \emph{Quelle}, $Z$ heißt \emph{Senke} von $\Theta$. $\Theta$ heißt \emph{zulässig}, wenn $\Theta(e) \leq c(e)$ für alle $e \in E$. Die \emph{Flussstärke} $S$ ist definiert durch
	\begin{align}
		S(\Theta):= \sum\limits_{x \in A} \sum\limits_{e \in E} \phi(x,e) \Theta(e)
	\end{align}
\end{definition}
	
\begin{satz}
	\label{satz:5-3}
	$G = (V,E,c)$ gerichtetes, endliches Netzwerk. Dann gilt für zulässiger Flüsse $\Theta$ von $A$ nach $Z$ mit $\abs{\Theta(e)}  \leq c(e), \forall e$:
	\begin{align}
		\max\limits_{\Theta \text{ zulässig}}\set{S(\Theta)}  = \min\set{\sum\limits_{e \in \Pi}c(e) \given \Pi \text{ ist ein $A$--$Z$ Cutset}}
	\end{align}
\end{satz}
\begin{beweis}
	Das Maximum existiert und wird angenommen. Die Menge aller Flüsse $T := \set{\Theta \colon E \to \RR^+ \given \Theta \text{ ist zulässiger $A$--$Z$--Fluss}}\subseteq \RR_+^{\abs{E}}$. Aus der Zulässigkeitsbedingung folgt $\Theta(e) \leq c(e), \forall e$. Damit ist die Menge aller zulässigen Flüsse abgeschlossen und beschränkt in $\Real^{\abs{E}}$, also kompakt. $S$ ist stetig in $\Theta$, somit wird das Maximum sowie das Minimum auf $T$ angenommen. 
	
	Sei $\Theta$ maximierend, sei $\Pi$ ein $A$--$Z$--Cutset. Dann ist $A'$ die Menge der Vertizes, die nicht durch \todo{malen malen malen} $\Pi$ von $A$ getrennt werden. Insbesondere ist $A \subseteq A'$. Dann gilt
	\begin{align}
		S(\Theta) = \sum\limits_{x \in A}\sum\limits_{c \in E} \phi(x,e) \Theta(e) = \sum\limits_{x \in A'} \sum\limits_{c \in E} \phi(x,e) \Theta(e)
	\end{align}
	da $A' \supseteq A$ und $\sum\limits_{e} \phi (x,e) \Theta(e) = 0$ für $x \in A' \backslash A.$
	\begin{align}
		\dots = \sum\limits_{c \in E}\Theta(e)\sum\limits_{x\in A'}\phi(x,e) \leq \sum\limits_{e \in \Pi} \Theta(e) \leq \sum\limits_{e \in \Pi} c(e)
	\end{align}
	Nun ist
	\begin{align}
		\sum\limits_{x \in A'} = 
			\begin{cases}
			0, & \text{wenn } e = (a',a'') \text{ mit } a',a'' \in A \text{ oder } a',a'' \notin A' \\
			1, & \text{wenn } c = (a',y), a' \in A', y\notin A'\\
			-1, & \text{wenn } e= (y,a'), a' \in A', y \notin A'
			\end{cases}
	\end{align}
	Für die andere Richtung definieren wir für $\Theta$, dass $x_0,x_1, \dots , x_k$ ein vergrößerbarer Pfad ist, wenn $x_0 \in A$ und für alle $i = 1,\dots,k$ gilt, dass entweder
	\begin{align}
		e = (x_{i-1}x_i) \in E \text{ und } \Theta(e) < c(e) \\
		e' = (x_i,x_{i-1}) \in E \text{ und } \Theta'(e) > 0.
	\end{align}
	
	Sei $B= \set{y \in V \given \exists \text{ vergrößerbarer Pfad von einem $x \in A$ nach $y$}}$, mit Nebenbedingung $A\subseteq B$.
	
	\underline{Behauptung:} $B \cap Z = \emptyset$. Angenommen $z \in B \cap Z$. Dann gibt es $\epsilon > 0$
	\begin{align}
		\epsilon = \min\Big\{\min\set{c(e) -\Theta(e), e = (x_{i-1},x_i)}, \min\set{\Theta(e'), e' = (x_i,x_{i-1})}\Big\}
	\end{align}
	
	Der Fluss \begin{align}
		\hat{\Theta(e)} = 
			\begin{cases}
				\Theta(e) + \epsilon, & \text{für } e = (x_{i-1},x_i)\\
 				\Theta(e) - \epsilon, & \text{für } e = (x_i,x_{i-1})
			\end{cases}
	\end{align}
	vergrößert dann den Fluss $\Theta(e)$ entlang des Pfades $x_0, \dots, x_e$. Das ist im Widerspruch zur Maximalität von $\Theta$. Also ist $B \subseteq Z^c$. Definiere
	\begin{align}
		\Pi = B \times B^c%\set{(b,b') \given b \in B, b' \in B^c \lor b \in B^c, b' \in B}
	\end{align} 
	Dann gilt $\Theta(e) = c(e)$ für $e \in B \times B^c$ und $\Theta(e) = 0$ für ein $e \in B^c \times B$. Mit der gleichen Rechnung wie eben erhalten wir 
	\begin{align} 
		S(\Theta) &= \sum\limits_{x\in A} \sum\limits_{e \in E} \phi(x,e) \Theta(e)\\ \marginnote{(*) \text{mit $A'$ wie eben}}
				&\overset{(*)}{=} \sum\limits_{x \in A'} \sum\limits_{e \in E} \phi(x,e) \Theta(e) \\
				&= \sum\limits_{e \in Pi}\Theta(e) = \sum\limits_{e \in \Pi}c(e) 
	\end{align}
\end{beweis}
Nun die Verallgemeinerung auf abzählbare Netzwerke. Dazu sei $G= (V,E,c)$ ein abzählbares Netzwerk, d.h. $V,E$ abzählbar. Weiter sei für alle $x$ 
\begin{align}
	\sum\limits_{e ^- = x} c(e) < \infty
\end{align}

dann gilt die folgende Verallgemeinerung des vorherigen Satzes.

\begin{satz}
	\label{satz:5-4}
	Sei $G$ ein zusammenhängendes, abzählbares Netzwerk wie oben. Dann gilt für $a$ (die Quelle)
	\begin{align}
		\max\set{S(\Theta) \given \Theta \text{ ist ein zulässiger Fluss von $a$ nach $\infty$}} \\
		=\inf\set{\sum\limits_{e\in\Pi}c(e) \given \Pi \text{ ist ein $a$--$\infty$ -- Cutset} }.
	\end{align}
\end{satz}

\begin{bemerkung}
	Tatsächlich ist die linke Seite ein Maximum, nicht nur ein Supremum. Denn ist $(\Theta_n)$ eine maximierende Folge auf endlichen $G_n \uparrow G$. Dann konvergiert $\Theta_n$ kantenweise und der Limes ist ein zulässiger $a \to \infty$ Fluss auf $G$.
\end{bemerkung}
\begin{beweis}
	
	Im ersten Schritt reduzieren wir unser Netzwerk, d.h. es gibt $\forall \epsilon > 0$ eine Menge an Kanten $D$ sodass $(V,E\backslash D,c)$ lokal--endlich ist \marginnote{lokal--endlich: Nur endlich viele Nachbarn an jedem Knoten} und $\sum\limits_{c \in D}c(e) < \epsilon$, denn sei $x_1,x_2, \dots$ eine Aufzählung von $V$, dann gibt es in jedem $x$ eine endliche Menge von Kanten $\mathcal{E}_{x_i} = \set{e,e^- = x_i}$ und 
	\begin{align}
		\sum\limits_{e^- = x_i, e \notin \mathcal{E}_{x_i}} c(e) < \epsilon \cdot 2^{-i},
	\end{align}
	wobei wir nur die Kanten behalten welche in einem $\mathcal{E}_{x_i}$ liegen.
	
	Sei nun $\mathcal{P}:= \set{\gamma: a \to \infty \given \gamma \text{ ist einfacher Pfad in } G'}$
	(Ein Pfad heißt einfach, wenn er keinen Vertex mehrfach sieht.)
	Auf $\mathcal{P}$ kann man eine Metrik definieren:
	\begin{align}
		d(\gamma,\gamma') = \inf\set{\frac{1}{n+1} \given \text{$\gamma$ und $\gamma'$ stimmen in den ersten n Schritten überein}}
	\end{align}	
	\begin{uebung}
		Das ist eine Metrik.
	\end{uebung}
	Bezüglich $d$ ist $\mathcal{P}$ kompakt. Zeige: $\mathcal{P}$ ist Folgen-kompakt.
	
	Sei $\Big(\enb{e^m_n}_n \Big)^m$ eine Folge von Pfaden in $\mathcal{P}$. Alle starten in $a$ in $G'$ ist lokal--endlich. Es gibt also eine Teilfolge $\Big(\enb{e^{m_1}_n}_n \Big)^{m_1}$ sodass alle Pfade dieser Teilfolge die gleiche 1. Kante haben. 
	Nun gibt es eine Teilfolge $\Big(\enb{e^{m_2}_n}_n \Big)^{m_2}$ von $\Big(\enb{e^{m_1}_n}_n \Big)^{m_1}$, sodass die Pfade dieser Teilfolge in den ersten zwei Kanten übereinstimmen (mit dem gleichen Argument), und so weiter. Ein Diagonalfolgenargument zeigt dann, dass es eine konvergente Teilfolge (in $d$) der gesamten Folge gibt. Also ist $(\mathcal{P},d)$ kompakt. Nun sei für $e \in E \backslash D$ 
	\begin{align}
		\Gamma_e = \set{\gamma \in P, e \in \gamma}.
	\end{align}
	Die $\Gamma_e$ sind offen, denn wenn $d_V(a,e^+) = n$, dann gehört mit $\gamma \in \Gamma_e$ auch $\gamma' \in \Gamma_e$, wenn $d_{\mathcal{P}}(\gamma,\gamma') \leq \frac{1}{n+2}$. Ist nun $\Pi$ ein Cutset, dann ist $(\Gamma_e)_{e \in \Pi}$ eine offene Überdeckung von $\mathcal{P}$. Also gibt es davon eine endliche Teilüberdekcung $(\Gamma^*_e)$, welche die Gestalt $(\Gamma^*_e)_{e\in\Pi^*}, \Pi^*$ endlich hat. Da $G'$ lokal--endlich ist und $\Pi^*$ endlich, ist die Menge $A' := \set{x \given x \text{ wird von $\Pi^*$ von $\infty$ getrennt}}$ endlich.
	
	\underline{Daher:}
	\begin{align}
		S(\Theta) = \sum\limits_{e \in E} \phi(a,e) \Theta(e)  &= \sum\limits_{x \in A'}\sum\limits_{e \in E\backslash D} \phi(x,e) \Theta(e) + \underbrace{\sum\limits_{x \in A'}\sum\limits_{e \in D} \phi(x,e) \Theta(e)}_{<\epsilon} \\ 
			&\leq \sum\limits_{e \in \Pi^*} \Theta(e) + \epsilon \\ 
			&\leq \sum\limits_{e \in \Pi} c(e) + \epsilon \underset{\epsilon \to 0}{\longrightarrow} \sum\limits_{e \in \Pi} c(e)
	\end{align}
	woraus der erste Teil der Behauptung folgt.	\todo{Beweis durchgehen}
	
	\marginnote{Beginn Vorlesung vom 09.05.2016}
	
	\enquote{$\geq$}: Die Idee ist, sich wieder auf den endlichen Fall zu konzentrieren. Dazu sei für ein zusammenhängendes Netzwerk $H$ mit $a \in H$
	\begin{align}
		C(H) = \inf\set{\sum\limits_{e \in \pi}c(e) \given \Pi \text{ ist ein $a$--$\infty$--Cutset in H}}
	\end{align}
	Nun bauen wir aus $G$ wieder ein lokal--endliches Netzwerk $G' = (V',E'c')$ mit $c' = c|_{E'}$, dergestalt, dass $V' = V$ und $E' = E \backslash D$ und $D$ ist eine Menge von Kanten, sodass $G'$ lokal--endlich ist und $\sum\limits_{e \in D} c(e) < \epsilon$. 
	
	1. Beobachtung: Wenn $\Pi'$ ein $a$--$\infty$--Cutset in $G'$ ist, dann ist $\Pi = \Pi' \cup D$ ein $a$--$\infty$--Cutset in $G \Rightarrow C(G) \leq C(G') +  \epsilon$.
	
	Sei nun wieder $(G'_n)_n$ eine aufsteigende Folge endlicher Netzwerke mit $a \in V'_n$ und $G'_n \uparrow G$. Wie üblich verkleben wir die Vertizes aus $V'\backslash V'_n$ zu einem Vertex $z_n$, behalten dabei Doppelkanten und lassen Schlaufen weg. Das resultierende Netzwerk heiße $G^W_n$.
	
	\underline{Behauptung:} \marginnote{Das ist das Kompaktheitsargument der vorherigen Vorlesung} $C(G')  =\inf C(G^W_n)$, denn ist $\Pi$ ein $a$--$\infty$--Cutset in $G'$, dann enthält $\Pi$ einen endlichen $a$--$\infty$--Cutset $\Pi'$. Dann ist aber auch $\sum\limits_{e \in \Pi'} c(e) \leq \sum\limits_{e \in \Pi} c(e)$, d.h. in der Berechnung von $C(G')$ braucht man nur über endliche Cutsets zu gehen. Diese sind aber alle irgendwann in dem $G^W_n$ enthalten. Für die $G_n^W$ weiß man aber nach Satz \ref{satz:5-3}, dass
	\begin{align}
		\max \set{S(\Theta) \given \Theta \text{ ist zulässiger $a$--$z_n$--Fluss in } G_n^W} = \max \set{\sum\limits_{e \in \Pi} c(e) \given \Pi \text{ ist ein $a$--$z_n$--Cutset in} G^W_n}
	\end{align} 
	
	ZU jedem der $G_n^W$ gibt es einen maximierenden Fluss $\Theta_n$. Diese Folge von Flüssen induziert eine Folge von Flüssen $\enb{\Theta_n}_n$ von $a \to \infty$ auf $G'$. Diese hat einen Häufungspunkt $\Theta$ Da $S(\cdot)$ stetig ist in $\Theta$ folgt
	\begin{align}
		S(\Theta) = \lim S(\Theta_n) = \inf C(G^W_n) = C(G') \geq C(G) - \epsilon
	\end{align}

	\end{beweis}
	Diesen Satz wollen wir (später) noch für spezielle Graphen anschauen. 

Zum Abschluss dieses Abschnitts wollen wir noch eine Umkehrung der Konstruktionsmethode aus Abschnitt \ref{chap:TransienzUndRekurrenz} betrachten. Dort haben wir zu Irrfahrten gehörige Flüsse betrachtet. 

Sei nun $\Theta$ ein Fluss auf einem ungerichteten Netzwerk $G$, d.h. $\Theta(e) = - \Theta(-e)$. $\Theta$ fließe von $a$ nach $z$, falls $G$ endlich ist und von $a$ nach $\infty$, falls $G$ unendlich ist. Zu diesem Fluss definieren wir eine Irrfahrt $(Y_n)$ mit $Y_0 = a$, $z$ absorbierend, falls $G$ endlich ist und 
\begin{align}
	\propE{>_n+1 = w \given Y_n = v} = \frac{\max\enb{\Theta(v,w),0}}{\Theta_{out}(v)}
\end{align}
wobei $\Theta_{out}(v) = \sum\limits_{e: e^-=v} \max\enb{\Theta(e),0}$.
Umgekehrt kann man die Methode aus Abschnitt \ref{chap:TransienzUndRekurrenz} verwenden, um aus $(Y_n)$ einen Fluss zu konstruieren
\begin{align}
	\Theta'(e) = \sum\limits_{k \geq 0}\p \Big( (Y_n,Y_{n+1}) = e\Big) - \p\Big((Y_{n+1},Y_n) = e\Big)
\end{align} 
\underline{Frage:} In welchem Zusammenhang stehen $\Theta'$ und $\Theta$?
\begin{definition}
	Ein Fluss heißt \emph{azyklisch}, falls es keinen gerichteten Kreis $x_1, \dots, x_k$ gibt, sodass $\Theta(x_i,x_{x+1}) > 0, \forall i$ \marginnote{$x_{k+1} = x_1$} 
\end{definition}

\begin{beispiel}
	Jeder Stromfluss ist azyklisch. Dies ist eine Übung.
\end{beispiel}

\begin{satz}
	Falls $\Theta$ ein azyklischer Fluss von $a$ nach $z$, bzw. $\infty$ ist, so gilt
	\begin{align}
		0 \leq \Theta'(e) \overset{(*)}{\leq} \Theta(e) \text{\qquad(für alle $\Theta(e) > 0$)}
	\end{align}
	und es gilt sogar Gleichheit bei (*), falls $\Theta$ der Einheitsstromfluss von $a$ nach $\infty$ ist.
\end{satz}
\begin{beweis}
	Falls $\Theta(e) > 0$, so folgt $\propE{(Y_n,Y_{n+1}) = e, \text{ für ein $n$}} \geq 0$, also $\Theta'(e) \geq 0$. 
	
	Sei nun $\Theta'(e) \leq \Theta(e).$ Dazu sei $p_N(e):= \propE{\exists n \leq N: (Y_n, Y_{n+1})  = e}$. $p_N(e)$ ist \enquote{schon fast} $\Theta'(e)$, denn da man keine Kante doppelt überquert ($\Theta'$ ist azyklisch) folgt $p_N(e) \uparrow \Theta(e)$. Wir zeigen $p_N(e) \leq \Theta(e), \forall N$. Für $N = 0$ stimmt das. 
	
	Für $x \in V$ sei $p_N(x) := \propE{\exists n \leq N: Y_n = x}.$ Ist nun bekannt, dass $p_N(e) \leq \Theta(e), \forall e$, so folgt für $x \neq a,z$
	\begin{align}
		p_{N+1}(x) = \sum\limits_{e:e^+=x} \leq \sum\limits_{e:e^+=x} =  \Theta_{out}(x),
	\end{align}
	Für $x=a$ ist $p_N(a) = \Theta_{out}(e), \forall N$. Mit $e^-=x$ folgt dann 
	\begin{align}
		p_{N+1}(e) = p_{N+1}(x) \cdot \frac{\Theta(e)}{\Theta_{out}(x)} \leq \frac{\Theta_{out}(x)}{\Theta_{out}(x)}\Theta(e) = \Theta(e) 
	\end{align}
	Ist $\Theta$ der Einheitsstromfluss von $a$ nch $\infty$, dann minimiert dieses $\mathcal{E}(\overset{\sim}{\Theta}) = \sum\limits_{e} \overset{\sim}{\Theta}^2(e)r(e)$, für alle Einheitsflüsse vom $a$ nach $\infty$, woraus $\overset{\sim}{\Theta} = \Theta$ folgt.
	
	Ist $G$ endlich dann ist auch $\Theta^{''} = \Theta - \Theta'$ ein Fluss. $\Theta^{''}$ ist quellenfrei und azyklisch, da $\Theta(e) \geq \Theta'(e), \forall e$ mit $\Theta(e) > 0$, kann $\Theta''(e) > 0$ nur gelten, wenn $\Theta(e) > 0 \Rightarrow \Theta''$ ist azyklisch.
	
	Nehmen wir an $\Theta''(e) > 0$ für ein $e$. Dann gibt es eine Kante $e_1$ mit $e^+=e_1^-$, sodass $\Theta(e_1) > 0$, sowie eine Kante $e_2$ mit $e_1^+=e_2^-$, sodass $\Theta(e_2) > 0$. Das geht nur endlich lange gut, bis man zum Ausgangspunkt zurückkommt, da $G$ endlich ist. $\lightning$ $\Rightarrow \Theta''$ ist azyklisch.
\end{beweis}

\begin{korollar}[Monotone Spannungspfade]
	Für einen Einheitsstromfluss $i$ auf einem unendlichen transienten Netzwerk von $a$ nach $\infty$ seien $v$ die zugehörigen Spannungen mit $v(\infty) = 0$. Dann gibt es für alle $x \in V$ einen Pfad entlang dessen $v$ monoton ist. Ist $x \in e$ \marginnote{$x \in e \Leftrightarrow x \in \set{e^-,e^+}$} mit $i(e) > 0$, dann ist die Spannung entlang des Pfades strikt monoton. 
\end{korollar}
\begin{beweis}
	Sei $(Y_n)$ der zugehörige Random Walk. Da $i$ azyklisch und ein Einheitsstromfluss ist, werden alle Kanten mit $i(e) > 0$ mit positiver Wahrscheinlichkeit überquert. Also werden alle 
	\begin{align}
		x \in W = \set{x\in V \given x \in e \text{ mit } i(e) > 0}
	\end{align}
	mit positiver Wahrscheinlichkeit getroffen. Damit ist die Spannung entlang des Pfades von $a$ nach $x$ strikt monoton. Die restlichen Punkte kann man mit einem $0$--Strompfad anschließen.
\end{beweis}

\section{Flüsse und Irrfahrten auf Bäumen}
\begin{definition}
	Ein \emph{Baum} ist ein kreisfreier, zusammenhängender Graph. Ein kreisfreier, nicht notwendig zusammenhängender Graph heißt \emph{Wald}. In einem gerichteten Baum sind alle Kanten von einem Knoten weg orientiert, dieser heißt \emph{Wurzel}.
	Es gibt auf Bäumen eine natürliche Graphendistanz $d$. Im gerichteten Fall sei für $e = (e^-,e^+), \abs{e} := d(0,e^+)$. Für $e=(e^-,e^+)$ heißt $e^-$ der \emph{Vorgänger} von $e^+$ und $e^+$ das \emph{Kind} von $e^-$. Knoten ohne Kinder heißen \emph{Blätter}.
\end{definition}\todo{Bilder?} 

Wir wollen auf gerichteten Graphen Flüsse betrachten. Das liegt daran, dass ein Fluss maximale Stärke auf einem Baum $T$ immer eine Variante besitzt, die keine Kante besitzt, wo der Fluss \enquote{entgegen der Baumrichtung} fließt. Auf Bäumen gibt es eine Art \enquote{Umkehrung} ds Nesh--Williams--Kriterium. Hat man eine Folge $(\Pi_n)_n$ disjunkter Cutsets von $0$ nach $\infty$, so gilt 
\begin{align}
	R(0 \leftrightarrow \infty) \geq \sum\limits_{n \geq 1} \enb{\sum\limits_{c \in \Pi_n} c(e)}^{-1}.
\end{align}
Auf Bäumen gilt
\begin{satz}
	Sei $T = (V,E,c)$ ein Netzwerk auf einem lokal--endlichen Baum. Sei $(w_n) \leq 1$ eine Folge nicht--negativer Zahlen mit $\sum\limits_{n=1}^{\infty} w_n < \infty$. Ist $\Theta$ ein Fluss auf $T$ mit 
	\begin{align}
		0 \leq \Theta(e) \leq w_{\abs{e}} c(e)
	\end{align}
	dann hat $\Theta$ endliche Energie.
\end{satz}

\underline{Vorüberlegung:} (Übung) Sei $T$ ein lokal--endlicher Baum (darauf ein Netzwerk gegeben). Sei $\Pi$ bezüglich $\subseteq$ ein minimaler Cutset, der $0$ von $\infty$ trennt. Dann gilt
\begin{align}
	S(\Theta) = \sum\limits_{e \in \Pi} \Theta(e)
\end{align}  

\begin{beweis}
	Es ist
	\begin{align}
		\mathcal{E}(\Theta) &= \sum\limits_{e \in E} \Theta^2(e) r(e) = \sum\limits_{n=1}^{\infty}\sum\limits_{e: \abs{e} = n} \Theta^2(e) r(e) \\
		&=\sum\limits_{n = 1}^{\infty} \sum\limits_{e: \abs{e} = n} \Theta(e)\enb{\Theta(e) \frac{1}{c(e)}} \\
		&\leq \sum\limits_{n = 1}^{\infty} w_n \sum\limits_{e: \abs{e} = n} \Theta(e) = S(\Theta) \sum\limits_{n=1}^{\infty} w_n < \infty
 	\end{align}
\end{beweis}

Wir wollen über Transienz von Irrfahrten auf $T$ reden. Frage: Wie muss man $c(e)$ für Transienz bzw. Rekurrenz wählen? \\
\underline{Motivierendes Beispiel:} Der $d$--reguläre Baum \todo{Bild}

Hier hat man $\abs{\set{e \given \abs{e} = n}} = d^n$. Es liegt also nahe, die Kantengewichte exponentiell fallend zu wählen, wenn man einen Rekurrenz--Transienz--Übergang wählen möchte. Ansatz; $c(e) = \lambda^{-\abs{e}}$. Wir wollen sehen, dass einen einen Phasenübergang in $\lambda$ gibt. Dazu sei $\lambda_c = \lambda_c(T)$ das kritische $\lambda$, sodass für $\lambda < \lambda_c(T)$ ein nicht--Null-Fluss von $0$ nach $\infty$ existiert und für $\lambda > \lambda_c(T)$ nicht. Wir haben schon zu Anfang von Kapitel \ref{chap:SpezielleNetzwerke} gesehen, falls für ein $\lambda$ ein Strom von $0$ nach $\infty$ fließt, dann gibt es einen zulässigen Fluss $\Theta$ von $0$ nach $\infty$. Also ist in diesem Fall $\lambda \leq \lambda_c$ 

Ist Umgekehrt $\lambda < \lambda_c$, dann wähle $\lambda' \in (\lambda,\lambda_c)$ und $w_n=\enb{\frac{\lambda}{\lambda'}}^n$. Dann ist $\sum w_n < \infty$. Da $\lambda' < \lambda_c$ existiert ein Nicht--Null--Fluss $\Theta$ mit $0 \leq \Theta(e) \leq (\lambda')^{-\abs{e}} = w_{\abs{e}}\lambda^{-\abs{e}}$. Also ist der Fluss von endlicher Energie. Somit gibt es einen Strom von $0$ nach $\infty$, also ist die Irrfahrt transient. Definiert man also
\begin{align}
	br(T) = \sup\set{\lambda \given \text{es gibt einen zulässigen nicht--Null-Fluss $\Theta$ von $0$ nach $\infty$ auf } T = \set{V,E,(\lambda^{-\abs{e}})}}
\end{align}
dann haben wir gezeigt:
\begin{satz}
	Sei $T=(V,E,c)$ ein lokal--endlicher, aber unendlicher Baum mit $c(e) = \lambda^{-\abs{e}}$, dann gilt
	\begin{align}
		\lambda &< br(T) \Rightarrow \text{so ist die Irrfahrt auf $T$ transient} \\
		\lambda &> br(T) \Rightarrow \text{so ist die Irrfahrt auf $T$ rekurrent} 
	\end{align}
\end{satz}
\begin{bemerkung} \quad \newline
	\begin{enumerate}[a)] 
		\item Nach MFNC gilt natürlich
			\begin{align}
				br(T) = \sup\set{\lambda \given \inf\limits_{\Pi:\Pi\text{ ist } 0 \leftrightarrow \infty \text{ Cutset}} \sum\limits_{e \in Pi} \lambda ^{-\abs{e}} > 0}
			\end{align}
		\item Wir haben gesehen, dass $br(T)$ für reguläre Bäume tatsächlich die Verzweigungszahl ist. Diese ist für allgemeine Bäume zunächst nicht ganz offensichtlich zu definieren. Ein Versuch wäre
			\begin{align}
				gr(T) = \lim\limits_{n \to \infty} \# \set{\text{Knoten in Generation }n}^{1/n} 
			\end{align}
			mit dem Nachteil, dass der Limes nicht existieren muss und nicht notwendig viel Information über die Struktur von $T$ vermittelt. Falls $gr(T)$ existiert (oder man \enquote{$\liminf$} statt \enquote{$\lim$} schreibt), gilt $br(T) \leq gr(T)$. Warum? Angenommen es gibt einen Nicht-Null-Fluss von $0$ nach $\infty$ mit $\Theta(e) \leq \lambda^{-\abs{e}}$. Dann
			\begin{align}
				\# \set{e \given \abs{e} = n} = \sum\limits_{\abs{e} = n} 1 \geq \sum\limits_{\abs{e} = n} \lambda^{\abs{e}} \Theta(e) = \lambda^n S(\Theta) 
			\end{align}
			ziehen wir die $n$'te Wurzel, bilden $\liminf$ und schicken $\lambda \to \lambda_c$ so ist $br(T) \leq gr(T).$ 
	\end{enumerate}
\end{bemerkung}
\begin{beispiel}
	Sei $T$ der Baum, der sich zu geraden Genrationen binär verzweigt und zu ungeraden jeder Knoten drei Kinder hat. Dann ist $br(T) = \sqrt{6}$, denn
	\begin{align}
		\#\set{\text{Knoten in Generation $n$}} = 2^{\lceil\frac{n}{2} \rceil} 3^{\lceil\frac{n}{2} \rceil} \approx \sqrt{6}^n 
	\end{align}
	d.h. $gr(T) = \sqrt{6}$ und $br(T) \leq \sqrt{6}$.
	Wähle:
	\begin{align}
		\Theta(e) = 	
						\begin{cases}
							\sqrt{6}^{-\abs{e}} &\text{, ungerade Generation} \\
							\frac{1}{3} 6^{\frac{- \abs{e} -1}{2}} &\text{, gerade Generation}
						\end{cases}
	\end{align}
	Dann ist $\Theta(e)$ ein Fluss von $0$ nach $\infty$ mit $\abs{\Theta(e)} \leq (\sqrt{6}) ^{-abs{e}}.$ Das zeigt: $br(T) = \sqrt{6}$.
\end{beispiel}













 