%!TEX root = WTG.tex
% -- Vorlage von: Phil Steinhorst, p.st@wwu.de
\documentclass[a4paper,twoside,index=totoc,toc=bibliography,fontsize=10,DIV=12,headinclude,BCOR=12mm,cleardoublepage=empty, draft]{scrreprt} %option draft hinzügen für ToDos

\usepackage[usenames,x11names]{xcolor} % Die Optionen definieren zusätzliche Farben (siehe Dokumentation)
\definecolor{dark_gray}{gray}{0.45}
\definecolor{light_gray}{gray}{0.6}
\definecolor{fb10_blue}{cmyk}{0.8,0.4,0.13,0.07}
\usepackage[final]{graphicx}

\usepackage{dsfont}

\usepackage{scrtime}

\usepackage{multicol}

\usepackage{lmodern}
\usepackage{mathtools,amssymb,amsthm} % Verbesserung von amsmath (die amsmath selbst lädt)
\mathtoolsset{showonlyrefs}
\usepackage[libertine,cmintegrals,bigdelims,varbb]{newtxmath}
\usepackage[no-math]{fontspec}
\usepackage{polyglossia} % moderner babel-ersatz
\setmainlanguage[spelling=new,babelshorthands=true,latesthyphen]{german}
\shorthandoff{"}
\defaultfontfeatures{Mapping=tex-text, Ligatures={Required,Common,Contextual}}
\setmainfont{LinLibertine}[Extension=.otf,UprightFont=*_R,BoldFont=*_RZ,ItalicFont=*_RI,BoldItalicFont=*_RZI]
\setsansfont{LinBiolinum}[Scale=MatchUppercase, Extension=.otf, UprightFont=*_R, BoldFont=*_RB, ItalicFont=*_RI,BoldItalicFont=*_RBO]
\setmonofont{Inconsolatazi4}[Scale=MatchUppercase,Extension=.otf,UprightFont=*-Regular,BoldFont=*-Bold,StylisticSet=1]
\usepackage[final]{microtype}

%%% Mathematikpakete und Einstellungen
\mathtoolsset{centercolon} % sorgt dafür dass := und =: besser aussehen
\usepackage{mathdots} % sorgt dafür, dass Punte wie zB \ddots besser aussehen
\usepackage{easybmat}
\usepackage{tensor}

%%% kommutative Diagramme
\usepackage{tikz-cd} %-- meiner Meinung nach das beste Paket für kommutative Diagramme
\tikzset{% um Kompatibilität mit Babel herzustellen und die angenehme "<label>"-Syntax zu nutzen
  every picture/.append style={
    execute at begin picture={\shorthandoff{"}},
    execute at end picture={\shorthandon{"}}
  }
}
\usetikzlibrary{quotes,babel,angles}
\usetikzlibrary{patterns}
\usetikzlibrary{arrows.meta}
\usetikzlibrary{matrix}
\tikzset{
	schraffiert/.style={pattern=north west lines,pattern color=#1},
	schraffiert/.default=black
}
\tikzset{>=Latex}

%%% Biber-Settings
\usepackage[%
	backend=biber,
	sortlocale=auto,
	natbib,
	hyperref,
	backref,
	backrefstyle=three+,
	style=alphabetic % eine unvollständige Auswahl von Styles: ieee, numeric, apa
	]%
{biblatex}
\setlength{\bibitemsep}{1em}     % Abstand zwischen den Literaturangaben
\setlength{\bibhang}{2em}        % Einzug nach jeweils erster Zeile
\addbibresource{literature.bib} % Literaturdatei einlesen

%%% Hyperref-Konfigration
\usepackage[hidelinks, pdfpagelabels, bookmarksopen=true, bookmarksnumbered=true, linkcolor=black, urlcolor=SkyBlue2, plainpages=false,pagebackref, citecolor=black, hypertexnames=true, pdfauthor={Lukas Schneider}, pdfborderstyle={/S/U}, linkbordercolor=SkyBlue2, colorlinks=false,backref=false]{hyperref}
\hypersetup{final}

%%% Aufzählung und Zitate
\usepackage[shortlabels]{enumitem}
\setlist[enumerate,description]{font=\sffamily\bfseries}
\usepackage[german=quotes]{csquotes}

\usepackage[textsize=small, obeyDraft]{todonotes}
\usepackage{marginnote}
\renewcommand*{\marginfont}{\color{gray} \footnotesize }
\setlength{\parindent}{0em} 

%%%% Kopf-/Fußzeilen
\usepackage{scrpage2}
\pagestyle{scrheadings}
\clearscrheadfoot 
\providecommand{\verfasser}{Lukas Schneider}
\newcommand{\fach}{Wahrscheinlichkeitstheorie auf Graphen}
\newcommand{\semester}{SoSe 2016}
\newcommand{\homepage}{http://wwwmath.uni-muenster.de/statistik/lehre/SS16/WTaufGraphen/}

\newcommand{\prof}{Prof.\ Dr.\ Matthias Löwe}
%--Konfiguration von scrheadings
\setheadsepline{1pt}[\color{light_gray}]
\pagestyle{scrheadings}
%\fancyhf[H, F]{}
\clearscrheadfoot

\providecommand{\shortFach}{WT auf Graphen}
\lehead{\includegraphics[height=0.6 cm,keepaspectratio]{../!config/Bilder/Logo_WWU_Muenster_light_gray.pdf}}
\rehead{\rule{0cm}{0.6cm}\footnotesize\sffamily\color{light_gray}\verfasser{} -- Mitschrift \shortFach}
\lohead{\rule{0cm}{0.6cm}\footnotesize\sffamily\color{light_gray}Stand: \today \; \thistime[:]}
\rohead{\includegraphics[height=0.6 cm,keepaspectratio]{../!config/Bilder/fb10logo_gray.pdf}}


\ofoot[{ \color{dark_gray} \LARGE \sffamily \thepage}]{{ \color{dark_gray} \LARGE \sffamily \thepage}} %hier wir auch der plain Stil bearbeitet!
\automark{section}
\ifoot{ \color{dark_gray} \small \leftmark}

%--Metadaten
\providecommand{\mail}{ls@schneider-a.com}
\author{\verfasser}
\titlehead{\includegraphics[height=1.5cm, keepaspectratio]{../!config/Bilder/Logo_WWU_Muenster.pdf}%
\hfill \includegraphics[height=1.3cm, keepaspectratio]{../!config/Bilder/fb10logo.pdf}}
\title{Skript \fach}
\subtitle{Mitschrift der Vorlesung  \enquote{\fach} von \prof}

\KOMAoptions{DIV=last}

%\setheadsepline{1pt} 
%\automark[section]{section} % definiert, welcher Text in den Kolumnentiteln erscheinen soll
%\rohead{\rightmark} 
%\lehead{\rightmark} 
%\ofoot[\pagemark]{\pagemark} 

% Indexerstellung
\usepackage{makeidx}
\newcommand{\Index}[1]{\textbf{#1}\index{#1}}
\makeindex
\renewcommand{\indexpagestyle}{scrheadings}






% -- theorem packages
\usepackage{amsthm}
\usepackage{thmtools,thm-restate}
\usepackage{mdframed}
\renewcommand{\listtheoremname}{Übersicht aller Aussagen}

% -- Theoreme als PDF-Lesezeichen
\usepackage{bookmark}
\bookmarksetup{open,numbered}
\makeatletter
\newcommand*{\theorembookmark}{%
	\bookmark[
	dest=\@currentHref,
	rellevel=1,
	keeplevel,
	]{%
		\thmt@thmname\space\csname the\thmt@envname\endcsname
		\ifx\thmt@shortoptarg\@empty
		\else
		\space(\thmt@shortoptarg)%
		\fi
	}%
}   
\makeatother

% -- Definition der einzelnen Umgebungen
\declaretheoremstyle[%
	headfont=\sffamily\bfseries,
	notefont=\normalfont\sffamily\scshape,
	bodyfont=\normalfont,
	headformat=\NUMBER\ \NAME\NOTE,
%	headpunct=.,
	headpunct={\\},
	postheadspace=1em,
	spaceabove=15pt,spacebelow=10pt,
	shaded={bgcolor=gray!20},
	% mdframed={%
	% 	backgroundcolor=gray!20,
	% 	innertopmargin=0pt,
	% 	innerleftmargin=0pt,
	% 	% roundcorner=5pt,
	% 	innerbottommargin=0pt,
	% 	splittopskip = 0pt,
	% 	% skipbelow=6pt,
	% 	% skipbelow=6pt,
	% 	topline=false,bottomline=false,leftline=false,rightline=false
	% },
	postheadhook=\theorembookmark]%
{mainstyle}
	\declaretheoremstyle[%
	headfont=\sffamily\bfseries,
	notefont=\normalfont\sffamily\scshape,
	bodyfont=\normalfont,
	headformat=\NUMBER\ \NAME\NOTE,
%	headpunct=.,
	headpunct={\\},
	postheadspace=1em,
	spaceabove=15pt,spacebelow=10pt,
	shaded={bgcolor=fb10_blue!20},
	postheadhook=\theorembookmark]%
	{mainstyle_blue}
	\declaretheoremstyle[%
	headfont=\sffamily\bfseries,
	notefont=\normalfont\sffamily\scshape,
	bodyfont=\normalfont,
	headformat=\NUMBER\ \NAME\NOTE,
	headpunct=.,
	postheadspace=1em,
	spaceabove=15pt,spacebelow=10pt,
	postheadhook=\theorembookmark]%
{mainstyle_unshaded}
	\declaretheoremstyle[%
	headfont=\sffamily\bfseries,
	notefont=\normalfont\sffamily\scshape,
	bodyfont=\normalfont,
	headformat=\NUMBER\NAME\NOTE,
%	headpunct=.,
	headpunct={\\},
	postheadspace=1em,
	spaceabove=15pt,spacebelow=10pt,
	% shaded={bgcolor=gray!20},
	postheadhook=\theorembookmark]%
	{mainstyle_unnumbered}
	\declaretheoremstyle[%
	headfont=\sffamily\bfseries,
	notefont=\normalfont\sffamily\scshape,
	bodyfont=\normalfont,
	headformat=swapnumber,
	headpunct=.,
	postheadspace=1em,
	spaceabove=15pt,spacebelow=10pt,
	shaded={bgcolor=gray!20},
	postheadhook=\theorembookmark,
	qed=\qedsymbol]%
{mainstyleB}
\declaretheorem[name=Definition,parent=section,style=mainstyle_blue]{definition}
\declaretheorem[name=Definition \& Proposition,refname=Proposition,sharenumber=definition,style=mainstyle_blue]{definitionP}
\declaretheorem[name=Definition,numbered=no,style=mainstyle_unnumbered]{definition*}
\declaretheorem[name=Theorem,sharenumber=definition,style=mainstyle]{theorem}
\declaretheorem[name=Theorem,numbered=no,style=mainstyle_unnumbered]{theorem*}
\declaretheorem[name=Proposition,sharenumber=definition,style=mainstyle]{proposition}
\declaretheorem[name=Lemma,sharenumber=definition,style=mainstyle]{lemma}
\declaretheorem[name=Satz,sharenumber=definition,style=mainstyle]{satz}
\declaretheorem[name=Satz,sharenumber=definition,style=mainstyle_unshaded]{satzUnshaded}
\declaretheorem[name=Definition,sharenumber=definition,style=mainstyle_unshaded]{definitionUnshaded}
\declaretheorem[name=Satz,numbered=no,style=mainstyle_unnumbered]{satz*}
\declaretheorem[name=Korollar,sharenumber=definition,style=mainstyle]{korollar}
\declaretheorem[name=Korollar,sharenumber=definition,style=mainstyleB]{korollarB}
\declaretheorem[name=Frage,numbered=no,style=mainstyle_unnumbered]{frage}
\declaretheorem[name=Frage,sharenumber=definition,style=mainstyle_unshaded]{frageA}
\declaretheorem[name=Erinnerung,sharenumber=definition,style=mainstyle_unshaded]{erinnerungA}
\declaretheorem[name=Ausblick,sharenumber=definition,style=mainstyle_unshaded]{ausblick}
\declaretheorem[name=Konvention,sharenumber=definition,style=mainstyle]{konvention}
\declaretheorem[name=Notation,sharenumber=definition,style=mainstyle_unshaded]{notation}
\declaretheorem[name=Bemerkung,sharenumber=definition,style=mainstyle_unshaded]{bemerkung}
\declaretheorem[name=Beispiel,sharenumber=definition,style=mainstyle_unshaded]{beispiel}
\declaretheorem[name=Regel,sharenumber=definition,style=mainstyle_unshaded]{regel}

% -- Beweise
\declaretheoremstyle[headfont=\bfseries\scshape,bodyfont=\normalfont,headpunct=:,postheadspace=1em,spacebelow=12pt,spaceabove=2pt,qed=\qedsymbol]{beweise}
\declaretheoremstyle[headfont=\sffamily\bfseries,bodyfont=\normalfont,headpunct=:,postheadspace=1em,spacebelow=10pt,spaceabove=10pt]{bemerkungen}
\declaretheorem[name=Beweis,numbered=no,style=beweise]{beweis}


\declaretheorem[name=Übung,numbered=no,style=bemerkungen]{uebung}
\declaretheorem[name=Erinnerung,numbered=no,style=bemerkungen]{erinnerung}


















%%-- Theorem-Pakete und Konfiguration
%\usepackage{thmtools}

%\declaretheoremstyle[%
%	headfont=\sffamily\bfseries,
%	notefont=\normalfont\sffamily,
%	bodyfont=\normalfont,
%	headformat=\NUMBER \ \NAME \NOTE,
%	headpunct={\\},
%	postheadspace=1ex,
%	spaceabove=15pt,spacebelow=10pt]%
%{mainstyle}
%\declaretheoremstyle[%
%	headfont=\sffamily\bfseries,
%	notefont=\normalfont\sffamily,
%	bodyfont=\normalfont,
%	headformat=\NAME \NOTE,
%	headpunct={\\},
%	postheadspace=1ex,
%	spaceabove=15pt,spacebelow=10pt]%
%{nonumber}
%\declaretheoremstyle[%
%	headfont=\sffamily\bfseries,
%	notefont=\normalfont\sffamily,
%	bodyfont=\normalfont,
%	headformat=\NAME \ \NOTE,
%	headpunct={\\},
%	postheadspace=1ex,
%	spaceabove=15pt,spacebelow=10pt]%
%{miscstyle}
%\declaretheoremstyle[%
%	headfont=\bfseries\scshape,
%	bodyfont=\normalfont,
%	headpunct=:,
%	postheadspace=1ex,
%	spacebelow=12pt,spaceabove=2pt,
%	qed=\qedsymbol]%
%{beweise}
%
%
%\declaretheorem[name=Definition,parent=section,style=mainstyle]{definition}
%\declaretheorem[name=Satz,sharenumber=definition,style=mainstyle]{satz}
%\declaretheorem[name=Korollar,sharenumber=definition,style=mainstyle]{korollar}
%\declaretheorem[name=Lemma,sharenumber=definition,style=mainstyle]{lemma}
%\declaretheorem[name=Proposition,sharenumber=definition,style=mainstyle]{proposition}
%\declaretheorem[name=Bemerkung,sharenumber=definition,style=mainstyle]{bemerkung}
%\declaretheorem[name=Beispiel,sharenumber=definition,style=mainstyle]{beispiel}
%\declaretheorem[name=Erinnerung,sharenumber=definition,style=mainstyle]{erinnerung}
%\declaretheorem[name=Beobachtung,sharenumber=definition,style=mainstyle]{beobachtung}
%\declaretheorem[name=Theorem,sharenumber=definition,style=mainstyle]{theorem}
%\declaretheorem[name=Aufgabe,parent=chapter,style=mainstyle]{aufgabe}
%\declaretheorem[name=Problem,sharenumber=definition,style=mainstyle]{problem}
%
%\declaretheorem[name=Beweis,numbered=no,style=beweise]{beweis}

% \declaretheorem[name=Erinnerung,numbered=no,style=nonumber]{no-erinnerung}


\usepackage{xparse}

\usepackage{setspace}
\usepackage{hyphenat}
\usepackage{bm}
\allowdisplaybreaks
\raggedbottom
\renewcommand{\phi}{\varphi}
\usepackage{chngcntr}
\counterwithout{equation}{chapter} % undo numbering system provided by phstyle.cls
\counterwithout{section}{chapter}
\renewcommand\thechapter{\Roman{chapter}}
\usepackage{faktor}

\usepackage{cancel}

\usepackage{subcaption}

%!TEX root = LA2_SS16
% Author: Phil Steinhorst, p.st@wwu.de

% Abk�rzungen
% ===========================================================
	\newcommand{\CC}{\mathbb{C}}
	\newcommand{\EE}{\mathbb{E}}
	\newcommand{\FF}{\mathbb{F}}
	\newcommand{\HH}{\mathcal{H}}
	\newcommand{\KK}{\mathbb{K}}
	\newcommand{\LL}{\mathbb{L}}
	\newcommand{\NN}{\mathbb{N}}
	\newcommand{\QQ}{\mathbb{Q}}
	\newcommand{\RR}{\mathbb{R}}
	\newcommand{\ZZ}{\mathbb{Z}}
	\newcommand{\oh}{\mathcal{O}}				% Landau-O
	\newcommand{\ind}{1\hspace{-0,8ex}1} 		% Indikatorfunktion (Doppeleins)
	\newcommand{\bewrueck}{"$\Leftarrow$":} 	% Beweis R�ckrichtung
	\newcommand{\bewhin}{"$\Rightarrow$":}		% Beweis Hinrichtung
	\newcommand{\setone}{\{1\}}					% Einsmenge
	\newcommand{\NT}{\trianglelefteq}			% Normalteiler
	\newcommand{\setzero}{\{0\}}				% Nullmenge
	\newcommand{\ol}[1]{\overline{#1}}
	\newcommand{\wt}[1]{\widetilde{#1}}
	\newcommand{\wh}[1]{\widehat{#1}}
% ===========================================================

% Operatoren
% ===========================================================
	\DeclareMathOperator{\Abb}{Abb}				% Menge der Abbildungen
	\DeclareMathOperator{\Bild}{Bild}			% Bild
	\DeclareMathOperator{\Char}{char} 			% Charakteristik
	\DeclareMathOperator{\Det}{\det\,\!}		% Determinante mit Subskript
	\DeclareMathOperator{\End}{End}				% Endomorphismen
	\DeclareMathOperator{\GL}{GL}				% allgemeine lineare Gruppe
	\DeclareMathOperator{\Hom}{Hom} 			% Homomorphismen
	\DeclareMathOperator{\id}{id} 				% Identit�t
	\DeclareMathOperator{\im}{im} 				% image
	\renewcommand{\Im}{\operatorname{Im}}		% Imagin�rteil
	\DeclareMathOperator{\Kern}{Kern}			% Kern
	\DeclareMathOperator{\LH}{LH}				% Lineare H�lle
	\DeclareMathOperator{\ord}{ord} 			% Ordnung
	\DeclareMathOperator{\pot}{\mathcal{P}}		% Potenzmenge
	\DeclareMathOperator{\Rang}{Rang}			% Rang
	\DeclareMathOperator{\rk}{rk}				% rank
	\renewcommand{\Re}{\operatorname{Re}}		% Realteil
	\DeclareMathOperator{\sgn}{sgn} 			% Signum
	\DeclareMathOperator{\sign}{sign} 			% Signum
	\DeclareMathOperator{\SL}{SL} 				% Spezielle lineare Gruppe
	\DeclareMathOperator{\SO}{S\oh} 			% Spezielle orthogonale Gruppe
	\DeclareMathOperator{\SU}{S\UU} 			% Spezielle unit�re Gruppe
	\DeclareMathOperator{\Sym}{Sym} 			% Symmetrische Gruppe
% ===========================================================

% Klammerungen und �hnliches
% ===========================================================
	\DeclarePairedDelimiter{\absolut}{\lvert}{\rvert}		% Betrag
	\DeclarePairedDelimiter{\ceiling}{\lceil}{\rceil}		% aufrunden
	\DeclarePairedDelimiter{\Floor}{\lfloor}{\rfloor}		% aufrunden
	\DeclarePairedDelimiter{\Norm}{\lVert}{\rVert}			% Norm
	\DeclarePairedDelimiter{\sprod}{\langle}{\rangle}		% spitze Klammern
	\DeclarePairedDelimiter{\enbrace}{(}{)}					% runde Klammern
	\DeclarePairedDelimiter{\benbrace}{\lbrack}{\rbrack}	% eckige Klammern
	\DeclarePairedDelimiter{\penbrace}{\{}{\}}				% geschweifte Klammern
	\newcommand{\Underbrace}[2]{{\underbrace{#1}_{#2}}} 	% bessere Unterklammerungen
	% Kurzschreibweisen f�r Faule und Code-Vervollst�ndigung
	\newcommand{\abs}[1]{\absolut*{#1}}
	\newcommand{\ceil}[1]{\ceiling*{#1}}
	\newcommand{\flo}[1]{\Floor*{#1}}
	\newcommand{\no}[1]{\Norm*{#1}}
	\newcommand{\sk}[1]{\sprod*{#1}}
	\newcommand{\enb}[1]{\enbrace*{#1}}
	\newcommand{\penb}[1]{\penbrace*{#1}}
	\newcommand{\benb}[1]{\benbrace*{#1}}
% ===========================================================

% Sonstiges
% ===========================================================
	\newcommand{\stack}[2]{\makebox[1cm][c]{$\stackrel{#1}{#2}$}}
	\DeclareMathOperator{\ev}{ev}
	\newcommand{\mat}[4]{\tensor*[^{#2}_{}]{#1}{^{#3}_{#4}}}
	\DeclareMathOperator{\diag}{diag}
	\DeclareMathOperator{\grad}{grad}
% ===========================================================
\DeclareMathOperator{\med}{med}					%median	
\DeclareMathOperator{\Bin}{Bin}					%Binomialvert.
\DeclareMathOperator{\BetaV}{Beta}				%Betavert.

\newcommand{\gauss}[1]{\left\lfloor#1\right\rfloor}	%Floor

\newcommand{\RG}{\ensuremath{G\enbrace{n,p}}} 	%Random Graph

%cases Umgebung anpassbar
\makeatletter
\renewcommand{\env@cases}[1][@{}l@{\quad}l@{}]{%
  \let\@ifnextchar\new@ifnextchar
  \left\lbrace
  \def\arraystretch{1.2}%
  \array{#1}%
}
\makeatother

%Anführungszeichen

\newcommand{\qte}[1]{\glqq #1\grqq}

%%Wkeitsbefehl definieren
%\newcommand{\propNEW}[2][\relax]{
%\ifx#3\relax
%	\ifx#2\relax
%		\ifx#1\relax
%			\ensuremath{\mathbb{P}}
%		\else
%			\ensuremath{\mathbb{P}\enb{#1}}
%		\fi
%%	\else
%%		\ifx#1\relax
%%			\ensuremath{\mathbb{P}_{#2}}
%%		\else
%%			\ensuremath{\mathbb{P}_{#2}\enb{#1}}
%%		\fi
%	\fi
%\else
%	\ifx#2\relax
%%		\ifx#1\relax
%%			\ensuremath{\mathbb{P}^{#3}
%%		\else
%%			\ensuremath{\mathbb{P}^{#3}\enb{#1}}
%%		\fi
%%	\else
%%		\ifx#1\relax
%%			\ensuremath{\mathbb{P}^{#3}_{#2}}
%%		\else
%%			\ensuremath{\mathbb{P}^{#3}_{#2}\enb{#1}}
%%		\fi
%	\fi			
%\fi
%}


\NewDocumentCommand{\propEckig}{m o o}{
	\IfNoValueTF{#2}
		{
			\IfNoValueTF{#3}
				{\ensuremath{\mathbb{P}\enb{\renewcommand\given{\;\middle\vert\;} #1}}}
				{\ensuremath{\mathbb{P}_{#3}\enb{\renewcommand\given{\;\middle\vert\;} #1}}}
		}
		{
			\IfNoValueTF{#3}
				{\ensuremath{\mathbb{P}^{#2}\enb{\renewcommand\given{\;\middle\vert\;} #1}}}
				{\ensuremath{\mathbb{P}^{#2}_{#3}\enb{\renewcommand\given{\;\middle\vert\;} #1}}}	
		}
	}
\NewDocumentCommand{\propRund}{m o o}{
	\IfNoValueTF{#2}
	{
		\IfNoValueTF{#3}
		{\ensuremath{\mathbb{P}\enb{\renewcommand\given{\;\middle\vert\;} #1}}}
		{\ensuremath{\mathbb{P}_{#3}\enb{\renewcommand\given{\;\middle\vert\;} #1}}}
	}
	{
		\IfNoValueTF{#3}
		{\ensuremath{\mathbb{P}^{#2}\enb{\renewcommand\given{\;\middle\vert\;} #1}}}
		{\ensuremath{\mathbb{P}^{#2}_{#3}\enb{\renewcommand\given{\;\middle\vert\;} #1}}}	
	}
}	


	
\newcommand{\prop}[1]{\propRund{#1}}
\newcommand{\propE}[1]{\propEckig{#1}}

\newcommand{\p}{\mathbb{P}}
%%Wkeitsbefehl definieren
%\newcommand{\prop}[1]{
%	\ifx#1\relax \ensuremath{\mathbb{P}}
%	\else \propRund{#1}
%	\fi}

%Wkeitsbefehl definieren Eckig
%\newcommand{\propE}[1]{
%	\ifx#1\relax \ensuremath{\mathbb{P}}
%	\else \ensuremath{\mathbb{P}\bracketeckig{#1}}	
%	\fi}

%EW definieren
\newcommand{\EW}[1]{
	\ifx#1\relax \ensuremath{\mathbb{E}}
	\else \ensuremath{\mathbb{E}\enb{#1}}	
	\fi}

%Var definieren
\newcommand{\Var}[1]{
	\ifx#1\relax \ensuremath{\mathbb{V}}
	\else \ensuremath{\mathbb{V}\enb{#1}}	
	\fi}

%Cov definieren
\newcommand{\Cov}[1]{
	\ifx#1\relax \ensuremath{Cov}
	\else \ensuremath{Cov\enb{#1}}	
	\fi}

%Cov definieren eckig
\newcommand{\CovE}[1]{
	\ifx#1\relax \ensuremath{Cov}
	\else \ensuremath{Cov\benb{#1}}	
	\fi}

%EW definieren Eckig
\newcommand{\EWE}[1]{
	\ifx#1\relax \ensuremath{\mathbb{E}}
	\else \ensuremath{\mathbb{E}\benb{#1}}	
	\fi}

%Var definieren Eckig
\newcommand{\VarE}[1]{
	\ifx#1\relax \ensuremath{\mathbb{V}}
	\else \ensuremath{\mathbb{V}\benb{#1}}	
	\fi}

%asymptotisch fast sicher
\newcommand{\afs}{\ifmmode \mathrm{\ a.f.s.} \else a.f.s. \fi}

%Landau big O
\newcommand{\bigO}[1]{\mathcal{O}\enb{#1}}
%Reele Zahlen
\newcommand{\Real}{\mathbb{R}}
\newcommand{\Nat}{\mathbb{N}}
\newcommand{\leit}[1]{\mathcal{C}\enb{#1}}
\newcommand{\greens}[1]{\mathcal{G}\enb{#1}} 
