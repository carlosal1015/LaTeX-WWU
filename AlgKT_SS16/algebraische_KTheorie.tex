%!TEX TS-program = xelatex
%!TEX TS-options = -shell-escape
%!TEX root = ../OpAlg2_SS16/operatoralgebren2.tex
\RequirePackage{fix-cm} 
\documentclass[a4paper, twoside, headsepline, index=totoc,toc=listof,toc=bibliography,toc=index, fontsize=10pt, cleardoublepage=empty, headinclude, DIV=12, BCOR=5mm, titlepage,draft]{scrartcl}
%!TEX root = ../AnaTopGeo_SS14/ana_top_geo.tex
\usepackage{scrtime} % KOMA, Uhrzeit ermoeglicht

%--Pakete zum "Programmieren"
% ======================================================================================
\usepackage{etoolbox}
\usepackage{letltxmacro}
\usepackage{ifthen}
% ======================================================================================

%--Farbdefinitionen und Grafiken (muss vor tikz geladen werden)
% ======================================================================================
\usepackage[usenames, table, x11names]{xcolor}
\definecolor{dark_gray}{gray}{0.45}
\definecolor{light_gray}{gray}{0.6}
\definecolor{fb10_blue}{cmyk}{0.8,0.4,0.13,0.07}
\usepackage[final]{graphicx}
\usepackage{adjustbox}
\newcommand{\cfbox}[2]{% coloured frame box
	\ifmmode
	\mathchoice{\adjustbox{cfbox=#1}{$\displaystyle#2$}}{\adjustbox{cfbox=#1}{$\textstyle#2$}}{\adjustbox{cfbox=#1}{$\scriptstyle#2$}}{\adjustbox{cfbox=#1}{$\scriptscriptstyle#2$}}
	\else
	\adjustbox{cfbox=#1}{#2}
	\fi
}
% ======================================================================================

%--Zum Zeichnen/ TikZ-Kram (vor polyglossia bzw. babel geladen werden)
% ======================================================================================
\usepackage{tikz}
\usepackage{tikz-cd}
\usetikzlibrary{external}
\tikzset{>=latex}
\usetikzlibrary{%
	shapes,
	arrows.meta,
	intersections,
	calc,
	3d,
	decorations.pathreplacing,decorations.markings,decorations.pathmorphing,
	angles,
	quotes,
}
\tikzexternalize[prefix=tikz/,up to date check=diff]
\pgfkeys{/pgf/images/include external/.code=\includegraphics{#1}}
\tikzset{external/system call={lualatex \tikzexternalcheckshellescape -halt-on-error -interaction=batchmode --shell-escape -jobname "\image" "\texsource"}}
\AtBeginEnvironment{tikzcd}{\tikzexternaldisable} % tikzexternalize fuer tikzcd deaktivieren, da inkompatibel
\AtEndEnvironment{tikzcd}{\tikzexternalenable}
\tikzset{% um Inkompatibilitaeten von quotes und polyglossia bzw. babel zu vermeiden
  every picture/.append style={
    execute at begin picture={\shorthandoff{"}},
    execute at end picture={\shorthandon{"}}
  }
}
\usepackage{pgfplots}
\usepgfplotslibrary{colormaps}
\newcommand*\circled[1]{\tikzexternaldisable\tikz[baseline=(char.base)]{\node[shape=circle,draw,inner sep=2pt] (char) {#1};}\tikzexternalenable}
% ======================================================================================



%-- Mathepakete etc.
% ======================================================================================
\usepackage[T1]{fontenc}
\renewcommand{\rmdefault}{zpltlf}
\usepackage{mathtools} % beinhaltet amsmath
\mathtoolsset{showonlyrefs,centercolon,showmanualtags}
\newtagform{brackets}[\textbf]{[}{]}
\usetagform{brackets}
\usepackage{fix-cm}
\usepackage[bbgreekl]{mathbbol}
\usepackage{amssymb,marvosym} 
\usepackage{nicefrac} % schräge Brüche
\usepackage{faktor}
\newcommand{\Faktor}[1]{\faktor[\textstyle]{#1}}
\usepackage{xfrac}
\usepackage{cancel}
\usepackage{mathdots} % Verbesserung von Punkten wie zB \ldots
\usepackage[bb=px]{mathalfa} % \mathbb als px font
\usepackage{centernot}
\usepackage{stackrel}
\DeclareSymbolFont{bbold}{U}{bbold}{m}{n}
\DeclareSymbolFontAlphabet{\mathbbold}{bbold}
\newcommand{\ind}{\mathbbold{1}} % charakteristische-Funktion-Eins
\def\mathul#1#2{\color{#1}\underline{{\color{black}#2}}\color{black}} %farbiges Untersteichen im Mathe-Modus
\renewcommand{\le}{\leqslant}
\renewcommand{\ge}{\geqslant}
% ======================================================================================


%-- Von xfrac erzeuge font warnings ignorieren
% ======================================================================================
\usepackage{silence}
\WarningFilter{latexfont}{Size substitutions with differences}
\WarningFilter{latexfont}{Font shape `U/bbold/m/n' in size}
% ======================================================================================


%-- Typographie/Polyglossia
% ======================================================================================
\usepackage[euler-digits]{eulervm} % vor fontspec laden!
\usepackage[no-math]{fontspec}
\usepackage{polyglossia} % moderner babel-ersatz
\setmainlanguage[spelling=new,babelshorthands=true]{german}
\shorthandoff{"}
\setotherlanguage{english}
\defaultfontfeatures{Mapping=tex-text, WordSpace={1.2}, Ligatures={Required,Common,Contextual},Extension=.otf} %


\setmainfont{TeXGyrePagellaX}[UprightFont=*-Regular,BoldFont=*-Bold,ItalicFont=*-Italic,BoldItalicFont=*-BoldItalic,ItalicFeatures={Style=Historic},Ligatures={Required,Common,Contextual,Historic}]
\setsansfont{texgyreadventor}[Scale=MatchUppercase, UprightFont=*-regular, BoldFont=*-bold, ItalicFont=*-italic, BoldItalicFont=*-bolditalic]
\setmonofont{SourceCodePro}[Scale=0.9,UprightFont=*-Regular, BoldFont=*-Semibold, ItalicFont=*-Light]
\usepackage{xltxtra}
\usepackage{fontawesome}
\usepackage[final]{microtype}
\usepackage[draft=false]{scrlayer-scrpage} 
\flushbottom
% ======================================================================================


%-- Aufzählungen
% ======================================================================================
\usepackage[shortlabels,inline]{enumitem}
\setlist[itemize,1]{label=\faCaretRight}
\setlist[enumerate]{font=\bfseries}
\setlist[description]{font=\normalfont\bfseries}
\usepackage{multicol}
% ======================================================================================


%-- Floats/Figures/Tabellen
% ======================================================================================
\usepackage{wrapfig}
\usepackage{float}
\usepackage[margin=10pt, font=small, labelfont={sf, bf}, format=plain, indention=1em]{caption}
\captionsetup[wrapfigure]{name=Abb. }
\usepackage{booktabs}
% ======================================================================================


%-- korrekte Anführungszeichen und Zitierbefehle
% ======================================================================================
\usepackage[autostyle,german=quotes,english=british]{csquotes}
% ======================================================================================


%--Indexverarbeitung
% ======================================================================================
\usepackage{makeidx}
\newcommand{\bet}[1]{\textbf{\emph{#1}}}
\newcommand{\Index}[1]{\bet{#1}\index{#1}}
\makeindex
\setindexpreamble{{\noindent\sffamily\small Die \emph{Seitenzahlen} sind mit Hyperlinks versehen und somit anklickbar} \par \bigskip}
\renewcommand{\indexpagestyle}{scrheadings}
% ======================================================================================


%-- Marginnotes/Todonotes/Footnotes
% ======================================================================================
\deffootnote[1.5em]{1.5em}{1.5em}{\textsuperscript{\thefootnotemark}\ }
\usepackage[fulladjust]{marginnote}
\renewcommand*{\marginfont}{\itshape\footnotesize}
\usepackage[textsize=small]{todonotes}
\usepackage{ragged2e}
\renewcommand*{\raggedleftmarginnote}{\RaggedLeft}
\renewcommand*{\raggedrightmarginnote}{\RaggedRight}
\LetLtxMacro{\oldtodo}{\todo}
\renewcommand{\todo}[2][]{\tikzexternaldisable\oldtodo[#1]{#2}\tikzexternalenable}
\LetLtxMacro{\oldmissingfigure}{\missingfigure}
\renewcommand{\missingfigure}[2][]{\tikzexternaldisable\oldmissingfigure[{#1}]{#2}\tikzexternalenable}
% ======================================================================================


% -- BibLaTeX
% ======================================================================================
\usepackage[%
	backend=biber,
	sortlocale=auto,
	natbib,
	hyperref,
	backref,
	style=alphabetic
	]%
{biblatex}
\renewcommand*{\mkbibnamelast}[1]{%
  \ifmknamesc{\textsc{#1}}{#1}}
\renewcommand*{\mkbibnameprefix}[1]{%
  \ifboolexpr{ test {\ifmknamesc} and test {\ifuseprefix} }
    {\textsc{#1}}
    {#1}}
\def\ifmknamesc{%
  \ifboolexpr{ test {\ifcurrentname{labelname}}
               or test {\ifcurrentname{author}}
               or ( test {\ifnameundef{author}} and test {\ifcurrentname{editor}} ) }}
\addbibresource{../!config/quellen.bib}
% ======================================================================================

%--Konfiguration von Hyperref und Cleveref
% ======================================================================================
\usepackage[hidelinks, pdfpagelabels,  bookmarksopen=true, bookmarksnumbered=true, linkcolor=black, urlcolor=SkyBlue2, plainpages=false,pagebackref, citecolor=black, hypertexnames=true, pdfauthor={Jannes Bantje}, pdfborderstyle={/S/U}, linkbordercolor=SkyBlue2, colorlinks=false,final,backref=false]{hyperref}
\usepackage[nameinlink,noabbrev]{cleveref}
\newcommand{\appendLink}[1]{#1\,\faExternalLink}
\newcommand{\hrefsym}[2]{\href{#1}{\texttt{\appendLink{#2}}}}
\newcommand{\hrefsymX}[2]{\href{#1}{\appendLink{#2}}}
\newcommand{\hrefsymmail}[2]{\href{#1}{\texttt{\faEnvelopeO\,#2}}}
\renewcommand{\url}[1]{\hrefsym{#1}{\nolinkurl{#1}}}
% ======================================================================================


% -- QR-Codes (hinter hyperref laden!)
% ======================================================================================
\usepackage{qrcode}
% ======================================================================================

%--Römische Zahlen
% ======================================================================================
\newcommand{\RM}[1]{\MakeUppercase{\romannumeral #1{}}}
% ======================================================================================

%-- Definition von diversen Mathe-Befehlen
% ======================================================================================
%!TEX root = mitschrift_main.tex

% -- Zum Finetuning von Befehlen
% ======================================================================================
\makeatletter
\newcommand{\raisemath}[1]{\mathpalette{\raisem@th{#1}}}
\newcommand{\raisem@th}[3]{\raisebox{#1}{$#2#3$}}
\makeatother
\makeatletter
\newcommand{\killDescendersM}[1]{\mathpalette{\killD@scendersM{#1}}}
\newcommand{\killD@scendersM}[2]{\raisebox{0pt}[\height][0pt]{$#2#1$}}
\makeatother
\DeclareRobustCommand{\minwidthbox}[2]{%
  \ifmmode
    \expandafter\mathmakebox
  \else
    \expandafter\makebox
  \fi
  [\ifdim#2<\width\width\else#2\fi]{#1}%
}
% ======================================================================================


%-- Klammerbefehle
% ======================================================================================
\DeclarePairedDelimiter{\abs}{\lvert}{\rvert}
\DeclarePairedDelimiter{\floor}{\lfloor}{\rfloor}
\DeclarePairedDelimiter{\ceil}{\lceil}{\rceil}
\DeclarePairedDelimiter\norm{\Vert}{\Vert}
\DeclarePairedDelimiter\enbrace{(}{)}
\DeclarePairedDelimiter\benbrace{[}{]}
\DeclarePairedDelimiter\bbenbrace{[\![}{]\!]}
\DeclarePairedDelimiter\lenbrace{<}{>}
\DeclarePairedDelimiter\angbrace{\langle}{\rangle}
\newcommand{\ssbrace}[1]{{\scriptscriptstyle\enbrace{#1}}}
\newcommand{\ssbbrace}[1]{{\scriptscriptstyle\benbrace{#1}}}
% ======================================================================================

%-- Mengen
% ======================================================================================
\newcommand\SetSymbol[1][]{\nonscript\:#1\vert\allowbreak\nonscript\:\mathopen{}}
\providecommand\given{} % to make it exist
\DeclarePairedDelimiterX\set[1]\{\}{\renewcommand\given{\SetSymbol[\delimsize]}#1}
% ======================================================================================

%-- Skalarprodukt (3 Varianten) 
% ======================================================================================
\DeclarePairedDelimiterX\sprod[2]{\langle}{\rangle}{#1\,\delimsize\vert\,#2}
\DeclarePairedDelimiterX\skal[2]{\langle}{\rangle}{#1\,,\,#2}
\makeatletter
\DeclareFontFamily{OMX}{MnSymbolE}{}
\DeclareSymbolFont{MnLargeSymbols}{OMX}{MnSymbolE}{m}{n}
\SetSymbolFont{MnLargeSymbols}{bold}{OMX}{MnSymbolE}{b}{n}
\DeclareFontShape{OMX}{MnSymbolE}{m}{n}{
    <-6>  MnSymbolE5
   <6-7>  MnSymbolE6
   <7-8>  MnSymbolE7
   <8-9>  MnSymbolE8
   <9-10> MnSymbolE9
  <10-12> MnSymbolE10
  <12->   MnSymbolE12
}{}
\DeclareFontShape{OMX}{MnSymbolE}{b}{n}{
    <-6>  MnSymbolE-Bold5
   <6-7>  MnSymbolE-Bold6
   <7-8>  MnSymbolE-Bold7
   <8-9>  MnSymbolE-Bold8
   <9-10> MnSymbolE-Bold9
  <10-12> MnSymbolE-Bold10
  <12->   MnSymbolE-Bold12
}{}
\let\llangle\@undefined
\let\rrangle\@undefined
\DeclareMathDelimiter{\llangle}{\mathopen}%
                     {MnLargeSymbols}{'164}{MnLargeSymbols}{'164}
\DeclareMathDelimiter{\rrangle}{\mathclose}%
                     {MnLargeSymbols}{'171}{MnLargeSymbols}{'171}
\makeatother
\DeclarePairedDelimiterX\sskal[2]{\llangle}{\rrangle}{#1\,,\,#2}
% ======================================================================================

%-- Abbildungsdefinition
% ======================================================================================
\newcommand{\mapdef}[5]{%
	\[
		\begin{array}{rcl}
			\textstyle #1 &\xrightarrow{\minwidthbox{#5}{2em}} & \textstyle #2 \\[0.5ex]
			\textstyle #3 &\xmapsto{\minwidthbox{\mbox{ }}{2em}} & \textstyle #4
		\end{array}
	\]
}
% ======================================================================================

%-- modifiziertes Stackrel 
% ======================================================================================
\newcommand{\StackText}[2]{\stackrel{\mbox{\scriptsize #1}}{#2}}
\newcommand{\StackTextClap}[2]{\stackrel{\mathclap{\mbox{\scriptsize #1}}}{#2}}
% ======================================================================================

%-- Blitz
% ======================================================================================
\newcommand{\light}{\text{\raisebox{-.3ex}{\Large\Lightning}}}
% ======================================================================================


%-- Underbrace u.Ä. als Befehl in LaTeX-Syntax (und ohne Spacingprobleme mit nachfolgenden Operatoren...)
% ======================================================================================
\newcommand{\Underbrace}[2]{{\underbrace{#1}_{#2}}}
\newcommand{\Underbracket}[2]{{\underbracket[0.7pt][2pt]{#1}_{#2}}}
\newcommand{\Overbracket}[2]{{\overbracket[0.7pt][2pt]{#1}^{#2}}}
% ======================================================================================


%-- Deklaration weiterer Operatoren (allgemein)
% ======================================================================================
\DeclareMathOperator{\re}{Re} % Realteil
\let\Re\relax
\DeclareMathOperator{\Re}{Re} % Realteil
\DeclareMathOperator{\im}{im} % Bild
\let\Im\relax
\DeclareMathOperator{\Im}{Im} % Bild
\DeclareMathOperator{\id}{id} % identische Abbildung
\DeclareMathOperator{\conj}{conj} % Konjugation
\DeclareMathOperator{\sgn}{sgn} % Signum
\DeclareMathOperator{\End}{End} % Endomorphismen
\DeclareMathOperator{\Hom}{Hom} % Homomorphismen
\DeclareMathOperator{\Iso}{Iso} % Isomorphismen
\DeclareMathOperator{\Aut}{Aut} % Automorphismen
\DeclareMathOperator{\Span}{span} % Span
\DeclareMathOperator{\coker}{coker} % Kokern
\DeclareMathOperator{\Tr}{Tr} % Spur,Trace
\DeclareMathOperator{\pr}{pr} % Projektion
\DeclareMathOperator{\diag}{diag} % Diagonalmatrix
\DeclareMathOperator{\Rg}{Rg} % Rang
\DeclareMathOperator{\const}{const} % konstante Abbildung
\DeclareMathOperator{\Spur}{Spur} % Spur
\DeclareMathOperator{\Arg}{Arg} % Argument
\DeclareMathOperator{\dist}{dist} % Distanz
\DeclareMathOperator{\supp}{supp} % Träger
\DeclareMathOperator{\Char}{char} % Charakteristik
% ======================================================================================


%-- Deklaration weiterer Operatoren (Differentiale etc.)
% ======================================================================================
\DeclareMathOperator{\grad}{grad} % Gradient
\DeclareMathOperator{\dive}{div} % Gradient
\DeclareMathOperator{\rot}{rot} % Rotation
\newcommand{\D}{\ensuremath{\mathrm{D}\mkern-1.0mu}} % Differential
\newcommand{\mathd}{\ensuremath{\mathrm{d}\mkern-1.0mu}} % äußere Ableitung
\newcommand{\Tmap}{\ensuremath{\mathrm{T}\mkern-0.85mu}} % Tangentialraum
\let\Tang\Tmap
\DeclareMathOperator{\Diff}{Diff}
\newcommand{\diff}[2]{\ensuremath{\frac{{\partial #1}}{{\partial #2}} }}
\newcommand{\diffd}[2]{\ensuremath{\frac{\mathd #1}{\mathd #2} }}
\DeclareMathOperator{\rank}{rank}
% ======================================================================================


%-- Deklaration weiterer Operatoren (Topologie)
% ======================================================================================
\newcommand*\interior[1]{\overset{\smash{\raisebox{-0.18ex}{$\scriptstyle\circ$}}}{#1}}
\newcommand{\sing}{{\raisemath{1.1pt}{\scriptscriptstyle\mathrm{sing}}}}
\newcommand{\pt}{\mathrm{pt}}
\DeclareMathOperator{\Zyl}{Zyl}
\newcommand{\rZyl}{\widetilde{\Zyl}}
\DeclareMathOperator{\Tel}{Tel}
\newcommand{\op}{\mathrm{op}}
\DeclareMathOperator{\Sp}{Sp}
\DeclareMathOperator{\Keg}{Keg}
\newcommand{\slashedi}{i\hspace{-3.5pt}/}
\newcommand{\cupp}{\smallsmile}
\newcommand{\capp}{\smallfrown}
\DeclareMathOperator*{\colim}{colim}
\DeclareMathOperator{\PD}{PD}
\newcommand{\lf}{\mathrm{lf}}
\DeclareMathOperator{\sig}{sig}
\DeclareMathOperator{\Tor}{Tor}
\DeclareMathOperator{\Ext}{Ext}
\DeclareMathOperator{\AW}{AW}
\DeclareMathOperator{\Proj}{Proj}
\DeclareMathOperator{\Gr}{Gr}
\DeclareMathOperator{\res}{res}
\DeclareMathOperator{\Spec}{Spec}
\DeclareMathOperator{\co}{co}
\DeclareMathOperator{\ch}{ch}
\DeclareMathOperator{\wOp}{w}
\DeclareMathOperator{\Ar}{Ar}
\newcommand{\actson}{\mathrel{\curvearrowright}}
\let\acts\actson
\let\action\actson
\DeclareMathSymbol{\bbDelta}{\mathord}{bbold}{"01}
\newcommand{\DDelta}{\bbDelta}
\DeclareMathOperator{\Star}{Star}
\DeclareMathOperator{\Link}{Link}
\DeclareMathOperator{\EPK}{EPK}
\DeclareMathOperator{\Vol}{Vol}
\newcommand{\cell}{{\raisemath{1.1pt}{\scriptscriptstyle\mathrm{cell}}}}
\DeclarePairedDelimiter{\homologieklasse}{\llbracket}{\rrbracket}
\newcommand{\rand}[1]{\ensuremath{\partial^{\scriptscriptstyle #1}}}
\DeclareMathOperator{\ab}{ab}
\DeclareMathOperator{\CW}{CW}
% ======================================================================================


%-- Deklaration von Operatoren (Liegruppen)
% ======================================================================================
\DeclareMathOperator{\GL}{GL}
\DeclareMathOperator{\SO}{SO}
\DeclareMathOperator{\Ad}{Ad}
\DeclareMathOperator{\ad}{ad}
\DeclareMathOperator{\On}{O}
\DeclareMathOperator{\Un}{U}
\DeclareMathOperator{\SU}{SU}
\DeclareMathOperator{\Mat}{Mat}
\DeclareRobustCommand{\Der}{\mathop{\mathfrak{der}}}
\DeclareMathOperator{\SL}{SL}
\DeclareMathOperator{\Graph}{Graph}
\DeclareMathOperator{\Int}{Int}
\DeclareRobustCommand{\intAlg}{\mathop{\mathfrak{int}}}
\DeclareMathOperator{\aut}{aut}
\DeclareMathOperator{\Rad}{Rad}
\DeclareMathOperator{\Nil}{Nil}
\DeclareMathOperator{\rad}{rad}
\DeclareMathOperator{\nil}{nil}
\DeclareMathOperator{\Ric}{Ric}
\DeclareMathOperator{\ric}{ric}
\newcommand{\bi}{\mathrm{bi}}
\DeclareMathOperator{\Isom}{Isom}
\DeclareMathOperator{\Sym}{Sym}
\newcommand{\opL}{\ensuremath{\mathrm{L}\mkern-0.6mu}}
% ======================================================================================

%-- Deklaration von Operatoren (Funktionalanalysis)
% ======================================================================================
\DeclareMathOperator{\tr}{tr}
\newcommand{\w}{\mkern1mu\mathrm{w}}
\newcommand{\sa}{\mathrm{sa}}
\newcommand{\vb}{\mathrm{v\mkern-2.5mu.b\mkern-1.5mu.}} % vollständig beschränkt
\newcommand{\so}{\mathrm{\mkern.3mu s\mkern-1.4mu.\mkern-.6mu o\mkern-1.7mu.}} % \newcommand{\so}{\mathrm{s.o.}}
\newcommand{\solim}{\so\text{-}\mkern-0.8mu\lim}
\newcommand{\wo}{\mathrm{w\mkern-3mu.\mkern-.4mu o\mkern-1.7mu.}}
\newcommand{\Top}[1]{\mathcal{T}_{\mkern-2.3mu #1}}
\newcommand{\weakT}[1]{\ensuremath{\mathcal{T}_{#1}^{\mkern+1.0mu\text{\raisebox{0.4ex}{$\mathrm{w}$}}}}}
\newcommand{\weakTstar}[1]{\ensuremath{\mathcal{T}_{#1}^{\mkern+1.0mu\text{\raisebox{0.4ex}{$\mathrm{w}$}}^*}}}
\newcommand{\TWeakStar}{\Top{\w^*}}
\newcommand{\TWeakOp}{\Top{\wo}}
\newcommand{\Tso}{\Top{\so}}
\newcommand{\finSub}{\subset\mkern-0.7mu \subset}
\DeclareMathOperator{\Inv}{Inv}
\newcommand{\simm}{{\hspace{-1.6pt}\raisemath{0.5pt}{\sim}}}
\newcommand{\plus}{{\hspace{-1.6pt}+}}
\DeclareMathOperator{\ev}{ev}
\DeclareMathOperator{\Alg}{Alg}
\DeclareMathOperator{\her}{her}
\newcommand{\subher}{\subset_{\her}}
\newcommand{\grenzw}[1]{\xrightarrow{\minwidthbox{#1}{1.4em}}}
\newcommand{\grenzwl}[1]{\xleftarrow{\minwidthbox{#1}{1.4em}}}
\newcommand{\grenzwIn}[1]{\grenzw{\raisemath{-2pt}{#1}}}
\newcommand{\MyTo}[1]{\tikzexternaldisable\mathbin{\tikz[baseline] \draw[-to,line width=.4pt] (0ex,0.94ex) -- (#1,0.94ex);}\tikzexternalenable}
\newcommand{\dlim}{%
    \mathchoice
      {\lim\limits_{\MyTo{4.2ex}}}% \displaystyle
      {\lim\limits_{\MyTo{2.8ex}}}% \textstyle
      {\lim\limits_{\MyTo{2.3ex}}}% \scriptstyle
      {\lim\limits_{\MyTo{2.3ex}}}% \scriptscriptstyle
}
\newcommand{\Dlim}{\killDescendersM{\dlim}}
\DeclareMathOperator{\sep}{sep}
\DeclareMathOperator{\diam}{diam}
\DeclareMathOperator{\conv}{conv}
\DeclareMathOperator{\Prim}{Prim}
\DeclareMathOperator{\hull}{hull}
\DeclareMathOperator{\red}{red}
\DeclarePairedDelimiterX\bra[1]{\langle}{\rvert}{#1\,}
\DeclarePairedDelimiterX\ket[1]{\lvert}{\rangle}{\,#1}
\DeclarePairedDelimiterX\bracket[2]{\langle}{\rangle}{#1\,\delimsize\vert\,#2}
\newcommand{\tensormax}{\mathbin{\otimes_{\max}}}
\newcommand{\tensormin}{\mathbin{\otimes_{\min}}}
\DeclareMathOperator{\Ped}{Ped}
\newcommand{\alg}{\mathrm{alg}}
\DeclareMathOperator{\CPC}{CPC}
\DeclareMathOperator{\CP}{CP}
\DeclareMathOperator{\UPC}{UPC}
\newcommand{\DeltaOp}{\mathbin{\Delta}}
\newcommand{\kernedP}{\mathcal{P}\mkern-2mu}
\newcommand{\Pinfty}{\kernedP_{\infty}}
\DeclareMathOperator{\Groth}{Groth}
\DeclareMathOperator{\rk}{rk}
\newcommand{\MvN}{\mathrm{MvN}}
% ======================================================================================

%-- Kategorien
% ======================================================================================
\DeclareMathOperator{\Mor}{Mor}
\DeclareMathOperator{\mor}{mor}
\DeclareMathOperator{\Obj}{Obj}
\DeclareMathOperator{\Ob}{Ob}
\newcommand{\TOP}{\textsc{Top}}
\newcommand{\HTOP}{\textsc{HTop}}
\newcommand{\VR}{\textsc{VR}}
\newcommand{\MOD}{\textsc{Mod}}
\newcommand{\Mod}[1]{#1\text{-}\MOD}
\newcommand{\MONOIDE}{\textsc{Monoide}}
\newcommand{\SET}{\textsc{Set}}
\newcommand{\MAN}{\textsc{Man}}
\newcommand{\GRUPPEN}{\textsc{Gruppen}}
\newcommand{\ABELGRUPPEN}{\textsc{Abel.Gruppen}}
\newcommand{\ABEL}{\textsc{Abel}}
\newcommand{\KAT}{\textsc{Kat}}
\newcommand{\FUN}{\textsc{Fun}}
\newcommand{\SIMP}{\textsc{Simp}}
\newcommand{\VEKT}{\textsc{Vekt}}
\newcommand{\CH}{\textsc{Ch}}
\newcommand{\CSTARUN}{C^*\text{-}\textsc{Alg}^{\raisemath{-2.5pt}{1}}}
\newcommand{\CSTAR}{C^*\text{-}\textsc{Alg}}
\newcommand{\AB}{\textsc{Ab}}
% ======================================================================================
% ======================================================================================



% -- theorem packages
% ======================================================================================
\usepackage{amsthm}
\usepackage{thmtools,thm-restate}
\usepackage{mdframed}
\renewcommand{\listtheoremname}{Übersicht aller Aussagen}
\usepackage{bookmark}
\bookmarksetup{open,numbered}
\makeatletter
\newcommand*{\theorembookmark}{%
  \bookmark[
    dest=\@currentHref,
    rellevel=1,
    keeplevel,
  ]{%
    \thmt@thmname\space\csname the\thmt@envname\endcsname
    \ifx\thmt@shortoptarg\@empty
    \else
      \space(\thmt@shortoptarg)%
    \fi
  }%
}   
\makeatother
% ======================================================================================

% -- Definition der einzelnen Theorem-Umgebungen
% ======================================================================================
\declaretheoremstyle[%
	headfont=\sffamily\bfseries,
	notefont=\normalfont\sffamily\scshape,
	bodyfont=\normalfont,
	headformat=\NUMBER\ \NAME\NOTE,
	headpunct=.,
	postheadspace=1em,
	spaceabove=15pt,spacebelow=10pt,
	shaded={bgcolor=gray!20},
	postheadhook=\theorembookmark]%
{mainstyle}
\declaretheoremstyle[%
	headfont=\sffamily\bfseries,
	notefont=\normalfont\sffamily\scshape,
	bodyfont=\normalfont,
	headformat=\NUMBER\ \NAME\NOTE,
	headpunct=.,
	postheadspace=1em,
	spaceabove=15pt,spacebelow=10pt,
	shaded={bgcolor=fb10_blue!20},
	postheadhook=\theorembookmark]%
{mainstyle_blue}
\declaretheoremstyle[%
	headfont=\sffamily\bfseries,
	notefont=\normalfont\sffamily\scshape,
	bodyfont=\normalfont,
	headformat=\NUMBER\ \NAME\NOTE,
	headpunct=.,
	postheadspace=1em,
	spaceabove=15pt,spacebelow=10pt,
	postheadhook=\theorembookmark]%
{mainstyle_unshaded}
\declaretheoremstyle[%
	headfont=\sffamily\bfseries,
	notefont=\normalfont\sffamily\scshape,
	bodyfont=\normalfont,
	headformat=\NUMBER\NAME\NOTE,
	headpunct=.,
	postheadspace=1em,
	spaceabove=15pt,spacebelow=10pt,
	% shaded={bgcolor=gray!20},
	postheadhook=\theorembookmark]%
{mainstyle_unnumbered}
\declaretheoremstyle[%
	headfont=\sffamily\bfseries,
	notefont=\normalfont\sffamily\scshape,
	bodyfont=\normalfont,
	headformat=swapnumber,
	headpunct=.,
	postheadspace=1em,
	spaceabove=15pt,spacebelow=10pt,
	shaded={bgcolor=gray!20},
	postheadhook=\theorembookmark,
	qed=\qedsymbol]%
{mainstyleB}
\declaretheoremstyle[%
	headfont=\bfseries\scshape,
	bodyfont=\normalfont,
	headpunct=:,
	postheadspace=1em,
	spacebelow=12pt,spaceabove=2pt,
	qed=\qedsymbol]%
{beweise}
\declaretheoremstyle[%
	headfont=\bfseries\scshape,
	bodyfont=\normalfont,
	headpunct=:,
	postheadspace=1em,
	spacebelow=12pt,spaceabove=2pt]%
{beweisskizze}
\declaretheoremstyle[%
	headfont=\sffamily\bfseries,
	bodyfont=\normalfont,
	headpunct=:,
	postheadspace=1em,
	spacebelow=10pt,spaceabove=10pt]%
{bemerkungen}
\declaretheorem[name=Definition,parent=section,style=mainstyle_blue]{definition}
\declaretheorem[name=Definition \& Proposition,refname=Proposition,sharenumber=definition,style=mainstyle_blue]{definitionP}
\declaretheorem[name=Definition,numbered=no,style=mainstyle_unnumbered]{definition*}
\declaretheorem[name=Theorem,sharenumber=definition,style=mainstyle]{theorem}
\declaretheorem[name=Theorem,numbered=no,style=mainstyle_unnumbered]{theorem*}
\declaretheorem[name=Proposition,sharenumber=definition,style=mainstyle,refname=Proposition]{proposition}
\declaretheorem[name=Lemma,sharenumber=definition,style=mainstyle]{lemma}
\declaretheorem[name=Satz,sharenumber=definition,style=mainstyle,refname=Satz]{satz}
\declaretheorem[name=Satz,sharenumber=definition,style=mainstyle_unshaded]{satzUnshaded}
\declaretheorem[name=Definition,sharenumber=definition,style=mainstyle_unshaded]{definitionUnshaded}
\declaretheorem[name=Satz,numbered=no,style=mainstyle_unnumbered]{satz*}
\declaretheorem[name=Korollar,sharenumber=definition,style=mainstyle,refname=Korollar]{korollar}
\declaretheorem[name=Korollar,sharenumber=definition,style=mainstyleB,refname=Korollar]{korollarB}
\declaretheorem[name=Frage,numbered=no,style=mainstyle_unnumbered]{frage}
\declaretheorem[name=Frage,sharenumber=definition,style=mainstyle_unshaded]{frageA}
\declaretheorem[name=Erinnerung,sharenumber=definition,style=mainstyle_unshaded]{erinnerungA}
\declaretheorem[name=Ausblick,sharenumber=definition,style=mainstyle_unshaded]{ausblick}
\declaretheorem[name=Konvention,sharenumber=definition,style=mainstyle]{konvention}
\declaretheorem[name=Notation,sharenumber=definition,style=mainstyle_unshaded]{notation}
\declaretheorem[name=Bemerkung,sharenumber=definition,style=mainstyle_unshaded,refname=Bemerkung]{bemerkung}
\declaretheorem[name=Bemerkung,numbered=no,style=mainstyle_unnumbered]{bemerkung*}
\declaretheorem[name=Beispiel,sharenumber=definition,style=mainstyle_unshaded,refname=Beispiel]{beispiel}
\declaretheorem[name=Beispiel,numbered=no,style=mainstyle_unnumbered]{beispiel*}
\declaretheorem[name=Exkurs,numbered=no,style=mainstyle_unnumbered]{exkurs*}
\declaretheorem[name=Beweis,numbered=no,style=beweise]{beweis}
\declaretheorem[name=Übung,numbered=no,style=bemerkungen]{uebung}
\declaretheorem[name=Erinnerung,numbered=no,style=bemerkungen]{erinnerung}

% english versions
\declaretheorem[name=Remark,sharenumber=definition,style=mainstyle_unshaded]{remark}
\declaretheorem[name=Remark,numbered=no,style=mainstyle_unnumbered]{remark*}
\declaretheorem[name=Example,sharenumber=definition,style=mainstyle_unshaded]{example}
\declaretheorem[name=Corollary,sharenumber=definition,style=mainstyle]{corollary}
\let\proof\relax
\declaretheorem[name=Proof,numbered=no,style=beweise]{proof}
\declaretheorem[name=Sketch of Proof,numbered=no,style=beweisskizze]{sketch}
% ======================================================================================

%--Inhaltsverzeichnis
% ======================================================================================
\usepackage[tocindentauto]{tocstyle}
\usetocstyle{KOMAlike}
% ======================================================================================

%-- Dinge, die erst am Ende getan werden dürfen
% ======================================================================================
\shorthandon{"}
\usepackage{ellipsis}
% ======================================================================================


\newcommand{\fach}{Algebraische $K$-Theorie}
\newcommand{\semester}{Sose 2016}
\newcommand{\homepage}{https://wwwmath.uni-muenster.de/reine/u/topos/lehre/WS2015-2016/Topologie2/}

\newcommand{\prof}{Prof.\ Dr.\ Arthur Bartels}
\publishers{\scalebox{11}{\Huge$K^{\raisebox{2.7pt}{\text{\small$\mathrm{alg}$}}}$}}
\input{../!config/mitschrift_headings.tex}

\begin{document}
\pagenumbering{Roman}
\maketitle
\begin{abstract}
\section*{Aktuelle Version verfügbar bei}
\newcommand{\dieBreite}{11cm}
\begin{minipage}{4cm}
	\qrcode[height=3.3cm, version=6]{https://gitlab.com/JaMeZ-B/LaTeX-WWU}
\end{minipage}
\hfill
\begin{minipage}{\dieBreite}
	% \includegraphics[height=0.6cm, keepaspectratio]{../!config/Bilder/wm_no_bg.pdf}
	\includegraphics[height=0.8cm, keepaspectratio]{../!config/Bilder/wm_no_bg.pdf}\\
	\url{https://gitlab.com/JaMeZ-B/LaTeX-WWU} \smallskip\\
	Das zentrale Repository des \enquote{\LaTeX-WWU}-Projekts befindet sich auf der Plattform GitLab.com.
	Neben der Koordination aller Beteiligten werden über diesen Dienst auch die PDFs gebaut, die in der Readme verlinkt sind.
\end{minipage}\\[1cm]
\begin{minipage}{4cm}
	\qrcode[height=3.3cm, version=6]{https://github.com/JaMeZ-B/latex-wwu}
\end{minipage}
\hfill
\begin{minipage}{\dieBreite}
	\includegraphics[height=0.6cm, keepaspectratio]{../!config/Bilder/github_octo.pdf}
	\includegraphics[height=0.6cm, keepaspectratio]{../!config/Bilder/GitHub_Logo.pdf}\\
	\url{https://github.com/JaMeZ-B/latex-wwu} \smallskip\\
	Die Entwicklung des \enquote{\LaTeX-WWU}-Projekts hat ursprünglich auf GitHub stattgefunden, ist mittlerweile aber zu GitLab gewechselt.
	Das GitHub-Repository wird stündlich automatisch aktualisiert, Merge-Requests werden aber nicht mehr entgegengenommen.
\end{minipage}\\[1cm]
% \begin{minipage}{4cm}
% 	\qrcode[height=3.3cm, version=6]{https://uni-muenster.sciebo.de/public.php?service=files&t=965ae79080a473eb5b6d927d7d8b0462}
% \end{minipage}
% \hfill
% \begin{minipage}{\dieBreite}
% 	\raisebox{-2pt}{\includegraphics[height=0.6cm, keepaspectratio]{../!config/Bilder/sciebo_logo.pdf}}
% 	\resizebox{!}{0.5cm}{\large \sffamily\textbf{sciebo}} {\sffamily\large die Campuscloud} \\
% 	\resizebox{\dieBreite}{!}{\footnotesize\url{https://uni-muenster.sciebo.de/public.php?service=files&t=965ae79080a473eb5b6d927d7d8b0462}}\smallskip\\
% 	Sciebo ist ein Dropbox-Ersatz der Hochschulen in NRW, der von der Uni Münster in leitender Position auf Basis der OpenSource-Software Owncloud aufgebaut wurde.
% \end{minipage}\\[1cm]
\hrule \mbox{ }\\[0.7cm]
\begin{minipage}{4cm}
	\qrcode[height=3.3cm, version=6]{\homepage}
\end{minipage}
\hfill
\begin{minipage}{\dieBreite}
	\resizebox{!}{0.5cm}{\large\sffamily\textbf{Vorlesungshomepage}}\\
	\resizebox{\dieBreite}{!}{\footnotesize\url{\homepage}}\smallskip\\
	Hier ist ein Link zur offiziellen Vorlesungshomepage.
\end{minipage}
\newpage
\section*{Vorwort --- Mitarbeit am Skript}
Dieses Dokument ist eine Mitschrift aus der Vorlesung \enquote{\fach, \semester}, gelesen von \prof. 
Der Inhalt entspricht weitestgehend dem Tafelanschrieb. 
Für die Korrektheit des Inhalts übernehme ich keinerlei Garantie! 
Für Bemerkungen und Korrekturen -- und seien es nur Rechtschreibfehler -- bin ich sehr dankbar. 
Korrekturen lassen sich prinzipiell auf drei Wegen einreichen: 
\begin{itemize}
	\item Persönliches Ansprechen in der Uni, Mails an \hrefsymmail{mailto:\mail}{\mail} (gerne auch mit annotieren PDFs) oder Kommentare auf \url{https://gitlab.com/JaMeZ-B/LaTeX-WWU}.
	\item \emph{Direktes} Mitarbeiten am Skript: Den Quellcode poste ich auf GitLab (siehe oben), also stehen vielfältige Möglichkeiten der Zusammenarbeit zur Verfügung:
	Zum Beispiel durch Kommentare am Code über die Website und die Kombination Fork und Merge-Request. 
	Wer sich verdient macht oder ein Skript zu einer Vorlesung, die ich nicht besuche, beisteuern will, dem gewähre ich gerne auch Schreibzugriff.
	
	Beachten sollte man dabei, dass dazu ein Account bei \url{gitlab.com} notwendig ist, der allerdings ohne Angabe von persönlichen Daten angelegt werden kann. 
	Wer bei GitLab (bzw. dem zugrunde liegenden Open-Source-Programm \enquote{\texttt{git}}) -- verständlicherweise -- Hilfe beim Einstieg braucht, dem helfe ich gerne weiter. 
	Es gibt aber auch zahlreiche empfehlenswerte Tutorials im Internet.\footnote{zB. \url{https://try.github.io/levels/1/challenges/1}, ist auf Englisch, aber dafür interaktiv}
	\item \emph{Indirektes} Mitarbeiten: \TeX-Dateien per Mail verschicken. 
	
	Dies ist nur dann sinnvoll, wenn man einen ganzen Abschnitt ändern möchte (zB. einen alternativen Beweis geben), da ich die Änderungen dann per Hand einbauen muss! Ich freue mich aber auch über solche Beiträge!
\end{itemize}
\section*{Literatur}
\begin{itemize}
	\item \citetitle{Rosenberg} von J. \textsc{\citeauthor{Rosenberg}} \cite{Rosenberg}
	\item \citetitle{Weibel} von Charles \textsc{\citeauthor{Weibel}} \cite{Weibel}
	\item \citetitle{MilnorKtheory} von John \textsc{\citeauthor{MilnorKtheory}} \cite{MilnorKtheory}
\end{itemize}
\end{abstract}

\tableofcontents
\cleardoubleoddemptypage

\pagenumbering{arabic}
\setcounter{page}{1}
\setcounter{footnote}{0}

\section{$K_0$ eines Ringes} % (fold)
\label{sec:1}

\begin{definition}[{name=[{projektiv}]}]
	Sei $R$ ein Ring.
	Ein $R$-Modul $P$ heißt \Index{projektiv}, falls er folgende Eigenschaft hat:
	Sei $f \colon M \twoheadrightarrow P$ $R$-linear und surjektiv. 
	Dann gibt es einen Spalt $s \colon P \to M$, das heißt eine $R$-lineare Abbildung $s$ mit $f \circ s = \id_P$.
\end{definition}

\begin{bemerkung}[{name=[{Äquivalenzen zu Projektivität}]}]
	Die zwei folgenden Bedingungen sind äquivalent zur Projektivität von $P$:
	\begin{enumerate}[1)]
		\item Sind $\varphi \colon P \to N$, $\psi \colon M \twoheadrightarrow N$ $R$-linear und $\psi$ surjektiv, so gibt es $\hat{\varphi} \colon P \to M$ mit $\psi \circ \hat{\varphi} = \varphi$.
		\[
			\begin{tikzcd}
				& M \dar[two heads,"\psi"] \\
				P \rar["\varphi"] \urar["\hat{\varphi}",dashed] & N
			\end{tikzcd}
		\]
		\item $P$ ist ein direkter Summand in einem freien Modul, das heißt es gibt einen $R$-Modul $Q$, sodass $P \oplus Q$ eine $R$-Basis besitzt.
	\end{enumerate}
\end{bemerkung}

\begin{definition}[{name=[{endlich erzeugter $R$-Modul}]}]
	Ein $R$-Modul heißt \bet{endlich erzeugt}\index{endlich erzeugter $R$-Modul}, wenn es eine endliche Teilmenge $S \subseteq M$ gibt, sodass $M$ der einzige Untermodul von $M$ ist, der $S$ enthält.
	Wir nennen $S$ ein \Index{endliches Erzeugendensystem} für $M$.
\end{definition}

\begin{bemerkung}[{name=[{endlich erzeugte freie Moduln}]}]
	Ein freier $R$-Modul ist genau dann endlich erzeugt, wenn er eine endliche Basis besitzt.
	Er ist dann isomorph zu $R^n$ für geeignetes $n \in \mathbb{N}$.\marginnote{dieses $n$ muss nicht eindeutig sein!}
\end{bemerkung}

\begin{satz}[{name=[{Äquivalenzen zur Projektivität endlich erzeugter Moduln}]},label=satz:proj_endlich_erz]
	Sei $P$ ein endlich erzeugter $R$-Modul.
	Dann sind äquivalent:
	\begin{enumerate}[1)]
		\item $P$ ist projektiv.
		\item $P$ ist direkter Summand in einem $R^n$.
		\item Es gibt eine $R$-lineare idempotente\marginnote{$p^2=p$} Abbildung $p \colon R^n \to R^n$ mit $P \cong \im p$.
	\end{enumerate}
\end{satz}
\begin{beweis}
	Sei zunächst $P$ projektiv.
	Ist $v_1,\ldots ,v_n$ ein endliches Erzeugendensystem von $P$, so erhalten wir $R$-lineare surjektive Abbildung $f \colon R^n \to P$, $f(r_1,\ldots,r_n)=\sum_{i=1}^n r_i \cdot v_i$.
	Da $P$ projektiv ist, gibt es einen Spalt $s$ für $f$ und es folgt $R^n \cong P \oplus \ker f$.
	
	Sei nun $\varphi \colon P \oplus Q \to R^n$ ein Isomorphismus.
	Sei $\tilde{p} \colon P \oplus  Q \to P \oplus Q$ die Projektion auf $P$, also $\tilde{p} (v,w) = (v,0)$ für $v \in P$, $w \in Q$.
	Dann ist $p := \varphi \circ \tilde{p} \circ \varphi^{-1} \in \End_R(R^n)$ idempotent und $\varphi|_{P} \colon P \to \im p$ ein Isomorphismus.
	
	Sei $p \colon R^n \to R^n$ idempotent.
	Wir zeigen, dass $\im p = p(R^n)$ projektiv ist.
	Sei $f \colon M \twoheadrightarrow p(R^n)$ $R$-linear und surjektiv.
	Dann gibt es $v_1,\ldots ,v_n \in M$ mit $f(v_i)= p(e_i)$.
	Sei $\hat{s} \colon R^n \to M$ gegeben durch $\hat{s}(r_1,\ldots ,r_n)=\sum_{i=1}^{n} r_i \cdot v_i$.
	Dann ist $f \circ \hat{s} =p$ und daher gilt für $w =p(w) \in \im p$
	\[
		f \enbrace[\big]{\hat{s}(w)} =p(w)=w
	\]
	Also ist $s:= \hat{s}|_{p(R^n)}$ der gesuchte Spalt.
\end{beweis}

\begin{bemerkung}[{name=[{direkte Summanden des $R^n$ sind endlich erzeugt}]}]
	Ist $P \subseteq R^n$ ein direkter Summand, so ist $P$ endlich erzeugt, da wir eine Surjektion $R^n \to P$ erhalten und das Bild der endlichen Basis der $R^n$ dann auch ein endliches Erzeugendensystem für $P$ ist.
\end{bemerkung}

\begin{bemerkung}[{name=[{Menge der Isomorphieklassen von endlich erzeugten projektiven Moduln}]}]
	Für jedes $n \in \mathbb{N}$ bilden die Untermoduln von $R^n$ eine Menge, insbesondere bilden die direkten Summanden eine Menge.
	Es folgt mit \autoref{satz:proj_endlich_erz}, dass auch die Isomorphieklassen von endlich erzeugten projektiven $R$-Moduln eine Menge bilden.
	Diese nennen wir $\Proj(R)$.
\end{bemerkung}

\begin{bemerkung}[{name=[{Proj(R) als abelsche Halbgruppe}]}]
	$\Proj(R)$ wird durch $\oplus$ zu einer abelschen Halbgruppe mit der Isomorphieklasse des Nullmoduls als neutralem Element.
\end{bemerkung}

\begin{satz}[name={Grothendiek-Konstruktion}]
	Sei $S$ eine abelsche Halbgruppe.
	Dann gibt es eine abelsche Gruppe $\Gr(S)$ zusammen mit einem Halbgruppenhomomorphismus $\varphi \colon S\to \Gr(S)$, der folgende universelle Eigenschaft erfüllt:
	Ist $A$ eine abelsche Gruppe und $\psi \colon S \to A$ ein Halbgruppenhomomorphismus, so gibt es einen eindeutigen Gruppenhomomorphismus $\hat{\psi} \colon \Gr(S) \to A$ mit $\psi = \hat{\psi} \circ \varphi$.
	\[
		\begin{tikzcd}[column sep=3em]
			S \rar["\varphi"] \drar["\psi"'] & \Gr(S) \dar[dashed,"\hat{\psi}"] \\
			& A
		\end{tikzcd}
	\]
\end{satz}
\begin{beweis}
	Wir nehmen an, dass $S$ ein neutrales Element besitzt.
	Wir definieren $\Gr(S)$ wie folgt: Auf $S \times S$ betrachten wir die Äquivalenzrelation 
	\[
		(s,t) \sim (s',t') :\iff \exists x \in S : s +t' + x = s'+t +x
	\] 
	und setzen $\Gr(S) := \sfrac{S\times S}{\sim}$.
	Wir schreiben $s-t$ für die Äquivalenzklasse von $(s,t)$.
	Die Addition auf $\Gr(S)$ ist durch $(s-t) + (s'-t') = s+s'  - (t+t')$ gegeben, das Inverse zu $s-t$ ist $t-s$.
	Der Halbgruppenhomomorphismus $\varphi \colon S \to \Gr(S)$ ist definiert durch $\varphi(s)= s - 0$.
	Ist $\psi \colon S \to A$ ein Halbgruppenhomomorphismus, so erhalten wir $\hat{\psi} \colon \Gr(S) \to A$ durch $\hat{\psi}(s-t) = \psi(s)- \psi(t)$.
\end{beweis}

\begin{definition}[{name=[{Grothendiek-Gruppe}]}]
	$\Gr(S)$ heißt die \Index{Grothendiek-Gruppe} oder \Index{Gruppenvervollständigung} von $S$.  
\end{definition}

\begin{bemerkung}[{name=[alternative Konstruktion mit freier abelscher Gruppe]}]
	Alternativ können wir auch die \Index{freie abelsche Gruppe} $\mathbb{Z}[S]$ benutzen, um $\Gr(S)$ zu konstruieren.
  Setze dazu
	\[
		\Gr(S) := \sfrac{\mathbb{Z}[S]}{A}
	\]
	wobei $A$ von $\set[\big]{s + t-(s +_S t) \given s,t \in S}$ erzeugt wird.
\end{bemerkung}

\begin{beispiel}[{name=[{für Grothendiek-Konstruktionen}]}]
	Es gilt
	\begin{multicols}{2}
		\begin{itemize}
			\item $\Gr(\mathbb{N},+) \cong \mathbb{Z}$
			\item $\Gr(\mathbb{N}_{\ge 17},+) \cong \mathbb{Z}$
			\item $\Gr(\mathbb{N}_{>0},\cdot ) \cong (\mathbb{Q}_{>0},\cdot )$
			\item $\Gr(\mathbb{N}_{\ge 0},\cdot) \cong 0$ (mit $x=0$)
			\item $\Gr(\mathbb{N} \cup \set*{\infty},+) \cong 0$
		\end{itemize}
	\end{multicols}
\end{beispiel}

\begin{bemerkung}[{name=[{Funktorialität}]}]
	$S \mapsto \Gr(S)$ ist ein Funktor von der Kategorie der abelschen Halbgruppen in die Kategorie der abelschen Gruppen:
	Ist $\psi \colon S \to S'$ ein Halbgruppenhomomorphismus, so ist $\Gr(\psi) \colon \Gr(S) \to \Gr(S')$, $s-t \mapsto \psi(s) - \psi(t)$ der induzierte Gruppenhomomorphismus.
  Dieser induzierte Gruppenhomomorphismus lässt sich auch mit der universellen Eigenschaft konstruieren.
\end{bemerkung}

\begin{definition}[{name=[{$K_0$ von Ringen}]},label=def:K0_ring]
	Sei $R$ ein Ring. 
	Wir definieren
	\[
		K_0(R) := \Gr \enbrace[\big]{\Proj(R),\oplus }
	\]
	Elemente in $K_0$ sind also formale Differenzen $[P]- [Q]$ von Isomorphieklassen von endlich erzeugten projektiven Moduln.
\end{definition}

Es gilt $[P] - [Q] = [P']- [Q']$ genau dann, wenn es einen endlich erzeugten projektiven Modul $X$ gibt mit
\[
	P \oplus Q' \oplus X \cong P' \oplus Q \oplus X
\]
Da jedes solche $X$ direkter Summand in einem endlich erzeugten freien Modul ist, genügt es für $X$ endlich erzeugte freie Moduln zu betrachten.
Insbesondere ist $[P]=0 \in K_0(R)$ genau dann, wenn es $n \in \mathbb{N}$ gibt mit $P \oplus R^n \cong R^n$.\todo{RevChap 1}

\begin{bemerkung}[{name=[{Funktoreigenschaften der 0-ten K-Theorie}]}]
	$K_0(R)$ ist ein Funktor: Ist $f \colon R \to R'$ ein Ringhomomorphismus, so erhalten wir eine induzierte Abbildung $\Proj(R) \to \Proj(R')$ durch $_R P \mapsto  R' \otimes_R {_R P}$, dabei wird $R'$ durch $f$ zu einem $R$-Rechtsmodul:
	\[
		r' \cdot r := r' \cdot f(r)
	\]
	Es ist nun $K_0(f) \colon K_0(R) \to K_0(R')$ gegeben durch 
	\(
		[P]- [Q] \longmapsto \benbrace*{R' \otimes_R P} - \benbrace*{R' \otimes_R Q}
	\).
\end{bemerkung}

\begin{bemerkung}[{name=[{kontravarianter Funktor}]}]
	Ist $f \colon R \to R'$ ein Ringhomomorphismus und $M'$ ein $R'$-Modul, so wird $M'$ zu einem $R$-Modul durch $r \cdot m' := f(r) \cdot m'$.
	Manchmal bezeichnet wir diesen $R$-Modul mit $\res_f M'$ oder $f^* M'$.
	Der Funktor $f^* \colon R'\text{-}\MOD \to R\text{-}\MOD$ erhält aber nicht notwendig endlich erzeugte bzw. projektive Moduln.
	Betrachte dazu die Inklusion $\mathbb{Z} \hookrightarrow \mathbb{Q}$. 
	Als $\mathbb{Q}$-Modul ist $\mathbb{Q}$ frei, als $\mathbb{Z}$-Modul nicht endlich erzeugt.
	
	Manchmal fasst man $f$ als eine Abbildung $f \colon \Spec R' \to \Spec R$ auf und schreibt $f_*$ für obigen Funktor.
\end{bemerkung}

\begin{satz}[{name=[{K0 von einem Körper}]}]
	Sei $F$ ein Körper.
	Dann erhalten wir einen Isomorphismus $\dim K_0(F) \to \mathbb{Z}$ mit 
	\[
		\dim\enbrace[\big]{[P]- [Q]} = \dim_F P - \dim_F Q
	\]
\end{satz}
\begin{beweis}
	Sei $[P]- [Q] = [P']- [Q']$ in $K_0(R)$.
	Dann gibt es einen endlich erzeugten Modul $V$ mit $P \oplus Q' \oplus V \cong P' \oplus Q \oplus V$.
	Insbesondere gilt 
	\[
		\dim P + \dim Q' + \dim V = \dim P' + \dim Q + \dim V,
	\]
	also $\dim P - \dim Q = \dim P' - \dim Q'$ und somit ist die Abbildung $\dim$ wohldefiniert auf $K_0(R)$.
	Wegen $\dim \enbrace*{[F^n]- [F^m]} = n-m$ ist $\dim$ surjektiv.
	
	Sei nun $[P]-[Q] \in K_0(R)$ mit $\dim \enbrace*{[P]- [Q]}=0$.
	Dann ist $\dim P - \dim Q =0$, also $P \cong Q$ und $[P]=[Q]$ bzw. $[P]-[Q]=0$.
\end{beweis}

\begin{definition}
	Sei $\iota \colon \mathbb{Z} \to K_0(R)$ der durch $n \mapsto \benbrace*{R^n}$ induzierte Gruppenhomomorphismus.
	Wir definieren 
	\[
		\tilde{K}_0(R) := \sfrac{K_0(R)}{\iota(\mathbb{Z})}
	\]
\end{definition}

\begin{bemerkung}
	\leavevmode
	\begin{enumerate}[1)]
		\item Im Allgemeinen ist $\iota$ weder injektiv noch surjektiv.
		\item $[P]=0 \in \tilde{K}_0(R)$ ist äquivalent zu $P$ ist \Index{stabil endlich erzeugt frei}, das heißt es gibt einen endlich erzeugten freien $R$-Modul $F$, sodass $F \oplus P$ frei ist.
		Insbesondere 
		\[
			\tilde{K}_0(R) =0 \iff \text{Jeder endlich erzeugte projektive Modul ist stabil endlich erzeugt frei} 
		\]
		\item Sei $R$ ein Ring, für den jeder endlich erzeugte projektive Modul frei von wohldefiniertem Rang ist, also $P \cong R^n$ für eindeutiges $n$.
		Dann definiert der Rang -- wie bei Körpern -- ein Inverses zu $\iota$ und wir erhalten $K_0(R)\cong \mathbb{Z}$.
		
		(Dies gilt zum Beispiel für Hauptidealringe oder lokale Ringe.)
		\item Gilt $R^n \cong R^m$ genau dann, wenn $n=m$ (also also wenn der Rang für endlich erzeugte freie Moduln wohldefiniert ist), so ist $\iota$ injektiv.
		Dies gilt zum Beispiel für kommutative Ringe oder Gruppenringen über kommutativen Ringen. 
	\end{enumerate}
\end{bemerkung}

\begin{beispiel}[{name={Eilenberg-Schwindel}}]
	Betrachte die Menge $\Proj^\infty(R)$ der Isomorphieklassen von projektiven $R$-Moduln mit einem abzählbaren Erzeugendensystem.
	Dann ist $\enbrace*{\Proj^\infty(R), \oplus }$ eine abelsche Halbgruppe.
	Dann ist aber $\Gr\enbrace*{\Proj^\infty(R),\oplus}=0$, denn zu jedem solchen Modul $P$ gibt es einen zweiten $Q$ mit $P \oplus Q \cong Q$, nämlich $Q := \bigoplus_{i=1}^\infty P$ ist auch projektiv und abzählbar erzeugt.\marginnote{man kann $Q$ auch frei wählen}
\end{beispiel}

\begin{definition}[{name=[Morita-äquivalent]}]
	Ringe $R$ und $S$ heißen \Index{Morita-äquivalent}, wenn es Bimoduln $_R X_S$ und $_S Y _R$ gibt, sodass 
	\begin{equation}
		\begin{split}
			{_R X_S} \otimes_S {_S Y _R} &\cong {_R R_R} \enspace\text{ als $R$-$R$-Bimodul} \\ 
			{_S Y _R} \otimes_R {_R X_S} &\cong {_S S_S} \enspace\text{ als $S$-$S$-Bimodul}
		\end{split} \tag{\#}\label{eq:def:morita}
	\end{equation}
\end{definition}

\begin{bemerkung}
	\eqref{eq:def:morita} impliziert, dass $_R X_S$ sowohl als $R$-Linksmodul als auch als $S$-Rechtsmodul endlich erzeugt projektiv ist,
	Gleiches gilt (umgekehrt) auch für ${_S X_R}$.
\end{bemerkung}

\begin{beispiel}
	Für jeden Ring $R$ ist $R$ Morita-äquivalent zum Ring $M_n(R)$ der $n \times n$-Matrizen über $R$ via ${_R X_{M_n(R)}}=R^n$ und $_{M_n(R)} Y_R = R^n$.
	Dabei wirkt $M_n(R)$ einmal auf $R^n$ als Spalten von links und einmal als Zeilen von rechts.
\end{beispiel}

\begin{satz}[{name=[K0 ist Morita-invariant]}]
	Sind $R$ und $S$ Morita-äquivalent, so gilt $K_0(R) \cong K_0(S)$.
\end{satz}
\begin{beweis}
	% Betrachte $\benbrace*{_R P} \mapsto \benbrace[\big]{{_S Y_R} \otimes_R {_R P}}$ ist ein Isomorphismus $\Proj(R) \to \Proj(S)$.
	% Die Inverse wird von $_R X _S$ induziert.
	Seien $x_i \in X$, $y_i \in Y$ und setze $\psi \enbrace*{\sum_{i=1}^{n} y_i \otimes x_i} = 1_S$.
	\[
		\begin{tikzcd}
			_R X \rar["\alpha"] & R^n & R^n \rar["\beta"] & X_R \\
			x \rar[mapsto] & \enbrace*{\varphi(x \otimes y_1), \ldots , \varphi(x \otimes y_n)} & (r_1, \ldots ,r_n) \rar[mapsto] & \sum_{i=1}^{n} r_i x_i
		\end{tikzcd}
	\]
	Damit ist $\beta \enbrace{\alpha(x)} = \sum_{i=1}^{n} \varphi(x \otimes y_i) x_i$.
	Betrachte nun das folgende Diagramm:\todo{Diagramm fertig machen}
	\[
		\begin{tikzcd}[sep=large]
			_R X \ar[rr,"\beta \circ \alpha"] \dar["\cong"] & & _R X \\
			{_R X} \otimes_S S \rar["\id \otimes \psi^{-1}"]  & {_R X} \otimes_S X \otimes_R X \rar["\varphi \otimes {\id}","\cong"'] & R \otimes_R X \uar["\cong"]
		\end{tikzcd}
	\]
	Damit ist $\alpha \circ \beta$ ein Isomorphismus.
\end{beweis}

\begin{korollarB}
	$K_0(R) \cong K_0 \enbrace[\big]{M_n(R)}$, ist $F$ ein Körper, so gilt $K_0 \enbrace[\big]{M_n(F)} \cong \mathbb{Z}$.
\end{korollarB}

\begin{satz}
	Sei $(\Lambda,\le)$ eine gerichtete Menge, $R \colon \Lambda \to \textsc{Ringe}$ ein Funktor.
	Dann gilt 
	\[
		K_0 \enbrace[\Big]{\colim_{\lambda \in \Lambda} R(\lambda)} \cong \colim_{\lambda \in \Lambda} K_0\enbrace[\big]{R(\lambda)}
	\]
\end{satz}
\begin{beweis}[Skizze]
	Man zeigt zunächst $\Proj \enbrace*{\colim_\Lambda R(\lambda)} = \colim_\Lambda \Proj \enbrace*{R(\lambda)}$.
	Sei $\mathcal{R} := \colim_\Lambda R(\lambda)$.
	Der entscheidende Punkt ist: Zu jedem endlich erzeugten projektiven Modul $\mathcal{R}$-Modul $\mathcal{M}$ gibt es $\lambda \in \Lambda$ und einen endlich erzeugten projektiven $R(\lambda)$-Modul $M(\lambda)$ mit $\mathcal{M} \cong \mathcal{R} \otimes_{R(\lambda)} M(\lambda)$.
	Dazu wählen wir $\tilde{p} \in M_n(\mathcal{R})$ idempotent mit $\mathcal{M} \cong \im \tilde{p}$.
	Dann gibt es $\lambda \in \Lambda$ mit $p(\lambda) \in M_n(R(\lambda))$ mit $\tilde{p} =f_\lambda(p(\lambda))$ unter $f_\lambda \colon M_n(R(\lambda)) \to M_n(\mathcal{R})$.
	Dann ist $\mathcal{M} \cong \im \tilde{p} \cong \mathcal{R} \otimes_{R(\lambda)} \im p(\lambda)$.
	Nun zeigt man, dass auch $\Gr$ mit gerichteten Kolimiten vertauscht.
\end{beweis}

\begin{beispiel}
	Sei $F$ ein Körper.
	Betrachte den Ringhomomorphismus $i_n \colon M_{2^n}(F) \to M_{2^{n+1}}(F)$ mit $i_n(A) = \begin{psmallmatrix} A & 0 \\ 0 & A \end{psmallmatrix}$.
	Sei $R= \colim_n M_n(F)$.
	Es folgt
	\[
		K_0(R) = \colim_{n \in \mathbb{N}} K_0 \enbrace*{M_n(F)} = \colim \Big(
		\begin{tikzcd}
			\mathbb{Z} \rar["(i_1)_*"] & \mathbb{Z} \rar["(i_2)_*"] & \mathbb{Z} \rar["(i_3)_*"]  & \mathbb{Z} \rar & \ldots 
		\end{tikzcd}
		\Big)
	\]
	Für $n \in \mathbb{N}$ sei $f_n \colon K_0(F) \to$ der durch die Morita-Äquivalenz $V \mapsto  F^n \otimes_R V$ induzierte Isomorphismus.
	Dann kommutiert
	\[
		\begin{tikzcd}
			\mathbb{Z} \dar["\cdot 2"] & K_0(F) \lar["\dim"'] \dar["\cdot 2"] \rar["f_n"] & K_0 \enbrace*{M_{2^n}(F)} \dar["(i_n)_*"] \\
			\mathbb{Z} & K_0(F) \lar["\dim"'] \rar["f_{n+1}"] & K_0 \enbrace*{M_{2^{n+1}} (F)}
		\end{tikzcd}
	\]
	Also $K_0(R) = \colim \colim \big(
		\begin{tikzcd}
			\mathbb{Z} \rar["\cdot 2"] & \mathbb{Z} \rar["\cdot 2"] & \mathbb{Z} \rar["\cdot 2"]  & \mathbb{Z} \rar & \ldots 
		\end{tikzcd}
		\big) = \mathbb{Z}\benbrace*{\frac{1}{2}}$.
\end{beispiel}
% section k_0_eines_ringes (end)
\newpage

\section{$K_0$ und Dedekind-Ringe} % (fold)
\label{sec:2}
\emph{In diesem Abschnitt ist der Ring $R$ immer nullteilerfrei, kommutativ und $Q := \mathrm{Quot}(R)$ bezeichnet den Quotientenkörper von $R$.}\todo{RevChap2}

\begin{definition}[{name=[{gebrochenes Ideal}]}]
	Ein \bet{gebrochenes Ideal}\index{Ideal!gebrochenes} von $R$ ist ein $R$-Untermodul $I \neq 0$ von $Q$, zu dem es $a \in R$ gibt mit $a I \subseteq R$.
	Jedes nichttriviale Ideal von $R$ ist auch ein gebrochenes Ideal.
	Zur besseren Unterscheidung nennt man dann Ideale von $R$ auch \bet{ganze Ideale}\index{Ideal!ganzes}.
	
	Für $\frac{a}{b} \in Q$, $a,b\neq 0$ ist $\frac{a R}{b}$ ein gebrochenes Ideal.
	Wir nennen solche gebrochenen Ideale \Index{gebrochene Hauptideale}.
\end{definition}

\begin{bemerkung}[{name=[{algebraische Struktur auf Menge der gebrochenen Ideale}]}]
	Die Menge der gebrochenen Ideale bildet bezüglich Multiplikation von gebrochenen Idealen $I \cdot J := \set[\big]{\sum_{i} a_i b_i \given a_i \in I, b_i \in J}$ eine abelsche Halbgruppe. 
	Das neutrale Element ist dabei $R$.
\end{bemerkung}

\begin{definition}[{name=[{Dedekind-Ring}]}]
	$R$ heißt \Index{Dedekind-Ring}\footnote{nach Julius Wilhelm Richard \textsc{Dedekind}, * 1831 † 1916, deutscher Mathematiker}, falls die gebrochenen Ideale eine Gruppe bezüglich Multiplikation bilden. 
\end{definition}

\begin{lemma}[{name=[Invereses eines gebrochenen Ideals]}]
	Sei $R$ ein Dedekind-Ring, $I$ ein gebrochenes Ideal.
	Dann ist $I^{-1} = \set*{a \in Q \given a I \subseteq R}$.
\end{lemma}
\begin{beweis}
	Sei $J:= \set*{a \in Q \given a I \subseteq R}$.
	Wegen $I^{-1} \cdot I \subseteq R$ ist $I^{-1} \subseteq J$ und daher $R = I \cdot I^{-1} \subseteq I \cdot J \subseteq R$, also $I \cdot J =R$, also $J=I^{-1}$.
\end{beweis}

\begin{definition}[{name=[{Klassengruppe}]}]
	Die \Index{Klassengruppe} eines Dedekind-Ringes ist definiert als 
	\[
		C(R):=  \frac{\text{Gruppe der gebrochenen Ideale}}{\text{Untergruppe der gebrochenen Hauptidealringe}} 
	\]
\end{definition}

\begin{lemma}[{name=[{Klassengruppe ist trivial gdw. $R$ ein Hauptidealring ist}]}]
	$C(R)=0 \iff R$ ist Hauptidealring.
\end{lemma}
\begin{beweis}
	Sei $I \subseteq R$ ein Ideal.
	Dann ist $I$ auch ein gebrochenes Ideal und aus $C(R)=0$ folgt, dass $I$ auch ein gebrochenes Hauptideal, also $I = \frac{aR}{b}$ ist.
	Wegen $\frac{a}{b} \in I \subseteq R$ ist $I$ ein Hauptideal.
	
	Sei umgekehrt $I \subseteq Q$ ein gebrochenes Ideal.
	Wähle $b \in R$ mit $b I \subseteq R$.
	Dann ist $b \cdot I=a R$, also $I= \frac{a R}{b}$ wie gewünscht. 
\end{beweis}

\begin{satz}[label=satz:2:1]
	Sei $F$ ein \Index{Zahlkörper}, das heißt $F$ ist eine endliche Körpererweiterung von $\mathbb{Q}$.
	Sei $R$ der Ring der ganzen Zahlen in $F$, das heißt $\alpha \in F$ liegt genau dann in $R$, wenn $\alpha$ Nullstelle eines normierten Polynoms über $\mathbb{Z}$ ist.
	Dann ist $R$ ein Dedekind-Ring.
\end{satz}
\begin{beweis}
	\emph{siehe \todo{Referenz \textsc{Rosenberg}}}
\end{beweis}

Wir wollen nun einen Zusammenhang zwischen der eben eingeführten Klassengruppe und der Gruppe $K_0$ eines Ringes herstellen, die in \hyperref[def:K0_ring]{\cref*{sec:1} \autoref{def:K0_ring}} definiert wurde.
Dies leistet der folgende Satz:

\begin{satz}[label=satz:klassengruppe_K_0]
	Sei $R$ ein Dedekind-Ring.
	Dann ist die Klassengruppe isomorph zu $\tilde{K}_0(R)$.
\end{satz}

\begin{lemma}[label=lem:ideale_dedekind]
	Sei $R$ ein Dedekind-Ring.
	Dann gilt
	\begin{enumerate}[a)]
		\item Ideale in $R$ sind endlich erzeugt projektiv als $R$-Moduln.
		\item Jeder projektive endlich erzeugte $R$-Modul ist isomorph zu einer Summe von Idealen 
		\[
			I_1 \oplus \ldots \oplus I_n
		\]
		\item Ist ein Ideal $I \subseteq R$ stabil frei (als $R$-Modul), dann ist $I$ ein Hauptideal.
	\end{enumerate}
\end{lemma}
\begin{beweis}
	\begin{enumerate}[a)]
		\item Sei $I \subseteq R$ ein Ideal. 
		Da $I^{-1} \cdot I=R$ ist, gibt es $a_1, \ldots, a_n \in I^{-1}$, $b_1, \ldots ,b_n \in I$ mit $1= a_1 b_1 + \ldots + a_n b_n$.
		Weiter ist $a_i I \subseteq R$ für alle $i$.
		Betrachte nun 
		\begin{align}
			p \colon R^n \longrightarrow I &\quad \text{ mit } p(r_1,\ldots ,r_n) =r_1 b_1+ \ldots r_n b_n \\
			s \colon I \longrightarrow R^n &\quad \text{ mit } s(b) = \enbrace*{b a_1, \ldots , b a_n}  
		\end{align}
		Dann ist $p \circ s(b) = b a_1 b_1 + \ldots + b a_n b_n = b \enbrace*{a_1 b_i + \ldots + a_n b_n}= b$.
		Also ist $p \circ s =\id$ und folglich ist $I$ endlich erzeugt projektiv als $R$-Modul.
		\item Wir zeigen per Induktion nach $k$: Ist $P \subseteq R^k$, so ist $P$ isomorph zu einer Summe von Idealen.
		
		Für $k=1$ ist die Aussage klar, da Untermoduln von $R$ Ideale sind. Für den Induktionsschritt $(k-1) \mapsto k$ sei $P \subseteq R^k$.
		Die Projektion auf die letzte Koordinate liefert kurze exakte Folgen 
		\[
			\begin{tikzcd}
				R^{k-1} \rar[hook] & R^k \rar["\pi"] & R \\
				P \cap R^{k-1}\uar[phantom,sloped,"\subseteq"] \rar & P \rar \uar[phantom,sloped,"\subseteq"] &  \pi(P) \uar[phantom,sloped,"\subseteq"]
			\end{tikzcd}
		\]
		Nach a) ist $\pi(P)$ projektiv und daher spaltet die unter Folge und wir können die Induktionsvorraussetzung anwenden.
		\item Vorbemerkung: Gibt es $a \in I$ mit $a^{-1} \in I^{-1}$, so ist $I$ ein Hauptideal, genauer ist $I = R \cdot a$.
		Sei dazu $x \in I$.
		Dann ist $x = (x a^{-1}) a$.
		
		Sei nun $f \colon R^n \hookrightarrow R^m$ mit $\im f = R^{m-1} \oplus I$.
		Wegen $I \neq 0$ wird $f$ über $Q$ invertierbar, insbesondere ist $n=m$.
		Sei $A \subset Q^{n \times n}$ die dazugehörigen Matrix.
		Da $A(R^n) \subseteq R^{n-1} \oplus I$ gilt, liegen alle Einträge von $A$ in $R$, die unterste Zeile sogar in $I$.
		Insbesondere $a := \det A \in I$
		
		Sei $B := A^{-1}$.
		Wegen $B \enbrace*{R^{n-1} \oplus I} \subseteq R^n$ liegen die Einträge der ersten $n-1$ Spalten in $R$ und die der letzten in $I^{-1}$.
		Insbesondere $a^{-1} = \det A^{-1} \in I^{-1}$.
		Nun ist $a \cdot a^{-1} = \det A \cdot \det A^{-1} = \det E_n =1$ und die Behauptung folgt aus der Vorbemerkung.\qedhere
	\end{enumerate}
\end{beweis}

\begin{proposition}[label=prop:summe_ideale]
	Sei $R$ ein Dedekind-Ring.
	Für Ideale $I_1$, $I_2$ gibt es dann einen $R$-linearen Isomorphismus $I_1 \oplus I_2 \cong I_1 \cdot I_2 \oplus R$.
\end{proposition}

\begin{korollar}[label=kor:isomorphismus_endl_erz_proj]
	Sei $R$ ein Dedekind-Ring.
	Dann ist jeder endlich erzeugte projektive Modul isomorph zu $R^n \oplus I$ für ein geeignetes $n$ und ein Ideal $I$.
\end{korollar}
\begin{beweis}
	Folgt mit Induktion aus \autoref{lem:ideale_dedekind} b) und \autoref{prop:summe_ideale}.
\end{beweis}

\begin{beweis}[name={von \autoref{satz:klassengruppe_K_0}}]
	Sei $f \colon C(R) \to \tilde{K}_0(R)$ die Abbildung, die ein gebrochenes Ideal $I \subseteq R$ die Klasse von $I$ als $R$-Modul $[I] \in \tilde{K}_0(R)$ zuordnet.
	Da $I\cong a \cdot I \subseteq R$ als $R$-Modul für geeignetes $a \in R$, ist $I$ nach \autoref{lem:ideale_dedekind} a) endlich erzeugt und projektiv.
	Ist $I \cong R \cdot a$ ein gebrochenes Hauptideal, so ist $I \cong R$ als $R$-Modul.
	\autoref{prop:summe_ideale} impliziert, dass $f$ ein Gruppenhomomorphismus ist und damit ist $f$ auch wohldefiniert.
	Wegen b) aus \autoref{lem:ideale_dedekind} (oder wegen \autoref{kor:isomorphismus_endl_erz_proj}) ist $f$ surjektiv.
	
	Zur Injektivität: Sei $I$ ein gebrochenes Ideal mit $[I]=0 \in \tilde{K}_0(R)$.
	Dann ist $I$ als $R$-Modul stabil frei.
	Ohne Beschränkung der Allgemeinheit können wir annehmen, dass $I$ ein Ideal ist.
	Nach c) aus \autoref{lem:ideale_dedekind} ist dann $I$ ein Hauptideal und $I$ ist trivial in $C(R)$.
\end{beweis}

\begin{satz}[label=satz:primfaktoren_ideale]
	Sei $R$ ein Dedekind-Ring.
	\begin{enumerate}[a)]
		\item Alle Primideale $\neq 0$ sind maximal.
		\item Jedes Ideal $\neq 0$ lässt sich eindeutig als Produkt von Primidealen schreiben.
	\end{enumerate}
\end{satz}
\begin{beweis}
	\begin{enumerate}[a)]
		\item Sei $0 \subsetneq P \subsetneq J \subseteq R$ mit $P$ prim.
		Wegen $P \subsetneq J$ folgt $J^{-1} P \subsetneq R$.
		Es ist $P = J \cdot \enbrace*{J^{-1} P}$.
		Da $P$ prim ist, folgt $J^{-1} P \subseteq P$.
		Es folgt $R= P P^{-1} \subseteq J \subseteq R$, also $J=R$.
		\item \emph{Existenz:} Sei $\mathcal{C}$ die Menge der Ideale $\neq 0$, die sich nicht als Produkt von Primidealen schreiben lassen.
		Wir zeigen $\mathcal{C}=\emptyset$.
		Angeommen, dies wäre nicht der Fall.
		Da Ideale in $R$ endlich erzeugt sind, finden wir mit Zorns Lemma ein maximales Element $I \in \mathcal{C}$.\todo{das geht auch ohne das Lemma von Zorn}
		
		Offenbar ist $I$ kein maximales Ideal, andernfalls wäre $I$ selbst prim.
		Sei $I \subsetneq J \subsetneq R$.
		Dann ist $I =J \enbrace*{J^{-1} I}$ und wegen $I \subseteq J$ folgt $J^{-1} I \subseteq R$.
		Aus $J \subsetneq R$ folgt durch Multiplikation mit $J^{-1} I$
		\[
			I \subsetneq J^{-1}I
		\] 
		Also gilt $J^{-1} I \notin \mathcal{C}$ und $J \notin \mathcal{C}$.
		Daher lassen sich beide als Produkt von Primidealen schreiben unt damit auch $I$.
		
		\emph{Eindeutigkeit:} Sei $P_1 \cdot \ldots \cdot P_m = Q_1 \cdot \ldots \cdot Q_n$ mit $P_i, Q_j$ prim.
		Dann ist
		\[
			P_1 \supseteq P_1 \cdot \ldots \cdot P_m = Q_1 \cdot \ldots \cdot Q_n
		\]
		Da $P_1$ prim ist, liegt eines der $Q_i$ in $P_1$, ohne Einschränkungen $Q_1 \subseteq P_1$.
		Da nach a) $Q_1$ maximal ist, folgt $Q_1=P_1$.
		Es folgt $P_2 \cdot \ldots \cdot P_m = Q_2 \cdot \ldots \cdot Q_n$ und die Behauptung folgt induktiv.
		\qedhere
	\end{enumerate}
\end{beweis}

\begin{lemma}[label=lem:]
	Sei $R$ ein kommutativer Ring.
	$P \subseteq R$ ein Primideal, $I,J$ Ideale mit $I J  \subseteq P$.
	Dann gilt $I \subseteq P$ oder $J \subseteq P$.
\end{lemma}
\begin{beweis}
	Angenommen dies gilt nicht, dann gibt es $a \in  I\setminus P$, $b \in J \setminus P$.
	Dann gilt $a \cdot b \in I \cdot J \subseteq P$, also $a \in P$ oder $b \in P$. Widerspruch!
\end{beweis}


\begin{lemma}[label=lem:ideale_dedekind_summe_invers]
    Sei $R$ ein Dedekind-Ring und $I$, $J$ Ideale in $R$.
    Dann gibt es $a \in I$ mit $I^{-1}a + J = R$.
\end{lemma}
\begin{beweis}
    Wegen der Primfaktorzerlegung für $J$ (siehe \autoref{satz:primfaktoren_ideale}) gibt es nur endlich viele maximale Ideale $P_1, \ldots ,P_n$ oberhalb von $J$.
    Wir zeigen: 
    \[
        \exists a \in I  \text{ mit } I^{-1}a \not\subseteq P_i \text{ für alle } i
    \]
    Dann kann $I^{-1}a +J$ in keinem der $P_i$ liegen und wegen $J \subseteq I^{-1}a + J$ kann $I^{-1}a +J$ auch in keinem anderen maximalen Ideal liegen.
    Also gilt $I^{-1}a +J=R$.
    
    Zur Konstruktion von $a$ wählen wir für jedes $i$ ein 
    \[
        a_i \in I P_1 \cdot \ldots \bcancel{P_i} \cdot \ldots \cdot P_n \setminus I \cdot P_1 \cdot \ldots \cdot P_n
    \]
    Dann ist $a_i I^{-1} \subseteq P_j$ für $j\neq i$, aber $a_iI^{-1} \not\subseteq P_i$, da andernfalls $a_i I^{-1} \subseteq \bigcap_{i=1}^n P_i = P_1 \cdot \ldots \cdot P_n$, also $a_i \in IP_1 \cdot \ldots \cdot P_n$.
    Es folgt für $a := a_1 + \ldots +a_n$ nun
    \(
        a I^{-1} \not\subseteq P_i
    \).
\end{beweis}

\begin{lemma}[label=lem:linearkombi_eins_dedekind]
    Seien $I_1$, $I_2$ nichttriviale Ideale in einem Dedekind-Ring $R$.
    Dann gibt es $a_i \in I_i$, $b_i \in I^{-1}_i$ mit $1 = a_2 b_2 -a_1 b_1$
\end{lemma}
\begin{beweis}
    Wähle $a_1 \in I_1$ mit $a_1 \neq 0$.
    Dann ist $J := a_1 I^{-1} \subseteq R$ ein Ideal.
    Für $I=I_2$ gibt es nach \autoref{lem:ideale_dedekind_summe_invers} $a_2 \in I_2$ mit $I^{-1}_2 a_2 + a_1 I_1^{-1} =R$.
    Also gibt es $b_i \in I_1^{-1}$ mit $1 = b_2 a_2 - a_1 b_1$.
\end{beweis}

\begin{beweis}[{name={von \autoref{prop:summe_ideale}}}]
    Sind $a_i$, $b_i$ wie in \autoref{lem:linearkombi_eins_dedekind}, so ist
    \[
        \begin{pmatrix}
            a_2 & a_1 \\
            b_1 & b_2
        \end{pmatrix} \colon I_1 \oplus I_2 \longrightarrow I_1 I_2 \oplus R
    \]
    ein Isomorphismus mit Inversem
    \[
        \begin{pmatrix*}[r]
            b_2 & -a_1 \\
            -b_1 & a_2
        \end{pmatrix*} \colon I_1 I_2 \oplus R \longrightarrow I_1 \oplus I_2 \qedhere
    \]
\end{beweis}

\begin{beispiel}
    Betrachte $R := \mathbb{Z} \benbrace*{\sqrt{-5}} \subseteq \mathbb{Q} \benbrace*{\sqrt{-5}} =: K$.
    \begin{enumerate}[a)]
        \item Sei $I \subseteq R$ ein nichttriviales Ideal.
        Sei $\alpha = a + b \cdot \sqrt{-5} \in Q$ mit $\alpha I \subseteq I$.
        Dann gilt $\alpha \in R$.
        \item Sei $I \subseteq R$ ein nichttriviales Ideal.
        Dann gibt es $c \in K \setminus R$ mit $c \cdot I \subseteq R$.
        \item $R$ ist ein Dedekind-Ring.
        \item $\enbrace*{2, 1 + \sqrt{-5}}$ ist ein projektiver $R$-Modul, der nicht stabil frei ist, also
        \[
            \benbrace[\Big]{\enbrace[\big]{2, 1+ \sqrt{-5}}} \neq 0 \in \tilde{K}_0(R)
        \]
        \item Es gilt $2 \benbrace[\Big]{\enbrace[\big]{2, 1+ \sqrt{-5}}} =0 \in \tilde{K}_0(R)$.
    \end{enumerate}
\end{beispiel}
\begin{beweis}
    \begin{enumerate}[a)]
        \item Als $\mathbb{Z}$-Modul hat $I$ Rang 2.
        Sei $\chi_\alpha \in \mathbb{Z} \benbrace*{\lambda}$ das charakteristische Polynom von $\alpha \colon I \to I$.
        Da $I \otimes_{\mathbb{Z}} \mathbb{Q} = K$ ist $\chi_\alpha$ auch das charakteristische Polynom über $\mathbb{Q}$ von $\alpha \colon K \to K$.
        Indem wir die $\mathbb{Q}$-Basis $\set*{1, \sqrt{-5}}$ von $K$ benutzen erhalten wir 
        \[
            \chi_\alpha(X) = \det \begin{pmatrix}
                X- a & 5b\\
                -b & X-a
            \end{pmatrix} = (X-a)^2 + 5 b^2 = X^2 -2 a X + (a^2 + 5b^2)
        \]
        Also $2a, a^2 + 5b^2 \in \mathbb{Z}$, insbesondere $20b^2 \in \mathbb{Z}$.
        Damit ist auch $2 b \in \mathbb{Z}$, es folgt $a^2 +b^2 \in \mathbb{Z}$ und daher
        \[
            (2a)^2 + (2b)^2 \equiv 0 \mod 4
        \]
        Wäre $2 a$ oder $2b$ ungerade, so wären beide ungerade und $(2a)^2 + (2b)^2 \equiv 2 \mod 4$.
        Also sind $2a$ und $2b$ gerade und $a,b \in \mathbb{Z}$.
        \item Als $\mathbb{Z}$-Modul hat $I$ Rang 2, insbesondere wird $I=(\alpha,\beta)$ von zwei Elementen erzeugt.
        Nun sind auch $(\alpha), (\beta) \subseteq (\alpha,\beta)$ Ideale von Rang 2.
        Ist $I=(\alpha)$, so nehmen wir $c := \sfrac{1}{\alpha}$.
        Andernfalls ist $\beta +(\alpha) \in \sfrac{R}{(\alpha)}$ nicht invertierbar und da $\sfrac{R}{(\alpha)}$ endlich ist, muss $\beta +(\alpha)$ ein Nullteiler sein.
        Wir finden also $b \in R$ mit $b \cdot \beta \in (\alpha)$.
        Setze nun $c := \sfrac{b}{\alpha}$.
        Dann gilt $\alpha c, \beta c \in R$, also $c I \subseteq R$.
        Wegen $(\alpha) \neq I$ ist $\beta \in (\alpha)$, also $\sfrac{b}{\alpha} \notin R$.
        \item Sei $I \subseteq R$ ein Ideal.
        Sei $J = \set*{\alpha \in K \given \alpha I \subseteq R}$.
        Es ist $I \cdot J =R$ zu zeigen.
        
        Sei $N \coloneqq \set*{\alpha \in K \given \alpha \cdot I \cdot J \subseteq R}$.
        Dann ist
        \[
            (N \cdot J) \cdot I = N \cdot (I \cdot J) \subseteq R
        \]
        also $N \cdot J \subseteq J$ nach Definition von $J$.
        Mit a) folgt $N \subseteq R$.
        Ist $J \cdot I \subsetneq R$, so folgt mit b) $N \subsetneq R$.
        Also ist $J \cdot I =R$. 
        \item \begin{enumerate}[(i)]
            \item Zu zeigen: $\enbrace*{2, 1+ \sqrt{-5}} \subsetneq R$. Es gilt
            \begin{align}
                2 \enbrace*{a + b\sqrt{-5}} + \enbrace*{1 + \sqrt{-5}} \enbrace*{\alpha + \beta \cdot \sqrt{-5}} &= 2a + 2 b\cdot \sqrt{-5} + \alpha - 5 \beta +(\alpha +\beta) \sqrt{-5} \\
                &= 2 a + \alpha - 5 \beta + \enbrace*{2b + \alpha+ \beta} \sqrt{-5}
            \end{align}
            Aus den Gleichungen $2 a + \alpha - 5 \beta = 1$ und $2 b +\alpha + \beta=0$ folgt
            \[
                2a -2b -6 \beta =1
            \]
            was einen Widerspruch darstellt, also $1 \notin \enbrace*{2, 1+ \sqrt{-5}}$.
            \item $\enbrace*{2, 1+ \sqrt{-5}}$ ist kein Hauptideal.
            Angenommen $\enbrace*{2, 1+ \sqrt{-5}} = (\alpha)$.
            Dann teilt $\alpha$ sowohl $2$ als auch $1+ \sqrt{-5}$ und $\abs*{\alpha}^2$ teilt $4$ und $6$, also $\norm*{\alpha}^2 \in \set*{1,2}$.
            Da $2$ nicht als $\norm*{a + b \sqrt{-5}}^2$ vorkommt, folgt $\abs*{\alpha}^2=1$ und $\alpha = \pm 1$.
            Aber $\enbrace*{2, 1+ \sqrt{-5}} \neq R$.
        \end{enumerate}
        \item Es gilt
        \begin{align}
            \enbrace*{2, 1+ \sqrt{-5}} \cdot \enbrace*{2, 1+ \sqrt{-5}} = \enbrace*{4, 2 + 2 \sqrt{-5}, -4 + 2\sqrt{-5}} = (2) 
        \end{align}
        Dies ist die $0$ in $\tilde{K}_0(R)$.
        \qedhere
    \end{enumerate}
\end{beweis}
% section 2 (end)

\newpage

\section{Endlichkeit von Kettenkomplexen} % (fold)
\label{sec:3}
In diesem Abschnitt sind Kettenkomplexe immer $\mathbb{N}$-graduiert. (Allgemeiner könnten wir nach unten beschränkte Kettenkomplexe betrachten)\todo{RevChap 3}

\begin{definition}
    Ein Kettenkomplex $C_*$ von $R$-Moduln heißt \bet{beschränkt}\index{Kettenkomplex!beschränkter}, wenn $C_j \neq 0$ nur für endlich viele $j$.
    Er heißt von \bet{endlichem Typ}\index{Kettenkomplex!von endlichem Typ}, falls er beschränkt ist und alle $C_j$ endlich erzeugt sind.
\end{definition}

\begin{definition}
    Sei $C_*$ ein Kettenkomplex von endlichem Typ von projektiven $R$-Moduln.
    Dann definieren wir die \Index{Euler-Charakteristik} von $C_*$ als 
    \[
        \chi(C_*) := \sum_j (-1)^j \cdot \benbrace[\big]{C_j} \in K_0(R)
    \]
    Wir schreiben $\tilde{\chi}(C_*)$ für das Bild von $\chi(C_*)$ in $\tilde{K}_0(R)$.
\end{definition}

\begin{bemerkung}
    Für $R=\mathbb{Z}$ ist $K_0(R) \cong \mathbb{Z}$ und $\chi(C_*)$ die Eulercharakteristik aus Topologie \RM{1}.    
    Ebenso: Für einen endlichen CW-Komplex $X$ ist $\chi(X) = \chi \enbrace*{C^{\cell}_*(X)}$ (unter $K_0(\mathbb{Z})\cong \mathbb{Z}$).
\end{bemerkung}

\begin{lemma}
    Sei $\begin{tikzcd} C_* \rar & C_*' \rar & C_*'' \end{tikzcd}$ eine kurze exakte Folge von Kettenkomplexen von endlichem Typ von projektiven $R$-Moduln.
    Dann gilt
    \[
        \chi(C_*') = \chi(C_*) + \chi(C_*'')
    \]
\end{lemma}
\begin{beweis}
    Da die $C_j$ projektiv sind, gilt $C_j' \cong C_j \oplus C_j''$, also $\benbrace*{C_j'} = \benbrace*{C_j} + \benbrace*{C_j''} \in K_0(R)$.
    Es gilt nun
    \[
        \chi(C_*') = \sum_{j} (-1)^j \cdot \benbrace*{C_j'} = \sum_j (-1)^j \cdot \enbrace*{ \benbrace*{C_j} + \benbrace*{C_j''}} = \chi(C_*) + \chi(C_*'') \qedhere
    \]
\end{beweis}

\begin{proposition}[label=prop:euler_homologie]
    Sei $C_*$ ein Kettenkomplex von endlichem Typ von projektiven Moduln.
    Seien alle Homologiegruppen von $C_*$ projektiv.
    Dann gilt 
    \[
        \chi(C_*) = \sum_j (-1)^j \cdot \benbrace[\big]{H_j(C_*)} \in K_0(R)
    \] 
\end{proposition}
\begin{beweis}
    Sei $Z_j \coloneqq \ker  \enbrace*{C_j \to C_{j-1}}$ und $B_j \coloneqq \im \enbrace*{Z_{j+1} \to Z_j}$.
    Wir erhalten kurze exakte Folgen
    \[
        \begin{tikzcd}
            0 \rar & Z_{j+1} \rar & C_{j+1} \rar & B_j \rar & 0
        \end{tikzcd}
    \]
    und 
    \[
        \begin{tikzcd}
            0 \rar & B_j \rar & Z_j \rar[two heads] & H_j(C_*) \rar & 0
        \end{tikzcd}
    \]
    Es folgt $Z_j \cong B_j \oplus H_j(C_*)$.
    Für $j<0$ ist $C_j=0$ und insbesondere ist $Z_0 =C_0$ projektiv und endlich erzeugt.
    Daher sind auch $B_0$ und $H_0(C_*)$ projektiv und endlich erzeugt.
    Es folgt $C_1 \cong Z_1 \oplus B_0$ und $Z_1$ ist projektiv und endlich erzeugt.
    Induktiv folgt, dass alle $C_j$, $B_j$ und $H_j(C_*)$ projektiv und endlich erzeugt sind.
    Mit $\benbrace*{C_{j+1}} = \benbrace*{B_j} + \benbrace*{Z_{j+1}}$ und $\benbrace*{Z_j} = \benbrace*{B_j} + \benbrace*{H_j(C_*)}$ folgt nun die Behauptung, vergleiche Topologie \RM{1}.
\end{beweis}

\begin{bemerkung}
    Es gilt $-\chi(C_*) = \chi (\Sigma C_*)$.
\end{bemerkung}

\begin{proposition}[label=prop:euler_invariant_kettenhomotopie]
    Seien $C_*$, $C_*'$ kettenhomotopieäquivalente Kettenkomplexe von endlichem Typ von projektiven $R$-Moduln.
    Dann gilt 
    \[
        \chi(C_*) = \chi(C_*') \in K_0(R)
    \]
\end{proposition}
\begin{beweis}
    Sei $\varphi \colon C_* \to C_*'$ eine Kettenhomotopieäquivalenz.
    Sei $\Keg(\varphi)_*$ der Kegel von $\varphi$.
    Wir erhalten eine kurze exakte Folge von Kettenkomplexen
    \[
        \begin{tikzcd}
            0 \rar & C_*' \rar & \Keg(\varphi)_* \rar & (\Sigma C)_* \rar & 0
        \end{tikzcd}
    \]
    Also gilt $\chi(\Keg(\varphi)_*) = \chi(C_*') + \chi((\Sigma C)_*) = \chi(C_*') - \chi(C_*)$.
    Da $\varphi$ in Homologie einen Isomorphismus induziert, ist $H_*(\Keg(\varphi)_*)=0$ und insbesondere projektiv.
    Nach \autoref{prop:euler_homologie} ist dann 
    \[
        \chi \enbrace[\big]{\Keg(\varphi)_*} = \sum_j (-1)^j \cdot  H_j \enbrace[\big]{\Keg(\varphi)_*} =0
    \]
    und die Behauptung folgt.
\end{beweis}

\begin{definition}[{name=[homologie-endlich]}]
    Ein Kettenkomplex $C_*$ von $R$-Moduln heißt \Index{homologie-endlich}, wenn er homotopieäquivalent zu einem Kettenkomplex $D_*$ von endlichem Typ von projektiven $R$-Moduln ist.
    Wir definieren dann 
    \[
        \chi(C_*) \coloneqq \chi(D_*) \in K_0(R)
    \]
\end{definition}

\begin{lemma}[{name=[{Euler-Charakteristik einer kurzen exakten Folge von Kettenkomplexen}]}]
    Sei 
    \(
    	\begin{tikzcd}[cramped]
            C \rar["i"] & C' \rar["\pi"] & C''
        \end{tikzcd}
    \) 
	eine kurze exakte Folge von homotopie-endlichen Kettenkomplexen von projektiven Moduln.
	Dann gilt
	\[
		\chi(C') = \chi(C) + \chi(C'') \in K_0(R)
	\]
\end{lemma}
\begin{beweis}
	Wir ersetzen $C$ und $C''$ in zwei Schritten durch Kettenkomplexe von endlichem Typ von projektiven $R$-Moduln.
	Sei $\varphi \colon D'' \to C''$ eine Kettenhomotpieäquivalenz, wobei $D''$ von endlichem Typ von projektiven Moduln ist.
	Sei $C' \oplus^{C''} D$ der Unterkomplex von $C' \oplus D$, der durch 
	\[
		(c',d'') \in C' \stackrel{C''}{\oplus} D :\Leftrightarrow \pi(c') =\varphi(d'')
	\]
	definiert wird.
	Wir erhalten eine kommutatives Diagramm mit kurzen exakten Zeilen
	\[
		\begin{tikzcd}[column sep=4em,row sep=2.8em]
			0 \rar & C \rar["i"] & C' \rar["\pi"] & C'' \rar & 0 \\
			0 \rar & C \rar["{c \mapsto (i(c),0)}"] \uar["\id"] & C' \oplus^{C''} D \rar[two heads] \uar & D'' \rar \uar["\varphi"] & 0
		\end{tikzcd}
	\]
	Mit der langen exakten Folge in Homologie und dem 5er-Lemma sehen wir, dass auch die mittlere Abbildung einen Isomorphismus in Homologie induziert.
	Da $C'$ und $C' \oplus^{C''} D$ projektiv sind, ist die mittlere Abbildung daher auch eine Kettenhomotopieäquivalenz.
	Wir können also nun annehmen, dass $C'$ von endlichem Typ von projektiven $R$-Moduln ist.
	
	Sei nun  $\varphi \colon C \to D$ eine Kettenhomotopieäquivalenz und $D$ von endlichem Typ von projektiven $R$-Moduln.
	Sei $D \oplus_C C'$ der Quotient von $D \oplus C'$ nach dem Unterkomplex $\Delta C = \set*{\enbrace*{\varphi(c), - i(c)} \given c \in C}$.
	Wir erhalten ein kommutatives Diagramm mit exakten Zeilen
	\[
		\begin{tikzcd}
			0 \rar &C \dar["\phi"] \rar["i"] & C' \rar["\pi"] \dar & C'' \rar \dar["\id"] & 0 \\
			0 \rar & D \rar & D \oplus_C C' \rar & C'' \rar & 0
		\end{tikzcd}
	\]
	Wie vorher ist auch die mittlere Abbildung eine Kettenhomotopieäquivalenz.
	
	Wir können also annehmen, dass $C$ und $C''$ vom endlichen Typ von projektiven $R$-Moduln sind.
	Dann ist aber auch $C'$ von endlichem Typ von projektiven $R$-Moduln und die Behauptung folgt aus \autoref{prop:euler_invariant_kettenhomotopie}.
\end{beweis}

\begin{bemerkung}
	Sei \(
		\begin{tikzcd}[cramped,sep=small]
			0 \rar & C \rar & C' \rar & C'' \rar & 0
		\end{tikzcd}
	\) eine kurze exakte Sequenz von Kettenkomplexen von projektiven Moduln.
	Aus dem Beweis des vorhergehenden Lemmas folgt: Sind $C$ und $C''$ homotopie-endlich, so ist auch $C'$ homotopie-endlich.
	Allgemeiner gilt: Sind zwei von drei Kettenkomplexen $C$, $C'$ und $C''$ homotopie-endlich, so auch der dritte.
	Dies folgt aus dem obigen Fall mit $\Sigma C \xrightarrow{\simeq} \Keg(\pi)$ und der kurzen exakten Folge 
	\[
		\begin{tikzcd}
			0 \rar & C'' \rar & \Keg(\pi) \rar & \Sigma C' \rar & 0
		\end{tikzcd}
	\]
	Ebenso $\Keg(i) \xrightarrow{\simeq} C''$.
\end{bemerkung}

\begin{satz}
	Sei $C_*$ ein Kettenkomplex von projektiven $R$-Moduln.
	Sei $C_*$ homotpie-endlich.
	Dann sind äquivalent
	\begin{enumerate}[(i)]
		\item $\tilde{\chi}(C_*) =0 \in \tilde{K}_0(R)$.
		\item $C_*$ ist kettenhomotopieäquivalent zu einem Kettenkomplex von endlichem Typ von \emph{freien} Moduln.
	\end{enumerate}
\end{satz}
\begin{beweis}
	Für die Implikation (ii) $\Rightarrow$ (i)  sei $C_* \simeq F_*$ mit $F_*$ von endlichem Typ von freien $R$-Moduln.
	Dann ist $\chi(C_*) = \chi(F_*)$ und insbesondere $\tilde{\chi}(C_*) = \tilde{\chi}(F_*)$.
	Aber es gilt
	\[
		\tilde{\chi}(F_*) = \sum_{j} (-1)^j \cdot \benbrace*{F_j} =0 \in \tilde{K}_0(R)
	\]
	Für die umgekehrte Implikation können wir ohne Einschränkungen annehmen, dass $C_*$ von endlichem Typ von projektiven $R$-Moduln ist.
	Sei $C_j=0$ für $j>n$.
	Sei $Q_0$ ein endlich erzeugter projektiver $R$-Modul mit $C_0 \oplus Q_0$ ist frei.
	Sei $Q_*$ der Kettenkomplex
	\[
		\begin{tikzcd}
			Q_0 & Q_0 \lar["\id"'] & 0 \lar & 0 \lar & \ldots \lar 
		\end{tikzcd}
	\]
	Dann ist $C_*$ kettenhomotopieäquivalent zu $C_* \oplus Q_*$ mit $(C_* \oplus Q_*)_0$ ist frei.
	Wir können also annehmen, dass $C_0$ frei ist.
	Genauso können wir annehmen, dass $C_j$ frei ist für $j<n$.
	Dann ist
	\[
		0 = \tilde{\chi}(C_*) = \sum_j (-1)^j \cdot \benbrace*{C_j} = (-1)^n \benbrace*{C_n} \in \tilde{K}_0(R)
	\]
	Damit ist $\benbrace*{C_n}=0 \in \tilde{K}_0(R)$, also ist $C_n$ stabil frei.
	Dann gibt es $F_n$ endlich erzeugt und frei, sodass $C_n \oplus F_n$ endlich erzeugt frei ist.
	Sei $F_*$ der Kettenkomplex
	\[
		\begin{tikzcd}
			0 &  \ldots \lar & 0 \lar & F_n \lar & F_n \lar["\id"'] & 0 \lar & \ldots \lar
		\end{tikzcd}
	\]
	Dann ist $C_*$ kettenhomotopieäquivalent zu $C_* \oplus F_*$ und die Kettenmodule von $C_* \oplus F_*$ sind alle frei.
\end{beweis}

\begin{definition}[{name=[dominierender Kettenkomplex]}]
	Ein Kettenkomplex $D_*$ \Index{dominiert} einen Kettenkomplex $C_*$, falls es Kettenabbildungen \(
		\begin{tikzcd}[cramped,sep=small]
			C_* \rar["i"] & D_* \rar["p"] & C_*
		\end{tikzcd}
	\) gibt mit $p \circ i \simeq \id_{C_*}$.
\end{definition}

\begin{definition}[{name=[endlich dominiert]}]
	Ein Kettenkomplex $C_*$ heißt \Index{endlich dominiert}, wenn er von einem Kettenkomplex von endlichem Typ von freien Moduln dominiert wird. 
\end{definition}

\begin{satz}
	Sei $C_*$ ein Kettenkomplex von projektiven $R$-Moduln.
	Dann sind äquivalent:
	\begin{enumerate}[(i)]
		\item $C_*$ ist homotopieäquivalent zu einen Kettenkomplex von endlichem Typ von projektiven Moduln.
		\item $C_*$ ist endlich dominiert.
	\end{enumerate}
\end{satz}
\begin{beweis}
	Für die erste Implikation müssen wir zeigen, dass jeder Kettenkomplex $C_*$ von endlichem Typ von projektiven Moduln endlich dominiert ist.
	Sei $n$ mit $C_j=0$ für $j>n$.
	Wähle für $j \le n$ $Q_j$ endlich erzeugt projektiv mit $Q_j \oplus C_j$ frei.
	Sei $P_*$ der Kettenkomplex 
	\[
		\begin{tikzcd}
			P_0 & P_1 \lar["0"'] & P_2 \lar["0"'] & \ldots \lar & P_n \lar["0"'] & 0 \lar & \ldots \lar
		\end{tikzcd}
	\]
	Dann dominiert $C_* \oplus P_*$ den Kettenkomplex $C_*$ und die Kettenmoduln von $(C_* \oplus P_*)$ sind frei.
	
	Für die umgekehrte Implikation zeigen wir zunächst per Induktion nach $k$ die Existenz von 
	\begin{enumerate}[a),itemsep=0pt]
		\item Kettenabbildungen \(
			\begin{tikzcd}[cramped,sep=small]
				C_* \rar["i"] & D_* \rar["p"] & C_*
			\end{tikzcd}
		\)
		\item einer Kettenhomotopie $s \colon {\id_{C_*}} \to p \circ i$.
		\item Für $j=0,\ldots ,k$ Abbildungen $t_j \colon D_j \to D_{j+1}$
	\end{enumerate}
	Sodass
	\begin{enumerate}[1),itemsep=0pt]
		\item $D_*$ ist von endlichem Typ von freien $R$-Moduln.
		\item ${\id_{D_j}} - {i_j \circ p_j} = \partial^D_{j+1} \circ t_j + t_{j-1} \circ \partial_j^D$ für $j=0,\ldots ,k$
		\item $p_{j+1} \circ t_j = s_j \circ p_j$ für $j=0,\ldots ,k$
	\end{enumerate}
	Für $j=-1$ folgt die Behauptung direkt aus (ii).
	Seien $i,p, D_*,s,t$ wie eben gegeben für $k-1$.
	\[
		\begin{tikzcd}[sep=huge]
			C_{k-1} \rar[bend left,"s_{k-1}"] \dar["i_{k-1}",xshift=.7ex] 
			& C_k \lar["\partial^C_k"'] \dar["i_k",xshift=.7ex] \rar[bend left,"s_{k}"]
			& C_{k+1} \dar["i_{k+1}",xshift=.7ex]  \lar["\partial^C_{k+1}"']
			& \ldots \lar \\
			D_{k-1} \rar[bend right,"t_{k-1}"'] \uar["p_{k-1}",xshift=-.7ex] 
			& D_k \lar["\partial_k^D"'] \uar["p_k",xshift=-.7ex]  \drar["t_k={\id}"',bend right,DodgerBlue3]
			& D_{k+1} \lar["\partial^D_{k+1}"'] \uar["p_{k+1}",xshift=-.7ex] \dar[phantom,sloped,"\bigoplus",DodgerBlue3]
			& \ldots \lar \\[-4ex]
			& & \color{DodgerBlue3} D_k \ar[uu,bend right=55,pos=.6,"y"',DodgerBlue3] \ular["x"',DodgerBlue3]
 		\end{tikzcd}
	\]
	Wir definieren $\hat{D}_*$ indem wir in Grad $k+1$ die direkte Summe mit $D_k$ bilden.
	Weiter definieren wir $\hat{t}_{k} \colon \hat{D}_k \to \hat{D}_{k+1}$ als die Inklusion $D_k \hookrightarrow D_{k+1} \oplus D_k$.
	Ebenso definieren wir $\hat{i}_{k+1}$ als die Komposition von $i_{k+1} \colon C_{k+1} \to D_{k+1}$ mit der Inklusion $D_{k+1} \hookrightarrow \hat{D}_{k+1}$.
	Wir müssen auch $\partial^D_{k+1}$ und $p_{k+1}$ auf $\hat{D}_{k+1}$ fortsetzen.
	Sei $\partial^{\hat{D}}_{k+1} = \enbrace*{\partial^D_{k+1},x}$ und $\hat{p}_{k+1} ? \enbrace*{p_{k+1},y}$ mit $x \colon D_k \to D_k$ und $y \colon D_k \to C_{k+1}$.
	Die Behauptung übersetzt sich nun in folgende Gleichungen:
	\begin{enumerate}[1)]
		\item $x + t_{k-1} \circ \partial_k^D = {\id_{D_k}} - i_k \circ p_k$
		\item $\partial_k^D \circ x=0$
		\item $\partial^C_{k+1} \circ y = p_k \circ x$
		\item $y= s_k \circ p_k$
	\end{enumerate}
	Wir setze also $x := {\id_{D_k}} - i_k \circ p_k - t_{k-1} \circ \partial^D_k$ und $y := s_k \circ p_k$.
	Wir müssen 2) und 3) nachrechnen:
	\begin{align}
		\partial^D_k \enbrace*{{\id_{D_k}} - i_k \circ p_k - t_{k-1} \circ \partial^D_k} &= \enbrace*{{\id_{D_{k-1}}} - i_{k-1} \circ p_{k-1} - \partial^D_k \circ t_{k-1}} \partial^D_k \\
		&= \enbrace*{{\id_{D_{k-1}}} - i_{k-1} p_{k-1} - \enbrace*{{\id_{D_{k-1}}} - i_{k-1} p_{k-1} - t_{k-2} \partial_{k-1}^D}} \partial^D_k \\ 
		&=0
	\end{align}
	und
	\begin{align}
		p_k \enbrace*{{\id_{D_k}} - i_k \circ p_k - t_{k-1} \partial^D_k} &= \enbrace*{{\id_{C_k}} - p_k \circ i_k} \circ p_k - p_k t_{k-1} \partial_k^D \\
		&= \enbrace*{s_{k-1} \circ \partial^C_k + \partial^C_{k+1} \circ s_k} \circ p_k - p_k \circ t_{k-1} \circ \partial^D_k \\
		&= \Underbrace{\enbrace[\big]{s_{k-1} \circ p_k - p_k \circ t_{k-1}}}{=0} \circ \partial^D_k + \partial_{k+1}^C \circ s_k \circ p_k = \partial^C_{k+1} \circ y
	\end{align}
	Mit $k \to \infty$ erhalten wir durch diese Konstruktion einen Kettenkomplex von endlich erzeugten freien Moduln, der kettenhomotopieäquivalent zu $C_*$ ist.
	Wir können also ohne Beschränkung der Allgemeinheit annehmen, dass die Kettenmoduln von $C_*$ endlich erzeugt frei sind.
	% \[
	% 	\begin{tikzcd}
	% 		C_0 & C_1 \lar & C_2 \lar & C_3 \lar & \ldots \lar & C_l \lar & C_{l+1} \lar & C_{l+2} \lar& \ldots \lar
	% 	\end{tikzcd}
	% \]
	Da $C_*$ endlich dominiert ist, ist $\id_{C_*}$ kettenhomotop zur einer Kettenabbildung $f \colon C_* \to C_*$ mit $f_k =0$ für große $k$.
	Für die zugehörige Kettenhomotopie $s_k \colon C_k \to C_{k+1}$ ist dann
	\[
		{\id_{C_k}} = \partial_{k+1} \circ s_k + s_{k-1} \circ \partial_k
	\]
	für genügend große $k$.
	Es folgt für genügend große $k$:
	\[
		\partial_{k+1} \circ s_k \circ \partial_{k+1} \circ s_k = \partial_{k+1}\enbrace*{-\partial_{k+2} s_{k+1} + {\id_{C_{k+1}}}} \circ s_k = \partial_{k+1} \circ s_k
	\]
	Also ist $\partial_{k+1} \circ s_k \colon C_k \to C_k$ eine Projektion.
	Offenbar ist $\im (\partial_{k+1} \circ s_k) \subseteq \ker \partial_k$.
	Für $x \in \ker \partial_k$ ist 
	\[
		\partial_{k+1} \circ s_k(x) = \enbrace*{{\id} - s_{k+1} \circ \partial_k} (x) =x
	\]
	Also ist $\partial_{k+1} \circ s_k$ eine Projektion auf $\ker \partial_k$, insbesondere ist $\ker \partial_k$ projektiv.
	Sei für festes großes $k$ nun $D_*$ definiert durch
	\[
		D_* = \begin{cases}
			C_* &\text{ falls }* \le k\\
			\ker \partial_k &\text{ falls } *=k+1 \\
			0 &\text{ sonst} 
		\end{cases}
	\]
	und Randabbildung
	\[
		\partial^D_* = \begin{cases}
			\partial^C_* &\text{ falls }*\le k\\
			\ker \partial_k \hookrightarrow C_k &\text{ falls } *=k+1 \\
			0 &\text{ sonst} 
		\end{cases}
	\]
	Wir erhalten $i \colon C_* \to D_*$ und $p \colon D_* \to C_*$ mit 
	\[
		i_* = \begin{cases}
			{\id_{C_*}} &\text{ falls }* \le k\\
			\partial_{k+1} &\text{ falls } * = k+1 \\
			0 & \text{ sonst} 
		\end{cases} \quad , \qquad 
		p_* = \begin{cases}
			\id_{C_*} &\text{ falls } * \le k\\
			s_k &\text{ falls } * =k+1 \\
			0 & \text{ sonst} 
		\end{cases}
	\]
	Dann ist $i \circ p = {\id_D}$ und wir erhalten eine Kettenhomotopie $\hat{s} \colon \id_{C_*} \simeq p \circ i$ durch $0$, falls $* \le k$ und $s_k$ sonst.
\end{beweis}
% section 3 (end)
\newpage

\section{\textsc{Wall}s Endlichkeitshindernis für topologische Räume} % (fold)
\label{sec:4}
\todo[inline]{RevChap 4}
\todo[inline]{Wer ist dieser Wall?}
\begin{definition}[{name=[{endlicher CW-Komplex}]}]
	Ein CW-Komplex $E$ heißt \bet{endlich}\index{endlicher CW-Komplex}, wenn er aus endlich vielen Zellen besteht. 
	Ein topologischer Raum $X$ heißt \bet{endlich dominiert}\index{endlich dominiert!topologischer Raum}, wenn es einen endlichen CW-Komplex $E$ gibt und stetige Abbildungen
	\[
		\begin{tikzcd}
			X \rar["i"] & E \rar["p"] & X
		\end{tikzcd}
	\]
	sodass $p \circ i$ homotop zu $\id_X$ ist.
\end{definition}

\begin{frage}
	Ist jeder endlich dominierte Raum homotopieäquivalent zu einem endlichen CW-Komplex?
\end{frage}

\begin{satz}[label=satz:endl_dom_endl_cw]
	Ist $X$ endlich dominiert, so ist $X$ homotopieäquivalent zu einem endlich dimensionalen CW-Komplex.
\end{satz}

\begin{definition}[{name=[{Abbildungszylinder}]}]
	Der \Index{Abbildungszylinder} einer stetigen Abbildung $f \colon X \to Y$ ist 
	\[
		\Zyl(f) \coloneqq \nicefrac{X \times [0,1] \amalg Y}{(x,1) \sim f(x)}
	\]
\end{definition}

\begin{bemerkung}
	\begin{enumerate}[(i)]
		\item Die Inklusion $i_1 \colon Y \hookrightarrow \Zyl(f)$ ist eine Homotopieäquivalenz.
		Die Inverse ist die Projektion $F \colon \Zyl(f) \to Y$, $F(x,t) = f(x)$, $F(y)=y$.
		\item Die Inklusion $i_0 \colon X = X \times \set*{0} \hookrightarrow \Zyl(f)$ ist homotop zur Komposition $i_1 \circ f$.
	\end{enumerate}
\end{bemerkung}

\begin{beispiel}
	Sei $p \colon X \to X$ mit $p^2 \simeq p$.
	Dann sind $i_0 \colon X \to \Zyl(p)$ und $i_1 \circ p \colon X \to \Zyl(p)$ homotop: 
	\[
		i_0 \circ p \simeq (i_1 \circ p) \circ p \simeq i_1 \circ p \simeq i_0
	\]
\end{beispiel}

\begin{definition}[{name=[{Abbildungsteleskop}]}]
	Das \Index{Abbildungsteleskop} einer stetigen Selbstabbildung $f \colon X \to X$ ist definiert als
	\[
		\Tel(f) \coloneqq \nicefrac{X \times [0,1] \times \mathbb{N}}{(x,1,n) \sim \enbrace*{f(x),0,n+1}}
	\]
	Für Abbildungen $f \colon X \to Y$, $g \colon Y \to X$ definieren wir genauso 
	\[
		\Tel(f,g) \coloneqq \nicefrac{(X \amalg Y) \times [0,1] \times \mathbb{N}}{
			\begin{array}{rl}
				(x,1,n) &\sim \enbrace*{f(x),0,n} \\
				(y,1,n) &\sim \enbrace*{g(y),0,n+1}
			\end{array}
		} 
	\]
\end{definition}
\missingfigure{das könnte man mal malen …}
\missingfigure{auch hier ein Bild …}

\begin{bemerkung}
	\begin{enumerate}[(i)]
		\item Es gilt $\Tel(f,g) \simeq \Tel(g,f) \simeq \Tel{g \circ f}$.
		\item Es gilt $\Tel(f) \simeq \Tel(g)$ für $f \simeq g$
		\item Es gilt $\Tel(\id_X) = X \times [0,\infty) \simeq X$.
	\end{enumerate}
\end{bemerkung}

\begin{definition}
	Sei
	\[
		\begin{tikzcd}
			X_1 \rar["f_1"] & X_2 \rar["f_2"] & X_3 \rar["f_3"] & X_4 \rar & \ldots 
		\end{tikzcd}
	\]
	ein Diagramm von stetigen Abbildungen.
	Dann definieren wir das Teleskop dieses Diagramms als 
	\[
		\Tel(f_1,f_2,\ldots ) \coloneqq \nicefrac{\coprod X_i \times [0,1]}{
			\substack{(x,1) \sim \enbrace*{f_i(x),0} \\ \text{für } x \in X_i }}
	\]
	Dann ist $\Tel(f) := \Tel(f,f,\ldots )$ und $\Tel(f,g) := \Tel(f,g,f,g, \ldots )$.
\end{definition}

\begin{bemerkung}
	$\Tel(f_1,f_2, \ldots )$ ist ein Modell für den Homotopie-Kolimes des obigen Diagramms und hat folgende Eigenschaft:
	\begin{enumerate}[1)]
		\item Für jedes $j$ erhalten wir eine Abbildung $i_j \colon X_j \to \Tel(f_1, f_2, \ldots)$ durch $i_j(x) = (x,0)$ und $i_j$ ist (kanonisch) homotop zu $i_{j+1} \circ f_j$.
		\item Ist $Y$ ein weiterer topologischer Raum und sind $g_j \colon X_j \to Y$ stetige Abbildungen für die $g_j$ homotop zu $g_{j+1} \circ f_j$ so erhalten wir eine Abbildung $F \colon \Tel(f_1,f_2,\ldots ) \to Y$ mit $g_j = F \circ i_j$.
	\end{enumerate}
\end{bemerkung}

\begin{bemerkung}
	Sei
	\[
		\begin{tikzcd}
			X_1 \dar["\alpha_1"] \rar["f_1"] & X_2 \rar["f_2"] \dar["\alpha_2"] & X_3 \dar["\alpha_3"] \rar["f_3"] & X_4 \dar["\alpha_4"] \rar["f_4"] & \ldots  \\
			Y_1 \rar["g_1"] & Y_2 \rar["g_2"] & Y_3 \rar["g_3"] & Y_4 \rar & \ldots 
		\end{tikzcd}
	\]
	ein kommutatives Diagramm, bei dem die $\alpha_i$ Homotopieäquivalenzen sind.
	Dann sind die Teleskope $\Tel(f_1, f_2, \ldots )$ und $\Tel(g_1, g_2, \ldots )$ homotopieäquivalent.
\end{bemerkung}

\begin{bemerkung}
	Es gilt 
	\begin{enumerate}[(i)]
		\item $\Tel(f,g) \simeq \Tel(g,f) \simeq \Tel(g \circ f)$
		\item Falls $f \simeq g$ ist, so ist $\Tel(f) \simeq \Tel(g)$
		\item $\Tel(\id_X) \simeq X$
	\end{enumerate}
\end{bemerkung}

\begin{beweis}[{name={von \autoref{satz:endl_dom_endl_cw}}}]
	Sei $\begin{tikzcd}[cramped]
		X \rar["i"] & E \rar["p"] & X
	\end{tikzcd}$
	eine endliche Dominierung von $X$.
	Sei $\pi = i \circ p \colon E \to E$ und $\pi' \colon E \to E$ eine zelluläre Approximation.\todo{Ref?}
	Dann ist $\Tel(\pi')$ ein endlichdimensionaler CW-Komplex und es gilt 
	\[
		\Tel(\pi') \simeq \Tel(\pi) \simeq \Tel(i \circ p) \simeq \Tel(i,p) \simeq \Tel(p,i) \simeq \Tel(p \circ i) \simeq \Tel(\id_X) \simeq X \qedhere
	\]
\end{beweis}

\begin{definition}[{name=[Gruppenring]}]
	Sei $\pi$ eine Gruppe.
	Der \Index{Gruppenring} $\mathbb{Z}\pi$  ist der Ring aller formalen $\mathbb{Z}$-Linearkombinationen mit endlichem Träger
	\[
		\sum_{g \in \pi} \lambda_g \cdot g
	\]
	mit $\lambda_g \in \mathbb{Z}$.
	Das Produkt auf $\mathbb{Z}\pi$ ist 
	\[
		\enbrace*{\sum\nolimits_g \lambda_g \cdot g} \enbrace*{\sum\nolimits_g \mu_g \cdot g} = \sum\nolimits_g \enbrace*{\sum\nolimits_{g= a b} \lambda_a \cdot \mu_b} \cdot g
	\]
	Insbesondere ist $g \cdot h = gh \in \mathbb{Z}\pi$.
\end{definition}

\begin{bemerkung}
	Sei $X$ ein zusammenhängender Raum mit universeller Überlagerung $\tilde{X} \to X$.
	Dann wirkt $\pi \coloneqq \pi_1(X)$ auf $\tilde{X}$.
	Durch diese Wirkung wird $C^\sing_*(\tilde{X})$ zu einem Kettenkomplex von $\mathbb{Z}\pi$-Moduln.
\end{bemerkung}

\begin{bemerkung}
	Achtung: Betrachtet man die Überlagerung $\tilde{X}=\mathbb{R} \to S^1 = X$.
	Dann ist die Homologie von $C_*^\sing(\tilde{X})$ einfach $H_*(\mathbb{R})$. Andererseits liefert $C_*^\sing(X) \otimes_\mathbb{Z} \mathbb{Z}\pi$ in Homologie
	$H_*(S^1) \otimes \mathbb{Z}\pi$.
	Es ist gilt also
	\[
		C_*^\sing(\tilde{X}) \not\simeq C_*^\sing(X) \otimes_\mathbb{Z} \mathbb{Z}\pi
	\]
\end{bemerkung}

\begin{bemerkung}
	Sei $E$ ein zusammenhängender CW-Komplex.
	Sei $p \colon \tilde{E} \to E$ die universelle Überlagerung von $E$.
	Dann ist auch $\tilde{E}$ ein CW-Komplex (mit $\tilde{E}^\ssbrace{k} \coloneqq \pi^{-1}(E^\ssbrace{k})$).
	Die Wirkung von $\pi = \pi_1(E)$ auf $\tilde{E}$ ist zellulär.
	Die induzierte Wirkung von $\pi$ auf der Menge der $k$-Zellen ist frei.
	Die Projektion $p \colon \tilde{E} \to E$ identifiziert 
	\(
		\nicefrac{\set*{\text{$k$-Zellen von } \tilde{E}}}{\pi}
	\)
	mit den $k$-Zellen von $E$.
	Insbesondere ist der zelluläre Kettenkomplex von $\tilde{E}$ ein Komplex von freien $\mathbb{Z}\pi$-Moduln.
	Ist $E$ ein endlicher CW-Komplex, so ist $C^{\cell}_*(\tilde{E})$ als $\mathbb{Z}\pi$-Kettenkomplex von endlichem Typ.
\end{bemerkung}

\begin{definition}
	Sei $X$ ein endlich dominierter Raum mit universeller Überlagerung $\tilde{X}\to X$.
	Sei $\pi \coloneqq \pi_1(X)$ die Fundamentalgruppe.
	Dann ist $C_*^\sing(\tilde{X})$ ein endlich dominierter $\mathbb{Z}\pi$-Kettenkomplex, wie wir gleich zeigen werden, und wir definieren
	\[
		\tilde{\chi}(X) \coloneqq \tilde{\chi} \enbrace*{C^\sing_*(\tilde{X})} \in \tilde{K}_0(\mathbb{Z}\pi)
	\]
\end{definition}
\todo{ref}
\todo[inline]{cell und sing aneinander anpassen!}
\begin{lemma}
	Sei $E$ ein CW-Komplex mit universeller Überlagerung $\tilde{E} \to E$.
	Sei $\pi \coloneqq \pi_1(E)$ die Fundamentalgruppe.
	Dann sind $C_*^\sing(\tilde{E})$ und $C_*^{\cell}(\tilde{E})$ als $\mathbb{Z}\pi$-Kettenkomplexe kettenhomotopieäquivalent.
\end{lemma}
\begin{beweis}
	Wir wissen schon, dass die Homologie der beiden Kettenkomplexe übereinstimmt.
	Sei 
	\[
		D_k \coloneqq \set*{\sigma \in C_k^\sing(\tilde{E}^\ssbrace{k}) \given \partial \sigma \in C^\sing_{k-1} \enbrace*{\tilde{E}^\ssbrace{k-1}}}
	\]
	Dann ist $\partial(D_k) \subseteq D_{k-1}$ und wir erhalten einen Unterkomplex $D_* \subseteq C_*^\sing(E)$.
	Weiter betrachten wir die Projektion 
	\[
		D_k \longrightarrow \ker \enbrace*{\partial \colon C_k^\sing \enbrace{\tilde{E}^\ssbrace{k}, \tilde{E}^\ssbrace{k-1}} \to C^\sing_{k-1} \enbrace{\tilde{E}^\ssbrace{k}, \tilde{E}^\ssbrace{k-1}}} \longrightarrow H_k \enbrace*{\tilde{E}^\ssbrace{k}, \tilde{E}^\ssbrace{k-1}} = C_k^{\cell}(\tilde{E})
	\]
	Wir erhalten eine Kettenabbildung $p \colon D_* \to C_*^{\cell}(\tilde{E})$.
	Es bleibt zu zeigen:
	\begin{enumerate}[(1),itemsep=1pt]
		\item $D_k$ ist projektiv für jedes $k$ über $\mathbb{Z}\pi$,
		\item die Inklusion $D_* \to C_*^\sing(\tilde{E})$ induziert einen Isomorphismus in Homologie und
		\item die Projektion $D_* \to C_*^{\cell}(\tilde{E})$ induziert einen Isomorphismus in Homologie.
	\end{enumerate}
	Wir zeigen an dieser Stelle nur (1): 
	Der Komplex $C^\sing_* \enbrace*{\tilde{E}^\ssbrace{k}, \tilde{E}^\ssbrace{k-1}}$ hat keine Homologie in Grad kleiner als $k$.
	Es folgt, dass $Z_k = \ker \partial \colon C^\sing_k\enbrace*{\tilde{E}^\ssbrace{k}, \tilde{E}^\ssbrace{k-1}} \to C^\sing_k \enbrace*{\tilde{E}^\ssbrace{k}, \tilde{E}^\ssbrace{k-1}}$ projektiv ist:
	\[
		\begin{tikzcd}[sep=1.9em]
			0 & C_0^\sing \enbrace*{\tilde{E}^\ssbrace{k},\tilde{E}^\ssbrace{k-1}} \lar & C_1^\sing\enbrace*{\tilde{E}^\ssbrace{k},\tilde{E}^\ssbrace{k-1}} \lar[two heads]& \ldots \lar & C_k^\sing \enbrace*{\tilde{E}^\ssbrace{k},\tilde{E}^\ssbrace{k-1}} \lar & Z_K \lar & 0 \lar 
		\end{tikzcd}
	\]
	Induktiv können wir diese exakte Folge verkürzen (indem wir die Surjektion in einem projektiven Moduln spalten) und erhalten, dass $Z_k$ ein direkter Summand in $C^\sing_k \enbrace*{\tilde{E}^\ssbrace{k},\tilde{E}^\ssbrace{k-1}}$ und damit projektiv ist.
	Die Projektivität von $D_k$ folgt nun aus der kurzen exakten Folge
	\[
		\begin{tikzcd}
			0 \rar & C^\sing_k \enbrace*{\tilde{E}^\ssbrace{k-1}}\rar & D_k \rar[two heads] & Z_k \rar & 0 
		\end{tikzcd} \qedhere
	\]
\end{beweis}

\begin{lemma}[label=lem:endl_dom_iso_fundamental]
	Sei $X$ endlich dominiert und wegzusammenhängend.
	Dann gibt es eine endliche Dominierung 
	\(
		\begin{tikzcd}[cramped,sep=small] 
			X \rar["i"] & E \rar["p"] & X 
		\end{tikzcd}
	\)
	mit $E$ wegzusammenhängend, so dass $p$ und $i$ Isomorphismen von Fundamentalgruppen induzieren.
\end{lemma}
\begin{beweis}
	Offenbar ist $p_* \colon \pi_1(E) \to \pi_1(X)$ surjektiv.
	Ist $\benbrace*{\alpha \colon S^1 \to E}$ im Kern von $p_*$, so gibt es eine stetige Fortsetzung $\beta \colon D^2 \to X$ von $p \circ \alpha$ und wir können dann $p$ auf $E \cup_\alpha D^2$ fortsetzen.
	Es ist dann $\pi_1(E \cup_\alpha D^2) \cong \sfrac{\pi_1(E)}{\langle\langle \alpha\rangle\rangle}$.
	Es bleibt zu zeigen: $\ker p_*$ ist normal endlich erzeugt.
	
	Da $p_*$ durch $i_*$ gespalten wird, ist $\pi_1(E) \cong \ker p_* \rtimes \pi_1(X)$.
	Da $E$ ein endlicher CW-Komplex ist, ist $\pi_1(E)$ endlich erzeugt.
	Wir finden dann auch ein endliches Erzeugendensystem 
	\[
		S = S_K \cup S_X
	\]
	mit $S_K \subseteq \ker p_*$ und $S_X \subseteq \pi_1(X)$.
	Jedes Element von $G$ lässt sich schreiben als Produkt von Elementen in $S_K$ und $S_X$.
	Da für $a \in S_X$ und $b \in S_K$ gilt $a \cdot b = b^a \cdot a$, lässt sich jedes $g \in \pi_1(E)$ schreiben als $g = \tilde{b} \cdot \tilde{a}$ mit $\tilde{b} \in \langle\langle S_K \rangle\rangle$, $\tilde{a} \in \langle S_X \rangle$.
	Ist $g \in \ker p_*$, so folgt $g \in \langle\langle S_K \rangle\rangle$.
	Also wird $\ker p_*$ normal von $S_K$ erzeugt.
\end{beweis}

\begin{bemerkung}
	Sei $X$ endlich dominiert mit universeller Überlagerung $\tilde{X} \to X$.
	Sei $\pi = \pi_1(X)$.
	Dann ist $C^\sing_*(\tilde{X})$ als $\mathbb{Z}\pi$-Kettenkomplex endlich dominiert.
	Sei \(
		\begin{tikzcd}[cramped,sep=small] 
			X \rar["i"] & E \rar["p"] & X 
		\end{tikzcd}
	\) wie in \autoref{lem:endl_dom_iso_fundamental}.
	Dann wird $C^\sing_*(\tilde{X})$ durch $C^\sing_*(\tilde{E})$ dominiert.
	Nun ist $C^\sing_*(\tilde{E}) \cong C^{\cell}_*(\tilde{E})$ und daher wird $C^\sing_*(\tilde{X})$ durch den Kettenkomplex $C^{\cell}_*(\tilde{E})$ von endlichem Typ von freien $\mathbb{Z}\pi$-Moduln dominiert.
\end{bemerkung}

\begin{definition}[{name=[{Walls Endlichkeitshindernis}]}]
	Sei $X$ ein endlich dominiert wegzusammenhängender Raum.
	Sei $\pi \coloneqq \pi_1 X$.
	\Index{Walls Endlichkeitshindernis} ist definiert als 
	\[
		\tilde{\chi}(X) \coloneqq \tilde{\chi} \enbrace*{C^\sing_*(\tilde{X})} \in \tilde{K}_0(\mathbb{Z} \pi)
	\]
\end{definition}

\begin{satz}[name={Wall},label=satz:wall]
	Sei $X$ endlich dominiert und wegzusammenhängend.
	Dann ist $X$ genau dann homotopieäquivalent zu einem endlichen CW-Komplex, wenn $\tilde{\chi}(X)=0 \in \tilde{K}_0(\mathbb{Z}\pi)$ ist.
\end{satz}
\begin{beweis}[{name={der einfachen Richtung}}]
	Ist $X$ homotopieäquivalent zu einem endlichen CW-Komplex $E$, so ist $C^\sing_*(\tilde{X}) \simeq C^{\cell}_*(\tilde{E})$ und 
	\[
		\tilde{\chi} \enbrace*{C^\sing_*(\tilde{X})} = \tilde{\chi} \enbrace*{C^{\cell}_*(\tilde{E})} = 0 \in \tilde{K}_0(\mathbb{Z}\pi) \qedhere
	\]
\end{beweis}

\begin{definition}
	Eine stetige Abbildung $f \colon X \to Y$ heißt \bet{$r$-zusammenhängend}\index{r-zusammenhängend@$r$-zusammenhängend}, falls $f_* \colon \pi_k \to \pi_k(Y)$ bijektiv ist für $k < r$ und surjektiv für $k =r$.
\end{definition}

\begin{definition}
	Für $f \colon X \to Y$ stetig definieren wir 
	\[
		\pi_k(f) \coloneqq \pi_k \enbrace*{\Zyl(f),X} \qquad \quad H_k(f) \coloneqq H_k \enbrace*{\Zyl(f),X}
	\]
\end{definition}

\begin{bemerkung}
	\begin{enumerate}[1)]
		\item Für $(X,A)$ ist $\pi_k(X,A)$ definiert als Homotopieklassen von punktierten Abbildungen $(D^k,S^{k-1}) \to (X,A)$.
		\item Der Satz von \textsc{Hurewicz} besagt, dass für einfach zusammenhängendes $X$ und $A$ die Hurewicz-Abbildung $\pi_r (X,A) \to H_r(X,A)$, $f \mapsto f_* (\mu_{D^r,S^{r-1}})$ ein Isomorphismus ist, falls $\pi_k(X,A)=0$ für $k < r$.
		\item Es gibt eine lange exakte Folge
		\[
			\begin{tikzcd}
				\ldots \rar & \pi_k(A) \rar & \pi_k(X) \rar & \pi_k(X,A) \rar & \pi_{k-1}(A) \rar & \ldots 
			\end{tikzcd}
		\]
		Insbesondere ist $f$ genau dann $r$-zusammenhängend, wenn $\pi_k(f)=0$ für $k <r$.
		\item Ist $f$ $r$-zusammenhängend und $r \ge 2$, so gilt $\pi_{r+1}(f) \cong \pi_{r+1}(\tilde{f}) \cong H_{r+1}(\tilde{f})$ nach Hurewicz.
		\[
			\begin{tikzcd}
				\tilde{X} \rar["\tilde{f}"] \dar & \tilde{Y} \dar \\
				X \rar["f"] & Y
			\end{tikzcd}
		\]
		\item Alternativ können wir $\pi_k(f)$ beschreiben als Homotopieklassen von Paaren von punktierten Abbildungen $\enbrace*{\phi \colon D^k \to Y, \varphi \colon S^{k-1} \to X}$ mit $\phi|_{S^{k-1}} = f \circ \varphi$
		\[
			\begin{tikzcd}
				S^{k-1} \dar[hook] \rar["\varphi"] & X \dar["f"] \\
				D^k \rar["\phi"] & Y
			\end{tikzcd}
		\]
	\end{enumerate}
\end{bemerkung}

\begin{lemma}
	Sei $f \colon X \to Y$ $r$-zusammenhängend und $r \ge 2$, $\benbrace*{\phi \colon D^{r+1} \to Y, \varphi \colon S^{r} \to X} \in \pi_{r+1}(f) \cong H_{r+1}(\tilde{f})$ und $X_\varphi \coloneqq X \cup_{\varphi} D^{r+1}$ und $f_\varphi \coloneqq f \cup_\varphi \phi \colon X_\varphi \to Y$.
	Weiter sei $\pi \coloneqq \pi_1 X = \pi_1 Y$ und $\mathbb{Z}\pi \benbrace*{\phi,\varphi} \subseteq \pi_{r+1}(f)$ der von $\benbrace*{\phi,\varphi} \in \pi_{r+1}(f)$ erzeugte $\mathbb{Z}\pi$-Untermodul.
	Dann induziert die Inklusion $X \hookrightarrow X_\varphi$ einen Isomorphismus 
	\[
		\pi_{r+1}(f_\varphi) \cong \nicefrac{\pi_{r+1}(f)}{\mathbb{Z}\pi \benbrace*{\phi,\varphi}}
	\]
\end{lemma}
\begin{beweis}
	Nachdem wir $Y$ durch $\Zyl(f)$ ersetzen, können wir annehmen, dass $f \colon X \hookrightarrow Y$ eine Inklusion ist. 
	Weiter ersetzen wir $X$ durch $\Zyl(\varphi)$ und $Y$ durch $\Zyl(\phi)$.
	Wir können nun annehmen, dass $X \subseteq X_\varphi \subseteq Y$ und $\tilde{X} \subseteq \tilde{X_\varphi} \subseteq \tilde{Y}$.
	Nach Voraussetzung ist 
	\[
		H_*(\widetilde{X_\varphi},\widetilde{X}) \cong H_*(\pi \times D^{r+1}, \pi \times S^r) \cong \bigoplus_\pi H_*(D^{r+1},S^r) \cong \begin{cases}
			\mathbb{Z}\pi &\text{ falls }*=r+1\\
			0 &\text{ sonst}
		\end{cases}
	\]
	Nun benutzen wir die lange exakte Folge zu $\enbrace*{\tilde{X} \subseteq \tilde{X_\varphi} \subseteq \tilde{Y}}$ und erhalten
	\[
		\begin{tikzcd}
			\mathbb{Z}\pi \cong H_{r+1} \enbrace*{\widetilde{X_\varphi}, \widetilde{X}} \rar H_{r+1} \enbrace*{\tilde{Y},\tilde{X}} \rar & H_{r+1} \enbrace*{\tilde{Y},\tilde{X_\varphi}} \rar & H_{r} \enbrace*{\tilde{X_\varphi},\tilde{X}}
		\end{tikzcd}
	\]
	Nach Konstruktion wird das Bild von $H_{r+1}(\tilde{X_\varphi},\tilde{X})$ in $H_{r+1} \enbrace*{\tilde{Y},\tilde{X_\varphi}}$ von $\benbrace*{\phi,\varphi}$ als $\mathbb{Z}\pi$-Modul erzeugt und die Behauptung folgt.
\end{beweis}

\begin{beweis}[{name={zweite Richtung von \autoref{satz:wall}}}]
	Ohne Einschränkungen sei $X$ ein CW-Komplex.
	Da nach Voraussetzung $\tilde{\chi} \enbrace*{C^\sing_*(\tilde{X})}=0$ ist, gibt es einen Kettenkomplex $D_*$ von endlichem Typ von freien $\mathbb{Z}\pi$-Moduln, der kettenhomotopieäquivalent zu $C_*^\sing(\tilde{X})$ und damit zu $C^\cell_*(\tilde{X})$ ist.
	Sei $f \colon D_* \to C_*^\cell(\tilde{X})$ eine solche Kettenhomotopieäquivalenz.
	
	Nun zeigen wir induktiv für alle $r \ge 2$.
	Es gibt einen $r$-dimensionalen endlichen CW-Komplex $E$ und eine $r$-zusammenhängende zelluläre Abbildung $g \colon E \to X$, so dass 
	\begin{enumerate}[(i)]
		\item $C_*^\cell(\tilde{E}) = D_{* \le r}$
		\item $C_*^\cell(\tilde{g}) = f_{* \le r}$
	\end{enumerate}
	Induktionsanfang $r=2$ (Skizze): Wir wissen, dass es eine endliche Dominierung $\begin{tikzcd}[cramped,sep=small] X \rar["i"] & E \rar["p"] & Y \end{tikzcd}$ gibt, sodass $p_* \colon \pi_1 E \to \pi_1 X$ ein Isomorphismus ist.
	Indem wir $E$ durch $E \cup \bigvee_{i=1}^r S^2$ ersetzen, können wir erreichen, dass $p_* \colon \pi_2 E \to \pi_2 X$ surjektiv ist (da $i_* \colon \pi_2 X \to \pi_2 E$ ein Spalt ist).
	Nun man $f_*$ so ändern, dass (i) und (ii) gelten.
	
	Induktionsschritt $r \mapsto r+1$: Sei $g \colon E \to X$ durch die Induktionsvoraussetzung gegeben.
	Wir ersetzen $X$ durch $\Zyl(g)$ und nehmen an, dass $E$ ein Unterkomplex von $X$ ist und dass $g$ die Inklusion $E\hookrightarrow X$ ist.
	Wir betrachten den Kegel $\tilde{g}_* \colon C^\cell_*(\tilde{E}) \to C^\cell_*(\tilde{X})$.
	Die Projektion $\Keg(\tilde{g}_*) \to C^\cell_*(\tilde{X})$ induziert eine Kettenabbildung $\Keg(\tilde{g}_*) \to C^\cell_*(\tilde{X},\tilde{E})$ und mit Hilfe der langen exakten Folge in Homologie sehen wir, dass diese einen Isomorphismus in Homologie induziert.
	Wegen $C^\cell_*(\tilde{E}) = D_{* \le r}$ ist $\Keg(\tilde{g}_*)$ ein Unterkomplex der zyklischen Kettenkomplexes $\Keg(\tilde{f}_*)$.
	\missingfigure{Diagramm mit direktem Summer übereinander}
	Es folgt, dass $\binom{f_*}{-\partial}\colon D_{r+1} \to \ker \enbrace*{\partial_{r+1}^{\Keg(\tilde{g}_*)}} \to H_{r+1}(\tilde{g}_*)$ surjektiv ist.
	Wähle nun eine $\mathbb{Z}\pi$-Basis $b_1, \ldots ,b_l$ von $D_{r+1}$ und realisiere die Bilder in $H_{r+1}(\tilde{g}_*)=\pi_{r+1}(g_*)$ durch  $(\phi_1,\varphi_1), \ldots , (\phi_l,\varphi_l) \colon (D^{r+1},S^r) \to \tilde{X},\tilde{E}$.
	Nun liefert die $l$-fache Anwendung der Konstruktion aus dem Lemma die gesuchte $(r+1)$-zusammenhängende Abbildung $g \cup \phi_1 \coprod \ldots \coprod \phi_l \colon E \cup_{\varphi_1 \coprod \ldots \coprod \varphi_l} \enbrace*{D^{r+1} \coprod \ldots \coprod D^{r+1}} \to X$.
\end{beweis}

\begin{bemerkung}
	Nach dem Satz von \textsc{Whitehead} ist eine Abbildung $f \colon X \to Y$ zwischen CW-Komplexen genau dann eine Homotopieäquivalenz, wenn sie (für alle Basispunkte in $X$ und) alle $k$ Isomorphismen $f_* \colon \pi_k(X) \to \pi_k(Y)$ induziert.
	
	Zum Beweis von Walls Satz: Sei $X$ endlich dominiert.
	Ohne Einschränkungen sei $X$ ein CW-Komplex.
	Wir haben einen endlichen CW-Komplex $E$ konstant in $f$ (?) und eine Abbildung $g \colon E \to X$ die $r$-zusammenhängend ist für alle $r$.
	Nach Whitehead ist $g$ dann eine Homotopieäquivalenz.
\end{bemerkung}

\begin{bemerkung}
	Es gelten folgende Aussagen, die allerdings nicht trivial zu zeigen sind:
	\begin{enumerate}[1)]
		\item Für eine endliche Gruppe $G$ ist auch $\tilde{K}_0(\mathbb{Z}[G])$ endlich.
		\item Für endliches $G$ ist $\tilde{K}_0(\mathbb{Z}[G]) \to \tilde{K}_0(\mathbb{Q}[G])$ immer die Nullabbildung.
		\item Sei $p$ eine Primzahl, $C_p = \sfrac{\mathbb{Z}}{p \mathbb{Z}}$ und $\xi = \exp(\sfrac{2 \pi i}{p}) \in \mathbb{C}$.
		Dann ist $\mathbb{Z}[\xi] \subseteq \mathbb{Q}(\xi)$ ein Dedekindring. 
		Der Ringhomomorphismus $\mathbb{Z}[C_p] \to \mathbb{Z}[\xi]$, der den Erzeugern von $C_p$ die Einheitswurzel $\xi$ zuordnet, induziert einen Isomorphismus auf $\tilde{K}_0$. Also
		\[
			\tilde{K}_0 \enbrace[\big]{\mathbb{Z}[C_p]} = \mathrm{Cl} \enbrace*{\mathbb{Z}[\xi]}
		\]
		Aber $\mathrm{Cl}(\mathbb{Z}[\xi])$ ist nur für endlich viele $p$ explizit bekannt:
		\begin{table}[htb!]
			\caption{$\mathrm{Cl}(\mathbb{Z}[\xi])$ für einige bekannte $p$}
			\label{table:bem:cyclo_p}
			\centering
			\begin{tabular}{p{2cm}cccccc}
				\toprule
				$p$ & $\le 19$ & $23$ & $29$ & $31$ & $37$ & $41$\\
				\midrule
				$\mathrm{Cl}(\mathbb{Z}[\xi])$ & $0$ &  $\sfrac{\mathbb{Z}}{3 \mathbb{Z}}$ & $\enbrace*{\sfrac{\mathbb{Z}}{3 \mathbb{Z}}}^3$ & $\sfrac{\mathbb{Z}}{9 \mathbb{Z}}$ & $\sfrac{\mathbb{Z}}{37 \mathbb{Z}}$ &  $\sfrac{\mathbb{Z}}{11 \mathbb{Z}} \oplus \sfrac{\mathbb{Z}}{11 \mathbb{Z}}$\\
				\bottomrule
			\end{tabular}
		\end{table}

	\end{enumerate}
\end{bemerkung}
% section 4 (end)
\newpage

\section{Waldhausen Kategorien} % (fold)
\label{sec:5}

\begin{definition}[{name=[{Waldhausen Kategorie}]}]
	Sei $\mathcal{C}$ eine Kategorie zusammen mit zwei ausgezeichneten Familien von Morphismen, den \bet{Kofaserungen}\index{Kofaserung} $\co \mathcal{C}$ und den \bet{schwachen Äquivalenzen}\index{schwache Äquivalenz} $\wOp \mathcal{C}$.
	Beide Familien sollten die Isomorphismen enthalten und abgeschlossen sein unter Komposition.
	Dann heißt $\mathcal{C}$ -- oder genauer $(\mathcal{C},\co \mathcal{C},\wOp \mathcal{C})$ -- eine \Index{Waldhausen-Kategorie}, falls folgende Axiome erfüllt sind:
	\begin{enumerate}[(i)]
		\item[(W1)] Es gibt ein \Index{Nullobjekt} $0 \in \mathbb{C}$, also ein Objekt $0$, dass sowohl initial als auch terminal ist, das heißt zu jedem Objekt $C$ in $\mathcal{C}$ gibt es jeweils genau einen Morphismus $0 \to C$ und $C \to 0$. 
		Wir verlangen weiter, dass $0 \hookrightarrow C$ immer eine Kofaserung ist.
		\item[(W2)] Sei $\!\begin{tikzcd}[sep=small] C & A \lar \rar[hook] & B \end{tikzcd}\!$ ein Diagramm in $\mathcal{C}$, sodass $A \hookrightarrow B$ eine Kofaserung ist.
		Dann existiert der Pushout
		\[
			\begin{tikzcd}
				A \rar[hook] \dar & B \dar  \\
				C \rar[hook] & C \cup_A B
			\end{tikzcd}
		\]
		und $C \hookrightarrow C \cup_A B$ ist eine Kofaserung.
		\item[(W3)] [Verkleben von schwachen Äquivalenzen] Sei 
		\[
			\begin{tikzcd}
				C \dar["\sim"] & A \lar \dar["\sim"] \rar[hook] & B \dar["\sim"] \\
				C' & A' \lar \rar[hook] & B'
			\end{tikzcd}
		\] 
		sodass $A \hookrightarrow B$, $A' \hookrightarrow B'$ Kofaserungen sind und die vertikalen Abbildungen schwache Äquivalenzen sind.
		Dann ist auch die induzierte Abbildung $B \cup_A C \xrightarrow{\sim} B' \cup_{A'} C'$ eine schwache Äquivalenz.
	\end{enumerate}
	Eine Waldhausen-Kategorie heißt \bet{saturiert}\index{Waldhausen-Kategorie!saturierte}, falls für komponierbare Morphismen $f \colon A \to B$, $g \colon B \to C$ gilt:
	Sind zwei der drei Morphismen $f$, $g$ und $ g \circ f$ schwache Äquivalenzen, so sind auch alle drei schwache Äquivalenzen.
\end{definition}

\begin{bemerkung}
	W1 und W2 haben die folgenden Konsequenzen:
	\begin{enumerate}[1)]
		\item Für $A,B \in \mathcal{C}$ existiert das Koprodukt $A \amalg B$ mittels
		\(
			\begin{tikzcd}
				0 \rar \dar & A \dar \\
				B \rar & A \amalg B
			\end{tikzcd}
		\)
		\item Für jede Kofaserung $A \hookrightarrow B$ existiert der Kokern $\sfrac{B}{A}$ mittels
		\(
			\begin{tikzcd}
				A \rar[hook] \dar & B \dar \\
				0 \rar & \sfrac{B}{A}
			\end{tikzcd}
		\)
		
		Wir nennen dann $A \hookrightarrow B \twoheadrightarrow \sfrac{B}{A}$ eine Kofaserfolge.
	\end{enumerate}
\end{bemerkung}

\begin{beispiel}
	\leavevmode
	\begin{enumerate}[(i)]
		\item Sei $R$ ein Ring, $\mathcal{P}_R$ die Kategorie der endlich erzeugten projektiven $R$-Moduln.
		Dann wird $\mathcal{P}_R$ zu einer Waldhausen-Kategorie, in der die Kofaserungen die Inklusionen von direktn Summanden sind und die schwachen Äquivalenzen die Isomorphismen. (Alternativ andere Kategorien von $R$-Moduln)
		\item Sei $R$ ein Ring.
		Die Kategorie $\ch_f \mathcal{P}_R$ der $R$-Kettenkomplexe von endlichem Typ von projektiven $R$-Moduln wird zu einer Waldhausen-Kategorie, in der die Kofaserungen die Kettenabbildungen sind, die in jedem Grad die Inklusion eines direkten Summanden sind, und die schwachen Äquivalenzen die Kettenhomotopieäquivalenzen sind. 
		\item Sei $\mathcal{R}$ die Kategorie der punktierten CW-Komplexe (mit zellulären punktierten Abbildungen).
		Dann wir $\mathcal{R}$ zu einer Waldhausen-Kategorie in der die Kofaserungen die Inklusionen von Unterkomplexen und die schwachen Äquivalenzen die Homotopieäquivalenzen sind.
		
		Genauso wird die Kategorie der endlichen punktierten CW-Komplexe zu einer Waldhausen-Kategorie.
	\end{enumerate}
\end{beispiel}

\begin{definition}[{name=[{schwache Äquivalenz von Objekten einer Waldhausen-Kategorie}]}]
	Wir sagen, dass Objekte $C$ und $C'$ einer Waldhausen-Kategorie $\mathcal{C}$ \Index{schwach äquivalent} sind, falls es ein \Index{Zick-Zack} von schwachen Äquivalenzen in $\mathcal{C}$ zwischen $C$ und $C'$ gilt:
	\[
		\begin{tikzcd}[row sep=small]
			& C_1 \dlar["\sim"'] \drar["\sim"] & & C_2 \dlar["\sim"'] \drar["\sim"] && \ldots \dlar["\sim"'] \drar["\sim"] && C_n \dlar["\sim"'] \drar["\sim"]\\
			C && C_1' && C_2'&& C_{n-1}' && C'
		\end{tikzcd}
	\]
\end{definition}

\begin{definition}[{name=[K-Theorie einer Waldhausen-Kategorie]}]
	Sei $\mathcal{C}$ eine Waldhausen-Kategorie, in der die schwachen Äquivalenzklassen eine Menge bilden.
	Dann definieren wir $K_0 \mathcal{C}$ als die abelsche Gruppe, die von schwachen Äquivalenzklassen $[C]_\sim$ in $\mathcal{C}$ erzeugt wird und die folgenden Relationen erfüllt: 
	Für jede Kofaserfolge $\begin{tikzcd}[cramped,sep=small] B \rar[hook] & C \rar & \sfrac{C}{B} \end{tikzcd}$ in $\mathcal{C}$ gilt
	\[
		[B]_\sim + \benbrace*{\sfrac{C}{B}}_\sim = \benbrace*{C}_\sim \in K_0 \mathcal{C}
	\]
	Ab sofort: Wann immer nötig betrachten wir nur Waldhausen-Kategorien, deren schwache Äquivalenzklassen von Objekten eine Menge sind.
\end{definition}

\begin{bemerkung}
	Ein Funktor $F \colon \mathcal{C} \to \mathcal{C}'$ zwischen Waldhausen-Kategorien heißt \Index{exakt}, wenn er Kofaserungen, schwache Äquivalenzen und Pushouts von Kofaserungen erhält.
	Die letzte Bedingung bedeutet: Ist 
	\[
		\begin{tikzcd}
			A \rar[hook] \dar & B \dar \\
			C \rar[hook] & B \cup_A C
		\end{tikzcd}
	\] 
	ein Pushoutdiagramm in $\mathcal{C}$, sodass $A \hookrightarrow B$ eine Kofaserung ist, so ist
	\[
		\begin{tikzcd}
			F(A) \rar[hook] \dar & F(C) \dar\\
			F(B) \rar[hook] & F(B \cup_A C)
		\end{tikzcd}
	\]
	ein Pushoutdiagramm in $\mathcal{C}'$.
	Insbesondere erhält $F$ dann Koprodukte und Kofaserungen und induziert einen Homomorphismus $F_* \colon K_0(\mathcal{C}) \to K_=(\mathcal{C}')$.
\end{bemerkung}

\begin{proposition}
	Sei $R$ ein Ring.
	Dann ist $K_0(R) \cong K_0(\mathcal{P}_R) \cong K_0 (\ch_f \mathcal{P}_R)$.
\end{proposition}

\todo[inline]{hier fehlt noch eine halbe Vorlesung!}

% section 5 (end)

\cleardoubleoddemptypage
\pagenumbering{Alph}
\setcounter{page}{1}
\cleardoubleoddemptypage
\appendix

\section{Anhang} % (fold)
\label{sec:anhang}
%!TEX root = ana_top_geo.tex

\subsection{Ausführlicher Beweis zu \cref{lem:kpt-schnitte}} % (fold)
\label{sub:kpt-schnitte}
Sei $X$ ein Hausdorffraum. Dann ist $X$ genau dann kompakt, wenn gilt: Hat eine Familie $\mathcal{A}$ von abgeschlossenen Teilmengen von $X$ die endliche 
Durchschnittseigenschaft, so gilt 
\[
	\bigcap_{A \in \mathcal{A}} A \not= \emptyset.
\]
\begin{beweis}
	Für die erste Implikation sei $X$ kompakt und $\mathcal{A}$ eine Familie von abgeschlossenen Mengen mit der endlichen Durchschnittseigenschaft.
	Angenommen $\bigcap_{A \in \mathcal{A}} A = \emptyset$.
	Dann gilt
	\[
		X = X \setminus \bigcap_{A \in \mathcal{A}} A = \bigcup_{A \in \mathcal{A}} X \setminus A.
	\]
	Nun ist $\mathcal{U} \coloneqq \set*{X \setminus A \given A \in \mathcal{A}}$ eine offene Überdeckung von $X$ und da $X$ kompakt ist, existiert $\mathcal{A}_0 \subset \mathcal{A}$ endlich, sodass
	\[
		X = \bigcup_{A \in \mathcal{A}_0} X \setminus A = X \setminus \underbrace{\bigcap_{A \in \mathcal{A}_0 } A }_{\neq \emptyset} \quad \light
	\]
	Für die umgekehrte Implikation sei nun $\mathcal{U} = \set{U_i}_{i \in I}$ eine offene Überdeckung von $X$.
	Angenommen für jede endliche Teilmenge $J \subseteq I$ gilt $X \neq \bigcup_{i \in J} U_i$.
	Betrachte nun $\mathcal{A} =  \set{X \setminus U_i}_{i \in I}$. Dann gilt nach Annahme
	\[
		\bigcap_{i \in J} X \setminus U_i = X \setminus \bigcup_{i \in J} U_i \neq \emptyset.
	\]
	Also hat $\mathcal{A}$ die endliche Durchschnittseigenschaft. Nach Vorraussetzung gilt dann
	\[
		\emptyset \not= \bigcap_{i \in I} X \setminus U_i = X \setminus \underbrace{\bigcup_{i \in I} U_i}_{= X} \quad \light \qedhere
	\]
\end{beweis}


\subsection[Blatt3, Aufgabe 4: Hilfssatz für den Hauptsatz der Algebra]{Blatt 3, Aufgabe 4} % (fold)
\label{sub:B3A4}
\emph{Diese Übungsaufgabe ist zentral für den Beweis des Hauptsatzes der Algebra, \cref{satz:hauptsatz-algebra}.} 

Sei $p(x)= x^n + a_{n-1} x^{n-1} + \ldots + a_1 x + a_0$ mit $n \in \mathbb{N}_0$ ein Polynom mit Koeffizienten $a_i \in \mathbb{C}$, dass \emph{keine} Nullstelle in $\mathbb{C}$ besitzt. 
Sei $S^1= \set*{z \in \mathbb{C} \given \abs*{z}=1}$.
\begin{enumerate}[(a)]
	\item $f \colon S^1 \to S^1$ gegeben durch $f(z) = \frac{p(z)}{\abs*{p(z)} } $ ist wohldefiniert und homotop zu einer konstanten Abbildung.
	\item $f$ ist homotop zur Abbildung $g_n \colon S^1 \to S^1$ mit $g_n(z)= z^n$.
\end{enumerate}
\minisec{Beweis}
\begin{enumerate}[(a)]
	\item \begin{description}
		\item[Wohldefiniertheit:] Sei $z \in S^1$ beliebig. Dann gilt
		\[
			\abs*{\frac{p(z)}{\abs*{p(z)} } } = \frac{1}{\abs*{p(z)} } \cdot \abs*{p(z)} =1,
		\]
		also ist $f(z) \in S^1$.
		\item[Homotop zu einer konstanten Abbildung:] Definiere $f_t \colon S^1 \to S^1$ für $t \in [0,1]$ durch 
		\[
			f_t(z) = \frac{p(t \cdot z)}{\abs*{p(t \cdot z)} } 
		\]
		Dies ist mit der gleichen Begründung wie oben wohldefiniert. 
		Außerdem ist $f_0(z)= \frac{a_0}{\abs*{a_0} } \in S^1 $ konstant und $f_1(z)= \frac{p(z)}{\abs*{p(z)} }=f(z)$. 
		Definiere nun $H \colon S^1 \times [0,1] \to S^1$ durch $H(x,t) \coloneqq f_t(x)$. 
		Dann ist $H$ stetig, da Polynome und $\abs*{.} $, sowie Multiplikation stetig sind. 
		$H$ ist die gesuchte Homotopie.
	\end{description}
	\item Sei $h \colon S^1 \times [0,1] \to \mathbb{C}$ gegeben durch $h(z,t) = z^n + \sum_{k=0}^{n-1} a_k z^k t^{n-k}$. 
	Dann gilt $h(z,0)=z^n \not= 0$, da $z \in S^1$.
	Für $t \neq 0$ gilt nun
	\begin{align*}
		h(z,t) = 0 \iff \frac{h(z,t)}{t^n} = 0 \iff \frac{z^n}{t^n} + \sum_{k=0}^{n-1} a_k \frac{z^k}{t^k} = 0 \iff p \enbrace*{\frac{z}{t}} = 0
	\end{align*}
	Aber nach Vorraussetzung gilt $p \enbrace*{\frac{z}{t}} \neq 0$. 
	Also $h(z,t) \neq 0$ für alle $t \in [0,1]$. 
	Definiere nun $H \colon S^1 \times [0,1]\to S^1$ durch $H(z,t) = \frac{h(z,t)}{\abs*{h(z,t)}}$. 
	Wie eben gezeigt, ist dies wohldefiniert und offensichtlich stetig. Da
	\[
		H(z,0) = \frac{z^n}{\abs*{z^n} } = z^n \quad \text{ und } \quad H(z,1) = \frac{h(z,1)}{\abs*{h(z,1)} } = \frac{p(z)}{\abs*{p(z)} } =f(z)
	\]
	ist $H$ die gesuchte Homotopie. \qedhere
\end{enumerate}

\subsection{Blatt 10, Aufgabe 3} % (fold)
\label{sub:B10A3}
\emph{Diese Übungsaufgabe lieferte den Beweis zu \cref{prop:iso-covering}.} \smallskip \\
Sei $p \colon \overline{X} \to X$ eine Überlagerung. 
Seien $\overline{x}_0  \in \overline{X}$ und $x_0= p(\overline{x}_0 )$ Basispunkte. 
Dann ist die induzierte Abbildung $\pi_n (p) \colon \pi_n(\overline{X}, \overline{x}_0) \to \pi_n(X,x_0)$ ein Isomorphismus für alle $n \ge 2$.
\minisec{Beweis}
Als Überlagerung ist $p$ stetig, also ist $\pi_n(p)$ ein Gruppenhomomorphismus nach \hyperref[prop:eig-hom-gruppen:enum:4]{ \cref*{prop:eig-hom-gruppen} \ref*{prop:eig-hom-gruppen:enum:4}}.
\begin{description}
	\item[Surjektivität:] Sei $[\omega] \in \pi_n(X,x_0)$, also $\omega \colon I^n \to X$ mit $\omega(\partial I^n) = \set{x_0}$. Betrachte $\omega$ nun als Abbildung $I^{n-1} \times [0,1] \to X$:
	\[
		\begin{tikzcd}[column sep=4em]
			I^{n-1} \times \set{0} \dar[hook] \rar["\mathrm{const}_{\overline{x}_0}"] & \overline{X} \dar["p"]\\
			I^{n-1} \times I \rar["\omega"] & X  
		\end{tikzcd}
	\]
	$\mathrm{const}_{\overline{x}_0} \colon I^{n-1} \times \set{0}$ ist eine Hebung von $\omega\big|_{I^{n-1} \times \set{0}} \equiv x_0$. 
	Nach dem Homotopiehebungssatz (\ref{satz:hebung-homotopie}) existiert eine Hebung $\overline{\omega} \colon I^{n-1} \times I \to \overline{X}$ von $\omega$ mit $\overline{\omega}\big|_{I^{n-1} \times \set{0}} \equiv \overline{x}_0 $. 
	Also gilt
	\[
		p \circ \overline{\omega} \big|_{\partial I^n} = \omega \big|_{\partial I^n} \equiv x_0 \enspace \Longrightarrow \enspace \overline{\omega} \big|_{\partial I^n} 
		\in p ^{-1}( \set{x_0} ) .
	\]
	Da $p^{-1}(\set{x_0})$ diskret und $\partial I^n$ für $n \ge 2$ zusammenhängend ist, muss $\overline{\omega} \big|_{\partial I^n}$ konstant sein. 
	Da $\overline{\omega}\big|_{I^{n-1} \times \set{0}} \equiv \overline{x}_0 $ gilt, folgt somit $\overline{\omega}(\partial I^n) = \set{\overline{x}_0}$. 
	Also ist $[\overline{\omega}] \in \pi_n(\overline{X},\overline{x}_0)$ und weiter gilt
	\[
		\pi_n(p) \enbrace*{[\overline{\omega}]} = [p \circ \overline{\omega} ] = [\omega] \in \pi_n(X,x_0). 
	\]
	\item[Injektivität:] Sei $[\omega] \in \ker \pi_n(p)$, also $[p \circ \omega] = [c_{x_0}]$. 
	Es existiert also eine Homotopie $H$ relativ $\partial I^n$ zwischen $p \circ \omega$ und $c_{x_0}$. 
	Offensichtlich ist $\omega$ eine Hebung von $p \circ \omega$. 
	Mit dem Homotopiehebungssatz erhalten wir eine Hebung $\overline{H}$ von $H$ mit $\overline{H}(-,0) = \omega$. 
	Weiter wissen wir, dass
	\[
		\overline{H} \big|_{\partial I^n \times [0,1]} \in p ^{-1}(\set{x_0} ) \quad \text{ und }\quad  \overline{H} \big|_{ I^n \times \set{1}} \in p ^{-1}(\set{x_0} )
	\]
	gelten muss, da $H = p \circ \overline{H}$ und $H(-,1)= c_{x_0} \equiv x_0$. 
	Mit dem gleichen Argument wie oben folgt, dass $\overline{H} \big|_{\partial I^n \times [0,1]}$ und $\overline{H} \big|_{ I^n \times \set{1}}$ konstant sind. 
	Für $z \in \partial I^n$ gilt nun
	\[
		\overline{H}(z,0) = \omega(z) = \overline{x}_0
	\]
	Da $\partial I^n \times [0,1] \cap I^n \times \set{1} \not= \emptyset$, muss also auch $\overline{H}(-,1) \equiv \overline{x}_0$ gelten. 
	Damit folgt $[\omega] = [c_{x_0}]$.\qedhere
\end{description}
\printindex
\printbibliography
\listoffigures
\todototoc
\listoftodos[To-do's und andere Baustellen]
\end{document}
