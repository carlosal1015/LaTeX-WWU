%!TEX TS-program = xelatex
%!TEX TS-options = -shell-escape
%!TEX root = ../OpAlg2_SS16/operatoralgebren2.tex
\RequirePackage{fix-cm} 
\documentclass[a4paper, twoside, headsepline, index=totoc,toc=listof,toc=bibliography,toc=index, fontsize=10pt, cleardoublepage=empty, headinclude, DIV=12, BCOR=5mm, titlepage,draft]{scrartcl}
%!TEX root = ../AnaTopGeo_SS14/ana_top_geo.tex
\usepackage{scrtime} % KOMA, Uhrzeit ermoeglicht

%--Pakete zum "Programmieren"
% ======================================================================================
\usepackage{etoolbox}
\usepackage{letltxmacro}
\usepackage{ifthen}
% ======================================================================================

%--Farbdefinitionen und Grafiken (muss vor tikz geladen werden)
% ======================================================================================
\usepackage[usenames, table, x11names]{xcolor}
\definecolor{dark_gray}{gray}{0.45}
\definecolor{light_gray}{gray}{0.6}
\definecolor{fb10_blue}{cmyk}{0.8,0.4,0.13,0.07}
\usepackage[final]{graphicx}
\usepackage{adjustbox}
\newcommand{\cfbox}[2]{% coloured frame box
	\ifmmode
	\mathchoice{\adjustbox{cfbox=#1}{$\displaystyle#2$}}{\adjustbox{cfbox=#1}{$\textstyle#2$}}{\adjustbox{cfbox=#1}{$\scriptstyle#2$}}{\adjustbox{cfbox=#1}{$\scriptscriptstyle#2$}}
	\else
	\adjustbox{cfbox=#1}{#2}
	\fi
}
% ======================================================================================

%--Zum Zeichnen/ TikZ-Kram (vor polyglossia bzw. babel geladen werden)
% ======================================================================================
\usepackage{tikz}
\usepackage{tikz-cd}
\usetikzlibrary{external}
\tikzset{>=latex}
\usetikzlibrary{%
	shapes,
	arrows.meta,
	intersections,
	calc,
	3d,
	decorations.pathreplacing,decorations.markings,decorations.pathmorphing,
	angles,
	quotes,
}
\tikzexternalize[prefix=tikz/,up to date check=diff]
\pgfkeys{/pgf/images/include external/.code=\includegraphics{#1}}
\tikzset{external/system call={lualatex \tikzexternalcheckshellescape -halt-on-error -interaction=batchmode --shell-escape -jobname "\image" "\texsource"}}
\AtBeginEnvironment{tikzcd}{\tikzexternaldisable} % tikzexternalize fuer tikzcd deaktivieren, da inkompatibel
\AtEndEnvironment{tikzcd}{\tikzexternalenable}
\tikzset{% um Inkompatibilitaeten von quotes und polyglossia bzw. babel zu vermeiden
  every picture/.append style={
    execute at begin picture={\shorthandoff{"}},
    execute at end picture={\shorthandon{"}}
  }
}
\usepackage{pgfplots}
\usepgfplotslibrary{colormaps}
\newcommand*\circled[1]{\tikzexternaldisable\tikz[baseline=(char.base)]{\node[shape=circle,draw,inner sep=2pt] (char) {#1};}\tikzexternalenable}
% ======================================================================================



%-- Mathepakete etc.
% ======================================================================================
\usepackage[T1]{fontenc}
\renewcommand{\rmdefault}{zpltlf}
\usepackage{mathtools} % beinhaltet amsmath
\mathtoolsset{showonlyrefs,centercolon,showmanualtags}
\newtagform{brackets}[\textbf]{[}{]}
\usetagform{brackets}
\usepackage{fix-cm}
\usepackage[bbgreekl]{mathbbol}
\usepackage{amssymb,marvosym} 
\usepackage{nicefrac} % schräge Brüche
\usepackage{faktor}
\newcommand{\Faktor}[1]{\faktor[\textstyle]{#1}}
\usepackage{xfrac}
\usepackage{cancel}
\usepackage{mathdots} % Verbesserung von Punkten wie zB \ldots
\usepackage[bb=px]{mathalfa} % \mathbb als px font
\usepackage{centernot}
\usepackage{stackrel}
\DeclareSymbolFont{bbold}{U}{bbold}{m}{n}
\DeclareSymbolFontAlphabet{\mathbbold}{bbold}
\newcommand{\ind}{\mathbbold{1}} % charakteristische-Funktion-Eins
\def\mathul#1#2{\color{#1}\underline{{\color{black}#2}}\color{black}} %farbiges Untersteichen im Mathe-Modus
\renewcommand{\le}{\leqslant}
\renewcommand{\ge}{\geqslant}
% ======================================================================================


%-- Von xfrac erzeuge font warnings ignorieren
% ======================================================================================
\usepackage{silence}
\WarningFilter{latexfont}{Size substitutions with differences}
\WarningFilter{latexfont}{Font shape `U/bbold/m/n' in size}
% ======================================================================================


%-- Typographie/Polyglossia
% ======================================================================================
\usepackage[euler-digits]{eulervm} % vor fontspec laden!
\usepackage[no-math]{fontspec}
\usepackage{polyglossia} % moderner babel-ersatz
\setmainlanguage[spelling=new,babelshorthands=true]{german}
\shorthandoff{"}
\setotherlanguage{english}
\defaultfontfeatures{Mapping=tex-text, WordSpace={1.2}, Ligatures={Required,Common,Contextual},Extension=.otf} %


\setmainfont{TeXGyrePagellaX}[UprightFont=*-Regular,BoldFont=*-Bold,ItalicFont=*-Italic,BoldItalicFont=*-BoldItalic,ItalicFeatures={Style=Historic},Ligatures={Required,Common,Contextual,Historic}]
\setsansfont{texgyreadventor}[Scale=MatchUppercase, UprightFont=*-regular, BoldFont=*-bold, ItalicFont=*-italic, BoldItalicFont=*-bolditalic]
\setmonofont{SourceCodePro}[Scale=0.9,UprightFont=*-Regular, BoldFont=*-Semibold, ItalicFont=*-Light]
\usepackage{xltxtra}
\usepackage{fontawesome}
\usepackage[final]{microtype}
\usepackage[draft=false]{scrlayer-scrpage} 
\flushbottom
% ======================================================================================


%-- Aufzählungen
% ======================================================================================
\usepackage[shortlabels,inline]{enumitem}
\setlist[itemize,1]{label=\faCaretRight}
\setlist[enumerate]{font=\bfseries}
\setlist[description]{font=\normalfont\bfseries}
\usepackage{multicol}
% ======================================================================================


%-- Floats/Figures/Tabellen
% ======================================================================================
\usepackage{wrapfig}
\usepackage{float}
\usepackage[margin=10pt, font=small, labelfont={sf, bf}, format=plain, indention=1em]{caption}
\captionsetup[wrapfigure]{name=Abb. }
\usepackage{booktabs}
% ======================================================================================


%-- korrekte Anführungszeichen und Zitierbefehle
% ======================================================================================
\usepackage[autostyle,german=quotes,english=british]{csquotes}
% ======================================================================================


%--Indexverarbeitung
% ======================================================================================
\usepackage{makeidx}
\newcommand{\bet}[1]{\textbf{\emph{#1}}}
\newcommand{\Index}[1]{\bet{#1}\index{#1}}
\makeindex
\setindexpreamble{{\noindent\sffamily\small Die \emph{Seitenzahlen} sind mit Hyperlinks versehen und somit anklickbar} \par \bigskip}
\renewcommand{\indexpagestyle}{scrheadings}
% ======================================================================================


%-- Marginnotes/Todonotes/Footnotes
% ======================================================================================
\deffootnote[1.5em]{1.5em}{1.5em}{\textsuperscript{\thefootnotemark}\ }
\usepackage[fulladjust]{marginnote}
\renewcommand*{\marginfont}{\itshape\footnotesize}
\usepackage[textsize=small]{todonotes}
\usepackage{ragged2e}
\renewcommand*{\raggedleftmarginnote}{\RaggedLeft}
\renewcommand*{\raggedrightmarginnote}{\RaggedRight}
\LetLtxMacro{\oldtodo}{\todo}
\renewcommand{\todo}[2][]{\tikzexternaldisable\oldtodo[#1]{#2}\tikzexternalenable}
\LetLtxMacro{\oldmissingfigure}{\missingfigure}
\renewcommand{\missingfigure}[2][]{\tikzexternaldisable\oldmissingfigure[{#1}]{#2}\tikzexternalenable}
% ======================================================================================


% -- BibLaTeX
% ======================================================================================
\usepackage[%
	backend=biber,
	sortlocale=auto,
	natbib,
	hyperref,
	backref,
	style=alphabetic
	]%
{biblatex}
\renewcommand*{\mkbibnamelast}[1]{%
  \ifmknamesc{\textsc{#1}}{#1}}
\renewcommand*{\mkbibnameprefix}[1]{%
  \ifboolexpr{ test {\ifmknamesc} and test {\ifuseprefix} }
    {\textsc{#1}}
    {#1}}
\def\ifmknamesc{%
  \ifboolexpr{ test {\ifcurrentname{labelname}}
               or test {\ifcurrentname{author}}
               or ( test {\ifnameundef{author}} and test {\ifcurrentname{editor}} ) }}
\addbibresource{../!config/quellen.bib}
% ======================================================================================

%--Konfiguration von Hyperref und Cleveref
% ======================================================================================
\usepackage[hidelinks, pdfpagelabels,  bookmarksopen=true, bookmarksnumbered=true, linkcolor=black, urlcolor=SkyBlue2, plainpages=false,pagebackref, citecolor=black, hypertexnames=true, pdfauthor={Jannes Bantje}, pdfborderstyle={/S/U}, linkbordercolor=SkyBlue2, colorlinks=false,final,backref=false]{hyperref}
\usepackage[nameinlink,noabbrev]{cleveref}
\newcommand{\appendLink}[1]{#1\,\faExternalLink}
\newcommand{\hrefsym}[2]{\href{#1}{\texttt{\appendLink{#2}}}}
\newcommand{\hrefsymX}[2]{\href{#1}{\appendLink{#2}}}
\newcommand{\hrefsymmail}[2]{\href{#1}{\texttt{\faEnvelopeO\,#2}}}
\renewcommand{\url}[1]{\hrefsym{#1}{\nolinkurl{#1}}}
% ======================================================================================


% -- QR-Codes (hinter hyperref laden!)
% ======================================================================================
\usepackage{qrcode}
% ======================================================================================

%--Römische Zahlen
% ======================================================================================
\newcommand{\RM}[1]{\MakeUppercase{\romannumeral #1{}}}
% ======================================================================================

%-- Definition von diversen Mathe-Befehlen
% ======================================================================================
%!TEX root = mitschrift_main.tex

% -- Zum Finetuning von Befehlen
% ======================================================================================
\makeatletter
\newcommand{\raisemath}[1]{\mathpalette{\raisem@th{#1}}}
\newcommand{\raisem@th}[3]{\raisebox{#1}{$#2#3$}}
\makeatother
\makeatletter
\newcommand{\killDescendersM}[1]{\mathpalette{\killD@scendersM{#1}}}
\newcommand{\killD@scendersM}[2]{\raisebox{0pt}[\height][0pt]{$#2#1$}}
\makeatother
\DeclareRobustCommand{\minwidthbox}[2]{%
  \ifmmode
    \expandafter\mathmakebox
  \else
    \expandafter\makebox
  \fi
  [\ifdim#2<\width\width\else#2\fi]{#1}%
}
% ======================================================================================


%-- Klammerbefehle
% ======================================================================================
\DeclarePairedDelimiter{\abs}{\lvert}{\rvert}
\DeclarePairedDelimiter{\floor}{\lfloor}{\rfloor}
\DeclarePairedDelimiter{\ceil}{\lceil}{\rceil}
\DeclarePairedDelimiter\norm{\Vert}{\Vert}
\DeclarePairedDelimiter\enbrace{(}{)}
\DeclarePairedDelimiter\benbrace{[}{]}
\DeclarePairedDelimiter\bbenbrace{[\![}{]\!]}
\DeclarePairedDelimiter\lenbrace{<}{>}
\DeclarePairedDelimiter\angbrace{\langle}{\rangle}
\newcommand{\ssbrace}[1]{{\scriptscriptstyle\enbrace{#1}}}
\newcommand{\ssbbrace}[1]{{\scriptscriptstyle\benbrace{#1}}}
% ======================================================================================

%-- Mengen
% ======================================================================================
\newcommand\SetSymbol[1][]{\nonscript\:#1\vert\allowbreak\nonscript\:\mathopen{}}
\providecommand\given{} % to make it exist
\DeclarePairedDelimiterX\set[1]\{\}{\renewcommand\given{\SetSymbol[\delimsize]}#1}
% ======================================================================================

%-- Skalarprodukt (3 Varianten) 
% ======================================================================================
\DeclarePairedDelimiterX\sprod[2]{\langle}{\rangle}{#1\,\delimsize\vert\,#2}
\DeclarePairedDelimiterX\skal[2]{\langle}{\rangle}{#1\,,\,#2}
\makeatletter
\DeclareFontFamily{OMX}{MnSymbolE}{}
\DeclareSymbolFont{MnLargeSymbols}{OMX}{MnSymbolE}{m}{n}
\SetSymbolFont{MnLargeSymbols}{bold}{OMX}{MnSymbolE}{b}{n}
\DeclareFontShape{OMX}{MnSymbolE}{m}{n}{
    <-6>  MnSymbolE5
   <6-7>  MnSymbolE6
   <7-8>  MnSymbolE7
   <8-9>  MnSymbolE8
   <9-10> MnSymbolE9
  <10-12> MnSymbolE10
  <12->   MnSymbolE12
}{}
\DeclareFontShape{OMX}{MnSymbolE}{b}{n}{
    <-6>  MnSymbolE-Bold5
   <6-7>  MnSymbolE-Bold6
   <7-8>  MnSymbolE-Bold7
   <8-9>  MnSymbolE-Bold8
   <9-10> MnSymbolE-Bold9
  <10-12> MnSymbolE-Bold10
  <12->   MnSymbolE-Bold12
}{}
\let\llangle\@undefined
\let\rrangle\@undefined
\DeclareMathDelimiter{\llangle}{\mathopen}%
                     {MnLargeSymbols}{'164}{MnLargeSymbols}{'164}
\DeclareMathDelimiter{\rrangle}{\mathclose}%
                     {MnLargeSymbols}{'171}{MnLargeSymbols}{'171}
\makeatother
\DeclarePairedDelimiterX\sskal[2]{\llangle}{\rrangle}{#1\,,\,#2}
% ======================================================================================

%-- Abbildungsdefinition
% ======================================================================================
\newcommand{\mapdef}[5]{%
	\[
		\begin{array}{rcl}
			\textstyle #1 &\xrightarrow{\minwidthbox{#5}{2em}} & \textstyle #2 \\[0.5ex]
			\textstyle #3 &\xmapsto{\minwidthbox{\mbox{ }}{2em}} & \textstyle #4
		\end{array}
	\]
}
% ======================================================================================

%-- modifiziertes Stackrel 
% ======================================================================================
\newcommand{\StackText}[2]{\stackrel{\mbox{\scriptsize #1}}{#2}}
\newcommand{\StackTextClap}[2]{\stackrel{\mathclap{\mbox{\scriptsize #1}}}{#2}}
% ======================================================================================

%-- Blitz
% ======================================================================================
\newcommand{\light}{\text{\raisebox{-.3ex}{\Large\Lightning}}}
% ======================================================================================


%-- Underbrace u.Ä. als Befehl in LaTeX-Syntax (und ohne Spacingprobleme mit nachfolgenden Operatoren...)
% ======================================================================================
\newcommand{\Underbrace}[2]{{\underbrace{#1}_{#2}}}
\newcommand{\Underbracket}[2]{{\underbracket[0.7pt][2pt]{#1}_{#2}}}
\newcommand{\Overbracket}[2]{{\overbracket[0.7pt][2pt]{#1}^{#2}}}
% ======================================================================================


%-- Deklaration weiterer Operatoren (allgemein)
% ======================================================================================
\DeclareMathOperator{\re}{Re} % Realteil
\let\Re\relax
\DeclareMathOperator{\Re}{Re} % Realteil
\DeclareMathOperator{\im}{im} % Bild
\let\Im\relax
\DeclareMathOperator{\Im}{Im} % Bild
\DeclareMathOperator{\id}{id} % identische Abbildung
\DeclareMathOperator{\conj}{conj} % Konjugation
\DeclareMathOperator{\sgn}{sgn} % Signum
\DeclareMathOperator{\End}{End} % Endomorphismen
\DeclareMathOperator{\Hom}{Hom} % Homomorphismen
\DeclareMathOperator{\Iso}{Iso} % Isomorphismen
\DeclareMathOperator{\Aut}{Aut} % Automorphismen
\DeclareMathOperator{\Span}{span} % Span
\DeclareMathOperator{\coker}{coker} % Kokern
\DeclareMathOperator{\Tr}{Tr} % Spur,Trace
\DeclareMathOperator{\pr}{pr} % Projektion
\DeclareMathOperator{\diag}{diag} % Diagonalmatrix
\DeclareMathOperator{\Rg}{Rg} % Rang
\DeclareMathOperator{\const}{const} % konstante Abbildung
\DeclareMathOperator{\Spur}{Spur} % Spur
\DeclareMathOperator{\Arg}{Arg} % Argument
\DeclareMathOperator{\dist}{dist} % Distanz
\DeclareMathOperator{\supp}{supp} % Träger
\DeclareMathOperator{\Char}{char} % Charakteristik
% ======================================================================================


%-- Deklaration weiterer Operatoren (Differentiale etc.)
% ======================================================================================
\DeclareMathOperator{\grad}{grad} % Gradient
\DeclareMathOperator{\dive}{div} % Gradient
\DeclareMathOperator{\rot}{rot} % Rotation
\newcommand{\D}{\ensuremath{\mathrm{D}\mkern-1.0mu}} % Differential
\newcommand{\mathd}{\ensuremath{\mathrm{d}\mkern-1.0mu}} % äußere Ableitung
\newcommand{\Tmap}{\ensuremath{\mathrm{T}\mkern-0.85mu}} % Tangentialraum
\let\Tang\Tmap
\DeclareMathOperator{\Diff}{Diff}
\newcommand{\diff}[2]{\ensuremath{\frac{{\partial #1}}{{\partial #2}} }}
\newcommand{\diffd}[2]{\ensuremath{\frac{\mathd #1}{\mathd #2} }}
\DeclareMathOperator{\rank}{rank}
% ======================================================================================


%-- Deklaration weiterer Operatoren (Topologie)
% ======================================================================================
\newcommand*\interior[1]{\overset{\smash{\raisebox{-0.18ex}{$\scriptstyle\circ$}}}{#1}}
\newcommand{\sing}{{\raisemath{1.1pt}{\scriptscriptstyle\mathrm{sing}}}}
\newcommand{\pt}{\mathrm{pt}}
\DeclareMathOperator{\Zyl}{Zyl}
\newcommand{\rZyl}{\widetilde{\Zyl}}
\DeclareMathOperator{\Tel}{Tel}
\newcommand{\op}{\mathrm{op}}
\DeclareMathOperator{\Sp}{Sp}
\DeclareMathOperator{\Keg}{Keg}
\newcommand{\slashedi}{i\hspace{-3.5pt}/}
\newcommand{\cupp}{\smallsmile}
\newcommand{\capp}{\smallfrown}
\DeclareMathOperator*{\colim}{colim}
\DeclareMathOperator{\PD}{PD}
\newcommand{\lf}{\mathrm{lf}}
\DeclareMathOperator{\sig}{sig}
\DeclareMathOperator{\Tor}{Tor}
\DeclareMathOperator{\Ext}{Ext}
\DeclareMathOperator{\AW}{AW}
\DeclareMathOperator{\Proj}{Proj}
\DeclareMathOperator{\Gr}{Gr}
\DeclareMathOperator{\res}{res}
\DeclareMathOperator{\Spec}{Spec}
\DeclareMathOperator{\co}{co}
\DeclareMathOperator{\ch}{ch}
\DeclareMathOperator{\wOp}{w}
\DeclareMathOperator{\Ar}{Ar}
\newcommand{\actson}{\mathrel{\curvearrowright}}
\let\acts\actson
\let\action\actson
\DeclareMathSymbol{\bbDelta}{\mathord}{bbold}{"01}
\newcommand{\DDelta}{\bbDelta}
\DeclareMathOperator{\Star}{Star}
\DeclareMathOperator{\Link}{Link}
\DeclareMathOperator{\EPK}{EPK}
\DeclareMathOperator{\Vol}{Vol}
\newcommand{\cell}{{\raisemath{1.1pt}{\scriptscriptstyle\mathrm{cell}}}}
\DeclarePairedDelimiter{\homologieklasse}{\llbracket}{\rrbracket}
\newcommand{\rand}[1]{\ensuremath{\partial^{\scriptscriptstyle #1}}}
\DeclareMathOperator{\ab}{ab}
\DeclareMathOperator{\CW}{CW}
% ======================================================================================


%-- Deklaration von Operatoren (Liegruppen)
% ======================================================================================
\DeclareMathOperator{\GL}{GL}
\DeclareMathOperator{\SO}{SO}
\DeclareMathOperator{\Ad}{Ad}
\DeclareMathOperator{\ad}{ad}
\DeclareMathOperator{\On}{O}
\DeclareMathOperator{\Un}{U}
\DeclareMathOperator{\SU}{SU}
\DeclareMathOperator{\Mat}{Mat}
\DeclareRobustCommand{\Der}{\mathop{\mathfrak{der}}}
\DeclareMathOperator{\SL}{SL}
\DeclareMathOperator{\Graph}{Graph}
\DeclareMathOperator{\Int}{Int}
\DeclareRobustCommand{\intAlg}{\mathop{\mathfrak{int}}}
\DeclareMathOperator{\aut}{aut}
\DeclareMathOperator{\Rad}{Rad}
\DeclareMathOperator{\Nil}{Nil}
\DeclareMathOperator{\rad}{rad}
\DeclareMathOperator{\nil}{nil}
\DeclareMathOperator{\Ric}{Ric}
\DeclareMathOperator{\ric}{ric}
\newcommand{\bi}{\mathrm{bi}}
\DeclareMathOperator{\Isom}{Isom}
\DeclareMathOperator{\Sym}{Sym}
\newcommand{\opL}{\ensuremath{\mathrm{L}\mkern-0.6mu}}
% ======================================================================================

%-- Deklaration von Operatoren (Funktionalanalysis)
% ======================================================================================
\DeclareMathOperator{\tr}{tr}
\newcommand{\w}{\mkern1mu\mathrm{w}}
\newcommand{\sa}{\mathrm{sa}}
\newcommand{\vb}{\mathrm{v\mkern-2.5mu.b\mkern-1.5mu.}} % vollständig beschränkt
\newcommand{\so}{\mathrm{\mkern.3mu s\mkern-1.4mu.\mkern-.6mu o\mkern-1.7mu.}} % \newcommand{\so}{\mathrm{s.o.}}
\newcommand{\solim}{\so\text{-}\mkern-0.8mu\lim}
\newcommand{\wo}{\mathrm{w\mkern-3mu.\mkern-.4mu o\mkern-1.7mu.}}
\newcommand{\Top}[1]{\mathcal{T}_{\mkern-2.3mu #1}}
\newcommand{\weakT}[1]{\ensuremath{\mathcal{T}_{#1}^{\mkern+1.0mu\text{\raisebox{0.4ex}{$\mathrm{w}$}}}}}
\newcommand{\weakTstar}[1]{\ensuremath{\mathcal{T}_{#1}^{\mkern+1.0mu\text{\raisebox{0.4ex}{$\mathrm{w}$}}^*}}}
\newcommand{\TWeakStar}{\Top{\w^*}}
\newcommand{\TWeakOp}{\Top{\wo}}
\newcommand{\Tso}{\Top{\so}}
\newcommand{\finSub}{\subset\mkern-0.7mu \subset}
\DeclareMathOperator{\Inv}{Inv}
\newcommand{\simm}{{\hspace{-1.6pt}\raisemath{0.5pt}{\sim}}}
\newcommand{\plus}{{\hspace{-1.6pt}+}}
\DeclareMathOperator{\ev}{ev}
\DeclareMathOperator{\Alg}{Alg}
\DeclareMathOperator{\her}{her}
\newcommand{\subher}{\subset_{\her}}
\newcommand{\grenzw}[1]{\xrightarrow{\minwidthbox{#1}{1.4em}}}
\newcommand{\grenzwl}[1]{\xleftarrow{\minwidthbox{#1}{1.4em}}}
\newcommand{\grenzwIn}[1]{\grenzw{\raisemath{-2pt}{#1}}}
\newcommand{\MyTo}[1]{\tikzexternaldisable\mathbin{\tikz[baseline] \draw[-to,line width=.4pt] (0ex,0.94ex) -- (#1,0.94ex);}\tikzexternalenable}
\newcommand{\dlim}{%
    \mathchoice
      {\lim\limits_{\MyTo{4.2ex}}}% \displaystyle
      {\lim\limits_{\MyTo{2.8ex}}}% \textstyle
      {\lim\limits_{\MyTo{2.3ex}}}% \scriptstyle
      {\lim\limits_{\MyTo{2.3ex}}}% \scriptscriptstyle
}
\newcommand{\Dlim}{\killDescendersM{\dlim}}
\DeclareMathOperator{\sep}{sep}
\DeclareMathOperator{\diam}{diam}
\DeclareMathOperator{\conv}{conv}
\DeclareMathOperator{\Prim}{Prim}
\DeclareMathOperator{\hull}{hull}
\DeclareMathOperator{\red}{red}
\DeclarePairedDelimiterX\bra[1]{\langle}{\rvert}{#1\,}
\DeclarePairedDelimiterX\ket[1]{\lvert}{\rangle}{\,#1}
\DeclarePairedDelimiterX\bracket[2]{\langle}{\rangle}{#1\,\delimsize\vert\,#2}
\newcommand{\tensormax}{\mathbin{\otimes_{\max}}}
\newcommand{\tensormin}{\mathbin{\otimes_{\min}}}
\DeclareMathOperator{\Ped}{Ped}
\newcommand{\alg}{\mathrm{alg}}
\DeclareMathOperator{\CPC}{CPC}
\DeclareMathOperator{\CP}{CP}
\DeclareMathOperator{\UPC}{UPC}
\newcommand{\DeltaOp}{\mathbin{\Delta}}
\newcommand{\kernedP}{\mathcal{P}\mkern-2mu}
\newcommand{\Pinfty}{\kernedP_{\infty}}
\DeclareMathOperator{\Groth}{Groth}
\DeclareMathOperator{\rk}{rk}
\newcommand{\MvN}{\mathrm{MvN}}
% ======================================================================================

%-- Kategorien
% ======================================================================================
\DeclareMathOperator{\Mor}{Mor}
\DeclareMathOperator{\mor}{mor}
\DeclareMathOperator{\Obj}{Obj}
\DeclareMathOperator{\Ob}{Ob}
\newcommand{\TOP}{\textsc{Top}}
\newcommand{\HTOP}{\textsc{HTop}}
\newcommand{\VR}{\textsc{VR}}
\newcommand{\MOD}{\textsc{Mod}}
\newcommand{\Mod}[1]{#1\text{-}\MOD}
\newcommand{\MONOIDE}{\textsc{Monoide}}
\newcommand{\SET}{\textsc{Set}}
\newcommand{\MAN}{\textsc{Man}}
\newcommand{\GRUPPEN}{\textsc{Gruppen}}
\newcommand{\ABELGRUPPEN}{\textsc{Abel.Gruppen}}
\newcommand{\ABEL}{\textsc{Abel}}
\newcommand{\KAT}{\textsc{Kat}}
\newcommand{\FUN}{\textsc{Fun}}
\newcommand{\SIMP}{\textsc{Simp}}
\newcommand{\VEKT}{\textsc{Vekt}}
\newcommand{\CH}{\textsc{Ch}}
\newcommand{\CSTARUN}{C^*\text{-}\textsc{Alg}^{\raisemath{-2.5pt}{1}}}
\newcommand{\CSTAR}{C^*\text{-}\textsc{Alg}}
\newcommand{\AB}{\textsc{Ab}}
% ======================================================================================
% ======================================================================================



% -- theorem packages
% ======================================================================================
\usepackage{amsthm}
\usepackage{thmtools,thm-restate}
\usepackage{mdframed}
\renewcommand{\listtheoremname}{Übersicht aller Aussagen}
\usepackage{bookmark}
\bookmarksetup{open,numbered}
\makeatletter
\newcommand*{\theorembookmark}{%
  \bookmark[
    dest=\@currentHref,
    rellevel=1,
    keeplevel,
  ]{%
    \thmt@thmname\space\csname the\thmt@envname\endcsname
    \ifx\thmt@shortoptarg\@empty
    \else
      \space(\thmt@shortoptarg)%
    \fi
  }%
}   
\makeatother
% ======================================================================================

% -- Definition der einzelnen Theorem-Umgebungen
% ======================================================================================
\declaretheoremstyle[%
	headfont=\sffamily\bfseries,
	notefont=\normalfont\sffamily\scshape,
	bodyfont=\normalfont,
	headformat=\NUMBER\ \NAME\NOTE,
	headpunct=.,
	postheadspace=1em,
	spaceabove=15pt,spacebelow=10pt,
	shaded={bgcolor=gray!20},
	postheadhook=\theorembookmark]%
{mainstyle}
\declaretheoremstyle[%
	headfont=\sffamily\bfseries,
	notefont=\normalfont\sffamily\scshape,
	bodyfont=\normalfont,
	headformat=\NUMBER\ \NAME\NOTE,
	headpunct=.,
	postheadspace=1em,
	spaceabove=15pt,spacebelow=10pt,
	shaded={bgcolor=fb10_blue!20},
	postheadhook=\theorembookmark]%
{mainstyle_blue}
\declaretheoremstyle[%
	headfont=\sffamily\bfseries,
	notefont=\normalfont\sffamily\scshape,
	bodyfont=\normalfont,
	headformat=\NUMBER\ \NAME\NOTE,
	headpunct=.,
	postheadspace=1em,
	spaceabove=15pt,spacebelow=10pt,
	postheadhook=\theorembookmark]%
{mainstyle_unshaded}
\declaretheoremstyle[%
	headfont=\sffamily\bfseries,
	notefont=\normalfont\sffamily\scshape,
	bodyfont=\normalfont,
	headformat=\NUMBER\NAME\NOTE,
	headpunct=.,
	postheadspace=1em,
	spaceabove=15pt,spacebelow=10pt,
	% shaded={bgcolor=gray!20},
	postheadhook=\theorembookmark]%
{mainstyle_unnumbered}
\declaretheoremstyle[%
	headfont=\sffamily\bfseries,
	notefont=\normalfont\sffamily\scshape,
	bodyfont=\normalfont,
	headformat=swapnumber,
	headpunct=.,
	postheadspace=1em,
	spaceabove=15pt,spacebelow=10pt,
	shaded={bgcolor=gray!20},
	postheadhook=\theorembookmark,
	qed=\qedsymbol]%
{mainstyleB}
\declaretheoremstyle[%
	headfont=\bfseries\scshape,
	bodyfont=\normalfont,
	headpunct=:,
	postheadspace=1em,
	spacebelow=12pt,spaceabove=2pt,
	qed=\qedsymbol]%
{beweise}
\declaretheoremstyle[%
	headfont=\bfseries\scshape,
	bodyfont=\normalfont,
	headpunct=:,
	postheadspace=1em,
	spacebelow=12pt,spaceabove=2pt]%
{beweisskizze}
\declaretheoremstyle[%
	headfont=\sffamily\bfseries,
	bodyfont=\normalfont,
	headpunct=:,
	postheadspace=1em,
	spacebelow=10pt,spaceabove=10pt]%
{bemerkungen}
\declaretheorem[name=Definition,parent=section,style=mainstyle_blue]{definition}
\declaretheorem[name=Definition \& Proposition,refname=Proposition,sharenumber=definition,style=mainstyle_blue]{definitionP}
\declaretheorem[name=Definition,numbered=no,style=mainstyle_unnumbered]{definition*}
\declaretheorem[name=Theorem,sharenumber=definition,style=mainstyle]{theorem}
\declaretheorem[name=Theorem,numbered=no,style=mainstyle_unnumbered]{theorem*}
\declaretheorem[name=Proposition,sharenumber=definition,style=mainstyle,refname=Proposition]{proposition}
\declaretheorem[name=Lemma,sharenumber=definition,style=mainstyle]{lemma}
\declaretheorem[name=Satz,sharenumber=definition,style=mainstyle,refname=Satz]{satz}
\declaretheorem[name=Satz,sharenumber=definition,style=mainstyle_unshaded]{satzUnshaded}
\declaretheorem[name=Definition,sharenumber=definition,style=mainstyle_unshaded]{definitionUnshaded}
\declaretheorem[name=Satz,numbered=no,style=mainstyle_unnumbered]{satz*}
\declaretheorem[name=Korollar,sharenumber=definition,style=mainstyle,refname=Korollar]{korollar}
\declaretheorem[name=Korollar,sharenumber=definition,style=mainstyleB,refname=Korollar]{korollarB}
\declaretheorem[name=Frage,numbered=no,style=mainstyle_unnumbered]{frage}
\declaretheorem[name=Frage,sharenumber=definition,style=mainstyle_unshaded]{frageA}
\declaretheorem[name=Erinnerung,sharenumber=definition,style=mainstyle_unshaded]{erinnerungA}
\declaretheorem[name=Ausblick,sharenumber=definition,style=mainstyle_unshaded]{ausblick}
\declaretheorem[name=Konvention,sharenumber=definition,style=mainstyle]{konvention}
\declaretheorem[name=Notation,sharenumber=definition,style=mainstyle_unshaded]{notation}
\declaretheorem[name=Bemerkung,sharenumber=definition,style=mainstyle_unshaded,refname=Bemerkung]{bemerkung}
\declaretheorem[name=Bemerkung,numbered=no,style=mainstyle_unnumbered]{bemerkung*}
\declaretheorem[name=Beispiel,sharenumber=definition,style=mainstyle_unshaded,refname=Beispiel]{beispiel}
\declaretheorem[name=Beispiel,numbered=no,style=mainstyle_unnumbered]{beispiel*}
\declaretheorem[name=Exkurs,numbered=no,style=mainstyle_unnumbered]{exkurs*}
\declaretheorem[name=Beweis,numbered=no,style=beweise]{beweis}
\declaretheorem[name=Übung,numbered=no,style=bemerkungen]{uebung}
\declaretheorem[name=Erinnerung,numbered=no,style=bemerkungen]{erinnerung}

% english versions
\declaretheorem[name=Remark,sharenumber=definition,style=mainstyle_unshaded]{remark}
\declaretheorem[name=Remark,numbered=no,style=mainstyle_unnumbered]{remark*}
\declaretheorem[name=Example,sharenumber=definition,style=mainstyle_unshaded]{example}
\declaretheorem[name=Corollary,sharenumber=definition,style=mainstyle]{corollary}
\let\proof\relax
\declaretheorem[name=Proof,numbered=no,style=beweise]{proof}
\declaretheorem[name=Sketch of Proof,numbered=no,style=beweisskizze]{sketch}
% ======================================================================================

%--Inhaltsverzeichnis
% ======================================================================================
\usepackage[tocindentauto]{tocstyle}
\usetocstyle{KOMAlike}
% ======================================================================================

%-- Dinge, die erst am Ende getan werden dürfen
% ======================================================================================
\shorthandon{"}
\usepackage{ellipsis}
% ======================================================================================


\newcommand{\fach}{Algebraische $K$-Theorie}
\newcommand{\semester}{Sose 2016}
\newcommand{\homepage}{https://wwwmath.uni-muenster.de/reine/u/topos/lehre/WS2015-2016/Topologie2/}

\newcommand{\prof}{Prof.\ Dr.\ Arthur Bartels}
\publishers{\scalebox{11}{\Huge$K^{\raisebox{2.7pt}{\text{\small$\mathrm{alg}$}}}$}}
\input{../!config/mitschrift_headings.tex}

\begin{document}
\pagenumbering{Roman}
\maketitle
\begin{abstract}
\section*{Aktuelle Version verfügbar bei}
\newcommand{\dieBreite}{11cm}
\begin{minipage}{4cm}
	\qrcode[height=3.3cm, version=6]{https://gitlab.com/JaMeZ-B/LaTeX-WWU}
\end{minipage}
\hfill
\begin{minipage}{\dieBreite}
	% \includegraphics[height=0.6cm, keepaspectratio]{../!config/Bilder/wm_no_bg.pdf}
	\includegraphics[height=0.8cm, keepaspectratio]{../!config/Bilder/wm_no_bg.pdf}\\
	\url{https://gitlab.com/JaMeZ-B/LaTeX-WWU} \smallskip\\
	Das zentrale Repository des \enquote{\LaTeX-WWU}-Projekts befindet sich auf der Plattform GitLab.com.
	Neben der Koordination aller Beteiligten werden über diesen Dienst auch die PDFs gebaut, die in der Readme verlinkt sind.
\end{minipage}\\[1cm]
\begin{minipage}{4cm}
	\qrcode[height=3.3cm, version=6]{https://github.com/JaMeZ-B/latex-wwu}
\end{minipage}
\hfill
\begin{minipage}{\dieBreite}
	\includegraphics[height=0.6cm, keepaspectratio]{../!config/Bilder/github_octo.pdf}
	\includegraphics[height=0.6cm, keepaspectratio]{../!config/Bilder/GitHub_Logo.pdf}\\
	\url{https://github.com/JaMeZ-B/latex-wwu} \smallskip\\
	Die Entwicklung des \enquote{\LaTeX-WWU}-Projekts hat ursprünglich auf GitHub stattgefunden, ist mittlerweile aber zu GitLab gewechselt.
	Das GitHub-Repository wird stündlich automatisch aktualisiert, Merge-Requests werden aber nicht mehr entgegengenommen.
\end{minipage}\\[1cm]
% \begin{minipage}{4cm}
% 	\qrcode[height=3.3cm, version=6]{https://uni-muenster.sciebo.de/public.php?service=files&t=965ae79080a473eb5b6d927d7d8b0462}
% \end{minipage}
% \hfill
% \begin{minipage}{\dieBreite}
% 	\raisebox{-2pt}{\includegraphics[height=0.6cm, keepaspectratio]{../!config/Bilder/sciebo_logo.pdf}}
% 	\resizebox{!}{0.5cm}{\large \sffamily\textbf{sciebo}} {\sffamily\large die Campuscloud} \\
% 	\resizebox{\dieBreite}{!}{\footnotesize\url{https://uni-muenster.sciebo.de/public.php?service=files&t=965ae79080a473eb5b6d927d7d8b0462}}\smallskip\\
% 	Sciebo ist ein Dropbox-Ersatz der Hochschulen in NRW, der von der Uni Münster in leitender Position auf Basis der OpenSource-Software Owncloud aufgebaut wurde.
% \end{minipage}\\[1cm]
\hrule \mbox{ }\\[0.7cm]
\begin{minipage}{4cm}
	\qrcode[height=3.3cm, version=6]{\homepage}
\end{minipage}
\hfill
\begin{minipage}{\dieBreite}
	\resizebox{!}{0.5cm}{\large\sffamily\textbf{Vorlesungshomepage}}\\
	\resizebox{\dieBreite}{!}{\footnotesize\url{\homepage}}\smallskip\\
	Hier ist ein Link zur offiziellen Vorlesungshomepage.
\end{minipage}
\newpage
\section*{Vorwort --- Mitarbeit am Skript}
Dieses Dokument ist eine Mitschrift aus der Vorlesung \enquote{\fach, \semester}, gelesen von \prof. 
Der Inhalt entspricht weitestgehend dem Tafelanschrieb. 
Für die Korrektheit des Inhalts übernehme ich keinerlei Garantie! 
Für Bemerkungen und Korrekturen -- und seien es nur Rechtschreibfehler -- bin ich sehr dankbar. 
Korrekturen lassen sich prinzipiell auf drei Wegen einreichen: 
\begin{itemize}
	\item Persönliches Ansprechen in der Uni, Mails an \hrefsymmail{mailto:\mail}{\mail} (gerne auch mit annotieren PDFs) oder Kommentare auf \url{https://gitlab.com/JaMeZ-B/LaTeX-WWU}.
	\item \emph{Direktes} Mitarbeiten am Skript: Den Quellcode poste ich auf GitLab (siehe oben), also stehen vielfältige Möglichkeiten der Zusammenarbeit zur Verfügung:
	Zum Beispiel durch Kommentare am Code über die Website und die Kombination Fork und Merge-Request. 
	Wer sich verdient macht oder ein Skript zu einer Vorlesung, die ich nicht besuche, beisteuern will, dem gewähre ich gerne auch Schreibzugriff.
	
	Beachten sollte man dabei, dass dazu ein Account bei \url{gitlab.com} notwendig ist, der allerdings ohne Angabe von persönlichen Daten angelegt werden kann. 
	Wer bei GitLab (bzw. dem zugrunde liegenden Open-Source-Programm \enquote{\texttt{git}}) -- verständlicherweise -- Hilfe beim Einstieg braucht, dem helfe ich gerne weiter. 
	Es gibt aber auch zahlreiche empfehlenswerte Tutorials im Internet.\footnote{zB. \url{https://try.github.io/levels/1/challenges/1}, ist auf Englisch, aber dafür interaktiv}
	\item \emph{Indirektes} Mitarbeiten: \TeX-Dateien per Mail verschicken. 
	
	Dies ist nur dann sinnvoll, wenn man einen ganzen Abschnitt ändern möchte (zB. einen alternativen Beweis geben), da ich die Änderungen dann per Hand einbauen muss! Ich freue mich aber auch über solche Beiträge!
\end{itemize}
\section*{Literatur}
\begin{itemize}
	\item \citetitle{Rosenberg} von J. \citeauthor{Rosenberg} \cite{Rosenberg}
	\item \citetitle{Weibel} von Charles \citeauthor{Weibel} \cite{Weibel}
	\item \citetitle{MilnorKtheory} von John \citeauthor{MilnorKtheory} \cite{MilnorKtheory}
\end{itemize}
\end{abstract}

\tableofcontents
\cleardoubleoddemptypage

\pagenumbering{arabic}
\setcounter{page}{1}
\setcounter{footnote}{0}

\section{$K_0$ eines Ringes} % (fold)
\label{sec:k_0_eines_ringes}

\begin{definition}[{name=[{projektiv}]}]
	Sei $R$ ein Ring.
	Ein $R$-Modul $P$ heißt \Index{projektiv}, falls er folgende Eigenschaft hat:
	Sei $f \colon M \twoheadrightarrow P$ $R$-linear und surjektiv. 
	Dann gibt es einen Spalt $s \colon P \to M$, das heißt eine $R$-lineare Abbildung $s$ mit $f \circ s = \id_P$.
\end{definition}

\begin{bemerkung}[{name=[{Äquivalenzen zu Projektivität}]}]
	Die zwei folgenden Bedingungen sind äquivalent zur Projektivität von $P$:
	\begin{enumerate}[1)]
		\item Sind $\varphi \colon P \to N$, $\psi \colon M \twoheadrightarrow N$ $R$-linear und $\psi$ surjektiv, so gibt es $\hat{\varphi} \colon P \to M$ mit $\psi \circ \hat{\varphi} = \varphi$.
		\[
			\begin{tikzcd}
				& M \dar[two heads,"\psi"] \\
				P \rar["\varphi"] \urar["\hat{\varphi}",dashed] & N
			\end{tikzcd}
		\]
		\item $P$ ist ein direkter Summand in einem freien Modul, das heißt es gibt einen $R$-Modul $Q$, sodass $P \oplus Q$ eine $R$-Basis besitzt.
	\end{enumerate}
\end{bemerkung}

\begin{definition}[{name=[{endlich erzeugter $R$-Modul}]}]
	Ein $R$-Modul heißt \bet{endlich erzeugt}\index{endlich erzeugter $R$-Modul}, wenn es eine endliche Teilmenge $S \subseteq M$ gibt, sodass $M$ der einzige Untermodul von $M$ ist, der $S$ enthält.
	Wir nennen $S$ ein \Index{endliches Erzeugendensystem} für $M$.
\end{definition}

\begin{bemerkung}[{name=[{endlich erzeugte freie Moduln}]}]
	Ein freier $R$-Modul ist genau dann endlich erzeugt, wenn er eine endliche Basis besitzt.
	Er ist dann isomorph zu $R^n$ für geeignetes $n \in \mathbb{N}$.\marginnote{dieses $n$ muss nicht eindeutig sein!}
\end{bemerkung}

\begin{satz}[{name=[{Äquivalenzen zur Projektivität endlich erzeugter Moduln}]},label=satz:proj_endlich_erz]
	Sei $P$ ein endlich erzeugter $R$-Modul.
	Dann sind äquivalent:
	\begin{enumerate}[1)]
		\item $P$ ist projektiv.
		\item $P$ ist direkter Summand in einem $R^n$.
		\item Es gibt eine $R$-lineare idempotente\marginnote{$p^2=p$} Abbildung $p \colon R^n \to R^n$ mit $P \cong \im p$.
	\end{enumerate}
\end{satz}
\begin{beweis}
	Sei zunächst $P$ projektiv.
	Ist $v_1,\ldots ,v_n$ ein endliches Erzeugendensystem von $P$, so erhalten wir $R$-lineare surjektive Abbildung $f \colon R^n \to P$, $f(r_1,\ldots,r_n)=\sum_{i=1}^n r_i \cdot v_i$.
	Da $P$ projektiv ist, gibt es einen Spalt $s$ für $f$ und es folgt $R^n \cong P \oplus \ker f$.
	
	Sei nun $\varphi \colon P \oplus Q \to R^n$ ein Isomorphismus.
	Sei $\tilde{p} \colon P \oplus  Q \to P \oplus Q$ die Projektion auf $P$, also $\tilde{p} (v,w) = (v,0)$ für $v \in P$, $w \in Q$.
	Dann ist $p := \varphi \circ \tilde{p} \circ \varphi^{-1} \in \End_R(R^n)$ idempotent und $\varphi|_{P} \colon P \to \im p$ ein Isomorphismus.
	
	Sei $p \colon R^n \to R^n$ idempotent.
	Wir zeigen, dass $\im p = p(R^n)$ projektiv ist.
	Sei $f \colon M \twoheadrightarrow p(R^n)$ $R$-linear und surjektiv.
	Dann gibt es $v_1,\ldots ,v_n \in M$ mit $f(v_i)= p(e_i)$.
	Sei $\hat{s} \colon R^n \to M$ gegeben durch $\hat{s}(r_1,\ldots ,r_n)=\sum_{i=1}^{n} r_i \cdot v_i$.
	Dann ist $f \circ \hat{s} =p$ und daher gilt für $w =p(w) \in \im p$
	\[
		f \enbrace[\big]{\hat{s}(w)} =p(w)=w
	\]
	Also ist $s:= \hat{s}|_{p(R^n)}$ der gesuchte Spalt.
\end{beweis}

\begin{bemerkung}[{name=[{direkte Summanden des $R^n$ sind endlich erzeugt}]}]
	Ist $P \subseteq R^n$ ein direkter Summand, so ist $P$ endlich erzeugt, da wir eine Surjektion $R^n \to P$ erhalten und das Bild der endlichen Basis der $R^n$ dann auch ein endliches Erzeugendensystem für $P$ ist.
\end{bemerkung}

\begin{bemerkung}[{name=[{Menge der Isomorphieklassen von endlich erzeugten projektiven Moduln}]}]
	Für jedes $n \in \mathbb{N}$ bilden die Untermoduln von $R^n$ eine Menge, insbesondere bilden die direkten Summanden eine Menge.
	Es folgt mit \autoref{satz:proj_endlich_erz}, dass auch die Isomorphieklassen von endlich erzeugten projektiven $R$-Moduln eine Menge bilden.
	Diese nennen wir $\Proj(R)$.
\end{bemerkung}

\begin{bemerkung}[{name=[{Proj(R) als abelsche Halbgruppe}]}]
	$\Proj(R)$ wird durch $\oplus$ zu einer abelschen Halbgruppe mit der Isomorphieklasse des Nullmoduls als neutralem Element.
\end{bemerkung}

\begin{satz}[name={Grothendiek-Konstruktion}]
	Sei $S$ eine abelsche Halbgruppe.
	Dann gibt es eine abelsche Gruppe $\Gr(S)$ zusammen mit einem Halbgruppenhomomorphismus $\varphi \colon S\to \Gr(S)$, der folgende universelle Eigenschaft erfüllt:
	Ist $A$ eine abelsche Gruppe und $\psi \colon S \to A$ ein Halbgruppenhomomorphismus, so gibt es einen eindeutigen Gruppenhomomorphismus $\hat{\psi} \colon \Gr(S) \to A$ mit $\psi = \hat{\psi} \circ \varphi$.
	\[
		\begin{tikzcd}[column sep=3em]
			S \rar["\varphi"] \drar["\psi"'] & \Gr(S) \dar[dashed,"\hat{\psi}"] \\
			& A
		\end{tikzcd}
	\]
\end{satz}
\begin{beweis}
	Wir nehmen an, dass $S$ ein neutrales Element besitzt.
	Wir definieren $\Gr(S)$ wie folgt: Auf $S \times S$ betrachten wir die Äquivalenzrelation 
	\[
		(s,t) \sim (s',t') :\iff \exists x \in S : s +t' + x = s'+t +x
	\] 
	und setzen $\Gr(S) := \sfrac{S\times S}{\sim}$.
	Wir schreiben $s-t$ für die Äquivalenzklasse von $(s,t)$.
	Die Addition auf $\Gr(S)$ ist durch $(s-t) + (s'-t') = s+s'  - (t+t')$ gegeben, das Inverse zu $s-t$ ist $t-s$.
	Der Halbgruppenhomomorphismus $\varphi \colon S \to \Gr(S)$ ist definiert durch $\varphi(s)= s - 0$.
	Ist $\psi \colon S \to A$ ein Halbgruppenhomomorphismus, so erhalten wir $\hat{\psi} \colon \Gr(S) \to A$ durch $\hat{\psi}(s-t) = \psi(s)- \psi(t)$.
\end{beweis}

\begin{definition}[{name=[{Grothendiek-Gruppe}]}]
	$\Gr(S)$ heißt die \Index{Grothendiek-Gruppe} oder \Index{Gruppenvervollständigung} von $S$.  
\end{definition}

\begin{bemerkung}[{name=[alternative Konstruktion mit freier abelscher Gruppe]}]
	Alternativ können wir auch die \Index{freie abelsche Gruppe} $\mathbb{Z}[S]$ benutzen, um $\Gr(S)$ zu konstruieren.
  Setze dazu
	\[
		\Gr(S) := \sfrac{\mathbb{Z}[S]}{A}
	\]
	wobei $A$ von $\set[\big]{s + t-(s +_S t) \given s,t \in S}$ erzeugt wird.
\end{bemerkung}

\begin{beispiel}[{name=[{für Grothendiek-Konstruktionen}]}]
	Es gilt
	\begin{multicols}{2}
		\begin{itemize}
			\item $\Gr(\mathbb{N},+) \cong \mathbb{Z}$
			\item $\Gr(\mathbb{N}_{\ge 17},+) \cong \mathbb{Z}$
			\item $\Gr(\mathbb{N}_{>0},\cdot ) \cong (\mathbb{Q}_{>0},\cdot )$
			\item $\Gr(\mathbb{N}_{\ge 0},\cdot) \cong 0$ (mit $x=0$)
			\item $\Gr(\mathbb{N} \cup \set*{\infty},+) \cong 0$
		\end{itemize}
	\end{multicols}
\end{beispiel}

\begin{bemerkung}[{name=[{Funktorialität}]}]
	$S \mapsto \Gr(S)$ ist ein Funktor von der Kategorie der abelschen Halbgruppen in die Kategorie der abelschen Gruppen:
	Ist $\psi \colon S \to S'$ ein Halbgruppenhomomorphismus, so ist $\Gr(\psi) \colon \Gr(S) \to \Gr(S')$, $s-t \mapsto \psi(s) - \psi(t)$ der induzierte Gruppenhomomorphismus.
  Dieser induzierte Gruppenhomomorphismus lässt sich auch mit der universellen Eigenschaft konstruieren.
\end{bemerkung}

\begin{definition}[{name=[{$K_0$ von Ringen}]}]
	Sei $R$ ein Ring. 
	Wir definieren
	\[
		K_0(R) := \Gr \enbrace[\big]{\Proj(R),\oplus }
	\]
	Elemente in $K_0$ sind also formale Differenzen $[P]- [Q]$ von Isomorphieklassen von endlich erzeugten projektiven Moduln.
\end{definition}

Es gilt $[P] - [Q] = [P']- [Q']$ genau dann, wenn es einen endlich erzeugten projektiven Modul $X$ gibt mit
\[
	P \oplus Q' \oplus X \cong P' \oplus Q \oplus X
\]
Da jedes solche $X$ direkter Summand in einem endlich erzeugten freien Modul ist, genügt es für $X$ endlich erzeugte freie Moduln zu betrachten.
Insbesondere ist $[P]=0 \in K_0(R)$ genau dann, wenn es $n \in \mathbb{N}$ gibt mit $P \oplus R^n \cong R^n$.\todo{RevChap 1}

\begin{bemerkung}[{name=[{Funktoreigenschaften der 0-ten K-Theorie}]}]
	$K_0(R)$ ist ein Funktor: Ist $f \colon R \to R'$ ein Ringhomomorphismus, so erhalten wir eine induzierte Abbildung $\Proj(R) \to \Proj(R')$ durch $_R P \mapsto  R' \otimes_R {_R P}$, dabei wird $R'$ durch $f$ zu einem $R$-Rechtsmodul:
	\[
		r' \cdot r := r' \cdot f(r)
	\]
	Es ist nun $K_0(f) \colon K_0(R) \to K_0(R')$ gegeben durch 
	\(
		[P]- [Q] \longmapsto \benbrace*{R' \otimes_R P} - \benbrace*{R' \otimes_R Q}
	\).
\end{bemerkung}

\begin{bemerkung}[{name=[{kontravarianter Funktor}]}]
	Ist $f \colon R \to R'$ ein Ringhomomorphismus und $M'$ ein $R'$-Modul, so wird $M'$ zu einem $R$-Modul durch $r \cdot m' := f(r) \cdot m'$.
	Manchmal bezeichnet wir diesen $R$-Modul mit $\res_f M'$ oder $f^* M'$.
	Der Funktor $f^* \colon R'\text{-}\MOD \to R\text{-}\MOD$ erhält aber nicht notwendig endlich erzeugte bzw. projektive Moduln.
	Betrachte dazu die Inklusion $\mathbb{Z} \hookrightarrow \mathbb{Q}$. 
	Als $\mathbb{Q}$-Modul ist $\mathbb{Q}$ frei, als $\mathbb{Z}$-Modul nicht endlich erzeugt.
	
	Manchmal fasst man $f$ als eine Abbildung $f \colon \Spec R' \to \Spec R$ auf und schreibt $f_*$ für obigen Funktor.
\end{bemerkung}

\begin{satz}[{name=[{K0 von einem Körper}]}]
	Sei $F$ ein Körper.
	Dann erhalten wir einen Isomorphismus $\dim K_0(F) \to \mathbb{Z}$ mit 
	\[
		\dim\enbrace[\big]{[P]- [Q]} = \dim_F P - \dim_F Q
	\]
\end{satz}
\begin{beweis}
	Sei $[P]- [Q] = [P']- [Q']$ in $K_0(R)$.
	Dann gibt es einen endlich erzeugten Modul $V$ mit $P \oplus Q' \oplus V \cong P' \oplus Q \oplus V$.
	Insbesondere gilt 
	\[
		\dim P + \dim Q' + \dim V = \dim P' + \dim Q + \dim V,
	\]
	also $\dim P - \dim Q = \dim P' - \dim Q'$ und somit ist die Abbildung $\dim$ wohldefiniert auf $K_0(R)$.
	Wegen $\dim \enbrace*{[F^n]- [F^m]} = n-m$ ist $\dim$ surjektiv.
	
	Sei nun $[P]-[Q] \in K_0(R)$ mit $\dim \enbrace*{[P]- [Q]}=0$.
	Dann ist $\dim P - \dim Q =0$, also $P \cong Q$ und $[P]=[Q]$ bzw. $[P]-[Q]=0$.
\end{beweis}

\begin{definition}
	Sei $\iota \colon \mathbb{Z} \to K_0(R)$ der durch $n \mapsto \benbrace*{R^n}$ induzierte Gruppenhomomorphismus.
	Wir definieren 
	\[
		\tilde{K}_0(R) := \sfrac{K_0(R)}{\iota(\mathbb{Z})}
	\]
\end{definition}

\begin{bemerkung}
	\leavevmode
	\begin{enumerate}[1)]
		\item Im Allgemeinen ist $\iota$ weder injektiv noch surjektiv.
		\item $[P]=0 \in \tilde{K}_0(R)$ ist äquivalent zu $P$ ist \Index{stabil endlich erzeugt frei}, das heißt es gibt einen endlich erzeugten freien $R$-Modul $F$, sodass $F \oplus P$ frei ist.
		Insbesondere 
		\[
			\tilde{K}_0(R) =0 \iff \text{Jeder endlich erzeugte projektive Modul ist stabil endlich erzeugt frei} 
		\]
		\item Sei $R$ ein Ring, für den jeder endlich erzeugte projektive Modul frei von wohldefiniertem Rang ist, also $P \cong R^n$ für eindeutiges $n$.
		Dann definiert der Rang -- wie bei Körpern -- ein Inverses zu $\iota$ und wir erhalten $K_0(R)\cong \mathbb{Z}$.
		
		(Dies gilt zum Beispiel für Hauptidealringe oder lokale Ringe.)
		\item Gilt $R^n \cong R^m$ genau dann, wenn $n=m$ (also also wenn der Rang für endlich erzeugte freie Moduln wohldefiniert ist), so ist $\iota$ injektiv.
		Dies gilt zum Beispiel für kommutative Ringe oder Gruppenringen über kommutativen Ringen. 
	\end{enumerate}
\end{bemerkung}

\begin{beispiel}[{name={Eilenberg-Schwindel}}]
	Betrachte die Menge $\Proj^\infty(R)$ der Isomorphieklassen von projektiven $R$-Moduln mit einem abzählbaren Erzeugendensystem.
	Dann ist $\enbrace*{\Proj^\infty(R), \oplus }$ eine abelsche Halbgruppe.
	Dann ist aber $\Gr\enbrace*{\Proj^\infty(R),\oplus}=0$, denn zu jedem solchen Modul $P$ gibt es einen zweiten $Q$ mit $P \oplus Q \cong Q$, nämlich $Q := \bigoplus_{i=1}^\infty P$ ist auch projektiv und abzählbar erzeugt.\marginnote{man kann $Q$ auch frei wählen}
\end{beispiel}

\begin{definition}[{name=[Morita-äquivalent]}]
	Ringe $R$ und $S$ heißen \Index{Morita-äquivalent}, wenn es Bimoduln $_R X_S$ und $_S Y _R$ gibt, sodass 
	\begin{equation}
		\begin{split}
			{_R X_S} \otimes_S {_S Y _R} &\cong {_R R_R} \enspace\text{ als $R$-$R$-Bimodul} \\ 
			{_S Y _R} \otimes_R {_R X_S} &\cong {_S S_S} \enspace\text{ als $S$-$S$-Bimodul}
		\end{split} \tag{\#}\label{eq:def:morita}
	\end{equation}
\end{definition}

\begin{bemerkung}
	\eqref{eq:def:morita} impliziert, dass $_R X_S$ sowohl als $R$-Linksmodul als auch als $S$-Rechtsmodul endlich erzeugt projektiv ist,
	Gleiches gilt (umgekehrt) auch für ${_S X_R}$.
\end{bemerkung}

\begin{beispiel}
	Für jeden Ring $R$ ist $R$ Morita-äquivalent zum Ring $M_n(R)$ der $n \times n$-Matrizen über $R$ via ${_R X_{M_n(R)}}=R^n$ und $_{M_n(R)} Y_R = R^n$.
	Dabei wirkt $M_n(R)$ einmal auf $R^n$ als Spalten von links und einmal als Zeilen von rechts.
\end{beispiel}

\begin{satz}[{name=[K0 ist Morita-invariant]}]
	Sind $R$ und $S$ Morita-äquivalent, so gilt $K_0(R) \cong K_0(S)$.
\end{satz}
\begin{beweis}
	% Betrachte $\benbrace*{_R P} \mapsto \benbrace[\big]{{_S Y_R} \otimes_R {_R P}}$ ist ein Isomorphismus $\Proj(R) \to \Proj(S)$.
	% Die Inverse wird von $_R X _S$ induziert.
	Seien $x_i \in X$, $y_i \in Y$ und setze $\psi \enbrace*{\sum_{i=1}^{n} y_i \otimes x_i} = 1_S$.
	\[
		\begin{tikzcd}
			_R X \rar["\alpha"] & R^n & R^n \rar["\beta"] & X_R \\
			x \rar[mapsto] & \enbrace*{\varphi(x \otimes y_1), \ldots , \varphi(x \otimes y_n)} & (r_1, \ldots ,r_n) \rar[mapsto] & \sum_{i=1}^{n} r_i x_i
		\end{tikzcd}
	\]
	Damit ist $\beta \enbrace{\alpha(x)} = \sum_{i=1}^{n} \varphi(x \otimes y_i) x_i$.
	Betrachte nun das folgende Diagramm:\todo{Diagramm fertig machen}
	\[
		\begin{tikzcd}[sep=large]
			_R X \ar[rr,"\beta \circ \alpha"] \dar["\cong"] & & _R X \\
			{_R X} \otimes_S S \rar["\id \otimes \psi^{-1}"]  & {_R X} \otimes_S X \otimes_R X \rar["\varphi \otimes {\id}","\cong"'] & R \otimes_R X \uar["\cong"]
		\end{tikzcd}
	\]
	Damit ist $\alpha \circ \beta$ ein Isomorphismus.
\end{beweis}

\begin{korollarB}
	$K_0(R) \cong K_0 \enbrace[\big]{M_n(R)}$, ist $F$ ein Körper, so gilt $K_0 \enbrace[\big]{M_n(F)} \cong \mathbb{Z}$.
\end{korollarB}

\begin{satz}
	Sei $(\Lambda,\le)$ eine gerichtete Menge, $R \colon \Lambda \to \textsc{Ringe}$ ein Funktor.
	Dann gilt 
	\[
		K_0 \enbrace[\Big]{\colim_{\lambda \in \Lambda} R(\lambda)} \cong \colim_{\lambda \in \Lambda} K_0\enbrace[\big]{R(\lambda)}
	\]
\end{satz}
\begin{beweis}[Skizze]
	Man zeigt zunächst $\Proj \enbrace*{\colim_\Lambda R(\lambda)} = \colim_\Lambda \Proj \enbrace*{R(\lambda)}$.
	Sei $\mathcal{R} := \colim_\Lambda R(\lambda)$.
	Der entscheidende Punkt ist: Zu jedem endlich erzeugten projektiven Modul $\mathcal{R}$-Modul $\mathcal{M}$ gibt es $\lambda \in \Lambda$ und einen endlich erzeugten projektiven $R(\lambda)$-Modul $M(\lambda)$ mit $\mathcal{M} \cong \mathcal{R} \otimes_{R(\lambda)} M(\lambda)$.
	Dazu wählen wir $\tilde{p} \in M_n(\mathcal{R})$ idempotent mit $\mathcal{M} \cong \im \tilde{p}$.
	Dann gibt es $\lambda \in \Lambda$ mit $p(\lambda) \in M_n(R(\lambda))$ mit $\tilde{p} =f_\lambda(p(\lambda))$ unter $f_\lambda \colon M_n(R(\lambda)) \to M_n(\mathcal{R})$.
	Dann ist $\mathcal{M} \cong \im \tilde{p} \cong \mathcal{R} \otimes_{R(\lambda)} \im p(\lambda)$.
	Nun zeigt man, dass auch $\Gr$ mit gerichteten Kolimiten vertauscht.
\end{beweis}

\begin{beispiel}
	Sei $F$ ein Körper.
	Betrachte den Ringhomomorphismus $i_n \colon M_{2^n}(F) \to M_{2^{n+1}}(F)$ mit $i_n(A) = \begin{psmallmatrix} A & 0 \\ 0 & A \end{psmallmatrix}$.
	Sei $R= \colim_n M_n(F)$.
	Es folgt
	\[
		K_0(R) = \colim_{n \in \mathbb{N}} K_0 \enbrace*{M_n(F)} = \colim \Big(
		\begin{tikzcd}
			\mathbb{Z} \rar["(i_1)_*"] & \mathbb{Z} \rar["(i_2)_*"] & \mathbb{Z} \rar["(i_3)_*"]  & \mathbb{Z} \rar & \ldots 
		\end{tikzcd}
		\Big)
	\]
	Für $n \in \mathbb{N}$ sei $f_n \colon K_0(F) \to$ der durch die Morita-Äquivalenz $V \mapsto  F^n \otimes_R V$ induzierte Isomorphismus.
	Dann kommutiert
	\[
		\begin{tikzcd}
			\mathbb{Z} \dar["\cdot 2"] & K_0(F) \lar["\dim"] \dar["\cdot 2"] \rar["f_n"] & K_0 \enbrace*{M_{2^n}(F)} \dar["(i_n)_*"] \\
			\mathbb{Z} & K_0(F) \lar["\dim"] \rar["f_{n+1}"] & K_0 \enbrace*{M_{2^{n+1}} (F)}
		\end{tikzcd}
	\]
	Also $K_0(R) = \colim \colim \big(
		\begin{tikzcd}
			\mathbb{Z} \rar["\cdot 2"] & \mathbb{Z} \rar["\cdot 2"] & \mathbb{Z} \rar["\cdot 2"]  & \mathbb{Z} \rar & \ldots 
		\end{tikzcd}
		\big) = \mathbb{Z}\benbrace*{\frac{1}{2}}$.
\end{beispiel}
% section k_0_eines_ringes (end)
\newpage

\section{$K_0$ und Dedekind-Ringe} % (fold)
\label{sec:2}
\emph{In diesem Abschnitt ist der Ring $R$ immer nullteilerfrei, kommutativ und $Q := \mathrm{Quot}(R)$ bezeichnet den Quotientenkörper von $R$.}\todo{RevChap2}

\begin{definition}[{name=[{gebrochenes Ideal}]}]
	Ein \bet{gebrochenes Ideal}\index{Ideal!gebrochenes} von $R$ ist ein $R$-Untermodul $I \neq 0$ von $Q$, zu dem es $a \in R$ gibt mit $a I \subseteq R$.
	Jedes nichttriviale Ideal von $R$ ist auch ein gebrochenes Ideal.
	Zur besseren Unterscheidung nennt man dann Ideale von $R$ auch \bet{ganze Ideale}\index{Ideal!ganzes}.
	Für $\frac{a}{b} \in Q$, $a,b\neq 0$ ist $\frac{a R}{b}$ ein bebrochenes Ideal.
	Wir nennen solche gebrochenen Ideale \Index{gebrochene Hauptideale}.
\end{definition}

\begin{bemerkung}
	Die Menge der gebrochenen Ideale bildet bezüglich Multiplikation von gebrochenen Idealen $I \cdot J := \set[\big]{\sum_{i} a_i b_i \given a_i \in I, b_i \in J}$ eine abelsche Halbgruppe mit neutralem Element $R$.
\end{bemerkung}

\begin{definition}[{name=[{Dedekind-Ring}]}]
	$R$ heißt \Index{Dedekind-Ring}\footnote{nach Julius Wilhelm Richard \textsc{Dedekind}, * 1831 † 1916, deutscher Mathematiker}, falls die gebrochenen Ideale eine Gruppe bezüglich Multiplikation bilden. 
\end{definition}

\begin{lemma}[{name=[Invereses eines gebrochenen Ideals]}]
	Sei $R$ ein Dedekind-Ring, $I$ ein gebrochenes Ideal.
	Dann ist $I^{-1} = \set*{a \in Q \given a I \subseteq R}$.
\end{lemma}
\begin{beweis}
	Sei $J:= \set*{a \in Q \given a I \subseteq R}$.
	Wegen $I^{-1} \cdot I \subseteq R$ ist $I^{-1} \subseteq J$ und daher $R = I \cdot I^{-1} \subseteq I \cdot J \subseteq R$, also $I \cdot J =R$, also $J=I^{-1}$.
\end{beweis}

\begin{definition}[{name=[{Klassengruppe}]}]
	Die \Index{Klassengruppe} eines Dedekind-Ringes ist definiert als 
	\[
		C(R):=  \frac{\text{Gruppe der gebrochenen Ideale}}{\text{Untergruppe der gebrochenen Hauptidealringe}} 
	\]
\end{definition}

\begin{lemma}
	$C(R)=0 \iff R$ ist Hauptidealring.
\end{lemma}
\begin{beweis}
	Sei $I \subseteq R$ ein Ideal.
	Dann ist $I$ auch ein gebrochenes Ideal und aus $C(R)=0$ folgt, dass $I$ auch ein gebrochenes Hauptideal, also $I = \frac{aR}{b}$ ist.
	Wegen $\frac{a}{b} \in I \subseteq R$ ist $I$ ein Hauptideal.
	
	Sei umgekehrt $I \subseteq Q$ ein gebrochenes Ideal.
	Wähle $b \in R$ mit $b I \subseteq R$.
	Dann ist $b \cdot I=a R$, also $I= \frac{a R}{b}$ wie gewünscht. 
\end{beweis}

\begin{satz}[label=satz:2:1]
	Sei $F$ ein \Index{Zahlkörper}, das heißt $F$ ist eine endliche Körpererweiterung von $\mathbb{Q}$.
	Sei $R$ der Ring der ganzen Zahlen in $F$, das heißt $\alpha \in F$ liegt genau dann in $R$, wenn $\alpha$ Nullstelle eines normierten Polynoms über $\mathbb{Z}$ ist.
	Dann ist $R$ ein Dedekind-Ring.
\end{satz}
\todo[inline]{Referenz Rosenberg}


\begin{satz}[label=satz:klassengruppe_K_0]
	Sei $R$ ein Dedekind-Ring.
	Dann ist die Klassengruppe isomorph zu $\tilde{K}_0(R)$.
\end{satz}

\begin{lemma}[label=lem:ideale_dedekind]
	Sei $R$ ein Dedekind-Ring.
	Dann gilt
	\begin{enumerate}[a)]
		\item Ideale in $R$ sind endlich erzeugt projektiv als $R$-Moduln.
		\item Jeder projektive endlich erzeugte $R$-Modul ist isomorph zu einer Summe von Idealen 
		\[
			I_1 \oplus \ldots \oplus I_n
		\]
		\item Ist ein Ideal $I \subseteq R$ stabil frei (als $R$-Modul), dann ist $I$ ein Hauptideal.
	\end{enumerate}
\end{lemma}
\begin{beweis}
	\begin{enumerate}[a)]
		\item Sei $I \subseteq R$ ein Ideal. 
		Da $I^{-1} \cdot I=R$ ist, gibt es $a_1, \ldots, a_n \in I^{-1}$, $b_1, \ldots ,b_n \in I$ mit $1= a_1 b_1 + \ldots + a_n b_n$.
		Weiter ist $a_i I \subseteq R$ für alle $i$.
		Betrachte nun 
		\begin{align}
			p \colon R^n \longrightarrow I &\quad \text{ mit } p(r_1,\ldots ,r_n) =r_1 b_1+ \ldots r_n b_n \\
			s \colon I \longrightarrow R^n &\quad \text{ mit } s(b) = \enbrace*{b a_1, \ldots , b a_n}  
		\end{align}
		Dann ist $p \circ s(b) = b a_1 b_1 + \ldots + b a_n b_n = b \enbrace*{a_1 b_i + \ldots + a_n b_n}= b$.
		Also ist $p \circ s =\id$ und folglich ist $I$ endlich erzeugt projektiv als $R$-Modul.
		\item Wir zeigen per Induktion nach $k$: Ist $P \subseteq R^k$, so ist $P$ isomorph zu einer Summe von Idealen.
		
		Für $k=1$ ist die Aussage klar, da Untermoduln von $R$ Ideale sind. Für den Induktionsschritt $(k-1) \mapsto k$ sei $P \subseteq R^k$.
		Die Projektion auf die letzte Koordinate liefert kurze exakte Folgen 
		\[
			\begin{tikzcd}
				R^{k-1} \rar[hook] & R^k \rar["\pi"] & R \\
				P \cap R^{k-1}\uar[phantom,sloped,"\subseteq"] \rar & P \rar \uar[phantom,sloped,"\subseteq"] &  \pi(P) \uar[phantom,sloped,"\subseteq"]
			\end{tikzcd}
		\]
		Nach a) ist $\pi(P)$ projektiv und daher spaltet die unter Folge und wir können die Induktionsvorraussetzung anwenden.
		\item Vorbemerkung: Gibt es $a \in I$ mit $a^{-1} \in I^{-1}$, so ist $I$ ein Hauptideal, genauer ist $I = R \cdot a$.
		Sei dazu $x \in I$.
		Dann ist $x = (x a^{-1}) a$.
		
		Sei nun $f \colon R^n \hookrightarrow R^m$ mit $\im f = R^{m-1} \oplus I$.
		Wegen $I \neq 0$ wird $f$ über $Q$ invertierbar, insbesondere ist $n=m$.
		Sei $A \subset Q^{n \times n}$ die dazugehörigen Matrix.
		Da $A(R^n) \subseteq R^{n-1} \oplus I$ gilt, liegen alle Einträge von $A$ in $R$, die unterste Zeile sogar in $I$.
		Insbesondere $a := \det A \in I$
		
		Sei $B := A^{-1}$.
		Wegen $B \enbrace*{R^{n-1} \oplus I} \subseteq R^n$ liegen die Einträge der ersten $n-1$ Spalten in $R$ und die der letzten in $I^{-1}$.
		Insbesondere $a^{-1} = \det A^{-1} \in I^{-1}$.
		Nun ist $a \cdot a^{-1} = \det A \cdot \det A^{-1} = \det E_n =1$ und die Behauptung folgt aus der Vorbemerkung.\qedhere
	\end{enumerate}
\end{beweis}

\begin{proposition}[label=prop:summe_ideale]
	Sei $R$ ein Dedekind-Ring.
	Für Ideale $I_1$, $I_2$ gibt es dann einen $R$-linearen Isomorphismus $I_1 \oplus I_2 \cong I_1 \cdot I_2 \oplus R$.
\end{proposition}

\begin{korollar}[label=kor:isomorphismus_endl_erz_proj]
	Sei $R$ ein Dedekind-Ring.
	Dann ist jeder endlich erzeugte projektive Modul isomorph zu $R^n \oplus I$ für ein geeignetes $n$ und ein Ideal $I$.
\end{korollar}
\begin{beweis}
	Folgt mit Induktion aus \autoref{lem:ideale_dedekind} b) und \autoref{prop:summe_ideale}.
\end{beweis}

\begin{beweis}[name={von \autoref{satz:klassengruppe_K_0}}]
	Sei $f \colon C(R) \to \tilde{K}_0(R)$ die Abbildung, die ein gebrochenes Ideal $I \subseteq R$ die Klasse von $I$ als $R$-Modul $[I] \in \tilde{K}_0(R)$ zuordnet.
	Da $I\cong a \cdot I \subseteq R$ als $R$-Modul für geeignetes $a \in R$, ist $I$ nach \autoref{lem:ideale_dedekind} a) endlich erzeugt und projektiv.
	Ist $I \cong R \cdot a$ ein gebrochenes Hauptideal, so ist $I \cong R$ als $R$-Modul.
	\autoref{prop:summe_ideale} impliziert, dass $f$ ein Gruppenhomomorphismus ist und damit ist $f$ auch wohldefiniert.
	Wegen b) aus \autoref{lem:ideale_dedekind} (oder wegen \autoref{kor:isomorphismus_endl_erz_proj}) ist $f$ surjektiv.
	
	Zur Injektivität: Sei $I$ ein gebrochenes Ideal mit $[I]=0 \in \tilde{K}_0(R)$.
	Dann ist $I$ als $R$-Modul stabil frei.
	Ohne Beschränkung der Allgemeinheit können wir annehmen, dass $I$ ein Ideal ist.
	Nach c) aus \autoref{lem:ideale_dedekind} ist dann $I$ ein Hauptideal und $I$ ist trivial in $C(R)$.
\end{beweis}

\begin{satz}
	Sei $R$ ein Dedekind-Ring.
	\begin{enumerate}[a)]
		\item Alle Primideale $\neq 0$ sind maximal.
		\item Jedes Ideal $\neq 0$ lässt sich eindeutig als Produkt von Primidealen schreibem.
	\end{enumerate}
\end{satz}
\begin{beweis}
	\begin{enumerate}[a)]
		\item Sei $0 \subsetneq P \subsetneq J \subseteq R$ mit $P$ prim.
		Wegen $P \subsetneq J$ folgt $J^{-1} P \subsetneq R$.
		Es ist $P = J \cdot \enbrace*{J^{-1} P}$.
		Da $P$ prim ist, folgt $J^{-1} P \subseteq P$.
		Es folgt $R= P P^{-1} \subseteq J \subseteq R$, also $J=R$.
		\item \emph{Existenz:} Sei $\mathcal{C}$ die Menge der Ideale $\neq 0$, die sich nicht als Produkt von Primidealen schreiben lassen.
		Wir zeigen $\mathcal{C}=\emptyset$.
		Angeommen, dies wäre nicht der Fall.
		Da Ideale in $R$ endlich erzeugt sind, finden wir mit Zorns Lemma ein maximales Element $I \in \mathcal{C}$.
		
		Offenbar ist $I$ kein maximales Ideal, andernfalls wäre $I$ selbst prim.
		Sei $I \subsetneq J \subsetneq R$.
		Dann ist $I =J \enbrace*{J^{-1} I}$ und wegen $I \subseteq J$ folgt $J^{-1} I \subseteq R$.
		Aus $J \subsetneq R$ folgt durch Multiplikation mit $J^{-1} I$
		\[
			I \subsetneq J^{-1}I
		\] 
		Also gilt $J^{-1} I \notin \mathcal{C}$ und $J \notin \mathcal{C}$.
		Daher lassen sich beide als Produkt von Primidealen schreiben unt damit auch $I$.
		
		\emph{Eindeutigkeit:} Sei $P_1 \cdot \ldots \cdot P_m = Q_1 \cdot \ldots \cdot Q_n$ mit $P_i, Q_j$ prim.
		Dann ist
		\[
			P_1 \supseteq P_1 \cdot \ldots \cdot P_m = Q_1 \cdot \ldots \cdot Q_n
		\]
		Da $P_1$ prim ist, liegt eines der $Q_i$ in $P_1$, ohne Einschränkungen $Q_1 \subseteq P_1$.
		Da nach a) $Q_1$ maximal ist, folgt $Q_1=P_1$.
		Es folgt $p_2 \cdot \ldots \cdot P_m = Q_2 \cdot \ldots \cdot Q_n$ und die Behauptung folgt induktiv.
		\qedhere
	\end{enumerate}
\end{beweis}

\begin{lemma}[label=lem:]
	Sei $R$ ein kommutativer Ring.
	$P \subseteq R$ ein Primideal, $I,J$ Ideale mit $I J  \subseteq P$.
	Dann gilt $I \subseteq P$ oder $J \subseteq P$.
\end{lemma}
\begin{beweis}
	Angenommen dies gilt nicht, dann gibt es $a \in  I\setminus P$, $b \in J \setminus P$.
	Dann gilt $a \cdot b \in I \cdot J \subseteq P$, also $a \in P$ oder $b \in P$. Widerspruch!
\end{beweis}
% section 2 (end)

\cleardoubleoddemptypage
\pagenumbering{Alph}
\setcounter{page}{1}
\cleardoubleoddemptypage
\appendix

\section{Anhang} % (fold)
\label{sec:anhang}
%!TEX root = ana_top_geo.tex

\subsection{Ausführlicher Beweis zu \cref{lem:kpt-schnitte}} % (fold)
\label{sub:kpt-schnitte}
Sei $X$ ein Hausdorffraum. Dann ist $X$ genau dann kompakt, wenn gilt: Hat eine Familie $\mathcal{A}$ von abgeschlossenen Teilmengen von $X$ die endliche 
Durchschnittseigenschaft, so gilt 
\[
	\bigcap_{A \in \mathcal{A}} A \not= \emptyset.
\]
\begin{beweis}
	Für die erste Implikation sei $X$ kompakt und $\mathcal{A}$ eine Familie von abgeschlossenen Mengen mit der endlichen Durchschnittseigenschaft.
	Angenommen $\bigcap_{A \in \mathcal{A}} A = \emptyset$.
	Dann gilt
	\[
		X = X \setminus \bigcap_{A \in \mathcal{A}} A = \bigcup_{A \in \mathcal{A}} X \setminus A.
	\]
	Nun ist $\mathcal{U} \coloneqq \set*{X \setminus A \given A \in \mathcal{A}}$ eine offene Überdeckung von $X$ und da $X$ kompakt ist, existiert $\mathcal{A}_0 \subset \mathcal{A}$ endlich, sodass
	\[
		X = \bigcup_{A \in \mathcal{A}_0} X \setminus A = X \setminus \underbrace{\bigcap_{A \in \mathcal{A}_0 } A }_{\neq \emptyset} \quad \light
	\]
	Für die umgekehrte Implikation sei nun $\mathcal{U} = \set{U_i}_{i \in I}$ eine offene Überdeckung von $X$.
	Angenommen für jede endliche Teilmenge $J \subseteq I$ gilt $X \neq \bigcup_{i \in J} U_i$.
	Betrachte nun $\mathcal{A} =  \set{X \setminus U_i}_{i \in I}$. Dann gilt nach Annahme
	\[
		\bigcap_{i \in J} X \setminus U_i = X \setminus \bigcup_{i \in J} U_i \neq \emptyset.
	\]
	Also hat $\mathcal{A}$ die endliche Durchschnittseigenschaft. Nach Vorraussetzung gilt dann
	\[
		\emptyset \not= \bigcap_{i \in I} X \setminus U_i = X \setminus \underbrace{\bigcup_{i \in I} U_i}_{= X} \quad \light \qedhere
	\]
\end{beweis}


\subsection[Blatt3, Aufgabe 4: Hilfssatz für den Hauptsatz der Algebra]{Blatt 3, Aufgabe 4} % (fold)
\label{sub:B3A4}
\emph{Diese Übungsaufgabe ist zentral für den Beweis des Hauptsatzes der Algebra, \cref{satz:hauptsatz-algebra}.} 

Sei $p(x)= x^n + a_{n-1} x^{n-1} + \ldots + a_1 x + a_0$ mit $n \in \mathbb{N}_0$ ein Polynom mit Koeffizienten $a_i \in \mathbb{C}$, dass \emph{keine} Nullstelle in $\mathbb{C}$ besitzt. 
Sei $S^1= \set*{z \in \mathbb{C} \given \abs*{z}=1}$.
\begin{enumerate}[(a)]
	\item $f \colon S^1 \to S^1$ gegeben durch $f(z) = \frac{p(z)}{\abs*{p(z)} } $ ist wohldefiniert und homotop zu einer konstanten Abbildung.
	\item $f$ ist homotop zur Abbildung $g_n \colon S^1 \to S^1$ mit $g_n(z)= z^n$.
\end{enumerate}
\minisec{Beweis}
\begin{enumerate}[(a)]
	\item \begin{description}
		\item[Wohldefiniertheit:] Sei $z \in S^1$ beliebig. Dann gilt
		\[
			\abs*{\frac{p(z)}{\abs*{p(z)} } } = \frac{1}{\abs*{p(z)} } \cdot \abs*{p(z)} =1,
		\]
		also ist $f(z) \in S^1$.
		\item[Homotop zu einer konstanten Abbildung:] Definiere $f_t \colon S^1 \to S^1$ für $t \in [0,1]$ durch 
		\[
			f_t(z) = \frac{p(t \cdot z)}{\abs*{p(t \cdot z)} } 
		\]
		Dies ist mit der gleichen Begründung wie oben wohldefiniert. 
		Außerdem ist $f_0(z)= \frac{a_0}{\abs*{a_0} } \in S^1 $ konstant und $f_1(z)= \frac{p(z)}{\abs*{p(z)} }=f(z)$. 
		Definiere nun $H \colon S^1 \times [0,1] \to S^1$ durch $H(x,t) \coloneqq f_t(x)$. 
		Dann ist $H$ stetig, da Polynome und $\abs*{.} $, sowie Multiplikation stetig sind. 
		$H$ ist die gesuchte Homotopie.
	\end{description}
	\item Sei $h \colon S^1 \times [0,1] \to \mathbb{C}$ gegeben durch $h(z,t) = z^n + \sum_{k=0}^{n-1} a_k z^k t^{n-k}$. 
	Dann gilt $h(z,0)=z^n \not= 0$, da $z \in S^1$.
	Für $t \neq 0$ gilt nun
	\begin{align*}
		h(z,t) = 0 \iff \frac{h(z,t)}{t^n} = 0 \iff \frac{z^n}{t^n} + \sum_{k=0}^{n-1} a_k \frac{z^k}{t^k} = 0 \iff p \enbrace*{\frac{z}{t}} = 0
	\end{align*}
	Aber nach Vorraussetzung gilt $p \enbrace*{\frac{z}{t}} \neq 0$. 
	Also $h(z,t) \neq 0$ für alle $t \in [0,1]$. 
	Definiere nun $H \colon S^1 \times [0,1]\to S^1$ durch $H(z,t) = \frac{h(z,t)}{\abs*{h(z,t)}}$. 
	Wie eben gezeigt, ist dies wohldefiniert und offensichtlich stetig. Da
	\[
		H(z,0) = \frac{z^n}{\abs*{z^n} } = z^n \quad \text{ und } \quad H(z,1) = \frac{h(z,1)}{\abs*{h(z,1)} } = \frac{p(z)}{\abs*{p(z)} } =f(z)
	\]
	ist $H$ die gesuchte Homotopie. \qedhere
\end{enumerate}

\subsection{Blatt 10, Aufgabe 3} % (fold)
\label{sub:B10A3}
\emph{Diese Übungsaufgabe lieferte den Beweis zu \cref{prop:iso-covering}.} \smallskip \\
Sei $p \colon \overline{X} \to X$ eine Überlagerung. 
Seien $\overline{x}_0  \in \overline{X}$ und $x_0= p(\overline{x}_0 )$ Basispunkte. 
Dann ist die induzierte Abbildung $\pi_n (p) \colon \pi_n(\overline{X}, \overline{x}_0) \to \pi_n(X,x_0)$ ein Isomorphismus für alle $n \ge 2$.
\minisec{Beweis}
Als Überlagerung ist $p$ stetig, also ist $\pi_n(p)$ ein Gruppenhomomorphismus nach \hyperref[prop:eig-hom-gruppen:enum:4]{ \cref*{prop:eig-hom-gruppen} \ref*{prop:eig-hom-gruppen:enum:4}}.
\begin{description}
	\item[Surjektivität:] Sei $[\omega] \in \pi_n(X,x_0)$, also $\omega \colon I^n \to X$ mit $\omega(\partial I^n) = \set{x_0}$. Betrachte $\omega$ nun als Abbildung $I^{n-1} \times [0,1] \to X$:
	\[
		\begin{tikzcd}[column sep=4em]
			I^{n-1} \times \set{0} \dar[hook] \rar["\mathrm{const}_{\overline{x}_0}"] & \overline{X} \dar["p"]\\
			I^{n-1} \times I \rar["\omega"] & X  
		\end{tikzcd}
	\]
	$\mathrm{const}_{\overline{x}_0} \colon I^{n-1} \times \set{0}$ ist eine Hebung von $\omega\big|_{I^{n-1} \times \set{0}} \equiv x_0$. 
	Nach dem Homotopiehebungssatz (\ref{satz:hebung-homotopie}) existiert eine Hebung $\overline{\omega} \colon I^{n-1} \times I \to \overline{X}$ von $\omega$ mit $\overline{\omega}\big|_{I^{n-1} \times \set{0}} \equiv \overline{x}_0 $. 
	Also gilt
	\[
		p \circ \overline{\omega} \big|_{\partial I^n} = \omega \big|_{\partial I^n} \equiv x_0 \enspace \Longrightarrow \enspace \overline{\omega} \big|_{\partial I^n} 
		\in p ^{-1}( \set{x_0} ) .
	\]
	Da $p^{-1}(\set{x_0})$ diskret und $\partial I^n$ für $n \ge 2$ zusammenhängend ist, muss $\overline{\omega} \big|_{\partial I^n}$ konstant sein. 
	Da $\overline{\omega}\big|_{I^{n-1} \times \set{0}} \equiv \overline{x}_0 $ gilt, folgt somit $\overline{\omega}(\partial I^n) = \set{\overline{x}_0}$. 
	Also ist $[\overline{\omega}] \in \pi_n(\overline{X},\overline{x}_0)$ und weiter gilt
	\[
		\pi_n(p) \enbrace*{[\overline{\omega}]} = [p \circ \overline{\omega} ] = [\omega] \in \pi_n(X,x_0). 
	\]
	\item[Injektivität:] Sei $[\omega] \in \ker \pi_n(p)$, also $[p \circ \omega] = [c_{x_0}]$. 
	Es existiert also eine Homotopie $H$ relativ $\partial I^n$ zwischen $p \circ \omega$ und $c_{x_0}$. 
	Offensichtlich ist $\omega$ eine Hebung von $p \circ \omega$. 
	Mit dem Homotopiehebungssatz erhalten wir eine Hebung $\overline{H}$ von $H$ mit $\overline{H}(-,0) = \omega$. 
	Weiter wissen wir, dass
	\[
		\overline{H} \big|_{\partial I^n \times [0,1]} \in p ^{-1}(\set{x_0} ) \quad \text{ und }\quad  \overline{H} \big|_{ I^n \times \set{1}} \in p ^{-1}(\set{x_0} )
	\]
	gelten muss, da $H = p \circ \overline{H}$ und $H(-,1)= c_{x_0} \equiv x_0$. 
	Mit dem gleichen Argument wie oben folgt, dass $\overline{H} \big|_{\partial I^n \times [0,1]}$ und $\overline{H} \big|_{ I^n \times \set{1}}$ konstant sind. 
	Für $z \in \partial I^n$ gilt nun
	\[
		\overline{H}(z,0) = \omega(z) = \overline{x}_0
	\]
	Da $\partial I^n \times [0,1] \cap I^n \times \set{1} \not= \emptyset$, muss also auch $\overline{H}(-,1) \equiv \overline{x}_0$ gelten. 
	Damit folgt $[\omega] = [c_{x_0}]$.\qedhere
\end{description}
\printindex
\printbibliography
\listoffigures
\todototoc
\listoftodos[To-do's und andere Baustellen]
\end{document}
