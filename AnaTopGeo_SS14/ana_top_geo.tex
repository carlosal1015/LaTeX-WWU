%!TEX TS-program = xelatex
%!TEX TS-options = -shell-escape
%!TEX root = ../OpAlg2_SS16/operatoralgebren2.tex
\RequirePackage{fix-cm} 
\documentclass[a4paper, twoside, headsepline, index=totoc,toc=listof,toc=bibliography,toc=index, fontsize=10pt, cleardoublepage=empty, headinclude, DIV=12, BCOR=5mm, titlepage,draft]{scrartcl}
%!TEX root = ../AnaTopGeo_SS14/ana_top_geo.tex
\usepackage{scrtime} % KOMA, Uhrzeit ermoeglicht

%--Pakete zum "Programmieren"
% ======================================================================================
\usepackage{etoolbox}
\usepackage{letltxmacro}
\usepackage{ifthen}
% ======================================================================================

%--Farbdefinitionen und Grafiken (muss vor tikz geladen werden)
% ======================================================================================
\usepackage[usenames, table, x11names]{xcolor}
\definecolor{dark_gray}{gray}{0.45}
\definecolor{light_gray}{gray}{0.6}
\definecolor{fb10_blue}{cmyk}{0.8,0.4,0.13,0.07}
\usepackage[final]{graphicx}
\usepackage{adjustbox}
\newcommand{\cfbox}[2]{% coloured frame box
	\ifmmode
	\mathchoice{\adjustbox{cfbox=#1}{$\displaystyle#2$}}{\adjustbox{cfbox=#1}{$\textstyle#2$}}{\adjustbox{cfbox=#1}{$\scriptstyle#2$}}{\adjustbox{cfbox=#1}{$\scriptscriptstyle#2$}}
	\else
	\adjustbox{cfbox=#1}{#2}
	\fi
}
% ======================================================================================

%--Zum Zeichnen/ TikZ-Kram (vor polyglossia bzw. babel geladen werden)
% ======================================================================================
\usepackage{tikz}
\usepackage{tikz-cd}
\usetikzlibrary{external}
\tikzset{>=latex}
\usetikzlibrary{%
	shapes,
	arrows.meta,
	intersections,
	calc,
	3d,
	decorations.pathreplacing,decorations.markings,decorations.pathmorphing,
	angles,
	quotes,
}
\tikzexternalize[prefix=tikz/,up to date check=diff]
\pgfkeys{/pgf/images/include external/.code=\includegraphics{#1}}
\tikzset{external/system call={lualatex \tikzexternalcheckshellescape -halt-on-error -interaction=batchmode --shell-escape -jobname "\image" "\texsource"}}
\AtBeginEnvironment{tikzcd}{\tikzexternaldisable} % tikzexternalize fuer tikzcd deaktivieren, da inkompatibel
\AtEndEnvironment{tikzcd}{\tikzexternalenable}
\tikzset{% um Inkompatibilitaeten von quotes und polyglossia bzw. babel zu vermeiden
  every picture/.append style={
    execute at begin picture={\shorthandoff{"}},
    execute at end picture={\shorthandon{"}}
  }
}
\usepackage{pgfplots}
\usepgfplotslibrary{colormaps}
\newcommand*\circled[1]{\tikzexternaldisable\tikz[baseline=(char.base)]{\node[shape=circle,draw,inner sep=2pt] (char) {#1};}\tikzexternalenable}
% ======================================================================================



%-- Mathepakete etc.
% ======================================================================================
\usepackage[T1]{fontenc}
\renewcommand{\rmdefault}{zpltlf}
\usepackage{mathtools} % beinhaltet amsmath
\mathtoolsset{showonlyrefs,centercolon,showmanualtags}
\newtagform{brackets}[\textbf]{[}{]}
\usetagform{brackets}
\usepackage{fix-cm}
\usepackage[bbgreekl]{mathbbol}
\usepackage{amssymb,marvosym} 
\usepackage{nicefrac} % schräge Brüche
\usepackage{faktor}
\newcommand{\Faktor}[1]{\faktor[\textstyle]{#1}}
\usepackage{xfrac}
\usepackage{cancel}
\usepackage{mathdots} % Verbesserung von Punkten wie zB \ldots
\usepackage[bb=px]{mathalfa} % \mathbb als px font
\usepackage{centernot}
\usepackage{stackrel}
\DeclareSymbolFont{bbold}{U}{bbold}{m}{n}
\DeclareSymbolFontAlphabet{\mathbbold}{bbold}
\newcommand{\ind}{\mathbbold{1}} % charakteristische-Funktion-Eins
\def\mathul#1#2{\color{#1}\underline{{\color{black}#2}}\color{black}} %farbiges Untersteichen im Mathe-Modus
\renewcommand{\le}{\leqslant}
\renewcommand{\ge}{\geqslant}
% ======================================================================================


%-- Von xfrac erzeuge font warnings ignorieren
% ======================================================================================
\usepackage{silence}
\WarningFilter{latexfont}{Size substitutions with differences}
\WarningFilter{latexfont}{Font shape `U/bbold/m/n' in size}
% ======================================================================================


%-- Typographie/Polyglossia
% ======================================================================================
\usepackage[euler-digits]{eulervm} % vor fontspec laden!
\usepackage[no-math]{fontspec}
\usepackage{polyglossia} % moderner babel-ersatz
\setmainlanguage[spelling=new,babelshorthands=true]{german}
\shorthandoff{"}
\setotherlanguage{english}
\defaultfontfeatures{Mapping=tex-text, WordSpace={1.2}, Ligatures={Required,Common,Contextual},Extension=.otf} %


\setmainfont{TeXGyrePagellaX}[UprightFont=*-Regular,BoldFont=*-Bold,ItalicFont=*-Italic,BoldItalicFont=*-BoldItalic,ItalicFeatures={Style=Historic},Ligatures={Required,Common,Contextual,Historic}]
\setsansfont{texgyreadventor}[Scale=MatchUppercase, UprightFont=*-regular, BoldFont=*-bold, ItalicFont=*-italic, BoldItalicFont=*-bolditalic]
\setmonofont{SourceCodePro}[Scale=0.9,UprightFont=*-Regular, BoldFont=*-Semibold, ItalicFont=*-Light]
\usepackage{xltxtra}
\usepackage{fontawesome}
\usepackage[final]{microtype}
\usepackage[draft=false]{scrlayer-scrpage} 
\flushbottom
% ======================================================================================


%-- Aufzählungen
% ======================================================================================
\usepackage[shortlabels,inline]{enumitem}
\setlist[itemize,1]{label=\faCaretRight}
\setlist[enumerate]{font=\bfseries}
\setlist[description]{font=\normalfont\bfseries}
\usepackage{multicol}
% ======================================================================================


%-- Floats/Figures/Tabellen
% ======================================================================================
\usepackage{wrapfig}
\usepackage{float}
\usepackage[margin=10pt, font=small, labelfont={sf, bf}, format=plain, indention=1em]{caption}
\captionsetup[wrapfigure]{name=Abb. }
\usepackage{booktabs}
% ======================================================================================


%-- korrekte Anführungszeichen und Zitierbefehle
% ======================================================================================
\usepackage[autostyle,german=quotes,english=british]{csquotes}
% ======================================================================================


%--Indexverarbeitung
% ======================================================================================
\usepackage{makeidx}
\newcommand{\bet}[1]{\textbf{\emph{#1}}}
\newcommand{\Index}[1]{\bet{#1}\index{#1}}
\makeindex
\setindexpreamble{{\noindent\sffamily\small Die \emph{Seitenzahlen} sind mit Hyperlinks versehen und somit anklickbar} \par \bigskip}
\renewcommand{\indexpagestyle}{scrheadings}
% ======================================================================================


%-- Marginnotes/Todonotes/Footnotes
% ======================================================================================
\deffootnote[1.5em]{1.5em}{1.5em}{\textsuperscript{\thefootnotemark}\ }
\usepackage[fulladjust]{marginnote}
\renewcommand*{\marginfont}{\itshape\footnotesize}
\usepackage[textsize=small]{todonotes}
\usepackage{ragged2e}
\renewcommand*{\raggedleftmarginnote}{\RaggedLeft}
\renewcommand*{\raggedrightmarginnote}{\RaggedRight}
\LetLtxMacro{\oldtodo}{\todo}
\renewcommand{\todo}[2][]{\tikzexternaldisable\oldtodo[#1]{#2}\tikzexternalenable}
\LetLtxMacro{\oldmissingfigure}{\missingfigure}
\renewcommand{\missingfigure}[2][]{\tikzexternaldisable\oldmissingfigure[{#1}]{#2}\tikzexternalenable}
% ======================================================================================


% -- BibLaTeX
% ======================================================================================
\usepackage[%
	backend=biber,
	sortlocale=auto,
	natbib,
	hyperref,
	backref,
	style=alphabetic
	]%
{biblatex}
\renewcommand*{\mkbibnamelast}[1]{%
  \ifmknamesc{\textsc{#1}}{#1}}
\renewcommand*{\mkbibnameprefix}[1]{%
  \ifboolexpr{ test {\ifmknamesc} and test {\ifuseprefix} }
    {\textsc{#1}}
    {#1}}
\def\ifmknamesc{%
  \ifboolexpr{ test {\ifcurrentname{labelname}}
               or test {\ifcurrentname{author}}
               or ( test {\ifnameundef{author}} and test {\ifcurrentname{editor}} ) }}
\addbibresource{../!config/quellen.bib}
% ======================================================================================

%--Konfiguration von Hyperref und Cleveref
% ======================================================================================
\usepackage[hidelinks, pdfpagelabels,  bookmarksopen=true, bookmarksnumbered=true, linkcolor=black, urlcolor=SkyBlue2, plainpages=false,pagebackref, citecolor=black, hypertexnames=true, pdfauthor={Jannes Bantje}, pdfborderstyle={/S/U}, linkbordercolor=SkyBlue2, colorlinks=false,final,backref=false]{hyperref}
\usepackage[nameinlink,noabbrev]{cleveref}
\newcommand{\appendLink}[1]{#1\,\faExternalLink}
\newcommand{\hrefsym}[2]{\href{#1}{\texttt{\appendLink{#2}}}}
\newcommand{\hrefsymX}[2]{\href{#1}{\appendLink{#2}}}
\newcommand{\hrefsymmail}[2]{\href{#1}{\texttt{\faEnvelopeO\,#2}}}
\renewcommand{\url}[1]{\hrefsym{#1}{\nolinkurl{#1}}}
% ======================================================================================


% -- QR-Codes (hinter hyperref laden!)
% ======================================================================================
\usepackage{qrcode}
% ======================================================================================

%--Römische Zahlen
% ======================================================================================
\newcommand{\RM}[1]{\MakeUppercase{\romannumeral #1{}}}
% ======================================================================================

%-- Definition von diversen Mathe-Befehlen
% ======================================================================================
%!TEX root = mitschrift_main.tex

% -- Zum Finetuning von Befehlen
% ======================================================================================
\makeatletter
\newcommand{\raisemath}[1]{\mathpalette{\raisem@th{#1}}}
\newcommand{\raisem@th}[3]{\raisebox{#1}{$#2#3$}}
\makeatother
\makeatletter
\newcommand{\killDescendersM}[1]{\mathpalette{\killD@scendersM{#1}}}
\newcommand{\killD@scendersM}[2]{\raisebox{0pt}[\height][0pt]{$#2#1$}}
\makeatother
\DeclareRobustCommand{\minwidthbox}[2]{%
  \ifmmode
    \expandafter\mathmakebox
  \else
    \expandafter\makebox
  \fi
  [\ifdim#2<\width\width\else#2\fi]{#1}%
}
% ======================================================================================


%-- Klammerbefehle
% ======================================================================================
\DeclarePairedDelimiter{\abs}{\lvert}{\rvert}
\DeclarePairedDelimiter{\floor}{\lfloor}{\rfloor}
\DeclarePairedDelimiter{\ceil}{\lceil}{\rceil}
\DeclarePairedDelimiter\norm{\Vert}{\Vert}
\DeclarePairedDelimiter\enbrace{(}{)}
\DeclarePairedDelimiter\benbrace{[}{]}
\DeclarePairedDelimiter\bbenbrace{[\![}{]\!]}
\DeclarePairedDelimiter\lenbrace{<}{>}
\DeclarePairedDelimiter\angbrace{\langle}{\rangle}
\newcommand{\ssbrace}[1]{{\scriptscriptstyle\enbrace{#1}}}
\newcommand{\ssbbrace}[1]{{\scriptscriptstyle\benbrace{#1}}}
% ======================================================================================

%-- Mengen
% ======================================================================================
\newcommand\SetSymbol[1][]{\nonscript\:#1\vert\allowbreak\nonscript\:\mathopen{}}
\providecommand\given{} % to make it exist
\DeclarePairedDelimiterX\set[1]\{\}{\renewcommand\given{\SetSymbol[\delimsize]}#1}
% ======================================================================================

%-- Skalarprodukt (3 Varianten) 
% ======================================================================================
\DeclarePairedDelimiterX\sprod[2]{\langle}{\rangle}{#1\,\delimsize\vert\,#2}
\DeclarePairedDelimiterX\skal[2]{\langle}{\rangle}{#1\,,\,#2}
\makeatletter
\DeclareFontFamily{OMX}{MnSymbolE}{}
\DeclareSymbolFont{MnLargeSymbols}{OMX}{MnSymbolE}{m}{n}
\SetSymbolFont{MnLargeSymbols}{bold}{OMX}{MnSymbolE}{b}{n}
\DeclareFontShape{OMX}{MnSymbolE}{m}{n}{
    <-6>  MnSymbolE5
   <6-7>  MnSymbolE6
   <7-8>  MnSymbolE7
   <8-9>  MnSymbolE8
   <9-10> MnSymbolE9
  <10-12> MnSymbolE10
  <12->   MnSymbolE12
}{}
\DeclareFontShape{OMX}{MnSymbolE}{b}{n}{
    <-6>  MnSymbolE-Bold5
   <6-7>  MnSymbolE-Bold6
   <7-8>  MnSymbolE-Bold7
   <8-9>  MnSymbolE-Bold8
   <9-10> MnSymbolE-Bold9
  <10-12> MnSymbolE-Bold10
  <12->   MnSymbolE-Bold12
}{}
\let\llangle\@undefined
\let\rrangle\@undefined
\DeclareMathDelimiter{\llangle}{\mathopen}%
                     {MnLargeSymbols}{'164}{MnLargeSymbols}{'164}
\DeclareMathDelimiter{\rrangle}{\mathclose}%
                     {MnLargeSymbols}{'171}{MnLargeSymbols}{'171}
\makeatother
\DeclarePairedDelimiterX\sskal[2]{\llangle}{\rrangle}{#1\,,\,#2}
% ======================================================================================

%-- Abbildungsdefinition
% ======================================================================================
\newcommand{\mapdef}[5]{%
	\[
		\begin{array}{rcl}
			\textstyle #1 &\xrightarrow{\minwidthbox{#5}{2em}} & \textstyle #2 \\[0.5ex]
			\textstyle #3 &\xmapsto{\minwidthbox{\mbox{ }}{2em}} & \textstyle #4
		\end{array}
	\]
}
% ======================================================================================

%-- modifiziertes Stackrel 
% ======================================================================================
\newcommand{\StackText}[2]{\stackrel{\mbox{\scriptsize #1}}{#2}}
\newcommand{\StackTextClap}[2]{\stackrel{\mathclap{\mbox{\scriptsize #1}}}{#2}}
% ======================================================================================

%-- Blitz
% ======================================================================================
\newcommand{\light}{\text{\raisebox{-.3ex}{\Large\Lightning}}}
% ======================================================================================


%-- Underbrace u.Ä. als Befehl in LaTeX-Syntax (und ohne Spacingprobleme mit nachfolgenden Operatoren...)
% ======================================================================================
\newcommand{\Underbrace}[2]{{\underbrace{#1}_{#2}}}
\newcommand{\Underbracket}[2]{{\underbracket[0.7pt][2pt]{#1}_{#2}}}
\newcommand{\Overbracket}[2]{{\overbracket[0.7pt][2pt]{#1}^{#2}}}
% ======================================================================================


%-- Deklaration weiterer Operatoren (allgemein)
% ======================================================================================
\DeclareMathOperator{\re}{Re} % Realteil
\let\Re\relax
\DeclareMathOperator{\Re}{Re} % Realteil
\DeclareMathOperator{\im}{im} % Bild
\let\Im\relax
\DeclareMathOperator{\Im}{Im} % Bild
\DeclareMathOperator{\id}{id} % identische Abbildung
\DeclareMathOperator{\conj}{conj} % Konjugation
\DeclareMathOperator{\sgn}{sgn} % Signum
\DeclareMathOperator{\End}{End} % Endomorphismen
\DeclareMathOperator{\Hom}{Hom} % Homomorphismen
\DeclareMathOperator{\Iso}{Iso} % Isomorphismen
\DeclareMathOperator{\Aut}{Aut} % Automorphismen
\DeclareMathOperator{\Span}{span} % Span
\DeclareMathOperator{\coker}{coker} % Kokern
\DeclareMathOperator{\Tr}{Tr} % Spur,Trace
\DeclareMathOperator{\pr}{pr} % Projektion
\DeclareMathOperator{\diag}{diag} % Diagonalmatrix
\DeclareMathOperator{\Rg}{Rg} % Rang
\DeclareMathOperator{\const}{const} % konstante Abbildung
\DeclareMathOperator{\Spur}{Spur} % Spur
\DeclareMathOperator{\Arg}{Arg} % Argument
\DeclareMathOperator{\dist}{dist} % Distanz
\DeclareMathOperator{\supp}{supp} % Träger
\DeclareMathOperator{\Char}{char} % Charakteristik
% ======================================================================================


%-- Deklaration weiterer Operatoren (Differentiale etc.)
% ======================================================================================
\DeclareMathOperator{\grad}{grad} % Gradient
\DeclareMathOperator{\dive}{div} % Gradient
\DeclareMathOperator{\rot}{rot} % Rotation
\newcommand{\D}{\ensuremath{\mathrm{D}\mkern-1.0mu}} % Differential
\newcommand{\mathd}{\ensuremath{\mathrm{d}\mkern-1.0mu}} % äußere Ableitung
\newcommand{\Tmap}{\ensuremath{\mathrm{T}\mkern-0.85mu}} % Tangentialraum
\let\Tang\Tmap
\DeclareMathOperator{\Diff}{Diff}
\newcommand{\diff}[2]{\ensuremath{\frac{{\partial #1}}{{\partial #2}} }}
\newcommand{\diffd}[2]{\ensuremath{\frac{\mathd #1}{\mathd #2} }}
\DeclareMathOperator{\rank}{rank}
% ======================================================================================


%-- Deklaration weiterer Operatoren (Topologie)
% ======================================================================================
\newcommand*\interior[1]{\overset{\smash{\raisebox{-0.18ex}{$\scriptstyle\circ$}}}{#1}}
\newcommand{\sing}{{\raisemath{1.1pt}{\scriptscriptstyle\mathrm{sing}}}}
\newcommand{\pt}{\mathrm{pt}}
\DeclareMathOperator{\Zyl}{Zyl}
\newcommand{\rZyl}{\widetilde{\Zyl}}
\DeclareMathOperator{\Tel}{Tel}
\newcommand{\op}{\mathrm{op}}
\DeclareMathOperator{\Sp}{Sp}
\DeclareMathOperator{\Keg}{Keg}
\newcommand{\slashedi}{i\hspace{-3.5pt}/}
\newcommand{\cupp}{\smallsmile}
\newcommand{\capp}{\smallfrown}
\DeclareMathOperator*{\colim}{colim}
\DeclareMathOperator{\PD}{PD}
\newcommand{\lf}{\mathrm{lf}}
\DeclareMathOperator{\sig}{sig}
\DeclareMathOperator{\Tor}{Tor}
\DeclareMathOperator{\Ext}{Ext}
\DeclareMathOperator{\AW}{AW}
\DeclareMathOperator{\Proj}{Proj}
\DeclareMathOperator{\Gr}{Gr}
\DeclareMathOperator{\res}{res}
\DeclareMathOperator{\Spec}{Spec}
\DeclareMathOperator{\co}{co}
\DeclareMathOperator{\ch}{ch}
\DeclareMathOperator{\wOp}{w}
\DeclareMathOperator{\Ar}{Ar}
\newcommand{\actson}{\mathrel{\curvearrowright}}
\let\acts\actson
\let\action\actson
\DeclareMathSymbol{\bbDelta}{\mathord}{bbold}{"01}
\newcommand{\DDelta}{\bbDelta}
\DeclareMathOperator{\Star}{Star}
\DeclareMathOperator{\Link}{Link}
\DeclareMathOperator{\EPK}{EPK}
\DeclareMathOperator{\Vol}{Vol}
\newcommand{\cell}{{\raisemath{1.1pt}{\scriptscriptstyle\mathrm{cell}}}}
\DeclarePairedDelimiter{\homologieklasse}{\llbracket}{\rrbracket}
\newcommand{\rand}[1]{\ensuremath{\partial^{\scriptscriptstyle #1}}}
\DeclareMathOperator{\ab}{ab}
\DeclareMathOperator{\CW}{CW}
% ======================================================================================


%-- Deklaration von Operatoren (Liegruppen)
% ======================================================================================
\DeclareMathOperator{\GL}{GL}
\DeclareMathOperator{\SO}{SO}
\DeclareMathOperator{\Ad}{Ad}
\DeclareMathOperator{\ad}{ad}
\DeclareMathOperator{\On}{O}
\DeclareMathOperator{\Un}{U}
\DeclareMathOperator{\SU}{SU}
\DeclareMathOperator{\Mat}{Mat}
\DeclareRobustCommand{\Der}{\mathop{\mathfrak{der}}}
\DeclareMathOperator{\SL}{SL}
\DeclareMathOperator{\Graph}{Graph}
\DeclareMathOperator{\Int}{Int}
\DeclareRobustCommand{\intAlg}{\mathop{\mathfrak{int}}}
\DeclareMathOperator{\aut}{aut}
\DeclareMathOperator{\Rad}{Rad}
\DeclareMathOperator{\Nil}{Nil}
\DeclareMathOperator{\rad}{rad}
\DeclareMathOperator{\nil}{nil}
\DeclareMathOperator{\Ric}{Ric}
\DeclareMathOperator{\ric}{ric}
\newcommand{\bi}{\mathrm{bi}}
\DeclareMathOperator{\Isom}{Isom}
\DeclareMathOperator{\Sym}{Sym}
\newcommand{\opL}{\ensuremath{\mathrm{L}\mkern-0.6mu}}
% ======================================================================================

%-- Deklaration von Operatoren (Funktionalanalysis)
% ======================================================================================
\DeclareMathOperator{\tr}{tr}
\newcommand{\w}{\mkern1mu\mathrm{w}}
\newcommand{\sa}{\mathrm{sa}}
\newcommand{\vb}{\mathrm{v\mkern-2.5mu.b\mkern-1.5mu.}} % vollständig beschränkt
\newcommand{\so}{\mathrm{\mkern.3mu s\mkern-1.4mu.\mkern-.6mu o\mkern-1.7mu.}} % \newcommand{\so}{\mathrm{s.o.}}
\newcommand{\solim}{\so\text{-}\mkern-0.8mu\lim}
\newcommand{\wo}{\mathrm{w\mkern-3mu.\mkern-.4mu o\mkern-1.7mu.}}
\newcommand{\Top}[1]{\mathcal{T}_{\mkern-2.3mu #1}}
\newcommand{\weakT}[1]{\ensuremath{\mathcal{T}_{#1}^{\mkern+1.0mu\text{\raisebox{0.4ex}{$\mathrm{w}$}}}}}
\newcommand{\weakTstar}[1]{\ensuremath{\mathcal{T}_{#1}^{\mkern+1.0mu\text{\raisebox{0.4ex}{$\mathrm{w}$}}^*}}}
\newcommand{\TWeakStar}{\Top{\w^*}}
\newcommand{\TWeakOp}{\Top{\wo}}
\newcommand{\Tso}{\Top{\so}}
\newcommand{\finSub}{\subset\mkern-0.7mu \subset}
\DeclareMathOperator{\Inv}{Inv}
\newcommand{\simm}{{\hspace{-1.6pt}\raisemath{0.5pt}{\sim}}}
\newcommand{\plus}{{\hspace{-1.6pt}+}}
\DeclareMathOperator{\ev}{ev}
\DeclareMathOperator{\Alg}{Alg}
\DeclareMathOperator{\her}{her}
\newcommand{\subher}{\subset_{\her}}
\newcommand{\grenzw}[1]{\xrightarrow{\minwidthbox{#1}{1.4em}}}
\newcommand{\grenzwl}[1]{\xleftarrow{\minwidthbox{#1}{1.4em}}}
\newcommand{\grenzwIn}[1]{\grenzw{\raisemath{-2pt}{#1}}}
\newcommand{\MyTo}[1]{\tikzexternaldisable\mathbin{\tikz[baseline] \draw[-to,line width=.4pt] (0ex,0.94ex) -- (#1,0.94ex);}\tikzexternalenable}
\newcommand{\dlim}{%
    \mathchoice
      {\lim\limits_{\MyTo{4.2ex}}}% \displaystyle
      {\lim\limits_{\MyTo{2.8ex}}}% \textstyle
      {\lim\limits_{\MyTo{2.3ex}}}% \scriptstyle
      {\lim\limits_{\MyTo{2.3ex}}}% \scriptscriptstyle
}
\newcommand{\Dlim}{\killDescendersM{\dlim}}
\DeclareMathOperator{\sep}{sep}
\DeclareMathOperator{\diam}{diam}
\DeclareMathOperator{\conv}{conv}
\DeclareMathOperator{\Prim}{Prim}
\DeclareMathOperator{\hull}{hull}
\DeclareMathOperator{\red}{red}
\DeclarePairedDelimiterX\bra[1]{\langle}{\rvert}{#1\,}
\DeclarePairedDelimiterX\ket[1]{\lvert}{\rangle}{\,#1}
\DeclarePairedDelimiterX\bracket[2]{\langle}{\rangle}{#1\,\delimsize\vert\,#2}
\newcommand{\tensormax}{\mathbin{\otimes_{\max}}}
\newcommand{\tensormin}{\mathbin{\otimes_{\min}}}
\DeclareMathOperator{\Ped}{Ped}
\newcommand{\alg}{\mathrm{alg}}
\DeclareMathOperator{\CPC}{CPC}
\DeclareMathOperator{\CP}{CP}
\DeclareMathOperator{\UPC}{UPC}
\newcommand{\DeltaOp}{\mathbin{\Delta}}
\newcommand{\kernedP}{\mathcal{P}\mkern-2mu}
\newcommand{\Pinfty}{\kernedP_{\infty}}
\DeclareMathOperator{\Groth}{Groth}
\DeclareMathOperator{\rk}{rk}
\newcommand{\MvN}{\mathrm{MvN}}
% ======================================================================================

%-- Kategorien
% ======================================================================================
\DeclareMathOperator{\Mor}{Mor}
\DeclareMathOperator{\mor}{mor}
\DeclareMathOperator{\Obj}{Obj}
\DeclareMathOperator{\Ob}{Ob}
\newcommand{\TOP}{\textsc{Top}}
\newcommand{\HTOP}{\textsc{HTop}}
\newcommand{\VR}{\textsc{VR}}
\newcommand{\MOD}{\textsc{Mod}}
\newcommand{\Mod}[1]{#1\text{-}\MOD}
\newcommand{\MONOIDE}{\textsc{Monoide}}
\newcommand{\SET}{\textsc{Set}}
\newcommand{\MAN}{\textsc{Man}}
\newcommand{\GRUPPEN}{\textsc{Gruppen}}
\newcommand{\ABELGRUPPEN}{\textsc{Abel.Gruppen}}
\newcommand{\ABEL}{\textsc{Abel}}
\newcommand{\KAT}{\textsc{Kat}}
\newcommand{\FUN}{\textsc{Fun}}
\newcommand{\SIMP}{\textsc{Simp}}
\newcommand{\VEKT}{\textsc{Vekt}}
\newcommand{\CH}{\textsc{Ch}}
\newcommand{\CSTARUN}{C^*\text{-}\textsc{Alg}^{\raisemath{-2.5pt}{1}}}
\newcommand{\CSTAR}{C^*\text{-}\textsc{Alg}}
\newcommand{\AB}{\textsc{Ab}}
% ======================================================================================
% ======================================================================================



% -- theorem packages
% ======================================================================================
\usepackage{amsthm}
\usepackage{thmtools,thm-restate}
\usepackage{mdframed}
\renewcommand{\listtheoremname}{Übersicht aller Aussagen}
\usepackage{bookmark}
\bookmarksetup{open,numbered}
\makeatletter
\newcommand*{\theorembookmark}{%
  \bookmark[
    dest=\@currentHref,
    rellevel=1,
    keeplevel,
  ]{%
    \thmt@thmname\space\csname the\thmt@envname\endcsname
    \ifx\thmt@shortoptarg\@empty
    \else
      \space(\thmt@shortoptarg)%
    \fi
  }%
}   
\makeatother
% ======================================================================================

% -- Definition der einzelnen Theorem-Umgebungen
% ======================================================================================
\declaretheoremstyle[%
	headfont=\sffamily\bfseries,
	notefont=\normalfont\sffamily\scshape,
	bodyfont=\normalfont,
	headformat=\NUMBER\ \NAME\NOTE,
	headpunct=.,
	postheadspace=1em,
	spaceabove=15pt,spacebelow=10pt,
	shaded={bgcolor=gray!20},
	postheadhook=\theorembookmark]%
{mainstyle}
\declaretheoremstyle[%
	headfont=\sffamily\bfseries,
	notefont=\normalfont\sffamily\scshape,
	bodyfont=\normalfont,
	headformat=\NUMBER\ \NAME\NOTE,
	headpunct=.,
	postheadspace=1em,
	spaceabove=15pt,spacebelow=10pt,
	shaded={bgcolor=fb10_blue!20},
	postheadhook=\theorembookmark]%
{mainstyle_blue}
\declaretheoremstyle[%
	headfont=\sffamily\bfseries,
	notefont=\normalfont\sffamily\scshape,
	bodyfont=\normalfont,
	headformat=\NUMBER\ \NAME\NOTE,
	headpunct=.,
	postheadspace=1em,
	spaceabove=15pt,spacebelow=10pt,
	postheadhook=\theorembookmark]%
{mainstyle_unshaded}
\declaretheoremstyle[%
	headfont=\sffamily\bfseries,
	notefont=\normalfont\sffamily\scshape,
	bodyfont=\normalfont,
	headformat=\NUMBER\NAME\NOTE,
	headpunct=.,
	postheadspace=1em,
	spaceabove=15pt,spacebelow=10pt,
	% shaded={bgcolor=gray!20},
	postheadhook=\theorembookmark]%
{mainstyle_unnumbered}
\declaretheoremstyle[%
	headfont=\sffamily\bfseries,
	notefont=\normalfont\sffamily\scshape,
	bodyfont=\normalfont,
	headformat=swapnumber,
	headpunct=.,
	postheadspace=1em,
	spaceabove=15pt,spacebelow=10pt,
	shaded={bgcolor=gray!20},
	postheadhook=\theorembookmark,
	qed=\qedsymbol]%
{mainstyleB}
\declaretheoremstyle[%
	headfont=\bfseries\scshape,
	bodyfont=\normalfont,
	headpunct=:,
	postheadspace=1em,
	spacebelow=12pt,spaceabove=2pt,
	qed=\qedsymbol]%
{beweise}
\declaretheoremstyle[%
	headfont=\bfseries\scshape,
	bodyfont=\normalfont,
	headpunct=:,
	postheadspace=1em,
	spacebelow=12pt,spaceabove=2pt]%
{beweisskizze}
\declaretheoremstyle[%
	headfont=\sffamily\bfseries,
	bodyfont=\normalfont,
	headpunct=:,
	postheadspace=1em,
	spacebelow=10pt,spaceabove=10pt]%
{bemerkungen}
\declaretheorem[name=Definition,parent=section,style=mainstyle_blue]{definition}
\declaretheorem[name=Definition \& Proposition,refname=Proposition,sharenumber=definition,style=mainstyle_blue]{definitionP}
\declaretheorem[name=Definition,numbered=no,style=mainstyle_unnumbered]{definition*}
\declaretheorem[name=Theorem,sharenumber=definition,style=mainstyle]{theorem}
\declaretheorem[name=Theorem,numbered=no,style=mainstyle_unnumbered]{theorem*}
\declaretheorem[name=Proposition,sharenumber=definition,style=mainstyle,refname=Proposition]{proposition}
\declaretheorem[name=Lemma,sharenumber=definition,style=mainstyle]{lemma}
\declaretheorem[name=Satz,sharenumber=definition,style=mainstyle,refname=Satz]{satz}
\declaretheorem[name=Satz,sharenumber=definition,style=mainstyle_unshaded]{satzUnshaded}
\declaretheorem[name=Definition,sharenumber=definition,style=mainstyle_unshaded]{definitionUnshaded}
\declaretheorem[name=Satz,numbered=no,style=mainstyle_unnumbered]{satz*}
\declaretheorem[name=Korollar,sharenumber=definition,style=mainstyle,refname=Korollar]{korollar}
\declaretheorem[name=Korollar,sharenumber=definition,style=mainstyleB,refname=Korollar]{korollarB}
\declaretheorem[name=Frage,numbered=no,style=mainstyle_unnumbered]{frage}
\declaretheorem[name=Frage,sharenumber=definition,style=mainstyle_unshaded]{frageA}
\declaretheorem[name=Erinnerung,sharenumber=definition,style=mainstyle_unshaded]{erinnerungA}
\declaretheorem[name=Ausblick,sharenumber=definition,style=mainstyle_unshaded]{ausblick}
\declaretheorem[name=Konvention,sharenumber=definition,style=mainstyle]{konvention}
\declaretheorem[name=Notation,sharenumber=definition,style=mainstyle_unshaded]{notation}
\declaretheorem[name=Bemerkung,sharenumber=definition,style=mainstyle_unshaded,refname=Bemerkung]{bemerkung}
\declaretheorem[name=Bemerkung,numbered=no,style=mainstyle_unnumbered]{bemerkung*}
\declaretheorem[name=Beispiel,sharenumber=definition,style=mainstyle_unshaded,refname=Beispiel]{beispiel}
\declaretheorem[name=Beispiel,numbered=no,style=mainstyle_unnumbered]{beispiel*}
\declaretheorem[name=Exkurs,numbered=no,style=mainstyle_unnumbered]{exkurs*}
\declaretheorem[name=Beweis,numbered=no,style=beweise]{beweis}
\declaretheorem[name=Übung,numbered=no,style=bemerkungen]{uebung}
\declaretheorem[name=Erinnerung,numbered=no,style=bemerkungen]{erinnerung}

% english versions
\declaretheorem[name=Remark,sharenumber=definition,style=mainstyle_unshaded]{remark}
\declaretheorem[name=Remark,numbered=no,style=mainstyle_unnumbered]{remark*}
\declaretheorem[name=Example,sharenumber=definition,style=mainstyle_unshaded]{example}
\declaretheorem[name=Corollary,sharenumber=definition,style=mainstyle]{corollary}
\let\proof\relax
\declaretheorem[name=Proof,numbered=no,style=beweise]{proof}
\declaretheorem[name=Sketch of Proof,numbered=no,style=beweisskizze]{sketch}
% ======================================================================================

%--Inhaltsverzeichnis
% ======================================================================================
\usepackage[tocindentauto]{tocstyle}
\usetocstyle{KOMAlike}
% ======================================================================================

%-- Dinge, die erst am Ende getan werden dürfen
% ======================================================================================
\shorthandon{"}
\usepackage{ellipsis}
% ======================================================================================


\newcommand{\fach}{Grundlagen der Analysis, Topologie, Geometrie}
\newcommand{\shortFach}{Analysis, Topologie, Geometrie}
\newcommand{\semester}{SoSe 2014}
\newcommand{\homepage}{https://wwwmath.uni-muenster.de/reine/u/topos/lehre/SS2014/AnaTopGeo/anatopgeo.html}

\newcommand{\prof}{Prof.\ Dr.\ Arthur Bartels}
\publishers{\scalebox{11}{\Huge$\pi_1$}}
\input{../!config/mitschrift_headings.tex}

\setlist{itemsep=1pt}

\begin{document}
\pagenumbering{Roman}
\maketitle
\begin{abstract}
\section*{Aktuelle Version verfügbar bei}
\newcommand{\dieBreite}{11cm}
\begin{minipage}{4cm}
	\qrcode[height=3.3cm, version=6]{https://gitlab.com/JaMeZ-B/LaTeX-WWU}
\end{minipage}
\hfill
\begin{minipage}{\dieBreite}
	% \includegraphics[height=0.6cm, keepaspectratio]{../!config/Bilder/wm_no_bg.pdf}
	\includegraphics[height=0.8cm, keepaspectratio]{../!config/Bilder/wm_no_bg.pdf}\\
	\url{https://gitlab.com/JaMeZ-B/LaTeX-WWU} \smallskip\\
	Das zentrale Repository des \enquote{\LaTeX-WWU}-Projekts befindet sich auf der Plattform GitLab.com.
	Neben der Koordination aller Beteiligten werden über diesen Dienst auch die PDFs gebaut, die in der Readme verlinkt sind.
\end{minipage}\\[1cm]
\begin{minipage}{4cm}
	\qrcode[height=3.3cm, version=6]{https://github.com/JaMeZ-B/latex-wwu}
\end{minipage}
\hfill
\begin{minipage}{\dieBreite}
	\includegraphics[height=0.6cm, keepaspectratio]{../!config/Bilder/github_octo.pdf}
	\includegraphics[height=0.6cm, keepaspectratio]{../!config/Bilder/GitHub_Logo.pdf}\\
	\url{https://github.com/JaMeZ-B/latex-wwu} \smallskip\\
	Die Entwicklung des \enquote{\LaTeX-WWU}-Projekts hat ursprünglich auf GitHub stattgefunden, ist mittlerweile aber zu GitLab gewechselt.
	Das GitHub-Repository wird stündlich automatisch aktualisiert, Merge-Requests werden aber nicht mehr entgegengenommen.
\end{minipage}\\[1cm]
% \begin{minipage}{4cm}
% 	\qrcode[height=3.3cm, version=6]{https://uni-muenster.sciebo.de/public.php?service=files&t=965ae79080a473eb5b6d927d7d8b0462}
% \end{minipage}
% \hfill
% \begin{minipage}{\dieBreite}
% 	\raisebox{-2pt}{\includegraphics[height=0.6cm, keepaspectratio]{../!config/Bilder/sciebo_logo.pdf}}
% 	\resizebox{!}{0.5cm}{\large \sffamily\textbf{sciebo}} {\sffamily\large die Campuscloud} \\
% 	\resizebox{\dieBreite}{!}{\footnotesize\url{https://uni-muenster.sciebo.de/public.php?service=files&t=965ae79080a473eb5b6d927d7d8b0462}}\smallskip\\
% 	Sciebo ist ein Dropbox-Ersatz der Hochschulen in NRW, der von der Uni Münster in leitender Position auf Basis der OpenSource-Software Owncloud aufgebaut wurde.
% \end{minipage}\\[1cm]
\hrule \mbox{ }\\[0.7cm]
\begin{minipage}{4cm}
	\qrcode[height=3.3cm, version=6]{\homepage}
\end{minipage}
\hfill
\begin{minipage}{\dieBreite}
	\resizebox{!}{0.5cm}{\large\sffamily\textbf{Vorlesungshomepage}}\\
	\resizebox{\dieBreite}{!}{\footnotesize\url{\homepage}}\smallskip\\
	Hier ist ein Link zur offiziellen Vorlesungshomepage.
\end{minipage}
\newpage
\section*{Vorwort --- Mitarbeit am Skript}
Dieses Dokument ist eine Mitschrift aus der Vorlesung \enquote{\fach, \semester}, gelesen von \prof. 
Der Inhalt entspricht weitestgehend dem Tafelanschrieb. 
Für die Korrektheit des Inhalts übernehme ich keinerlei Garantie! 
Für Bemerkungen und Korrekturen -- und seien es nur Rechtschreibfehler -- bin ich sehr dankbar. 
Korrekturen lassen sich prinzipiell auf drei Wegen einreichen: 
\begin{itemize}
	\item Persönliches Ansprechen in der Uni, Mails an \hrefsymmail{mailto:\mail}{\mail} (gerne auch mit annotieren PDFs) oder Kommentare auf \url{https://gitlab.com/JaMeZ-B/LaTeX-WWU}.
	\item \emph{Direktes} Mitarbeiten am Skript: Den Quellcode poste ich auf GitLab (siehe oben), also stehen vielfältige Möglichkeiten der Zusammenarbeit zur Verfügung:
	Zum Beispiel durch Kommentare am Code über die Website und die Kombination Fork und Merge-Request. 
	Wer sich verdient macht oder ein Skript zu einer Vorlesung, die ich nicht besuche, beisteuern will, dem gewähre ich gerne auch Schreibzugriff.
	
	Beachten sollte man dabei, dass dazu ein Account bei \url{gitlab.com} notwendig ist, der allerdings ohne Angabe von persönlichen Daten angelegt werden kann. 
	Wer bei GitLab (bzw. dem zugrunde liegenden Open-Source-Programm \enquote{\texttt{git}}) -- verständlicherweise -- Hilfe beim Einstieg braucht, dem helfe ich gerne weiter. 
	Es gibt aber auch zahlreiche empfehlenswerte Tutorials im Internet.\footnote{zB. \url{https://try.github.io/levels/1/challenges/1}, ist auf Englisch, aber dafür interaktiv}
	\item \emph{Indirektes} Mitarbeiten: \TeX-Dateien per Mail verschicken. 
	
	Dies ist nur dann sinnvoll, wenn man einen ganzen Abschnitt ändern möchte (zB. einen alternativen Beweis geben), da ich die Änderungen dann per Hand einbauen muss! Ich freue mich aber auch über solche Beiträge!
\end{itemize}
\end{abstract}

\tableofcontents
\cleardoubleoddemptypage

\pagenumbering{arabic}
\setcounter{page}{1}
\setcounter{footnote}{0}

\section{Topologische Räume} % (fold)
\label{sec:top_raume}

\begin{definition}[label=def:metrischer,{name=[metrischer Raum]}]
	Ein \Index{metrischer Raum} $(X,d)$ ist eine Menge $X$ mit einer Abbildung, \Index{Metrik} genannt,  $d \colon X \times X \to [0,\infty)$ mit den folgenden Eigenschaften:
	\begin{enumerate}[(i)]
		\item $\forall x,y \in X : d(x,y) = d(y,x)$,
		\item $\forall x,y \in X : d(x,y)=0 \iff x=y$ und
		\item $\forall x,y,z \in X : d(x,z) \le d(x,y) + d(y,z)$  (Dreiecksungleichung)
	\end{enumerate}
\end{definition}

\begin{definition}[{name=[Vektorraumnorm]}]
	Sei $V$ ein $\mathbb{R}$-Vektorraum. 
	Eine \Index{Norm} auf $V$ ist eine Abbildung $\norm{.} \colon V \to [0,\infty)$ mit den folgenden Eigenschaften:
	\begin{enumerate}[(i)]
		\item $\forall v \in V, \lambda  \in \mathbb{R} : \norm{\lambda  \cdot v} = \abs{\lambda}  \cdot \norm{v}$,
		\item $\forall v,w \in V : \norm{v+w} \le \norm{v} + \norm{w}$ (Dreiecksungleichung)
		\item $\forall v \in V : \norm{v} = 0 \iff v=0$
	\end{enumerate} 
	Durch $d(v,w) \coloneqq \norm{v-w} $ erhalten wir eine Metrik auf $V$ wie man sich leicht klarmacht.
\end{definition}

\begin{beispiel}[{name=[Normen auf dem euklischen Raum]}]
	Auf $\mathbb{R}^n$ gibt es verschiedene Normen und damit auch verschiedene Metriken: 
	Für $x= (x_1, \ldots ,x_n) \in \mathbb{R}^n$ definiert man
	\begin{enumerate}[(i)]
		\item $\norm{x}_2 = \sqrt{\sum_{i=1}^{n} x_i^2}$
		\item $\norm{x}_1  = \sum_{i=1}^{n} \abs{x_i} $
		\item $\norm{x}_\infty = \max \set[\big]{\abs{x_i} \given i=1, \ldots , n}$
	\end{enumerate}
\end{beispiel}

\begin{beispiel}[{name=[Metriken verschiedenster Art]}]
	Man kann Metriken auch anderweitig definieren:
	\begin{enumerate}[(i)]
		\item Auf der \Index{1-Sphäre} \(
			S^1 \coloneqq \set*{z \in \mathbb{C} \given \abs*{z} = 1} 
		\)
		wird durch $d(z,z') \coloneqq \min \set*{\abs*{\theta} \given \theta \in \mathbb{R} : z = e^{i \theta} \cdot z'} $ eine Metrik definiert.
		\item Ist $X$ ein metrischer Raum und $A$ eine Teilmenge von $X$, so wird $A$ durch die Einschränkung der Metrik auf $A$ zu einem metrischen Raum. 
		Wir sagen dann $A$ ist ein Unterraum von $X$.
		\item Sei $X$ eine beliebige Menge. Durch
		\[
			d(x,y) \coloneqq \begin{cases}
				0, &\text{ falls } x=y\\
				1, &\text{ falls } x \neq y
			\end{cases}
		\]
		wird auf $X$ eine Metrik, die \bet{diskrete Metrik}\index{Metrik!diskrete}, definiert.
		\item Sei $p$ eine Primzahl. 
		Jedes $x \neq 0 \in \mathbb{Q}$ lässt sich eindeutig schreiben als $x = \frac{a}{b} p^n$ mit $n,a,b \in \mathbb{Z}, b \neq 0$ und $a,b,p$ paarweise teilerfremd. 
		Dann heißt
		\[
			\abs*{x}_p \coloneqq p^{-n} 
		\]
		der \bet{$p$-adische Betrag}\index{p-adischer Betrag@$p$-adischer Betrag} von $x$. 
		Setzt man $\abs*{0}_p \coloneqq 0 $, so erhält man durch $d_p(x,y) \coloneqq \abs*{x-y}_p $ die $p$-adische Metrik auf $\mathbb{Q}$. 
	\end{enumerate}
\end{beispiel}

\begin{definition}[{name=[Isometrie und Stetigkeit]}]
	Seien $(X,d_X)$ und $(Y,d_Y)$ zwei metrische Räume. 
	Eine Abbildung $f \colon X \to Y$ heißt eine \Index{Isometrie}, falls für alle $x,x' \in X$
	\[
		d_Y \enbrace[\big]{f(x), f(x')} = d_X (x,x')
	\]
	gilt.
	$f$ heißt \Index{stetig}, falls für alle $x_0 \in X$ gilt:
	\[
		\forall \varepsilon >0 : \exists \delta >0 : d_X(x,x_0) < \delta  \implies d_Y \enbrace[\big]{f(x), f(x_0)} < \varepsilon 
	\]
\end{definition}

\begin{definition}[{name=[Offene Mengen in metrischen Räumen]},label=def:offen-metrisch]
	Eine Teilmenge $U$ eines metrischen Raumes $X$ heißt \Index{offen}, falls gilt 
	\[
		\forall x \in U : \exists \delta  > 0 \text{ mit } B_\delta (x) = \set[\big]{y \in X \given d(x,y) < \delta } \subseteq U   
	\]
\end{definition}

\begin{lemma}[{name=[Charakterisierung von Stetigkeit über offene Mengen]}]
	Sei $f \colon X \to Y$  eine Abbildung zwischen metrischen Räumen. Dann sind äquivalent:
	\begin{enumerate}[(i)]
		\item $f$ ist stetig
		\item Urbilder offener Mengen in $Y$ sind offen in $X$; also 
		\[
			\forall \, U \subseteq Y \text{ offen} : f^{-1}(U) \subseteq X \text{ offen}
		\]
	\end{enumerate}
\end{lemma}
\begin{beweis}
	\emph{Siehe Analysis II.}
\end{beweis}

\begin{definition}[{name=[topologischer Raum]}]
	Ein \Index{topologischer Raum} $(X, \mathcal{O})$ ist eine Menge $X$ zusammen mit einer Familie $\mathcal{O}$ von Teilmengen von $X$, sodass gilt:
	\begin{enumerate}[(i)]
		\item $\emptyset, X \in \mathcal{O}$
		\item $U,V \in \mathcal{O} \implies U \cap V \in \mathcal{O}$
		\item Ist $I$ eine Indexmenge und $U_i \in \mathcal{O}$ für $i \in I$, so gilt $\bigcup_{i \in I} U_i \in \mathcal{O}$.
	\end{enumerate} 
	$\mathcal{O}$ heißt dann eine \Index{Topologie} auf $X$. 
	$U \subseteq X$ heißt \Index{offen}, falls $U \in \mathcal{O}$. 
	$A \subseteq X$ heißt \Index{abgeschlossen}, falls $X \setminus A$ offen ist.
\end{definition}

\begin{beispiel}[{name=[topologische Räume]}]
	\begin{enumerate}[(i)]
		\item Jeder metrische Raum wird durch 
		\[
			\mathcal{O} \coloneqq \set[\big]{U \subseteq X \given U \text{ ist offen im Sinne von \cref{def:offen-metrisch}}} 
		\]
		zu einem topologischen Raum.
		\item Sei $X$ eine beliebige Menge.
		\begin{enumerate}[(a)]
			\item Die \bet{grobe Topologie}\index{Topologie!grobe} ist $\mathcal{O}_{\text{grob}} \coloneqq \set{\emptyset, X}$.
			\item Die \bet{diskrete Topologie}\index{Topologie!diskrete} ist $\mathcal{O}_{\text{diskret}} \coloneqq \mathcal{P}(X)$.
			\item Die \bet{koendliche Topologie}\index{Topologie!koendliche}  ist 
			$\mathcal{O}_{\text{koendl.}} \coloneqq \set{U \subseteq X \given X \setminus U \text{ endlich}} \cup \set{\emptyset}$.
		\end{enumerate}
	\end{enumerate}
\end{beispiel}

\begin{definition}[{name=[Stetigkeit in topologischen Räumen]}]
	Eine Abbildung $f \colon X \to Y$ zwischen topologischen Räumen heißt \Index{stetig}, wenn Urbilder von offener Mengen offen sind.
\end{definition}

\begin{lemma}[{name=[Verknüpfung stetiger Abbildungen]}]
	Seien $f \colon X \to Y$ und $g \colon Y \to Z$ stetige Abbildungen. 
	Dann ist auch $g \circ f \colon X \to Z$ stetig.
\end{lemma}
\begin{beweis}
	Sei $U \subseteq Z$ offen. 
	Dann ist $g ^{-1}(U) \subseteq Y$ offen, da $g$ stetig ist. 
	Da auch $f$ stetig ist, gilt 
	$\enbrace{g \circ f } ^{-1} (U) = f ^{-1} \enbrace[\big]{g^{-1} (U)} \subseteq X$ offen.
\end{beweis}

\begin{definition}[{name=[Homöomorphismus]}]
	Seien $X,Y$ topologische Räume. 
	Eine bijektive stetige Abbildung $f \colon X \to Y$ heißt \Index{Homöomorphismus}, falls auch ihre Umkehrabbildung  $f^{-1} \colon Y \to X$ 
	stetig ist.
	
	Gibt es einen solchen Homöomorphismus, so heißen $X$ und $Y$ \Index{homöomorph} und wir schreiben $X \cong Y$, andernfalls $X \not\cong Y$.
\end{definition}

\begin{beispiel}[{name=[Homöomorphismen]}]
	\begin{enumerate}[(i)]
		\item Es gilt $(0,1) \cong (0,\infty) \cong (-\infty,0) \cong \mathbb{R}$. 
		Diese Homöomorphismen kann man leicht explizit hinschreiben.
		\item Es gilt $(0,1)\not\cong [0,1) \not\cong [0,1] \not\cong (0,1)$. 
		Dies zeigen wir in einer Übungsaufgabe.
		\item Es gilt
		\[
			\mathbb{R}^n \cong \mathbb{R}^m \iff n=m
		\]
		Diese vermeintlich harmlose Aussage (topologische Invarianz der Dimension) ist so schwer zu beweisen, dass wir im Rahmen dieser Vorlesung nur Spezialfälle zeigen können!
	\end{enumerate}
\end{beispiel}

\begin{definition}[{name=[Basis einer Topologie]}]
	Sei $X$ ein topologischer Raum. 
	Eine Familie $\mathcal{U}$ von offenen Teilmengen von $X$ heißt eine \Index{Basis der Topologie}, falls für jede Teilmenge $W \subseteq X$
	äquivalent sind:
	\begin{enumerate}[(1)]
		\item $W$ ist offen.
		\item $\forall x \in W  : \exists U \in \mathcal{U}$ mit $x \in U \subseteq  W \iff W = \bigcup_{\substack{U \in \mathcal{U} \\ U \subseteq W}} U$ 
	\end{enumerate}
	Man sagt $X$ erfüllt das \bet{zweite Abzählbarkeitsaxiom}\index{zweites Abzählbarkeitsaxiom}, falls $X$ eine abzählbare Basis der Topologie besitzt.
\end{definition}

\begin{beispiel}[{name=[Basis der Topologie eines metrischen Raumes]}]
	Sei $X$ ein metrischer Raum. 
	Dann ist $\set*{B_\delta (x) \given x \in X, \delta > 0}$ eine Basis der Topologie von $X$ nach \cref{def:offen-metrisch}. 
	Gibt es eine abzählbare dichte Teilmenge $X_0\subseteq X$, so ist 
	\[
		\set*{B_{1/n}(x) \given x \in X_0, n \in \mathbb{N}}
	\]
	eine abzählbare Basis der Topologie von $X$ und $X$ erfüllt das zweite Abzählbarkeitsaxiom.
\end{beispiel}

\begin{proposition}[label=prop:basis-topo,{name=[Charakterisierung einer Basis der Topologie]}]
	Sei $X$ eine Menge und $\mathcal{U} $ eine Familie von Teilmengen von $X$ mit $X = \bigcup_{U \in \mathcal{U} } U$. 
	Dann ist $\mathcal{U}$ genau dann die Basis einer Topologie $\mathcal{O}$ auf $X$, wenn $\mathcal{U}$ folgende Bedingungen erfüllt:
	\begin{equation}
		\forall U,V \in \mathcal{U} : \forall x \in U \cap V : \exists W \in \mathcal{U}  \text{ mit } x \in W \subseteq U \cap V \label{eq:prop-basis} \tag{\#}
	\end{equation}
\end{proposition}
\begin{beweis}
	Sei $\mathcal{U} $ die Basis der Topologie $\mathcal{O} $ und $U,V \in \mathcal{U} $. $\Rightarrow U,V$ offen, also ist auch $U \cap V$ offen. 
	Da $\mathcal{U} $ eine Basis der Topologie ist, gibt es zu jedem $x \in U \cap V$ ein $W \in \mathcal{U} $ mit $x \in W \subseteq U \cap V$. Daher gilt \eqref{eq:prop-basis}. 
	
	Sei umgekehrt \eqref{eq:prop-basis} erfüllt. Definiere $\mathcal{O} $ durch
	\[
		W \in \mathcal{O} :\Longleftrightarrow \forall x \in W : \exists U \in \mathcal{U} : x \in U \subseteq W.  
	\]
	Dann ist $\emptyset \in \mathcal{O}$. 
	Wegen $X = \bigcup_{U \in \mathcal{U}} U$ gilt auch $X \in \mathcal{O}$. 
	Weiter ist $\mathcal{O}$ unter Vereinigungen abgeschlossen.
	Seien $W_1, W_2 \in \mathcal{O}$ und $x \in W_1 \cap W_2$. Dann gilt
	\begin{align*}
		x \in W_1, W_1 \text{ offen } &\implies  \exists U_1 \in \mathcal{U} : x \in U_1 \subseteq W_1 \\
		x \in W_2, W_2 \text{ offen } &\implies \exists U_2 \in \mathcal{U} : x \in U_2 \subseteq W_2 
	\end{align*}
	Also $x \in U_1 \cap U_2$. Mit \eqref{eq:prop-basis} folgt: $\exists W \in \mathcal{U} $ mit $x \in W \subseteq U_1 \cap U_2 \subseteq W_1 \cap W_2$.
\end{beweis}

\begin{bemerkung}[{name=[Eindeutigkeit der Topologie aus \cref{prop:basis-topo}]}]
	Die Topologie $\mathcal{O}$ in \cref{prop:basis-topo} wird eindeutig durch $\mathcal{U}$ bestimmt.
\end{bemerkung}

\begin{beispiel}[{name=[Topologien definiert durch Basen]}]
	\begin{itemize}
		\item Sei $\mathbb{R}^\mathbb{N}$ der $\mathbb{R}$-Vektorraum aller reellen Folgen. 
		Für eine Konstante $\delta >0$, eine Zahl $n \in \mathbb{N}$ und Punkte $\alpha_1, \ldots , \alpha_n \in \mathbb{R}$ sei
		\[
			U_{n, \delta , \alpha_1, \ldots , \alpha_n} \coloneqq \set[\big]{(x_i)_{i \in \mathbb{N}} \given \abs*{x_i - \alpha_i} < \delta \text{ für } i=1, \ldots ,n } 
		\]
		Dann erfüllt $\mathcal{U} \coloneqq \set*{U_{n, \delta , \alpha_1, \ldots , \alpha_n} \given n \in \mathbb{N}, \alpha_i \in \mathbb{R}, \delta >0}$ die Bedingung \eqref{eq:prop-basis} und ist die Basis der \Index{Topologie der punktweisen Konvergenz}.
		\item Sei $C(\mathbb{R},\mathbb{R})$ der $\mathbb{R}$-Vektorraum aller stetigen Abbildungen. 
		Zu $[a,b] \subset \mathbb{R}$, $\delta >0$, $g \colon [a,b] \to \mathbb{R}$ stetig sei
		\[
			U_{a,b, \delta , g} \coloneqq \set*{f \colon \mathbb{R} \to \mathbb{R} \text{ stetig} \given \forall t \in [a,b] : \abs[\big]{f(t)- g(t)}< \delta}. 
		\]
		Dann erfüllt $\mathcal{U} \coloneqq \set*{U_{a,b,\delta ,g}}$ die Bedingung \eqref{eq:prop-basis} und ist die Basis der \Index{Topologie der gleichmäßigen Konvergenz} auf kompakten Intervallen.
	\end{itemize}
\end{beispiel}

\begin{definition}[{name=[{Inneres, Abschluss und Rand einer Teilmenge}]}]
	Sei $Y$ eine Teilmenge eines topologischen Raums $X$.
	\begin{align*}
		\mathring Y &:= \set*{y \in Y \given  \exists U \subseteq X \text{ offen mit } y \in U \subseteq Y} \text{ heißt das \Index{Innere} von $Y$.} \\ 
		\overline{Y} &:= \set*{x \in X \given \forall U \subseteq X \text{ offen mit } x \in U : U \cap Y \not= \emptyset}    \text{ heißt \Index{Abschluss} von $Y$.} \\
		\partial Y &:= \overline{Y} \setminus \mathring Y \text{ heißt der \Index{Rand} von $Y$.} 
	\end{align*}
\end{definition}

\begin{bemerkung}[{name=[{Eigenschaften von Innerem, Abschluss und Rand}]}]
	Es gilt
	\begin{enumerate}[1)]
		\item $\mathring Y = X \setminus (\overline{X \setminus Y} )$, $\overline{Y} = X \setminus (X \setminus Y)^\circ$.
		\item $\mathring Y = \bigcup_{\substack{U \subseteq Y \\ U \text{ offen}}} U$ ist offen.
		\item $\overline{Y} = \bigcap_{Y \subseteq A, A \text{ abg.}} A$ ist abgeschlossen.
		\item $\partial Y = \overline{Y} \setminus \mathring Y $ ist abgeschlossen.
	\end{enumerate}
\end{bemerkung}

\begin{definition}[{name=[Umgebung]}]
	Sei $X$ ein topologischer Raum und $x \in X$. 
	$V \subseteq X$ heißt eine \Index{Umgebung} von $x$, falls es $U \subseteq X$ offen gibt mit $x \in U \subseteq V$. 
	Ist $V$ offen, so heißt $V$ eine \bet{offene Umgebung}\index{Umgebung!offene Umgebung} von $x$.
\end{definition}

\begin{definition}[{name=[Hausdorffraum]}]
	Ein topologischer Raum $X$ heißt \Index{hausdorffsch} (oder ein \Index{Hausdorffraum}), falls es zu jedem Paar $x,y \in X, x \neq y$ offene Umgebungen $U$ von $x$ und $V$ von $y$ gibt mit $U \cap V = \emptyset$.
\end{definition}

Metrische Räume sind stets hausdorffsch.
Ist $\abs*{X} \ge 2 $ so ist $(X, \mathcal{O}_{\text{grob}})$ nicht hausdorffsch.

\begin{definition}[{name=[topologische Mannigfaltigkeit]}]
	Ein Hausdorffraum $M$, der das zweite Abzählbarkeitsaxiom erfüllt, heißt eine \Index{topologische Mannigfaltigkeit} der Dimension $n$ (oder eine $n$-Mannigfaltigkeit), falls er lokal homöomorph zum $\mathbb{R}^n$ ist; d.h. $\forall x \in M$ existiert eine offene Umgebung $U$ von $x$ mit $U \cong \mathbb{R}^n$.
\end{definition}
% section topologische_raume (end)

\newpage

\section{Konstruktion topologischer Räume} % (fold)
\label{sec:konst_topo}

\begin{definition}[{name=[Teilraumtopologie]}]
	Sei $X$ ein topologischer Raum und $A \subseteq X$.\marginnote{\emph{Achtung:} $U \subseteq A$ offen $\centernot\implies U \subseteq X$ offen!}
	Die \Index{Spurtopologie}, \Index{Teilraumtopologie} oder \Index{Unterraumtopologie} auf $A$ besteht aus allen Teilmengen von $A$ der Form $A \cap U$ mit $U \subseteq X$ offen. 
	Mit dieser Topologie heißt $A$ ein \Index{Unterraum} von $X$.
\end{definition}

\begin{bemerkung}[{name=[Eigenschaften der Inklusion]}]
	Sei $i \colon A \hookrightarrow X$ die Inklusion.
	Dann ist $i$ stetig und falls $Y$ ein weiterer topologischer Raum ist und $f \colon Y \to A$ eine Abbildung, so gilt
	\[
		f \text{ stetig} \iff i \circ  f \colon Y \to X \text{ stetig}
	\]
\end{bemerkung}

\begin{definition}[{name=[Produkttopologie]}]
	Seien $X,Y$ topologische Räume. 
	Eine Basis für die \Index{Produkttopologie} auf $X \times Y$ ist 
	\[
		\mathcal{U} \coloneqq \set[\big]{U \times V \given U \subseteq X \text{ offen }, V \subseteq Y \text{ offen}}. 
	\]
\end{definition}

Dies können wir auf das Produkt beliebig vieler topologischer Räume verallgemeinern:

\begin{definition}[{name=[Produkttopologie]}]
	Seien $X_i$ für $i \in I$ topologische Räume. Die \Index{Produkttopologie} auf ihrem Produkt 
	\[
		\prod_{i \in I} X_i = \set[\big]{(x_i)_{i \in I} \given x_i \in X_i} 
	\]
	hat als Basis alle Mengen der Form $\prod_{i \in I} U_i$ mit
	\begin{enumerate}[(i)]
		\item $U_i \subseteq X_i$ ist offen
		\item Für fast alle $i$ ist $U_i = X_i$ (also für alle bis auf endlich viele $i$).
	\end{enumerate}
\end{definition}

\begin{bemerkung}[{name=[universelle Eigenschaft der Produkttopologie]}]
	Seien $p_j \colon \prod_{i \in I} X_i \to X_j$ die Projektionen auf die einzelnen Koordinaten. 
	Dann sind die $p_j$ alle stetig und die folgende universelle Eigenschaft ist erfüllt:
	
	Ist $Y$ ein weiterer topologischer Raum und $f \colon Y \to \prod_{i \in I} X_i$ eine Abbildung, so gilt:
	\[
		f \text{ stetig} \iff \forall j : f_j \coloneqq  p_j \circ f \text{ stetig} 
	\]
\end{bemerkung}

\begin{bemerkung}[{name=[Topologie des euklidischen Raumes]}]
	Die übliche Topologie auf $\mathbb{R}^n = \prod_{i =1}^n \mathbb{R}$ stimmt mit der eben definierten Produkttopologie überein.
\end{bemerkung}

\begin{beispiel}[{name=[Torus]}]
	Mit Produkten lassen sich viele interessante topologische Räumen \enquote{bauen}; setze zum Beispiel
	\[
		T^n \coloneqq \underbrace{S^1 \times \ldots \times S^1}_{n} = \prod_{i=1}^n S^1
	\]
	$T^n$ heißt der \bet{$n$-Torus}\index{Torus}. Der $n$-Torus ist eine (glatte) $n$-Mannigfaltigkeit.
	\tikzsetnextfilename{Torus}
	\begin{figure}[hbt]
		\centering{\begin{tikzpicture}[xscale=0.9, yscale=0.6]
		    \begin{axis}[hide axis, view={40}{50}]
		       \addplot3[surf,
		       colormap/cool,
		       samples=60,
		       domain=0:2*pi,y domain=0:2*pi,
		       z buffer=sort,
			   shader=faceted,]
		       ({(2+cos(deg(x)))*cos(deg(y+pi/2))}, 
		        {(2+cos(deg(x)))*sin(deg(y+pi/2))}, 
		        {sin(deg(x))});
		    \end{axis}
		\end{tikzpicture}
		\caption[Der Torus $T^2$]{Der Torus $T^2$, \hrefsym{http://tex.stackexchange.com/questions/348/how-to-draw-a-torus}{Quelle}}}
	\end{figure}
\end{beispiel}

\begin{definition}[{name=[Homotopie]}]
	Seien $X$ und $Y$ topologische Räume und $(f_t)_{t \in [0,1]}$ eine Familie von stetigen Abbildungen $f_t \colon X \to Y$. 
	Wir sagen, dass die $f_t$ stetig von $t$ abhängen, falls 
	\[
		H \colon X \times [0,1] \longrightarrow Y \text{ mit } H(x,t) = f_t(x)
	\]
	stetig bezüglich der Produkttopologie ist. 
	In diesem Fall heißen $f_0$ und $f_1$ \Index{homotop} und $H$ eine \Index{Homotopie} zwischen $f_0$ und $f_1$.
\end{definition}

Beispielsweise sind je zwei Abbildungen $f,g \colon X \to \mathbb{R}^n$ homotop; eine Homotopie wird gegeben durch $H(x,t) \coloneqq (1-t)\cdot f(x) + t \cdot g(x)$. 
Wir werden später sehen, dass die Identität $\id \colon S^1 \to S^1$ nicht homotop zu einer konstanten Abbildung ist.

\begin{definition}[{name=[Quotiententopologie]}]
	Sei $X$ ein topologischer Raum, $M$ eine Menge und $q \colon X \to M$ eine surjektive Abbildung. 
	Die offenen Mengen der \Index{Quotiententopologie} auf $M$ (bezüglich $q$) sind alle $U \subseteq M$ für die $q^{-1} (U ) \subseteq X$ offen ist.
\end{definition}

Die Quotiententopologie ist gerade so definiert, dass $q \colon X \to M$ stetig ist.
Außerdem ist wieder eine universelle Eigenschaft erfüllt, nämlich die folgende:

Ist $Y$ ein weiterer topologischer Raum und $f : M \to Y$ eine Abbildung, so gilt 
\[
	f \text{ stetig} \iff f \circ q \text{ stetig}
\]
Die Quotiententopologie wird oft wie folgt eingesetzt:
Sei $\sim$ eine Äquivalenzrelation auf dem topologischen Raum $X$. 
Dann ist die Äquivalenzklassenabbildung $q \colon X \to \sfrac{X}{\sim}$, $x \mapsto [x]_\sim$ surjektiv.
Insbesondere wird $\sfrac{X}{\sim}$ durch die Quotiententopologie zu einem topologischen Raum.

\begin{beispiel}[{name=[Räume konstruiert als Quotienten]}]
	Betrachte $X = [0,1] \times [0,1]$.
	\begin{enumerate}[(i)]
		\item Definiere $(s,t) \sim (s', t') :\Leftrightarrow (s=s' \text{ und } t=t')$ oder $(s=0, s'=1, t=t')$.
		Dann erhalten wir einen Zylinder
		\[
			\sfrac{X}{\sim} \cong \enspace
			\vcenter{\hbox{\begin{tikzpicture}[yscale=0.2,xscale=0.3]
			\draw (-2,6) -- (-2,0) arc (180:360:2cm and 0.5cm) -- (2,6) ++ (-2,0) circle (2cm and 0.5cm);
			\draw[densely dashed] (-2,0) arc (180:0:2cm and 0.5cm);
			\end{tikzpicture}}} \enspace
			 \cong \enspace \vcenter{\hbox{\begin{tikzpicture}[scale=0.7]
			 	\draw (0,0) circle[radius=0.9];
				\draw (0,0) circle[radius=1.3];
				\draw (0,0.9) -- (0,1.3);
			 \end{tikzpicture}}}
		\]
		Anschaulich haben wir zwei gegenüberliegende Seiten \enquote{zusammengeklebt}.
		\item Definieren wird stattdessen $(s,t) \sim (s',t') :\Leftrightarrow (s=s' \text{ und }t=t')$ oder $(s=0, s'=1 \text{ und } t=1-t')$. Dann erhalten wir das Möbiusband, siehe \cref{fig:moebius}.
		Hier haben wir anschaulich gesprochen $X$ verdreht und dann zusammengeklebt.
		\begin{figure}[hbt]
			\centering
			\tikzsetnextfilename{Möbiusband}
			\begin{tikzpicture}[yscale=0.5, xscale=0.9]
				\begin{axis}[
				    hide axis,
				    view={40}{40}
				]
				\addplot3 [
				    surf, shader=faceted interp,
				    point meta=x,
				    colormap/greenyellow,
				    samples=80,
				    samples y=5,
				    z buffer=sort,
				    domain=0:360,
				    y domain=-0.5:0.5
				] (
				    {(1+0.5*y*cos(x/2)))*cos(x)},
				    {(1+0.5*y*cos(x/2)))*sin(x)},
				    {0.5*y*sin(x/2)});

				\addplot3 [
				    samples=70,
				    domain=-145:180,
				    samples y=0,
				    thick
				] (
				    {cos(x)},
				    {sin(x)},
				    {0});
				\end{axis}
			\end{tikzpicture}
			\caption[Möbius-Band]{Möbius-Band, \hrefsym{http://tex.stackexchange.com/questions/118563/moebius-strip-using-tikz}{Quelle}}\label{fig:moebius}
		\end{figure}
		\item Sei $\mathbb{R}P^n$ die Menge aller $1$-dimensionalen Unterräume des $\mathbb{R}^{n+1}$. 
		Wir erhalten eine surjektive Abbildung 
		\[
			q \colon \mathbb{R}^{n+1}\setminus \set{0} \to \mathbb{R}P^n \enspace,  \quad  q(v) \coloneqq \langle v\rangle
		\]
		$\mathbb{R}P^n$ mit der Quotiententopologie bezüglich $q$ heißt der \Index{reell projektive Raum} der Dimension $n$. Er ist eine $n$-Mannigfaltigkeit.\todo{Kap 15 referenzieren}
		% \hfill (siehe auch \hyperref[154:enum:3]{\ref*{sub:154} (\ref*{154:enum:3})})
		\item Betrachte auf $\mathbb{R}$ die Relation $x \sim y :\Leftrightarrow x-y \in \mathbb{Q}$. 
		Den Raum der Äquivalenzklassen bezeichnen wir mit $\sfrac{\mathbb{R}}{\mathbb{Q}}$. 
		Dann ist $\sfrac{\mathbb{R}}{\mathbb{Q}}$ mit der Quotiententopologie nicht hausdorffsch, obwohl $\mathbb{R}$ selbstverständlich hausdorffsch ist.
		
		\emph{Übung:} Die Quotiententopologie auf $\nicefrac{\mathbb{R}}{\mathbb{Q}}$ ist die grobe Topologie.
		\item\label{210:enum:5} Sei $f \colon X \to X$ eine stetige Abbildung. 
		Betrachte auf $X \times [0,1]$ die Äquivalenzrelation 
		\[
			(x,t) \sim (x',t') : \Longleftrightarrow (x=x' \text{ und } t=t') \text{ oder } (t=0, t'=1 \text{ und } x'= f(x))
		\] 
		Der Quotient $T_f \coloneqq \sfrac{X \times [0,1]}{\sim}$ heißt der \Index{Abbildungstorus} von $f$. 
	
		Beispiel: Betrachte $f \colon [{-1,1}] \to [{-1,1}]$ gegeben durch $f(x)=-x$. 
		Dann ist $T_f$ das Möbiusband von eben.
	\end{enumerate}
\end{beispiel}



% section konst_topo (end)


\cleardoubleoddemptypage
\pagenumbering{Alph}
\setcounter{page}{1}
\cleardoubleoddemptypage
\appendix

\section{Anhang} % (fold)
\label{sec:anhang}
% %!TEX root = ana_top_geo.tex

\subsection{Ausführlicher Beweis zu \cref{lem:kpt-schnitte}} % (fold)
\label{sub:kpt-schnitte}
Sei $X$ ein Hausdorffraum. Dann ist $X$ genau dann kompakt, wenn gilt: Hat eine Familie $\mathcal{A}$ von abgeschlossenen Teilmengen von $X$ die endliche 
Durchschnittseigenschaft, so gilt 
\[
	\bigcap_{A \in \mathcal{A}} A \not= \emptyset.
\]
\begin{beweis}
	Für die erste Implikation sei $X$ kompakt und $\mathcal{A}$ eine Familie von abgeschlossenen Mengen mit der endlichen Durchschnittseigenschaft.
	Angenommen $\bigcap_{A \in \mathcal{A}} A = \emptyset$.
	Dann gilt
	\[
		X = X \setminus \bigcap_{A \in \mathcal{A}} A = \bigcup_{A \in \mathcal{A}} X \setminus A.
	\]
	Nun ist $\mathcal{U} \coloneqq \set*{X \setminus A \given A \in \mathcal{A}}$ eine offene Überdeckung von $X$ und da $X$ kompakt ist, existiert $\mathcal{A}_0 \subset \mathcal{A}$ endlich, sodass
	\[
		X = \bigcup_{A \in \mathcal{A}_0} X \setminus A = X \setminus \underbrace{\bigcap_{A \in \mathcal{A}_0 } A }_{\neq \emptyset} \quad \light
	\]
	Für die umgekehrte Implikation sei nun $\mathcal{U} = \set{U_i}_{i \in I}$ eine offene Überdeckung von $X$.
	Angenommen für jede endliche Teilmenge $J \subseteq I$ gilt $X \neq \bigcup_{i \in J} U_i$.
	Betrachte nun $\mathcal{A} =  \set{X \setminus U_i}_{i \in I}$. Dann gilt nach Annahme
	\[
		\bigcap_{i \in J} X \setminus U_i = X \setminus \bigcup_{i \in J} U_i \neq \emptyset.
	\]
	Also hat $\mathcal{A}$ die endliche Durchschnittseigenschaft. Nach Vorraussetzung gilt dann
	\[
		\emptyset \not= \bigcap_{i \in I} X \setminus U_i = X \setminus \underbrace{\bigcup_{i \in I} U_i}_{= X} \quad \light \qedhere
	\]
\end{beweis}


\subsection[Blatt3, Aufgabe 4: Hilfssatz für den Hauptsatz der Algebra]{Blatt 3, Aufgabe 4} % (fold)
\label{sub:B3A4}
\emph{Diese Übungsaufgabe ist zentral für den Beweis des Hauptsatzes der Algebra, \cref{satz:hauptsatz-algebra}.} 

Sei $p(x)= x^n + a_{n-1} x^{n-1} + \ldots + a_1 x + a_0$ mit $n \in \mathbb{N}_0$ ein Polynom mit Koeffizienten $a_i \in \mathbb{C}$, dass \emph{keine} Nullstelle in $\mathbb{C}$ besitzt. 
Sei $S^1= \set*{z \in \mathbb{C} \given \abs*{z}=1}$.
\begin{enumerate}[(a)]
	\item $f \colon S^1 \to S^1$ gegeben durch $f(z) = \frac{p(z)}{\abs*{p(z)} } $ ist wohldefiniert und homotop zu einer konstanten Abbildung.
	\item $f$ ist homotop zur Abbildung $g_n \colon S^1 \to S^1$ mit $g_n(z)= z^n$.
\end{enumerate}
\minisec{Beweis}
\begin{enumerate}[(a)]
	\item \begin{description}
		\item[Wohldefiniertheit:] Sei $z \in S^1$ beliebig. Dann gilt
		\[
			\abs*{\frac{p(z)}{\abs*{p(z)} } } = \frac{1}{\abs*{p(z)} } \cdot \abs*{p(z)} =1,
		\]
		also ist $f(z) \in S^1$.
		\item[Homotop zu einer konstanten Abbildung:] Definiere $f_t \colon S^1 \to S^1$ für $t \in [0,1]$ durch 
		\[
			f_t(z) = \frac{p(t \cdot z)}{\abs*{p(t \cdot z)} } 
		\]
		Dies ist mit der gleichen Begründung wie oben wohldefiniert. 
		Außerdem ist $f_0(z)= \frac{a_0}{\abs*{a_0} } \in S^1 $ konstant und $f_1(z)= \frac{p(z)}{\abs*{p(z)} }=f(z)$. 
		Definiere nun $H \colon S^1 \times [0,1] \to S^1$ durch $H(x,t) \coloneqq f_t(x)$. 
		Dann ist $H$ stetig, da Polynome und $\abs*{.} $, sowie Multiplikation stetig sind. 
		$H$ ist die gesuchte Homotopie.
	\end{description}
	\item Sei $h \colon S^1 \times [0,1] \to \mathbb{C}$ gegeben durch $h(z,t) = z^n + \sum_{k=0}^{n-1} a_k z^k t^{n-k}$. 
	Dann gilt $h(z,0)=z^n \not= 0$, da $z \in S^1$.
	Für $t \neq 0$ gilt nun
	\begin{align*}
		h(z,t) = 0 \iff \frac{h(z,t)}{t^n} = 0 \iff \frac{z^n}{t^n} + \sum_{k=0}^{n-1} a_k \frac{z^k}{t^k} = 0 \iff p \enbrace*{\frac{z}{t}} = 0
	\end{align*}
	Aber nach Vorraussetzung gilt $p \enbrace*{\frac{z}{t}} \neq 0$. 
	Also $h(z,t) \neq 0$ für alle $t \in [0,1]$. 
	Definiere nun $H \colon S^1 \times [0,1]\to S^1$ durch $H(z,t) = \frac{h(z,t)}{\abs*{h(z,t)}}$. 
	Wie eben gezeigt, ist dies wohldefiniert und offensichtlich stetig. Da
	\[
		H(z,0) = \frac{z^n}{\abs*{z^n} } = z^n \quad \text{ und } \quad H(z,1) = \frac{h(z,1)}{\abs*{h(z,1)} } = \frac{p(z)}{\abs*{p(z)} } =f(z)
	\]
	ist $H$ die gesuchte Homotopie. \qedhere
\end{enumerate}

\subsection{Blatt 10, Aufgabe 3} % (fold)
\label{sub:B10A3}
\emph{Diese Übungsaufgabe lieferte den Beweis zu \cref{prop:iso-covering}.} \smallskip \\
Sei $p \colon \overline{X} \to X$ eine Überlagerung. 
Seien $\overline{x}_0  \in \overline{X}$ und $x_0= p(\overline{x}_0 )$ Basispunkte. 
Dann ist die induzierte Abbildung $\pi_n (p) \colon \pi_n(\overline{X}, \overline{x}_0) \to \pi_n(X,x_0)$ ein Isomorphismus für alle $n \ge 2$.
\minisec{Beweis}
Als Überlagerung ist $p$ stetig, also ist $\pi_n(p)$ ein Gruppenhomomorphismus nach \hyperref[prop:eig-hom-gruppen:enum:4]{ \cref*{prop:eig-hom-gruppen} \ref*{prop:eig-hom-gruppen:enum:4}}.
\begin{description}
	\item[Surjektivität:] Sei $[\omega] \in \pi_n(X,x_0)$, also $\omega \colon I^n \to X$ mit $\omega(\partial I^n) = \set{x_0}$. Betrachte $\omega$ nun als Abbildung $I^{n-1} \times [0,1] \to X$:
	\[
		\begin{tikzcd}[column sep=4em]
			I^{n-1} \times \set{0} \dar[hook] \rar["\mathrm{const}_{\overline{x}_0}"] & \overline{X} \dar["p"]\\
			I^{n-1} \times I \rar["\omega"] & X  
		\end{tikzcd}
	\]
	$\mathrm{const}_{\overline{x}_0} \colon I^{n-1} \times \set{0}$ ist eine Hebung von $\omega\big|_{I^{n-1} \times \set{0}} \equiv x_0$. 
	Nach dem Homotopiehebungssatz (\ref{satz:hebung-homotopie}) existiert eine Hebung $\overline{\omega} \colon I^{n-1} \times I \to \overline{X}$ von $\omega$ mit $\overline{\omega}\big|_{I^{n-1} \times \set{0}} \equiv \overline{x}_0 $. 
	Also gilt
	\[
		p \circ \overline{\omega} \big|_{\partial I^n} = \omega \big|_{\partial I^n} \equiv x_0 \enspace \Longrightarrow \enspace \overline{\omega} \big|_{\partial I^n} 
		\in p ^{-1}( \set{x_0} ) .
	\]
	Da $p^{-1}(\set{x_0})$ diskret und $\partial I^n$ für $n \ge 2$ zusammenhängend ist, muss $\overline{\omega} \big|_{\partial I^n}$ konstant sein. 
	Da $\overline{\omega}\big|_{I^{n-1} \times \set{0}} \equiv \overline{x}_0 $ gilt, folgt somit $\overline{\omega}(\partial I^n) = \set{\overline{x}_0}$. 
	Also ist $[\overline{\omega}] \in \pi_n(\overline{X},\overline{x}_0)$ und weiter gilt
	\[
		\pi_n(p) \enbrace*{[\overline{\omega}]} = [p \circ \overline{\omega} ] = [\omega] \in \pi_n(X,x_0). 
	\]
	\item[Injektivität:] Sei $[\omega] \in \ker \pi_n(p)$, also $[p \circ \omega] = [c_{x_0}]$. 
	Es existiert also eine Homotopie $H$ relativ $\partial I^n$ zwischen $p \circ \omega$ und $c_{x_0}$. 
	Offensichtlich ist $\omega$ eine Hebung von $p \circ \omega$. 
	Mit dem Homotopiehebungssatz erhalten wir eine Hebung $\overline{H}$ von $H$ mit $\overline{H}(-,0) = \omega$. 
	Weiter wissen wir, dass
	\[
		\overline{H} \big|_{\partial I^n \times [0,1]} \in p ^{-1}(\set{x_0} ) \quad \text{ und }\quad  \overline{H} \big|_{ I^n \times \set{1}} \in p ^{-1}(\set{x_0} )
	\]
	gelten muss, da $H = p \circ \overline{H}$ und $H(-,1)= c_{x_0} \equiv x_0$. 
	Mit dem gleichen Argument wie oben folgt, dass $\overline{H} \big|_{\partial I^n \times [0,1]}$ und $\overline{H} \big|_{ I^n \times \set{1}}$ konstant sind. 
	Für $z \in \partial I^n$ gilt nun
	\[
		\overline{H}(z,0) = \omega(z) = \overline{x}_0
	\]
	Da $\partial I^n \times [0,1] \cap I^n \times \set{1} \not= \emptyset$, muss also auch $\overline{H}(-,1) \equiv \overline{x}_0$ gelten. 
	Damit folgt $[\omega] = [c_{x_0}]$.\qedhere
\end{description}
\printindex
\printbibliography
\listoffigures
% \todototoc
% \listoftodos[To-do's und andere Baustellen]
\end{document}
