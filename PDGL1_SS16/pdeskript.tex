%!TEX TS-program = xelatex
%skript für Die Vorlesung Partielle Differentialgleichungen
\newcommand{\Semester}{SoSe 2016}
\newcommand{\fach}{Partielle Differentialgleichungen}
\newcommand{\prof}{Prof.\ Zeppieri}

\input{../!config/VorlagenTim/preambel.tex}

\numberwithin{equation}{section}
\numberwithin{figure}{section}

\begin{document}

\maketitle
\cleardoubleoddemptypage

\pagenumbering{Alph}
\section*{Vorwort --- Mitarbeit am Skript}
Dieses Dokument ist eine Mitschrift aus der Vorlesung \enquote{\fach, \Semester}, gelesen von \prof. 
Der Inhalt entspricht weitestgehend dem Tafelanschrieb. 
Für die Korrektheit des Inhalts werde ich keinerlei Garantie übernehmen. Dieses Skript wird allerdings zusätzlich von \prof \, Korrektur gelesen. 
Für Bemerkungen und Korrekturen -- und seien es nur Rechtschreibfehler -- bin ich sehr dankbar. 
Bitte durch persönliches Ansprechen oder per Mail an \href{mailto:keil.menden@web.de}{keil.menden@web.de}.  

\newpage

\tableofcontents
\cleardoubleoddemptypage
\pagenumbering{arabic}
\setcounter{page}{1}

\section{Einleitung} 
\label{sec:einleitung}
\begin{definition}[PDGL]
	Sei $\Omega$ eine offene Teilmenge des $\mathbb{R}^n$ mit $n \geq 2$ und $ k \in \mathbb{N}$. 
	Eine partielle Differentialgleichung (PDGL) ist eine Gleichung der Form
	\begin{equation}\label{(1)}
		E(x,u(x),Du(x),\dots,D^ku(x))=0 \qquad \text{für alle }x \in \Omega,
	\end{equation}
	für eine unbekannte Funktion $u$ (und ihre Ableitungen $Du,\dots,D^ku$ bis zur Ordnung $k$), wobei $E: \Omega \times \mathbb{R} \times \mathbb{R}^n \times \dots \times \mathbb{R}^{n^k} \to \mathbb{R}$ ein gegebene Funktion ist. 
	Die höchste Ableitungsordnung, die in \eqref{(1)} auftritt, nennt man die Ordnung der PDGL.
\end{definition}

\begin{notation}
\begin{itemize}
	\item $D = (\partial_1, \dots , \partial_n)$, $\partial_i = \partial_{x_i} = \diff{}{x_i}$ für $i=1,\dots,n$.
	\item $D^k:= \set[D^{\alpha }]{\abs{\alpha }=k}$.
	\item $\alpha \in \mathbb{N}_0^n$ ist ein Multiindex mit $\alpha = (\alpha_1, \dots, \alpha_n)$ und
	\[
		\abs{\alpha} = \sum^{n}_{i=1}\alpha_i \qquad \qquad D^{\alpha}= \prod\limits_{i=1}^{n}\partial_i^{\alpha_i} 
		= \partial_1^{\alpha_1} \cdots \partial_n^{\alpha_n}.
	\]
\end{itemize}
\end{notation}

\begin{bemerkung}
Für $n=1$ wird \eqref{(1)} auf eine gewöhnliche Differentialgleichung (ODE) reduziert: $u: \Omega \subseteq \mathbb{R} \to \mathbb{R}$ mit
\[
	E(x,u(x),u'(x),\dots,u^{(k)}(x))=0 \qquad \text{in } \Omega.
\]
\end{bemerkung}

\begin{definition}[System von PDE's]
	Ein System von PDE's ist eine Gleichung der Form \eqref{(1)}, wenn die unbekannte Funktion $u$ und die gegebene Funktion $E$ vektorwertig sind, d.h. $u: \Omega \to \mathbb{R}^m$ für ein $m \geq 1$, $u = (u_1, \dots, u_m)$ und
	\[
		E: \Omega \times \mathbb{R}^m \times \dots \times \mathbb{R}^{m \times n^k} \to \mathbb{R}^m.
	\]
\end{definition}
\begin{definition}[Klassifikation für PDE's]
		Die PDE \eqref{(1)} heißt:
		\begin{enumerate}[(i)]
			\item LINEAR, falls sie in $u$ und allen ihren Ableitungen linear ist und die dazugehörigen Koeffizienten nur von $x$ abhängen,
			d.h., falls man sie in der Form
			\begin{equation}
				\sum_{\abs{\alpha}\leq k}^{}a_{\alpha}(x) D^{\alpha}u(x) - f(x) = 0 \qquad \text{in } \Omega \label{(2)}
			\end{equation} 
			schreiben kann, wobei $a_{\alpha}$ ($\abs{\alpha}\leq k$) und $f$ gegebene Funktionen sind.
			Im Fall $f=0$ nennt man \eqref{(2)} homogen, sonst inhomogen.
			\item SEMILINEAR, falls sie in der höchsten Ableitung $D^ku$ linear ist und die dazugehörigen Koeffizienten nur von $x$ abhängen,
			d.h. falls man sie in der Form
			\begin{equation}
				\sum_{\abs{\alpha}=k}^{}a_{\alpha}(x)D^{\alpha}u(x) + E_0(x,u(x),Du(x),\dots,D^{k-1}u(x))=0 \qquad \text{in }\Omega
			\end{equation}
			schreiben kann, wobei $a_{\alpha}(x)$ ($\abs{\alpha}=k$) und $E_0$ gegebene Funktionen sind.
			\item QUASILINEAR, falls sie linear in der höchsten Ableitung $D^ku$ ist und die dazugehörigen Koeffizienten nur von $x$, $u$, 
			und $D^lu$ mit $\abs{l}\leq k-1$ abhängen, d.h. falls man sie in der Form
			\begin{small}
			\begin{equation}
				\sum_{\abs{\alpha}=k}^{}a_{\alpha}(x,u(x),Du(x),\dots,D^{k-1}u(x))D^{\alpha}u(x) + E_0(x,u(x),Du(x),\dots,D^{k-1}u(x)) = 0
			\end{equation}
			\end{small}
			schreiben kann, wobei $a_{\alpha}$ und $E_0$ gegebene Funktionen sind.
			\item VOLL NICHTLINEAR in allen anderen Fällen.
			
		\end{enumerate}
\end{definition}

\begin{bemerkung}
	\[
		\set{\text{lineare PDE}} \subseteq \set{\text{semilineare PDE}} \subseteq \set{\text{quasilineare PDE}} \subseteq \set{\text{PDE}}.
	\]
\end{bemerkung}

\begin{definition}[Typeinteilung für PDE's zweiter Ordnung]
		Sei $\Omega$ eine offene Teilmenge des $\mathbb{R}^n$ und $E: \Omega \times \mathbb{R} \times \mathbb{R}^n \times \mathbb{R}^{n \times n} \to \mathbb{R}$ eine gegebene Funktion, so dass $p \mapsto E(x,u,z,p)$ für alle $(x,u,z) \in \Omega \times \mathbb{R}\times \mathbb{R}^n$ von der Klasse $C^1(\mathbb{R}^{n \times n})$ ist. Die PDE
		\begin{equation}
			E(x,u(x),Du(x),D^2u(x))=0 \qquad \text{in }\Omega 
		\end{equation}
		heißt
		\begin{enumerate}[(i)]
			\item ELLIPTISCH, falls die $(n \times n)$-Matrix 
			\begin{equation}
				\left( \diff{E}{p_{ij}}(x,u,z,p) \right)_{i,j=1,\dots,n} \label{(3)}
			\end{equation}
			für alle $(x,u,z,p) \in \Omega \times \mathbb{R} \times \mathbb{R}^n \times \mathbb{R}^{n \times n}$ positiv definit ist.
			\item HYPERBOLISCH, falls die Matrix in \eqref{(3)} genau einen negativen und $(n-1)$ positive Eigenwerte besitzt. (auch anders herum möglich)
			\item PARABOLISCH, falls man sie in der Form
			\begin{equation}
				D_1u(x)= \bar{E}(x,u(x),D'u(x),(D')^2u(x)) \qquad \text{in } \Omega
			\end{equation}
			%%%% unvollständig
			schreiben kann, wobei $D'=(\partial_1,\dots,\partial_n)$ und $\bar{E}: \Omega \times \mathbb{R} \times \mathbb{R}^{n-1} \times \mathbb{R}^{(n-1) \times (n-1)} \to \mathbb{R}$
			von der Klasse $C^1$ ist und 
			\begin{equation}
				\left( \diff{\bar{E}}{p_{ij}}(x,u,z',p') \right)_{i,j=1,\dots,n}.
			\end{equation}
		\end{enumerate}
\end{definition}

\begin{beispiel}[Typeinteilung für lineare PDE's zweiter Ordnung]
Sei $\Omega$ eine offene Teilmenge des $\mathbb{R}^n$ und seien $a_{ij}$, $b_i$ (für $i,j=1,\dots,n$), $c$ und $f$ skalarwertige Funktionen auf $\Omega$.
\begin{enumerate}
	\item Die lineare PDE der Form
	\[
		\sum^{n}_{i,j=1} a_{ij}(x) D_{ij}u(x) + \sum^{n}_{i=1}b_i(x) D_iu(x) + c(x)u(x) - f(x) = 0 \qquad \text{in } \Omega
	\]
	ist elliptisch, falls die Matrix $\left(a_{ij}(x)\right)_{i,j=1,\dots,n}$ für alle $x \in \Omega$ positiv definit ist.
	\item Die lineare PDE der Form 
	\[
		D_11u(x) - \sum^{n}_{i,j=1} a_{ij}(x)D_{ij}u(x) + \sum^{n}_{i=1}b_i(x) D_iu(x) + c(x)u(x) - f(x) = 0 \qquad \text{in }\Omega
	\]
	ist hyperbolisch falls die Matrix $\left( a_{ij}(x) \right)_{i,j=2,\dots,n}$ für alle $x \in \Omega$ positiv definit ist.
	\item Die lineare PDE der Form 
	\begin{equation}
		D_1u(x) - \sum^{n}_{i,j=2}a_{ij}(x)D_{ij}u(x) + \sum^{n}_{i=2}b_i(x)D_i(x) + c(x)u(x)- f(x)= 0 \qquad \text{in } \Omega
	\end{equation}
	ist parabolisch, falls die Matrix $\left( a_{ij}(x) \right)_{i,j=2,\dots,n}$ für alle $x \in \Omega$ positiv definit ist.
 \end{enumerate}
\end{beispiel}

\begin{definition}[klassische Lösung]
	Eine Funktion $u: \Omega \to  \mathbb{R}$ heißt klassische Lösung der PDE der Form 
	\begin{equation}
		E(x,u(x),Du(x),\dots,D^ku(x))=0 \qquad  \text{in } \Omega, \label{(4)}
	\end{equation}
	falls $u \in  C^k(\Omega)$ gilt und die Gleichung \eqref{(4)} überall in $\Omega$ erfüllt ist.
\end{definition}

\begin{beispiel}
	$D_1D_2u(x)=0$ in $\mathbb{R}^2$, $u(x_1,x_2)=v_1(x_1)+v_2(x_2)$ für beliebige Funktionen $v_1,v_2 \in C^2(\mathbb{R})$.
\end{beispiel}  

\begin{nb} 
\[
	B(x,u(x),Du(x),\dots,Du^{k-1}(x))=0 \qquad  \text{auf }\partial \Omega,
\] wobei $B: \partial \Omega \times \mathbb{R} \times \mathbb{R}^n \times \dots \times \mathbb{R}^{n^{k-1}} \to \mathbb{R}$ eine vorgegebene Funktion ist.
\end{nb}

\begin{bemerkung}
	Ein Problem ( = PDE + Nebenbedinung) heißt wohlgestellt (well-posed) im Sinne von Hadamerd, wenn zu gegebenen Daten eine eindeutige Lösung existiert und sie stetig von den Daten abhängt.
	Wir stellen uns in der Vorlesung folgende Fragen:
	\begin{itemize}
		\item Existenz und Eindeutigkeit von Lösungen?
		\item Hängen die Lösungen stetig von den Daten ab?
		\item Wie können Lösungen dargestellt werden?
		\item Wie ist das qualitative Verhalten von Lösungen?
	\end{itemize}
\end{bemerkung}

\begin{beispiele}
	\begin{enumerate}[1.]
		\item Laplacegleichung (linear, elliptisch):
		\begin{equation}
			\Delta u = 0 \qquad \text{in }\Omega
		\end{equation}
		modelliert das elektrische Feld im Vakuum. Inhomogen ist dies die Poissongleichung und es gilt $\Delta u = \diver(Du)$.
		\item Monge-Ampère-Gleichung (voll nichtlinear):
		\begin{equation}
			\det(D^2u)=f(x,u,Du) \qquad \text{in }\Omega
		\end{equation}
		für Transportprobleme und Differentialgeometrie.
		\item Wärmeleitungsgleichung (linear, parabolisch):
		\begin{equation}
			\partial_t u - \Delta u = 0 \qquad \text{in } \mathbb{R}^+ \times \Omega,
		\end{equation}
		wobei $\Omega \subseteq  \mathbb{R}^n$, $t$= Zeitkoordinate, $x$= Raumkoordinate modelliert die Verteilung von Wärme. $u(t,x)$ die Temperatur im Punkt $x \in \Omega$ zum Zeitpunkt $t \in \mathbb{R}^+$.
		\item Reaktions-Diffusions-Gleichung (semilinear, parabolisch):
		\begin{equation}
			\partial_t u - \Delta u = f(x,t,u) \qquad \text{in } \mathbb{R}^+ \times \Omega.
		\end{equation}
		\item System der Navier-Stokes-Gleichungen (seminlinear, parabolisch):
		\begin{equation}
			\partial_t u - \Delta u + u \cdot Du = -Dp \qquad \text{in } \mathbb{R}^+ \times \Omega,
		\end{equation}
		wobei $u: \mathbb{R}^+ \times \Omega \to  \mathbb{R}^n$, $p: \mathbb{R}^+ \times \Omega \to \mathbb{R}$ und $\Omega \subseteq  \mathbb{R}^n$. Modelliert die Strömung von inkompressiblen Flüssigkeiten, $u$ = Geschwindigkeit, $p$ = Druck.
		\item Transportgleichung (linear):
		\begin{equation}
			\partial_t u + b(t,x) \cdot Du = 0 \qquad \text{in } \mathbb{R}^+ \times \Omega.
		\end{equation}
		\item Wellengleichung (linear, hyperbolisch):
		\begin{equation}
			\partial_{tt}u - \Delta u = 0 \qquad \text{in } \mathbb{R}^+ \times \Omega.
		\end{equation}
		Für $n=2$ modelliert sie die Schwingung einer elastischen Membran. Für $n=3$ modelliert sie die Ausbreitung von Wellen (Licht und Wasser).
	\end{enumerate}
\end{beispiele}

\section{Laplacegleichung} 
\label{sec:laplacegleichung}

\begin{equation}
	\Delta u = 0 \qquad \text{in }\Omega,
\end{equation}
wobei $\Omega \subseteq \mathbb{R}^n$, offen, $n \geq 2$ und $\Delta u = \sum^{n}_{i=1} \diff{^2u}{x_i^2} = \sum^{n}_{i=1} a_{ij} \diff{^2u}{x_i \partial x_j},$ wobei $a_{ij}= I_n \in \mathbb{R}^{n \times n}$.

\begin{definition}[Harmonische Funktion]
	Sei $\Omega$ eine offene Teilmenge des $\mathbb{R}^n$ mit $n \geq 2$ und $u \in C^2(\Omega)$. 
	Man nennt $u$ \underline{harmonisch}, falls $\Delta u = 0$ in $\Omega$ gilt. Falls lediglich $\Delta u \geq 0$ in $\Omega$ gilt, nennt man $u$ subharmonisch und falls $\Delta u \leq 0$ in $\Omega$ gilt, superharmonisch.
\end{definition}
	
\begin{beispiele}
	\begin{enumerate}[1.]
		\item Jede affine Funktion ($u(x) = b \cdot x + x, b \in \mathbb{R}^n , c \in \mathbb{R}$) ist harmnoisch in $\mathbb{R}^n$.
		\item Sei $A \in \mathbb{R}^{n \times n}$. Die Matrix $u(x) = Ax \cdot x$ ist genau dann harmonisch / subharmonich / superharmonisch, falls $ \Tr(A)= 0$ / $ \Tr(A) \geq  0 $ / $ \Tr(A) \geq  0 $ gilt. (weil in diesem Fall $\Delta u = \Tr(A)$).
		\item Die Funktionen $u(x_1,x_2)= e^{ax_1}\sin(ax_2)$, $u(x_1,x_2)= e^{ax_1} \cos(ax_2)$ für alle $a \in \mathbb{R}$ sind harmonisch in $\mathbb{R}^2$.
		\begin{align}
			\diff{^2u}{x_1^2} = a^2 e^{ax_1} \sin(ax_2) \qquad \text{,} \qquad \diff{u}{x_2} &= a e^{ax_1} \cos(ax_2) 
			\qquad \text{,} \qquad \diff{^2u}{x_2^2} = - a^2 e^{ax_1} \sin(ax_2), \\
			\Rightarrow \Delta u &= 0 \qquad \text{in } \mathbb{R}^2.
		\end{align}
		\item Real- und Imaginärteil holomorpher Funktionen sind harmonisch. Sei dazu $\Omega \in \mathbb{C}$ offen und $f: \Omega \to \mathbb{C}$ eine holomorphe Funktion.
		\begin{equation}
			f ( x_1 + i x_2) = u(x_1,x_2) + i v(x_1,x_2), 
		\end{equation}
		wobei $u,v : \Omega \to  \mathbb{R}$ gilt. Da $f$ holomorph ist, erfüllen $u$ und $v$ die Cauchy-Riemann Differentialgleichungen:
		\begin{align}
			(*)
			\begin{cases}
				\diff{u}{x_1}= \diff{v}{x_2}, \\
				\diff{u}{x_2}= -\diff{v}{x_1}	\label{(stern)}
			\end{cases}
		\end{align}
		und sind beliebig oft differenzierbar in $\Omega$.
		\begin{equation}
			\Delta u = \diff{^2u}{x_1^2} + \diff{^2u}{x_2^2} \stackrel{(*)}{=} \diff{}{x_1} \left( \diff{v}{x_2} \right) + \diff{}{x_2} \left( - \diff{v}{x_1} \right) = 0
		\end{equation}
		und ebenso
		\[
			\Delta v = 0 \qquad \text{in } \Omega.
		\]
	\end{enumerate}
\end{beispiele}

\subsection{Die Fundamentallösung der Laplacegleichung} 
\label{sub:die_fundamentallosung_der_laplacegleichung}
Wir betrachten die Laplacegleichung in $\mathbb{R}^n$ mit ($n \geq 2$) und suchen eine Rotationssymmetrische Lösung, d.h. $u(x) = v(r)$, wobei $r := \abs{x}= \sqrt{x_1^2 + \dots x_n^2}$ und $v : \mathbb{R}^+ \to \mathbb{R}$. Es gilt
\begin{equation}
	\diff{r}{x_i} = \frac{2x_i}{2\sqrt{x_1^2 + \dots x_n^2} } = \frac{x_i}{r}
\end{equation}
für alle $i=1,\dots,n$ und für $x \neq 0$.
\begin{align}
	\diff{u}{x_i} = \diff{v(r)}{x_i} &= v'(r) \diff{r}{x_i} = v'(r) \frac{x_i}{r}, \\
	\diff{^2u}{x_i^2} = \diff{}{x_i} \left( v'(r) \frac{x_i}{r} \right) = v''(r) \frac{x_i^2}{r^2} + &v'(r) \left( r - \frac{x_i^2}{r} \right) \frac{1}{r^2} = v''(r) \frac{x_i^2}{r^2} + v'(r) \left( \frac{1}{r} - \frac{x_i^2}{r^2} \right)
\end{align}
Dann gilt

\begin{align*}
	\Delta u = \sum^{n}_{i=1} \diff{^2u}{x_i^2} &= \sum^{n}_{i=1} \left( v''(r) \frac{x_i^2}{r^2} + v'(r) \left( \frac{1}{r} - \frac{x_i^2}{r^2} \right) \right) \\
	&= v''(r) + v'(r) \left( \frac{n-1}{r} \right)
\end{align*}
Aus $\Delta u = 0$ in $\mathbb{R}^n$ erhalten wir eine ODE:
\begin{equation}
	v''(r) + v'(r)\left( \frac{n-1}{r} \right) = 0  \label{(5)}
\end{equation}
Wenn $v' \neq 0$, kann man \eqref{(5)} für $v'$ lösen
\begin{align}
	\frac{v''(r)}{v'(r)} &= \frac{1-n}{r} \\
	\diffd{}{r}\left( \lg(\abs{v'(r)}) \right) &= \diffd{}{r} \left( (1-n) \lg(r) \right)
\end{align}
Integration liefert 
\begin{equation}
	\lg \abs{v'(r)} = \lg( r ^{1-n}) + \alpha
\end{equation}
mit Integrationskonstante $\alpha$.
\begin{align*}
	&\hphantom{\Rightarrow} e^{\lg \abs{v'(r)}} = e^{\lg(r^{1-n})+2}, \\
	&\Rightarrow \abs{v'(r)} = e^{\alpha} r^{1-n}, \\
	&\Rightarrow v'(r) = \beta r^{1-n},
\end{align*}
für eine beliebige Konstante $\beta \in \mathbb{R}$. Es gilt
\begin{equation}
	v'(r) = \begin{cases}
		\frac{\beta}{r}, &\text{ falls } n = 2,\\
		\beta r^{1-n}, &\text{ falls } n \geq 3		
	\end{cases}
\end{equation}
und daraus folgt
\begin{equation}
	v'(r) = \begin{cases}
		\beta \lg(r)+ \gamma, &\text{ falls } n = 2,\\
		\beta r^{2-n} + \gamma , &\text{ falls } n \geq 3,		
	\end{cases}
\end{equation}
wobei $\beta,\gamma \in \mathbb{R}$. \\
Wir setzen $ \gamma = 0$ (für $n \geq 3$ gilt $v(r) \to 0$ für $r \to +\infty$). Um $\beta$ zu bestimmen, verwendet man 
\begin{equation}
	-1 = \int_{\partial B_k(0)}^{} \Delta u \cdot \nu \,\mathrm{d}S. \label{(6)}
\end{equation}
$\nu$ ist der äußere Einheitsnormalenvektor an $\partial B_r(0)$ und das Integral ist ein Oberflächenintegral. ( $ \Delta = D$). \\
\[
	\nu = \frac{x}{\abs{x}} \qquad \text{,} \qquad \Delta u = \nu'(\abs{x}) \frac{x}{\abs{x}}.
\]
Aus \eqref{(6)} folgt 
\begin{equation}
	-1 = \int\limits_{\substack{\partial B_r(0)\\ \abs{x}=r}}^{} v'(\abs{x}) \underset{=1}{\underbrace{\frac{x}{\abs{x}} \cdot \frac{x}{\abs{x}}}} \,\mathrm{d}S  
	= v'(r) \underset{= r^{n-1}S_n}{\underbrace{\int_{\partial B_r(0)}^{} \,\mathrm{d}S }},
\end{equation}
wobei $S_n$ das $(n-1)$- dimensionale Volumen der Einheitsphäre bezeichnet. [siehe Übung]

\begin{equation}
	-1 = v'(r) r^{n-1} S_n = \begin{cases}
		\frac{\beta}{r}r S_2 = \beta S_2 = \beta 2 \pi, &\text{ falls } n=2,\\
		\beta (2-n) \frac{r^{1-n}r^{n-1} }{  S_n}, &\text{ falls } n \geq 3.
	\end{cases}
\end{equation}
Für $n=2$ folgt
\[
	\beta = - \frac{1}{2 \pi}
\] 
und sonst folgt
\[
	\beta = -\frac{1}{(2-n)S_n} = \frac{1}{(n-2)n\omega_n}.
\]
Hierbei ist $\omega_n$ das $n$-dimensionale Volumen der Einheitskugel $B_1(0)$. [Übung]

\begin{align*}
	v(r) = \begin{cases}
		-\frac{1}{2 \pi} \lg(r), &\text{ falls }n=2,\\
		\frac{1}{n (n-2) \omega_n } r^{2-n}, &\text{ falls }n \geq 3.
	\end{cases}
\end{align*}

\begin{definition}[Fundamentallösung der Laplacegleichung]
	Die Funktion $\Phi: \mathbb{R}^n \setminus \set{0} \to \mathbb{R}$, definiert als
	\begin{equation}
		\Phi(x) := \begin{cases}
			-\frac{1}{2 \pi} \lg \abs{x}, &\text{ falls }n=2\\
			\frac{1}{n (n-2) \omega_n } \abs{x}^{2-n}, &\text{ falls }n \geq 3
		\end{cases}
	\end{equation}
	heißt die Fundamentallösung der Laplacegleichung.
\end{definition}

\begin{bemerkung}
	$\Phi$ hat bei $x=0$ eine Singularität aber $\Phi , \abs{ \nabla \Phi} \in L^1_{\text{loc}}(\mathbb{R}^n)$ und $ \abs{ \nabla^2 \Phi} \not\in L^1_{\text{loc}}(\mathbb{R}^n) $. Tatsächlich ist im Fall $n=2$ für $\abs{x} < 1$
	\[
		\abs{ \Phi (x)} = \abs{ - \frac{1}{2 \pi} lg(x)} = - \frac{1}{2 \pi} lg \abs{x} = \frac{1}{2 \pi} lg \frac{1}{\abs{x}} \leq  \frac{C}{\abs{x}} \in L^1(B_1(0)) 
	\]
	weil $ \frac{1}{\abs{x}^2} \in L^1(B_1(0)) \Leftrightarrow a < n$ . \\
	Für $n \geq 3$ gilt
	\begin{align*}
		\abs{ \Phi} &\leq \frac{C}{\abs{x}^{n-2}} \in L^1(B_1(0)), \\
		\abs{  \nabla  \Phi} &\leq \frac{C}{\abs{x}^{n-1}} \in  L^1 ( B_1(0)), 
	\end{align*}
	dagegen 
	\begin{equation}
		\abs{  \nabla^2 \Phi } \simeq \frac{1}{\abs{x}^n} \not\in L^1(B_1(0)).
	\end{equation}
\end{bemerkung}

\subsection{Mittelwerteigenschaft} 
\label{sub:mittelwerteigenschaft}
\begin{equation}
	u(x_0) = \fint_{\partial B_r(x_0)}^{} u \,\mathrm{d}S \qquad  \text{,} \qquad u(x_0) = \fint_{B_r(x_0)}^{} u \,\mathrm{d}x
\end{equation}
für $u$ harmonisch, wobei 
\begin{align*}
	\fint_{\partial B_r(x_0)}^{} u \,\mathrm{d}S &= \frac{1}{r^{n-1}S_n} \int_{\partial B_r(x_0)}^{}  u\,\mathrm{d}S, \\
	\fint_{B_r(x_0)}^{} u \,\mathrm{d}x &= \frac{1}{r^n \omega_n} \int_{B_r(x_0)}^{}u \,\mathrm{d}x.
\end{align*}

\begin{definition*}[$C^1$-Rand]
	Sei $\Omega \in \mathbb{R}^n$ offen. Wir sagen, dass $\Omega$ einen $C^1$-Rand hat, falls gilt: \\
	Für alle $p \in \partial \Omega$ gibt es eine offene Menge $U \subseteq \mathbb{R}^n$ und eine Funktion $f \in C^1(U)$ so dass $ \nabla f(x) \neq 0$ für alle $x \in U$ gilt und
	\begin{align*}
		\partial \Omega \cap U &= \set[x \in U]{f(x)=0}, \\
		\Omega \cap U &= \set[x \in U]{f(x) < 0}.
	\end{align*}
\end{definition*}

\begin{bemerkung}
	$\partial \Omega$ ist eine $(n-1)$-dimensionale $C^1$-Mannigfaltigkeit und $\partial \Omega$ liegt lokal nur auf einer Seite von $\Omega$.
\end{bemerkung}

\begin{lemma}
	Sei $\Omega$ eine offene Teilmenge in $\mathbb{R}^n$, $B_R(x_0) \subseteq \Omega$ und $u \in C^2(\Omega)$. Definiert man $\varphi : (0,R) \to \mathbb{R}$ mittels
	\[
		\varphi(r) = \fint_{\partial B_r(x_0)}^{} u \,\mathrm{d}S \qquad \text{für } r \in (0,R),
	\]
	so gelten
	\begin{enumerate}[(i)]
		\item $\varphi(r) \to  u (x_0)$ für $ r \to 0^+$,
		\item $\varphi'(r) = \frac{r}{n} \fint_{B_r(x_0)}^{} \Delta u \,\mathrm{d}x$.
	\end{enumerate}
	\end{lemma}
	\begin{beweis}
		\begin{enumerate}[(i)]
			\item \begin{align*}
				\abs{ \varphi(r)-u(x_0)} &= \abs{ \fint_{\partial B_r(x_0)}^{}u \,\mathrm{d}S - u(x_0)} \\
				&= \abs{ \fint_{\partial B_r(x_0)}^{} (u(x)-u(x_0)) \,\mathrm{d}S} \\
				& \leq \fint_{\partial B_r(x_0)}^{} \abs{u(x)-u(x_0)} \,\mathrm{d}S \\
				& \leq  \underset{=1}{\underbrace{\abs{\fint_{\partial B_r(x_0)}^{} \,\mathrm{d}S}}} \max \abs{u(x)-u(x_0)} \stackrel{r \to 0^+}{\to} 0
			\end{align*}
		\item Es gilt
		\begin{equation}
			\varphi(r) = \frac{1}{r^{n-1}S_n} \int_{\partial B_r(x_0)}^{}u(x) \,\mathrm{d}S(x) \stackrel{x=x_0 + r y}{=} \frac{1}{S_n} \int_{\partial B_r(x_0)}^{} u(x_0 + ry) \,\mathrm{d}S(y).
		\end{equation}
		$u(x_0 + ry)$ ist differenzierbar nach $r$ für alle $y \in \partial B_1(0)$. \\
		$\diffd{}{r}u(x_0 + ry) =  \nabla u(x_0+ry) \cdot y$ ist integrierbar auf $ \partial B_1(0)$.
		\begin{align*}
			\varphi'(r) &\stackrel{\hphantom{x=x_0+ry}}{=} \diffd{}{r} \left( \frac{1}{S_n} \int_{\partial B_1(0)}^{} u(x_0 + ry) \,\mathrm{d}S(y) \right) \\
			& \stackrel{\hphantom{x=x_0+ry}}{=} \frac{1}{S_n} \int_{\partial B_1(0)}^{} \diffd{}{r} u(x_0 + ry) \,\mathrm{d}S(y) \\
			& \stackrel{\hphantom{x=x_0+ry}}{=} \frac{1}{S_n} \int_{\partial B_1(0)}^{} \nabla u(x_0 + r y) \cdot y \,\mathrm{d}S(y) \\
			& \stackrel{x=x_0+ry}{=} \frac{1}{S_n r^{n-1}} \int_{\partial B_r(x_0)}^{}  \nabla u(x) \cdot \frac{x-x_0}{r} \,\mathrm{d}S(x) \\
			&\stackrel{\hphantom{x=x_0+ry}}{=} \frac{1}{S_n r^{n-1}} \int_{ B_r(x_0)}^{} \diver (  \nabla u(x)) \,\mathrm{d}x
		\end{align*}
		Hierbei ist $\frac{x-x_0}{r}$ die äußere Einheitsnormalenableitung an $\partial B_k(x_0)$ im Punkt x. 
		Wegen $u \in C^2(\Omega)$ gilt $ \nabla u \in C^1( \overline{B_k(x_0)})$ wobei $\overline{B_k(x_0)} \subseteq \Omega$. \\
		Es gilt außerdem $\diver(  \nabla u) = \Delta u$ und somit
		\begin{equation}
			\varphi'(r) \stackrel{\text{Gauß}}{=} \frac{r}{S_n r^n} \int_{B_k(x_0)}^{} \Delta u \,\mathrm{d}x 
			\stackrel{\omega_n = S_n n}{=} \frac{r}{n} \fint_{B_k(x_0)}^{} \Delta u \,\mathrm{d}x.
		\end{equation}
		\end{enumerate}
		
	\end{beweis}
\begin{korollar}[Mittelwerteigenschaft]
	Sei $\Omega$ eine offene Teilmenge in $\mathbb{R}^n$, $\overline{B_r(x_0)} \subseteq \Omega$ ($B_r(x_0) \subset \subset \Omega$), $u \in C^2(\Omega)$.
	\begin{enumerate}[(i)]
		\item Falls $ \Delta u = 0 $ in $\Omega$ gilt, so folgt
		\begin{equation}
			u(x_0) = \fint_{\partial B_r(x_0)}^{} u \,\mathrm{d}S = \fint_{B_r(x_0)}^{} u 	\,\mathrm{d}x.
		\end{equation}
		\item Falls $\Delta u \geq 0$ in $\Omega$ gilt, so folgt
		\begin{equation}
			u(x_0) \geq \fint_{\partial B_r(x_0)}^{} u \,\mathrm{d}S \qquad \text{,} \qquad u(x_0) \leq \fint_{B_r(x_0)}^{}u \,\mathrm{d}x.
		\end{equation}
		\item Falls $\Delta u > 0$ in $\Omega$ gilt, so folgt
		\begin{equation}
			u(x_0) < \fint_{\partial B_r(x_0)}^{} u\,\mathrm{d}S \qquad \text{,} \qquad u(x_0) < \fint_{B_r(x_0)}^{} u \,\mathrm{d}x.
		\end{equation}
	\end{enumerate}
	\end{korollar}
	\begin{beweis}
		\begin{enumerate}[(i)]
			\item folgt aus (ii) angewandt auf $u$ und $-u$.
			\item Aus $\Delta u \geq 0$ folgt, dass $\varphi$ monoton wachsend ist.
			\begin{equation}
				u(x_0) = \lim_{S \to 0^+} \varphi(s) \leq \varphi(\rho) = \fint_{\partial B_{\rho}(x_0)}^{}u \,\mathrm{d}S \qquad \forall\, \rho \in (0,r].
			\end{equation}
			Insbesondere folgt direkt
			\[
				u(x_0) \geq \fint_{\partial B_r(x_0)}^{}u \,\mathrm{d}S.
			\]
			Die zweite Ungleichung in (ii) folgt durch Integration.
			\begin{align*}
				u(x_0)\abs{B_r(x_0)} &= u(x_0) \omega_n r^n \\ &= u(x_0) \omega_n \int_{0}^{r}n \rho^{n-1} \,\mathrm{d}\rho \\ 
				&= n\omega_n \int_{0}^{r}u(x_0)\rho^{n-1} \,\mathrm{d}\rho  \\
				&\leq n \omega_n \int_{0}^{r}\fint_{\partial B_{\rho}(x_0)}^{} u \,\mathrm{d}S \rho^{n-1} \,\mathrm{d}\rho \\
				&= n \omega_n \int_{0}^{r} \frac{1}{S_n} \int_{B_{\rho}(x_0)}^{} u \,\mathrm{d}S \,\mathrm{d}\rho \\
				&= \int_{0}^{r} \int_{B_{\rho}(x_0)}^{}u \,\mathrm{d}S \,\mathrm{d}\rho \\
				&= \int_{B_{r}(x_0)}^{}u \,\mathrm{d}x,
			\end{align*}
			wobei der letzte Schritt die Anwendung von Polarkoordinaten beinhaltet. (iii) folgt analog. In diesem Fall ist $\varphi$ streng monoton wachsend.
			\end{enumerate}
	\end{beweis}

\begin{satz}
	Sei $\Omega$ eine offene Teilmenge in $\mathbb{R}^n$ und $u \in C^2(\Omega)$. Die folgenden Eigenschaften sind äquivalent:
	\begin{enumerate}[(i)]
		\item $u$ ist harmonisch, d.h. es gilt 
		\begin{equation}
			\Delta u = 0 \qquad \text{in } \Omega.
		\end{equation}
		\item $u$ erfüllt die sphärische Mittelwerteigenschaft, d.h. es gilt
		\begin{equation}
			u(x_0) = \fint_{\partial B_r(x_0)}^{}u \,\mathrm{d}S
		\end{equation}
		für alle $x_0 \in \Omega$ und $r>0$ mit $B_r(x_0) \subset \subset \Omega$
		\item $u$ erfüllt die Mittelwerteigenschaft auf Kugeln, d.h. es gilt
		\begin{equation}
			u(x_0) = \fint_{B_r(x_0)}^{}u \,\mathrm{d}x
		\end{equation}
		für alle $x_0 \in \Omega$ und $r > 0$ mit $B_r(x_0) \subset \subset \Omega$.
		\item $u$ erfüllt die Mittelwerteigenschaft auf \underline{kleinen} Kugeln, d.h. für jedes $x_0 \in \Omega$ existiert eine positive Zahl von $R(x_0) < \dist(x_0,\partial \Omega)$, so dass für alle $r < R(x_0)$ gilt
		\begin{equation}
			u(x_0) = \fint_{B_r(x_0)}^{}u \,\mathrm{d}x.
		\end{equation}
	\end{enumerate}
\end{satz}
\begin{beweis}
	\begin{description}
		\item[\underline{(i) $\Rightarrow$ (ii)}:] War gerade die Aussage (i) von Korrolar $2.4$.
		\item[\underline{(ii) $\Rightarrow$ (iii)}:] folgt mittels der Polarkoordinaten:
		\begin{align*}
			\int_{B_r(x_0)}^{}u \,\mathrm{d}x &= \int_{0}^{r} \int_{\partial B_r(x_0)}^{} u \,\mathrm{d}S \,\mathrm{d}\rho  \\
			&= \int_{0}^{r} \rho^{n-1}S_n \underset{=u(x_0)}{\underbrace{\fint_{\partial B_{\rho}(x_0)}^{} u \,\mathrm{d}S}} \,\mathrm{d} \rho \\
			&= \int_{0}^{r}S_n \rho^{n-1} u(x_0)\,\mathrm{d}\rho \\
			&= S_n \frac{r^n}{n} u(x_0).
		\end{align*}
		Es folgt \[
			u(x_0) = \fint_{B_r(x_0)}^{} u \,\mathrm{d}x.
		\]
		\item[\underline{(iii) $\Rightarrow$ (iv)}:] ist offensichtlich.
		\item[\underline{(iv) $\Rightarrow$ (i)}:] folgt durch ein Widerspruchsargument: \\
		Angenommen, es gilt (iv) und $\Delta u(x_0) \neq 0$ für ein $x_0 \in \Omega$, zum Beispiel $ \Delta u(x_0) > 0$. 
		Da $u \in C^2(\Omega)$ ist, findet man eine Kugel $B_r(x_0)$ mit $r < R(x_0)$, so dass
		\[
			\Delta u(x) > 0 \qquad \text{in } B_r(x_0).
		\]
		Außerdem folgt dann aus Korollar $2.4$, dass 
		\[
			u(x_0)< \fint_{\partial B_r(x_0)}^{} u \,\mathrm{d}S.
		\]
		Dies ist ein Widerspruch.
	\end{description}
\end{beweis}

\subsection{Folgerung aus der Mittelwerteigenschaft} 
\label{sub:folgerung_aus_der_mittelwerteigenschaft}
\begin{satz}[Maximumsprinzipien]
	Sei $\Omega$ eine beschränkte, offene Teilmenge des $\mathbb{R}^n$ und $u \in C^0(\bar{\Omega}) \cap C^2(\Omega)$ eine subharmonische Funktion (d.h. $ \Delta u \geq 0$ in $\Omega$). Dann gilt:
	\begin{enumerate}[(i)]
		\item Das schwache Maximumsprinzip:
		\[
			\max_{\bar{\Omega}} u = \max_{\partial \Omega}u.
		\]
		\item Das starke Maximumsprinzip: \\
		Ist $\Omega$ zusammenhängend und existiert ein innerer Punkt $x_0 \in \Omega$ mit \[
			u(x_0) = \max_{ \bar{\Omega}} u,
		\]so ist $u$ konstant.
	\end{enumerate}
\end{satz}
\begin{bemerkung}
	Entsprechend gelten Minimumsprinzipien für subharmonische Funktionen.
\end{bemerkung}
\begin{definition*}
	Eine Teilmenge $\Omega$ in $\mathbb{R}^n$ heißt zusammenhängend,
	falls sie sich nicht als disjunkte Vereinigung zweier nicht leerer $\Omega$-offene (= relativ offen in $\Omega$) Mengen schreiben lässt. \\
	\\
	Wenn $\Omega = O_1 \cup O_2$ mit $O_1,O_2$ offen und $O_1 \cap O_2 = \emptyset$, so folgt entweder $O_1 = \emptyset$ oder $O_2 = \emptyset$.
\end{definition*}
\begin{bemerkung}
	Diese Definition ist äquivalent zu: \\
	\\
	$\Omega$ und $\emptyset$ sind die beiden einzigen Mengen, die zugleich $\Omega$-offen und $\Omega$-abgeschlossen.
\end{bemerkung}

\begin{beweis}
	\begin{description}
		\item[\underline{(ii) $\Rightarrow$ (i)}:] Beweis durch ein Wiederspruchsargument: \\
		Angenommen, es gelte 
		\begin{equation}
			\max_{\bar{\Omega}}u > \max_{\partial \Omega} u.
		\end{equation}
		($\geq$ ist trivial, da $\partial \Omega \subseteq \bar{\Omega}$). \\
		Dann gäbe es ein Punkt $x_0 \in \Omega$ mit $u(x_0) = \max_{\bar{\Omega}}u$ auf die Zusammenhangskomponente von $\Omega(x_0)$, wäre also
		\begin{align*}
			\max_{\overline{\Omega(x_0)}}u &= \max_{\bar{\Omega}}u \\
			&> \max_{\partial \Omega}u \\
			&\geq  \max_{ \partial \Omega(x_0)}u.
		\end{align*}
		Dies gilt wegen $ \partial \Omega(x_0) \subseteq \partial \Omega$. \\
		Aus dem starken Maximumsprinzip folgt, dass $u$ Konstant in $\Omega(x_0)$ ist und dies ist ein Widerspruch zu der obigen Aussage
		\[
			\max_{\overline{\Omega(x_0)}}u > \max_{\partial \Omega(x_0)} u.
		\]
		\item[\underline{(ii)}:] Sei $x_0 \in \Omega$ mit $u(x_0) = \max_{\bar{\Omega}}u =: M$. Dann gilt
		\begin{align*}
			M = u(x_0) \stackrel{(*)}{\leq } \fint_{B_r(x_0)}^{} u \,\mathrm{d}x \leq  \fint_{B_r(x_0)}^{} M \,\mathrm{d}x = M 
		\end{align*}
		mit $\overline{B_r(x_0)} \subseteq \Omega$. Hierbei folgt $(*)$ aus der Mittelwerteigenschaft mit $ \Delta u \geq 0$. $M \leq M$ ist allerdings nur möglich, 
		wenn $u \equiv M$ auf $B_r(x_0)$ ist. Sei
		\[
			\Omega_M := \set[x \in \Omega]{u(x)=M}.
		\]
		Dann ist $\Omega_M$ offen also findet man für jedes $x_0 \in \Omega_M$ eine offene Kugel $B_r(x_0) \subseteq \Omega_M$ finden. \\
		Für $x_0 \in \Omega_M$ existiert $r>0$, so dass auf $B_r(x_0)$ $u(x) \equiv M$ gilt. 
		Daraus folgt $B_r(x_0) \subseteq  \Omega_M$ und somit ist $\Omega_M$ offen. \\
		Da $u$ stetig ist, ist $\Omega_M$ relativ abgeschlossen in $\Omega$. 
		Und weil $\Omega$ zusammenhängend ist, gilt somit entweder $\Omega_M = \Omega$ oder $\Omega_M = \emptyset$.
		Da aber $x_0 \in \Omega_M$ gilt somit $\Omega_M = \Omega$ und so $u \equiv M$ auf $\Omega$.
	\end{description}
\end{beweis}

\begin{align}\label{poi}\tag{p}
	\Delta u &= f \qquad \text{in }\Omega \qquad \quad \text{(Poisson-Gleichung),}\\
	u&=g \qquad \text{auf }\partial \Omega \qquad \text{(Dirichlet-Randbedingungen).}
 \end{align}

\begin{korollar}[Eindeutigkeit von Lösungen von (p)]
	Sei $\Omega \subseteq \mathbb{R}^n$ offen und beschränkt. Seien $f \in C^0(\Omega)$ und $g \in C^0(\partial \Omega)$ gegebene Funktionen. Seien $u_1,u_2 \in C^2(\Omega)\cap C^0(\bar{\Omega})$ zwei Lösungen von \eqref{poi}. Dann folgt
	\begin{equation}
		u_1 \equiv u_2.
	\end{equation}
\end{korollar}
\begin{beweis}
	Sei $w := u_1 - u_2$. Dann gilt
	\begin{align}
		\Delta w &= 0 \qquad \text{in }\Omega, \\
		w &= 0 \qquad \text{auf }\partial. \Omega
	\end{align}
	Da $w$ subharmonisch ist, gilt
	\begin{equation}
		\max_{\bar{\Omega}} w = \max_{\partial \Omega} w = 0.
	\end{equation}
	Da $-w$ subharmonisch ist, gilt außerdem
	\begin{equation} 
		\max_{\bar{\Omega}}(-w) = \max_{\partial \Omega}(-w)
	\end{equation}
	und \begin{equation}
		\max_{\bar{\Omega}}(-w) = \min_{\bar{\Omega}}(w) = \min_{\partial \Omega}(w) = 0,
	\end{equation}
	\begin{equation}
		\Rightarrow w \equiv 0 \qquad \text{in }\Omega,
	\end{equation}
	\[
		\Rightarrow u_1 \equiv u_2.
	\]
	\end{beweis}
Stetig Abhängigkeit von Randbedingungen in der $C^0-Norm$:
\begin{align*}
	\norm{u_1-u_2}_{C^0(\bar{\Omega})} &= \max_{\bar{\Omega}} \abs{u_1-u_2} \\
	&= \max \set{ \max_{\bar{\Omega}} (u_1-u_2), \max_{\bar{\Omega}}(u_2-u_1) } \\
	&= \max \set{ \max_{\bar{\Omega}} (u_1-u_2), -\min_{\bar{\Omega}}(u_1-u_2) } \\
	&= \max \set{ \max_{\partial \Omega} (u_1-u_2), -\min_{\partial \Omega}(u_1-u_2) } \\
	&= \max \set{ \max_{\partial \Omega} (g_1-g_2), -\min_{\partial \Omega}(g_1-g_2) } \\
	&= \max \abs{g_1-g_2} \\
	&= \norm{g_1-g_2}_{C^0(\Omega)}.
\end{align*}
Es folgt
\[
	\norm{u_1-u_2}_{C^0(\bar{\Omega})} = \norm{g_1-g_2}_{C^0(\partial \Omega)}.
\]


\begin{satz}[Harneck-Ungleichung]
	Sei $\bar{V}$ kompakt und $\bar{V} \subseteq \Omega$. Sei $\Omega \subseteq  \mathbb{R}^n$ offen und $V \subset \subset \Omega $ offen und
	zusammenhängend. Dann gibt es eine Konstante $C = C(V) >0$, so dass
	\begin{equation}
		\sup_{V}u \leq C \inf_{V}u
	\end{equation}
	für alle nichtnegativen, harmonischen Funktionen $u$ auf $\Omega$. Insbesondere sind alle Funktionswerte von $u$ in $V$ vergleichbar, das heißt es gilt
	\begin{equation}
		u(y) \leq C u(x) \leq C^2 u(y)
	\end{equation}
	für alle $x,y \in V$.
\end{satz}
\begin{beweis}
	\begin{description}
		\item[$\underline{\text{Schritt }1:}$] Wir setzen $r:= \frac{\dist(V,\partial \Omega)}{4} > 0$. $V \subset \subset \Omega$. Dann folgt für $x_0,x_1 \in V$, dass $\abs{x_1-x_0}< r$ und somit
		\begin{equation}
			B_{2r}(x_1) \subset \subset \Omega ,
		\end{equation}
		\begin{equation}
			B_r(x_1) \subseteq B_{2r}(x_1).
		\end{equation}
		Es gilt
		\begin{equation}
			y \in B_r(x_0) \Rightarrow \abs{y-x_1} \leq \abs{y-x_0} + \abs{x_1 - x_0} < 2r.
		\end{equation}
		Da $u$ harmonisch ist, gilt
		\begin{align*}
			u(x_0) = \fint_{B_r(x_0)}^{} u \,\mathrm{d}x &= \frac{1}{\omega_n r^n} \int_{B_r(x_0)}^{}u\,\mathrm{d}x \stackrel{u \geq 0}{=} \frac{2^n}{2^n \omega_n r^n}
			\int_{B_{2r}(x_0)}^{}u \,\mathrm{d}x \\ &= 2^n \fint_{B_{2r}(x_0)}^{}u \,\mathrm{d}x = 2^n u(x_1),
		\end{align*}
		wobei die letzte Gleichheit aus der Mittelwerteigenschaft auf $B_{2r}(x_1)$ folgt. Insgesamt gilt somit
		\[
			u(x_0) \leq 2^n u(x_1).
		\]
		\item[$\underline{\text{Schritt }2:}$] $x,y \in V$. 
		Da $\bar{V}$ kompakt ist und $V$ zusammenhängend ist, existiert eine \underline{endliche} Überdeckung von $V$ 
		durch eine Kette von offenen Kugeln $B_j := B_{\frac{k}{2}}(x_j)$ für $j=1,\dots,N$, so dass $B_j \cap B_{j+1} \neq \emptyset$ für $j=1,\dots,N-1$.
		\begin{equation}
			\abs{x_j - x_{j+1}} < r \qquad \stackrel{\text{Schritt }1}{\Rightarrow } \qquad u(x_j) \leq 2^n u(x_{j+1}).
		\end{equation}
		Sind $x,y \in V$ beliebige Punkte, so können wir Zwischenpunkte $x_1,\dots,x_N$ so wählen, dass $\abs{x-x_1}<r, \abs{x_2-x_1}<r, \dots, \abs{y-x_N}<r$.
		\begin{equation}
			u(x) \leq 2^n u(x_1) \leq (2^n)^2 u(x_2) \dots \qquad \Rightarrow  \qquad u(x) \leq \left(2^n \right)^N u(y),
		\end{equation} 
		\begin{equation}
			\Rightarrow \sup_V u \leq \left( 2^n \right)^N \inf_{V}u.
		\end{equation}
	\end{description}
\end{beweis}
\subsection{Ein schwaches Maximumsprinzip für lineare elliptische Gleichungen} 
\label{sub:ein_schwaches_maximumsprinzip_fur_lineare_elliptische_gleichungen}
Sei $\Omega \subseteq \mathbb{R}^n$ offen und $x \in \Omega$.
\[
	Lu := - \sum^{n}_{i,j=1}a_{ij}(x)D_{ij}u(x) + \sum^{n}_{i=1}b_i(x)D_iu(x).
\]
Wir nehmen an, dass $A= (a_{ij})_{i,j=1,\dots,n}$ stetig, beschränkt symmetrisch ($A^T = A$) und elliptisch ist, im Sinne: \\
es existiert ein $\alpha$, so dass für alle $x \in \Omega$ und $\xi \in \mathbb{R}^n$ gilt
\[
	A(x) \xi \cdot \xi  = \sum^{n}_{i,j=1} a_{ij}(x)\xi_i\xi_j \geq \alpha \abs{\xi}^2.
\]
Außerdem sei $b=(b_1,\dots,b_n)$ stetig und beschränkt.
\begin{bemerkung}
	$A \in I_n$, $b=0$. Dann $Lu= - \Delta u$.
\end{bemerkung}

\begin{satz}[Schwaches Maximumsprinzip]
	Sei $\Omega$ eine offene, beschränkte Teilmenge des $\mathbb{R}^n$ und $u \in C^2(\Omega) \cap C^0(\bar{\Omega})$. Falls $Lu \leq 0$ gilt , so folgt
	\begin{equation}
		\max_{\bar \Omega}u = \max_{ \partial \Omega}u.
	\end{equation}
	Entsprechend gilt ein schwaches Maximumsprinzip, falls $Lu \geq 0$ gilt.
\end{satz}
\begin{beweis}
	Wir nehmen an, dass $x_0 \in \Omega$ existiert mit $u(x_0)= \max_{\bar{\Omega}}u > \max_{ \partial \Omega}u$.
	\begin{description}
		\item[\underline{Schritt $1$:}]Herleitung eines Widerspruchs im Fall $Lu < 0$. $Du(x_0)=0$, weil $x_0$ ein innerer Maximumspunkt von $u$ ist. $D^2u(x_0)$ ist negativ semidefinit, d.h 
		\begin{equation}
			D^2u(x_0) \xi \cdot \xi \leq 0 \qquad \forall\, \xi \in \mathbb{R}^n.
		\end{equation}  
		Da $A$ symmetrisch und positiv definit ist gilt 
		\[
			O^T A(x_0) O = \begin{pmatrix}
				d_1 & & 0 \\ & \ddots & \\ 0 & & d_n
			\end{pmatrix} \qquad \text{mit } d_i > 0 \qquad \text{und } O \text{ Orthogonale Matrix,}
		\]
		\[
			A(x_0)= O \begin{pmatrix}
				d_1 & & 0 \\ & \ddots & \\ 0 & & d_n
			\end{pmatrix} O^T.
		\]
		Also
		\[
			a_{ij}(x_0) = \sum^{n}_{k=1}O_{ik}d_kO_{jk} \qquad \forall\, i,j=1,\dots,n,
		\]
		\begin{align*}
			Lu(x_0) &= - \sum^{n}_{i,j=1} a_{ij}(x_0) D_{ij}u(x_0) + \sum^{n}_{i=1}b_i(x_0)D_iu(x_0)  \\
			&= - \sum^{n}_{i,j=1} a_{ij}(x_0)D_{ij}u(x_0) \\
			&= - \sum_{k=1}^{n} d_k \underset{\underset{\leq 0}{\underbrace{D^2u(x_0)O_k \cdot O_k}}}{\underbrace{\sum^{n}_{i,j=1} O_{ik}O_{jk} D_{ij}u(x_0)}} > 0.
		\end{align*}
		Dies steht im Widerspruch zu $Lu < 0$. Also falls $Lu < 0$, dann gilt
		\[
			\max_{\bar{\Omega}}u = \max_{ \partial \Omega}u.
		\]
		\item[\underline{Schritt $2$:}] Herleitung eines Widerspruchs im allgemeinen Fall. Sei $Lu \leq 0$ und sei 
		\[
			u_{\varepsilon}(x):= u(x) + \varepsilon e^{\lambda x_1},
		\]
		wobei $x = (x_1, \dots, x_n)$, $\varepsilon > 0$ und $\lambda \in \mathbb{R}$ ein freier Parameter.
		\begin{align*}
			Lu_{\varepsilon}(x) &= Lu(x) + L \varepsilon e^{\lambda x_1} \leq \varepsilon L e^{\lambda x_1} \\ 
			&= - \varepsilon \lambda^2 a_{11}(x) e^{\lambda x_1} + \varepsilon \lambda b_1(x) e^{\lambda x_1} \\
			&= e \lambda e^{\lambda x_1} ( - \lambda a_{11}(x)+ b_1(x)).
		\end{align*}
		Aus \begin{equation}
			a_{11}(x)= A(x)e_1 \cdot e_1 \geq \alpha \abs{e_1}^2 = \alpha
		\end{equation}
		folgt
		\[
			-a_{11}(x) \leq -\alpha
		\]
		und aus $\abs{b_1} \leq \norm{b}_{L^{\infty}(\Omega,\mathbb{R}^n)}$ folgt dann
		\[
			Lu_{\varepsilon}(x) \leq  \varepsilon \lambda e^{\lambda x_1}(-\lambda \alpha + \norm{b}_{L^{\infty}(\Omega,\mathbb{R}^n)}).
		\]
		Es muss gelten 
		\[
			-\lambda \alpha + \norm{b}_{L^{\infty}(\Omega,\mathbb{R}^n)}<0 \qquad \Leftrightarrow \qquad \lambda > \frac{\norm{b}_{L^{\infty}(\Omega,\mathbb{R}^n)}}{\alpha}.
		\]
		Also folgt 
		\[
			Lu_{\varepsilon} < 0 \qquad \forall\, \varepsilon >0 
		\]
		und mit Schritt $1$ gilt dann
		\[
			\max_{\bar{\Omega}}u_{\varepsilon} = \max_{ \partial \Omega}u_{\varepsilon}.
		\]
		Allerdings gilt \[
			\max_{\bar{\Omega}}u > \max_{ \partial \Omega}u
		\]
		und somit für kleine $\varepsilon >0$ \[
			\max_{\bar{\Omega}}u_{\varepsilon} > \max_{ \partial \Omega}u_{\varepsilon}.
		\]
		Dies ist der Widerspruch zu oben.
 	\end{description}
\end{beweis}
\subsection{Regularität harmonischer Funktionen} 
\label{sub:regularitat_harmonischer_funktionen}
Wir werden zeigen, dass wenn $u$ harmonisch auf $\Omega$ ist folgt, dass $u \in C^{\infty}(\Omega)$.
\minisec{Glättungen}
$\eta \in C^{\infty}_0(B_1(0))$ Glättungskern, d.h eine nicht-negative, radialsymmetrische ($\eta(x)=\tilde\eta(\abs{x})$) Funktion mit
\[
	\int_{\mathbb{R}^n}^{}\eta \,\mathrm{d}x = 1
\]	
\begin{beispiel}
	\begin{equation}
		\eta(x) = \begin{cases}
			C e^{\frac{1}{\abs{x}^2-1}}, &\text{ falls }\abs{x}<1\\
			0 , &\text{ falls }\abs{x}\geq 1\\
		\end{cases}
	\end{equation}
	ist der Standard-Glättungskern mit einer Normierungskonstante $C$. FÜr $\varepsilon > 0$ definiert man die assoziierten \underline{skalierten} Glättungskerne als
	\[
		\eta_{\varepsilon}:= \frac{1}{\varepsilon^n}\eta \left( \frac{x}{\varepsilon} \right)
	\]
	mit $\eta_{\varepsilon} \in C_0^{\infty}(B_{\varepsilon}(0))$ und die anderen Eigenschaften bleiben erhalten.
\end{beispiel}

%%%%%%% 25.04.16 %%%%%%%%

\begin{definition*}[$\varepsilon$-Glättung]
	Sei $\Omega$ eine offene Teilmenge im $\mathbb{R}^n$, $\varepsilon >0$ und 
	\[
		\Omega_{\varepsilon} := \set[x \in \Omega]{\dist(x,\partial \Omega)> \varepsilon}.
	\]
	Für $f \in L^1_{\text{loc}}(\Omega)$ definiert man die $\varepsilon$-Glättung $f \varepsilon$ von $f$ als die Faltung von $f$ mit dem Glättungskern $\eta_{\varepsilon}$, also
	\[
		f_{\varepsilon}(x):= \eta_{\varepsilon} * f := \int_{\Omega}^{}\eta_{\varepsilon}(x-y)f(y) \,\mathrm{d}y \qquad \text{für alle }x \in \Omega_{\varepsilon}.
	\]
\end{definition*}

\begin{bemerkung}[Eigenschaften von $\varepsilon$-Glättungen]
	\begin{enumerate}[1)]
		\item $f_{\varepsilon} \in C^{\infty}(\Omega _{\varepsilon})$ mit 
		$D^{\alpha}f _{\varepsilon}(x)= \int_{\Omega}^{}D^{\alpha} \eta_{\varepsilon}(x-y)f(y) \,\mathrm{d}y$ für beliebige Multiindizes $\alpha \in \mathbb{N}_0^n$.
		\item $f_{\varepsilon} \to f$ fast überall in $\Omega$.
		\item $f_{\varepsilon} \to f$ in $L^1_{\text{loc}}(\Omega)$
	\end{enumerate}
\end{bemerkung}

\begin{satz}
	Sei $\Omega$ eine offene Teilmenge des $\mathbb{R}^n$ und $u \in C^0(\Omega)$ eine Funktion, die die sphärische Mittelwerteigenschaft erfüllt, d.h.
	\begin{equation}
		u(x_0) = \fint_{\partial B_r(x_0)}^{} u\,\mathrm{d}S \qquad \forall\, B_r(x_0) \subset \subset \Omega. 
	\end{equation}
	Dann gilt
	\begin{equation}
		u(x_0) = u_{\varepsilon}(x_0) \qquad \forall\, x_0 \in \Omega \text{ und } \varepsilon < \dist(x_0,\partial \Omega),
	\end{equation}
	wobei $u_{\varepsilon}$ die $\varepsilon$-Glättung bezeichnet. Insbesondere ist $u$ also von der Klasse $C^{\infty}(\Omega)$ und harmonisch auf $\Omega$.
\end{satz}
\begin{beweis}
	$x_0 \in \Omega$, $\varepsilon < \dist(x_0, \partial \Omega)$. \\
	Wegen der Radialsymmetrie von $\eta _{\varepsilon}$ und weil $\eta _{\varepsilon}(x_0-x)= \text{konstant}$ für alle $x \in \partial B_r(x_0)$ gilt
	\begin{align*}
		u_{\varepsilon}(x_0) &\stackrel{\hphantom{\substack{\text{Mittelwert-} \\ \text{Eigenschaft}}}}{=} \int_{B_{\varepsilon}(x_0)}^{}\eta_{\varepsilon}(x_0-y) u(y) 
		\,\mathrm{d}y \\
		&\stackrel{\hphantom{\substack{\text{Mittelwert-} \\ \text{Eigenschaft}}}}{=} \int_{0}^{\varepsilon} \int_{\partial B_r(x_0)}^{} \eta_{\varepsilon}(x_0-y)u(y)
		 \,\mathrm{d}S(y) \,\mathrm{d}r \\
		&\stackrel{\hphantom{\substack{\text{Mittelwert-} \\ \text{Eigenschaft}}}}{=} \int_{0}^{\varepsilon} 
		\left(\eta_{\varepsilon}\underset{=r}{\underbrace{(x_0-y)}} \cdot \int_{\partial B_r(x_0)}^{}u(y) \,\mathrm{d}S(y) \right) \,\mathrm{d}r \\
		&\stackrel{\hphantom{\substack{\text{Mittelwert-} \\ \text{Eigenschaft}}}}{=} \int_{0}^{\varepsilon} \left( \eta_{\varepsilon}(x_0-y)S_nr^{n-1} \cdot 
		\underset{=u(x_0)}{\underbrace{\fint_{\partial B_r(x_0)}^{}u(y) \,\mathrm{d}S(y)}} \right) \,\mathrm{d}r \\
		& \stackrel{\substack{\text{Mittelwert-} \\ \text{Eigenschaft}}}{=} \int_{0}^{\varepsilon}u(x_0) \cdot \int_{\partial B_r(x_0)}^{} \eta_{\varepsilon}(x_0-y)
		\,\mathrm{d}S(y) \,\mathrm{d}r \\
		&\stackrel{\hphantom{\substack{\text{Mittelwert-} \\ \text{Eigenschaft}}}}{=} 
		\underset{=1}{\underbrace{\int_{B_{\varepsilon}(x_0)}^{} \eta_{\varepsilon}(x_0-x) \,\mathrm{d}x}} \\
		&\stackrel{\hphantom{\substack{\text{Mittelwert-} \\ \text{Eigenschaft}}}}{=} u(x_0)
	\end{align*}
\end{beweis}

\begin{bemerkung}
	Der Satz macht keine Annahme über die Randwerte von $u$, diese müssen nicht glatt sein (und können sogar unstetig sein).
\end{bemerkung}

Die Aussage von Satz 2.10 bleibt gültig, wenn die Funktion $u$ die Mittelwerteigenschaft auf Kugeln erfüllt.

\begin{korollar}
	Sei $\Omega \subseteq \mathbb{R}^n$ offen und $u \in C^0(\Omega)$ eine Funktion, die die Mittelwerteigenschaft auf Kugeln erfüllt, d.h.
	\begin{equation}
		u(x_0) = \fint_{B_r(x_0)}^{}u \,\mathrm{d}x \qquad \forall\, B_r(x_0) \subset \subset \Omega.
	\end{equation}
	Dann gilt
	\begin{equation}
		u(x_0) = u_{\varepsilon}(x_0) \qquad \forall\, x_0 \in \Omega, \varepsilon < \dist(x_0,\partial \Omega).
	\end{equation}
\end{korollar}

\begin{beweis}
	Es genügt zu zeigen, dass auch in diesem Fall die sphärische Mittelwerteigenschaft erfüllt ist. 
	Dazu definieren wir für eine beliebige Kugel $B_r(x_0) \subset \subset \Omega$ die Funktion $\psi: (0,r) \to \mathbb{R}$ mit
	\begin{equation}
		\psi(\rho) := \int_{\partial B_{\rho}(x_0)}^{} (u(x)-u(x_0)) \,\mathrm{d}S(x)
	\end{equation}
	$\psi$ ist stetig, weil $u$ stetig ist. 
	Außerdem gilt mit $x= x_0 + \rho y$ für alle $R \in (0,r)$
	\begin{equation}
		\int_{0}^{R} \psi(\rho) \,\mathrm{d}\rho = \int_{0}^{R} \int_{\partial B_1(0)}^{} (u(x_0+\rho y)-u(x_0)) \rho^{n-1} \,\mathrm{d}S \,\mathrm{d}\rho
		= \int_{B_R(x_0)}^{}(u(x)-u(x_0)) \,\mathrm{d}x = 0.
	\end{equation}
	Damit muss $\psi \equiv 0$ auf $(0,r)$ gelten und somit
	\[
		u(x_0) = \fint_{\partial B_r(x_0)}^{} u(x) \,\mathrm{d}S(x)
	\]
\end{beweis}
Eine einfache Folgerung ist nun, dass Harmonizität unter gleichmäßiger Konvergenz erhalten bleibt.

\begin{korollar}[Konvergenzsatz von Weierstraß]
	Sei $\Omega \subseteq \mathbb{R}^n$ offen und zusammenhängend. Sei $(u_k)_{k \in \mathbb{N}}$ eine Folge harmonischer Funktionen auf $\Omega$, die (lokal) gleichmäßig gegen eine Funktion $u$ konvergiert. Dann ist $u$ harmonisch auf $\Omega$.
\end{korollar}

\begin{beweis}
	Für jede Kugel $B_r(x_0) \subset \subset \Omega$ gilt 
	\[
		u(x_0)= \lim_{k \to \infty}u_k(x_0) = \lim_{k \to \infty} \fint_{B_r(x_0)}^{}u_k \,\mathrm{d}x = \fint_{B_r(x_0)}^{}u \,\mathrm{d}x.
	\]
	Dies gilt, weil $u$ stetig ist (gleichmäßiger Limes stetiger Funktionen). 
	Somit erfüllt $u$ die Mittelwerteigenschaft auf Kugeln und mit Korollar 2.11 folgt dann die Behauptung.
\end{beweis}

\begin{korollar}[Harnack'scher Konvergenzsatz]
	Sei $\Omega \subseteq \mathbb{R}^n$ offen und zusammenhängend. Sei $(u_k)_{k \in \mathbb{N}}$ eine monoton wachsende Folge harmonischer Funktionen. 
	Gibt es ein $x_0 \in \Omega$, so dass $(u_k(x_0))_{k \in \mathbb{N}}$ beschränkt (und damit konvergent) ist, so konvergiert $(u_k)_{k \in \mathbb{N}}$ auf jeder
	zusammenhängenden offenen Menge $V \subset \subset \Omega$ gleichmäßig gegen eine harmonische Funktion auf $\Omega$.
\end{korollar}

\begin{beweis}
	Wegen der Monotonie ist $u_k-u_j$ für $k > j$ eine nicht-negative harmonische Funktion. Sei $V \subset \subset \Omega$ offen und zusammenhängend. 
	Sei o.B.d.A $x_0 \in V$ (andernfalls können wir $ \tilde V$ mit $V \subset \tilde V \subset \subset \Omega$ mit $x_0 \in \tilde V$ konstruieren)
	
	
	\begin{align*}
		0 &\stackrel{\hphantom{x_0 \in V}}{\leq} \sup_V(u_k-u_j) \stackrel{\text{Harnack}}{\leq} c(V) \inf_V(u_k-u_j) \\
		& \stackrel{x_0 \in V}{\leq} c(V) (u_k(x_0)-u_j(x_0)) \\
		&\stackrel{\hphantom{x_0 \in V}}{\leq} c(V) \varepsilon
	\end{align*}
	Da $(u_k(x_0))_{k \in \mathbb{N}}$ nach Voraussetzung eine Cauchy-Folge ist folgt, dass $(u_k)$ eine Cauchy-Folge bezüglich der Supremumsnorm auf $V$ ist. 
	Damit ist sie gleichmäßig konvergent auf $V$. Nach Korollar 2.12 ihr Limes eine harmonische Funktion. \\
	Wähle $V = B_r(x_0) \subset \subset \Omega$, dann folgt
	\begin{equation}
		u_k(x_0) = \fint_{B_r(x_0)}^{}u_k \,\mathrm{d}x 
	\end{equation}
	und wegen gleichmäßiger Konvergenz folgt dann
	\begin{equation}
	u(x_0) = \fint_{B_r(x_0)}^{}u \,\mathrm{d}x.
	\end{equation}
	Somit ist $u$ harmonisch auf $\Omega$, weil $u$ stetig ist.
\end{beweis}

\begin{satz}[Hermann Weyl]
	Sei $\Omega \subseteq \mathbb{R}^n$ offen und $u \in L^1_{\text{loc}}(\Omega)$ eine Funktion, für die 
	\begin{equation}
		\int_{\Omega}^{} u \Delta \varphi \,\mathrm{d}x = 0 \qquad \forall\, \varphi \in C^{\infty}_0(\Omega)
	\end{equation}
	erfüllt ist. 
	Dann ist $u$ harmonisch in $\Omega$ (streng genommen: dann existiert eine Funktion $\tilde u$, die harmonisch ist mit $u = \tilde u$ fast überall in $\Omega$)
\end{satz}

\begin{beweis}
	Wir zeigen zunächst, dass die Glättungen $u_{\varepsilon}$ harmonisch in $\Omega_{\varepsilon}$ sind. Nach Übung gilt für $x \in \Omega_{\varepsilon}$
	\[
		\Delta u_{\varepsilon}(x) = ( \Delta \eta_{\varepsilon} * u)(x) = \int_{\Omega}^{} \Delta \eta_{\varepsilon}(x-y)u(y) \,\mathrm{d}y.
	\]
	Nach Voraussetzung mit $\varphi(y)= \eta_{\varepsilon}(x-y) \in C^{\infty}_0(\Omega)$ folgt 
	\[
		\Delta u_{\varepsilon}(x) \equiv 0 \qquad \text{in }\Omega_{\varepsilon}.
	\]
	Es gilt für fast alle $x_0 \in \Omega$ und alle $r < \dist(x_0, \partial \Omega)$ ($B_r(x_0) \subset \subset \Omega$) 
	\begin{equation}
		u(x_0) \stackrel{\substack{\text{fast überall}\\\text{Konvergenz}\\\text{von }u_{\varepsilon}}}{=}
		\lim_{\varepsilon \to 0} u_{\varepsilon}(x_0) 
		\stackrel{\substack{\text{Mittelwerteigenschaft} \\\text{der Familie }u_{\varepsilon}}}{=} \lim_{\varepsilon \to 0} 
		\fint_{B_r(x_0)}^{}u_{\varepsilon}(x) \,\mathrm{d}x 
		\stackrel{\substack{L^1_{\text{loc}}\text{-Konvergenz}\\ \text{von }u_{\varepsilon}}}{=}
		\fint_{B_r(x_0)}^{} u(x) \,\mathrm{d}x.
	\end{equation}
	Das heißt $u$ erfüllt für fast alle $x_0 \in \Omega$ die Mittelwerteigenschaft auf Kugeln. 
	Definiert man nun $\tilde u: \Omega \to \mathbb{R}$ durch
	\[
		\tilde u(x_0):= \fint_{B_r(x_0)}^{}u \,\mathrm{d}x
	\]
	mit $r:= \frac{\dist(x_0, \partial \Omega)}{2}$, so gilt 
	\begin{equation}
		u = \tilde u
	\end{equation}
	fast überall in $\Omega$. Wegen der Absolutstetigkeit der Integrale ist $\tilde u$ stetig und erfüllt die Mittelwerteigenschaft (auf Kugeln) überall in $\Omega$.
	Damit folgt die Harmonizität von $\tilde u$ aus Korollar 2.11.
\end{beweis}
Nun werden wir lokale Abschätzungen für höhere Ableitungen harmnoischer Funktionen beweisen.

%%%%%% 28.05.2016
\begin{satz}[Innere Abschätzung für Ableitungen harmonischer Funktionen]
	Sei $\Omega \subseteq \mathbb{R}^n$ offen und $u \in C^2(\Omega)$ eine harmonische Funktion auf $\Omega$. Dann gilt für jeden Multiindex $\alpha$ mit $ \abs{\alpha}=k$ und jede Kugel $B_r(x_0) \subset \subset \Omega$
	\begin{equation}
		\abs{D^{\alpha}u(x_0)} \leq c(n,k)r^{-n-k} \norm{u}_{L^1(B_r(x_0))}
	\end{equation}
	mit Konstanten 
	\begin{align}
		c(n,k) &:= \frac{\left( 2^{(n+1)}nk \right)^k}{\omega_n} \qquad \text{für }k=1,\dots,n \\
		c(n,0) &:= \frac{1}{\omega_n}
	\end{align}
\end{satz}

\begin{beweis}
	Wir beweisen die Abschätzungen mit Induktion über $k$. 
	Beachte, dass mit $u$ auch jede Ableitung $D^{\alpha}u$ harmonisch ist (weil z.B $u \in C^{\infty}$ ist $\Delta \diff{u}{x_i}= \diff{}{x_i}\Delta u = 0$) 
	und analog für alle anderen Ableitungen.
	\begin{description}
		\item[Der Fall $k=0$:] Die Behauptung folgt sofort aus der Mittelwerteigenschaft auf Kugeln:
		\begin{equation}
			u(x_0) = \frac{1}{r^n \omega_n} \int_{B_r(x_0)}^{}u \,\mathrm{d}
		\end{equation} 
		Also 
		\begin{equation}
			\abs{u(x_0)} \leq c(u,0)r^{-n} \norm{u}_{L^1(B_r(x_0))}
		\end{equation}
		\item[Der Fall $k=1$:] Für $i=1,\dots,n$ ist wegen oben $D_iu$ harmonisch und es gilt
		\begin{align*}
			\abs{D_i u(x_0)} & \stackrel{\hphantom{\text{P.I.}}}{=} \abs{ \fint_{B_{\frac{r}{2}}(x_0)}^{} D_iu(x) \,\mathrm{d}x} \\
			& \stackrel{\hphantom{\text{P.I.}}}{=} \frac{2^n}{\omega_n r^n} \abs{\int_{B_{\frac{r}{2}}(x_0)}^{}D_i u(x) \,\mathrm{d}x} \\
			& \stackrel{\text{P.I.}}{=} \abs{\int_{ \partial B_{\frac{r}{2}}(x_0)}^{} u \nu_i\,\mathrm{d}S} \\
			& \stackrel{\hphantom{\text{P.I.}}}{\leq} \frac{2^n}{\omega_n r^n} \left( \frac{r}{2} \right)^{n-1}S_n \sup_{\partial B_{\frac{r}{2}}(x_0)}\abs{u} \\
			& \stackrel{\hphantom{\text{P.I.}}}{=} \frac{2n}{r} \sup_{\partial B_{\frac{r}{2}}(x_0)}\abs{u}
		\end{align*}
		Es gilt für alle $x \in \partial B_{\frac{r}{2}}(x_0)$, dass $B_{\frac{r}{2}}(x) \subset B_r(x_0)$ und somit erhalten wir aus dem ersten Schritt
		\begin{equation}
			\abs{u(x)} \leq \frac{\left( \frac{r}{2} \right)^{-n}}{\omega_n} \norm{u}_{L^1(B_{\frac{r}{2}}(x))} 
			\leq \frac{\left( \frac{r}{2} \right)^{-n}}{\omega_n} \norm{u}_{L^1(B_r(x_0))},
		\end{equation}
		für alle $x \in \partial B_{\frac{r}{2}}(x_0)$ und damit
		\begin{equation}
			\sup_{x \in \partial B_{\frac{r}{2}}(x_0)}\abs{u(x)} \leq \frac{2^n}{r^n \omega_n} \norm{u}_{L^1(B_r(x_0))}.
		\end{equation}
		Insgesamt folgt die Behauptung mit
		\begin{equation}
			\abs{D_iu(x_0)}
			 = \underset{\frac{2^{n+1}n}{\omega_n}= c(n,1)}{\underbrace{\frac{2n}{r} \frac{\left( \frac{r}{2} \right)^{-n}}{\omega_n}}} \norm{u}_{L^1(B_r(x_0))}
		\end{equation}
		\item[Der Fall $k \geq 2$:]Ähnlich wie im Fall $k=1$ kann man die Ableitung der Ordnung $k$ durch das Supremum einer Ableitung der Ordnung $k-1$ auf einer
		kleinere Kugel abschätzen.
		So erhalten wir
		\[
			\abs{D^{\alpha}(x_0)} \leq  \frac{kn}{r} \sup_{\partial B_{\frac{r}{k}}(x_0)} \abs{D^{\beta}u}
		\]
		für einen Multiindex $\beta$ mit $\beta_j = \alpha_{j-1}$ für ein $j=1,\dots,n$ und $\beta_i = \alpha_i$ für $i \neq j$. Damit gilt $\abs{\beta}= k-1$. \\
		Nach Induktionsannahme gilt ähnlich wie im Fall $k=1$
		\begin{equation}
			\sup_{\partial B_{\frac{r}{k}}(x_0)} \abs{D^{\beta}u} \leq \frac{c(n,k-1)k^{n+1-1}}{\left( k-1 \right)^{n+k-1}r^{n+k-1}} \norm{u}_{L^1(B_r(x_0))}.
		\end{equation}
		Insgesamt foglt dann die Behauptung.
	\end{description}
\end{beweis}

\begin{satz}[Satz von Liouville]
	Sei $u \in C^2(\mathbb{R}^n)$ harmonisch. Ist $u$ beschränkt, so ist $u$ konstant.
\end{satz}
\begin{beweis}
	Sei $x_0 \in \mathbb{R}^n$ und $r >0$. Wie oben gilt für alle $k > 0$
	\begin{align*}
		\abs{D_iu(x_0)} &\leq c(n,1) \frac{1}{r^{n+1}} \norm{u}_{L^1(B_r(x_0))} \\
		& \leq \frac{1}{r^{n+1}} \int_{B_r(x_0)}^{} \sup_{\mathbb{R}^n} \abs{u} \,\mathrm{d}x \\
		&\leq \frac{1}{r^{n+1}}\, \underset{\leq M}{\underbrace{\sup_{\mathbb{R}^n}\abs{u}}} \,\omega_n r^n \\
		&\leq \frac{c}{r}.
	\end{align*}
	Also gilt für alle $x_0 \in \mathbb{R}^n$ und für alle $i=1,\dots,n$
	\[
		\lim_{k \to \infty}\abs{D_iu(x_0)} = 0
	\]
	Damit gilt 
	\[
		 \nabla u \equiv 0 
	\]
	in $\mathbb{R}^n$ und somit ist $u$ konstant.
\end{beweis}

\begin{definition}
	Sei $\Omega \in \mathbb{R}^n$ offen. Eine Funktion $f : \Omega \to \mathbb{R}$ heißt analytisch in einem Punkt $x \in \Omega$, 
	falls $f$ sich lokal durch seine Taylorreihe darstellen lässt, also falls es ein $r \in (0, \dist(x,\partial \Omega))$ existiert, so dass
	\begin{equation}
		f(y) = \sum_{n \in \mathbb{N}^n}^{} \frac{}{} \frac{1}{\alpha !}D^{\alpha}f(x)(y-x)^{\alpha} \qquad \text{für alle }y \in B_r(x) \text{ gilt.}
	\end{equation}
	Falls $f$ in allen $x$ analytisch ist, so heißt $f$ analytisch in $\Omega$.
\end{definition}

\begin{satz}
 	   Sei $\Omega \subseteq \mathbb{R}^n$ offen und $u \in C^2(\Omega)$ harmonisch auf $\Omega$. Dann ist $u$ analytisch in $\Omega$.
\end{satz}

\begin{beweis}
	siehe Blatt $3$ Aufgabe $4$.
\end{beweis}

Für $\Omega \subseteq \mathbb{R}^n$ offen, beschränkt und regulär betrachten wir für gegebene $f,g$ das Poisson-Problem
\begin{align}
	- \Delta u &= f \qquad \text{in }\Omega \\
	u &= g \qquad \text{auf }\Omega
\end{align}

\subsection{Darstellungsformel für Lösungen der Poissongleichung} 
\label{sub:darstellungsformel_fur_losungen_der_poissongleichung}

\begin{satz}[Greensche Darstellungsformel]
	Sei $\Omega $ eine offene, beschränkte Teilmenge des $\mathbb{R}^n$ mit $C^1$-Rand und $h \in C^2(\Omega) \cap C^1(\bar{\Omega})$ mit $ \Delta h \in L^1(\Omega)$.
	Dann gilt für alle $x \in \Omega$:
	\begin{equation}
		h(x) = - \int_{\Omega}^{} \Phi(x-y) \Delta h(y) \,\mathrm{d}y 
		+ \int_{\partial \Omega}^{} \left( \Phi(x-y)  \nabla h(y) - h(y)  \nabla_x \Phi(x-y) \right) \cdot \nu(y) \,\mathrm{d}S,
	\end{equation}
	wobei $\Phi$ die Fundamentallösung der Lagrange-Gleichung bezeichnet.
\end{satz}

\begin{beweis}
	Zu vorgegebenen $x \in \Omega$ betrachten wir ein fixiertes $\varepsilon < \min \set{1,\dist(x, \partial \Omega)}$.
	\[
		\Omega = B_{\varepsilon}(x) \cup (\Omega \setminus B_{\varepsilon}(x))
	\]
	Dann gilt
	\begin{align*}
		\int_{\Omega}^{} \Phi(x-y) \Delta h(y) \,\mathrm{d}y 
		&= \underset{=:A_{\varepsilon}}{\underbrace{\int_{B_{\varepsilon}(x)}^{} \Phi(x-y) \Delta h(y) \,\mathrm{d}y }}
		+ \underset{=:C_{\varepsilon}}{\underbrace{\int_{ \Omega \setminus B _{\varepsilon}(x)}^{} \Phi(x-y) \Delta h(y) \,\mathrm{d}y}}
	\end{align*}
	Wir zeigen $\abs{A _{\varepsilon}}\to 0$ für $\varepsilon \to 0$ mit
	\begin{equation}
		\abs{A_{\varepsilon}} \leq c \norm{\Delta^2 h}_{L^{\infty}(B_{\varepsilon}(x))} \int_{B_{\varepsilon}(x)}^{}\Phi(x-y) \,\mathrm{d}y
	\end{equation}
	und mit $\abs{y} \leq \varepsilon$ gilt
	\begin{align}
		\int_{B_{\varepsilon}(x)}^{}\Phi(x-y) \,\mathrm{d}y = \int_{B_{\varepsilon}(0)}^{} \Phi(y) \,\mathrm{d}y &= \begin{cases}
			c \int_{B_{\varepsilon}(0)}^{} -\lg(y) \,\mathrm{d}y, &\text{ falls } n =2\\
			c \int_{B_{\varepsilon}(0)}^{} \abs{y}^{2n} \,\mathrm{d}y , &\text{ falls } n \geq 2
			\end{cases} \\
			&= \begin{cases}
				c \abs{\lg \varepsilon} \varepsilon^2, &\text{ falls }n=2\\
				c \varepsilon^{2-n} \varepsilon^n, &\text{ falls }n \geq 2\\
			\end{cases}.
	\end{align}
	Daraus folgt dann \[
		A_{\varepsilon} \to 0 \qquad \text{für }\varepsilon \to 0.
	\]
	$c$ ist hier eine von $\varepsilon$ unabhängige Konstante. \\
	Als nächstes betrachten wir $C_{\varepsilon}$. Mit Hilfe der Greenschen Formel 
	\[
		\int_{U}^{} ( u \Delta v - v \Delta u) \,\mathrm{d}y = \int_{\partial U}^{}(u  \nabla  v - v  \nabla u) \cdot \nu \,\mathrm{d}S
	\]
	gilt mit $U = \Omega \setminus B _{\varepsilon}(x)$,$u(y) = \Phi(x-y)$ und $v(y)= h(y)$
	\begin{align*}
		C _{\varepsilon} &= \int_{\Omega \setminus B _{\varepsilon}(x)}^{} \Phi(x-y) \Delta h(y)  \,\mathrm{d}y
		 - \int_{\Omega \setminus B _{\varepsilon}(x)}^{} h(y) \underset{=0, x \neq y}{\underbrace{\Delta_y \Phi(x-y)}} \,\mathrm{d}y \\
		 &= \int_{\partial ( \Omega \setminus B _{\varepsilon}(x))}^{} \left( \Phi(x-y)  \nabla h(y) - h(y)  \nabla_y \Phi(x-y) \right) \cdot \nu(y) \,\mathrm{d}S \\
		 &= \int_{\partial \Omega}^{} (\Phi(x-y)  \nabla h(y) - h(y)  \nabla_y \Phi(x-y)) \cdot \nu(y) \,\mathrm{d}S \\
		 & \qquad \qquad - \underset{D_{\varepsilon}}{\underbrace{\int_{\partial B_{\varepsilon}(x)}^{} \Phi(x-y)  \nabla h(y) \cdot \nu(y) \,\mathrm{d}S}} \\
		 & \qquad \qquad + \underset{E_{\varepsilon}}{\underbrace{\int_{\partial B_{\varepsilon}(x)}^{} h(y)  \nabla_y \Phi(x-y) \cdot \nu(y) \,\mathrm{d}S}}
	\end{align*}
	Nun gilt es zu zeigen, dass
	\begin{enumerate}[(i)]
		\item $\abs{D_{\varepsilon}} \to 0$ für $\varepsilon \to 0$
		\item $E_\varepsilon \to -h(x)$ für $\varepsilon \to 0$
	\end{enumerate}
	\begin{beweis}
		\begin{enumerate}[(i)]
			\item Es gilt
			\begin{align*}
				\abs{D_{\varepsilon}} &\leq \norm{  \nabla h}_{L^{\infty}(B_1(x))} \int_{ \partial B_{\varepsilon}(x)}^{} \Phi(y) \,\mathrm{d}S \\ &\leq \begin{cases}
					c \norm{ \nabla h}_{L^{\infty}(B_1(x))} \int_{\partial B_{\varepsilon}(0)}^{} 
					- \lg \abs{y} \,\mathrm{d}S \leq C \abs{ \lg \varepsilon}\varepsilon, &\text{ falls }n=2\\
					c \norm{ \nabla h}_{L^{\infty}(B_1(x))} \int_{\partial B_{\varepsilon}(0)}^{} \abs{y}^{2-n} \,\mathrm{d}S 
					\leq  c \varepsilon^{2-n} \varepsilon^{n-1} = c \varepsilon , &\text{ falls } n \geq 2.
				\end{cases}
			\end{align*}
			Insgesamt also $\abs{D_{\varepsilon}} \to 0$ für $\varepsilon \to 0$
			\item Es gilt
			\[
				\Phi(y) = \begin{cases}
					-\frac{1}{2 \omega_2} \lg \abs{y}, &\text{ falls }n=2\\
					\frac{1}{n(n-2)\omega_n} \abs{y}^{2-n} , &\text{ falls } n \geq 3.
				\end{cases}
			\]
			und für $n \geq  3$ außerdem
			\begin{equation}
				 \nabla \Phi(y) = - \frac{1}{n \omega_n} \abs{y}^{1-n} \frac{y}{\abs{y}}.
			\end{equation}
			Damit
			\begin{equation}
				 \nabla_y \Phi(x-y) = \frac{1}{n \omega_n} \abs{x-y}^{1-n} \frac{x-y}{\abs{x-y}} = \frac{1}{n \omega_n} \frac{x-y}{\abs{x-y}^n}.
			\end{equation}
			Für $\nu$ gilt
			\begin{equation}
				\nu(y) = \frac{y-x}{\abs{y-x}}  
			\end{equation}
			und somit für $y \in \partial B_{\varepsilon}(x)$
			\begin{equation}
				 \nabla_y \Phi(x-y) \cdot \nu(y) = - \frac{1}{n \omega_n} \frac{1}{\abs{x-y}^{n-1}} =- \frac{1}{n \omega_n} \frac{1}{\varepsilon^{n-1}}.
			\end{equation}
			Für $E_{\varepsilon}$ gilt damit folgt aus der Stetigkeit von $h$
			\begin{equation}
				E _{\varepsilon} = - \frac{1}{n \omega_n \varepsilon^{n-1}} \int_{\partial B_{\varepsilon}(x)}^{} h(y) \,\mathrm{d}S(y) 
				= - \fint_{\partial B_{\varepsilon}(x)}^{} -h(y) \,\mathrm{d}S(y) \stackrel{\varepsilon \to 0}{\to } -h(x)
			\end{equation}
		\end{enumerate}
	\end{beweis}
	Insgesamt folgt die Behauptung des Satzes.
\end{beweis}	

\begin{korollar}
	Sei $f \in C^2_0(\mathbb{R}^n)$ und $u : \mathbb{R}^n \to \mathbb{R}$ gegeben durch 
	\[
		u(x) = ( \Phi * f)(x) := \int_{\mathbb{R}^n}^{} \Phi(x-y) f(y) \,\mathrm{d}y =\int_{\mathbb{R}^n}^{} \Phi(y)f(x-y) \,\mathrm{d}y.
	\]
	Dann gilt $u \in C^2(\mathbb{R}^n)$ und $- \Delta u = f$ in $\mathbb{R}^n$.
\end{korollar}
\begin{beweis}
	Zunächst zeigen wir $u \in C^2(\mathbb{R}^n)$. 
	$\Phi(y)f(x-y)$ ist stetig differenzierbar nach $x_i$ für $i=1,\dots,n$ und $y \neq x$, weil $f \in C^2_0(\mathbb{R}^n)$.
	\begin{equation}
		\diff{}{x_i} \left( \Phi(y)f(x-y) \right) = \Phi(y) \diff{f(x-y)}{x_i}
	\end{equation}
	ist integrierbar, weil $f$ einen kompakten Träger besitzt und $\Phi \in L^1_{\text{loc}}(\mathbb{R}^n)$. Damit erhalten wir für alle $i=1,\dots,n$
	\begin{equation}
		\diff{u}{x_i}(x) = \int_{\mathbb{R}^n}^{} \Phi(y) \diff{f(x-y)}{x_i} \,\mathrm{d}y. 
	\end{equation}
	Analog gilt für alle $i,j=1,\dots,n$
	\begin{equation}
		\diff{^2u}{x_i \partial x_j} (x) = \int_{\mathbb{R}^n}^{}\Phi(y) \diff{^2f(x-y)}{x_i \partial x_j} \,\mathrm{d}y.
	\end{equation}
	und damit $u \in C^2(\mathbb{R}^n)$. Für die zweite Behauptung folgern wir
	\begin{align}
		\Delta u(x) &= \int_{\mathbb{R}^n}^{} \Phi(y) \Delta_x f(x-y) \,\mathrm{d}y \\ 
		&= \int_{\mathbb{R}^n}^{} \Phi(y) \Delta_y f(x-y) \,\mathrm{d}y \\
		&= \int_{\mathbb{R}^n}^{} \Phi(x-y)\Delta f(y) \,\mathrm{d}y
	\end{align}
	Wir wollen nun $- \Delta u = f$ in $\mathbb{R}^n$ zeigen. Dafür bemerken wir zunächst, dass wegen $f \in C^2_0(\mathbb{R}^n)$ auch $f \in C_0^2(B_R(0))$ für 
	ein hinreichend großes $R > 0$ gilt. Mit Satz 2.19 erhalten wir 
	\begin{equation}
		- f(x) = \underset{= \Delta u(x)}{\underbrace{\int_{B_R(0)}^{} \Phi(x-y) \Delta f(y) \,\mathrm{d}y }} 
	\end{equation}
	Die Randintegrale verschwinden wegen der kompakten Träger für $f$ und $  \nabla f$. 
\end{beweis}

%%%%% 02.05.2016

\begin{bemerkung}
	\begin{enumerate}[(i)]
		\item Für $n \geq3$ ist $u$ beschränkt und $\lim_{\abs{x} \to \infty}u(x)=0$.
		\item Für $n=2$ ist $u$ potentiell und beschränkt.
		\item Jede andere beschränkte Lösung der Poisson-Gleichung auf $\mathbb{R}^n$ unterscheidet sich nur für eine additive Konstante.
	\end{enumerate}
	\begin{beweis}
		\begin{enumerate}[(i)]
			\item Sei $n \geq  3$ und $\Phi(y)= \frac{1}{n(n-2)\omega_n}\frac{1}{\abs{y}^{n-2}}$ für $y \neq 0$ die Fundamentallösung der Poisson-Gleichung. 
			Da $f \in C^2_0(\mathbb{R}^n)$ und somit $ \supp(f) \subset B_R(0)$ für ein hinreichend großes $R > 0$. Wegen $B_R(x) \subset B_{2R}(0)$ gilt
			\begin{align*}
				\abs{u(x)} &= \abs{ \int_{B_R(0)}^{} \Phi(x-y)f(y) \,\mathrm{d}}y \\
				& \leq \norm{f}_{L^{\infty}(\mathbb{R}^n)} \int_{B_R(0)}^{}\abs{\Phi(x-y)} \,\mathrm{d}y \\
				&= \norm{f}_{L^{\infty}(\mathbb{R}^n)} \int_{B_R(x)}^{} \abs{\Phi(y)} \,\mathrm{d}y \\
				& \leq \norm{f}_{L^{\infty}(\mathbb{R}^n)} \underset{\leq M}{\underbrace{\int_{B_{2R}(0)}^{} \abs{\Phi(y)} \,\mathrm{d}y}} \\
				& \leq C
			\end{align*}
			für $x \in B_R(0)$. Für $x \in \mathbb{R}^n \setminus B_R(0)$ gilt mit $y \in B_R(0)$ und $\abs{x}> R$
			\begin{equation}
				\abs{x-y} \geq \abs{x} - \abs{y} > \abs{x} - R > 0 
			\end{equation}
			und damit
			\begin{align*}
				\abs{u(x)} &\leq \norm{f}_{L^{\infty}(\mathbb{R}^n)} \int_{B_R(0)}^{} \abs{ \Phi(x-y)} \,\mathrm{d}y \\
				&\leq c \norm{f}_{L^{\infty}(\mathbb{R}^n)} \int_{B_R(0)}^{} \frac{1}{\abs{x-y}^{n-2}} \,\mathrm{d}y \\
				& \leq  c \norm{f}_{L^{\infty}(\mathbb{R}^n)} \int_{B_R(0)}^{} \frac{1}{\left( \abs{x}-R \right)^{n-2}} \,\mathrm{d}y \\
				& = c \norm{f}_{L^{\infty}(\mathbb{R}^n)} \frac{1}{(\abs{x}-R)^{n-2}} \stackrel{\abs{x}\to \infty}{\to} 0 .
			\end{align*}
			\item Sei nun $n=2$. Beachte, dass nun für $y \neq 0$
			\[
				\Phi(y) = - \frac{1}{2R} \lg \abs{y}.
			\] Ist $f$ beispielsweise $f \leq 0$ mit $f \leq -1$ in $B_1(0)$ und $\supp (f) \subset B_2(0)$ so gilt $\Phi(y) \leq 0$ für $\abs{y}> 1$ 
			Außerdem gilt für $\abs{x} > 3$
			\[
				\abs{x-y} \geq \abs{x}-\abs{y} > 3- 2 = 1
			\]
			\begin{equation}
				\Rightarrow \Phi(x-y) \leq 0
			\end{equation}
			\begin{equation}
				\Rightarrow \Phi(x-y)f(y) \geq 0 \qquad \text{ für } x \in \mathbb{R}^n \setminus B_3(0), y \in B_2(0).
			\end{equation}
			Damit wegen $\abs{x-y} > \abs{x}-1$
			\begin{align*}
					u(x) &= \int_{B_2(0)}^{} \Phi(x-y) f(y) \,\mathrm{d}y \\
					& \geq \int_{B_1(0)}^{} \Phi(x-y)f(y) \,\mathrm{d}y \\
					& \geq  \int_{B_1(0)}^{} \abs{\Phi(x-y)} \,\mathrm{d}y \\
					& \geq  \int_{B_1(0)}^{} c \lg ( \abs{x}-1) \,\mathrm{d}y \\
					&= c \lg(\abs{x}-1),
			\end{align*}
			also insgesamt für $\abs{x} \to \infty$
			\begin{equation}
				u(x) \to  \infty.
			\end{equation}
		\item Wir nehmen an, dass $u_1,u_2$ zwei beschränkte Lösungen der Poisson-Gleichung sind. 
		Die Funktion $u_1-u_2$ ist beschränkt und harmonisch in $\mathbb{R}^n$ und daher nach dem Satz von Liouville konstant. Also ist 
		\begin{equation}
			u_1 = u_2 + \text{Konstante}.
		\end{equation}
		\end{enumerate}
	\end{beweis}
\end{bemerkung}

Wir betrachten nun wieder die Poisson-Gleichung mit Dirichlet-Randbedingungen.
\begin{align*}
	\begin{cases}
		- \Delta u = f, &\text{ falls }x \in \Omega\\
		u =g, &\text{ falls } x \in \partial \Omega,
	\end{cases}
\end{align*}
wobei $\Omega \subseteq \mathbb{R}^n$ offen, beschränkt und mit $C^1$-Rand. Außerdem seien $f,g$ hier reguläre Funktionen.
Wir wollen eine Darstellungsformel für die Lösung finden. Wir haben bereits bewiesen, dass diese Gleichung höchstens eine Lösung hat.

\begin{definition}[Greensche Funktion für $\Omega$]
	Sei $\Omega$ eine offene Menge in $\mathbb{R}^n$. 
	Eine Funktion $G : \set[(x,y) \in \Omega \times \Omega]{x \neq y} \to \mathbb{R}$ heißt Greensche Funktion für $\Omega$, falls für alle $x \in \Omega$ gilt:
	\begin{enumerate}[(i)]
		\item $y \to G(x,y)- \Phi(x-y)$ ist von der Klasse $C^2(\Omega) \cap C^1(\bar{\Omega})$ und harmonisch in $\Omega$,
		\item $y \to G(x,y)$ hat Nullrandwerte auf $\partial \Omega$, d.h. es gilt 
		\[
			\lim_{y \to y_0}G(x,y) = 0 \qquad \text{ für } y_0 \in \partial \Omega.
		\]
		(und bei unbeschränktem $\Omega$ auch für $y_0 = \infty$)
	\end{enumerate}
\end{definition}
\begin{bemerkung}
	\begin{enumerate}[1)]
		\item $y \to G(x,y) - \Phi(x-y)$ heißt die Korrektorfunktion. 
		Ist $\Omega$ beschränkt, so ist nach der Eindeutigkeit der Lösung der Poissongleichung die Korrektorfunktion 
		(und damit auch die Greensche Funktion) eindeutig, falls sie existiert.
		\begin{equation}
			\Delta_y ( G(x,y) - \Phi(x-y)) = 0 \qquad  \Rightarrow \qquad \Delta_y G(x,y) = \Delta_y \Phi(x-y) = 0 \qquad \text{ für }x \neq y
		\end{equation}
		Also ist $y \to G(x,y)$ harmonisch. Es gilt außerdem
		\[
			G(x,y) - \Phi(x-y)  \Big|_{\partial \Omega}^{} = - \Phi(x-y)  \Big|_{\partial \Omega}^{}
		\]
		\item $y \to G(x,y)$ ist von der Klasse $C^2(\Omega \setminus \set{x}) \cap C^1( \bar{\Omega} \setminus \set{x})$ und besitzt die gleiche Singularität in $x$
		wie $\Phi$.
	\end{enumerate}
\end{bemerkung}

\begin{satz}[Greensche Lösungsformel]
	Sei $\Omega$ eine beschränkte offene Menge in $\mathbb{R}^n$ mit $C^1$-Rand und $G$ die Greensche Funktion für $\Omega$ (falls existent). 
	Ist $u \in C^2(\Omega) \cap C^1(\bar{\Omega})$ eine Lösung der Poissongleichung. 
	Für Funktionen $f \in C^{\infty}(\Omega) \cap L^1(\Omega)$ und $g \in C^0(\partial \Omega)$ gilt für alle $x \in \Omega$
	\[
		u(x) = \int_{\Omega}^{} G(x,y)f(y) \,\mathrm{d}y - \int_{\partial \Omega}^{}  \nabla_y G(x,y) g(y) \cdot \nu(y) \,\mathrm{d}S(y).
	\]
\end{satz}
\begin{beweis}
	Wir benutzen die Greensche Formel und Satz 2.19. Die Greensche Formel wenden wir auf die harmonische Funktion $y \to G(x,y) - \Phi(x-y)$ und die Lösung $u$ an. 
	Sei $v(y):= G(x,y) - \Phi(x-y)$ und $w(y)= u(y)$. Dann gilt
	\begin{align*}
		\int_{\Omega}^{}(G(x,y)-\Phi(x-y)) \Delta u \,\mathrm{d}y 
		&= \int_{\partial \Omega}^{} \left[ (G(x,y)- \Phi(x-y))  \nabla u - u  \nabla_y (G(x,y)-\Phi(x-y)) \right] \cdot \nu(y)  \,\mathrm{d}S(y) \\
		& \qquad \qquad - \int_{\Omega}^{} G(x,y)f(y) \,\mathrm{d}y - \int_{\Omega}^{} \Phi(x-y) \Delta u(y) \,\mathrm{d}y  \\
		& = - \int_{\partial \Omega}^{} \left[ \Phi(x-y)  \nabla u(y) + u(y)  \nabla_y \Phi(x-y) \right] \cdot \nu(y)  \,\mathrm{d}S(y) \\
		& \qquad \qquad -\int_{\partial \Omega}^{} g  \nabla_y G(x,y) \cdot \nu(y)  \,\mathrm{d}S(y).		
	\end{align*}
	Insgesamt folgt die Behauptung.
\end{beweis}
Wir wollen nun zeigen, dass die Funktion $G$ symmetrisch ist, also $G(x,y)=G(y,x)$. Dies führt dann zu folgendem Resultat.
$y \to G(x,y)$ ist harmonisch in $\Omega \setminus \set{x}$ und somit $ \Delta_y G(x,y)=0$. Dann ist auch $x \to G(x,y)$ harmonisch in $\Omega \setminus \set{y}$.

\begin{lemma}
	Sei $\Omega$ eine offene Teilmenge des $\mathbb{R}^n$, $G$ die Greensche Funktion für $\Omega$ und $B_r(x) \subset \subset \Omega$. 
	Ist $h \in C^2(B_r(x)) \cap C^1( \overline{B_r(x)})$, so gilt
	\begin{equation}
		\lim_{\varepsilon \to 0} \int_{\partial B _{\varepsilon}(x)}^{} \left( G(x,y)  \nabla h(y) - h(y)  \nabla_y G(x,y) \right) \cdot \nu(y) \,\mathrm{d}S(y) =
		h(x).
	\end{equation}
\end{lemma}
\begin{beweis}
	Aus dem Beweis von Satz von 2.19 folgt
	\begin{align*}
		\lim_{\varepsilon \to 0} \underset{-D_{\varepsilon}- E_{\varepsilon}}{\underbrace{\int_{\partial B_{\varepsilon}(x)}^{}\left( \Phi(x-y) 
		 \nabla h(y) - h(y)  \nabla_y \Phi(x-y) \right) \cdot \nu(y) \,\mathrm{d}S(y)}}=h(x).
	\end{align*}
	$h$ und die Korrektorfunktion $G(x,y)-\Phi(x-y)$ sind regulär.
	\begin{align}
		\int_{ \partial B_{\varepsilon}(x)}^{} &\left( \left( \underset{ C^1(\overline{B_r(x)})}{\underbrace{G(x,y)- \Phi(x-y)}} \right) 
		 \underset{ C^0(\overline{B_r(x)})}{\underbrace{\nabla h(y) }} - \underset{ C^1(\overline{B_r(x)})}{\underbrace{h(y)}} 
		  \underset{ C^0(\overline{B_r(x)})}{\underbrace{\nabla_y \left( G(x-y) -\Phi(x-y) \right)}} \right) \cdot \nu(y) \,\mathrm{d}S(y) \\
		   &\leq M S_n \varepsilon^{n-1} \stackrel{\varepsilon \to 0}{\to } 0
	\end{align}
\end{beweis}

\begin{satz}[Symmetrie der Greenschen Funktion]
	Ist $G$ die Greensche Funktion zu einer beschränkten, offenen Menge $\Omega \subseteq \mathbb{R}^n$ mit $C^1$-Rand, so gilt 
	\[
	G(x,y)=G(y,x)  	
	\]  
	für alle $x,y \in \Omega$ mit  $x \neq y$.
\end{satz}
\begin{beweis}
	Für feste $x,y \in \Omega$ mit $x \neq y$ definieren wir die Hilfsfunktionen 
	\begin{align*}
		v(z) &= G(x,z), \qquad z \in \bar{\Omega} \setminus \set{x} \\
		w(z) &= G(y,t), \qquad z \in \bar{\Omega} \setminus \set{y}
	\end{align*}
	Dann gilt
	\begin{align*}
		\Delta v(z) &= 0 \qquad \text{ in } \Omega \setminus \set{x}, \qquad v  \Big|_{\partial \Omega}^{}= 0 \\
		\Delta w(z) &= 0 \qquad \text{ in } \Omega \setminus \set{y}, \qquad w  \Big|_{\partial \Omega}^{}= 0 
	\end{align*}
	Für $\varepsilon < \min \set{ \dist(x, \partial \Omega) , \dist(y, \partial \Omega), \frac{\dist(x,y)}{2} }$ folgt
	\[
		\overline{B_\varepsilon(x)} \cup \overline{B_\varepsilon(y)} \subseteq \Omega \qquad
		 \qquad \overline{B_\varepsilon(x)} \cup \overline{B_\varepsilon(y)} = \emptyset 
	\]
	Die Greensche Formel angewandt auf die Menge $\Omega \setminus \overline{B_\varepsilon(x)} \cup \overline{B_\varepsilon(y)}$ liefert
	\begin{align*}
		0 &= \int_{\Omega \setminus (\overline{B_\varepsilon(x)} \cup \overline{B_\varepsilon(y)})}^{} (v \Delta w - w \Delta v) \,\mathrm{d}z  \\
		&= \int_{\partial ( \Omega \setminus \overline{B_\varepsilon(x)} \cup \overline{B_\varepsilon(y)})}^{} (v(z)  \nabla w(z)- w(z)  \nabla v(z))\cdot \nu(z)
		 \,\mathrm{d}S(z)\\ 
		& \stackrel{*}{=} \int_{\partial B_\varepsilon(x) }^{} (v(z)  \nabla w(z)- w(z)  \nabla v(z))\cdot \nu(z)
		 \,\mathrm{d}S(z) \\
		& \qquad + \int_{\partial B_\varepsilon(y) }^{} (v(z)  \nabla w(z)- w(z)  \nabla v(z))\cdot \nu(z)
		 \,\mathrm{d}S(z) \\
		&= \int_{\partial B_\varepsilon(x) }^{} (G(x,z)  \nabla w(z)- w(z)  \nabla G(x,z))\cdot \nu(z)
		 \,\mathrm{d}S(z) \\
		& \qquad + \int_{\partial B_\varepsilon(y) }^{} (v(z)  \nabla G(y,z)- G(y,z)  \nabla v(z))\cdot \nu(z)
		 \,\mathrm{d}S(z) \\
		 &= w(x)- v(y) \qquad \text{ für } \varepsilon \to  0
		\end{align*}
		Es folgt
		\[
			0 = w(x)- v(y)
		\]
		Also \[
			w(x)= G(y,x) = v(y) = G(x,y)
		\]
\end{beweis}
\cleardoubleoddemptypage
\pagenumbering{Alph}
\setcounter{page}{1}

\end{document}