%!TEX TS-program = xelatex
%skript für Die Vorlesung Partielle Differentialgleichungen
\newcommand{\Semester}{SoSe 2016}
\newcommand{\fach}{Partielle Differentialgleichungen}
\newcommand{\prof}{Prof.\ Zeppieri}

\input{../!config/VorlagenTim/preambel.tex}

\numberwithin{equation}{section}
\numberwithin{figure}{section}

\begin{document}

\maketitle
\cleardoubleoddemptypage

\pagenumbering{Alph}
\section*{Vorwort --- Mitarbeit am Skript}
Dieses Dokument ist eine Mitschrift aus der Vorlesung \enquote{\fach, \Semester}, gelesen von \prof. 
Der Inhalt entspricht weitestgehend dem Tafelanschrieb. 
Für die Korrektheit des Inhalts werde ich keinerlei Garantie übernehmen. Dieses Skript wird allerdings zusätzlich von \prof \, Korrektur gelesen. 
Für Bemerkungen und Korrekturen -- und seien es nur Rechtschreibfehler -- bin ich sehr dankbar. 
Bitte durch persönliches Ansprechen oder per Mail an \href{mailto:keil.menden@web.de}{\nolinkurl{keil.menden@web.de}}.  

\newpage

\tableofcontents
\cleardoubleoddemptypage
\pagenumbering{arabic}
\setcounter{page}{1}

%%%% lecture until 25.04.2016 %%%%%
\input{splits/pde1.tex}

%%%%%%% lecture 25.04.16 %%%%%%%%
%%%%% lecture 25.04.2016

\begin{definition*}[$\varepsilon$-Glättung]
	Sei $\Omega$ eine offene Teilmenge im $\mathbb{R}^n$, $\varepsilon >0$ und 
	\[
		\Omega_{\varepsilon} := \set[x \in \Omega]{\dist(x,\partial \Omega)> \varepsilon}.
	\]
	Für $f \in L^1_{\text{loc}}(\Omega)$ definiert man die $\varepsilon$-Glättung $f \varepsilon$ von $f$ als die Faltung von $f$ mit dem Glättungskern $\eta_{\varepsilon}$, also
	\[
		f_{\varepsilon}(x):= \eta_{\varepsilon} * f := \int_{\Omega}^{}\eta_{\varepsilon}(x-y)f(y) \,\mathrm{d}y \qquad \text{für alle }x \in \Omega_{\varepsilon}.
	\]
\end{definition*}

\begin{bemerkung}[Eigenschaften von $\varepsilon$-Glättungen]
	\begin{enumerate}[1)]
		\item $f_{\varepsilon} \in C^{\infty}(\Omega _{\varepsilon})$ mit \\
		$D^{\alpha}f _{\varepsilon}(x)= \int_{\Omega}^{}D^{\alpha} \eta_{\varepsilon}(x-y)f(y) \,\mathrm{d}y$ für beliebige Multiindizes $\alpha \in \mathbb{N}_0^n$.
		\item $f_{\varepsilon} \to f$ fast überall in $\Omega$.
		\item $f_{\varepsilon} \to f$ in $L^1_{\text{loc}}(\Omega)$
	\end{enumerate}
\end{bemerkung}

\begin{satz}
	Sei $\Omega$ eine offene Teilmenge des $\mathbb{R}^n$ und $u \in C^0(\Omega)$ eine Funktion, die die sphärische Mittelwerteigenschaft erfüllt, d.h.
	\begin{equation}
		u(x_0) = \fint_{\partial B_r(x_0)}^{} u\,\mathrm{d}S \qquad \forall\, B_r(x_0) \subset \subset \Omega. 
	\end{equation}
	Dann gilt
	\begin{equation}
		u(x_0) = u_{\varepsilon}(x_0) \qquad \forall\, x_0 \in \Omega \text{ und } \varepsilon < \dist(x_0,\partial \Omega),
	\end{equation}
	wobei $u_{\varepsilon}$ die $\varepsilon$-Glättung bezeichnet. Insbesondere ist $u$ also von der Klasse $C^{\infty}(\Omega)$ und harmonisch auf $\Omega$.
\end{satz}
\begin{beweis}
	$x_0 \in \Omega$, $\varepsilon < \dist(x_0, \partial \Omega)$. \\
	Wegen der Radialsymmetrie von $\eta _{\varepsilon}$ und weil $\eta _{\varepsilon}(x_0-x)= \text{konstant}$ für alle $x \in \partial B_r(x_0)$ gilt
	\begin{align*}
		u_{\varepsilon}(x_0) &\stackrel{\hphantom{\substack{\text{Mittelwert-} \\ \text{Eigenschaft}}}}{=} \int_{B_{\varepsilon}(x_0)}^{}\eta_{\varepsilon}(x_0-y) u(y) 
		\,\mathrm{d}y \\
		&\stackrel{\hphantom{\substack{\text{Mittelwert-} \\ \text{Eigenschaft}}}}{=} \int_{0}^{\varepsilon} \int_{\partial B_r(x_0)}^{} \eta_{\varepsilon}(x_0-y)u(y)
		 \,\mathrm{d}S(y) \,\mathrm{d}r \\
		&\stackrel{\hphantom{\substack{\text{Mittelwert-} \\ \text{Eigenschaft}}}}{=} \int_{0}^{\varepsilon} 
		\left(\eta_{\varepsilon}\underset{=r}{\underbrace{(x_0-y)}} \cdot \int_{\partial B_r(x_0)}^{}u(y) \,\mathrm{d}S(y) \right) \,\mathrm{d}r \\
		&\stackrel{\hphantom{\substack{\text{Mittelwert-} \\ \text{Eigenschaft}}}}{=} \int_{0}^{\varepsilon} \left( \eta_{\varepsilon}(x_0-y)S_nr^{n-1} \cdot 
		\underset{=u(x_0)}{\underbrace{\fint_{\partial B_r(x_0)}^{}u(y) \,\mathrm{d}S(y)}} \right) \,\mathrm{d}r \\
		& \stackrel{\substack{\text{Mittelwert-} \\ \text{Eigenschaft}}}{=} \int_{0}^{\varepsilon}u(x_0) \cdot \int_{\partial B_r(x_0)}^{} \eta_{\varepsilon}(x_0-y)
		\,\mathrm{d}S(y) \,\mathrm{d}r \\
		&\stackrel{\hphantom{\substack{\text{Mittelwert-} \\ \text{Eigenschaft}}}}{=} u(x_0)
		\underset{=1}{\underbrace{\int_{B_{\varepsilon}(x_0)}^{} \eta_{\varepsilon}(x_0-x) \,\mathrm{d}x}} \\
		&\stackrel{\hphantom{\substack{\text{Mittelwert-} \\ \text{Eigenschaft}}}}{=} u(x_0)
	\end{align*}
\end{beweis}

\begin{bemerkung}
	Der Satz macht keine Annahme über die Randwerte von $u$, diese müssen nicht glatt sein (und können sogar unstetig sein).
\end{bemerkung}

Die Aussage von Satz 2.10 bleibt gültig, wenn die Funktion $u$ die Mittelwerteigenschaft auf Kugeln erfüllt.

\begin{korollar}
	Sei $\Omega \subseteq \mathbb{R}^n$ offen und $u \in C^0(\Omega)$ eine Funktion, die die Mittelwerteigenschaft auf Kugeln erfüllt, d.h.
	\begin{equation}
		u(x_0) = \fint_{B_r(x_0)}^{}u \,\mathrm{d}x \qquad \forall\, B_r(x_0) \subset \subset \Omega.
	\end{equation}
	Dann gilt
	\begin{equation}
		u(x_0) = u_{\varepsilon}(x_0) \qquad \forall\, x_0 \in \Omega, \varepsilon < \dist(x_0,\partial \Omega).
	\end{equation}
\end{korollar}

\begin{beweis}
	Es genügt zu zeigen, dass auch in diesem Fall die sphärische Mittelwerteigenschaft erfüllt ist. 
	Dazu definieren wir für eine beliebige Kugel $B_r(x_0) \subset \subset \Omega$ die Funktion $\psi: (0,r) \to \mathbb{R}$ mit
	\begin{equation}
		\psi(\rho) := \int_{\partial B_{\rho}(x_0)}^{} (u(x)-u(x_0)) \,\mathrm{d}S(x)
	\end{equation}
	$\psi$ ist stetig, weil $u$ stetig ist. 
	Außerdem gilt mit $x= x_0 + \rho y$ für alle $R \in (0,r)$
	\begin{equation}
		\int_{0}^{R} \psi(\rho) \,\mathrm{d}\rho = \int_{0}^{R} \int_{\partial B_1(0)}^{} (u(x_0+\rho y)-u(x_0)) \rho^{n-1} \,\mathrm{d}S \,\mathrm{d}\rho
		= \int_{B_R(x_0)}^{}(u(x)-u(x_0)) \,\mathrm{d}x = 0.
	\end{equation}
	Damit muss $\psi \equiv 0$ auf $(0,r)$ gelten und somit
	\[
		u(x_0) = \fint_{\partial B_r(x_0)}^{} u(x) \,\mathrm{d}S(x)
	\]
\end{beweis}
Eine einfache Folgerung ist nun, dass Harmonizität unter gleichmäßiger Konvergenz erhalten bleibt.

\begin{korollar}[Konvergenzsatz von Weierstraß]
	Sei $\Omega \subseteq \mathbb{R}^n$ offen und zusammenhängend. Sei $(u_k)_{k \in \mathbb{N}}$ eine Folge harmonischer Funktionen auf $\Omega$, die (lokal) gleichmäßig gegen eine Funktion $u$ konvergiert. Dann ist $u$ harmonisch auf $\Omega$.
\end{korollar}

\begin{beweis}
	Für jede Kugel $B_r(x_0) \subset \subset \Omega$ gilt 
	\[
		u(x_0)= \lim_{k \to \infty}u_k(x_0) = \lim_{k \to \infty} \fint_{B_r(x_0)}^{}u_k \,\mathrm{d}x = \fint_{B_r(x_0)}^{}u \,\mathrm{d}x.
	\]
	Dies gilt, weil $u$ stetig ist (gleichmäßiger Limes stetiger Funktionen). 
	Somit erfüllt $u$ die Mittelwerteigenschaft auf Kugeln und mit Korollar 2.11 folgt dann die Behauptung.
\end{beweis}

\begin{korollar}[Harnack'scher Konvergenzsatz]
	Sei $\Omega \subseteq \mathbb{R}^n$ offen und zusammenhängend. Sei $(u_k)_{k \in \mathbb{N}}$ eine monoton wachsende Folge harmonischer Funktionen. 
	Gibt es ein $x_0 \in \Omega$, so dass $(u_k(x_0))_{k \in \mathbb{N}}$ beschränkt (und damit konvergent) ist, so konvergiert $(u_k)_{k \in \mathbb{N}}$ auf jeder
	zusammenhängenden offenen Menge $V \subset \subset \Omega$ gleichmäßig gegen eine harmonische Funktion auf $\Omega$.
\end{korollar}

\begin{beweis}
	Wegen der Monotonie ist $u_k-u_j$ für $k > j$ eine nicht-negative harmonische Funktion. Sei $V \subset \subset \Omega$ offen und zusammenhängend. 
	Sei o.B.d.A $x_0 \in V$ (andernfalls können wir $ \tilde V$ mit $V \subset \tilde V \subset \subset \Omega$ mit $x_0 \in \tilde V$ konstruieren)
	
	
	\begin{align*}
		0 &\stackrel{\hphantom{x_0 \in V}}{\leq} \sup_V(u_k-u_j) \stackrel{\text{Harnack}}{\leq} c(V) \inf_V(u_k-u_j) \\
		& \stackrel{x_0 \in V}{\leq} c(V) (u_k(x_0)-u_j(x_0)) \\
		&\stackrel{\hphantom{x_0 \in V}}{\leq} c(V) \varepsilon
	\end{align*}
	Da $(u_k(x_0))_{k \in \mathbb{N}}$ nach Voraussetzung eine Cauchy-Folge ist folgt, dass $(u_k)$ eine Cauchy-Folge bezüglich der Supremumsnorm auf $V$ ist. 
	Damit ist sie gleichmäßig konvergent auf $V$. Nach Korollar 2.12 ist ihr Limes eine harmonische Funktion. \\
	Wähle $V = B_r(x_0) \subset \subset \Omega$, dann folgt
	\begin{equation}
		u_k(x_0) = \fint_{B_r(x_0)}^{}u_k \,\mathrm{d}x 
	\end{equation}
	und wegen gleichmäßiger Konvergenz folgt dann
	\begin{equation}
	u(x_0) = \fint_{B_r(x_0)}^{}u \,\mathrm{d}x.
	\end{equation}
	Somit ist $u$ harmonisch auf $\Omega$, weil $u$ stetig ist.
\end{beweis}

\begin{satz}[Hermann Weyl]
	Sei $\Omega \subseteq \mathbb{R}^n$ offen und $u \in L^1_{\text{loc}}(\Omega)$ eine Funktion, für die 
	\begin{equation}
		\int_{\Omega}^{} u \Delta \varphi \,\mathrm{d}x = 0 \qquad \forall\, \varphi \in C^{\infty}_0(\Omega)
	\end{equation}
	erfüllt ist. 
	Dann ist $u$ harmonisch in $\Omega$ (streng genommen: dann existiert eine Funktion $\tilde u$, die harmonisch ist mit $u = \tilde u$ fast überall in $\Omega$)
\end{satz}

\begin{beweis}
	Wir zeigen zunächst, dass die Glättungen $u_{\varepsilon}$ harmonisch in $\Omega_{\varepsilon}$ sind. Nach Übung gilt für $x \in \Omega_{\varepsilon}$
	\[
		\Delta u_{\varepsilon}(x) = ( \Delta \eta_{\varepsilon} * u)(x) = \int_{\Omega}^{} \Delta \eta_{\varepsilon}(x-y)u(y) \,\mathrm{d}y.
	\]
	Nach Voraussetzung mit $\varphi(y)= \eta_{\varepsilon}(x-y) \in C^{\infty}_0(\Omega)$ folgt 
	\[
		\Delta u_{\varepsilon}(x) \equiv 0 \qquad \text{in }\Omega_{\varepsilon}.
	\]
	Es gilt für fast alle $x_0 \in \Omega$ und alle $r < \dist(x_0, \partial \Omega)$ ($B_r(x_0) \subset \subset \Omega$) 
	\begin{equation}
		u(x_0) \stackrel{\substack{\text{fast überall}\\\text{Konvergenz}\\\text{von }u_{\varepsilon}}}{=}
		\lim_{\varepsilon \to 0} u_{\varepsilon}(x_0) 
		\stackrel{\substack{\text{Mittelwerteigenschaft} \\\text{der Familie }u_{\varepsilon}}}{=} \lim_{\varepsilon \to 0} 
		\fint_{B_r(x_0)}^{}u_{\varepsilon}(x) \,\mathrm{d}x 
		\stackrel{\substack{L^1_{\text{loc}}\text{-Konvergenz}\\ \text{von }u_{\varepsilon}}}{=}
		\fint_{B_r(x_0)}^{} u(x) \,\mathrm{d}x.
	\end{equation}
	Das heißt $u$ erfüllt für fast alle $x_0 \in \Omega$ die Mittelwerteigenschaft auf Kugeln. 
	Definiert man nun $\tilde u: \Omega \to \mathbb{R}$ durch
	\[
		\tilde u(x_0):= \fint_{B_r(x_0)}^{}u \,\mathrm{d}x
	\]
	mit $r:= \frac{\dist(x_0, \partial \Omega)}{2}$, so gilt 
	\begin{equation}
		u = \tilde u
	\end{equation}
	fast überall in $\Omega$. Wegen der Absolutstetigkeit der Integrale ist $\tilde u$ stetig und erfüllt die Mittelwerteigenschaft (auf Kugeln) überall in $\Omega$.
	Damit folgt die Harmonizität von $\tilde u$ aus Korollar 2.11.
\end{beweis}
Nun werden wir lokale Abschätzungen für höhere Ableitungen harmonischer Funktionen beweisen.



%%%%%% lecture 28.04.2016 %%%%%
%%%% Lecture 28.05.2016

\begin{satz}[Innere Abschätzung für Ableitungen harmonischer Funktionen]
	Sei $\Omega \subseteq \mathbb{R}^n$ offen und $u \in C^2(\Omega)$ eine harmonische Funktion auf $\Omega$. Dann gilt für jeden Multiindex $\alpha$ mit $ \abs{\alpha}=k$ und jede Kugel $B_r(x_0) \subset \subset \Omega$
	\begin{equation}
		\abs{D^{\alpha}u(x_0)} \leq c(n,k)r^{-n-k} \norm{u}_{L^1(B_r(x_0))}
	\end{equation}
	mit Konstanten 
	\begin{align}
		c(n,k) &:= \frac{\left( 2^{(n+1)}nk \right)^k}{\omega_n} \qquad \text{für }k=1,\dots,n \\
		c(n,0) &:= \frac{1}{\omega_n}
	\end{align}
\end{satz}

\begin{beweis}
	Wir beweisen die Abschätzungen mit Induktion über $k$. 
	Beachte, dass mit $u$ auch jede Ableitung $D^{\alpha}u$ harmonisch ist (weil z.B $u \in C^{\infty}$ ist $\Delta \diff{u}{x_i}= \diff{}{x_i}\Delta u = 0$) 
	und analog für alle anderen Ableitungen.
	\begin{description}
		\item[Der Fall $k=0$:] Die Behauptung folgt sofort aus der Mittelwerteigenschaft auf Kugeln:
		\begin{equation}
			u(x_0) = \frac{1}{r^n \omega_n} \int_{B_r(x_0)}^{}u \,\mathrm{d}
		\end{equation} 
		Also 
		\begin{equation}
			\abs{u(x_0)} \leq c(u,0)r^{-n} \norm{u}_{L^1(B_r(x_0))}
		\end{equation}
		\item[Der Fall $k=1$:] Für $i=1,\dots,n$ ist wegen oben $D_iu$ harmonisch und es gilt
		\begin{align*}
			\abs{D_i u(x_0)} & \stackrel{\hphantom{\text{P.I.}}}{=} \abs{ \fint_{B_{\frac{r}{2}}(x_0)}^{} D_iu(x) \,\mathrm{d}x} \\
			& \stackrel{\hphantom{\text{P.I.}}}{=} \frac{2^n}{\omega_n r^n} \abs{\int_{B_{\frac{r}{2}}(x_0)}^{}D_i u(x) \,\mathrm{d}x} \\
			& \stackrel{\text{P.I.}}{=} \abs{\int_{ \partial B_{\frac{r}{2}}(x_0)}^{} u \nu_i\,\mathrm{d}S} \\
			& \stackrel{\hphantom{\text{P.I.}}}{\leq} \frac{2^n}{\omega_n r^n} \left( \frac{r}{2} \right)^{n-1}S_n \sup_{\partial B_{\frac{r}{2}}(x_0)}\abs{u} \\
			& \stackrel{\hphantom{\text{P.I.}}}{=} \frac{2n}{r} \sup_{\partial B_{\frac{r}{2}}(x_0)}\abs{u}
		\end{align*}
		Es gilt für alle $x \in \partial B_{\frac{r}{2}}(x_0)$, dass $B_{\frac{r}{2}}(x) \subset B_r(x_0)$ und somit erhalten wir aus dem ersten Schritt
		\begin{equation}
			\abs{u(x)} \leq \frac{\left( \frac{r}{2} \right)^{-n}}{\omega_n} \norm{u}_{L^1(B_{\frac{r}{2}}(x))} 
			\leq \frac{\left( \frac{r}{2} \right)^{-n}}{\omega_n} \norm{u}_{L^1(B_r(x_0))},
		\end{equation}
		für alle $x \in \partial B_{\frac{r}{2}}(x_0)$ und damit
		\begin{equation}
			\sup_{x \in \partial B_{\frac{r}{2}}(x_0)}\abs{u(x)} \leq \frac{2^n}{r^n \omega_n} \norm{u}_{L^1(B_r(x_0))}.
		\end{equation}
		Insgesamt folgt die Behauptung mit
		\begin{equation}
			\abs{D_iu(x_0)}
			 = \underset{\frac{2^{n+1}n}{\omega_n}= c(n,1)}{\underbrace{\frac{2n}{r} \frac{\left( \frac{r}{2} \right)^{-n}}{\omega_n}}} \norm{u}_{L^1(B_r(x_0))}
		\end{equation}
		\item[Der Fall $k \geq 2$:]Ähnlich wie im Fall $k=1$ kann man die Ableitung der Ordnung $k$ durch das Supremum einer Ableitung der Ordnung $k-1$ auf einer
		kleinere Kugel abschätzen.
		So erhalten wir
		\[
			\abs{D^{\alpha}(x_0)} \leq  \frac{kn}{r} \sup_{\partial B_{\frac{r}{k}}(x_0)} \abs{D^{\beta}u}
		\]
		für einen Multiindex $\beta$ mit $\beta_j = \alpha_{j-1}$ für ein $j=1,\dots,n$ und $\beta_i = \alpha_i$ für $i \neq j$. Damit gilt $\abs{\beta}= k-1$. \\
		Nach Induktionsannahme gilt ähnlich wie im Fall $k=1$
		\begin{equation}
			\sup_{\partial B_{\frac{r}{k}}(x_0)} \abs{D^{\beta}u} \leq \frac{c(n,k-1)k^{n+1-1}}{\left( k-1 \right)^{n+k-1}r^{n+k-1}} \norm{u}_{L^1(B_r(x_0))}.
		\end{equation}
		Insgesamt foglt dann die Behauptung.
	\end{description}
\end{beweis}

\begin{satz}[Satz von Liouville]
	Sei $u \in C^2(\mathbb{R}^n)$ harmonisch. Ist $u$ beschränkt, so ist $u$ konstant.
\end{satz}
\begin{beweis}
	Sei $x_0 \in \mathbb{R}^n$ und $r >0$. Wie oben gilt für alle $k > 0$
	\begin{align*}
		\abs{D_iu(x_0)} &\leq c(n,1) \frac{1}{r^{n+1}} \norm{u}_{L^1(B_r(x_0))} \\
		& \leq \frac{1}{r^{n+1}} \int_{B_r(x_0)}^{} \sup_{\mathbb{R}^n} \abs{u} \,\mathrm{d}x \\
		&\leq \frac{1}{r^{n+1}}\, \underset{\leq M}{\underbrace{\sup_{\mathbb{R}^n}\abs{u}}} \,\omega_n r^n \\
		&\leq \frac{c}{r}.
	\end{align*}
	Also gilt für alle $x_0 \in \mathbb{R}^n$ und für alle $i=1,\dots,n$
	\[
		\lim_{k \to \infty}\abs{D_iu(x_0)} = 0
	\]
	Damit gilt 
	\[
		 \nabla u \equiv 0 
	\]
	in $\mathbb{R}^n$ und somit ist $u$ konstant.
\end{beweis}

\begin{definition}
	Sei $\Omega \in \mathbb{R}^n$ offen. Eine Funktion $f : \Omega \to \mathbb{R}$ heißt analytisch in einem Punkt $x \in \Omega$, 
	falls $f$ sich lokal durch seine Taylorreihe darstellen lässt, also falls es ein $r \in (0, \dist(x,\partial \Omega))$ existiert, so dass
	\begin{equation}
		f(y) = \sum_{n \in \mathbb{N}^n}^{} \frac{}{} \frac{1}{\alpha !}D^{\alpha}f(x)(y-x)^{\alpha} \qquad \text{für alle }y \in B_r(x) \text{ gilt.}
	\end{equation}
	Falls $f$ in allen $x$ analytisch ist, so heißt $f$ analytisch in $\Omega$.
\end{definition}

\begin{satz}
 	   Sei $\Omega \subseteq \mathbb{R}^n$ offen und $u \in C^2(\Omega)$ harmonisch auf $\Omega$. Dann ist $u$ analytisch in $\Omega$.
\end{satz}

\begin{beweis}
	siehe Blatt $3$ Aufgabe $4$.
\end{beweis}

Für $\Omega \subseteq \mathbb{R}^n$ offen, beschränkt und regulär betrachten wir für gegebene $f,g$ das Poisson-Problem
\begin{align}
	- \Delta u &= f \qquad \text{in }\Omega \\
	u &= g \qquad \text{auf }\Omega
\end{align}

\subsection{Darstellungsformel für Lösungen der Poissongleichung} 
\label{sub:darstellungsformel_fur_losungen_der_poissongleichung}

\begin{satz}[Greensche Darstellungsformel]
	Sei $\Omega $ eine offene, beschränkte Teilmenge des $\mathbb{R}^n$ mit $C^1$-Rand und $h \in C^2(\Omega) \cap C^1(\bar{\Omega})$ mit $ \Delta h \in L^1(\Omega)$.
	Dann gilt für alle $x \in \Omega$:
	\begin{equation}
		h(x) = - \int_{\Omega}^{} \Phi(x-y) \Delta h(y) \,\mathrm{d}y 
		+ \int_{\partial \Omega}^{} \left( \Phi(x-y)  \nabla h(y) - h(y)  \nabla_x \Phi(x-y) \right) \cdot \nu(y) \,\mathrm{d}S,
	\end{equation}
	wobei $\Phi$ die Fundamentallösung der Lagrange-Gleichung bezeichnet.
\end{satz}

\begin{beweis}
	Zu vorgegebenen $x \in \Omega$ betrachten wir ein fixiertes $\varepsilon < \min \set{1,\dist(x, \partial \Omega)}$.
	\[
		\Omega = B_{\varepsilon}(x) \cup (\Omega \setminus B_{\varepsilon}(x))
	\]
	Dann gilt
	\begin{align*}
		\int_{\Omega}^{} \Phi(x-y) \Delta h(y) \,\mathrm{d}y 
		&= \underset{=:A_{\varepsilon}}{\underbrace{\int_{B_{\varepsilon}(x)}^{} \Phi(x-y) \Delta h(y) \,\mathrm{d}y }}
		+ \underset{=:C_{\varepsilon}}{\underbrace{\int_{ \Omega \setminus B _{\varepsilon}(x)}^{} \Phi(x-y) \Delta h(y) \,\mathrm{d}y}}
	\end{align*}
	Wir zeigen $\abs{A _{\varepsilon}}\to 0$ für $\varepsilon \to 0$ mit
	\begin{equation}
		\abs{A_{\varepsilon}} \leq c \norm{\Delta^2 h}_{L^{\infty}(B_{\varepsilon}(x))} \int_{B_{\varepsilon}(x)}^{}\Phi(x-y) \,\mathrm{d}y
	\end{equation}
	und mit $\abs{y} \leq \varepsilon$ gilt
	\begin{align}
		\int_{B_{\varepsilon}(x)}^{}\Phi(x-y) \,\mathrm{d}y = \int_{B_{\varepsilon}(0)}^{} \Phi(y) \,\mathrm{d}y &= \begin{cases}
			c \int_{B_{\varepsilon}(0)}^{} -\lg(y) \,\mathrm{d}y, &\text{ falls } n =2\\
			c \int_{B_{\varepsilon}(0)}^{} \abs{y}^{2n} \,\mathrm{d}y , &\text{ falls } n \geq 2
			\end{cases} \\
			&= \begin{cases}
				c \abs{\lg \varepsilon} \varepsilon^2, &\text{ falls }n=2\\
				c \varepsilon^{2-n} \varepsilon^n, &\text{ falls }n \geq 2\\
			\end{cases}.
	\end{align}
	Daraus folgt dann \[
		A_{\varepsilon} \to 0 \qquad \text{für }\varepsilon \to 0.
	\]
	$c$ ist hier eine von $\varepsilon$ unabhängige Konstante. \\
	Als nächstes betrachten wir $C_{\varepsilon}$. Mit Hilfe der Greenschen Formel 
	\[
		\int_{U}^{} ( u \Delta v - v \Delta u) \,\mathrm{d}y = \int_{\partial U}^{}(u  \nabla  v - v  \nabla u) \cdot \nu \,\mathrm{d}S
	\]
	gilt mit $U = \Omega \setminus B _{\varepsilon}(x)$,$u(y) = \Phi(x-y)$ und $v(y)= h(y)$
	\begin{align*}
		C _{\varepsilon} &= \int_{\Omega \setminus B _{\varepsilon}(x)}^{} \Phi(x-y) \Delta h(y)  \,\mathrm{d}y
		 - \int_{\Omega \setminus B _{\varepsilon}(x)}^{} h(y) \underset{=0, x \neq y}{\underbrace{\Delta_y \Phi(x-y)}} \,\mathrm{d}y \\
		 &= \int_{\partial ( \Omega \setminus B _{\varepsilon}(x))}^{} \left( \Phi(x-y)  \nabla h(y) - h(y)  \nabla_y \Phi(x-y) \right) \cdot \nu(y) \,\mathrm{d}S \\
		 &= \int_{\partial \Omega}^{} (\Phi(x-y)  \nabla h(y) - h(y)  \nabla_y \Phi(x-y)) \cdot \nu(y) \,\mathrm{d}S \\
		 & \qquad \qquad - \underset{D_{\varepsilon}}{\underbrace{\int_{\partial B_{\varepsilon}(x)}^{} \Phi(x-y)  \nabla h(y) \cdot \nu(y) \,\mathrm{d}S}} \\
		 & \qquad \qquad + \underset{E_{\varepsilon}}{\underbrace{\int_{\partial B_{\varepsilon}(x)}^{} h(y)  \nabla_y \Phi(x-y) \cdot \nu(y) \,\mathrm{d}S}}
	\end{align*}
	Nun gilt es zu zeigen, dass
	\begin{enumerate}[(i)]
		\item $\abs{D_{\varepsilon}} \to 0$ für $\varepsilon \to 0$
		\item $E_\varepsilon \to -h(x)$ für $\varepsilon \to 0$
	\end{enumerate}
	\begin{beweis}
		\begin{enumerate}[(i)]
			\item Es gilt
			\begin{align*}
				\abs{D_{\varepsilon}} &\leq \norm{  \nabla h}_{L^{\infty}(B_1(x))} \int_{ \partial B_{\varepsilon}(x)}^{} \Phi(y) \,\mathrm{d}S \\ &\leq \begin{cases}
					c \norm{ \nabla h}_{L^{\infty}(B_1(x))} \int_{\partial B_{\varepsilon}(0)}^{} 
					- \lg \abs{y} \,\mathrm{d}S \leq C \abs{ \lg \varepsilon}\varepsilon, &\text{ falls }n=2\\
					c \norm{ \nabla h}_{L^{\infty}(B_1(x))} \int_{\partial B_{\varepsilon}(0)}^{} \abs{y}^{2-n} \,\mathrm{d}S 
					\leq  c \varepsilon^{2-n} \varepsilon^{n-1} = c \varepsilon , &\text{ falls } n \geq 3.
				\end{cases}
			\end{align*}
			Insgesamt also $\abs{D_{\varepsilon}} \to 0$ für $\varepsilon \to 0$
			\item Es gilt
			\[
				\Phi(y) = \begin{cases}
					-\frac{1}{2 \omega_2} \lg \abs{y}, &\text{ falls }n=2\\
					\frac{1}{n(n-2)\omega_n} \abs{y}^{2-n} , &\text{ falls } n \geq 3.
				\end{cases}
			\]
			und für $n \geq  3$ außerdem
			\begin{equation}
				 \nabla \Phi(y) = - \frac{1}{n \omega_n} \abs{y}^{1-n} \frac{y}{\abs{y}}.
			\end{equation}
			Damit
			\begin{equation}
				 \nabla_y \Phi(x-y) = \frac{1}{n \omega_n} \abs{x-y}^{1-n} \frac{x-y}{\abs{x-y}} = \frac{1}{n \omega_n} \frac{x-y}{\abs{x-y}^n}.
			\end{equation}
			Für $\nu$ gilt
			\begin{equation}
				\nu(y) = \frac{y-x}{\abs{y-x}}  
			\end{equation}
			und somit für $y \in \partial B_{\varepsilon}(x)$
			\begin{equation}
				 \nabla_y \Phi(x-y) \cdot \nu(y) = - \frac{1}{n \omega_n} \frac{1}{\abs{x-y}^{n-1}} =- \frac{1}{n \omega_n} \frac{1}{\varepsilon^{n-1}}.
			\end{equation}
			Für $E_{\varepsilon}$ gilt damit folgt aus der Stetigkeit von $h$
			\begin{equation}
				E _{\varepsilon} = - \frac{1}{n \omega_n \varepsilon^{n-1}} \int_{\partial B_{\varepsilon}(x)}^{} h(y) \,\mathrm{d}S(y) 
				= - \fint_{\partial B_{\varepsilon}(x)}^{} -h(y) \,\mathrm{d}S(y) \stackrel{\varepsilon \to 0}{\to } -h(x)
			\end{equation}
		\end{enumerate}
	\end{beweis}
	Insgesamt folgt die Behauptung des Satzes.
\end{beweis}	

\begin{korollar}
	Sei $f \in C^2_0(\mathbb{R}^n)$ und $u : \mathbb{R}^n \to \mathbb{R}$ gegeben durch 
	\[
		u(x) = ( \Phi * f)(x) := \int_{\mathbb{R}^n}^{} \Phi(x-y) f(y) \,\mathrm{d}y =\int_{\mathbb{R}^n}^{} \Phi(y)f(x-y) \,\mathrm{d}y.
	\]
	Dann gilt $u \in C^2(\mathbb{R}^n)$ und $- \Delta u = f$ in $\mathbb{R}^n$.
\end{korollar}
\begin{beweis}
	Zunächst zeigen wir $u \in C^2(\mathbb{R}^n)$. 
	$\Phi(y)f(x-y)$ ist stetig differenzierbar nach $x_i$ für $i=1,\dots,n$ und $y \neq x$, weil $f \in C^2_0(\mathbb{R}^n)$.
	\begin{equation}
		\diff{}{x_i} \left( \Phi(y)f(x-y) \right) = \Phi(y) \diff{f(x-y)}{x_i}
	\end{equation}
	ist integrierbar, weil $f$ einen kompakten Träger besitzt und $\Phi \in L^1_{\text{loc}}(\mathbb{R}^n)$. Damit erhalten wir für alle $i=1,\dots,n$
	\begin{equation}
		\diff{u}{x_i}(x) = \int_{\mathbb{R}^n}^{} \Phi(y) \diff{f(x-y)}{x_i} \,\mathrm{d}y. 
	\end{equation}
	Analog gilt für alle $i,j=1,\dots,n$
	\begin{equation}
		\diff{^2u}{x_i \partial x_j} (x) = \int_{\mathbb{R}^n}^{}\Phi(y) \diff{^2f(x-y)}{x_i \partial x_j} \,\mathrm{d}y.
	\end{equation}
	und damit $u \in C^2(\mathbb{R}^n)$. Für die zweite Behauptung folgern wir
	\begin{align}
		\Delta u(x) &= \int_{\mathbb{R}^n}^{} \Phi(y) \Delta_x f(x-y) \,\mathrm{d}y \\ 
		&= \int_{\mathbb{R}^n}^{} \Phi(y) \Delta_y f(x-y) \,\mathrm{d}y \\
		&= \int_{\mathbb{R}^n}^{} \Phi(x-y)\Delta f(y) \,\mathrm{d}y
	\end{align}
	Wir wollen nun $- \Delta u = f$ in $\mathbb{R}^n$ zeigen. Dafür bemerken wir zunächst, dass wegen $f \in C^2_0(\mathbb{R}^n)$ auch $f \in C_0^2(B_R(0))$ für 
	ein hinreichend großes $R > 0$ gilt. Mit Satz 2.19 erhalten wir 
	\begin{equation}
		- f(x) = \underset{= \Delta u(x)}{\underbrace{\int_{B_R(0)}^{} \Phi(x-y) \Delta f(y) \,\mathrm{d}y }} 
	\end{equation}
	Die Randintegrale verschwinden wegen der kompakten Träger für $f$ und $  \nabla f$. 
\end{beweis}



%%%%% 02.05.2016
%%%%% lecture 02.05.2016 %%%%%

\begin{bemerkung}
	\begin{enumerate}[(i)]
		\item Für $n \geq3$ ist $u$ beschränkt und $\lim_{\abs{x} \to \infty}u(x)=0$.
		\item Für $n=2$ ist $u$ potentiell und beschränkt.
		\item Jede andere beschränkte Lösung der Poisson-Gleichung auf $\mathbb{R}^n$ unterscheidet sich nur für eine additive Konstante.
	\end{enumerate}
	\begin{beweis}
		\begin{enumerate}[(i)]
			\item Sei $n \geq  3$ und $\Phi(y)= \frac{1}{n(n-2)\omega_n}\frac{1}{\abs{y}^{n-2}}$ für $y \neq 0$ die Fundamentallösung der Poisson-Gleichung. 
			Da $f \in C^2_0(\mathbb{R}^n)$ und somit $ \supp(f) \subset B_R(0)$ für ein hinreichend großes $R > 0$. Wegen $B_R(x) \subset B_{2R}(0)$ gilt
			\begin{align*}
				\abs{u(x)} &= \abs{ \int_{B_R(0)}^{} \Phi(x-y)f(y) \,\mathrm{d}}y \\
				& \leq \norm{f}_{L^{\infty}(\mathbb{R}^n)} \int_{B_R(0)}^{}\abs{\Phi(x-y)} \,\mathrm{d}y \\
				&= \norm{f}_{L^{\infty}(\mathbb{R}^n)} \int_{B_R(x)}^{} \abs{\Phi(y)} \,\mathrm{d}y \\
				& \leq \norm{f}_{L^{\infty}(\mathbb{R}^n)} \underset{\leq M}{\underbrace{\int_{B_{2R}(0)}^{} \abs{\Phi(y)} \,\mathrm{d}y}} \\
				& \leq C
			\end{align*}
			für $x \in B_R(0)$. Für $x \in \mathbb{R}^n \setminus B_R(0)$ gilt mit $y \in B_R(0)$ und $\abs{x}> R$
			\begin{equation}
				\abs{x-y} \geq \abs{x} - \abs{y} > \abs{x} - R > 0 
			\end{equation}
			und damit
			\begin{align*}
				\abs{u(x)} &\leq \norm{f}_{L^{\infty}(\mathbb{R}^n)} \int_{B_R(0)}^{} \abs{ \Phi(x-y)} \,\mathrm{d}y \\
				&\leq c \norm{f}_{L^{\infty}(\mathbb{R}^n)} \int_{B_R(0)}^{} \frac{1}{\abs{x-y}^{n-2}} \,\mathrm{d}y \\
				& \leq  c \norm{f}_{L^{\infty}(\mathbb{R}^n)} \int_{B_R(0)}^{} \frac{1}{\left( \abs{x}-R \right)^{n-2}} \,\mathrm{d}y \\
				& = c \norm{f}_{L^{\infty}(\mathbb{R}^n)} \frac{1}{(\abs{x}-R)^{n-2}} \stackrel{\abs{x}\to \infty}{\to} 0 .
			\end{align*}
			\item Sei nun $n=2$. Beachte, dass nun für $y \neq 0$
			\[
				\Phi(y) = - \frac{1}{2R} \lg \abs{y}.
			\] Ist $f$ beispielsweise $f \leq 0$ mit $f \leq -1$ in $B_1(0)$ und $\supp (f) \subset B_2(0)$ so gilt $\Phi(y) \leq 0$ für $\abs{y}> 1$ 
			Außerdem gilt für $\abs{x} > 3$
			\[
				\abs{x-y} \geq \abs{x}-\abs{y} > 3- 2 = 1
			\]
			\begin{equation}
				\Rightarrow \Phi(x-y) \leq 0
			\end{equation}
			\begin{equation}
				\Rightarrow \Phi(x-y)f(y) \geq 0 \qquad \text{ für } x \in \mathbb{R}^n \setminus B_3(0), y \in B_2(0).
			\end{equation}
			Damit wegen $\abs{x-y} > \abs{x}-1$
			\begin{align*}
					u(x) &= \int_{B_2(0)}^{} \Phi(x-y) f(y) \,\mathrm{d}y \\
					& \geq \int_{B_1(0)}^{} \Phi(x-y)f(y) \,\mathrm{d}y \\
					& \geq  \int_{B_1(0)}^{} \abs{\Phi(x-y)} \,\mathrm{d}y \\
					& \geq  \int_{B_1(0)}^{} c \lg ( \abs{x}-1) \,\mathrm{d}y \\
					&= c \lg(\abs{x}-1),
			\end{align*}
			also insgesamt für $\abs{x} \to \infty$
			\begin{equation}
				u(x) \to  \infty.
			\end{equation}
		\item Wir nehmen an, dass $u_1,u_2$ zwei beschränkte Lösungen der Poisson-Gleichung sind. 
		Die Funktion $u_1-u_2$ ist beschränkt und harmonisch in $\mathbb{R}^n$ und daher nach dem Satz von Liouville konstant. Also ist 
		\begin{equation}
			u_1 = u_2 + \text{Konstante}.
		\end{equation}
		\end{enumerate}
	\end{beweis}
\end{bemerkung}

Wir betrachten nun wieder die Poisson-Gleichung mit Dirichlet-Randbedingungen.
\begin{align*}
	\begin{cases}
		- \Delta u = f, &\text{ falls }x \in \Omega\\
		u =g, &\text{ falls } x \in \partial \Omega,
	\end{cases}
\end{align*}
wobei $\Omega \subseteq \mathbb{R}^n$ offen, beschränkt und mit $C^1$-Rand. Außerdem seien $f,g$ hier reguläre Funktionen.
Wir wollen eine Darstellungsformel für die Lösung finden. Wir haben bereits bewiesen, dass diese Gleichung höchstens eine Lösung hat.

\begin{definition}[Greensche Funktion für $\Omega$]
	Sei $\Omega$ eine offene Menge in $\mathbb{R}^n$. 
	Eine Funktion $G : \set[(x,y) \in \Omega \times \Omega]{x \neq y} \to \mathbb{R}$ heißt Greensche Funktion für $\Omega$, falls für alle $x \in \Omega$ gilt:
	\begin{enumerate}[(i)]
		\item $y \to G(x,y)- \Phi(x-y)$ ist von der Klasse $C^2(\Omega) \cap C^1(\bar{\Omega})$ und harmonisch in $\Omega$,
		\item $y \to G(x,y)$ hat Nullrandwerte auf $\partial \Omega$, d.h. es gilt 
		\[
			\lim_{y \to y_0}G(x,y) = 0 \qquad \text{ für } y_0 \in \partial \Omega.
		\]
		(und bei unbeschränktem $\Omega$ auch für $y_0 = \infty$)
	\end{enumerate}
\end{definition}
\begin{bemerkung}
	\begin{enumerate}[1)]
		\item $y \to G(x,y) - \Phi(x-y)$ heißt die Korrektorfunktion. 
		Ist $\Omega$ beschränkt, so ist nach der Eindeutigkeit der Lösung der Poissongleichung die Korrektorfunktion 
		(und damit auch die Greensche Funktion) eindeutig, falls sie existiert.
		\begin{equation}
			\Delta_y ( G(x,y) - \Phi(x-y)) = 0 \qquad  \Rightarrow \qquad \Delta_y G(x,y) = \Delta_y \Phi(x-y) = 0 \qquad \text{ für }x \neq y
		\end{equation}
		Also ist $y \to G(x,y)$ harmonisch. Es gilt außerdem
		\[
			G(x,y) - \Phi(x-y)  \Big|_{\partial \Omega}^{} = - \Phi(x-y)  \Big|_{\partial \Omega}^{}
		\]
		\item $y \to G(x,y)$ ist von der Klasse $C^2(\Omega \setminus \set{x}) \cap C^1( \bar{\Omega} \setminus \set{x})$ und besitzt die gleiche Singularität in $x$
		wie $\Phi$.
	\end{enumerate}
\end{bemerkung}

\begin{satz}[Greensche Lösungsformel]
	Sei $\Omega$ eine beschränkte offene Menge in $\mathbb{R}^n$ mit $C^1$-Rand und $G$ die Greensche Funktion für $\Omega$ (falls existent). 
	Ist $u \in C^2(\Omega) \cap C^1(\bar{\Omega})$ eine Lösung der Poissongleichung. 
	Für Funktionen $f \in C^{\infty}(\Omega) \cap L^1(\Omega)$ und $g \in C^0(\partial \Omega)$ gilt für alle $x \in \Omega$
	\[
		u(x) = \int_{\Omega}^{} G(x,y)f(y) \,\mathrm{d}y - \int_{\partial \Omega}^{}  \nabla_y G(x,y) g(y) \cdot \nu(y) \,\mathrm{d}S(y).
	\]
\end{satz}
\begin{beweis}
	Wir benutzen die Greensche Formel und Satz 2.19. Die Greensche Formel wenden wir auf die harmonische Funktion $y \to G(x,y) - \Phi(x-y)$ und die Lösung $u$ an. 
	Sei $v(y):= G(x,y) - \Phi(x-y)$ und $w(y)= u(y)$. Dann gilt
	\begin{align*}
		\int_{\Omega}^{}(G(x,y)-\Phi(x-y)) \Delta u \,\mathrm{d}y 
		&= \int_{\partial \Omega}^{} \left[ (G(x,y)- \Phi(x-y))  \nabla u - u  \nabla_y (G(x,y)-\Phi(x-y)) \right] \cdot \nu(y)  \,\mathrm{d}S(y) \\
		& \qquad \qquad - \int_{\Omega}^{} G(x,y)f(y) \,\mathrm{d}y - \int_{\Omega}^{} \Phi(x-y) \Delta u(y) \,\mathrm{d}y  \\
		& = - \int_{\partial \Omega}^{} \left[ \Phi(x-y)  \nabla u(y) + u(y)  \nabla_y \Phi(x-y) \right] \cdot \nu(y)  \,\mathrm{d}S(y) \\
		& \qquad \qquad -\int_{\partial \Omega}^{} g  \nabla_y G(x,y) \cdot \nu(y)  \,\mathrm{d}S(y).		
	\end{align*}
	Insgesamt folgt die Behauptung.
\end{beweis}
Wir wollen nun zeigen, dass die Funktion $G$ symmetrisch ist, also $G(x,y)=G(y,x)$. Dies führt dann zu folgendem Resultat.
$y \to G(x,y)$ ist harmonisch in $\Omega \setminus \set{x}$ und somit $ \Delta_y G(x,y)=0$. Dann ist auch $x \to G(x,y)$ harmonisch in $\Omega \setminus \set{y}$.

\begin{lemma}
	Sei $\Omega$ eine offene Teilmenge des $\mathbb{R}^n$, $G$ die Greensche Funktion für $\Omega$ und $B_r(x) \subset \subset \Omega$. 
	Ist $h \in C^2(B_r(x)) \cap C^1( \overline{B_r(x)})$, so gilt
	\begin{equation}
		\lim_{\varepsilon \to 0} \int_{\partial B _{\varepsilon}(x)}^{} \left( G(x,y)  \nabla h(y) - h(y)  \nabla_y G(x,y) \right) \cdot \nu(y) \,\mathrm{d}S(y) =
		h(x).
	\end{equation}
\end{lemma}
\begin{beweis}
	Aus dem Beweis von Satz von 2.19 folgt
	\begin{align*}
		\lim_{\varepsilon \to 0} \underset{-D_{\varepsilon}- E_{\varepsilon}}{\underbrace{\int_{\partial B_{\varepsilon}(x)}^{}\left( \Phi(x-y) 
		 \nabla h(y) - h(y)  \nabla_y \Phi(x-y) \right) \cdot \nu(y) \,\mathrm{d}S(y)}}=h(x).
	\end{align*}
	$h$ und die Korrektorfunktion $G(x,y)-\Phi(x-y)$ sind regulär.
	\begin{align}
		\int_{ \partial B_{\varepsilon}(x)}^{} &\left( \left( \underset{ C^1(\overline{B_r(x)})}{\underbrace{G(x,y)- \Phi(x-y)}} \right) 
		 \underset{ C^0(\overline{B_r(x)})}{\underbrace{\nabla h(y) }} - \underset{ C^1(\overline{B_r(x)})}{\underbrace{h(y)}} 
		  \underset{ C^0(\overline{B_r(x)})}{\underbrace{\nabla_y \left( G(x-y) -\Phi(x-y) \right)}} \right) \cdot \nu(y) \,\mathrm{d}S(y) \\
		   &\leq M S_n \varepsilon^{n-1} \stackrel{\varepsilon \to 0}{\to } 0
	\end{align}
\end{beweis}

\begin{satz}[Symmetrie der Greenschen Funktion]
	Ist $G$ die Greensche Funktion zu einer beschränkten, offenen Menge $\Omega \subseteq \mathbb{R}^n$ mit $C^1$-Rand, so gilt 
	\[
	G(x,y)=G(y,x)  	
	\]  
	für alle $x,y \in \Omega$ mit  $x \neq y$.
\end{satz}
\begin{beweis}
	Für feste $x,y \in \Omega$ mit $x \neq y$ definieren wir die Hilfsfunktionen 
	\begin{align*}
		v(z) &= G(x,z), \qquad z \in \bar{\Omega} \setminus \set{x} \\
		w(z) &= G(y,z), \qquad z \in \bar{\Omega} \setminus \set{y}
	\end{align*}
	Dann gilt
	\begin{align*}
		\Delta v(z) &= 0 \qquad \text{ in } \Omega \setminus \set{x}, \qquad v  \Big|_{\partial \Omega}^{}= 0 \\
		\Delta w(z) &= 0 \qquad \text{ in } \Omega \setminus \set{y}, \qquad w  \Big|_{\partial \Omega}^{}= 0 
	\end{align*}
	Für $\varepsilon < \min \set{ \dist(x, \partial \Omega) , \dist(y, \partial \Omega), \frac{\dist(x,y)}{2} }$ folgt
	\[
		\overline{B_\varepsilon(x)} \cup \overline{B_\varepsilon(y)} \subseteq \Omega \qquad
		 \qquad \overline{B_\varepsilon(x)} \cap \overline{B_\varepsilon(y)} = \emptyset 
	\]
	Die Greensche Formel angewandt auf die Menge $\Omega \setminus \overline{B_\varepsilon(x)} \cup \overline{B_\varepsilon(y)}$ liefert
	\begin{align*}
		0 &= \int_{\Omega \setminus (\overline{B_\varepsilon(x)} \cup \overline{B_\varepsilon(y)})}^{} (v \Delta w - w \Delta v) \,\mathrm{d}z  \\
		&= \int_{\partial ( \Omega \setminus \overline{B_\varepsilon(x)} \cup \overline{B_\varepsilon(y)})}^{} (v(z)  \nabla w(z)- w(z)  \nabla v(z))\cdot \nu(z)
		 \,\mathrm{d}S(z)\\ 
		& \stackrel{*}{=} \int_{\partial B_\varepsilon(x) }^{} (v(z)  \nabla w(z)- w(z)  \nabla v(z))\cdot \nu(z)
		 \,\mathrm{d}S(z) \\
		& \qquad + \int_{\partial B_\varepsilon(y) }^{} (v(z)  \nabla w(z)- w(z)  \nabla v(z))\cdot \nu(z)
		 \,\mathrm{d}S(z) \\
		&= \int_{\partial B_\varepsilon(x) }^{} (G(x,z)  \nabla w(z)- w(z)  \nabla G(x,z))\cdot \nu(z)
		 \,\mathrm{d}S(z) \\
		& \qquad + \int_{\partial B_\varepsilon(y) }^{} (v(z)  \nabla G(y,z)- G(y,z)  \nabla v(z))\cdot \nu(z)
		 \,\mathrm{d}S(z) \\
		 &= w(x)- v(y) \qquad \text{ für } \varepsilon \to  0
		\end{align*}
		Es folgt
		\[
			0 = w(x)- v(y)
		\]
		Also \[
			w(x)= G(y,x) = v(y) = G(x,y)
		\]
\end{beweis}

%%%%% 09.05.2016
%%%%% lecture 09.05.2016 %%%%%

\begin{definition}
	Sei $B_r(a)$ eine Kugel im $\mathbb{R}^n$ und $x \in \mathbb{R}^n \setminus \set{a}$. Der Punkt $x^* \in \mathbb{R}^n \setminus \set{a}$ der definiert ist durch
	\[
		x^* := a + r^2 \frac{x-a}{\abs{x-a}^2}
	\]
	heißt Spiegelungspunkt von $x$ bezüglich der Sphäre $ \partial B_r(a)$.
	Für $x \in \partial B_r(a)$ gilt
	\[
		x^* = a + \frac{r^2}{\abs{a-x}^2}(x-a) = a + x -a = x
	\]
	Es gilt
	\begin{align*}
		x^*-a &= \frac{r^2}{\abs{x-a}^2}(x-a) \\
		\abs{x^*-a} &= \frac{r^2}{\abs{x-a}^2}\abs{x-a} \\
		\abs{x-a}\abs{x^*-a} &= r^2
	\end{align*}
	%%%fehlt eine Zeichnung
\end{definition}
\begin{bemerkung}
	\begin{enumerate}[(i)]
		\item Aus $\abs{x^*-a} \abs{x-a} = r^2$ folgt $\abs{x^* -a } = \frac{r^2}{\abs{x-a}}$. 
		Dies bedeutet insbesondere, dass für $x \in  \mathbb{R}^n \setminus \left( \partial B_r(a) \cup \set{a} \right)$ genau einen der beiden Punkte $x,x^*$ 
		in der Kugel $B_r(a)$ und der andere in $\mathbb{R}^n \setminus B_r(a)$ liegen muss.
		\item \begin{align*}
			x^{**} &= x \\
			x^{**} &= a + r^2 \frac{x^* - a}{\abs{x^*-a}^2} = a + \frac{r^2}{\abs{x-a}^2}\left( \frac{r^2}{\abs{x-a}^2} \right)(x-a) = a + x-a =x
		\end{align*}
		\item Für Punkte $y \in \partial B_r(a)$ gilt
		\[
			\abs{x^*-y} = r \frac{\abs{y-x}}{x-a}
		\]
	\end{enumerate}
\end{bemerkung}
Wir definieren nun 
\[
	\Psi^*(y):= \begin{cases}
		- \Phi \left( \frac{\abs{x-a}}{r}(y-x^*) \right), &\text{ falls } x \in B_r(a) \setminus \set{a} \\
		- \Phi (r e_1) , &\text{ falls } x = a, y \in \overline{B_r(a)}
	\end{cases},
\]
wobei $\Phi$ die Fundamentallösung der Laplacegleichung bezeichnet. 
$\Phi$ hat eine Singularität in $ \frac{\abs{x-a}}{r} (y - x^*)= 0$ mit $x^* \in \mathbb{R}^n \setminus B_r(a)$. Außerdem gilt
\[
	\frac{\abs{x-a}}{r}(y- x^*) \neq 0
\]
für alle $y \in B_r(a)$ und $x \in B_r(a) \setminus \set{a}$. \\
$y \to \Psi^*(y)$ ist glatt und harmonisch in $B_r(a)$ für alle $x \in B_r(a)$. \\
Wir wollen nun zeigen, dass \[
	\Psi^*(y) + \Phi(x-y)=0 \qquad \text{für } y \in \partial B_r(a) \text{ und } x \in B_r(a)
\]
\begin{beweis}
	Für $y \in \partial B_r(a)$ und $x = a$ gilt
	\begin{align*}
		\Psi^*(y)+ \Phi(x-y) &= - \Phi(re_1) + \Phi(a-y) = 0,
	\end{align*}
	weil $\abs{re_1} = \abs{a-y} = r$ und $\Phi$ nur vom Betrag des Arguments abhängt. \\
	Für $y \in \partial B_r(a)$ und $x \in B_r(a) \setminus \set{a}$ gilt außerdem
	\begin{align*}
		\abs{\frac{\abs{x-a}}{r}(y-x^*)} &= \frac{\abs{x-a}}{r} \abs{y-x^*} \\
		&= \frac{\abs{x-a}}{r} \frac{\abs{y-x}}{\abs{x-a}}r  \\ 
		&= \abs{y-x}
	\end{align*}
	und somit
	\[
		- \Phi \left( \frac{\abs{x-a}}{r}(y-x^*) \right) + \Phi(x-y)= 0.
	\]
	\end{beweis}
	Insgesamt erhalten wir als Resultat, dass $\Psi^*$ eine Korrektorfunktion ist und wir definieren die Greensche Funktion für $B_r(a)$ durch
	\[
		G_{B_r(a)}(x,y) := \Psi^*(y) + \Phi(x-y),
	\]
	wobei hier $y \in B_r(a)$ und $x \in B_r(a)$. \\
	Die Greensche Funktion für $B_r(a)$ lautet
	\[
		G_{B_r(a)} := \begin{cases}
			 \Phi(x-y)- \Phi \left( \frac{\abs{x-a}}{r}(y-x^*) \right) , &\text{ falls }x \in B_r(a) \setminus \set{a}\\
			 \Phi(x-y)- \Phi(re_1) , &\text{ falls }x =a
		\end{cases}
	\]
	für $x,y \in B_r(a)$ und $x \neq y$. Für $n \geq 3$, $x \in  B_r(a) \setminus \set{a}$ gilt
	\[
		G_{B_r(a)}(x,y) = \frac{1}{n (n-2) \omega_n} \left( \abs{x-y}^{2-n} - \frac{ \abs{x-a}^{2-n}}{r^{2-n}} \abs{y - x^*}^{2-n} \right).
	\]
	Außerdem gilt
	\begin{align*}
		 \nabla_y G(x,y) &= \frac{1}{\omega_n n(n-2)} \left( (2-n) \abs{x-y}^{1-n} \frac{y-x}{\abs{x-y}} 
		 - (2-n) \frac{\abs{x-a}^{2-n}}{r ^{2-n}} \abs{y-x^*}^{1-n} \frac{y-x^*}{\abs{y-x^*}} \right) \\
		 &= - \frac{1}{\omega_nn} \left( \frac{y-x}{\abs{x-y}^n} - \frac{\abs{x-a}^{2-n}}{r^{2-n}} \frac{y- x^*}{\abs{y- x^*}^n} \right) \\
		 &= - \frac{1}{\omega_nn} \left( \frac{y-x}{\abs{x-y}^n} - \frac{\abs{x-a}^{2-n}}{r^{2-n}} \frac{y-x^*}{r^n \abs{y-x}} \abs{x-a}^n \right) \\
		 &= - \frac{1}{\omega_n n \abs{y-x}^n} \left( y-x - \frac{\abs{x-a}^2}{r^2}(y-x^*) \right) \\
		 &= - \frac{1}{\omega_n n \abs{y-x}^n} \left( y-x - \frac{\abs{x-a}^2}{r^2} \left( y - a - \frac{r^2}{\abs{x-a}^2}(x-a) \right) \right).
	\end{align*}
	Mit $\nu(y)= \frac{y-a}{r}$ gilt somit
	\begin{align*}
		 \nabla_y G(x,y) \cdot \nu(y) &= - \frac{1}{r \omega_n n \abs{y-x}^n} \left( y -x - \frac{\abs{x-a}^2}{r^2}(y-a) + (x-a) \right) \cdot (y-a) \\
		 &= - \frac{1}{r \omega_n n \abs{y-x}^n} \left( (y-x)(y-a) - \frac{\abs{x-a}^2}{r^2} \abs{y-a}^2 \abs{y-a} + (x-a)(y-a) \right) \\
		 &= - \frac{1}{r \omega_n n \abs{x-y}^n} \left( \underset{\abs{y-a}^2}{\underbrace{(y-x+x-a) \cdot (y-a)}} - \abs{x-a}^2 \right).
	\end{align*}
	Man kann leicht zeigen, dass das gleiche Endergebnis sich für $x=a$ und $n = 2$ ergibt. \\
	Wir definieren nun den Poissonkern für die Kugel $B_r(a)$ durch
	\[
		K_{B_r(a)}(x,y) := -  \nabla_y G_{B_r(a)}(x,y) \nu (y) = \frac{1}{\omega_n n \abs{y-x}^n r} \left( \abs{y-a}^2 - \abs{x-a}^2 \right)
	\]
	für $x \in B_r(a)$ und $y \in \partial B_r(a)$. \\
	Sei $y \in C^0(\partial B_r(a))$ und $u \in C^2(B_r(a)) \cap C^1( \overline{B_r(a)})$ mit 
	\[
		\begin{cases}
			\Delta u = 0, &\text{ in }B_r(a)\\
			 u = g, &\text{ auf } \partial B_r(a)
		\end{cases}
	\]
	Dann gilt
	\[
		u(x) = \int_{\partial B_r(a) }^{} K_{B_r(a)}(x,y)g(y) \,\mathrm{d}S(y).
	\]
	Wichtig ist an diesem Punkt, dass dies kein Existenzresultat ist.
	\begin{satz}[Poisson-Integralformel für Kugeln]
		Sei $g \in C^0(\partial B_r(a))$ und sei $u$ die Funktion definiert durch die obige Formel. Dann gelten
		\begin{enumerate}[(i)]
			\item $ \Delta u = 0$ in $B_r(a)$
			\item $u  \Big|_{\partial B_r(a)}^{} = g$ 
		\end{enumerate}
	\end{satz}
	\begin{beweis}
		\begin{enumerate}[(i)]
			\item $ y \to G(x,y)$ ist harmonisch für $y \neq x$ und $G$ ist symmetrisch. Dann folgt also dass auch $x \to G(x,y)$ harmonisch für $y \neq x$ ist.
			\[
				K_{B_r(a)}(x,y) = -  \nabla_y G(x,y) \cdot \nu(y) 
			\]
			$\diff{}{y_i}G(x,y)$ ist harmonisch für alle $i = 1, \dots,n$, denn
			\[
				\Delta_x \diff{}{y_i}G(x,y) = \diff{}{y_i} \Delta_x G(x,y) = 0.
			\]
			Vertauschung von Differentiation und Integration liefert
			\begin{align*}
				\Delta u (x) &= \Delta \int_{\partial B_r(a)}^{} K_{B_r(a)}(x,y)g(y) \,\mathrm{d}S(y) \\
				&= \int_{\partial B_r(a)}^{} \underset{=0}{\underbrace{ \Delta_x K_{B_r(a)}(x,y)}}g(y) \,\mathrm{d}S(y) = 0.
			\end{align*}
			\item Wir wollen 
			\[
				\lim_{x \to x_0} \abs{u(x)-g(x_0)}=0
			\]
			zeigen. Sei $x_0 \in  \partial B_r(a)$ und $\varepsilon >0$ mit
			\[
				\abs{g(y)-g(x_0)} < \varepsilon \qquad \qquad \forall\, y \in \partial B_r(a) \cap B_{2 \delta}(x_0)
			\]
			für ein $\delta > 0$ (wegen der Stetigkeit von $g$ auf $\partial B_r(a)$). 
			\[
				K_{B_r(a)}(x,y) \geq 0 \qquad , \qquad \int_{\partial B_r(a)}^{} K_{B_r(a)}(x,y) \,\mathrm{d}S(y) = 1
			\]
			Außerdem ist $u \equiv 1$ Lösung der Laplacegleichung. \\
			Sei $x \in B_r(a) \cap B_{\varepsilon}(x_0)$ Dann gilt
			\begin{align*}
				\abs{u(x)-g(x_0)} &= \abs{\int_{\partial B_r(a)}^{} K(x,y)(g(x)-g(x_0)) \,\mathrm{d}S(y)} \\
				&\leq \int_{\partial B_r(a)}^{} K(x,y)\abs{g(y)-g(x_0)} \,\mathrm{d}S(y) \\
				&\leq \int_{\partial B_r(a) \cap B_{2 \delta}(x_0)}^{} K(x,y) \varepsilon \,\mathrm{d}S(y) 
				+ \int_{\partial B_r(a) \setminus B_{2 \delta}(x_0)}^{} K(x,y) 2 \max_{\partial B_r(a)} \abs{g} \,\mathrm{d}S(y) \\
				&\leq  \varepsilon \underset{=1}{\underbrace{\int_{\partial B_r(a)}^{} K(x,y) \,\mathrm{d}S(y) }}
				+ 2 \max_{\partial B_r(a)}\abs{g} \int_{\partial B_r(a) \setminus B_{2 \delta }(x_0)}^{} K(x,y) \,\mathrm{d}S(y) .
			\end{align*}
			Wir haben $\abs{y-x_0} \geq 2 \delta $ für $y \in \partial B_r(a) \setminus B_{2 \delta }(x_0)$ und $\abs{x-x_0} < \varepsilon$. Damit folgt
			\[
				\abs{y-x} \geq  \abs{y-x_0}- \abs{x-x_0} \geq \delta 
			\]
			\[
				\Rightarrow K(x,y) \leq \frac{1}{n \omega_n r} \frac{r^2 - \abs{x-a}^2}{r \delta ^n}
			\]
			und damit 
			\begin{align*}
				\int_{\partial B_r(a) \setminus B_{2 \delta }(x_0)}^{} K(x,y) \,\mathrm{d}S(y) 
				&\leq \int_{\partial B_r(a) \setminus B_{2 \delta }(x_0)}^{} \frac{1}{\omega_n n r }\frac{ r^2 - \abs{x-a}^2}{r \delta^n } \,\mathrm{d}S(y) \\
				&\leq c \frac{r^2 - \abs{x-a}^2}{r \delta^n} ,
			\end{align*}
			wobei $x \in B_r(a)$. Der Grenzübergang $x \to x_0$ ergibt dann
			\[
				\lim_{x \to x_0} \abs{u(x)-g(x_0)} \leq \varepsilon
			\]
			und damit folgt
			\[
				\lim_{x \to x_0} \abs{u(x)- g(x_0)} = 0
			\]
		\end{enumerate}
	\end{beweis}

%%%%% 6
%%%%% lecture 6 %%%%%

\subsection*{Greensche Funktion für den Halbraum} 
\label{sec:section_name}
Definiere 
\[
	\mathbb{R}^n_+ := \set[(x_1,\dots,x_n) \in \mathbb{R}^n]{x_n > 0}
\]

\begin{definition}
	Zu $x \in \mathbb{R}^n$ heißt der Punkt $\bar{x} \in \mathbb{R}^n$, der definiert ist als 
	\[
		\bar{x} := (x_1, \dots, x_{n-1},-x_n) 
	\]
	der Spiegelungspunkt von $x$ bezüglich $\partial \mathbb{R}^n_+$. 
\end{definition}
	Für $x,y \in \mathbb{R}^n_+$ und $\Psi^*(y):= \Phi(\bar{x}-y)$ gilt
	\begin{enumerate}[(i)]
		\item $y \to \Psi^*(y)$ ist glatt weil $\bar{x} \in \mathbb{R}^n \setminus \mathbb{R}^n_+$ (also $\bar{x} \neq y$) und harmonisch
		\item Für $y \in \partial \mathbb{R}^n_+$ gilt
		\begin{align*}
			\abs{\bar{x}-y} &= \sqrt{(x_1-y_1)^2+ \dots + (x_{n-1}-y_{n-1})^2 + (-x_n)^2} \\
			&= \sqrt{(x_1-y_1)^2+ \dots + (x_{n-1}-y_{n-1})^2 + x_n^2} \\
			&= \abs{x-y}
		\end{align*}
		und es folgt für $y \in \partial \mathbb{R}^n_+$ \[
			-\Phi(\bar{x}-y)+  \Phi(x-y) = 0.
		\] 
		Es folgt, dass $\Psi^*$ eine Korrektorfunktion ist und wir definieren die Greensche Funktion als
		\[
			G_{\mathbb{R}^n_+}(x,y) := \Phi(x-y) - \Phi(\bar{x},y) \qquad \text{für alle }x,y \in \mathbb{R}^n_+, x \neq y
		\]
		Dann gilt
		\[
				G_{\mathbb{R}^n_+}(x,y) = \begin{cases}
					\frac{1}{n \omega_n} \left( -  \lg \abs{x-y} + \lg \abs{\bar{x}-y} \right), &\text{ falls }n=2\\
					\frac{1}{n(n-2)\omega_n} \left( \abs{x-y}^{2-n}- \abs{\bar{x}-y}^{2-n} \right), &\text{ falls }n \geq 3
				\end{cases}
		\]
		\[
			 \nabla_y G_{\mathbb{R}^n_+}(x,y) = - \frac{1}{n \omega_n} \left( \frac{y-x}{\abs{x-y}^n} - \frac{y - \bar{x}}{\abs{\bar{x}-y}^n} \right)
			 = - \frac{1}{n \omega_n \abs{x-y}^n} (-x + \bar{x})
		\]
		Außerdem gilt
		\[
			  \nabla_y G_{\mathbb{R}^n_+}(x,y) \cdot \underset{=-e_n}{\underbrace{\nu(y)}} = - \frac{2 x_n}{n \omega_n \abs{x-y}^n}
		\]
		Den Poissonkern für den Halbraum definieren wir für $x \in \mathbb{R}^n_+$ und $y \in  \partial \mathbb{R}^n_+$ durch
		\[
			K_{\mathbb{R}^n_+}(x,y) := -  \nabla_y G(x,y) \cdot \nu(y) = \frac{2x_n}{n \omega_n \abs{x-y}^n}
		\]
	\end{enumerate}

\begin{satz}[Poisson-Integralform für den Halbraum]
	Sei $g \in C^{\infty}(\mathbb{R}^{n-1})\cap L^{\infty}(\mathbb{R}^{n-1})$. Dann definiert für $x \in \mathbb{R}^n_+$
	\[
		u(x):= \int_{\partial \mathbb{R}^n_+}^{} K_{\mathbb{R}^n_+}(x,y)g(y) \,\mathrm{d}S(y)
	\]
	eine beschränkte harmonische Funktion $u \in C^2(\mathbb{R}^n_+) \cap C^0(\bar{\mathbb{R}^n_+})$ mit $u =g$ auf $ \partial \mathbb{R}^n_+$.
\end{satz}
\begin{beweis}
	Analog zum Beweis von Satz 2.26 bzw Evans Satz $38$, Satz $14$
\end{beweis}

\[
	\text{(P)} \begin{cases}
		- \Delta u = f, &\text{ in }\Omega\\
		u = g , & \text{ auf } \partial \Omega
	\end{cases}, \qquad \text{\underline{Energiemethode}}
\]
Sei $\Omega \in \mathbb{R}^n$ offen, $u \in L^2(\Omega)$, $ \nabla u \in L^2(\Omega; \mathbb{R}^n)$, $f \in L^2(\Omega)$. Wir definieren das Dirichlet-Funktional durch
\[
	I_f(u):= \frac{1}{2} \int_{\Omega}^{} \abs{ \nabla u}^2 \,\mathrm{d}x - \int_{\Omega}^{} f u \,\mathrm{d}x
\]
und 
\[
	A_g := \set[ w \in C^2(\Omega) \cap C^1(\bar{\Omega})]{w=g \text{ auf } \partial \Omega}
\]
\begin{satz}[Dirichlet-Prinzip]
	Sei $\Omega \subseteq \mathbb{R}^n$ eine beschränkte, offene Menge mit $C^1$-Rand, $f \in C^0(\partial \Omega)$. 
	Eine Funktion $u \in A_g$ ist ein Minimierer von $I_f$ auf $A_g$ genau dann, wenn $ - \Delta u = f$ in $\Omega$ und $u =g$ auf $\partial \Omega$.
	Dies bedeutet, dass $u$ eine Lösung von (P) ist.
\end{satz}

\begin{beweis}
	\begin{description}
		\item[$\Leftarrow $:] Wir nehmen an, dass $u$ eine Lösung von (P) ist und wir wollen zeigen, dass 
		\[
			I_f(u) \leq I_f(v) \qquad \forall\,  v \in A_g .
		\] 
		Sei $v \in A_g$ und $v = u + (v-u)$. Dann folgt mit partieller Integration
		\begin{align*}
			I_f(v) &= \frac{1}{2} \int_{\Omega}^{} \abs{ \nabla  u}^2 \,\mathrm{d}x + \frac{1}{2} \int_{ \Omega}^{} \abs{  \nabla (v-u)}^2 \,\mathrm{d}x 
			+ \int_{ \Omega}^{}  \nabla u  \nabla (v-u) \,\mathrm{d}x - \int_{\Omega}^{}fu \,\mathrm{d}x - \int_{\Omega}^{}f(v-u) \,\mathrm{d}x \\
			&= I_f(u) + \frac{1}{2} \int_{\Omega}^{} \abs{  \nabla (v-u)} \,\mathrm{d}x 
			+ \int_{\partial \Omega}^{} \underset{\substack{=0, \\ \text{auf } \partial \Omega, \\\text{weil }v,u \in A_g}}{\underbrace{(v-u)}} 
			\nabla u \cdot \nu \,\mathrm{d}S - \int_{\Omega}^{}f(v-u) \,\mathrm{d}x \\
			&= I_f(u) + \frac{1}{2} \int_{\Omega}^{} \abs{  \nabla (v-u)}^2 \,\mathrm{d}x - \int_{\Omega}^{} ( \nabla u + f)(v-u) \,\mathrm{d}
		\end{align*}
		und insgesamt folgt 
		\[
			I_f(v) \geq I_f(u) \qquad \forall\, v \in A_g.
		\]
		Somit ist $u$ ein Minimierter von $I_f$.
		\item[$\Rightarrow $:] Wir nehmen an, dass $u$ ein Minimierer von $I_f$ ist also 
		\[
			I_f(u) \leq I_f(v) \qquad \forall\, v \in A_g.
		\]
		Insbesondere 
		\[
			I_f(-u) \leq I_f(u + t \phi) \qquad \forall\,  t \in \mathbb{R}, \varphi \in C^{\infty}_0(\Omega)
		\]
		und $u + t \varphi \in A_g$. \\
		Wir wollen nun zeigen, dass $u$ eine Lösung von (P) ist. Sei $F: \mathbb{R} \to \mathbb{R}$ mit
		\begin{align*}
			F(t) &:= I_f(u + t \varphi) \\ &= \frac{1}{2} \int_{\Omega}^{} \abs{  \nabla u}^2 \,\mathrm{d}x 
			+ \frac{t^2}{2} \int_{\Omega} \abs{ \nabla \varphi}^2 \,\mathrm{d}x + t \int_{\Omega}^{}  \nabla u \cdot  \nabla \varphi \,\mathrm{d}x
			- \int_{\Omega}^{} fu \,\mathrm{d}x - t \int_{\Omega}^{}f \varphi \,\mathrm{d}x.
		\end{align*}
		Es gilt somit $F \in C^1(\Omega)$ und außerdem
		\[
			F(0) = I_f(u) \leq I_f(u + t \varphi) = F(t) \qquad \forall\, t \in \mathbb{R}.
		\]
		Damit ist $t = 0$ ein Minimierer von $F$ und somit ist $F'(0)=0$. Dann erhalten wir auch
		\begin{align*}
			0 &= F'(t)  \Big|_{t=0}^{} \\ &= \diffd{}{t}F(t)  \Big|_{t=0}^{} \\ &= t \int_{\Omega}^{} \abs{ \nabla \varphi}^2 \,\mathrm{d}x
			+ \int_{\Omega}^{}  \nabla u \cdot  \nabla \varphi \,\mathrm{d}x - \int_{\Omega}^{} f \varphi \,\mathrm{d}x  \Big|_{t=0}^{} \\
			&= \int_{\Omega}^{}  \nabla u \cdot  \nabla \varphi \,\mathrm{d}x - \int_{\Omega}^{}f \varphi \,\mathrm{d}x
		\end{align*}
	und es folgt
	\[
		\int_{\Omega}^{}  \nabla u \cdot  \nabla \varphi \,\mathrm{d}x - \int_{\Omega}^{} f \varphi \,\mathrm{d}x = 0 \qquad \forall\, \varphi \in C^{\infty}_0(\Omega).
	\]
	Mit partieller Integration erhalten wir für alle $\varphi \in C^{\infty}_0(\Omega)$
	\[
		0 = - \int_{\Omega}^{}  \Delta u  \varphi\,\mathrm{d}x - \int_{\Omega}^{}f \varphi \,\mathrm{d}x = - \int_{\Omega}^{}(  \Delta u + f) \varphi \,\mathrm{d}x,
	\]
	 weil $ \Delta u + f$ stetig ist. Somit
	 \[
	 	\Delta u + f = 0 \qquad \text{in } \Omega,
	 \]
	 denn wäre zum Beispiel $ \Delta u (x_0) + f(x_0) > 0$ für ein $x_0 \in \Omega$. Dann folgt, weil $ \Delta u + f$ stetig ist, dass
	 $\Delta u + f > 0$ in $B_{\delta }(x_0)$ für ein kleines $\delta > 0$. 
	 Betrachte dann eine Testfunktion $\Psi_{\delta}$ mit $\Psi_{\delta } > 0$ passend mit kompaktem Träger auf $B_{\delta }(x_0)$ . 
	 Dann gilt aber
	 \[
	 	\int_{\Omega}^{}( \Delta u + f) \Psi_{\delta } \,\mathrm{d}x = \int_{B_{\delta }(x_0)}^{} ( \Delta u + f)\Psi_{\delta } \,\mathrm{d} > 0.
	 \]
	\end{description}
\end{beweis}

Der Beweis, dass dieser Minimierer wirklich existiert, benötigt die Theorie von Sobolevräumen und schwachen Lösungen. Dieses wird in der Vorlesung "Variationsrechnung" im Wintersemester behandelt. 

\section{Wärmeleitungsgleichung} 
\label{sec:warmeleitungsgleichung}
Nun werden wir die Wärmeleitungsgleichung behandeln, die definiert ist durch
\[
	\partial_t u - \Delta u  = 0 \qquad \text{in }I \times \Omega
\]
mit $u = u(t,x)$, wobei $t>0$ eine Zeitkonstante ist und $x \in \Omega \subseteq \mathbb{R}^n$ eine Raumkonstante bezeichnet. Außerdem ist $I = (0,T]$ mit $T>0$ und in diesem Falle $ \Delta u = \Delta_x u$. Die Wärmeleitungsgleichung ist eine lineare, parabolische partielle differentialgleichung zweiter Orndung.

\begin{definition}
	Sei $\Omega \subseteq \mathbb{R}^n$, $T >0$.
	\begin{enumerate}[(i)]
		\item Der parabolische Zylinder $\Omega_T$ ist definiert als
		\[
			\Omega_T := (0,T] \times \Omega \subseteq \mathbb{R}^{n+1}
		\]
		\item der parabolische Rand von $\Omega_T$ (Mantel vom parabolischen Zylinder) ist definiert als
		\[
			\partial_p \Omega_T = ( \set{0} \times \Omega) \cup ( [0,T] \times \partial \Omega) \subset \partial \Omega .
		\]
		$\Omega_p \Omega_T$ ist abgeschlossen und falls $T < \infty$, 
		so unterscheidet sich $\partial_p \Omega_T$ von $\partial \Omega_T$ genau um die Menge $\set{T} \times \Omega$.
	\end{enumerate}
\end{definition}

\begin{bemerkung}
	Da $t$ eine Zeitkoordinate ist, ist es sinnvoll, Randbedingungen nur auf $\partial_p \Omega_T$ zu fordern. 
	Es wäre unnatürlich Randbedingungen auf $\set{T} \times \Omega$ zu fordern. (Man fordert Anfangswerte aber keine Endwerte!)
\end{bemerkung}
Für eine Teilmenge $P \subseteq \mathbb{R}^{n+1}$ definieren wir für $i,j=1,\dots,n$ 
\[
	C^2_1(P):= \set[u \in C^1(P)]{ D_iD_j u \in C^0(P)}
\]

\subsection{Kalorische Funktionen} 
\label{sub:kalorische_funktionen}

\begin{definition}
	Sei $\Omega$ eine offene Teilmenge in $\mathbb{R}^n$, $T >0$ und $u \in C^2_1(\Omega_T)$. Man bezeichnet $u$ als kalorisch, falls $\partial_t - \Delta u =0$ in 
	$\Omega_T$ gilt. Falls ledigleich die Ungleichung $\partial_t - \Delta u \leq 0$ gilt, so nennt man sie subkalorisch und falls $\partial_t - \Delta u \geq 0$ gilt, 
	superkalorisch.
\end{definition}

\begin{bemerkung}[Invarianzen der Lösungseigenschaften]
	Ist $u \in C_1^2(\Omega_T)$ kalorisch, so führen die folgenden Transformationen der Lösung wieder zu Lösungen der Wärmeleitungsgleichungen. 
	(auf dem transformierten parabolischen Zylinder). 
	\begin{enumerate}[(i)]
		\item Translationen in Zeit und Ort für $\eta >0$, $x_0 \in \mathbb{R}^n$.
		\[
			u_{0,x_0}(t,x): = u(t,x-x_0) \qquad , \qquad u_{(\eta,0)}(t,x) := u(t-\eta,x).
		\]
		\item Rotationen bezüglich der Raumvariablen für $R \in \text{SO}(n)$
		\[
			u_R(t,x) = u(t,Rx).
		\]
		\item Inhomogene Dilatation für $\lambda >0$.
		\[
			u_{\lambda}(t,x) := \lambda^n u ( \lambda^2, \lambda x)
		\]
		(Das ist auch wahrt ohne $\lambda^n$. mit der angegebenen Skalierung bleibt das Integral
		\[
			\int_{\mathbb{R}^n}^{} u _{\lambda}(t,x) \,\mathrm{d}x = \int_{\mathbb{R}^n}^{} u(\lambda^2t,y) \,\mathrm{d}y
		\] erhalten, zum skalierten Zeitpunkt.)
	\end{enumerate}
\end{bemerkung}

\begin{beispiele}
	\begin{enumerate}[1)]
		\item Jede harmonische Funktion $u \in C^2(\Omega)$ ist (aufgefasst als Funktion auf $\Omega_T$) kalorisch auf $\Omega_T$. 
		Dies sind genau die kalorischen Funktionen, die invariant unter Translationen in der Zeit sind.
		%% noch nicht fertig.
		\[
			\int_{\Omega}^{}f \,\mathrm{d}x = \sum^{n}_{i=1} 
		\]
		$u \in C^2(\Omega) \cap C^0(\bar{\Omega})$
	\end{enumerate}
\end{beispiele}


%%%%% 23.05.2016
%%%% 23.05.2016 %%%%

Falls das Definitionsgebiet und die Funktion $u$ invariant unter der entsprechenden Operation sind, so ergeben sich Lösungen mit Symmetrien, die eine einfache
Differentialgleichung lösen.

\begin{enumerate}[(i)]
	\item Falls $u(t,x)=u(t-r,x)$ für alle $r \in \mathbb{R}$, dann $u(t,x)=v(x)$ für $v : \Omega \to \mathbb{R}$ und $ \Delta u = 0$.
	\item Falls $u(t,x)= u(t,x-x_0)$ für alle $x_0 \in \mathbb{R}^n$, dann $u(t,x)= w(t)$ mit $w_t(t)=0$. Also ist $w$ konstant.
	\item Falls $u(t,x)=u(t,kx)$ für alle $k \in \text{SO}(n)$, 
	dann $u(t,x)=v(t,\abs{x})$ für ein $v : \mathbb{R}^+ \times \mathbb{R}^+ \to \mathbb{R}$ und $v$ erfüllt für $r= \abs{x} >0$ und $t >0$
	\[
		v_t(t,r) - \underset{=\Delta u(t,r)}{\underbrace{\left( \partial^2_r v(t,r)+ \frac{n-1}{r} \partial_r v(t,r) \right)}}=0.
	\]

	\item Falls $u(t,x)= \lambda^n u(\lambda^2 t, \lambda x)$ für alle $\lambda > 0$ (mit der speziellen Wahl $\lambda = t^{- \frac{1}{2}}$), so existiert $\tilde v : \mathbb{R}^n \to \mathbb{R}$ mit
	\[
		u(t,x) = t^{- \frac{n}{2}} \tilde v ( t^{- \frac{1}{2}}x) \qquad \forall\, t>0, x \in \mathbb{R}^n.
	\]
	Also
	\[
		- \frac{n}{2} \tilde v(y) - \frac{1}{2} y \cdot  \nabla \tilde v(y)- \Delta \tilde v(y) = 0 \qquad \text{ in } \mathbb{R}^n
	\]
	Mit $\lambda = t^{- \frac{1}{2}}$ gilt
	\begin{align*}
		u(t,x) &= \lambda^n u(\lambda^2 t, \lambda x) \\
		&= t^{- \frac{n}{2}} u(t^{-1}t, t^{- \frac{1}{2}} x) \\
		&= t^{- \frac{n}{2}} u(1, t^{- \frac{1}{2}} x) \\
		&= t^{- \frac{n}{2}} \tilde v(t^{- \frac{1}{2}} x)
	\end{align*}
	\[
		u_t(t,x) = - \frac{n}{2} t^{- \frac{n}{2} - 1} \tilde v(t^{- \frac{1}{2}} x) + t^{- \frac{n}{2}}  \nabla \tilde v(t^{- \frac{1}{2}} x) \cdot x 
		\left( - \frac{1}{2} t^{- \frac{1}{2}-1} \right)
	\]
	Damit gilt
	\[
		\Delta_x u(t,x) = t^{- \frac{n}{2}} \Delta \tilde v(t^{- \frac{1}{2}} x) \cdot t^{-1}
	\]
	und somit mit $y = t^{- \frac{1}{2}} x \in \mathbb{R}^n$
	\begin{align*}
		u_t(t,x) - \Delta u(t,x) &= 0 \\
		-\frac{n}{2} t^{- \frac{n}{2}-1} \tilde v(t^{- \frac{1}{2}} x) + t^{- \frac{n}{2}-1}  
		\nabla  \tilde v(t^{- \frac{1}{2}} x) \cdot (-\frac{1}{2} x t^{- \frac{1}{2}}) - t^{- \frac{n}{2}-1}  \Delta \tilde v(t^{- \frac{1}{2}}x) &= 0 \\
		-\frac{n}{2} \tilde v(t^{- \frac{1}{2}}x)+  \nabla \tilde v(t^{- \frac{1}{2}} x) \cdot \left( - \frac{1}{2} x t^{- \frac{1}{2}} \right)- \Delta \tilde v(t^{- \frac{1}{2}}x) &=0 .
	\end{align*}
	
	\item Falls $u$ sowohl unter Rotationen im Raum als auch unter inhomogenen Dilatationen (wie in (iv)) ist, so existiert eine Funktion $w: \mathbb{R}^+ \to \mathbb{R}$ mit
	\[
		u(t,x)= t^{- \frac{n}{2}} w(t^{- \frac{1}{2}} \abs{x})
	\]
	für alle $t >0$ und $x \in \mathbb{R}^n$. $w$ erfüllt die folgende ODE:
	\[
		\underset{u_t}{\underbrace{-\frac{n}{2} w(s)- \frac{1}{2} s w'(s)}} - \underset{\Delta u}{\underbrace{\left( w''(s) + \frac{n-1}{s} w'(s) \right)}} 
		= 0 \qquad \forall\, s >0 
	\]
	mit $s = t^{- \frac{1}{2}} \abs{x}$. Multipliziere diese nun mit $s^{n-1}$ so ergibt sich
	\begin{align*}
		- \frac{n}{2} w(s) s^{n-1} + \frac{1}{2} s^n w'(s) + w''(s) s^{n-1} + (n-1) s^{n-2} w'(s) &= 0 \\
		\frac{1}{2} \left( w(s) s^n \right) + \left( w'(s) s^{n-1} \right)' &=0  \\
		\frac{1}{2} w(s) s^n + w'(s) s^{n-1} = \text{Konstante} = 0
	\end{align*}
	und wir erhalten eine ODE
	\[
		w'(s) = \frac{1}{2}w(s)s \qquad \Rightarrow \qquad w(s) = c e^{-\frac{s^2}{4}} \qquad c \in \mathbb{R}
	\]
	und somit
	\[
		u(t,x) = t^{- \frac{n}{2}} w(t^{- \frac{1}{2}} \abs{x}) = c t^{- \frac{n}{2}} e^{- \frac{\abs{x}}{4t}}
	\]
\end{enumerate}

\begin{definition}
	Die Funktion $\Psi : \mathbb{R}^+ \times \mathbb{R}^n \to \mathbb{R}$ definiert als 
	\[
		\Psi(t,x) = \frac{1}{(4 \pi t)^{\frac{n}{2}}} e^{-\frac{\abs{x}^2}{4t}}
	\]
	heißt die Fundamentallösung der Wärmeleitungsgleichung
\end{definition}
\begin{bemerkung}
	\item $ \Psi_t(t,x) = \Delta \Psi(t,x) = 0$ für alle $t>0$ und $x \in \mathbb{R}^n$.
	\item Für die Anfangswerte gilt
	\[
		\lim_{t \to 0} \Psi(t,0) = \infty \qquad , \qquad \lim_{t \to 0} \Psi(t,x) = 0 \qquad \forall\, x \neq 0, x \in \mathbb{R}^n
	\]
	\item \[
		\int_{\mathbb{R}^n}^{} \Psi(t,x) \,\mathrm{d}x = 1,
	\]
	denn mit Transformationssatz ($ y = \frac{x}{2 \sqrt{t}}$) gilt
	\begin{align*}
		\int_{\mathbb{R}^n}^{}\Psi(t,x) \,\mathrm{d}x &= \frac{1}{(4 \pi t)^{\frac{n}{2}}} \int_{\mathbb{R}^n}^{}e^{-\frac{\abs{x}^2}{4t}} \,\mathrm{d}x \\
		&= \frac{1}{(\pi)^{\frac{n}{2}}} \int_{\mathbb{R}^n}^{} e^{- \abs{y}^2} \,\mathrm{d}y \\
		&= (\pi)^{- \frac{n}{2}} \int_{\mathbb{R}^n}^{} e^{-(y_1^2 + \dots + y_n^2)} \,\mathrm{d}y_1 \dots \mathrm{d}y_n \\
		&= (\pi)^{- \frac{n}{2}} \left( \underset{= \sqrt{\pi}}{\underbrace{\int_{\mathbb{R}}^{} e^{y_1^2} \,\mathrm{d}y_1}} \right)^n = 1
	\end{align*}
\end{bemerkung}
Wir haben schon gesehen, dass wenn $u$ harmonisch auf $\Omega$ ist, dann gilt
\[
	u(x_0) = \fint_{B_r(x_0)}^{} u(x) \,\mathrm{d}x \qquad \forall\, B_r(x_0) \subset \subset \Omega.
\]
Wir wollen nun eine ähnliche Eigenschaft für kalorische Funktionenb beweisen. Dafür definieren wir
\[
	B_r(x_0) = \set[x \in \mathbb{R}^n]{\abs{x-x_0}< r} = \set[x \in \mathbb{R}^n]{\Psi(x-x_0) > \Phi(re_1)},
\]
wobei $\Phi$ die Fundamentallösung der WLG ist. Für $n \geq 3$ gilt in dieser Menge
\[
	\frac{1}{n(n-2)\omega_n} \abs{x- x_0}^{2-n} > \frac{1}{n(n-2)\omega_n} r ^{2-n}
\]
und dies ist äquivalent zu $\abs{x-x_0}<r$.\\
$B_r(x_0)$ stellt genau die $\Phi(re_1)$- Superniveaumenge.

\begin{definition}[Wärmeleitungskugel]
	Sei $t_0 \in \mathbb{R}, x_0 \in \mathbb{R}^n$ und $r >0$. Wir definieren die Wärmeleitungskugel $W_r(t_0,x_0)$ an $(t_0,x_0)$ (nicht das Zentrum) als
	\[
		W_r(t_0,x_0) := \set[(t,x) \in \mathbb{R}^{n+1}]{ t<t_0 \text{ und } \Psi(t_0-t, x_0-x)> r^{-n}}
	\]
\end{definition}
\begin{bemerkung}
	\begin{enumerate}[(i)]
		\item $(t_0,x_0)$ gehört nicht zu der offenen Menge $W_t(t_0,x_0)$, sondern zu \\ $\partial W_r(t_0,x_0)$.
		\item Monotonie: Für $r < \tilde r$ gilt $ W_r(t_0,x_0) \subset W_{\tilde r}(t_0,x_0)$.
		\item Translationsverhalten: $W_k(t_0,x_0) = (t_0,x_0)+ W_r(0,0)$.
		\item Parabolische Reskalierung: $(t,x) \in W_r(0,0)$ ist äquivalent zu $( r^{-2}t, r^{-1}x) \in W_1(0,0)$.
		%% Begrüundung fehlt.
		\item Explizite Darstellung von $W_r(0,0)$:
		\[
			W_r(0,0) = \set[(t,x) \in \mathbb{R}^{n+1}]{t<0, \Psi(-t,x)> r^{-n}}
		\] 
		\begin{align*}
			& \qquad \Psi(-t,x) = \left( -4 \pi t \right)^{- \frac{1}{2}} e^{\frac{\abs{x}^2}{4t}} > r^{-n} \\
			\Leftrightarrow & \qquad e^{\frac{\abs{x}^2}{4t}} > (-4 \pi t)^{\frac{n}{2}} \\
			\Leftrightarrow & \qquad \frac{\abs{x}^2}{4t} > \lg \left( (-4 \pi t)^{\frac{n}{2}} r^{-n} \right) \\
			\Leftrightarrow & \qquad \frac{\abs{x}^2}{4t} > \lg (-4 \pi t)^{\frac{n}{2}} + \lg (r^{-n}) = \frac{n}{2} \lg (-4 \pi t)- n \lg ( r)
		\end{align*}
		Definiere nun $b_r: \mathbb{R}^- \times \mathbb{R}^n \to \mathbb{R}$ mit
		\[
			b_r(t,x):= \frac{\abs{x}^2}{4t} + n \lg r - \frac{n}{2} \lg(-4 \pi t)
		\]
		Damit
		\begin{align*}
			W_r(0,0) &= \set[(t,x) \in \mathbb{R}^{n+1}]{b_r(t,x)>0} \\
			&= \set[(t,x) \in \mathbb{R}^{n+1}]{- \frac{r^2}{4 \pi} < t < 0 , \abs{x}^2 < 2nt \lg \left( - \frac{r^2}{4 \pi t} \right)}
		\end{align*}
		$W_r(0,0)$ ist beschränkt ($W_r(x_0,t_0)= (x_0,t_0)+ W_r(0,0)$). Außerdem gilt
		\[
			\partial W_r(0,0) = \set{(0,0)} \cup \set[(t,x) \in \mathbb{R}^{n+1}]{b_r(t,x)=0}.
		\]
		\item Gewichtetes Volumen: $W_r(0,0)$ hat bezüglich des gewichteten Maßes $\abs{x}^2 t^{-2} \, \mathrm{d}t \, \mathrm{d}x$ ein Volumen von $4 r^n$ also
		\[
			\int_{W_r(0,0)}^{} \frac{\abs{x}^2}{t^2} \,\mathrm{d}t \, \mathrm{d}x = 4 r^n
		\]
	\end{enumerate}
\end{bemerkung}

\begin{lemma}
	Sei $R >0$ und $u \in C^2_1(W_R(0,0))$. Definiert man $\psi : (0,R) \to \mathbb{R}$ mittels
	\[
		\psi(r) := \frac{1}{4 r^n} \int_{W_r(0,0)}^{} u(t,x) \frac{\abs{x}^2}{t^2} \,\mathrm{d}t \,\mathrm{d}x \qquad \text{ für } r \in (0,R)
	\]
	so gelten
	\begin{enumerate}[(i)]
		\item \[
			\lim_{k \to 0^+} \psi(r) = u(0,0)
		\]
		\item \[
			\psi'(r) = \frac{n}{r^{n+1}} \int_{W_r(0,0)}^{} (-u_t(t,x)+ \Delta u(t,x))b_r(t,x) \,\mathrm{d}t \,\mathrm{d}x,
		\] wobei $b_r$ die Funktion aus der obigen Bemerkung (v) ist.
	\end{enumerate}
\end{lemma}
\begin{beweis}
	\begin{enumerate}[(i)]
		\item Es gilt
		\begin{align*}
			\abs{ \psi(r) - w(0,0)} &= \abs{ \frac{1}{4r^n} \int_{W_r(0,0)}^{} (u(t,x)- u(0,0)) \frac{\abs{x}^2}{t^2} \,\mathrm{d}t \, \mathrm{d}x} \\
			& \leq \frac{1}{4 r^n} \int_{W_r(0,0)}^{} \abs{u(t,x)-u(0,0)} \frac{\abs{x}^2}{t^2} \,\mathrm{d}t \, \mathrm{d}x \\
			& \leq \sup_{W_r(0,0)} \abs{u(t,x)-u(0,0)} \to 0 \qquad \text{ für } r \to 0^+, \text{ da } u \text{ stetig ist.}
		\end{align*}
		\item Mit Transformation $t= r^2 s$ und $x = ry$ gilt
		\begin{align*}
			\psi(r) &= \frac{1}{4 r^n} \int_{W_1(0,0)}^{}u(r^2s,ry) \frac{r^2 \abs{y}^2}{r^4 s^2} r^2 r^n \,\mathrm{d}s \,\mathrm{d}y \\
			&= \frac{1}{4} \int_{W_1(0,0)}^{} u(r^2s,ry) \frac{\abs{y}^2}{s^2} \,\mathrm{d}s \,\mathrm{d}y.
		\end{align*}
		Wegen $u \in C^2_1(W_R(0,0))$ ist die Funktion $r \mapsto  u(r^2s,ry)$ ist differenzierbar für alle $(s,y) \in W_1(0,0)$. Außerdem sind
		$u(r^2s,ry)$ und $\diffd{}{r}u$ integrierbar über $W_1(0,0)$ und daraus folgt, dass die Ableitung von $\psi$ existiert mit
		\[
			\diffd{\psi}{r} = \frac{1}{4} \int_{W_1(0,0)}^{} \diffd{}{r} u(r^2s,ry) \frac{\abs{y}^2}{s^2} \,\mathrm{d}s \,\mathrm{d}y
		\]
		Berechne dazu nun
		\begin{align*}
			\diffd{}{r}u(r^2s,ry) = u_t(r^2s,ry)2rs+  \nabla u(r^2s,ry)\cdot y.
		\end{align*}
		und erhalte
		\begin{align*}
			\diffd{\psi}{r} &= \frac{1}{4} \int_{W_1(0,0)}^{} ( u_t(r^2s,ry)2rs +  \nabla u(r^2s,ry) \cdot y) \frac{\abs{y}^2}{s^2} \,\mathrm{d}s \,\mathrm{d}y \\
			&= \frac{1}{4} \int_{W_r(0,0)}^{} (u_t(t,x)2 \frac{t}{r} + 
			\nabla u(t,x) \cdot \frac{x}{r}) \frac{\abs{x}}{r^2} \frac{r^4}{t^2} \frac{\mathrm{d}t \, \mathrm{d}x}{r^2r^n}  \\
			&= \frac{1}{4r^{n+1}} \int_{W_r(0,0)}^{} ( u_t(t,x)2t +  \nabla u(t,x) \cdot x) \frac{\abs{x}}{t^2} \,\mathrm{d}t \,\mathrm{d}x \\
			&= \frac{1}{r^{n+1}} \int_{W_r(0,0)}^{} \left( u_t(t,x) \frac{\abs{x}^2}{2t}
			+  \nabla u(t,x) \cdot \frac{\abs{x}^2 x}{4t^2} \right) \,\mathrm{d}t  \, \mathrm{d}x
		\end{align*}
		Es gilt
		\begin{align*}
			\diff{b_r}{t}(t,x) = - \frac{\abs{x}^2}{4t^2} - \frac{n}{2} \frac{1}{t} \qquad , \qquad  \nabla b_r(t,x) = \frac{x}{2t}
		\end{align*}
		Und wegen \[
			\frac{\abs{x}^2}{2t} = \frac{x \cdot x}{2t} =  \nabla b_r(t,x) \cdot x
		\]
		und 
		\[
			\frac{\abs{x}^2 x}{4t^2} = - \partial_tb_r(t,x)x -n \frac{x}{2t} = - \partial_t b_r(t,x) \cdot x - n  \nabla b_r(t,x)
		\]
		es gilt somit insgesamt 
		\begin{align*}
			\diffd{\psi}{r} &= \frac{1}{r^{n+1}} \int_{W_r(0,0)}^{} u_t(t,x)  \nabla b_r(t,x) \cdot x 
			+  \nabla u(t,x) \left( - \partial_t b_r(t,x) \cdot x - n  \nabla b_r(t,x) \right) \,\mathrm{d}t \,\mathrm{d}x \\
			&= \frac{1}{r^{n+1}} \int_{W_r(0,0)}^{}- b_r(t,x) \diver(u_t(t,x)x) 
			+  \nabla u(t,x) \left( - \partial_t b_r(t,x) \cdot x - n  \nabla b_r(t,x) \right) \,\mathrm{d}t \,\mathrm{d}x \\
			&= \frac{1}{r^{n+1}} \int_{W_r(0,0)}^{} \left( -b_r(t,x)  \nabla u_t(t,x) \cdot x - b_r(t,x)u_t(t,x)n \right)
			+ \left( \partial_t  \nabla u(t,x) x - n  \nabla u(t,x) \cdot  \nabla b_r(t,x) \right) \,\mathrm{d}t \,\mathrm{d}x \\
			&= \frac{1}{r^{n+1}} \int_{W_r(0,0)}^{} \left( - u_t(t,x)n b_r(t,x) + n \Delta u(t,x) b_r(t,x) \right) \,\mathrm{d}t \,\mathrm{d}x.
		\end{align*}
		Die Randintegrale verschwinden jeweils wegen $b_r(t,x) = 0$ auf $\partial W_r(0,0)$.
		%%%% 30.05.2016 %%%%
		
	\end{enumerate}
\end{beweis}

%%%% 30.05.2016
%%% 30.05.2016 %%%

\begin{satz}[Mittelwerteigenschaft]
	Sei $\Omega$ eine offene Teilmenge des $\mathbb{R}^n$, $T >0$, \\ $W_r(t_0,x_0) \subset \subset \Omega_T$ und $u \in C^2_1(\Omega_T)$.
	\begin{enumerate}[(i)]
		\item Falls $u_t - \Delta u = 0$ in $\Omega_T$ gilt, so folgt 
		\[
			u(t_0,x_0) = \frac{1}{4r^n} \int_{W_r(t_0,x_0)}^{} u(t,x) \frac{\abs{x-x_0}^2}{(t-t_0)^2} \,\mathrm{d}t \,\mathrm{d}x
		\]
		\item Falls $u_t - \Delta u \leq 0$ in $\Omega_T$ gilt, so folgt \[
			u(t_0,x_0) \leq \frac{1}{4r^n} \int_{W_r(t_0,x_0)}^{} u(t,x) \frac{\abs{x-x_0}^2}{(t-t_0)^2} \,\mathrm{d}t \,\mathrm{d}x
		\] 
		\item Falls $u_t - \Delta u < 0$ in $\Omega_T$ gilt, so folgt 
		\[
			u(t_0,x_0) < \frac{1}{4r^n} \int_{W_r(t_0,x_0)}^{} u(t,x) \frac{\abs{x-x_0}^2}{(t-t_0)^2} \,\mathrm{d}t \,\mathrm{d}x
		\]
	\end{enumerate}
\end{satz}
\begin{beweis}
	\begin{enumerate}[(i)]
		\item folgt aus (ii) angewandt auf $u$ und $-u$.
		\item Sei $v(t,x) := u(t+ t_0 , x +x_0)$. Wenn $u$ subKalonisch ist, so auch $v$. Dieses ist äquivalent zu 
		\[
			v_t - \Delta v \leq 0 \qquad , \qquad \psi(r) = \frac{1}{4r^n} \int_{W_r(0,0)}^{} v(t,x) \frac{\abs{x}^2}{t^2} \,\mathrm{d}t \,\mathrm{d}x. 
		\]
		Wegen $\psi'(r)>0$ ist $\psi(r)$ monoton wachsend. Damit gilt für alle $r >0$
		\[
			\lim_{\rho \to 0} \psi(\rho) \leq \psi(r).
		\]
		Mit Lemma $3.5$ folgt dann $v(0,0) \leq \psi(r)$ also
		\[
			v(0,0) = u(t_0,x_0) \leq \frac{1}{4r^n} \int_{W_r(0,0)}^{} u(t+t_0,x+ x_0) \frac{\abs{x}^2}{t^2} \,\mathrm{d}t \,\mathrm{d}x
		\]
		Mit $s = t+t_0$ und $y = x + x_0$ folgt die Aussage.
		\item Analog zu (ii).
	\end{enumerate}
\end{beweis}

\begin{korollar}
	Sei $\Omega$ eine offene Teilmenge des $\mathbb{R}^n$, $T>0$ und $u \in C^2_1(\Omega_T)$. Dann sind äquivalent:
	\begin{enumerate}[(i)]
		\item $u$ ist Kalonisch.
		\item $u$ erfüllt die Mittelwerteigenschaft auf Wärmeleitungskugeln, d.h für alle \\ $W_t(t_0,x_0) \subset \subset \Omega_T$ gilt
		\[
			u(t_0,x_0) = \frac{1}{4r^n} \int_{W_r(t_0,x_0)}^{} u(t,x) \frac{\abs{x-x_0}^2}{(t-t_0)^2} \,\mathrm{d}t \,\mathrm{d}x.
		\]
	\end{enumerate}
\end{korollar}

\begin{beweis}
	Dieser Beweis geht analog zu Satz $2.5$.
\end{beweis}

\subsubsection{Folgerungen aus der Mittelwerteigenschaft} 
\label{sub:folgerungen_aus_der_mittelwerteigenschaft}
Nun werden wir Maximumsprinzipien formulieren.

\begin{satz}[Maximumsprinzipien]
	Sei $\Omega$ eine beschränkte, offene Teilmenge des $\mathbb{R}^n$ und $ u \in C^2_1(\Omega_T) \cap C^0( \bar{\Omega_T})$ eine subKalonische Funktion
	( $ u_t - \Delta u \leq 0$ in $\Omega_T$) Dann gilt:
	\begin{enumerate}[(i)]
		\item das schwache Maximumsprinzip:
		\[
			\max_{\bar{\Omega}_T} u = \max_{\partial_p \Omega_T} u
		\]
		$\partial_p \Omega_T$ bezeichnet sozusagen den Mantel von $\Omega_T$.
		\item das starke Maximumsprinzip: \\
		Ist $\Omega$ zusammenhängend und existiert $(t_0,x_0) \in \Omega_T$ mit 
		\[
			u(t_0,x_0) = \max_{\overline{\Omega}_T}u,
		\]
		so ist $u$ konstant auf $\Omega_{t_0}= [0,t_0] \times \Omega$
	\end{enumerate}
\end{satz}
\begin{bemerkung}
	Die Aussage des starken Maximumsprinzips besagt lediglich, dass $u$ Konstant zu jedem früheren Zeitpunkt sein muss.
\end{bemerkung}

\begin{beweis}
	\begin{description}
		\item[(ii) $\Rightarrow$ (i):]Wir zeigen dies mit einem Wiederspruchsargument. Wäre (i) falsch, so gäbe es $(t_0,x_0)$ mit 
		\[
			u(t_0,x_0) = \max_{\bar{\Omega_T}} u > \max_{\partial_p \Omega_T}
		\]  
		Wir nennen $\Omega(x_0)$ die Zusammenhangskomponente von $x_0$ in $\Omega$. Setze
		\[
			\Omega(x_0)_{t_0} := [0,t_0] \times \Omega(x_0)
		\]
		Wir erhalten wegen $\partial \Omega(x_0) \subseteq  \partial \Omega$
		\[
			\max_{\overline{\Omega(x_0)_{t_0}}}u = \max_{\overline{\Omega_T}}u > \max_{\partial_p \Omega_T} u \geq \max_{\partial_p \Omega(x_0)_{t_0}}u 
		\]
		einen Widerspruch, da $u$ nach (ii) konstant auf $\Omega(x_0)_{t_0}$ sein müsste.
		\item[(i) $\Rightarrow$ (ii)] Sei $(t_0,x_0) \subseteq \Omega_T$ mit 
		\[
			u(t_0,x_0) = \max_{\overline{\Omega_T}}u =: M.
		\]
		Sei $r >0$ mit $W_r(t_0,x_0) \subset \subset \Omega_T$.
		\[
			M = u(t_0,x_0) \leq  \frac{1}{4r^n} \int_{W_r(t_0,x_0)}^{}u(t,x) \frac{\abs{x-x_0}^2}{(t-t_0)^2} \,\mathrm{d}t \,\mathrm{d}x \leq M
		\]
		wegen $u \leq M$ und weil das Integral $1$ ergibt. Weiter erhalten wir
		\[
			\frac{1}{4r^n} \int_{W_r(t_0,x_0)}^{} (u(t,x)-M) \frac{\abs{x-x_0}^2}{(t-t_0)^2} \,\mathrm{d}t \,\mathrm{d}x =0
		\]
		und weil $u-M \leq 0$ auf $W_r(t_0,x_0)$ folgt somit $u-M \equiv 0$ auf $W_r(t_0,x_0)$. 
		Da außerdem $u(t_0,x_0)=M$ gilt $u \equiv M$ auf $W_r(t_0,x_0) \subset \subset \Omega_T$.
		Wir wollen zeigen, dass $u \equiv M$ auf $\Omega_{t_0}$ ist. Dafür gehen wir in zwei Schritten vor.
		\begin{description}
			\item[$1$. Schritt:] Wir betrachten $(t_1,x_1)$ so dass $t_1 < t_0$ und die gesamte Linie, 
			die $(t_0,x_0)$ mit $(t_1,x_1)$ verbindet in $\Omega_{t_0}$ liegt. 
			\[
				L((t_0,x_0),(t_1,x_1)) : = \set[\tau (t_1,x_1) + (1- \tau) (t_0,x_0)]{\tau \in [0,1]} \subseteq \Omega_{t_0}.
			\]
			Die Behauptung ist nun: 
			\[
				u \equiv M \qquad \text{ auf } L((t_0,x_0),(t_1,x_1))
			\]
			Setze dafür $f : [0,1] \to \mathbb{R}$ stetig mit
			\[
				f(\tau) := u(\tau (t_1,x_1) + (1- \tau) (t_0,x_0).
			\]
			Dann ist 
			\[
				f^{-1}(\set{M}) = \set[\tau \in [0,1]]{f(\tau) = M}.
			\]
			$f^{-1}(\set{M})$ ist abgeschlossen und nichtleer (da $f(0)= u(t_0,x_0) = M$). 
			Außerdem ist $f^{-1}(\set{M})$ relativ offen (mithilfe der Mittelwerteigenschaft wie oben). Damit gilt
			\[
				f^{-1}(\set{M}) = [0,1] \qquad \Leftrightarrow \qquad u \equiv M \text{ auf } L.
			\]
			\item[$2$. Schritt] Wir betrachten einen beliebigen Punkt $(t^{*},x^{*}) \in \Omega_{t_0}$. 
			In diesem Fall kann $(t_0,x_0)$ mit $(t^{*},x^{*})$ über endlich viele Punkte $(t_1,x_1), \dots, (t_m,x_m)$ verbunden werden 
			(da $\Omega$ zusammenhängend ist). Wir setzen $t_{m+1}:= t^{*}$ und $x_{m+1}:= x^{*}$, so dass $t_{k+1} < t_k$ für alle $k \in \set{1,\dots,m}$ gilt
			und alle Linien $L((t_k,x_k),(t_{k+1},x_{k+1})) \subset \Omega_{t_k}$ liegen. Nach dem $1$. Schritt gilt $u \equiv M$ auf $L_k$ für alle $k$ und somit 
			$u(t^{*},x^{*}) = M$. Wegen der Stetigkeit von $u$ gilt dies auch für alle Punkte in $\set{t_0} \times \Omega$ und somit 
			\[
				u \equiv M \qquad \text{ auf } \Omega_{t_0}.
 			\] 
		\end{description}
	\end{description}
\end{beweis}

%%%% 02.06.2016- 13.06.2016
%%% Vorlesung 2.6.16 - 13.6.16

Wir betrachten das Anfangswertproblem

\[
	\begin{cases}
		u_t(t,x) - \Delta u(t,x) = f, &\text{ in }\Omega_T\\
		u = g, &\text{ auf }\partial_p \Omega_T
	\end{cases}
\]

\begin{korollar}[Eindeutigkeit]
	Sei $\Omega$ eine beschränkte, offene Teilmenge des $\mathbb{R}^n$, $T>0$ und $u,v \in C_1^2(\Omega_T) \cap C^0(\Omega_T)$ sodass
	\[
		\begin{cases}
			u_t- \Delta u = v_t - \Delta v, &\text{ in }\Omega_T\\
			u = v ,&\text{ auf } \partial_p \Omega_T
		\end{cases}
	\]
	Dann gilt \[
		u \equiv v
	\]
\end{korollar}
\begin{beweis}
	Sei $w:= v-u$. Dann gilt $w=0$ auf $ \partial_p \Omega_T$ damit gilt
	\[
		\max_{ \partial_p \Omega_T} w = \min_{ \partial_p \Omega_T} w = 0
	\]
	Damit ist $w$ kalonisch und mit dem Maximumsprinzip, angewandt auf $-w$ und $w$ folgt
	\[
		\max_{\bar{\Omega}_T} w = \min_{\bar{\Omega}_T} w = 0
	\]
	und somit
	\[
		w \equiv 0 \qquad \text{ in } \Omega_T.
	\]
\end{beweis}

\subsection{Darstellungsformel von Lösungen im Ganzraumfall $\Omega_T = \mathbb{R}^{+} \times \mathbb{R}^n$} 
\label{sub:darstellungsformel_von_losungen_im_ganzraumfall_omega_t_mathbb_r_times_mathbb_r_n}
Wir betrachten nun das Anfangswertproblem 
\[
	\begin{cases}
		u_t - \Delta u = f , &\text{ in }(0,\infty) \times \mathbb{R}^n,\\
		u = g, &\text{ auf } \set{0} \times \mathbb{R}^n,
	\end{cases}
\]
aufgeteilt mit $u = u_1 + u_2$ in 
\[
	(1)\begin{cases}
		{u_1}_t - \Delta u_1 = 0, &\text{ in }\mathbb{R}^{+} \times \mathbb{R}^n\\
		u_1 = g, &\text{ auf } \set{0} \times \mathbb{R}^n
	\end{cases}
\]
\[
	(2)\begin{cases}
		{u_2}_t - \Delta u_2 = f, &\text{ in }\mathbb{R}^{+} \times \mathbb{R}^n\\
		u_2 = 0, &\text{ auf } \set{0} \times \mathbb{R}^n
	\end{cases}
\]

\begin{satz}
	Sei $g \in L^p(\mathbb{R}^n)$ für ein $p \in [1, \infty]$. Dann ist die Funktion $u: \mathbb{R}^{+} \times \mathbb{R}^n \to \mathbb{R}$, die definiert ist durch
	\[
		u(t,x) := (\psi(t,\cdot)*g)(x) = \int_{\mathbb{R}^n}^{} \psi(t,x-y)g(y) \,\mathrm{d}y, 
	\]
	wobei $\psi$ die Fundamentallösung der Wärmeleitungsgleichung ist, von der Klasse \\ $C^{\infty}(\mathbb{R}^{+} \times \mathbb{R}^n)$ und erfüllt die Wärmeleitungsgleichung auf $\mathbb{R}^{+}\times \mathbb{R}^n$.
\end{satz}
\begin{beweis}
	Betrachte
	\[
		y \mapsto \psi(t,x-y) \in L^q(\mathbb{R}^n) \qquad \forall\, t>0,\,\, y \in \mathbb{R}^n, \,\, \forall\, q \in [1,\infty].	
	\]
	Es gilt
	\[
		\int_{\mathbb{R}^n}^{} \abs{ \psi(t,x-y)g(y)} \,\mathrm{d}y \stackrel{\text{Hölder}}{\leq} \norm{\psi}_{L^q(\mathbb{R}^n)} \norm{g}_{L^p(\mathbb{R}^n)}
	\]
	und damit ist
	\[
		y \mapsto \psi(t,x-y)g(y)
	\]
	integrierbar auf $\mathbb{R}^n$, für alle $t>0$, $x \in \mathbb{R}^n$. Damit ist $u$ wohldefiniert. Nun beweisen wir, dass $u \in C^{\infty}(\mathbb{R}^{+} \times \mathbb{R}^n)$. Es gilt
	\begin{itemize}
		\item $\psi \in C^{\infty}(\mathbb{R}^+ \times \mathbb{R}^n)$
		\item $\partial_t^k D_x^{\alpha}u(t,\cdot) \in L^q(\mathbb{R}^n), \qquad \forall\,  k \in \mathbb{N}, \alpha \in \mathbb{N}^n, \qquad \forall\,  q \in [1,\infty]$
	\end{itemize}
	Also ist
	\[
		y \mapsto \underset{\in L^q(\mathbb{R}^n)}{\underbrace{\partial_t^k D^{\alpha}_x \psi(t,x-y) }}\underset{\in L^p(\mathbb{R}^n)}{\underbrace{g(y)}}
	\]
	integrierbar auf $\mathbb{R}^n$, mit $\alpha \in \mathbb{N}^n, t>0, x \in \mathbb{R}^n$. Damit kann die Ableitung reingezogen werden und
	\[
		\partial_t^k D_x^{\alpha} u(t,x) = \int_{\mathbb{R}^n}^{} \partial_t^k D_x^{\alpha} \psi(t,x-y)g(y) \,\mathrm{d}y,
		\qquad \forall\, k \in \mathbb{N},\, \alpha \in \mathbb{N}^n
	\]
	Und somit
	\[
		u_t(t,x) - \Delta u(t,x) = \int_{\mathbb{R}^n}^{} \underset{=0}{\underbrace{(\psi_t(t,x-y) - \Delta_x \psi(t,x-y))}}g(y) \,\mathrm{d}y.
	\]
	Damit ist $u$ eine Lösung der Wärmeleitungsgleichung.
\end{beweis}

\begin{satz}
	Bezeichnet $u$ die Funktion aus Satz $3.10$ mit $g \in C^{\infty}(\mathbb{R}^n) \cap L^{\infty}(\mathbb{R}^n)$, so gilt die folgende Konvergenz:
	\[
		\lim_{(t,x) \to (0,x_0)} u(t,x)=g(x_0), 
	\]
	für alle $x_0 \in \mathbb{R}^n$.
\end{satz}
\begin{beweis}
	Sei $x_0 \in \mathbb{R}^n$, $\varepsilon >0$, $\delta >0$:
	\[
		\abs{g(x_0)-g(y)} < \varepsilon, \qquad \forall\, y: \abs{y-x_0} < \delta 
	\]
	Wir wollen nun $\abs{u(t,x)-g(x_0)} \to 0$ für $(t,x) \to (0,x_0)$. \\
	Für $x \in B_{\frac{\delta }{2}}$ gilt
	\begin{align*}
		\abs{u(t,x)-g(x_0)} &= \abs{\int_{\mathbb{R}^n}^{} \abs{\psi(t,x-y)(g(y)-g(x_0))} \,\mathrm{d}y} \\
		&\leq \int_{\mathbb{R}^n}^{} \abs{\psi(t,x-y)(g(y)-g(x_0))} \,\mathrm{d}y \\
		&\leq \int_{B_{\delta}(x_0)}^{} \psi(t,x-y) \abs{g(y)-g(x_0)} \,\mathrm{d}y \\
		& \qquad \qquad + \int_{\mathbb{R}^n \setminus B_{\delta }(x_0)}^{} \abs{ \psi(t,x-y) (g(y)-g(x_0))} \,\mathrm{d}y \\
		& \leq \varepsilon \underset{=1}{\underbrace{\int_{\mathbb{R}^n}^{} \psi(t,x-y) \,\mathrm{d}y}} 
		+ \int_{\mathbb{R}^n \setminus B_{\delta }(x_0)}^{} \psi(t,x-y)\abs{g(y)-g(x_0)} \,\mathrm{d}y \\
		&\leq  \varepsilon + 2 \norm{g}_{L^{\infty}(\mathbb{R}^n)} \int_{\mathbb{R}^n \setminus B_{\delta }(x_0)}^{} \psi(t,x-y) \,\mathrm{d}y
	\end{align*}
	Betrachte nun, dass wegen $\abs{y-x_0} < \delta$ und 
	\[
		\abs{y-x_0} \leq \abs{y-x} + \abs{x-x_0} \leq \abs{y-x}+ \frac{\delta }{2} < \abs{y-x} + \frac{1}{2}\abs{y-x}
	\]
	\[
		\frac{1}{2} \abs{y-x_0}< \abs{y-x}
	\]
	Weiter gilt
	\begin{align*}
		\int_{\mathbb{R}^n \setminus B_{\delta }(x_0)}^{} \psi(t,x-y) \,\mathrm{d}y 
		&= \frac{1}{(4 \pi t)^{\frac{n}{2}}} \int_{\mathbb{R}^n \setminus B_{\delta }(x_0)} e^{\frac{-\abs{x-y}^2}{4t^2}} \,\mathrm{d}y \\
		&\leq \frac{1}{(4 \pi t)^{\frac{n}{2}}} \int_{\mathbb{R}^n \setminus B_{\delta }(x_0)}^{} e^{\frac{-\abs{y-x_0}^2}{16t^2}} \,\mathrm{d}y
	\end{align*}
	Wir wenden nun die Transformation $z:= \frac{y-x_0}{\sqrt{t}}$ an und erhalten
	\[
		\int_{\mathbb{R}^n \setminus B_{\delta }(x_0)}^{} \psi(t,x-y) \,\mathrm{d}y= \frac{1}{(4 \pi t)^{\frac{n}{2}}} 
		\int_{\mathbb{R}^n \setminus B_{\frac{\delta }{\sqrt{t}}}(0)}^{} e^{- \frac{\abs{z}^2}{16}} \,\mathrm{d}z \, (t)^{\frac{n}{2}}.
	\]
	Da $e^{- \frac{\abs{z}^2}{16}}$ integrierbar über $\mathbb{R}^n$ ist, gilt 
	\[
		\int_{\mathbb{R}^n \setminus B_{\frac{\delta }{\sqrt{t}}}(0)}^{} e^{- \frac{\abs{z}^2}{16}} \,\mathrm{d}z \stackrel{t \to 0^+}{\to } 0
	\]
	Damit finden wir $\tilde \delta >0$, so dass 
	\[
		\int_{\mathbb{R}^n \setminus B_{\frac{\tilde \delta }{\sqrt{t}}}(0)}^{} e^{- \frac{\abs{z}^2}{16}} \,\mathrm{d}z < \varepsilon,
		 \qquad \forall\, t \in (0,\tilde \delta ]
	\]
	Und somit
	\[
		\abs{u(t,x)-g(x_0)} \leq 2 \varepsilon, \qquad \forall\, x \in B_{\frac{\delta }{2}(x_0)}, \,t \in (0, \tilde \delta]
	\]
	\[
		\Rightarrow \lim_{\substack{t \to 0^+ \\ x \to x_0}} \abs{u(t,x)-g(x_0)} = 0.
	\]
	Damit ist
	\[
		u(t,x)= \int_{\mathbb{R}^n}^{} \psi(t,x-y)g(y) \,\mathrm{d}y
	\]
	eine Lösung von dem Anfangswertproblem $(1)$.
\end{beweis}
\begin{satz}[Duhamelsches Prinzip]
	Sei $f \in C_1^2(\mathbb{R}^{+} \times \mathbb{R}^n)$ mit $\supp(f) \subset \subset [0,\infty] \times \mathbb{R}^n$. Dann ist die Funktion $u : \mathbb{R}^{+} \times \mathbb{R}^n \to \mathbb{R}$, die definiert ist durch
	\[
		u(t,x)= \int_{0}^{t} \int_{\mathbb{R}^n}^{} \psi(t-s,x-y)f(s,y) \,\mathrm{d}y \,\mathrm{d}s
	\]
	von der Klasse $C_1^2(\mathbb{R}^{+} \times \mathbb{R}^n) \cap C^0([0,\infty)\times \mathbb{R}^n)$ und erfüllt
	\begin{enumerate}[(i)]
		\item $u_t- \Delta u =f$ in $\mathbb{R}^{+} \times \mathbb{R}^n$
		\item $\lim\limits_{(x,t) \to (x_0,0)} u(t,x) =0$ für alle $x_0 \in \mathbb{R}^n$
	\end{enumerate}
	Dies bedeutet, dass $u$ eine Lösung von $(2)$ ist.
\end{satz}
\begin{beweis}
	Es gilt
	\[
		u(t,x)= \int_{0}^{t} \int_{\mathbb{R}^n}^{} \psi(s,y)f(t-s,x-y) \,\mathrm{d}y \,\mathrm{d}s
	\]
	Die Leipnitzregel für Parameterintegrale besagt:
	\[
		\diffd{}{t} \int_{a(t)}^{b(t)} g(t,x) \,\mathrm{d}x = \int_{a(t)}^{b(t)} g_t(t,x) \,\mathrm{d}x + g(t,b(t))b'(t) - g(t,a(t))a'(t).
	\]
	Somit
	\begin{align*}
		u_t(t,x) &= \int_{0}^{t} \int_{\mathbb{R}^n}^{} \psi(s,y)f_t(t-s,x-y) \,\mathrm{d}y \,\mathrm{d}s +
		\int_{\mathbb{R}^n}^{}\psi(t,y)f(0,x-y) \,\mathrm{d}y \\
		\diff{^2u(t,x)}{x_i \partial x_j} &= \int_{0}^{t} \int_{\mathbb{R}^n}^{} \psi(s,y) \partial_{x_i} \partial_{x_j} f(t-s,x-y) \,\mathrm{d}y \,\mathrm{d}s
	\end{align*}
	und damit $u \in C^2_1(\mathbb{R}^{+} \times \mathbb{R}^n)$. 
	Für die $C^0([0,\infty)\times \mathbb{R}^n)$-Regularität müssen wir die stetige Fortsetzbarkeit für $t \to 0^+$ durch $0$ nachweisen. \\
	Es gilt
	\[
		\abs{u(t,x)} \leq  \int_{0}^{t} \underset{=1}{\underbrace{\int_{\mathbb{R}^n}^{} \psi(s,y) \,\mathrm{d}s}} \,\mathrm{d}y \, \norm{f}_{L^{\infty}(\mathbb{R}^n)} 
		\leq t \norm{f}_{L^{\infty}(\mathbb{R}^n)} \stackrel{t \to 0}{\to} 0
	\]
	Damit folgt $u \in C^0([0,\infty) \times \mathbb{R}^n)$ und damit auch (ii). Wir beweisen nun noch (i). \\
	Für $\varepsilon \in (0,t)$ gilt
	\begin{align*}
		u_t(t,x)- \Delta u(t,x) &= \int_{0}^{t} \int_{\mathbb{R}^n}^{} \psi(s,y)(\underset{=-f_s(t-s,x-y)}{\underbrace{f_t(t-s,x-y)}} 
		+ \underset{=-\Delta_y f(t-s,x-y)}{\underbrace{ \Delta_x f(t-s,x-y))}} \,\mathrm{d}y \,\mathrm{d}s \\
		& \qquad \qquad + \int_{\mathbb{R}^n}^{} \psi(t,y) f(0,x-y) \,\mathrm{d}y \\
		&= \int_{0}^{\varepsilon} \int_{\mathbb{R}^n}^{} \psi(s,y) (-f_s(t-s,x-y) - \Delta_y f(t-s,x-y)) \,\mathrm{d}y \,\mathrm{d}s \\
		& \qquad + \int_{\varepsilon}^{t} \int_{\mathbb{R}^n}^{} \psi(s,y) (-f_s(t-s,x-y) - \Delta_y f(t-s,x-y)) \,\mathrm{d}y \,\mathrm{d}s \\
		& \qquad + \int_{\mathbb{R}^n}^{} \psi(t,y)f(0,x-y) \,\mathrm{d}y \\
		&=: A_{\varepsilon}+ B _{\varepsilon} + C.
	\end{align*}
	Zunächst gilt 
	\[
		\abs{A _{\varepsilon}} \leq ( \underset{=:C'}{\underbrace{\norm{f_s}_{L^{\infty}(\mathbb{R}^{+} \times \mathbb{R}^n)} +
		 \norm{D^{\alpha}f}_{L^{\infty}(\mathbb{R}^{+} \times \mathbb{R}^n)}}}) \cdot \int_{0}^{\varepsilon} \int_{\mathbb{R}^n}^{} 
		 \psi(s,y) \,\mathrm{d}s \,\mathrm{d}y = \varepsilon C' \stackrel{\varepsilon \to 0}{\to } 0
	\]
	und außerdem mit partieller Integration und Leipnizregel
	\begin{align*}
			B _{\varepsilon} &= \int_{\varepsilon}^{t} \int_{\mathbb{R}^n}^{}
		 (\underset{\substack{=0, \\ \text{da $\psi$ kalonisch}}}{\underbrace{ \partial_s \psi(s,y) - \Delta \psi(s,y)}})
		 f(t-s,x-y) \,\mathrm{d}y \,\mathrm{d}s \\ & \qquad - \int_{\mathbb{R}^n}^{} \psi(t,y) f(0,x-y) \,\mathrm{d}y 
		 + \int_{\mathbb{R}^n}^{} \psi(\varepsilon,y)f(t-\varepsilon,x-y) \,\mathrm{d}y \\
		 &= -C + \int_{\mathbb{R}^n}^{} \psi(\varepsilon,y)(f(t-\varepsilon,x-y)-f(t,x-y)) \,\mathrm{d}y \\
		 & \qquad + \int_{\mathbb{R}^n}^{} \psi(\varepsilon,y)(f(t,x-y)) \,\mathrm{d}y.
	\end{align*}
	Insgesamt erhalten wir
	\begin{align*}
		u_t - \Delta u &= \lim_{\varepsilon \to 0} ( \underset{=(I)}{\underbrace{\int_{\mathbb{R}^n}^{} \psi(\varepsilon,y)(f(t-\varepsilon,x-y)-f(t,x-y)) 
		\,\mathrm{d}y}} \\ & \qquad \underset{=(II)}{\underbrace{\int_{\mathbb{R}^n}^{} \psi(\varepsilon,x-y) f(t,y) \,\mathrm{d}y}} ).
	\end{align*}
	\[
		(I) \leq \sup_{z \in \mathbb{R}^n} \abs{f(t-\varepsilon,z)-f(t,z)} \to 0
	\]
	für $\varepsilon \to 0$, da $f$ gleichmäßig stetig ist. Und wegen Satz $3.11$ gilt
	\[
		(II) \to f(t,x)
	\]
	Insgesamt ergibt dies die Behauptung
\end{beweis}
\begin{satz}[Allgemeine Lösungsformel]
	Seien $g \in C^0(\mathbb{R}^n) \cap L^{\infty}(\mathbb{R}^n)$ und \\ $f \in C_1^2([0,\infty]\times \mathbb{R}^n)$ mit $\supp(f) \subset \subset [0,\infty) \times \mathbb{R}^n$. Dann ist die Funktion $u: \mathbb{R}^{+} \times \mathbb{R}^n \to \mathbb{R}$, definiert durch
	\[
		u(t,x)= \int_{\mathbb{R}^n}^{} \psi(t,x-y)g(y) \,\mathrm{d}y + \int_{0}^{t}\int_{\mathbb{R}^n}^{} \psi(t-s,x-y)f(s,y) \,\mathrm{d}y \,\mathrm{d}s
	\]
	von der Klasse $C_1^2(\mathbb{R}^{+} \times \mathbb{R}^n) \cap C^0(\mathbb{R}^{+} \times \mathbb{R}^n)$ und erfüllt das Anfangswertproblem
	\[
		\begin{cases}
			u_t- \Delta u =f, &\text{ in }\mathbb{R}^{+} \times \mathbb{R}^n\\
			u = g, &\text{ auf } \set{0} \times \mathbb{R}^n
		\end{cases}
	\]
\end{satz}
%%% neue Vorlesung

Energiemethode 
\[
	\begin{cases}
		u_t - \Delta u = f, &\text{ in }\Omega_T, f \in C^0(\Omega_T), g \in C^0(\partial_p \Omega_T), T >0\\
		u = g ,&\text{ auf } \partial_p \Omega_T, \Omega \subseteq \mathbb{R}^n \text{ offen, beschränkt und mit $C^1$-Rand}
	\end{cases}
\]
\begin{satz}
	Seien $u,v$ zwei $C^2(\overline{\Omega_T})$ Lösungen von (P). Dann gilt $u \equiv v$.
\end{satz}
\begin{beweis}
	$w:= u -v$. Dann gilt 
	\[
		(P)'\begin{cases}
			w_t -\Delta w = 0, &\text{ in }\Omega_T\\
			w = 0, &\text{ auf } \partial_p \Omega_T
		\end{cases}
	\]
	Sei
	\[
		e(t):= \int_{\Omega}^{} w^2(t,x) \,\mathrm{d}x, \qquad \forall\, t \in [0,T].
	\]
	Betrachte
	\begin{align*}
		\diffd{e(t)}{t} & \stackrel{\hphantom{(P)'}}{=} \int_{\Omega}^{} 2 w(t,x) w_t(t,x) \,\mathrm{d}x  \\
		&\stackrel{(P)'}{=} 2 \int_{\Omega}^{} w(t,x) \Delta w(t,x) \,\mathrm{d}x \\
		&\stackrel{\text{P.I.\hphantom{...}}}{=} -2 \int_{\Omega}^{}  \nabla  w(t,x) \cdot  \nabla w(t,x) \,\mathrm{d}x \\
		&\stackrel{\hphantom{(P)'}}{=} -2 \int_{\Omega}^{} \abs{ \nabla w(t,x)}^2 \,\mathrm{d}x \leq 0
	\end{align*}
	Damit ist $e(t)$ monoton fallend und so folgt
	\[
		e(t) \leq e(0), \qquad \forall\, t \in [0,T].
	\]
	\begin{align*}
		& \qquad e(0) = \int_{\Omega}^{} w^2(0,x) \,\mathrm{d}x \stackrel{\substack{\text{Rand-}\\\text{bedingung}}}{=} 0 \\
		\Rightarrow& \qquad 0 \leq e(t) \leq 0 \\
		\Rightarrow & \qquad  e(t) = 0, \qquad \forall\, t \\
		\Rightarrow & \qquad  w \equiv 0  
	\end{align*}
\end{beweis}
\newpage
\section{Die Wellengleichung (linear, hyperbolisch)} 
\label{sec:die_wellengleichung_linear_hyperbolisch}

Für $t>0$, $x \in \Omega$ und $\Omega \subseteq \mathbb{R}^n$ offen lautet die homogene Wellengleichung
\[
	u_{tt} - \Delta u = 0.
\]
Die inhomogene hingegen
\[
	u_{tt} - \Delta u = f,
\]
wobei $u = u(t,x)$ und $ \Delta u = \Delta_x u$. \\
Betrachte das Problem

\[
	(P) \qquad  \begin{cases}
		u_{tt} - \Delta u = 0, &\text{ in }(0,\infty) \times \mathbb{R}^n\\
		u = g, \,\, u_t = h ,&\text{ auf } \set{0} \times \mathbb{R}^n \\
		u(0,x) = g(x), \,\, u_t(0,x) = h(x) ,&\text{ für } x \in \mathbb{R}^n
	\end{cases}
\]

 \subsection{Darstellungsformel für die Lösung im Grenzfall $(0,\infty) \times \mathbb{R}^n$} 
 \label{sub:darstellungsformel_fur_die_losung_im_grenzfall}

Für $n=1$ betrachten wir
\[
	(P) \qquad  \begin{cases}
		u_{tt}- u_{xx} = 0, &\text{ in }(0,\infty) \times \mathbb{R}\\
		u(0,x) = g(x), \,\,u_t(0,x) = h(x) ,&\text{ für } x \in \mathbb{R}
	\end{cases}
\]

\begin{align*}
u \text{ ist eine Lösung von (P)} &\Leftrightarrow u_{tt}-u_{xx}=0 + \text{RB} \\
& \Leftrightarrow ( \partial_t + \partial_x)( \partial_t - \partial_x)u = 0 + \text{RB} \\
& \Rightarrow \begin{cases}
	\partial_t v + \partial_x v =0, &\text{ für }t>0, x \in \mathbb{R}\\
	v(0,x)= (h-g')(x), &\text{ für }x \in \mathbb{R}
\end{cases}(*)
\end{align*} 
Dies ist die homogene Transportgleichung in einer Raumdimension.

\begin{satz}[Transportgleichung]
	Seien $b \in \mathbb{R}^n$. $ \tilde f \in C^1( (0, \infty) \times \mathbb{R}^n)$ und \\ $ \tilde g \in C^1((0,\infty) \times \mathbb{R}^n)$. Das Anfangswertproblem für die inhomogene Transportgleichung
	\[
		\begin{cases}
			u_t + b \cdot  \nabla u = \tilde f, &\text{ in }(0,\infty) \times \mathbb{R}^n\\
			u = \tilde g, &\text{ auf } \set{0} \times \mathbb{R}^n
			
		\end{cases}
	\]
	bestitzt eine eindeutige Lösung in $C^1((0,\infty) \times \mathbb{R}^n)$ die gegeben ist durch
	\[
		u(t,x) = \tilde g(x-tb) + \int_{0}^{t} \tilde f(s,x+(s-t)b) \,\mathrm{d}s
	\]
\end{satz}
\begin{beweis}
	Blatt 2, Aufg 3.
\end{beweis}
Aus (*) und Satz $4.1$ folgt bei $b = 1$
\[
	v(t,x) = (h-g')(x-t) 
\]
für $t>0$ und $x \in \mathbb{R}^n$. Also
\[
	v(t,x) = \partial_t v(t,x) - \partial_x v(t,x) = h(x-t) - g'(x-t)
\]
für $t>0$, $x \in \mathbb{R}$, $u(0,x) = g(x)$ für $x \in \mathbb{R}$.
Dies ist die inhomogene Transportgleichung in einer Raumdimension. \\
Aus Satz $4.1$ folgt bei $b = -1$
\begin{align*}
	u(t,x) &= g(x+t) + \int_{0}^{t}(h(\underset{\tilde s := x -2s + t}{\underbrace{x-(s-t)-s}}) - g'(x-(s-t)-s)) \,\mathrm{d}s \\
	&= g(x+t) - \frac{1}{2} \int_{x-t}^{x+t} h(\tilde s) \,\mathrm{d}\tilde s - \frac{1}{2} \int_{x-t}^{x+t} g'(\tilde s) \,\mathrm{d} \tilde s \\
	&= g(x+t) + \frac{1}{2} \int_{x-t}^{x+t} h(\tilde s) \,\mathrm{d}\tilde s + \frac{1}{2} (g(x-t)-g(x+t))
\end{align*}
Es folgt
\[
	u(t,x) = \frac{1}{2} (g(x+t)+ g(x-t)) + \frac{1}{2} \int_{x-t}^{x+t}h(\tilde s) \,\mathrm{d}\tilde s
\]
Dies ist die d'Alembertsche Formel.
\begin{satz}
	Seien $g \in C^2(\mathbb{R})$, $h \in C^1(\mathbb{R})$ und $u: [0, \infty) \times \mathbb{R} \to  \mathbb{R}$ definiert durch
	\[
		u(t,x)= \frac{1}{2} (g(x+t)+ g(x-t))+ \frac{1}{2} \int_{x-t}^{x+t} h(s) \,\mathrm{d}s
	\]
	Dann gelten
	\begin{enumerate}[(i)]
		\item $u$ ist von der Klasse $C^2((0,\infty) \times \mathbb{R})$ 
		\item $u$ löst die homogene Wellengleichung auf $(0,\infty) \times \mathbb{R}$
		\item $u=g$ und $u_t = h$ auf $\set{0} \times \mathbb{R}$ (und $u$ ist damit die eindeutige Lösung von (P) für $n=1$)
	\end{enumerate}
\end{satz}
\begin{bemerkung}
	\begin{enumerate}[(i)]
		\item Die Lösung $u$ ist von der Form
		\[
			u(t,x) = u_1(x+t) + u_2(x-t)
		\]
		für $u_1, u_2 \in C^2((0,\infty) \times \mathbb{R})$, besteht also aus eine nach links und einer nach rechts laufenden Welle. Jede Funktion $u$ dieser Form löst
		die homogene Wellengleichung. Damit setzt sich die allgemeine Lösung der Wellengleichung für $n=1$ immer aus Lösungen der beiden Gleichungen erster Ordnung 
		\begin{align*}
			\partial_t u_1 + \partial_x u_1 &= 0 \\
			\partial_t u_2 + \partial_x u_2 &= 0 
		\end{align*}
		zusammen.
		\item Es gibt keinen Regularisierungseffekt der Wellengleichung (im Gegensatz zur Laplace- oder Wärmeleitungsgleichung). Für $g \in C^2$ und $h \in C^1$ ist 
		$u$ lediglich von der Klasse $C^2([0, \infty] \times \mathbb{R})$, aber nicht "besser".
	\end{enumerate}
\end{bemerkung}
\begin{korollar}
	Seien $g \in C^2([0, \infty))$ mit $g''(0)=0 = g(0)$ und $h \in C^1([0,\infty))$ mit $h(0)=0$. Dann ist die Funktion $u$, definiert durch 
	\[
		u(t,x) = \begin{cases}
			\frac{1}{2}(g(x+t)+g(x-t)) + \frac{1}{2} \int_{x-t}^{x+t}h(s) \,\mathrm{d}s, &\text{ falls } x \geq t \geq 0\\
			\frac{1}{2}(g(x+t)-g(t-x)) + \frac{1}{2} \int_{t-x}^{x+t}h(s) \,\mathrm{d}s, &\text{ falls } t \geq x \geq 0
		\end{cases}
	\]
	die eindeutige $C^2([0,\infty) \times \mathbb{R})$- Lösung von 
	\[
		u_{tt}- u_{xx} = 0 \text{ in } [0, \infty) \times (0,\infty), \qquad u(0,x) = g(x)
	\]
	\[
		u_t(0,x) = h(x), \qquad x \in (0, \infty), \qquad u(t,0) = 0, \qquad t \geq 0
	\]
\end{korollar}
\begin{beweis}

Wir setzen $g$, $h$ durch ungerade Spiegelung fort, d.h wir definieren
\[
	\bar{g} (x) = \begin{cases}
		g(x), &\text{ für }x > 0,\\
		-g(-x), &\text{ für }x <0
		
	\end{cases}
\]
\[
	\bar{h} (x) = \begin{cases}
		h(x), &\text{ für }x \geq 0,\\
		-h(-x), &\text{ für }x <0
		
	\end{cases}
\]
Es ist
\[
	\bar{g} (0) = \lim_{x \to 0^-} \bar{g} (x) = 0 \qquad \Rightarrow \qquad \bar{g} \in C^0(\mathbb{R})
\]
\[
	\bar{g}'(x) = \begin{cases}
		g'(x), &\text{ für }x \geq 0\\
		g'(-x), &\text{ für }x <0
		
	\end{cases} \qquad \Rightarrow \qquad g' \in C^0(\mathbb{R}),
\]	
Es folgt
\[
	\bar{g}'' \in C^0(\mathbb{R}) \qquad \Rightarrow \qquad \bar{g} \in C^2(\mathbb{R}), \qquad h \in C^1(\mathbb{R}).
\]
\[
	\Rightarrow u(t,x) = \frac{1}{2} (\bar{g} (x+t) + \bar{g}(x-t)) + \frac{1}{2} \int_{x-t}^{x+t} \bar{h}(s) \,\mathrm{d}s
\]
ist eine Lösung von $u_{tt}- u_{xx} =0$ in $(0,\infty) \times \mathbb{R}$ und somit auch in $(0, \infty) \times (0, \infty)$. \\
Für $x \geq 0$ betrachten wir zwei Fälle
\begin{align*}
	x-t \geq 0 & \Leftrightarrow x \geq t \geq 0 \\
	& \Rightarrow u(t,x)= \frac{1}{2} (g(x+t) + g(x-t)) + \frac{1}{2} \int_{x-t}^{x+t} h(s) \,\mathrm{d}s
\end{align*}
\begin{align*}
	x-t \leq  0 & \Leftrightarrow t \geq x \geq 0 \\
	& \Rightarrow u(t,x)= \frac{1}{2} (g(x+t) - g(t-x)) + \frac{1}{2} \int_{x-t}^{x+t} \bar{h}(s) \,\mathrm{d}s
\end{align*}
Außerdem gilt
\begin{align*}
	\frac{1}{2} \int_{x-t}^{x+t} \bar{h}(s) \,\mathrm{d}s &= \frac{1}{2} \left( \int_{0}^{x+t}\bar{h}(s) \,\mathrm{d}s + \int_{x-t}^{0} \bar{h}(s) \,\mathrm{d}s \right)
	\\ &= \frac{1}{2} \left( \int_{0}^{x+t}h(s) \,\mathrm{d}s + \int_{x-t}^{0}-h(-s) \,\mathrm{d}s \right) \\ &= \frac{1}{2} \int_{t-x}^{x+t}h(s) \,\mathrm{d}s
\end{align*}
\[
	u(t,0) = \frac{1}{2} (g(t) -g(t)) = 0
\]
\end{beweis}

\subsubsection{Die Methode der sphärischen Mittelwerte für $n \geq 2$} 
\label{ssub:die_methode_der_spharischen_mittelwerte_fur_n_geq_2}
\[
	(P) \qquad  \begin{cases}
		u_{tt}- \Delta u = 0, &\text{ in }(0,\infty) \times \mathbb{R}^n\\
		u(0,x) = g(x), &\text{ für } x \in \mathbb{R}^n \\
		u_t(0,x)=h(x), &\text{ für } x \in \mathbb{R}^n
		\end{cases}
\]
Für $x \in \mathbb{R}^n$, $r>0$, $t>0$ definiere
\begin{align*}
	U_x(t,r) &:= \fint_{\partial B_r(x)}^{} u(t,y) \,\mathrm{d}S(y), \\
	G_x(r) &:= \fint_{\partial B_r(x)}^{} g(y) \,\mathrm{d}S(y), \\
	H_x(r) &:= \fint_{\partial B_r(x)}^{} h(y) \,\mathrm{d}S(y)
\end{align*}

\begin{lemma}[Euler-Poisson-Darbaux-Gleichung]
	Sei $u \in C^m((0,\infty) \times \mathbb{R}^n)$ mit $m \geq 2$ eine Lösung von (P). Für alle $ x \in \mathbb{R}^n$ gilt dann $U_x \in C^m((0,\infty) \times [0,\infty))$ und $U_x$ erfüllt das Anfangswertproblem 
	\begin{align*}
		\partial_t^2 U_x(t,r)- \partial_r^2 U_x(t,r) - \frac{n-1}{r} \partial_r U_x(t,r) &= 0 \qquad \text{ in } \mathbb{R}^{+} \times \mathbb{R}^{+} \\
		U_x(0,r)&= G_x(r), \qquad r>0 \\
		\partial U_x(0,r) &= H_x(r), \qquad r>0
	\end{align*}
\end{lemma}
\begin{beweis}
Wir beweisen $U_x \in C^m((0,\infty) \times \mathbb{R}^n)$:
\[
	U_x(t,r) = \fint_{\partial B_1(0)}^{} u (t,x+r \tilde y)r^n \,\mathrm{d}S(\tilde y)
\]	
$B_1(0)$ ist beschränkt
\[
	\Rightarrow \qquad \diff{^{k+l}}{t^k \partial r^l} U_x(t,r) = \fint_{\partial B_1(0)}^{} \diff{^{k+l}}{t^k \partial r^l} u(t,x+r \tilde y) \,\mathrm{d}S(\tilde y)
\]
ist stetig für $k+l \leq m$ und daraus folgt die Behauptung. \\ 
Für den Beweis der Anfangsdaten gilt
\[
	U_x(0,r) = \fint_{\partial B_r(x)}^{} u(0,y) \,\mathrm{d}S(y) = \fint_{\partial B_r(x)}^{} g(y) \,\mathrm{d}S(y) = G_x(r)
\]
und
\begin{align*}
	\partial_r U_x(0,r)& = \fint_{\partial B_1(0)}^{} \partial_t u(0,x+ r \tilde y) r^n \,\mathrm{d}S(\tilde y) \\ &= 
	\fint_{\partial B_1(0)}^{} h(x+ r \tilde y)r^n \,\mathrm{d}S(\tilde y) \\ &= \fint_{\partial B_r(x)}^{} h(y) \,\mathrm{d}S(y) \\ &= H_x(r).
\end{align*}
Nun gilt
\[
	\partial_t^2U_x(t,x) = \fint_{\partial B_r(x)}^{} \partial_t^2 u(t,y) \,\mathrm{d}S(y)
\]
\begin{align*}
	\partial_r U_x(t,r) &\stackrel{\text{Lemma }2.3}{=} \frac{r}{n} \fint_{B_r(x)}^{} \Delta u(t,y) \,\mathrm{d}y \\
	& \stackrel{\hphantom{\text{Lemma }2.3}}{=} \frac{r}{n} \frac{1}{\omega_n r^n} \int_{0}^{r} \int_{\partial B_{\rho}(x)}^{} \Delta u(t,y) \,\mathrm{d}S(y) \,\mathrm{d}\rho \\
	& \stackrel{\hphantom{\text{Lemma }2.3}}{=} \frac{r^{1-n}}{n \omega_n} \int_{0}^{r} \int_{\partial B_{\rho}(x)}^{} \Delta u(t,y) \,\mathrm{d}S(y) \,\mathrm{d}\rho
\end{align*}

\begin{align*}
	\partial_r ( \partial_r U_x(t,r)) &= \frac{1-n}{n \omega_n r^n} \int_{B_r(x)}^{} \Delta u(t,y) \,\mathrm{d}y 
	+ \frac{r^{1-n}}{n \omega_n} \int_{\partial B_r(x)}^{} \Delta u(t,y) \,\mathrm{d}S(y) \\
	&= \frac{1-n}{n} \fint_{B_r(x)}^{} \Delta u(t,y) \,\mathrm{d}y + \fint_{\partial B_r(x)}^{} \Delta u(t,y) \,\mathrm{d}S(y) \\
	&= \frac{1-n}{r} \partial_r U_x(t,r) + \fint_{\partial B_r(x)}^{} \Delta u(t,y) \,\mathrm{d}S(y) \\
	&= \frac{1-n}{r} \partial_r U_x(t,r) + \partial_t^2 U_x(t,r)
\end{align*}
\end{beweis}
Wir definieren nun eine Funktion $\tilde U_x$, so dass
\[
	\partial_t^2 \tilde U_x(t,r) - \partial_r^2 \tilde U_x(t,r) = 0 \qquad \text{ in } (0,\infty) \times (0,\infty)
\]
$n=3:$
\begin{align*}
	\tilde U_x(t,r) &:= r U_x(t,r), \\
	\tilde G_x(r) &:= r G_x(r), \\
	\tilde H_x(r) &:= r H_x(r).
\end{align*}
\begin{korollar}
	Sei $u \in C^m([0,\infty)\times \mathbb{R}^3)$ mit $m \geq 2$ eine Lösung von (P). Für alle $x \in \mathbb{R}^3$ gilt dann $ \tilde U_x \in C^m([0,\infty) \times [0, \infty))$ und $\tilde U_x$ erfüllt das Anfangsrandwertproblem
	\[
		\begin{cases}
			\partial_t^2 \tilde U_x(t,r) - \partial^2_r \tilde U_x(t,r)=0, &\text{ in }(0, \infty) \times (0, \infty)\\
			\tilde U_x(0,r) = \tilde G_x(r), \, \partial \tilde U_x(0,r) = \tilde H_x(r), &\text{ für }r >0 \\
			\tilde U_x(t,0) = 0 , &\text{ für }t>0
		\end{cases}
	\]
\end{korollar}
\begin{beweis}
	$\tilde U_x \in C^m$ funktioniert wie in Lemma $4.4$ und die Anfangs und Randbedingungen sind klar.
	\begin{align*}
			\partial^2_t \tilde U_x(t,r) & \stackrel{\hphantom{\text{Lemma }4.4}}{=} r \partial_t^2 U_x(t,r) \\
			&\stackrel{\text{Lemma }4.4}{=} r [ \partial_r^2 U_x(t,r) + \frac{2}{r} \partial_r U_x(t,r)] \\
			& \stackrel{\hphantom{\text{Lemma }4.4}}{=} r \partial_r^2 U_x(t,r) + 2 \partial_rU_x(t,r) \\
			& \stackrel{\hphantom{\text{Lemma }4.4}}{=} \partial_r(r \partial_r U_x(t,r)+ U_x(t,r)) \\
			& \stackrel{\hphantom{\text{Lemma }4.4}}{=} \partial_r (\partial_r (\underset{= \tilde U_x(t,r)}{\underbrace{r U_x(t,r)}})) \\
			& \stackrel{\hphantom{\text{Lemma }4.4}}{=} \partial_r^2 \tilde U_x(t,r)
	\end{align*}
\end{beweis}
Wir bemerken, dass 
\[
	\lim_{r \to 0} \tilde G_x(r) = \lim_{r \to 0} \tilde G_x''(r) \qquad , \qquad \lim_{r \to 0} \tilde H_x(r) = 0
\]

Wir dürfen die Darstellungsformel für Lösungen der Wellengleichung auf $(0,\infty) \times (0,\infty)$ (Korollar 4.3) verwenden. \\
Damit erhalten wir für $t \geq r > 0$
\[
	\tilde U_x(t,r) = \frac{1}{2} ( \tilde G_x(t-r)- \tilde G_x(t-r)) + \frac{1}{2} \int_{t-r}^{t+r} \tilde H_x(s) \,\mathrm{d}s
\]
Da $u$ stetig ist, gilt
\[
	u(t,x)= \lim_{r \to 0} \fint_{\partial B_r(x)}^{} u(t,y) \,\mathrm{d}S(y) = \lim_{r \to 0} U_x(t,r)
\]
\begin{align*}
	u(t,x) &= \lim_{r \to 0} \left( \frac{\tilde G_x(t+r)- \tilde G_x(t-r)}{2 r} + \frac{1}{2r} \int_{t-r}^{t+r} \tilde H_x(s) \,\mathrm{d}s \right) \\
	&= \lim_{r \to 0} \left( \frac{\tilde G_x(t+r) - \tilde G_x(t-r)}{2r} + \fint_{t-r}^{t+r} \tilde H_x(s) \,\mathrm{d}s \right) \\
	&= \tilde G_x'(t) + \tilde H_x(t) \\
	&= \partial_t \left( t \fint_{\partial B_t(x)}^{} g(y) \,\mathrm{d}S(y) \right) + t \fint_{\partial B_t(x)}^{} h(y) \,\mathrm{d}S(y)
\end{align*}
Wir berechnen nun 
\begin{align*}
	\partial_t \left(t \fint_{\partial B_t(x)}^{} g(y) \,\mathrm{d}S(y) \right)
\end{align*}
\begin{align*}
 t \fint_{\partial B_t(x)}^{} g(y) \,\mathrm{d}S(y)
	\stackrel{y= x+t \tilde y}{=} \fint_{\partial B_1(0)}^{} g(x + t \tilde y) \,\mathrm{d}S( \tilde y)
\end{align*}
Damit gilt
\begin{align*}
	\partial_t \left(t \fint_{\partial B_t(x)}^{} g(y) \,\mathrm{d}S(y) \right) & 
	\stackrel{\hphantom{\tilde y = \frac{y-x}{t}}}{=} \fint_{\partial B_1(0)}^{} g(x+t \tilde y) \,\mathrm{d}S( \tilde y) 
	+  t \fint_{\partial B_1(0)}^{}  \nabla g(x + t \tilde y) \cdot \tilde y \,\mathrm{d}S( \tilde y) \\
	& \stackrel{\tilde y = \frac{y-x}{t}}{=} \fint_{B_t(x)}^{} g(y) \,\mathrm{d}S(y) + t \fint_{\partial B_t(x)}^{}  \nabla g(y) \frac{y-x}{t} \,\mathrm{d}S(y)
\end{align*}
\[
	(K) \qquad u(t,x)= \fint_{\partial B_t(x)}^{} g(y) \,\mathrm{d}S(y) + \fint_{\partial B_t(x)}^{}  \nabla g(y) (y-x) \,\mathrm{d}S(y) + t \fint_{\partial B_t(x)}^{} h(y) \,\mathrm{d}S(y)
\]
Dies ist die Kirchhoffsche Formel für $n=3$.

\begin{satz}
	Seien $g \in C^3(\mathbb{R}^3)$, $h \in C^2(\mathbb{R}^3)$ und $u: (0,\infty) \times \mathbb{R}^3 \to \mathbb{R}$ definiert durch (K). Dann gelten
	\begin{enumerate}[(i)]
		\item $ u \in C^2([0, \infty) \times \mathbb{R}^3) $
		\item $u(0,x)=g(x)$, $u_t(0,x) =h(x)$ für $x \in \mathbb{R}^3$
		\item $u_{tt}(t,x)- \Delta u(t,x)=0$ für alle $(t,x) \in (0,\infty) \times \mathbb{R}^3$
	\end{enumerate}
\end{satz}
\begin{beweis}
	\begin{enumerate}[(i)]
		\item $\surd$ 
		\item +(iii) \begin{description}
			\item[Schritt 1:] $g \equiv 0$. Dann gilt
			\[
				u(t,x)= t \fint_{\partial B_t(x)}^{} h(y) \,\mathrm{d}S(y) = t \fint_{\partial B_1(0)}^{} h(x+t \tilde y) \,\mathrm{d}S( \tilde y)
			\] 
			\[
				\lim_{t \to 0} u(t,x) = 0 = g \qquad \surd
			\]
			\begin{align*}
				\Delta u(t,x) &= t \fint_{\partial B_t(x)}^{} \Delta h(y) \,\mathrm{d}S(y), \\
				\partial_t u(t,x) &= \fint_{\partial B_1(0)}^{} h(x+ t \tilde y) \,\mathrm{d}S( \tilde y) + t \fint_{\partial B_1(0)}^{}  \nabla h(x+t \tilde y)\cdot \tilde y \,\mathrm{d}S( \tilde y)
			\end{align*}
			\[
				\Rightarrow \qquad \lim_{t \to 0} \partial_t u(t,x) = h(x) \qquad \surd
			\]
		Es gilt
		\begin{align*}
			\partial_t u(t,x)& \stackrel{\tilde y = \frac{y-x}{t}}{=} \frac{1}{4 \pi} \int_{\partial B_1(0)}^{} h(x+t \tilde y) \,\mathrm{d}S( \tilde y) + 
			\frac{t}{4 \pi} \frac{1}{t^2} \int_{ \partial B_t(x)}^{}  \nabla  h(y) \frac{y-x}{t} \,\mathrm{d}S(y) \\
			&\stackrel{\text{Gauß}}{=} \frac{1}{4 \pi} \int_{\partial B_1(0)}^{} h(x+ t \tilde y) \,\mathrm{d}S( \tilde y) 
			+ \frac{1}{4 \pi t} \int_{B_t(x)}^{} \Delta h(y) \,\mathrm{d}y \\
			&\stackrel{\hphantom{Gauß}}{=} \frac{1}{4 \pi} \int_{ \partial B_1(0)}^{} h(x+t \tilde y) \,\mathrm{d}S( \tilde y) +
			\frac{1}{4 \pi t} \int_{0}^{t} \int_{\partial B_{\rho}}^{} \Delta h(y) \,\mathrm{d}S(y) \,\mathrm{d}\rho
		\end{align*}
		\begin{align*}
			\partial_{tt} u(t,x) &= \frac{1}{4 \pi} \int_{ \partial B_1(0)}^{}  \nabla h(x+ t \tilde y) \cdot \tilde y \,\mathrm{d}S( \tilde y)
			+ \frac{1}{4 \pi t} \int_{\partial B_t(x)}^{} \Delta h(y) \,\mathrm{d}S(y) \\
			& \qquad + \frac{1}{4 \pi} \int_{B_t(x)}^{} \Delta h(y) \,\mathrm{d}y \\
			&= \frac{1}{4 \pi t} \frac{t^2}{t^2} \int_{\partial B_t(x)}^{} \Delta h(y) \,\mathrm{d}S(y) \\
			&= t \fint_{\partial B_t(x)}^{} \Delta h(y) \,\mathrm{d}S(y) \\
			&= \Delta u(t,x)  
		\end{align*}
		für alle $(t,x) \in (0,\infty) \times \mathbb{R}^3$.
		\item[Schritt $2$:] $h \equiv 0$. \\
		Sei $v : [0,\infty) \times \mathbb{R}^3 \to \mathbb{R}$ definiert durch
		\[
			v(t,x) = t \fint_{\partial B_1(0)}^{} g(x+ t \tilde y) \,\mathrm{d}S( \tilde y)
		\] 
		Aus Schritt 1 folgt für alle $(t,x) \in (0,\infty) \times \mathbb{R}^3$ \[
			v_{tt}(t,x) - \Delta v(t,x) = 0 
		\]
		Dann gilt
		\begin{align*}
			v_t(t,x) &= \fint_{\partial B_1(0)}^{} g(x+ t \tilde y) \,\mathrm{d}S(\tilde y) 
			+ t \int_{\partial B_1(0)}^{}  \nabla g(x+t \tilde y) \cdot \tilde y \,\mathrm{d}S( \tilde y) \\
			&= u(t,x)
		\end{align*}
		\begin{align*}
			u_t(t,x) &= \partial_t^2 v(t,x) \\
			&= \partial_t v_{tt}(t,x) \\
			&= \partial_t \Delta v(t,x) \\
			& \stackrel{v \in C^3}{=} \Delta v_t(t,x) \\
			&= \Delta u(t,x)
		\end{align*}
		\[
			\lim_{t \to 0} u(t,x) = g(x) \qquad \surd
		\]
		Wir erhalten 
		\[
			\lim_{t \to 0} v(t,x) = 0
 		\]
		aus der Definition und $v_{tt}(t,x)  \Big|_{t=0}^{} = \Delta v(t,x)  \Big|_{t=0}^{}=0$, weil $v(0,x) \equiv 0$ ist.
		\[
			v_{tt}(t,x) = u_t(t,x) \qquad  \Rightarrow \qquad \lim_{t \to 0} u_t(t,x)=0 =h. 
		\]
		\end{description}
		Der Beweis folgt aus der Linearität der Wellengleichung.
	\end{enumerate}
\end{beweis}
\begin{bemerkung}
	\begin{enumerate}[(i)]
		\item Um eine Lösung $u \in L^2((0,\infty) \times \mathbb{R}^3)$ zu erhalten, braucht man Anfangsdaten $g \in C^3(\mathbb{R}^3)$, $h \in C^2(\mathbb{R}^3)$ \\
		(Regularitätsverlust)
		\item $u(t,x)$ hängt nur von den Anfangsdaten $g$, $ \nabla g$, $h$ auf $\partial B_t(x)$ ab.
		\item Für $n>3$ ungerade gibt es eine analoge formel mit $g \in C^{\frac{n+3}{2}}(\mathbb{R}^n)$, $h \in C^{\frac{n+1}{2}}$. [Evans, S. 77].
		\[
			(P) \begin{cases}
				u_{tt}(t,x)- \Delta u(t,x) = 0, &\text{ in }(0, \infty) \times \mathbb{R}^2\\
				u(0,x) = g(x), \qquad u_t(0,x)= h(x), & \text{ für }x \in \mathbb{R}^2
			\end{cases}
		\]
		Ist $u \in C^2([0,\infty) \times \mathbb{R}^2)$ eine Lösung von (P), so löst die Funktion
		\[
			\bar{u} (t,\underset{\in  \mathbb{R}^3}{\underbrace{\bar{x}}}) := u(t,\underset{\in \mathbb{R}^2}{\underbrace{x}})
		\]
		das Problem
		\[
			\begin{cases}
				\overline{u_{tt}}(t,\bar{x})- \Delta \bar{u}(t,\bar{x}), &\text{ in }(0,\infty) \times \mathbb{R}^3\\
				\bar{u}(0,\bar{x})= \bar{g}(\bar{x}), \qquad \overline{u_t}(0,\bar{x}) = \bar{h}(\bar{x}), &\text{ für } \bar{x} \in \mathbb{R}^3,
			\end{cases}
		\]
		wobei $\bar{h}(\bar{x}):= g(x)$ und $\bar{h}(\bar{x}):= h(x)$. \\
		Die Kirchhoffsche Formel zeigt daher
		\[
			\bar{u}(t,x,0) = \partial_t \left( t \fint_{\partial B_t(x,0)}^{} \bar{g}(\bar{y}) \,\mathrm{d}S(\bar{y}) \right)
			+ t \fint_{\partial B_t(x,0)}^{} \bar{h}(\bar{y}) \,\mathrm{d}S(\bar{y}) 
		\]
		\[
			(*) \qquad \fint_{B_t(x,0)}^{} \bar{g}(\bar{y}) \,\mathrm{d}S( \bar{ y}) 
			= t \frac{1}{4 \pi t^2} \int_{\partial B_t(x,0)}^{} g( \bar{y}) \,\mathrm{d}S( \bar{y})
		\]
		Wir benutzen eine Parametrisierung 
		\[
			\varphi(y) = \sqrt{t^2 - \abs{y-x}^2}
		\]
		von $ \partial B_t(x,0) \cap \set{y_3 \geq 0}$ über $B_t(x)$. \\
		Dann gilt wegen
		\[
			 \nabla \varphi = \frac{-2 (y-x)}{2 \sqrt{t^2-\abs{y-x}^2}} = - \frac{y-x}{\sqrt{t^2-\abs{y-x}^2}}
		\]
		\[
			\Rightarrow \qquad \sqrt{1+ \abs{ \nabla \varphi}^2} = \sqrt{\frac{t^2}{t^2-\abs{y-x}^2}}
		\]
		und somit insgesamt
		\begin{align*}
			(*) &= 2 \frac{1}{4 \pi t} \int_{B_t(x)}^{} g(y) \sqrt{1+ \abs{ \nabla \varphi(y)}^2} \,\mathrm{d}y \\
			&= \frac{1}{2 \pi t} \int_{B_t(x)}^{} g(y) \sqrt{\frac{t^2}{t^2-\abs{y-x}^2}} \,\mathrm{d}y 
		\end{align*}
		\[
			t \fint_{B_t(x)}^{} h( \bar{y}) \,\mathrm{d}S(\bar{y}) 
			= \frac{1}{2} \frac{1}{\pi t} \int_{B_t(x)}^{} h(y) \sqrt{\frac{t^2}{t^2 - \abs{y-x}^2}} \,\mathrm{d}y 
			= \frac{1}{2} t^2 \fint_{B_t(x)}^{} \frac{h(y)}{\sqrt{t^2-\abs{y-x}^2}} \,\mathrm{d}y
		\]
		\begin{align*}
			\frac{1}{2} \partial_t \left( \frac{1}{\pi} \int_{B_t(x)}^{} \frac{g(y)}{t^2-\abs{y-x}^2} \,\mathrm{d}y \right) & \stackrel{y = x + t \tilde y}{=} 
			\frac{1}{2} \partial_t \left( \frac{1}{\pi} \int_{B_1(0)}^{} \frac{g(x + t \tilde y)}{\sqrt{t^2 - t^2 \abs{\tilde y}^2}}t^2 \,\mathrm{d} \tilde y \right) \\
			& \stackrel{\hphantom{y = x + t \tilde y}}{=} 
			\frac{1}{2} \partial_t \left(  \frac{t}{ \pi} \int_{B_1(0)}^{} \frac{g(x+ t \tilde y)}{\sqrt{1- \abs{ \tilde y}^2}} \,\mathrm{d} \tilde y \right) \\
			& \stackrel{\hphantom{y = x + t \tilde y}}{=} \frac{1}{2} \left( \frac{1}{\pi} \int_{B_1(0)}^{} \frac{g(x+ t \tilde y)}{\sqrt{1 - \abs{ \tilde y}^2}} 
			\right. \,\mathrm{d}S( \tilde y) \\ & \qquad \qquad  \left .
		+ \frac{t}{\pi} \int_{B_1(0)}^{} \frac{ \nabla  g(x+ t \tilde y) \cdot \tilde y}{\sqrt{1- \abs{ \tilde y}^2}} \,\mathrm{d} \tilde y \right) \\
			 & \stackrel{\hat{y}= \frac{y-x}{t}}{=} \frac{1}{2} \left( \frac{1}{t^2 \pi} \int_{B_t(x)}^{} \frac{g(y)}{ \sqrt{1- \frac{\abs{y-x}^2}{t^2}}}
		 \frac{\,\mathrm{d}y}{t^2} \right. \hphantom{)} \\ & \qquad \qquad  \left. + \frac{t}{\pi t^2} \int_{B_t(x)}^{} \frac{ \nabla g(y) \cdot \frac{y-x}{t}}{\sqrt{1-\frac{\abs{y-x}^2}{t^2}}} 
			 \,\mathrm{d}y \right) \\
			 & \stackrel{\hphantom{y = x + t \tilde y}}{=} \frac{1}{2} \fint_{B_t(x)}^{} \frac{t g(y)}{ \sqrt{t^2 - \abs{y-x}^2}} \,\mathrm{d}y \\
			 & \qquad \qquad \frac{1}{2} \fint_{B_t(x)}^{} \frac{t  \nabla g \cdot (y-x)}{\sqrt{t^2 - \abs{y-x}^2}} \,\mathrm{d}y
		\end{align*}
		Damit gilt
		\[
			(1)\qquad u(t,x)= \frac{1}{2} \fint_{B_t(x)}^{} \frac{t g(y) + t  \nabla g(y) \cdot (y-x) + t^2 h(y)}{\sqrt{t^2 - \abs{y-x}^2}} \,\mathrm{d}y
		\]
		Dies ist die Poissonsche Formel für $n=2$
	\end{enumerate}
\end{bemerkung}

\begin{satz}
	Seien $ g \in C^3(\mathbb{R}^2)$, $h \in C^2(\mathbb{R}^2)$ und $u: (0,\infty) \times \mathbb{R}^2 \to \mathbb{R}$ definiert durch (1). Dann gelten:
	\begin{enumerate}[(i)]
		\item $u \in C^2([0,\infty) \times \mathbb{R}^2)$
		\item $u_{tt} - \Delta u = 0$ in $[0,\infty) \times \mathbb{R}^2$
		\item $u(0,x)=g(x)$, $u_t(0,x) = h(x)$, $x \in \mathbb{R}^2$
	\end{enumerate} 
\end{satz}
\begin{beweis}
	Folgt aus Satz $4.6$
\end{beweis}
\begin{bemerkung}
	\begin{enumerate}[(i)]
		\item Die Lösung hängt von den Anfangsdaten $g$, $ \nabla g$, $h$ in der ganzen Kugel $B_t(x)$ ab.
		\item Für $n$ gerade gibt es eine analoge Formel mit $g \in C^{\frac{n+k}{2}}(\mathbb{R}^n)$, $h \in C^{\frac{n+2}{2}}(\mathbb{R}^n)$ [Evans, S.80]
	\end{enumerate}
\end{bemerkung}
\subsubsection{Die inhomogene Wellengleichung} 
\label{ssub:die_inhomogene_wellengleichung}

\[
	\begin{cases}
		u_{tt}(x)- \Delta u(t,x) = f(t,x), &\text{ in }(0,\infty) \times \mathbb{R}^n\\
		u(0,x) = g(x), \, u_t(0,x) = h(x), &\text{ für }x \in \mathbb{R}^n
	\end{cases}
\]
Wegen der Linearität der Wellengleichung betrachten wir
\[
	(P)_f \qquad \begin{cases}
		u_{tt}(t,x) - \Delta u(t,x) = f(t,x), &\text{ in }(0,\infty) \times \mathbb{R}^n\\
		u(0,x) = u_t(0,x) = 0, \text{ für } x \in \mathbb{R}^n
	\end{cases}
\]
Wir definieren eine Familie von Funktionen $(v(t,x;s))_{s >0}$ als Lösungen der Differentialgleichung
\[
	(P)_s \qquad \begin{cases}
		v_{tt}(t,x;s)- \Delta v(t,x;s) = 0, &\text{ in }(s, \infty) \times \mathbb{R}^n\\ 
		v(s,x;s) = 0 ,&\text{ für } x \in \mathbb{R}^n \\
		v_t(s,x;s) = f(s,x) ,&\text{ für } x \in \mathbb{R}^n
	\end{cases}
\]

\begin{satz}
	Sei $f \in C^{\lfloor \frac{n}{2} \rfloor +1}([0,\infty) \times \mathbb{R}^n)$ und $u: (0, \infty) \times \mathbb{R}^n \to \mathbb{R}$ definiert durch
	\[
		u(t,x):= \int_{0}^{t} v(t,x;s) \,\mathrm{d}s 
	\]
	mit $v$ Lösung von $(P)_s$. Dann gilt $u \in C^2([0,\infty) \times \mathbb{R}^n)$ und $u$ löst $(P)_f$
\end{satz}
\begin{beweis}
	$u \in C^2$ klar wegen Regularität von $v$. \\
	\[
		\lim_{t \to 0^+} u(t,x) = 0, \qquad u_t(t,x) = v(t,x;t) + \int_{0}^{t} v_t(t,x;s) \,\mathrm{d}s
	\]
	Aus der Leipnitzregel für die Ableitungen folgt dann
	\[
		 \qquad \lim_{t \to 0^+} u_t(t,x) = 0
	\]
	\begin{align*}
		u_{tt}(t,x) &\leq v_t(t,x;t) + \int_{0}^{t} v_{tt}(t,x;s) \,\mathrm{d}s  \\
		&= f(t,x) + \int_{0}^{t} \Delta_x v(t,x;s) \,\mathrm{d}s \\
		&= f(t,x) + \Delta u(t,x)
	\end{align*}
\end{beweis}
\minisec{Explizite Darstellungsformel für die Lösung von $(P)_f$}
\begin{description}
	\item[$n=1$:]
	\[
		v(t,x;s) = \frac{1}{2} \int_{x-(t-s)}^{x+(t-s)} f(s,y) \,\mathrm{d}y
	\]
	(Der Anfangszeitpunkt liegt nun bei $t=s$)
	\begin{align*}
		u(t,x)&= \int_{0}^{t}v(t,x;s) \,\mathrm{d}s \\ &= \frac{1}{2} \int_{0}^{t} \int_{x-(t-s)}^{x+(t-s)} f(s;y) \,\mathrm{d}y \,\mathrm{d}s \\
		&= \frac{1}{2} \int_{0}^{t} \int_{x- \tilde s}^{x + \tilde s} f(t- \tilde s;y) \,\mathrm{d}y \,\mathrm{d} \tilde s
	\end{align*}
	\item[$n=3$:] 
	\[
		v(t,x;s) = (t-s) \fint_{\partial B_{(t-s)}(x)}^{} f(s;s) \,\mathrm{d}S(y)
	\]
	Dann gilt
	\begin{align*}
		u(t,x) & \stackrel{\hphantom{\tilde s = t-s}}{=} \int_{0}^{t} v(t,x;s) \,\mathrm{d}s \\
		& \stackrel{\hphantom{\tilde s = t-s}}{=} \int_{0}^{t}(t-s) \fint_{\partial B_{(t-s)}(x)}^{} f(s;y) \,\mathrm{d}S(y) \,\mathrm{d}s \\
		& \stackrel{\tilde s = t-s}{=} \frac{1}{4 \pi} \int_{0}^{t} \frac{1}{\tilde s} \int_{\partial B_{\tilde s}(x)}^{} f(t- \tilde s,y)
		 \,\mathrm{d}S(y) \,\mathrm{d} \tilde s \\
		& \stackrel{ \bar{s} = \abs{y-x}}{=} \frac{1}{4 \pi} \int_{0}^{t} \int_{\partial B_{\tilde s}(x)}^{} \frac{1}{\abs{y-x}} f(t-\abs{y-x};y) \,\mathrm{d}S(y) \,\mathrm{d} \bar{s} \\
	&\stackrel{\hphantom{\tilde s = t-s}}{=}	\frac{1}{4 \pi} \int_{B_{\bar{s}}(x)}^{} \frac{f(t- \abs{y-x};y)}{\abs{y-x}} \,\mathrm{d}y
	\end{align*}
\end{description}

\subsection{Die Wellengleichung auf allgemeinen Gebieten} 
\label{sub:die_wellengleichung_auf_allgemeinen_gebieten}
Betrachte für $\Omega \subseteq \mathbb{R}^n$ offen, beschränkt und mit $C^1$-Rand und für $T >0$
\[
	\begin{cases}
		u_{tt}(t,x) - \Delta u(t,x) = f(t,x), &\text{ in }\Omega_T = (0,T] \times \Omega\\
		u=g ,&\text{ auf} \partial_p \Omega_T, \\
		u_t=h, &\text{ auf } \set{0} \times \Omega		
	\end{cases}
\]
\begin{bemerkung}
	Für $T= \infty$, $ \Omega = \mathbb{R}^n$ gilt $ \partial _p \Omega_T = \set{0} \times \mathbb{R}^n = \set{0} \times \Omega$.
\end{bemerkung}
Zu Der \underline{Energiemethode} gehört die \underline{Eindeutigkeit} und die \underline{Ausbreitungsgeschwindigkeit} 

\minisec{Eindeutigkeit:}
\begin{lemma}
	Sei $\Omega$ eine offene, beschränkte Teilmenge des $\mathbb{R}^n$ mit $C^1$-Rand, $T>0$ und $u \in C^2(\overline{\Omega_T})$ eine Lösung von
	\[
		\begin{cases}
			u_{tt}- \Delta u = 0, &\text{ in }\Omega_T\\
			u=0, &\text{ auf }[0,T] \times \partial \Omega
		\end{cases}
	\]
	Dann erfüllt 
	\[
		e(t) := \frac{1}{2} \int_{\Omega}^{} (u_t(t,x))^2 + ( \nabla u(t,x))^2 \,\mathrm{d}x
	\]
	für alle $t \in [0,T]$ die Identität
	\[
		e(t) = e(0) 
	\]
\end{lemma}
\begin{beweis}
	Es gilt
	\begin{align*}
		\diffd{}{t}e(t) &= \int_{\Omega}^{}( u_t u_{tt} +  \nabla u \cdot \underset{=  \nabla u_t}{\underbrace{\partial_t (  \nabla u)}}) \,\mathrm{d}x \\
		&= \int_{\Omega}^{}( u_t u_{tt} +  \nabla u \cdot  \nabla u_t) \,\mathrm{d}x \\ 
		&= \underset{=0}{\underbrace{\int_{\Omega}^{} (u_t u_{tt} - \Delta u u_t) \,\mathrm{d}x}} 
		+ \underset{=0}{\underbrace{ \int_{ \partial \Omega}^{} u_t  \nabla u \cdot \nu \,\mathrm{d}S(x)}} \\
		&=0
	\end{align*}
	Damit ist $e$ konstant und damit
	\[
		e(t) = e(0) \qquad \forall\, t \in [0,T].
	\]
\end{beweis}

%%%% 13.06.2016- 30.06.2016
\begin{satz}
	Sei $\Omega$ eine offene, beschränkte Teilmenge im $\mathbb{R}^n$ mit $C^1$-Rand. Sind $u,v \in C^2( \overline{\Omega_T})$ Lösungen von (P), so gilt $u=v$, $T>0$,
	$f \in C^0(\Omega_T)$, $g \in C^2( \partial_p \Omega_T)$, $h \in C^1(\Omega)$.
\end{satz}
\begin{beweis}
	Sei $w:= u-v \in C^2(\overline{\Omega_T})$. Dann gilt \[
		\begin{cases}
			w_{tt}- \Delta w = 0, &\text{ in }\Omega_T\\
			w = 0,&\text{ auf } \partial_p \Omega_T \\
			w_t=0 , &\text{ auf } \partial_p \Omega_T
		\end{cases}
	\]
	\[
		\stackrel{w \in C^2(\overline{\Omega_T})}{\Rightarrow} \qquad w =0 \text{ auf } [0,T] \times \partial \Omega
	\]
	\[
		\stackrel{\text{Lemma }4.9}{\Rightarrow } \qquad e(t) = e(0) = \int_{\Omega}^{} ((\underset{=0}{\underbrace{w_t(0,x)}})^2 
		+ \underset{=0}{\underbrace{(  \nabla w(0,x))^2}}) \,\mathrm{d}x = 0
	\]
	\[
		\Rightarrow \qquad e(t)=0 \qquad \forall\, t \in [0,T]
	\]
	\[
		0 = e(t) = \int_{\Omega}^{}((w_t(t,x))^2+ ( \nabla w(t,x))^2) \,\mathrm{d}x
	\]
	\[
		\Rightarrow \qquad w_t=0, \qquad  \nabla w=0 \qquad \text{in }\Omega_T
	\]
	\[
		\Rightarrow \qquad w = \text{konstante} = 0
	\]
\end{beweis}
\minisec{Endliche Ausbreitungsgeschwindigkeit:}
\begin{satz}
	Sei $(t_0,x_0) \in (0, \infty) \times \mathbb{R}^n$ und 
	\[
		C(t_0,x_0) := \set[(t,x)]{0 \leq t \leq t_0,\,\abs{x-x_0} \leq t_0-t}
	\]
	der sogenannte Vergangenheitskegel/Einflusskegel. \\
	Sei $u \in C^2((0,\infty) \times \mathbb{R}^n)$ eine Lösung von $u_{tt}-\Delta u =0$ in $(0,\infty) \times \mathbb{R}^n$ falls $u = u_t=0$ auf 
	$\set{0} \times B_{t_0}(x_0)$, dann gilt $u \equiv 0$ in $C(t_0,x_0)$
\end{satz}

\begin{bemerkung}
	Die Lösung der Wellengleichung hängt nicht von Störungen außerhalb des Einflusskegels ab. Damit haben Störungen endliche Ausbreitungsgeschwindigkeit.
\end{bemerkung}
\begin{beweis}
	Sei
	\begin{align*}
		e(t) :=& \frac{1}{2} \int_{B_{t_0-t}(x_0)}^{} ((u_t(t,x))^2 + \abs{ \nabla  u(t,x)}^2) \,\mathrm{d}x \\
		=& \frac{1}{2} \int_{0}^{t_0-t} \int_{\partial B_{\rho}(x_0)}^{} ((u_t(t,x))^2 + \abs{ \nabla u(t,x)}^2) \,\mathrm{d}S(x) \,\mathrm{d} \rho \\
		=& \frac{1}{2} \int_{0}^{t_0-t} f(t,\rho) \,\mathrm{d}\rho
	\end{align*}
	wegen $0 \leq t \leq t_0$. \\
	Es gilt
	\begin{align*}
		\diffd{e(t)}{t} &= \frac{1}{2} \int_{0}^{t_0-t} f_t(t,\rho)  \,\mathrm{d}\rho - \frac{1}{2}f(t,t_0-t) \\
		&= \frac{1}{2} \int_{0}^{t_0-t} \int_{\partial B_\rho(x_0)}^{} ( 2 u_t u_{tt} + 2  \nabla u  \nabla u_t) \,\mathrm{d}S(x) \,\mathrm{d}\rho \\
		& \qquad \qquad - \frac{1}{2}\int_{B_{t_0-t}(x_0)}^{}((u_t(t,x))^2 + \abs{ \nabla u(t,x)}^2) \,\mathrm{d}S(x) \\
		&= \int_{B_{t_0-t}(x_0)}^{}(u_t u_{tt}+  \nabla u  \nabla u_t) \,\mathrm{d}x - \frac{1}{2} \int_{\partial B_{t_0-t}(x_0)}^{} 
		((u_t)^2 + \abs{ \nabla u}^2) \,\mathrm{d}S(x)
	\end{align*}
	\begin{align*}
		\diffd{e(t)}{t} &= \int_{B_{t_0-t}(x_0)}^{} u_t( u_{tt}- \Delta u) \,\mathrm{d}x + \int_{\partial B_{t_0-t}(x_0)}^{} u_t  \nabla u \cdot \omega
		 \,\mathrm{d}S(x) \\ & \qquad \qquad - \frac{1}{2} \int_{\partial B_{t_0-t}(x_0)}^{}((u_t)^2 + \abs{  \nabla u}^2) \,\mathrm{d}S(x) \leq 0
	\end{align*}
	wegen
	\[
		u_t  \nabla \cdot \omega \leq \abs{u_t  \nabla u \cdot \omega} \leq \abs{u_t}\abs{ \nabla u} \stackrel{\text{Young}}{\leq } \frac{1}{2} ((u_t)^2 + \abs{ \nabla u}^2)
	\]
	Damit ist $e$ monoton fallend, also 
	\[
		e(t) \leq e(0) = \int_{B_{t_0}(x_0)}^{} ((u_t)^2 + \abs{ \nabla u}^2) \,\mathrm{d}x =0
	\]
	\[
		\Rightarrow \qquad 0 \leq  e(t) \leq 0 \qquad \forall\, t \in [0,t_0)
	\]
	\[
		\Rightarrow \qquad u_t=0, \, \nabla u=0 \qquad \text{in } C(t_0,x_0) \qquad \forall\, x \in B_{t_0-t}(x_0), \,t \in [0,t_0)
	\]
	\[
		\Rightarrow \qquad u=0 \qquad \text{in }C(t_0,x_0)
	\]
\end{beweis}
\[
	(1) \begin{cases}
		E(x,u(x), \nabla u(x))=0, &\text{ in }\Omega\\
		u=g,&\text{ auf }\partial \Omega
	\end{cases}
\]
Mit $E$ und $g$ "regulär", $\Omega \subseteq \mathbb{R}^n$ offen und mit $C^1$-Rand.
\newpage
\section{PDE's erster Ordnung} 
\label{sec:pde_s_erster_ordnung}
In diesem Kapitel werden wir die Methode der Charakteristiken kennenlernen. Dies bedeutet aus der Differentialgleichung (1) eine ODE zu machen. 
\subsection{Vorbereitungen} 
\label{sub:vorbereitungen}
\subsubsection{ODE's} 
\label{ssub:ode_s}
Sei $G \subseteq \mathbb{R} \times \mathbb{R}^n$, $f: G \to  \mathbb{R}^n$ mit $f= (f_1,\dots,f_n)$ stetig.
\[
	(2) \qquad y' = f(x,y) \qquad \Leftrightarrow \qquad \begin{cases}
		y_1'&=f_1(x,y_1,\dots,y_n)\\
		&\vdots \\
		y_n' &= f_n(x,y_1,\dots,y_n)
	\end{cases}\qquad \text{System von ODE's}
\]
Eine Lösung von (2) ist eine Funktion $y \in C^1(I,\mathbb{R}^n)$ auf einem Intervall $I \subseteq \mathbb{R}$ mit folgenden Eigenschaften:
\begin{enumerate}[(i)]
	\item Der Graph von $y$ ist in $G$ enthalten also 
	\[
		(x,y(x)) \subseteq G \qquad \forall\, x \in I
	\]
	\item $y'=f(x,y(x))$ für alle $x \in I$
\end{enumerate}
\begin{satz}[Zurückführung auf Integralgleichung]
	Sei $G \subseteq \mathbb{R}\times \mathbb{R}^n$, $f: G \to \mathbb{R}^n$ eine stetige Funktion und $(x_0,y_0) \subseteq G$ gegeben. Dann gilt: \\
	Eine stetige Funktion $y: I \to \mathbb{R}^n$, die auf einem Intervall $I \subseteq \mathbb{R}$ mit $x_0 \in I$ definiert ist, und deren Graph in $G$ enthalten
	 ist, ist genau dann Lösung von
	\[
		(3) \qquad  \begin{cases}
			y' = f(x,y), &\text{ in }I\\
			y(x_0) = y_0,
		\end{cases}
	\]wenn die folgende Integralgleichung gilt
	\[
		(4) \qquad y(x) = y_0 + \int_{x_0}^{x} f(t,y(t)) \,\mathrm{d}t \qquad \forall\, x \in I
	\]
\end{satz}
\begin{beweis}
	\begin{description}
		\item[$(4) \Rightarrow (3)$:]$y(x_0)=y_0$, $t \mapsto f(t,y(t))$ ist stetig in $I$. es gilt
		\[
			\diffd{}{x} \int_{x_0}^{x}f(t,y(t)) \,\mathrm{d}t = f(x,y(x)) = y'(x)
		\] 
		\item[$(3) \Rightarrow (4)$] \[
			y'(x) = f(x,y) \qquad \Rightarrow \qquad \int_{x_0}^{x} f(t,y(t)) \,\mathrm{d}t = y(x) - y_0
		\]
	\end{description}
\end{beweis}
\begin{definition*}
	Sei $G \subseteq \mathbb{R}\times \mathbb{R}^n$. Eine Funktion $f : G \to \mathbb{R}^n$ heißt Lipschitzstetig bzgl der zweiten Variablen mit der Lipschitzkonstante $L>0$, wenn
	\[
		\abs{f(x,y_1)-f(x,y_2)} \leq L \abs{y_1-y_2} \qquad \forall\, (x,y_1), \,(x,y_2) \in G.
	\]
	$f$ ist lokal Lipschitzstetig bzgl. der zweiten Variablen, falls jeder Punkt $(\bar{x},\bar{y}) \in G$ eine Umgebung $U$ besitzt, so dass $f$ in $G \cap U$ Lipschitzstetig bzgl. der zweiten Variablen ist.
\end{definition*}

\begin{bemerkung}
	Sei $G \subseteq \mathbb{R} \times \mathbb{R}^n$ offen und $f: G \to  \mathbb{R}^n$ eine bzgl der Variablen $(y_1, \dots, y_n)$ stetig partiell differenzierbare Funktion. Dann ist $f$ lokal Lipschitzstetig in $G$.
\end{bemerkung}

\begin{satz}[lokaler Existenz- und Eindeutigkeitssatz von Picard-Lindelöf]
	Sei $G \subseteq \mathbb{R} \times \mathbb{R}^n$ offen und $f: G \to \mathbb{R}^n$ eine stetige Funktion, die lokal Lipschitzstetig bzgl. der zweiten Variablen ist. Dann gibt es zu jedem $(x_0,y_0) \in G$ ein $\varepsilon >0$ und eine eindeutige Lösung $y : [x_0- \varepsilon, x_0 + \varepsilon] \to \mathbb{R}^n$ von
	\[
		\begin{cases}
			y'(x)&=f(x,y)\\
			y(x_0)&= y_0
		\end{cases} \qquad \text{Cauchy-Problem}
	\]
\end{satz}
\begin{beweis}
	mithilfe des Banachschen Fixpunktsatzes. $C^0([x_0- \varepsilon, x_0 + \varepsilon];\mathbb{R}^n)$ ist ein Banachraum mit 
	\[
		\norm{\varphi}_{\infty} := \sup_{\abs{x-x_0} \leq \varepsilon} \abs{\varphi(x)} 
	\]
	(für ein noch zu bestimmtes $\varepsilon >0 $). Weil $G$ offen ist, existieren $\delta >0$ und $r >0$ so dass
	\[
		R_{\delta ,r} = \set[(x,y) \in \mathbb{R} \times \mathbb{R}^n]{\abs{x-x_0} \leq \delta , \abs{y-y_0} \leq r} \subseteq G
	\]
	und die Funktion $f$ in $R_{\delta ,r}$ Lipschitzstetig mit einer gewissen Konstante $L >0$ ist.
	\[
		\abs{f(x,y) \leq M} \qquad  \forall\, (x,y) \in R_{\delta ,r}
	\]
	Wir setzen $\varepsilon := \min \set{\delta , \frac{r}{M}, \frac{1}{2L}}$ und 
	\[
		B:= \set[\varphi \in C^0([x_0- \varepsilon, x_0 + \varepsilon];\mathbb{R}^n)]{\norm{\varphi-x_0}_{\infty} \leq r}
	\]
	$B$ ist eine abgeschlossene Teilmenge des Banachraums $C^0([x_0- \varepsilon, x_0 + \varepsilon]; \mathbb{R}^n)$. Damit ist $B$ ein vollständiger Raum.
	Sei $ T:B \to B$ mit $T(\varphi) = \psi$ wobei
	\[
		\varphi(x) : = y_0 + \int_{x_0}^{x} f(t,\varphi(t)) \,\mathrm{d}t, \qquad x \in [x_0- \varepsilon, x_0 + \varepsilon]
	\]
	\begin{enumerate}[(i)]
		\item $T$ ist wohldefiniert: $\norm{\varphi-y_0}_{\infty} \leq r$
		\[
			\Rightarrow \qquad (x,\varphi(x)) \subseteq R_{\varepsilon,r} \stackrel{\varepsilon \leq \delta }{\Rightarrow } R_{\delta ,r} \subseteq G
		\]
		Damit ist $f(t,\varphi(t))$ für alle $t \in [x_0- \varepsilon, x_0 + \varepsilon]$ definiert und als Funktion von $t$ stetig.
		\[
			\abs{y-y_0} = \abs{\int_{x_0}^{x} f(t,\varphi(t)) \,\mathrm{d}t} \leq M \underset{\leq \varepsilon}{\underbrace{\abs{x-x_0}}} \leq r
		\]
		Damit liegt $\psi$ in $B$.
		\item $T$ ist eine Kontaktion: Es gilt für alle $x \in [x_0 - \varepsilon , x_0 + \varepsilon]$
		\begin{align*}
			\abs{T(\varphi_1)(x)-T(\varphi_2)(x)} &\leq \abs{\int_{x_0}^{x} \abs{f(t,\varphi_1(t))- f(t,\varphi_2(t))} \,\mathrm{d}t} \\
			&\leq \abs{\int_{x_0}^{x} L \abs{ \varphi_1(t)-\varphi_2(t)} \,\mathrm{d}t} \\
			&\leq  L \norm{\varphi_1- \varphi_2}_{\infty} \underset{\leq \varepsilon}{\underbrace{\abs{x-x_0}}} \\
			&\leq \frac{1}{2} \norm{\varphi_1- \varphi_2}_{\infty}
		\end{align*}
		\[
			\Rightarrow \qquad \norm{T(\varphi_1)-T(\varphi_2)}_{\infty} \leq \frac{1}{2}\norm{ \varphi_1- \varphi_2}_{\infty}
		\]
		Damit ist $T$ eine Kontraktion.
	\end{enumerate}
	Aus dem Banachschen Fixpunktsatz folgt nun, dass genau ein $ y \in B$ existiert, mit
	\[
		Ty=y 
	\] 
	Also für alle $x \in [x_0 - \varepsilon , x_0 + \varepsilon]$
	\[
		y(x) = y_0 + \int_{x_0}^{x} f(t,y(t)) \,\mathrm{d}t \qquad \Leftrightarrow \qquad y'=f(x,y) \text{ und } y(x_0) = y_0
	\]
\end{beweis}
\begin{beispiel}
	Betrachte
	\[
		\begin{cases}
			y'&=3y^{\frac{3}{2}} \\ 
			y(0)&= 0
		\end{cases}
	\]
	Dann ist $f(x,y) = 3 y^{\frac{3}{2}}$. Mit der Trennung der Variablen gilt
	\begin{align*}
		\diffd{y}{x} &= 3 y^{\frac{3}{2}} \\
		\Rightarrow \qquad \int_{}^{} \frac{1}{3} y^{- \frac{2}{3}} \,\mathrm{d}y &= \int_{}^{} \,\mathrm{d}x \\
		\Rightarrow \qquad \frac{1}{3} \frac{y^{- \frac{2}{3}+1}}{- \frac{2}{3}+1} &= x + c \\
		\Rightarrow \qquad y^{\frac{1}{3}} &= x+ c \\
		\Rightarrow \qquad y &= (x+c)^3 \\
		\Rightarrow \qquad y(x) &= x^3
	\end{align*}
	Außerdem ist auch $y \equiv 0$ eine Lösung, also
	\[
		y_a(x) = \begin{cases}
			(x+a)^3, &\text{ falls }x<-a\\
			0, &\text{ falls }-a \leq x < a, \,a >0 \\
			(x-a)^3, &\text{ falls }x \geq a
		\end{cases}
	\]
	Der Eindeutigkeitssatz gilt nicht! Wäre $f$ lokal Lipschitzstetig, dann $\exists\, 0 \in U,\, L>0$, sodass
	\[
		\abs{f(y)-f(0)} \leq L \abs{y-0} \qquad \forall\, y \in U
	\]
	\[
		3 \abs{y^{\frac{2}{3}}} \leq L \abs{y} \qquad \Rightarrow \qquad 3 \abs{y^{-\frac{1}{3}}} \leq  L \qquad \forall\, y \in U,
	\]
	Aber $\abs{y^{- \frac{1}{3}}} \to  \infty$ für $y \to 0$. Dies bedeutet, dass $f$ in keiner Umgebung von $0$ eine Lipschitzbedingung erfüllt.
\end{beispiel}
\begin{bemerkung}[globale Version von Picard-Lindelöf]
	Ist $f: [a,b] \times \mathbb{R}^n \to \mathbb{R}^n$ stetig und Lipschitzstetig bzgl. der zweiten Variablen, sowie $x_0 \in [a,b]$, so bestizt das Cauchy-Problem
	eine eindeutige globale Lösung $y \in C^1([a,b], \mathbb{R}^n)$
\end{bemerkung}

\begin{satz}[Satz über implizite Funktionen]
	Sei $\Omega$ eine offene Menge in $\mathbb{R}^n \times \mathbb{R}^m$ und $f \in C^1(\Omega, \mathbb{R}^n)$ mit $f(x_1,\dots,x_n,y_1, \dots, y_n)$. Ist $(x_0,y_0) \in \Omega$ mit 
	\[
		\det  \nabla f(x_0,y_0) \neq 0,
	\] 
	so existieren offene Mengen $\Theta_1 \subseteq \mathbb{R}^n $, $ \Theta_2 \subseteq \mathbb{R}^m$ mit $(x_0,y_0) \in \Theta_1 \times \Theta_2$ sowie eine
	Funktion $g \in C^1(\Theta_1, \Theta_2)$, so dass gilt:
	\[
		\set[(x,y) \in \Theta_1 \times \Theta_2]{f(x,y)=f(x_0,y_0)} = \set[(x,g(x))]{x \in \Theta_1}
	\]
\end{satz}

\begin{satz}[Satz der lokalen Umkehrabbildung]
	Sei $\Omega \subseteq \mathbb{R}^n$ offen und $f \in C^1(\Omega,\mathbb{R}^n)$. Ist $x_0 \in \Omega$ mit $\det  \nabla f(x_0) \neq 0$, so existieren offene Mengen
	$\Theta_1, \, \Theta_2 \subseteq  \mathbb{R}^n$, so dass $x_0 \in \Theta_1$ gilt und $f: \Theta_1 \to \Theta_2$ ein Diffeomorphismus ist. Dies bedeutet, dass eine Funktion $f^{-1} \in C^1(\Theta_2, \Theta_1)$ existiert mit 
	\[
		f^{-1} \circ f = \id, \qquad f \circ f^{-1} = \id, \qquad \text{auf } \Theta_1 \text{ bzw } \Theta_2
	\]
	Es gilt außerdem 
	\[
		 \nabla f^{-1}(f(x)) = ( \nabla f(x))^{-1} \qquad \forall\, x \in \Theta_1.
	\]
	Falls zusätzlich $f \in C^k(\Omega; \mathbb{R}^n)$ für $k \geq 2$ gilt, so gilt auch $f ^{-1} \in C^k(\Theta_2,\Theta_1)$
 \end{satz}
Betrachte nun für $E \in C^2( \bar{\Omega} \times \mathbb{R} \times \mathbb{R}^n)$
\[
	(2) \qquad \begin{cases}
		E(x,u(x), \nabla u(x))=0, &\text{ in }\Omega \subseteq \mathbb{R}^n \text{ offen und mit $C^1$-Rand}\\
		u=g,&\text{ auf }\Gamma \subseteq \partial \Omega
	\end{cases}
\]
\subsection{Lokale Existenz mittels Charakteristiken} 
\label{ssub:lokale_existenz_mittels_charakteristiken}
Wir nehmen an, dass $u \in C^2$ eine Lösung von $(2)$ ist; $\gamma$ verbindet hierbei $x_0$ und $x$. 
\minisec{Aufstellung der charakteristischen Gleichung:}
$u \in C^2 \cap C^1(\bar{\Omega})$, $E \in C^2(\bar{\Omega} \times \mathbb{R} \times \mathbb{R}^n)$ sei Lösung von $E(x,u(x), \nabla u(x)) = 0$ in $\Omega$. Sei $\gamma \in C^1(I;\Omega)$, $I \subseteq \mathbb{R}$ Intervallkurve in $\Omega$. \\
\begin{align*}
	v(s) &:= u(\gamma) \qquad \text{Kurve in }\mathbb{R}, \\
	z(s) &:=  \nabla u(\gamma(s)) \qquad \text{Kurve in }\mathbb{R}^n.	
\end{align*}
\begin{align*}
	v'(s) &=  \nabla u(\gamma(s)) \cdot \gamma'(s), \\
	z(s) &= (\partial_{x_1}u(\gamma(s)), \dots, \partial_{x_n}u(\gamma(s))), \\
	z_i'(s) &=  \nabla (\partial_{x_i} u(\gamma(s))) \cdot \gamma'(s)
\end{align*}
Wir verwenden die Notation $E = E(x,v,z)$ und $ \nabla_x E$, $\partial_v E$, $ \nabla_z E$. Nun leiten wir \\$E(x,u(x), \nabla u(x))$ nach $x_i$ ab.
\begin{align*}
	0 &= \partial_{x_i} E(x, u(x),  \nabla u(x)) + \partial_{v} E(x,u(x), \nabla u(x))\partial_{x_i} u(x) \\
	&\qquad +  \nabla_z E(x,u(x), \nabla u(x)) \cdot \underset{=  \nabla (\partial_{x_i}u(x))}{\underbrace{\partial_{x_i}( \nabla u(x))}}
\end{align*}
Die Auswertung auf der Kurve $\gamma$ ergibt:
\[
	0 = \partial_{x_i} E(\gamma,v,z) + \partial_v E(\gamma,v,z)z_i +  \nabla_z E(\gamma, v,z) \cdot  \underset{(*)}{\underbrace{\nabla (\partial_{x_i}u) \circ \gamma}}
\]
Wir wollen eine Gleichung erhalten, in der nur $\gamma$, $v$, $z$ und deren Ableitungen vorkommen. Dafür müssen wir den Term $(*)$ eliminieren. Dafür setzen wir
\begin{description}
	\item[(i)]$\gamma'(s)=  \nabla _z E(\gamma,v,z)$ und daraus folgt
	\item[(iii)] \begin{align*}
		z_i' &=  \nabla (\partial_{x_i} u(\gamma(s)))\cdot  \nabla _z E(\gamma,v,z) \\
		&= - \partial_{x_i} E(\gamma,v,z) - \partial_v E(\gamma,v,z)z_i \qquad \forall\, i=1,\dots,n
	\end{align*} 
	\item[(ii)] $v'(s) = z(s) \cdot  \nabla_z E(\gamma,v,z)$
\end{description}
Es ergeben sich $2n+1$ Unbekannte und $2n+1$ gewöhnliche Differentialgleichungen.

\begin{satz}[Struktur der charakteristischen Gleichungen]
	Sei $u \in C^2(\Omega) \cap C^1(\bar{\Omega})$ eine Lösung von $E(x,u(x), \nabla u(x))= 0 $ in $\Omega$. Sei außerdem $I$ ein Intervall in $\mathbb{R}$, 
	$\gamma \in C^1(I;\Omega)$, $v:= u \circ \gamma$, $z:=  \nabla u \circ \gamma$. Ist $\gamma$ eine Lösung von 
	\begin{description}
		\item[(i)] $\gamma'(s) =  \nabla _z E(\gamma,v,z)$ in $I$, so lösen $v$ und $z$ auf $I$ die Differentialgleichungen
		\item[(ii)] $v'(s)=  \nabla_z E(\gamma,v,z) \cdot z$,
		\item[(iii)] $z'(s) = -  \nabla_x E(\gamma,v,z) - \partial_v E(\gamma,v,z)z$
	\end{description}
\end{satz}
\begin{bemerkung}
	Das System (i)-(iii) ist geschlossen und besitzt nach dem Satz von Picard-Lindelöf eine eindeutige Lösung (zu gegebenen Anfangsdaten).
\end{bemerkung}
\begin{definition*}
	Die Funktionen $\gamma = (\gamma_1, \dots, \gamma_n)$, $v$ und $z = (z_1, \dots, z_n)$ heißen die Charakteristiken zur Gleichung $E(x,u(x), \nabla u(x))$.
\end{definition*}

\begin{beispiel}
	\begin{enumerate}[1.]
		\item Lineare Gleichung \[
			b(x) \cdot  \nabla u(x) + c(x) u(x) = 0,
		\]also $E(x,v,z) = b(x) \cdot z + c(x) v$ und somit
		\[
			b(\gamma(s)) \cdot z(s) + c (\gamma(s))v(s) = 0
		\]
		\begin{enumerate}[(i)]
			\item $\gamma'(s) = b(\gamma(s))$
			\item $v'(s) = b(\gamma(s)) \cdot z(s) = -c (\gamma(s))v(s)$
		\end{enumerate}
		Hier haben wir nur $n+1$ Gleichungen für die $n+1$ Unbekannten $\gamma$ und $v$.
		\item Die Transportlgleichung
		\[
			\partial_t u(t,x) + b  \nabla u(t,x) = 0, \qquad b \in \mathbb{R}^n,\, c=0
		\]
		mit $b \in \mathbb{R}^n$, $c=0$. Setze $\hat{b}:= (1,b)$ und $Du:= ( \partial_t u,  \nabla u)$. Wir haben also die Gleichung
		\[
			\hat{b} Du(t,x) = 0
		\]
		\begin{enumerate}[(i)]
			\item $\gamma'(s) = \hat{b} = (1,b)$, mit $\gamma : [0,S] \to (0,\infty) \times \mathbb{R}^n$
			\item $v'(s) = 0$, woraus folgt, dass $v$ konstant ist.
		\end{enumerate}
		Für $t>0$, $x \in \mathbb{R}^n$, Anfangsdaten $u(0,x) = g(x)$ für alle $x \in \mathbb{R}^n$ gilt dann
		\[
			\gamma(0) = (0,x_0), \qquad x_0 \in \mathbb{R}^n
		\]
		und damit
		\[
			v(0) = u(\gamma(0))= u(0,x_0) = g(x_0)
		\]
		\[
			\Rightarrow v(s) = v(0) = g(x_0), \qquad \gamma(s) = (s, sb + x_0)
		\]
		\[
			v(s) = u(\gamma(s)) = u(s, sb+x_0) = g(x_0)
		\]
		\[
			u(s,sb+ x_0) = g(x_0).
		\]
		Setze $s:=t$, $x:= sb+x_0$ Dann $x_0 = x - tb$
		\[
			\Rightarrow u(t,x)= g(x-tb)
		\]
		\minisec{Übung:}$n=2$,
		\[
			\begin{cases}
				u_x(x,y)+ u_y(x,y) =2,\\
				u(x,0)= x^2
			\end{cases}
		\]
		Dann
		\[
			E(x,v,z) = z_1+z_2 -2 = z \cdot (1,1) - 2
		\]
		\[
			\gamma'(s) = (1,1), \qquad \gamma(0) = (x_0,0), \qquad x_0 \in \mathbb{R}
		\]
		und 
		\[
			v'(s) = b(\gamma(s)) \cdot z = 2, \qquad v(0) = x_0^2
		\]
		Es folgt
		\[
			\gamma(s) = (s+x_0,s), \qquad v(s) = 2s + x_0^2
		\]
		\[
			\Rightarrow \qquad u(\gamma(s)) = u(s+x_0,s) = 2s+ x_0^2
		\]
		Setze $x:= s+ x_0$, $y:=s$. Dann folgt $x_0= x-y$ und
		\[
			u(x,y) = 2y+ (x-y)^2
		\]
		\item Quasilineare Gleichung 
		\[
			b(x,u(x)) \cdot  \nabla u(x) + c(x,u(x)) = 0,
		\]
		Dann \[
			E(x,v,z) = b(x,v) \cdot z + c(x,v)
		\]
		Die charakteristischen Gleichungen sind dann
		\begin{enumerate}[(i)]
			\item $\gamma'(s) = b(\gamma(s),v(s))$
			\item $v'(s) = b(\gamma(s),v(s)) \cdot z(s) = -c (\gamma(s),v(s))$
		\end{enumerate}
		Auch hier werden nun die Gleichungen für $\gamma$ und $v$ benötigt.
	\end{enumerate}
\end{beispiel}

%%%neue Vorlesung

\[
	(1) \qquad E(x,u(x), \nabla u(x)) = 0 \qquad \text{in }\Omega
\]
Suche charakteristische Gleichungen zu (1):
\[
	(S) \qquad \begin{cases}
		\gamma'(s) &=  \nabla_z E(\gamma,v,z) \\
		v'(s) &=  \nabla_z E(\gamma,v,z) \cdot z \\
		z'(s) &= -  \nabla_x E(\gamma,v,z) - \partial_v E(\gamma,v,z) \cdot z
	\end{cases}
\]
Damit erhält man $2n+1$ Gleichungen. Mit $\Omega = \mathbb{R}^n_+$
\[
(2) \qquad 	\begin{cases}
		E(x,u(x), \nabla u(x)) = 0, &\text{ in }\Omega\\
		u=g, &\text{ auf } \Gamma \subseteq \partial \Omega
	\end{cases}
\]
\[
	u(x_0) = g(x_0) \qquad \forall\, x_0 \in  \partial \mathbb{R}^n_+
\]
\[
	I:= [0,s], \qquad \gamma(0) = x_0, \qquad v(0) = u(\gamma(0)) = u(x_0) = g(x_0)
\]
\[
	z_i(0) = \partial_{x_i} g(x_0), \qquad i=1,\dots,n
\]
Das sind $2n$ Anfangsbedindungen und $2n+1$ Gleichungen. Wir brauchen auch $z_n(0)$
\[
	E(x,u(x), \nabla u(x)) = 0 \qquad \forall\, x \in \mathbb{R}^n_+ \qquad \Rightarrow \qquad E(x_0,u(x_0), \nabla u(x_0)) = 0,
\]
da $E$ stetig bis zum Rand ist. Daher sollte $z_n(0)$ zu den bereits fixierten Anfangsbedingungen so bestimmt sein, dass
\[
	E(\gamma(0),v(0),z(0))=0.
\]
%%%% Vorlesung 24.6

\begin{definition*}
	Ein Vektor $(x_0,v_0,z_0) \in \mathbb{R}^{2n+1}$ heißt zulässig für das Anfangswertproblem (2) (wird noch nachgeliefert), falls \[
		g(x_0)=v_0, \qquad (z_0)_i = \partial_{x_i}g(x_0), \qquad \text{für }i=1,\dots,n-1
	\]
	und
	\[
		E(x_0,v_0,z_0) = 0
	\]
	gelten. Diese Bedingungen werden als Kompatibilitätsbedingungen bezeichnet.
\end{definition*}

\begin{bemerkung}
	Die Werte für $(x_0,v_0,(z_0)_{i=1,\dots,n-1})$ sind eindeutig bestimmt. Die Bedingung an $(z_0)_n$ ist jedoch eine im Allgemeinen nichtlineare Gleichung, für die
	keine oder mehrere Lösungen existieren können
\end{bemerkung}
\[
	(S), (x_0,v_0,z_0) \text{ zulässig} \qquad \stackrel{\text{Picard-Lindelöf}}{\Rightarrow } \qquad \exists\,! \text{ Lösung von (S) mit Anfangsbedingungen }\]
	\[
		\begin{cases}
				\gamma(0)&= x_0\\
				v(0)&=v_0 \\
				z(0)&=z_0 \\
		\end{cases}
	\]
Haben wir einen zulässigen Vektor $(x_0,v_0,z_0)$ so können wir mithilfe von Picard-Lindelöf die Lösung der Charakteristischen Gleichungen zu diesen Anfangswerten bestimmen. Wir wollen allerdings eine Lösung von (2) konstruieren, also benötigen wir zulässige Vektoren für alle Randpunkte auf $\partial \mathbb{R}^n_+$.

\begin{lemma}
	Ist $(x_0,v_0,z_0)$ ein zulässiger Vektor für das Anfangswertproblem (2), so existiert ein $\delta >0$ und eine eindeutige Lösung $\bar{z}$ von 
	$\bar{z_i}(y) = \partial_{x_i}g(y)$ für $i=1,\dots,n-1$ und $E(y,g(y),\bar{z}(y))=0$ für alle $y \in B_{\delta }(x_0) \cap \partial \mathbb{R}^n_+$, falls 
	$\partial_{z_n} E(x_0,v_0,z_0) \neq 0$ gilt. \\
	In diesem Fall heißt der Vektor $(x_0,v_0,z_0)$ nichtcharakteristisch.
\end{lemma}
\begin{beweis}
	Sei
	\[
		G: \mathbb{R}^{n-1} \times \mathbb{R}^n \to \mathbb{R}^n, \qquad G(\hat{y},z)= (G_1(\hat{y},z), \dots, G_n(\hat{y},z)) 
	\]
	\begin{align*}
		G_1(\hat{y},z) &:= z_i - \partial g(\hat{y},0), \qquad i=1,\dots,n \\
		G_n(\hat{y},z) &:= E((\hat{y},0),g(\hat{y},0),z)
	\end{align*}
	Da $(x_0,v_0,z_0)$ zulässig ist, folgt $G((x_0)_1, \dots, (x_0)_{n-1},z_0)=0$.
	\[
		 \nabla _z G = (\partial_{z_j}G_i)_{i,j=1,\dots,n} \in \mathbb{R}^{n \times n}
	\]
	mit
	\begin{align*}
		\partial_{z_j} G_i ( \hat{y}, z) &= \delta _{ij} = \begin{cases}
			1, &\text{ falls }i = j\\
			0, &\text{ falls }i \neq j
		\end{cases} \\
		\partial_{z_j} G_n(\hat{y},z) &= \partial_{z_j} E((\hat{y},0),g(\hat{y},0),z) \qquad j=1,\dots,n
	\end{align*}
	Somit
	\[
		 \nabla_z G((x_0)_1, \dots, (x_0)_{n-1},z_0) = \begin{pmatrix}
		 	1 & 0 & \dots & \dots & 0 \\
			0 & 1 & \dots & \dots & 0 \\
			\vdots & \vdots & & & \vdots \\
			\partial_{z_1}E(x_0,v_0,z_0) & \dots & \dots & \dots & \partial_{z_n} E(x_0,v_0,z_0)
		 \end{pmatrix}
	\]
	und \[
		\det  \nabla_z G((x_0)_1,\dots,(x_0)_{n-1},z_0) = \partial_{z_n}E(x_0,v_0,z_0) \neq 0
	\]
	Daher existiert nach Satz $5.3$ ein $\delta >0$ und eine Funktion $\bar{z}$ so dass $G(\hat{y},\bar{z}(\hat{y}))=0$ für alle $\hat{y} \in \mathbb{R}^{n-1}$ 
	mit $\abs{(\hat{y},0)-x_0} < \delta $
\end{beweis}

\begin{bemerkung}
	Im Fall der quasilinearen Gleichung $b(x,u(x)) \cdot  \nabla u(x) + c(x,u(x)) = 0$ reduziert sich die Bedingung aus Lemma $5.6$ auf $b_n(x_0,g(x_0)) \neq 0$.
\end{bemerkung}

\minisec{Konstruktion lokaler Lösungen von (2)}
Wir nehmen an, dass es einen zulässigen und nicht charakteristischen Vektor $(x_0,v_0,z_0)$ gibt. Wir lösen die Familie von ODEs:
\begin{align*}
	\diffd{}{s} \gamma(\hat{y},s) = \gamma' ( \hat{y}, s) &=  \nabla _z E(\gamma(\hat{y},s),v(\hat{y},s),z(\hat{y},s)) \\
	v'(\hat{y},s) &=  \nabla _z E(\gamma(\hat{y},s),v(\hat{y},s),z(\hat{y},s)) \cdot z(\hat{y},s) \\
	z'(\hat{y},s) &= -  \nabla_x E( \gamma(\hat{y},s),v(\hat{y},s),z(\hat{y},s))- \partial_v E(\gamma(\hat{y},s),v(\hat{y},s),z(\hat{y},s))z(\hat{y},s)
\end{align*}
mit Anfangsbedinungen
\begin{align*}
	\begin{cases}
		\gamma(\hat{y},0) &= (\hat{y},0) \\
		v(\hat{y},0) &= g(\hat{y},0) \\
		z(\hat{y},0) &= \bar{z}(\hat{y},0)
	\end{cases}
\end{align*}
wobei $s \in [0,\delta ]$, $y \in  \partial \mathbb{R}^n_+$, $\hat{y} = (y_1, \dots, y_{n-1})$ mit $\abs{y-x_0} < \delta$. Die Funktion $\bar{z}$ und $\delta >0$ stammen hierbei aus Lemma $5.6$. \\
Wir nehmen einen Punkt $x \in \mathbb{R}^n_+$. \[
	\hat{y}(x), s(x): \qquad \gamma(\hat{y}(x),s(x)) = x.
\]
Wir wollen nun $u(x)$ in einer Umgebung in $\mathbb{R}^n_+$ von $x_0$ konstruieren. Das nächste Lemma garantiert, dass $(\gamma(\hat{y},s))_{\hat{y},s}$ eine Umgebung von $x_0$ in $\mathbb{R}^n_+$ überdecken. 

\begin{lemma}
	Sei $(x_0,v_0,z_0)$ ein zulässiger und nicht charakteristischer Vektor für (2). Dann existiert ein offenes Intervall $I \subseteq \mathbb{R}$ mit $0 \in I$, eine Umgebung $W$ von $x_0$ in $\partial \mathbb{R}^n_+$ und eine Umgebung $V$ von $x_0$ in $\mathbb{R}^n_+$, so dass für jedes $x \in V$ ein eindeutiges $s = s(x) >0$ in $I$ und $y = y(x) \in W$ existieren, so dass
	\[
		x = \gamma(\hat{y}(x),s(x)), \qquad \hat{y}(x) = (y(x)_1, \dots, y(x)_{n-1})
	\]
und die Abbildung $x \mapsto (\hat{y},s)$ von der Klasse $C^2$ ist.
\end{lemma}
\begin{beweis}
	$\gamma$ erfüllt $\gamma(\hat{x_0},0)= x_0$ mit $\hat{x_0}= ((x_0)_1, \dots , (x_0)_{n-1})$. Daher folgt die Behauptung des Lemmas nach dem Satz über die Umkehrabbildung, falls
	\[
		\det  \nabla _{(\hat{y},s)} \gamma(\hat{x_0},0) \neq 0
	\]
	gilt. Wegen der Anfangsbedingung $\gamma(\hat{y},0)= (\hat{y},0)$ gilt für $i=1,\dots,n-1$ und $j=1,\dots,n$
	\begin{align*}
		\partial_{y_i} \gamma_j ( \hat{y}, 0 ) &= \delta _{ij} \\
		\partial_s \gamma_j(\hat{y},s) = \gamma'_j(\hat{y},s) &= \partial_{z_j}E
	\end{align*}
	Es folgt für $j=1,\dots,n$
	\[
		\partial_s \gamma_j (\hat{x_0},0) = \partial_{z_j} E(x_0,v_0,z_0).
	\]
	\[
		 \nabla _{(\hat{y},s)} \gamma ( \hat{x_0},0) = \begin{pmatrix}
		 	1 & 0 & \dots & \partial_{z_1}E(x_0,v_0,z_0) \\
			0 & 1 & \dots & \partial_{z_2}E(x_0,v_0,z_0) \\
			0 & 0 & 1 & \dots \\
			0 & 0 & \dots & \partial_{z_n}E(x_0,v_0,z_0)
		 \end{pmatrix}
	\]
	und somit 
	\[
		\det (  \nabla_{(\hat{y},s)} \gamma(\hat{x_0},0)) = \partial_{z_n} E(x_0,v_0,z_0) \neq 0.
	\]
	Damit folgt die Behauptung. 
\end{beweis}

\minisec{Idee:}
Wir lösen die Charakteristische Gleichungen, die zu $\hat{y}(x)$ gehören und werten $v$ bei $s(x)$ aus. So erhalten wir einen Kandidaten $v(\hat{y}(x),s(x)) = u(x)$ 
für eine lokale Lösung von (2).

\begin{satz}[Evans Satz $107$]
	Sei $(x_0,v_0,z_0)$ ein zulässiger und nichtcharakteristischer Vektor, und sei $V$ die Umgebung von $x_0$ in $\mathbb{R}^n_+$ aus Lemma $5.7$. Dann ist die Funktion $u: V \to \mathbb{R}$ definiert als 
	\[
		u(x) : = v( \hat{y}(x), s(x))
	\]
	mit $(\hat{y}(x),s(x)) \in \mathbb{R}^{n-1} \times \mathbb{R}^+$ aus Lemma $5.7$, von der Klasse $C^2(V)$ eine lokale Lösung von (2) in $V$, d.h
	\[
		\begin{cases}
			E(x,u(x), \nabla u(x)) = 0 , &\text{ in }V\\
			u(x) = g(x), & \text{ auf } \partial \mathbb{R}^n_+ \cap V
			
		\end{cases}.
	\]
\end{satz}

%%%% neue Vorlesung

	\[
		\begin{cases}
			E(x,u(x), \nabla u(x)) = 0 , &\text{ in }\mathbb{R}^n_+\\
			u(x) = g(x), & \text{ auf } \partial \mathbb{R}^n_+ \cap V
		\end{cases}.
	\]
$\partial_{z_n} E(x_0,v_0,z_0) \neq 0$ für $(x_0,v_0,z_0) \in \partial \mathbb{R}^n_+ \times \mathbb{R} \times \mathbb{R}^n$.

\begin{beispiele}
	\begin{enumerate}[1.]
		\item \[
			\begin{cases}
				u_{x_2} + u = 0, &\text{ in }\mathbb{R}^2_+\\
				u=g , & \text{ auf } \partial \mathbb{R}^2_+
			\end{cases}
		\]
		Dann gilt $E(x,v,z) = z_2 +v$
		\[
			x_0 \in \partial \mathbb{R}^2_+, \qquad v_0 = g(x_0), \qquad (z_0)_1 = g'(x_0), \qquad \partial \mathbb{R}^2_+ = \set{x_2 = 0}
		\]
		\[
			(z_0)_2 + v_0 = 0
		\]
		Damit sind alle Vektoren $(x_0,v_0,z_0)$ zulässig. Außerdem gilt $\partial_{z_2}E(x_0,v_0,z_0) = 1 \neq 0$. Alle Vektoren $(x_0,v_0,z_0) \in \partial \mathbb{R}^2_+ \times \mathbb{R} \times \mathbb{R}^2$ sind nichtcharakteristisch.
		\[
			x_0 \in \partial \mathbb{R}^2_+, \qquad x_0 = (x_1,0), \qquad x_1 \in \mathbb{R}
		\]
		\[
			\gamma'(s) =  \nabla_z E = (0,1), \qquad \gamma(0) = (x_1,0) 
		\]
		\[
			v'(s) =  \nabla_z E \cdot z = (0,1) \cdot z = z_2 = -v , \qquad v(0)= g(x_1)
		\]
		\[
			\Rightarrow \qquad \gamma(s) = (x_1,s), \qquad v(s) = c e^{-s}, \qquad v(0) = g(x_1)
		\]
		\[
			\Rightarrow \qquad v(s) = g(x_1) e^{-s}
		\]
		Also gilt $(x_1,x_2) = x \in \mathbb{R}^2_+$ und damit folgt $x= \gamma(x_2)$. Wir erhalten
		\[
			u(x_1,x_2) = u(\gamma(x_2)) = v(x_2) = g(x_1) e^{-x_2}
		\]für $g \in C^1(\partial \mathbb{R}^2_+)$. 
		\[
			u(x_1,x_2) = g(x_1)e^{-x_2} 
		\]
		ist also die globale klassische Lösung.
		\item
		\[
		(2) \qquad 	\begin{cases}
				u_{x_1}+u =0, &\text{ in }\mathbb{R}^2_+\\
				u=g, &\text{ auf } \partial \mathbb{R}^2_+
				
			\end{cases}
		\]
		Wie gewohnt erhalten wir $E(x,v,z) = z_1 + v$. Es gilt somit $\partial_{z_2}E \equiv  0$. Damit sind alle Vektoren $(x_0,v_0,z_0) \in \partial \mathbb{R}^2_+ \times \mathbb{R} \times \mathbb{R}^2$ charakteristisch.
		\[
			\begin{cases}
				\gamma'(s) &= (1,0) \\
				\gamma(0) &= (x_1,0) \\
				v'(s) &= -v(s) \\
				v(0) &=g(x_1)
			\end{cases} \qquad (x_1,0)= x_0 \in \partial \mathbb{R}^2_+
		\]
		Es gilt
		\[
			\gamma(s) = (s+ x_1,0),
		\]
		was bedeutet, dass $\gamma$ komplett am Rand verläuft. 
		\[
			v(s) = g(x_1) e^{-s}, \qquad u(\gamma(s)) = v(s), \qquad u(s+x_1,0) = g(x_1) e^{-s}
		\]
		Es gilt
		\[
			u(\bar{x_1},0) \stackrel{s = \bar{x_1}-x_1}{=} u((s+x_1),0) \stackrel{s = \bar{x_1}-x_1}{=} g(x_1)e^{-(\bar{x_1}-x_1)}
		\]
		und somit
		\[
			u(\bar{x_1},0) = g(x_1) e^{-(\bar{x_1}-x_1)} 
		\]
		für $x_1 < \bar{x_1}$. 
		\[
			u(\bar{x_1},0) = g(\bar{x_1}) \qquad \Rightarrow \qquad g(x_1) e^{-(\bar{x_1}-x_1)} = g(\bar{x_1}) \qquad (*)
		\]
		Möglich für $g(x_1) = e^{-x_1}$ und daraus folgt
		\[
			e^{-x_1}e^{-(\bar{x_1}-x_1)} = e^{-\bar{x_1}} = g(\bar{x_1})
		\]
		für die Anfangsdaten $g(x_1) = e^{-x_1}$ ist $(*)$ erfüllt und eine globale Lösung von $(2)$ ist gegeben durch
		\[
			u(x_1,x_2) = h(x_2) e^{-x_1} 
		\]
		für eine beliebige $C^1$-Funktion $h$: $h(0)=1$. Es folgt
		\[
			u(x,0) = e^{-x_1}
		\]
		Dies bedeutet, dass wir unendlich viele globale Lösungen erhalten. Für $g \equiv c \in \mathbb{R}$ gilt	
		\[
			c e^{-(\bar{x_1}-x_1)} = c \qquad \Rightarrow \qquad e^{-(\bar{x_1}-x_1)} =1
		\]
		aber $\bar{x_1} \neq x_1$.
	\end{enumerate}
\end{beispiele}

\subsection{Skalare Erhaltungsgleichungen} 
\label{sub:skalare_erhaltungsgleichungen}
\[
	(1) \qquad u_t(t,x) + (F(u))_x = 0
\] für $t >0$, $x \in \mathbb{R}$ und $F \in C^1(\mathbb{R})$.
Für $u \in C^1(\mathbb{R}_+ \times \mathbb{R})$ Lösung von $(1)$ gilt 
\begin{align*}
	\int_{a}^{b} u_t(t,x) \,\mathrm{d}x &= - \int_{a}^{b} (F(u))_x \,\mathrm{d}x \\
	&= F(u(t,a))- F(u(t,b)) \\ &= \diffd{}{t} \int_{a}^{b} u  \,\mathrm{d}x \qquad \forall\, [a,b] \subseteq \mathbb{R}
\end{align*}
Die Änderung der Gesamtmasse in einem beliebigen Intervall $[a,b]$ ist nur gegeben durch den Fluss $F$ von $u$ durch die Randpunkte $a$ und $b$. (= Erhaltungsgleichung).
\[
	(2) \qquad \begin{cases}
		u_t + (F(u))_x = 0, &\text{ in }\mathbb{R}_+ \times \mathbb{R}\\
		u=g , &\text{ auf } \set{0} \times \mathbb{R}
	\end{cases}
\]
Falls $u \in C^1(\mathbb{R}_+ \times \mathbb{R})$ eine Lösung von $(2)$ ist, so gilt
\[
	(3) \qquad u_t + (F(u))_x = u_t +F'(u) u_x  = 0 \qquad \text{ in } \mathbb{R}_+ \times \mathbb{R}
\]
und somit können wir schreiben 
\[
	E(x,v,z) = z_1 + F'(v) z_2
\]
Wir erhalten die Charakteristiken
\[
	\begin{cases}
		\gamma(s) &= (1, F'(v(s)))\\
		\gamma(0) &= (0,x_0) \\
		v'(s) &=  \nabla _z E \cdot z = (1, F'(v(s))) \cdot z(s) = 0, \qquad v(s) = v(0) = g(x_0) \\
		v(0) &= g(x_0)
 	\end{cases}
\]
es folgt
\[
	\gamma(s) = (s, F'(g(x_0))s + x_0).
\]
Damit ist 
\[
	u(s,F'(g(x_0))s + x_0) = g(x_0) 
\]
eine Lösungsformel für eine klassische Lösung.
\begin{bemerkung}
	\begin{enumerate}[(i)]
		\item Die Charakteristiken sind Geraden
		\item Die Lösung ist konstant entlang von $\gamma(s)$.
	\end{enumerate}
\end{bemerkung}
\begin{enumerate}[1.)]
	\item $F$ Linear: $F(u) = au$ mit $a \in \mathbb{R}$. Dann ist $F'(u) = a$.
	$\gamma(s)=(s,as+x_0)$ sind dann parallel zueinander mit konstanter Steigung.
	\[
		v(s,as+x_0) = g(x_0), \qquad t=s, \, x=at+x_0 \qquad \Rightarrow \qquad u(t,x) = g(x-at)
	\]
	globale klassische Lösung von $(2)$. ($g \in C^1$ $\Rightarrow$ $u \in C^1$ ). \\
	\[
		u_t + F'(u) u_x =0 \qquad \Rightarrow \qquad u_t+ a u_x =0 
	\]
	die Transportgleichung.
	\item $F$ Nichtlinear: $F(u)= \frac{1}{2} u^2$, also $u_t + u u_x = 0$ die Bewegungsgleichung. $F'(u)=u$. Es folgt
	\[
		\gamma(s) = (s,g(x_0)s+x_0)
	\]
 	das heißt die Steigung der Charakteristiken hängt von $g$ ab.
	\[
		u(s,g(x_0)s+x_0) = g(x_0)
	\]
	Es existieren $x_0$,$\bar{x_0}$ sodass $g(x_0) \neq g(\bar{x_0})$ und $g(x_0)s+ x_0 = g(\bar{x_0})s + \bar{x_0}$. In diesem Fall gibt es (mindestens) zwei unterschiedliche Vorschriften $(g(x_0),g(\bar{x_0}))$ für den Wert von $u$ am Schnittpunkt, sodass keine globale klassische Lösung existieren kann. Wir berechnen den ersten Zeitpunkt, an dem die Charakteristiken sich schneiden:
	\[
		g(x_0)t + x_0 = g(\bar{x_0})t+ x_0
	\]
	\[
		\Leftrightarrow \qquad t(g(x_0)-g(\bar{x_0})) = \bar{x_0}-x_0 \qquad \Rightarrow \qquad t = \frac{\bar{x_0}-x_0}{g(x_0)-g(\bar{x_0})}
	\]
	\begin{align*}
		T:&= \inf_{x_0 \neq \bar{x_0}} \left( \frac{\bar{x_0}-x_0}{g(x_0)-g(\bar{x_0})} \right) \\
		&= \inf_{x_0 \neq \bar{x_0}} \left( - \frac{1}{\frac{g(x_0)- g(\bar{x_0})}{x_0-\bar{x_0}}} \right) \\
		& = - \sup_{x_0 \neq \bar{x_0}} \left( \frac{1}{\frac{g(x_0)-g(\bar{x_0})}{x_0-\bar{x_0}}} \right) \\
		&= - \frac{1}{\inf_{x_0 \neq \bar{x_0}}\left( \frac{g(x_0)-g(\bar{x_0})}{x_0-\bar{x_0}} \right)}
	\end{align*}
	Falls $0 < T < \infty$ folgt, dass eine klassische Lösung nur für $t < T$ existiert. Für $t=T$ ist die Lösung unstetig.
	\begin{beispiel}
		\[
			\begin{cases}
				u_t + uu_x = 0, &\text{ in }\mathbb{R}_+ \times \mathbb{R}\\
				u(0,x) = g(x), & \text{ für } x \in \mathbb{R}
			\end{cases}
		\]
		mit
		\[
			g(x) = \begin{cases}
				1, &\text{ falls }x \leq 0\\
				1-x, &\text{ falls } 0 \leq x \leq 1 \\
				0, &\text{ falls }x >1
			\end{cases}
		\]
		Für $x \in (0,1)$ gilt $g(x_0) - g(\bar{x_0}) = 1 - x_0 - 1+ \bar{x_0}$.
		Es folgt
		\[
			\frac{g(x_0)-g(\bar{x_0})}{x_0- \bar{x_0}} = \frac{\bar{x_0}-x_0}{x_0-\bar{x_0}} = -1
		\]
		und somit
		\[
			T=1
		\]
		Hier fehlt eine hilfreiche Skizze.  \\
		Wir haben eine Lösung nur für $t<1:$
		\[
			u(t,x) = \begin{cases}
				1, &\text{ für }x \leq t, \, t \in [0,1)\\
				0, &\text{ für }x \geq 1, \, t \in [0,1)\\
				\frac{1-x}{1-t},& \text{ für }t \leq x \leq 1, \, t \in [0,1),
			\end{cases}
		\]
		da 
		\[
			u(t,(1-x_0)t+x_0) = g(x_0) = 1-x_0
		\]
		\[
			x = (1-x_0)t + x_0 = t + x_0 (1-t)
		\]
		\[
			\Rightarrow x_0 = \frac{x-t}{1-t}
		\]
		\[
			u(t,x)= 1- \frac{x-t}{1-t} = \frac{1-t-x+t}{1-t} = \frac{1-x}{1-t}
		\]
	\end{beispiel}
	\begin{bemerkung}
		\begin{itemize}
			\item Auch für $g \in C^{\infty}(\mathbb{R})$ haben wir keine klassische Lösung.
			\item Auch eine Lösung, die zum Zeitpunkt glatt war, entwickelt in endlicher Zeit Unstetigkeiten.
		\end{itemize}
	\end{bemerkung}
\end{enumerate}
\minisec{Erinnerung:}
Betrachten 
\[
	(2) \qquad \begin{cases}
		u_t + (F(u))_x = 0, &\text{ in }\mathbb{R}_+ \times \mathbb{R}\\
		u=g , &\text{ auf } \set{0} \times \mathbb{R}
	\end{cases}
\]
\begin{description}
	\item[Problem:]Selbst für stetige Anfangsdaten $g$ entwickeln Lösungen von (2) Unstetigkeiten innerhalb endlicher Zeit.
	\item[Idee:] Führe schwachen Lösungsbegriff ein. Dazu Multipliziere die Gleichung mit einer Testfunktion $\varphi \in C^{\infty}_0([0,\infty) \times \mathbb{R})$
	und integriere in Raum und Zeit.
	\item[Motivation] Sei $u \in C^1(\mathbb{R}^+ \times \mathbb{R})$ glatte Lösung von (2) und $\varphi \in C^{\infty}_0([0, \infty) \times \mathbb{R})$. Dann gilt
	\[
		0 = \int_{0}^{+ \infty} \int_{\mathbb{R}}^{} (u_t + F(u)_x) \varphi \,\mathrm{d}x \,\mathrm{d}t \stackrel{\text{P.I}}{=} \int_{0}^{\infty} \int_{\mathbb{R}}^{}
		u \varphi_t + F(u) \varphi_x \,\mathrm{d}x \,\mathrm{d}t - \int_{\mathbb{R}}^{} u(0,x)\varphi(0,x) \,\mathrm{d}x
	\] 
	Die rechte Seite macht auch für $u \in L^{\infty}(\mathbb{R}^+ \times \mathbb{R})$ Sinn. Dies motiviert die folgende Definition.
\end{description}
\begin{definition*}
	Sei $u \in L^{\infty}(\mathbb{R}^+ \times \mathbb{R})$. Wir sagen, dass $u$ eine schwache Lösung von (2) ist, falls für alle $\varphi \in C^{\infty}_0([0,\infty) \times \mathbb{R})$ gilt
	\[
		\int_{0}^{\infty} \int_{\mathbb{R}}^{} (u \varphi_t + F(u) \varphi_x) \,\mathrm{d}x \,\mathrm{d}t + \int_{\mathbb{R}}^{} g(x) \varphi(x) \,\mathrm{d}x = 0
	\]
\end{definition*}
\begin{bemerkung}
	\begin{enumerate}[(i)]
		\item Schwache Lösungen müssen weder stetig, noch differenzierbar sein.
		\item Jede klassiche Lösung ist auch eine schwache Lösung.
		\item Schwache Lösungen, die lokal in $C^1$ sind, sind auch lokale klassische Lösungen von (2). 
	\end{enumerate}
\end{bemerkung}
\begin{lemma}
	Sei $u \in L^{\infty}(\mathbb{R}^+ \times \mathbb{R})$ eine schwache Lösung von (2). Weiter existiere $ O \subseteq  \mathbb{R}^+ \times \mathbb{R}$ offen, sodass $u \in C^1(O)$. Dann löst $u$ das AWP (2) in $O$ im klassischen Sinne.
\end{lemma}
\begin{beweis}
	Sei $u \in L^{\infty}(\mathbb{R}^+ \times \mathbb{R})$ schwache Lösung von (2). Dann gilt für alle $\varphi \in C^{\infty}_0([0,\infty) \times \mathbb{R})$
	\[
		\int_{0}^{\infty} \int_{\mathbb{R}}^{} u \varphi_t + F(u) \varphi_x \,\mathrm{d}x \,\mathrm{d}t + \int_{\mathbb{R}}^{} g(x)\varphi(0,x) \,\mathrm{d}x = 0.
	\]
	Insbesondere gilt für alle $\psi \in C^{\infty}_0(O) \subseteq C_0^{\infty}([0,\infty)\times \mathbb{R})$
	\[
		0 = \int_{O}^{} u \psi_t + F(u) \psi_x \,\mathrm{d}x \,\mathrm{d}t \stackrel{\text{P.I}}{=} \int_{O}^{} (u_t+F(u)_x)\psi \,\mathrm{d}x \,\mathrm{d}t
	\]
	Damit gilt
	\[
		u_t + F(u)_x = 0 \qquad \text{ in }O
	\]
\end{beweis}
\begin{lemma}[Rankine-Hugoniot Sprungbedingung]
	Sei $u \in L^{\infty}(\mathbb{R}^+ \times \mathbb{R})$ schwache Lösung von (2). Sei $O \subseteq \mathbb{R}^+ \times \mathbb{R}$ offen und $C$ eine $C^1$-Kurve, 
	die $O$ in zwei disjunkte offene Teilmengen $O_l$ und $O_r$ zerlegt, d.h 
	\[
	 	O = O_l \, \dot\cup \, C \, \dot\cup \, O_r
	\]
	Seien weiter $u_l \in C^1(O_l) \cap C^0(O_l \cup C)$ und $u_r \in C^1(O_r) \cap C^0(O_r \cup C)$ mit $u = u_l$ in $O_l$ und $u= u_r$ in $O_r$. Dann muss gelten
	\[
		\begin{pmatrix}
			u_l - u_r \\
			F(u_l)- F(u_r)
		\end{pmatrix} \cdot \nu = 0, \qquad \text{auf }C
	\]
	wobei $\nu$ die äußere Normale an $O_l$ ist.
\end{lemma}
\begin{beweis}
	Nach Lemma $5.9$ wissen wir, dass folgenes gilt
	\begin{align*}
		\partial_t u_l + F(u_l)_x &= 0 \qquad \text{ in }O_l \\
		\partial_t u_r + F(u_r)_x &= 0 \qquad \text{ in }O_r
	\end{align*}
	Weiter gilt für alle $\varphi \in C_0^{\infty}(O)$
	\begin{align*}
		0 &= \int_{O}^{} u \varphi_t + F(u) \varphi_x \,\mathrm{d}x\,\mathrm{d}t \\
		&= \int_{O_l}^{} u_l \varphi_t + F(u_l) \varphi_x \,\mathrm{d}x\,\mathrm{d}t  +  \int_{O_r}^{} u_r \varphi_t + F(u_r) \varphi_x \,\mathrm{d}x\,\mathrm{d}t \\
		&= \int_{O_l}^{} \begin{pmatrix}
			u_l \\ F(u_l)
		\end{pmatrix} \cdot  \nabla_{t,x} \varphi \,\mathrm{d}x \, \mathrm{d}t + \int_{O_r}^{} \begin{pmatrix}
			u_r \\ F(u_r)
		\end{pmatrix} \cdot  \nabla_{t,x} \varphi \,\mathrm{d}x \\
		&= - \int_{O_l}^{} \underset{=0}{\underbrace{(\partial_t u_l + F(u_l)_x)}} \varphi \,\mathrm{d}x \,\mathrm{d}t + \int_{C}^{} \begin{pmatrix}
			u_l \\ F(u_l)
		\end{pmatrix} \cdot \nu \varphi \,\mathrm{d}S \\
		& \qquad - \int_{O_r}^{} \underset{=0}{\underbrace{(\partial_t u_r + F(u_r)_x)}} \varphi \,\mathrm{d}x \,\mathrm{d}t - \int_{C}^{} \begin{pmatrix}
					u_r \\ F(u_r)
				\end{pmatrix} \cdot \nu \varphi \,\mathrm{d}S
	\end{align*}
	Damit gilt
	\[
		\int_{C}^{} \begin{pmatrix}
			u_l - u_r \\ F(u_l)-F(u_r)
		\end{pmatrix} \cdot \nu \varphi \,\mathrm{d}S=0 \qquad \forall\, \varphi \in C^{\infty}_0(O)
	\]
	\[
		\Rightarrow \begin{pmatrix}
			u_l-u_r \\ F(u_l)- F(u_r)
		\end{pmatrix} \cdot \nu =0 \qquad \text{ auf }C.
	\]
\end{beweis}

\begin{bemerkung}
	Die Rankine-Hugoniot Sprungbedingung nimmt für $C = \set[(t,\lambda t)]{t >0}$ für ein $\lambda \in \mathbb{R}$ die einfache Form
	\[
		F(u_l)-F(u_r) = \lambda (u_l-u_r) \qquad \text{ auf }C
	\]
	an.
 
Diese einfache Darstellung wollen wir nun nutzen, um die lokale Lösung der Burgers-Gleichung mit Anfangsdaten
\[
	g(x) = \begin{cases}
		1, &\text{ falls }x \leq 0\\
		1-x, , &\text{ falls }0 \leq x \leq 1 \\
		0, &\text{ falls }x \geq 1
	\end{cases}
\]
und \[
	F(u) = \frac{1}{2} u^2
\]
zu einer globalen schwachen Lösung fortzusetzen. \\
Idee: Wir wissen bereits, dass für $t \in [0,1)$ eine klassische Lösung gegeben ist durch
\[
	u_1(t,x) = \begin{cases}
		1, &\text{ falls }x \leq t, \,t \in [0,1)\\
		\frac{1-x}{1-t} , &\text{ falls }t \leq x \leq 1, \, t \in [0,1) \\
		0, &\text{ falls }x \geq  1, \,t \in [0,1)
		\end{cases}
\]
das heißt bis zum Zeitpunkt $t=1$ (welches der früheste Zeitpunkt ist, an dem sich die Charakteristiken schneiden) gibt es keine Probleme, jeder Punkt $(t,x)$ liegt auf einer eindeutig bestimmten Charakteristik und wir ordnen dann den Wert zu, den $g$ auf dieser Charakteristik für $t=0$ annimmt. Für $t=1$ können wir nun ein neues AWP betrachten, genauer suchen wir nach einer schwachen Lösung $u_2$ von $u_t+F(u)_x = 0$ mit Anfangsdaten
\[
	u_2(1,x) = \begin{cases}
		1, &\text{ falls }x \leq 0\\
		0, &\text{ falls }x \geq 0
	\end{cases} 
\]
Einfache Idee: $u_2$ ist stückweise konstant und spingt entlang einer geeigneten Kurve $C$ von $1$ nach $0$. Mit der R-H Sprungbedingung muss gelten:
\[
	F(1)- F(0) = \frac{1}{2} \stackrel{!}{=} \frac{1}{2} ( 1 - 0) = \lambda.
\]
Deshalb muss diese Kurve mit der Sprungbedingung die konstante Steigung $\frac{1}{2}$ haben. In Formeln:
\[
	u(t,x) = \begin{cases}
		u_1(t,x), &\text{ falls }0 \leq t<1\\
		u_2(t,x), &\text{ falls }t \geq 1
	\end{cases},
\]
wobei
\[
	u_1(t,x) = \begin{cases}
		1, &\text{ falls }x \leq t, \,t \in [0,1)\\
		\frac{1-x}{1-t} , &\text{ falls }t \leq x \leq 1, \, t \in [0,1) \\
		0, &\text{ falls }x \geq  1, \,t \in [0,1)
		\end{cases}
\]
\[
	u_2(t,x) = \begin{cases}
		1, &\text{ falls }x \leq \frac{t}{2}+ \frac{1}{2}, \,t \geq 1\\
		0, &\text{ falls }x > \frac{t}{2} + \frac{1}{2}, \,t \geq 1
	\end{cases}
\]
Es gilt dann:
\[
	u_2(1,x) = \begin{cases}
		1, &\text{ falls }x \leq 0\\
		0, &\text{ falls }x \geq 0
	\end{cases} 
\]
\end{bemerkung}

\newpage
\section{Sobolevräume} 
\label{sec:sobolevraume}

Ziel: Übertrage schwachen Lösungsbegriff auch auf andere PDEs. Dazu erweisen sich die Sobolevräume als geeignete Funktionenräume für schwache Lösungen.
\subsection{Wiederholung: Lebesgueräume} 
\label{sub:wiederholung_lebesgueraume}
Literatur:
\begin{itemize}
	\item Alt: Lineare Funktionalanalysis Kapitel $1$.
	\item Brezys: Functional Analysis, Kapitel $4$.
\end{itemize}

\begin{definition*}[$L^p$-Räume]
	Sei $\Omega \subseteq  \mathbb{R}^n$, $p \in [1,\infty)$. Dann definieren wir 
	\[
		L^p(\Omega) := \set[u: \Omega \to \mathbb{R} \text{ messbar}]{ \int_{\Omega}^{} \abs{u}^p \,\mathrm{d}x < + \infty}
	\]
	und 
	\[
		\norm{u}_{L^p(\Omega)} := \left( \int_{\Omega}^{}\abs{u}^p \,\mathrm{d}x \right)^{\frac{1}{p}}.
	\]
	Weiter setzen wir
	\[
		L^{\infty}(\Omega) := \set[u: \Omega \to \mathbb{R} \text{ messbar}]{\esssup_{x \in \Omega}\abs{u(x)} < + \infty}
	\]
	wobei
	\begin{align*}
		\esssup_{x \in \Omega} \abs{u(x)} &= \inf_{N \text{ Nullmenge}} \sup_{x \in \Omega \setminus N} \abs{u(x)} \\
		&= \inf \set[C >0]{\abs{u(x)}\leq C \text{ fast überall in }\Omega} \\
		&=: \norm{u}_{L^{\infty}}
	\end{align*}
\end{definition*}
\begin{bemerkung}
	\begin{enumerate}[(i)]
		\item Die Funktionen $u \in L^p(\Omega)$ sind nicht stetig (insbesondere können sie auf Nullmengen abgeändert werden). Wir sagen, dass $u,v$ in $L^p$
		gleich sind, falls $u=v$ fast überall. Genauer betrachten Äquivalenzklassen. (Dies wird nicht in dieser Vorlesung behandelt)
		\item $L^p_{\text{loc}}(\Omega)$ ist der Raum, der alle Funktionen $u: \Omega \to  \mathbb{R}$ enthält, sodass $u  \Big|_{\chi_K}^{} \in L^p(\Omega)$ 
		für alle $K \subseteq \Omega$ kompakt, wobei $\chi_K$ die charakteristische Funktion auf $K$ ist.
	\end{enumerate}
\end{bemerkung}
\begin{satz}[Vollständigkeit]
	Sei $\Omega \subseteq  \mathbb{R}^n$, $1 \leq p \leq \infty$. Dann ist $(L^p(\Omega),\norm{.}_{L^p(\Omega)})$ ein Banachraum. Insbesondere existiert für jede Cauchyfolge $(u_n)_{n \in \mathbb{N}}$ in $L^p(\Omega)$ bezüglich $\norm{.}_{L^p(\Omega)}$, so existiert ein $u \in L^p(\Omega)$ mit 
	\[
		\norm{u_n - u}_{L^p(\Omega)} \to 0.
	\]
	Für eine Teilfolge $(u_{n_k})_k$ von $(u_n)_n$ gilt außerdem $u_{n_k} \to u$ fast überall.
\end{satz}

\begin{bemerkung}
	\begin{enumerate}[(i)]
		\item Um die Dreiecksungleichung zu beweisen, benutzt man die Hölder- \\Ungleichung:\\
		Für $p \in [1,\infty]$ und $p'$ so, dass $\frac{1}{p}+ \frac{1}{p'} =1$, wobei $p'=\infty$ für $p=1$ und anders herum. Seien außerdem $u \in L^p(\Omega)$ und 
		$v \in L^{p'}(\Omega)$. Dann gilt
		\[
			\abs{\int_{\Omega}^{}uv \,\mathrm{d}x} \leq \int_{\Omega}^{}\abs{uv} \,\mathrm{d}x \leq \norm{u}_{L^p(\Omega)} \, \norm{v}_{L^{p'}(\Omega)}
		\]
		\item Für $p \in (1, \infty)$ ist $L^{p'}(\Omega)$ gerade der Dualraum von $L^p(\Omega)$, d.h
		\[
			L^{p'}(\Omega) = (L^p(\Omega))'
		\]
		\item Für $p=2$ ist $L^2(\Omega)$ ein Hilbertraum mit dem Skalarprodukt
		\[
			\skal{u}{v}_{L^2(\Omega)} := \int_{\Omega}^{}uv \,\mathrm{d}x
		\]
		Außerdem gilt für $p=2$, dass $p' = 2$, also
		\[
			L^2(\Omega) = (L^2(\Omega))'.
		\]
	\end{enumerate}
\end{bemerkung}
\begin{satz}[Dichtheit glatter Funktionen]
	Sei $\Omega \subseteq \mathbb{R}^n$ offen, $1 \leq p < \infty$. Dann ist der Raum $C^{\infty}_0(\Omega) \subseteq L^p(\Omega)$ dicht bezüglich $\norm{.}_{L^p(\Omega)}$, d.h. für jedes $u \in L^p(\Omega)$ existiert eine Folge $(u_n)_n$ in $C^{\infty}_0(\Omega)$ mit
	\[
		\norm{u_n-u}_{L^p(\Omega)} \to 0
	\]
	Das Resultat gilt nicht für $p=\infty$!
\end{satz}

\begin{satz}[Hauptsatz der Variationsrechnung]
	Sei $\Omega \subseteq \mathbb{R}^n$ offen, $u \in L^1_{\text{loc}}(\Omega)$ mit 
	\[
		\int_{\Omega}^{} u \varphi \,\mathrm{d}x = 0 \qquad \forall\, \varphi \in C^{\infty}_0(\Omega)
	\]
	Dann gilt $u=0$ fast überall in $\Omega$.
\end{satz}
\begin{beweis}[Für $u \in L^2(\Omega) \subseteq L^1_{\text{loc}}$]
	Sei $u \in L^2(\Omega)$ mit
	\[
		\int_{\Omega}^{} u \varphi \,\mathrm{d}x = 0 \qquad \forall\, \varphi \in C^{\infty}_0(\Omega)
	\]
	Wähle $\varepsilon >0$ beliebig. Dann existiert nach $6.2$ ein $\varphi \in C^{\infty}_0(\Omega)$ mit $\norm{u-\varphi}_{L^2(\Omega)} < \varepsilon$. Dann 
	\[
		\norm{u}^2_{L^2(\Omega)} = \int_{\Omega}^{} u^2 \,\mathrm{d}x = \int_{u}^{}u (u-\varphi) \,\mathrm{d}x \stackrel{\text{Hölder}}{\leq } \norm{u}_{L^2(\Omega)} \underset{< \varepsilon}{\underbrace{\norm{u-\varphi}_{L^2(\Omega)}}}
	\]
\end{beweis}
\subsection{Schwache Ableitungen} 
\label{sub:schwache_ableitungen}
\minisec{Motivation:} Sei $u \in C^1(\Omega)$, $\varphi \in C^{\infty}_0(\Omega)$. Dann gilt
\[
	\int_{\Omega}^{}u_{x_i}\varphi \,\mathrm{d}x = - \int_{\Omega}^{}u \varphi_{x_i} \,\mathrm{d}x \qquad \forall\,  i=1,\dots,n.
\]
Das rechte Integral ist definiert für $u \in L^1_{\text{loc}}(\Omega)$.
\begin{definition*}[Schwache Ableitung]
	Sei $\Omega \subseteq \mathbb{R}^n$, $u \in L^1_{\text{loc}}(\Omega)$. Dann zeigen wir, dass $u$ schwach differenzierbar ist, falls für alle $i \in \set{1,\dots,n}$ ein $v_i \in L^1_{\text{loc}}(\Omega)$ existiert mit
	\[
		\int_{\Omega}^{}u \varphi_{x_i} \,\mathrm{d}x = - \int_{\Omega}^{}v_i \varphi \,\mathrm{d}x \qquad \forall\, \varphi \in C^{\infty}_0(\Omega).
	\]
	$v_i \in L^1_{\text{loc}}(\Omega)$ ist durch die obige Gleichung eindeutig estimmt und wir setzten $\diff{}{x_i}u = v_i$ und nennen $v_i$ die schwache Ableitung von $u$ nach $x_i$. Weiter setzen wir
	\[
		 \nabla u := ( \diff{}{x_1}u, \dots , \diff{}{x_n} u).
	\]
\end{definition*}

\begin{bemerkung}
	\begin{enumerate}[(i)]
		\item Die schwache Ableitung ist eindeutig. (Konsequenz aus $6.3$)
		\item Ist $u \in C^1(\Omega)$, so stimmt die klassische Ableitung mit der schwachen Ableitung überein.
	\end{enumerate}
\end{bemerkung}
\begin{beispiel}
	\begin{enumerate}[(i)]
		\item Sei $\Omega = (-1,1) \subseteq \mathbb{R}^n$, $u(x) = \abs{x}$, $u \in L^1(\Omega)$. Wähle $\varphi \in C^{\infty}_0(\Omega)$. Dann gilt
		\begin{align*}
			\int_{-1}^{1} \abs{x} \varphi'(x) \,\mathrm{d}x &= - \int_{-1}^{0} x \varphi'(x) \,\mathrm{d}x + \int_{0}^{1} x \varphi'(x) \,\mathrm{d}x \\
			&= \underset{=0}{\underbrace{\varphi(-1)}} + \int_{-1}^{0} \varphi(x) \,\mathrm{d}x + \underset{=0}{\underbrace{\varphi(1)}} - \int_{0}^{1} \varphi(x) \,\mathrm{d}x \\
			&= - \int_{-1}^{0} - \varphi(x) \,\mathrm{d}x - \int_{0}^{1} \varphi(x) \,\mathrm{d}x \\
			&= - \int_{-1}^{1} \sign(x) \varphi(x) \,\mathrm{d}x
		\end{align*}
		Nun ist also $x \mapsto \sign(x)$ in $L^1_{\text{loc}}(\Omega)$ und somit ist $u'(x) = \sign(x)$ die schwache Ableitung der Betragsfunktion.
		\item Sei wieder $\Omega = (-1,1)$ und setze
		\[
			H(x) := \begin{cases}
				1, &\text{ falls }x \geq 0\\
				0, &\text{ falls }x < 0
			\end{cases}
		\]
		Wir behaupten, dass $H$ keine schwache Ableitung besitzt. \\
		\underline{Annahme:} Es existiert $\delta \in L^1_{\text{loc}}(\Omega)$ und 
		\[
			\int_{-1}^{1} H(x)\varphi'(x) \,\mathrm{d}x = - \int_{-1}^{1} \delta (x) \varphi(x) \,\mathrm{d}x \qquad \forall\, \varphi \in C^{\infty}_0(\Omega)
		\]
		Sei nun $\psi \in C^{\infty}_0(0,1)$. Setze 
		\[
			\tilde \psi(x) := \begin{cases}
				\psi(x), &\text{ falls }x \geq 0\\
				0, &\text{ falls } -1 \leq x \leq 0
			\end{cases}
		\]
		Dann gilt $\tilde \psi \in C^{\infty}_0(\Omega)$. Nach Annahme muss gelten
		\[
			- \int_{-1}^{1} \delta (x) \tilde \psi(x) \,\mathrm{d}x = \int_{-1}^{1} H(x) \tilde \psi'(x) \,\mathrm{d}x = \int_{0}^{1} \psi'(x) \,\mathrm{d}x = \psi(1)-\psi(0) = 0
		\]
		Damit folgt aber auch
		\[
			- \int_{0}^{1}\delta (x) \psi(x) \,\mathrm{d}x = 0
		\]
		Mit dem Hauptsatz der Variationsrechnung gilt dann $\delta =0$ fast überall in $(0,1)$. \\
		Analog sei $\psi in C^{\infty}_0(-1,0)$ beliebig. Setze
		\[
			\tilde \psi(x) := \begin{cases}
				\psi(x), &\text{ falls }x \leq 0\\
				0, &\text{ falls } 0 \leq x \leq 1
			\end{cases}
		\]
		Dann folgt wieder $ \tilde \psi \in C^{\infty}_0(\Omega)$ und es gilt
		\[
			- \int_{-1}^{0}\delta (x) \psi(x) \,\mathrm{d}x = - \int_{-1}^{1} \delta (x)  \tilde \psi(x) \,\mathrm{d}x = \int_{-1}^{1} H(x) \tilde \psi'(x) \,\mathrm{d}x = 0
		\]
		Damit gilt wieder $\delta  =0$ fast überall in $(-1,0)$. Starten wir nun mit einem beliebigen $\varphi \in C^{\infty}_0(\Omega)$, so gilt
		\[
			0 = \int_{-1}^{1} \delta (x) \varphi(x) \,\mathrm{d}x = - \int_{-1}^{1}H(x) \psi'(x) \,\mathrm{d}x = - \int_{0}^{1} \varphi'(x) \,\mathrm{d}x = \varphi(0).
		\]
		Und dies ist ein Widerspruch. 
	\end{enumerate}
\end{beispiel}

\subsection{Definition und Eigenschaften von Sobolevräumen} 
\label{sub:definition_und_eigenschaften_von_sobolevraumen}
Wir wollen nun die Sobolevräume einführen. Diese sind Unterräume von Lebesgueräumen.
\begin{definition*}
	Sei $\Omega \subseteq \mathbb{R}^n$ und $ p \in [1, + \infty]$. Wir setzen
	\begin{align*}
		W^{1,p}(\Omega) :=& \set[u \in L^p(\Omega)]{ \diff{}{x_i}u \text{ existiert in }L^p(\Omega), \forall\, 1 \leq i \leq n} \\
		=& \set[u \in L^p(\Omega)]{ \nabla u \text{ existiert in }L^p(\Omega, \mathbb{R}^n)}
	\end{align*}
	Weiter setzen wir:
	\[
		\norm{u}_{W^{1,p}(\Omega)} := \begin{cases}
			\left( \norm{u}_{L^p}^p + \sum^{n}_{i=1}\norm{ \diff{}{x_i}u}_{L^p}^p \right)^{\frac{1}{p}}, &\text{ falls }1 \leq p < \infty\\
			\norm{u}_{L^{\infty}} + \sum^{n}_{i=1}\norm{\diff{}{x_i}u}_{L^{\infty}}, &\text{ falls }p = \infty
		\end{cases}
	\]
	$W^{1,p}(\Omega)$ heißt Sobolevraum.
\end{definition*}
\begin{satz}[Vollständigkeit]
	Sei $\Omega \subseteq \mathbb{R}^n$, $1 \leq p \leq  \infty$. Dann ist $(W^{1,p}(\Omega), \norm{.}_{W^{1,p}})$ ein Banachraum.
\end{satz}
\begin{beweis}
	Es ist klar, dass $(W^{1,p}(\Omega), \norm{.}_{W^{1,p}(\Omega)})$ ein normierter Vektorraum ist. Zu zeigen bleibt also die Vollständigkeit. Dazu sei $(u_k)_k$ eine 
	Cauchyfolge in $W^{1,p}(\Omega)$ bzgl. $\norm{.}_{W^{1,p}(\Omega)}$. Dann sind $(u_k)_k$ und $(\diff{}{x_i}u_k)_k$ für alle $i=1,\dots,n$ Cauchy-Folgen in $L^p(\Omega)$ bzgl. $\norm{.}_{L^p(\Omega)}$. Denn:
	\begin{align*}
		\norm{u_k-u_m}_{L^p(\Omega)} \leq  \norm{u_k-u_m}_{W^{1,p}(\Omega)} < \varepsilon
	\end{align*}
	für $k,m$ hinreichend groß. Außerdem gilt
	\[
		\norm{\diff{}{x_i}u_k- \diff{}{x_i}u_m}_{L^p(\Omega)} \leq \norm{u_k - u_m}_{W^{1,p}} < \varepsilon
	\]
	für hinreichend große $k$ und $m$. Mit Satz $6.1$ (Vollständigkeit von $L^p$) existiert $u \in L^p(\Omega)$ und $v_1, \dots , v_n$ in $L^p(\Omega)$ mit
	\[
		\norm{u_k-u}_{L^p(\Omega)} \to 0, \qquad \norm{\diff{}{x_i}u_k-v_i}_{L^p(\Omega)} \to 0 \qquad \forall\, i=1,\dots,n.
	\]
	Wir zeigen dazu
	\[
		\int_{\Omega}^{} u \diff{}{x_i}\varphi \,\mathrm{d}x = - \int_{\Omega}^{}v_i \varphi \,\mathrm{d}x \qquad \forall\, \varphi \in C^{\infty}_0(\Omega).
	\]
	Es gilt wegen
	\[
		u_k \to u \qquad \text{ in }L^p(\Omega).
	\]
	\begin{align*}
		\int_{\Omega}^{} u \diff{}{x_i}\varphi \,\mathrm{d}x &= \lim_{k \to \infty} \int_{\Omega}^{}u_k \diff{}{x_i}\varphi \,\mathrm{d}x \\
		&= \lim_{k \to \infty} - \int_{\Omega}^{} \diff{}{x_i}u_k \varphi \,\mathrm{d}x , \qquad \text{da }u_k \in W^{1,p}(\Omega) \\
		&= - \int_{\Omega}^{}v_i \varphi \,\mathrm{d}x, \qquad \text{da }\diff{}{x_i}u_k \to v_i \text{ in } L^p(\Omega)
	\end{align*}
	Daraus folgt dann
	\[
		u \in W^{1,p}(\Omega) \qquad \text{ und } \qquad \norm{u_k-u}_{W^{1,p}(\Omega)} \to 0.
	\]
\end{beweis}
\begin{bemerkung}
	\begin{enumerate}[(i)]
		\item Der Raum $W^{1,2}(\Omega)$ ist ein Hilbertraum, dessen Norm durch das folgende Skalarprodukt induziert wird:
		\begin{align*}
			\skal{u}{v}_{W^{1,2}(\Omega)} &= \int_{\Omega}^{}u v \,\mathrm{d}x + \sum^{n}_{i=1} \int_{\Omega}^{} \diff{}{x_i} u \diff{}{x_i}v \,\mathrm{d}x \\
			&= \int_{\Omega}^{} uv \,\mathrm{d}x+ \int_{\Omega}^{}  \nabla u \cdot  \nabla v \,\mathrm{d}x \qquad \forall\, u,v \in W^{1,2}(\Omega).
		\end{align*}
		Man findet häufig die Notation 
		\[
			H^1(\Omega) = W^{1,2}(\Omega).
		\]
		\item Wir haben die folgenden Inklusionen für ein beschränktes $\Omega$.
		\[
			C^1(\Omega) \subseteq W^{1,\infty}(\Omega) \subseteq  W^{1,p}(\Omega) \subseteq L^p(\Omega) \qquad \forall\, 1 \leq p <\infty.
		\]
	\end{enumerate}
\end{bemerkung}
Wir wollen zeigen, dass für $n=1$ alle $W^{1,p}(\Omega)$ Funktionen einen stetigen Repräsentanten haben. Das heißt, wenn $u \in W^{1,p}((a,b))$, dann existiert ein 
$ \tilde u \in C^0([a,b])$ und $\tilde u = u$ fast überall in $(a,b)$. Um das zu beweisen, benötigen wir das folgende Lemma, welches eine Folgerung aus dem Hauptsatz der Variationsrechnung ist.
\begin{lemma}
	Sei $u \in L^1_{\text{loc}}((a,b))$ und gelte
	\[
		\int_{a}^{b} u \varphi' \,\mathrm{d}x = 0, \qquad \forall\, \varphi \in C^{\infty}_0((a,b)),
	\]
	dann gilt $u(x)= c$ fast überall in $(a,b)$ für eine Konstante $c \in \mathbb{R}$.
	
\end{lemma}
\begin{beweis}
	Zeige zunächst: Ist $u \in L^1_{\text{loc}}((a,b))$ und 
	\[
		\int_{a}^{b} u \psi \,\mathrm{d}x = 0 \qquad \forall\, \psi \in C^{\infty}_0((a,b))  \qquad (**)
	\]
	mit
	\[
		\int_{a}^{b}\psi(x) \,\mathrm{d}x = 0,
	\]
	so ist $u$ konstant fast überall in $(a,b)$. Sei dazu $w \in C^{\infty}_0((a,b))$ und $f \in C^{\infty}_0((a,b))$ mit 
	\[
		\int_{a}^{b}f(x) \,\mathrm{d}x =1.
	\]
	Setze \[
		\psi(x) := w(x) - \int_{a}^{b} w(y) \,\mathrm{d}y f(x).
	\]
	Dann ist $\psi \in C^{\infty}_0((a,b))$ mit 
	\begin{align*}
		\int_{a}^{b} \psi(x) \,\mathrm{d}x &= \int_{a}^{b}w(x) \,\mathrm{d}x - \int_{a}^{b} \left( \int_{a}^{b} w(y) \,\mathrm{d}y  \right) f(x) \,\mathrm{d}x \\
		&= \int_{a}^{b}w(x) \,\mathrm{d}x - \int_{a}^{b} w(y) \,\mathrm{d}y \underset{=1}{\underbrace{\int_{a}^{b} f(x) \,\mathrm{d}x}} =0
	\end{align*}
	Nach Voraussetzung gilt dann:
	\begin{align*}
		0 &= \int_{a}^{b}u(x)\psi(x) \,\mathrm{d}x  \\ 
		&= \int_{a}^{b} u(x)w(x) \,\mathrm{d}x - \int_{a}^{b}\left( \int_{a}^{b}w(y) \,\mathrm{d}y \right) u(x)f(x) \,\mathrm{d}x \\
		&= \int_{a}^{b}u(x)w(x) \,\mathrm{d}x - \int_{a}^{b} \left( \int_{a}^{b}u(x)f(x) \,\mathrm{d}x \right) w(y) \,\mathrm{d}y 
	\end{align*}
	Somit
	\[
		\int_{a}^{b}\left( u(y)- \int_{a}^{b} u(x)f(x) \,\mathrm{d}x \right) w(y) \,\mathrm{d}y = 0 \qquad \forall\, w \in C^{\infty}_0((a,b))
	\]
	und es folgt weiter mit dem Hauptsatz der Variationsrechnung
	\[
		u = \underset{\text{Konstante}}{\underbrace{\int_{a}^{b}u(x)f(x) \,\mathrm{d}x}} \qquad \text{fast überall in }(a,b).
	\]
	Sei nun $u \in L^1_{\text{loc}}((a,b))$ wie im Lemma. Zeige $u$ erfüllt $(**)$. Sei dazu $ \psi \in C^{\infty}_0((a,b))$ mit $\int_{a}^{b}\psi(x) \,\mathrm{d}x =0$.
	Setze
	\[
		\varphi(x):= \int_{a}^{x}\psi(t) \,\mathrm{d}t.
	\]
	Dann ist $\varphi \in C^{\infty}_0((a,b))$ und 
	\[
		\varphi'(x) = \psi(x) - \underset{=0}{\underbrace{\psi(a)}} = \psi(x)
	\]
	Es folgt
	\[
		\int_{a}^{b} u(x) \psi(x) \,\mathrm{d}x = \int_{a}^{b} u(x)\psi'(x) \,\mathrm{d}x =0.
	\]
\end{beweis}

\begin{satz}[Stetiger Repräsentant]
	Sei $1 \leq p \leq \infty$, $u \in W^{1,p}((a,b))$, wobei $(a,b) \subseteq \mathbb{R}$. Dann existiert ein $\tilde u \in C^0([a,b])$ mit $u = \tilde u$ fast überall in $(a,b)$ und 
	\[
		\tilde u(x) - \tilde u(y) = \int_{x}^{y}u'(t) \,\mathrm{d}t \qquad \forall\, x,y \in [a,b]
	\]
\end{satz}
\begin{beweis}
	Sei $u \in W^{1,p}((a,b))$, $c \in (a,b)$. Setze
	\[
		v(x):= \int_{c}^{x}u'(t) \,\mathrm{d}t.
	\]
	Dann ist $v \in C^0([a,b])$ auf Grund der gleichmäßigen Stetigkeit des Lebesque-Integrals. Zeige
	\[
		\int_{a}^{b} v(x) \varphi(x) \,\mathrm{d}x = - \int_{a}^{b}u'(x) \varphi(x) \,\mathrm{d}x
	\]
	also $u'$ ist die schwache Ableitung von $v$. Es gilt
	\begin{align*}
		\int_{a}^{b}v(x) \varphi'(x) \,\mathrm{d}x &= \int_{a}^{b} \left( \int_{c}^{x} u'(t) \,\mathrm{d}t \right) \varphi'(x) \,\mathrm{d}x \\
		&= \int_{a}^{c} \left( \int_{c}^{x} u'(t) \,\mathrm{d}t \right) \varphi'(x) \,\mathrm{d}x + \int_{c}^{b} \left( \int_{c}^{x} u'(t) \,\mathrm{d}t \right) \varphi'(x) \,\mathrm{d}x
	\end{align*}
	Mit Fubini erhalten wir
	\begin{align*}
		\int_{a}^{c} \left( \int_{c}^{x}u'(t) \,\mathrm{d}t \right) \varphi'(x) \,\mathrm{d}x &= - \int_{a}^{c}\left( \int_{x}^{c}u'(t) \,\mathrm{d}t \right) \varphi'(x) \,\mathrm{d}x \\ 
		&= - \int_{a}^{c} \left( \int_{a}^{t} \varphi'(x) \,\mathrm{d}x \right)u'(t) \,\mathrm{d}t \\
		&= - \int_{a}^{c} \left( \varphi(t) - \underset{=0}{\underbrace{\varphi(a)}} \right) u'(t) \,\mathrm{d}t \\
		&= - \int_{a}^{c} \varphi(t) u'(t) \,\mathrm{d}t
	\end{align*}
\end{beweis}
\cleardoubleoddemptypage
\pagenumbering{Alph}
\setcounter{page}{1}

\end{document}