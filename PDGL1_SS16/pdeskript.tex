%!TEX TS-program = xelatex
%skript für Die Vorlesung Partielle Differentialgleichungen
\newcommand{\Semester}{SoSe 2016}
\newcommand{\fach}{Partielle Differentialgleichungen}
\newcommand{\prof}{Prof.\ Zeppieri}

\input{../!config/VorlagenTim/preambel.tex}

\numberwithin{equation}{section}

\begin{document}

\maketitle
\cleardoubleoddemptypage

\pagenumbering{Alph}
\section*{Vorwort --- Mitarbeit am Skript}
Dieses Dokument ist eine Mitschrift aus der Vorlesung \enquote{\fach, \Semester}, gelesen von \prof. Der Inhalt entspricht weitestgehend dem Tafelanschrieb. Für die
Korrektheit des Inhalts werde ich keinerlei Garantie übernehmen. Dieses Skript wird allerdings zusätzlich von \prof \, Korrektur gelesen. Für Bemerkungen und Korrekturen -- und seien es nur Rechtschreibfehler -- bin ich sehr dankbar. Bitte durch persönliches Ansprechen oder per Mail an \href{mailto:keil.menden@web.de}{keil.menden@web.de}.

\newpage

\tableofcontents
\cleardoubleoddemptypage
\pagenumbering{arabic}
\setcounter{page}{1}

\section{Einleitung} 
\label{sec:einleitung}
\begin{definition}[PDGL]
	Sei $\Omega$ eine offene Teilmenge des $\mathbb{R}^n$ mit $n \geq 2$ und $ k \in \mathbb{N}$. Eine partielle Differentialgleichung (PDGL) ist eine Gleichung der Form
	\begin{equation}\label{(1)}
		E(x,u(x),Du(x),\dots,D^ku(x))=0 \qquad \text{für alle }x \in \Omega.
	\end{equation}
	für eine unbekannte Funktion $u$ (und ihre Ableitungen $Du,\dots,D^ku$ bis zur Ordnung $k$), wobei $E: \Omega \times \mathbb{R} \times \mathbb{R}^n \times \dots \times \mathbb{R}^{n^k} \to \mathbb{R}$ ein gegebene Funktion ist. Die höchste Ableitungsordnung, die in \eqref{(1)} auftritt, nennt man die Ordnung der PDGL.
\end{definition}

\begin{notation}
\begin{itemize}
	\item $D = (\partial_1, \dots , \partial_n)$, $\partial_i = \partial_{x_i} = \diff{}{x_i}$ für $i=1,\dots,n$
	\item $D^k:= \set[D^{\alpha }]{\abs{\alpha }=k}$
	\item $\alpha \in \mathbb{N}_0^n$ ist ein Multiindex mit $\alpha = (\alpha_1, \dots, \alpha_n)$ und
	\[
		\abs{\alpha} = \sum^{n}_{i=1}\alpha_i \qquad \qquad D^{\alpha}= \prod\limits_{i=1}^{n}\partial_i^{\alpha_i} = \partial_1^{\alpha_1} \cdots \partial_n^{\alpha_n}
	\]
\end{itemize}
\end{notation}

\begin{bemerkung}
Für $n=1$ wird \eqref{(1)} auf eine gewöhnliche Differentialgleichung (ODE) reduziert: $u: \Omega \subseteq \mathbb{R} \to \mathbb{R}$ mit
\[
	E(x,u(x),u'(x),\dots,u^{(k)}(x))=0 \qquad \text{in } \Omega 
\]
\end{bemerkung}

\begin{definition}[System von PDE's]
	Ein System von PDE's ist eine Gleichung der Form \eqref{(1)}, wenn du unbekannte Funktion $u$ und die gegebene Funktion $E$ vektorwertig sind, d.h. $u: \Omega \to \mathbb{R}^m$ für ein $m \geq 1$, $u = (u_1, \dots, u_m)$ und
	\[
		E: \Omega \times \mathbb{R}^m \times \dots \times \mathbb{R}^{m \times n^k} \to \mathbb{R}^m
	\]
\end{definition}
\begin{definition}[Klassifikation für PDE's]
		Die PDE \eqref{(1)} heißt:
		\begin{enumerate}[(i)]
			\item LINEAR, falls sie in $u$ und allen ihren Ableitungen linear ist und die dazugehörigen Koeffizienten nur von $x$ abhängen, d.h., falls man sie in der Form
			\begin{equation}
				\sum_{\abs{\alpha}\leq k}^{}a_{\alpha}(x) D^{\alpha}u(x) - f(x) = 0 \qquad \text{in } \Omega \label{(2)}
			\end{equation} 
			schreiben kann, wobei $a_{\alpha}$ ($\abs{\alpha}\leq k$) und $f$ gegebene Funktionen sind.
			Im Fall $f=0$ nennt man \eqref{(2)} homogen, sonst inhomogen.
			\item SEMILINEAR, falls sie in der höchsten Ableitung $D^ku$ linear ist und die dazugehörigen Koeffizienten nur von $x$ abhängen, d.h. falls man sie in der Form
			\begin{equation}
				\sum_{\abs{\alpha}=k}^{}a_{\alpha}(x)D^{\alpha}u(x) + E_0(x,u(x),Du(x),\dots,D^{k-1}u(x))=0 \qquad \text{in }\Omega
			\end{equation}
			schreiben kann, wobei $a_{\alpha}(x)$ ($\abs{\alpha}=k$) und $E_0$ gegebene Funktionen sind.
			\item QUASILINEAR, falls sie linear in der höchsten Ableitung $D^ku$ ist und die dazugehörigen Koeffizienten nur von $x$, $u$, und $D^lu$ mit $\abs{l}\leq k-1$ abhängen, 
			d.h. falls man sie in der Form
			\begin{small}
			\begin{equation}
				\sum_{\abs{\alpha}=k}^{}a_{\alpha}(x,u(x),Du(x),\dots,D^{k-1}u(x))D^{\alpha}u(x) + E_0(x,u(x),Du(x),\dots,D^{k-1}u(x)) = 0 
			\end{equation}
			\end{small}
			schreiben kann, wobei $a_{\alpha}$ und $E_0$ gegebene Funktionen sind.
			\item VOLL NICHTLINEAR in allen anderen Fällen
			
		\end{enumerate}
\end{definition}

\begin{bemerkung}
	\[
		\set{\text{lineare PDE}} \subseteq \set{\text{semilineare PDE}} \subseteq \set{\text{quasilineare PDE}} \subseteq \set{\text{PDE}}
	\]
\end{bemerkung}

\begin{definition}[Typeinteilung für PDEs zweiter Ordnung]
		Sei $\Omega$ eine offene Teilmenge des $\mathbb{R}^n$ und $E: \Omega \times \mathbb{R} \times \mathbb{R}^n \times \mathbb{R}^{n \times n} \to \mathbb{R}$ eine gegebene Funktion, so dass $p \mapsto E(x,u,z,p)$ für alle $(x,u,z) \in \Omega \times \mathbb{R}\times \mathbb{R}^n$ von der Klasse $C^1(\mathbb{R}^{n \times n})$ ist. Die PDE
		\begin{equation}
			E(x,u(x),Du(x),D^2u(x))=0 \qquad \text{in }\Omega 
		\end{equation}
		heißt
		\begin{enumerate}[(i)]
			\item ELLIPTISCH, falls die $(n \times n)$-Matrix 
			\begin{equation}
				\left( \diff{E}{p_{ij}}(x,u,z,p) \right)_{i,j=1,\dots,n} \label{(3)}
			\end{equation}
			für alle $(x,u,z,p) \in \Omega \times \mathbb{R} \times \mathbb{R}^n \times \mathbb{R}^{n \times n}$ positiv definit ist.
			\item HYPERBOLISCH, falls die Matrix in \eqref{(3)} genau einen negativen und $(n-1)$ positive Eigenwerte besitzt. (auch anders herum möglich)
			\item PARABOLISCH, falls man sie in der Form
			\begin{equation}
				D_1u(x)= \overline{E}(x,u(x),D'u(x),(D')^2u(x)) \qquad \text{in } \Omega
			\end{equation}
			%%%% unvollständig
			schreiben kann, wobei $D'=(\partial_1,\dots,\partial_n)$ und $\overline{E}: \Omega \times \mathbb{R} \times \mathbb{R}^{n-1} \times \mathbb{R}^{(n-1) \times (n-1)} \to \mathbb{R}$
			von der Klasse $C^1$ ist und 
			\begin{equation}
				\left( \diff{\overline{E}}{p_{ij}}(x,u,z',p') \right)_{i,j=1,\dots,n} 
			\end{equation}
		\end{enumerate}
\end{definition}

\begin{beispiel}[Typeinteilung für lineare PDEs zweiter Ordnung]
Sei $\Omega$ eine offene Teilmenge des $\mathbb{R}^n$ und seien $a_{ij}$, $b_i$ (für $i,j=1,\dots,n$), $c$ und $f$ skalarwertige Funktionen auf $\Omega$
\begin{enumerate}
	\item Die lineare PDE der Form
	\[
		\sum^{n}_{i,j=1} a_{ij}(x) D_{ij}u(x) + \sum^{n}_{i=1}b_i(x) D_iu(x) + c(x)u(x) - f(x) = 0 \qquad \text{in } \Omega
	\]
	ist elliptisch, falls die Matrix $\left(a_{ij}(x)\right)_{i,j=1,\dots,n}$ für alle $x \in \Omega$ positiv definit ist.
	\item Die lineare PDE der Form 
	\[
		D_11u(x) - \sum^{n}_{i,j=1} a_{ij}(x)D_{ij}u(x) + \sum^{n}_{i=1}b_i(x) D_iu(x) + c(x)u(x) - f(x) = 0 \qquad \text{in }\Omega
	\]
	ist hyperbolisch falls die Matrix $\left( a_{ij}(x) \right)_{i,j=2,\dots,n}$ für alle $x \in \Omega$ positiv definit ist.
	\item Die lineare PDE der Form 
	\begin{equation}
		D_1u(x) - \sum^{n}_{i,j=2}a_{ij}(x)D_{ij}u(x) + \sum^{n}_{i=2}b_i(x)D_i(x) + c(x)u(x)- f(x)= 0 \qquad \text{in } \Omega
	\end{equation}
	ist parabolisch, falls die Matrix $\left( a_{ij}(x) \right)_{i,j=2,\dots,n}$ für alle $x \in \Omega$ positiv definit ist.
 \end{enumerate}
\end{beispiel}

\begin{definition}[klassische Lösung]
	Eine Funktion $u: \Omega \to  \mathbb{R}$ heißt klassische Lösung der PDE der Form 
	\begin{equation}
		E(x,u(x),Du(x),\dots,D^ku(x))=0 \qquad  \text{in } \Omega \label{(4)}
	\end{equation}
	falls $u \in  C^k(\Omega)$ gilt und die Gleichung \eqref{(4)} überall in $\Omega$ erfüllt ist.
\end{definition}

\begin{beispiel}
	$D_1D_2u(x)=0$ in $\mathbb{R}^2$, $u(x_1,x_2)=v_1(x_1)+v_2(x_2)$ für beliebige Funktionen $v_1,v_2 \in C^2(\mathbb{R})$.
\end{beispiel}  

\begin{nb} 
\[
	B(x,u(x),Du(x),\dots,Du^{k-1}(x))=0 \qquad  \text{auf }\partial \Omega,
\] wobei $B: \partial \Omega \times \mathbb{R} \times \mathbb{R}^n \times \dots \times \mathbb{R}^{n^{k-1}} \to \mathbb{R}$ eine vorgegebene Funktion ist.
\end{nb}

\begin{bemerkung}
	Ein Problem ( = PDE + Nebenbedinung) heißt wohlgestellt (well-posed) im Sinne von Hadamerd, wenn zu gegebenen Daten eine eindeutige Lösung existiert und sie stetig von den Daten abhängt.
	Wir stellen uns in der Vorlesung folgende Fragen:
	\begin{itemize}
		\item Existenz und Eindeutigkeit von Lösungen
		\item Hängen die Lösungen stetig von den Daten ab?
		\item Wie können Lösungen dargestellt werden?
		\item Wie ist das qualitative Verhalten von Lösungen?
	\end{itemize}
\end{bemerkung}

\begin{beispiele}
	\begin{enumerate}[1.]
		\item Laplacegleichung (linear, elliptisch):
		\begin{equation}
			\Delta u = 0 \qquad \text{in }\Omega
		\end{equation}
		modelliert das elektrische Feld im Vakuum. Inhomogen ist dies die Poissongleichung und es gilt $\Delta u = \diver(Du)$
		\item Monge-Ampère-Gleichung (voll nichtlinear):
		\begin{equation}
			\det(D^2u)=f(x,u,Du) \qquad \text{in }\Omega
		\end{equation}
		für Transportprobleme und Differentialgeometrie.
		\item Wärmeleitungsgleichung (linear, parabolisch):
		\begin{equation}
			\partial_t u - \Delta u = 0 \qquad \text{in } \mathbb{R}^+ \times \Omega,
		\end{equation}
		wobei $\Omega \subseteq  \mathbb{R}^n$, $t$= Zeitkoordinate, $x$= Raumkoordinate modelliert die Verteilung von Wärme. $u(t,x)$ die Temperatur im Punkt $x \in \Omega$ zum Zeitpunkt
		$t \in \mathbb{R}^+$.
		\item Reaktions-Diffusions-Gleichung (semilinear, parabolisch):
		\begin{equation}
			\partial_t u - \Delta u = f(x,t,u) \qquad \text{in } \mathbb{R}^+ \times \Omega
		\end{equation}
		\item System der Navier-Stokes-Gleichungen (seminlinear, parabolisch):
		\begin{equation}
			\partial_t u - \Delta u + u \cdot Du = -Dp \qquad \text{in } \mathbb{R}^+ \times \Omega,
		\end{equation}
		wobei $u: \mathbb{R}^+ \times \Omega \to  \mathbb{R}^n$, $p: \mathbb{R}^+ \times \Omega \to \mathbb{R}$ und $\Omega \subseteq  \mathbb{R}^n$. Modelliert die Strömung von inkompressiblen Flüssigkeiten, $u$ = Geschwindigkeit, $p$ = Druck
		\item Transportgleichung (linear):
		\begin{equation}
			\partial_t u + b(t,x) \cdot Du = 0 \qquad \text{in } \mathbb{R}^+ \times \Omega 
		\end{equation}
		\item Wellengleichung (linear, hyperbolisch):
		\begin{equation}
			\partial_{tt}u - \Delta u = 0 \qquad \text{in } \mathbb{R}^+ \times \Omega,
		\end{equation}
		Für $n=2$ modelliert sie die Schwingung einer elastischen Membran. Für $n=3$ modelliert sie die Ausbreitung von Wellen (Licht und Wasser).
	\end{enumerate}
\end{beispiele}

\section{Laplacegleichung} 
\label{sec:laplacegleichung}

\begin{equation}
	\Delta u = 0 \qquad \text{in }\Omega
\end{equation}
wobei $\Omega \subseteq \mathbb{R}^n$, offen, $n \geq 2$ und $\Delta u = \sum^{n}_{i=1} \diff{^2u}{x_i^2} = \sum^{n}_{i=1} a_{ij} \diff{^2u}{x_i \partial x_j}$ wobei $a_{ij}= I_n \in \mathbb{R}^{n \times n}$

\begin{definition}[Harmonische Funktion]
	Sei $\Omega$ eine offene Teilmenge des $\mathbb{R}^n$ mit $n \geq 2$ und $u \in C^2(\Omega)$. Man nennt $u$ \underline{harmonisch}, falls $\Delta u = 0$ in $\Omega$ gilt. Falls lediglich $\Delta u \geq 0$ in $\Omega$ gilt, nennt man $u$ subharmonisch und falls $\Delta u \leq 0$ in $\Omega$ gilt, superharmonisch.
\end{definition}
	
\begin{beispiele}
	\begin{enumerate}[1.]
		\item Jede affine Funktion ($u(x) = b \cdot x + x, b \in \mathbb{R}^n , c \in \mathbb{R}$) ist harmnoisch in $\mathbb{R}^n$.
		\item Sei $A \in \mathbb{R}^{n \times n}$. Die Matri $u(x) = Ax \cdot x$ ist genau dann harmonisch / subharmonich / superharmonisch, falls $ \Tr(A)= 0$ / $ \Tr(A) \geq  0 $ / $ \Tr(A) \geq  0 $ gilt. (weil in diesem Fall $\Delta u = \Tr(A)$).
		\item Die Funktionen $u(x_1,x_2)= e^{ax_1}\sin(ax_2)$, $u(x_1,x_2)= e^{ax_1} \cos(ax_2)$ für alle $a \in \mathbb{R}$ sind harmonisch in $\mathbb{R}^2$.
		\begin{align}
			\diff{^2u}{x_1^2} = a^2 e^{ax_1} \sin(ax_2) \qquad \text{,} \qquad \diff{u}{x_2} &= a e^{ax_1} \cos(ax_2) 
			\qquad \text{,} \qquad \diff{^2u}{x_2^2} = - a^2 e^{ax_1} \sin(ax_2)  \\
			\Rightarrow \Delta u &= 0 \qquad \text{in } \mathbb{R}^2
		\end{align}
		\item Real- und Imaginärteil holomorpher Funktionen sind harmonisch. Sei dazu $\Omega \in \mathbb{C}$ offen und $f: \Omega \to \mathbb{C}$ eine holomorphe Funktion.
		\begin{equation}
			f ( x_1 + i x_2) = u(x_1,x_2) + i v(x_1,x_2) 
		\end{equation}
		wobei $u,v : \Omega \to  \mathbb{R}$ gilt. Da $f$ holomorph ist, erfüllen $u$ und $v$ die Cauchy-Riemann Differentialgleichungen:
		\begin{align}
			(*)
			\begin{cases}
				\diff{u}{x_1}= \diff{v}{x_2} \\
				\diff{u}{x_2}= -\diff{v}{x_1}	\label{(stern)}
			\end{cases}
		\end{align}
		und sind beliebig oft differenzierbar in $\Omega$
		\begin{equation}
			\Delta u = \diff{^2u}{x_1^2} + \diff{^2u}{x_2^2} \stackrel{(*)}{=} \diff{}{x_1} \left( \diff{v}{x_2} \right) + \diff{}{x_2} \left( - \diff{v}{x_1} \right) = 0
		\end{equation}
		und ebenso
		\[
			\Delta v = 0 \qquad \text{in } \Omega
		\]
	\end{enumerate}
\end{beispiele}

\subsection{Die Fundamentallösung der Laplacegleichung} 
\label{sub:die_fundamentallosung_der_laplacegleichung}
Wir betrachten die Laplacegleichung in $\mathbb{R}^n$ mit ($n \geq 2$) und suchen eine Rotationssymmetrische Lösung, d.h. $u(x) = v(r)$, wobei $r := \abs{x}= \sqrt{x_1^2 + \dots x_n^2}$ und $v : \mathbb{R}^+ \to \mathbb{R}$. Es gilt
\begin{equation}
	\diff{r}{x_i} = \frac{2x_i}{2\sqrt{x_1^2 + \dots x_n^2} } = \frac{x_i}{r}
\end{equation}
für alle $i=1,\dots,n$ und für $x \neq 0$
\begin{align}
	\diff{u}{x_i} = \diff{v(r)}{x_i} &= v'(r) \diff{r}{x_i} = v'(r) \frac{x_i}{r}, \\
	\diff{^2u}{x_i^2} = \diff{}{x_i} \left( v'(r) \frac{x_i}{r} \right) = v''(r) \frac{x_i^2}{r^2} + &v'(r) \left( r - \frac{x_i^2}{r} \right) \frac{1}{r^2} = v''(r) \frac{x_i^2}{r^2} + v'(r) \left( \frac{1}{r} - \frac{x_i^2}{r^2} \right)
\end{align}
Dann gilt

\begin{align*}
	\Delta u = \sum^{n}_{i=1} \diff{^2u}{x_i^2} &= \sum^{n}_{i=1} \left( v''(r) \frac{x_i^2}{r^2} + v'(r) \left( \frac{1}{r} - \frac{x_i^2}{r^2} \right) \right) \\
	&= v''(r) + v'(r) \left( \frac{n-1}{r} \right)
\end{align*}

Aus $\Delta u = 0$ in $\mathbb{R}^n$ erhalten wir eine ODE:
\begin{equation}
	v''(r) + v'(r)\left( \frac{n-1}{r} \right) = 0  \label{(5)}
\end{equation}
Wenn $v' \neq 0$, kann man \eqref{(5)} für $v'$ lösen
\begin{align}
	\frac{v''(r)}{v'(r)} &= \frac{1-n}{r} \\
	\diffd{}{r}\left( \lg(\abs{v'(r)}) \right) &= \diffd{}{r} \left( (1-n) \lg(r) \right)
\end{align}
Integration liefert 
\begin{equation}
	\lg \abs{v'(r)} = \lg( r ^{1-n}) + \alpha
\end{equation}
mit Integrationskonstante $\alpha$.
\begin{align*}
	&\hphantom{\Rightarrow} e^{\lg \abs{v'(r)}} = e^{\lg(r^{1-n})+2} \\
	&\Rightarrow \abs{v'(r)} = e^{\alpha} r^{1-n} \\
	&\Rightarrow v'(r) = \beta r^{1-n}
\end{align*}
für eine beliebige Konstante $\beta \in \mathbb{R}$. Es gilt
\begin{equation}
	v'(r) = \begin{cases}
		\frac{\beta}{r}, &\text{ falls } n = 2\\
		\beta r^{1-n}, , &\text{ falls } n \geq 3		
	\end{cases}
\end{equation}
und daraus folgt
\begin{equation}
	v'(r) = \begin{cases}
		\beta \lg(r)+ \gamma, &\text{ falls } n = 2\\
		\beta r^{2-n} + \gamma, , &\text{ falls } n \geq 3		
	\end{cases}
\end{equation}
wobei $\beta,\gamma \in \mathbb{R}$. \\
Wir setzen $ \gamma = 0$ (für $n \geq 3$ gilt $v(r) \to 0$ für $r \to +\infty$). Um $\beta$ zu bestimmen, verwendet man 
\begin{equation}
	-1 = \int_{\partial B_k(0)}^{} \Delta u \cdot \nu \,\mathrm{d}S \label{(6)}
\end{equation}
$\nu$ ist der äußere Einheitsnormalenvektor an $\partial B_r(0)$ und das Integral ist ein Oberflächenintegral. ( $ \Delta = D$). \\
\[
	\nu = \frac{x}{\abs{x}} \qquad \text{,} \qquad \Delta u = \nu'(\abs{x}) \frac{x}{\abs{x}}
\]
Aus \eqref{(6)} folgt 
\begin{equation}
	-1 = \int\limits_{\substack{\partial B_r(0)\\ \abs{x}=r}}^{} v'(\abs{x}) \underset{=1}{\underbrace{\frac{x}{\abs{x}} \cdot \frac{x}{\abs{x}}}} \,\mathrm{d}S  
	= v'(r) \underset{= r^{n-1}S_n}{\underbrace{\int_{\partial B_r(0)}^{} \,\mathrm{d}S }}
\end{equation}
Wobei $S_n$ das $(n-1)$- dimensionale Volumen der Einheitsphäre bezeichnet. [siehe Übung]

\begin{equation}
	-1 = v'(r) r^{n-1} S_n = \begin{cases}
		\frac{\beta}{r}r S_2 = \beta S_2 = \beta 2 \pi, &\text{ falls } n=2\\
		\beta (2-n) \frac{r^{1-n}r^{n-1} }{  S_n}, &\text{ falls } n \geq 3
	\end{cases}
\end{equation}
Für $n=2$ folgt
\[
	\beta = - \frac{1}{2 \pi}
\] 
und sonst folgt
\[
	\beta = -\frac{1}{(2-n)S_n} = \frac{1}{(n-2)n\omega_n}
\]
Hierbei ist $\omega_n$ das $n$-dimensionale Volumen der Einheitskugel $B_1(0)$ [Übung]

\begin{align*}
	v(r) = \begin{cases}
		-\frac{1}{2 \pi} \lg(r), &\text{ falls }n=2\\
		\frac{1}{n (n-2) \omega_n } r^{2-n}, &\text{ falls }n \geq 3
	\end{cases}
\end{align*}

\begin{definition}[Fundamentallösung der Laplacegleichung]
	Die Funktion $\Phi: \mathbb{R}^n \setminus \set{0} \to \mathbb{R}$ definiert als
	\begin{equation}
		\Phi(x) := \begin{cases}
			-\frac{1}{2 \pi} \lg \abs{x}, &\text{ falls }n=2\\
			\frac{1}{n (n-2) \omega_n } \abs{x}^{2-n}, &\text{ falls }n \geq 3
		\end{cases}
	\end{equation}
	heißt die Fundamentallösung der Laplacegleichung.
\end{definition}

\begin{bemerkung}
	$\Phi$ hat bei $x=0$ eine Singularität aber $\Phi , \abs{ \nabla \Phi} \in L^1_{\text{loc}}(\mathbb{R}^n)$ und $ \abs{ \nabla^2 \Phi} \not\in L^1_{\text{loc}}(\mathbb{R}^n) $. Tatsächlich ist im Fall $n=2$ für $\abs{x} < 1$
	\[
		\abs{ \Phi (x)} = \abs{ - \frac{1}{2 \pi} lg(x)} = - \frac{1}{2 \pi} lg \abs{x} = \frac{1}{2 \pi} lg \frac{1}{\abs{x}} \leq  \frac{C}{\abs{x}} \in L^1(B_1(0)) 
	\]
	weil $ \frac{1}{\abs{x}^2} \in L^1(B_1(0)) \Leftrightarrow a < n$ . \\
	Für $n \geq 3$ gilt
	\begin{align*}
		\abs{ \Phi} &\leq \frac{C}{\abs{x}^{n-2}} \in L^1(B_1(0)) \\
		\abs{  \nabla  \Phi} &\leq \frac{C}{\abs{x}^{n-1}} \in  L^1 ( B_1(0)) 
	\end{align*}
	dagegen 
	\begin{equation}
		\abs{  \nabla^2 \Phi } \simeq \frac{1}{\abs{x}^n} \not\in L^1(B_1(0))
	\end{equation}
\end{bemerkung}

\subsection{Mittelwerteigenschaft} 
\label{sub:mittelwerteigenschaft}
\begin{equation}
	u(x_0) = \fint_{\partial B_r(x_0)}^{} u \,\mathrm{d}S \qquad  \text{,} \qquad u(x_0) = \fint_{B_r(x_0)}^{} u \,\mathrm{d}x
\end{equation}
für $u$ harmonisch, wobei 
\begin{align*}
	\fint_{\partial B_r(x_0)}^{} u \,\mathrm{d}S &= \frac{1}{r^{n-1}S_n} \int_{\partial B_r(x_0)}^{}  u\,\mathrm{d}S \\
	\fint_{B_r(x_0)}^{} u \,\mathrm{d}x &= \frac{1}{r^n \omega_n} \int_{B_r(x_0)}^{}u \,\mathrm{d}x
\end{align*}

\begin{definition*}[$C^1$-Rand]
	Sei $\Omega \in \mathbb{R}^n$ offen. Wir sagen, dass $\Omega$ einen $C^1$-Rand hat, falls gilt: \\
	Für alle $p \in \partial \Omega$ gibt es eine offene Menge $U \subseteq \mathbb{R}^n$ und eine Funktion $f \in C^1(U)$ so dass $ \nabla f(x) \neq 0$ für alle $x \in U$ gilt und
	\begin{align*}
		\partial \Omega \cap U &= \set[x \in U]{f(x)=0} \\
		\Omega \cap U &= \set[x \in U]{f(x) < 0}
	\end{align*}
\end{definition*}

\begin{bemerkung}
	$\partial \Omega$ ist eine $(n-1)$-dimensionale $C^1$-Mannigfaltigkeit und $\partial \Omega$ liegt lokal nur auf einer Seite von $\Omega$.
\end{bemerkung}

\begin{lemma}
	Sei $\Omega$ eine offene Teilmenge in $\mathbb{R}^n$, $B_R(x_0) \subseteq \Omega$ und $u \in C^2(\Omega)$. Definiert man $\varphi : (0,R) \to \mathbb{R}$ mittels
	\[
		\varphi(r) = \fint_{\partial B_r(x_0)}^{} u \,\mathrm{d}S \qquad \text{für } r \in (0,R)
	\]
	so gelten
	\begin{enumerate}[(i)]
		\item $\varphi(r) \to  u (x_0)$ für $ r \to 0^+$
		\item $\varphi'(r) = \frac{r}{n} \fint_{B_r(x_0)}^{} \Delta u \,\mathrm{d}x$ 
	\end{enumerate}
	\end{lemma}
	\begin{beweis}
		\begin{enumerate}[(i)]
			\item \begin{align*}
				\abs{ \varphi(r)-u(x_0)} &= \abs{ \fint_{\partial B_r(x_0)}^{}u \,\mathrm{d}S - u(x_0)} \\
				&= \abs{ \fint_{\partial B_r(x_0)}^{} (u(x)-u(x_0)) \,\mathrm{d}S} \\
				& \leq \fint_{\partial B_r(x_0)}^{} \abs{u(x)-u(x_0)} \,\mathrm{d}S \\
				& \leq  \underset{=1}{\underbrace{\abs{\fint_{\partial B_r(x_0)}^{} \,\mathrm{d}S}}} \max \abs{u(x)-u(x_0)} \stackrel{r \to 0^+}{\to} 0
			\end{align*}
		\item Es gilt
		\begin{equation}
			\varphi(r) = \frac{1}{r^{n-1}S_n} \int_{\partial B_r(x_0)}^{}u(x) \,\mathrm{d}S(x) \stackrel{x=x_0 + r y}{=} \frac{1}{S_n} \int_{\partial B_r(x_0)}^{} u(x_0 + ry) \,\mathrm{d}S(y)
		\end{equation}
		$u(x_0 + ry)$ ist differenzierbar nach $r$ für alle $y \in \partial B_1(0)$. \\
		$\diffd{}{r}u(x_0 + ry) =  \nabla u(x_0+ry) \cdot y$ ist integrierbar auf $ \partial B_1(0)$.
		\begin{align*}
			\varphi'(r) &\stackrel{\hphantom{x=x_0+ry}}{=} \diffd{}{r} \left( \frac{1}{S_n} \int_{\partial B_1(0)}^{} u(x_0 + ry) \,\mathrm{d}S(y) \right) \\
			& \stackrel{\hphantom{x=x_0+ry}}{=} \frac{1}{S_n} \int_{\partial B_1(0)}^{} \diffd{}{r} u(x_0 + ry) \,\mathrm{d}S(y) \\
			& \stackrel{\hphantom{x=x_0+ry}}{=} \frac{1}{S_n} \int_{\partial B_1(0)}^{} \nabla u(x_0 + r y) \cdot y \,\mathrm{d}S(y) \\
			& \stackrel{x=x_0+ry}{=} \frac{1}{S_n r^{n-1}} \int_{\partial B_r(x_0)}^{}  \nabla u(x) \cdot \frac{x-x_0}{r} \,\mathrm{d}S(x) \\
			&\stackrel{\hphantom{x=x_0+ry}}{=} \frac{1}{S_n r^{n-1}} \int_{\partial B_k(x_0)}^{} \div (  \nabla u(x)) \,\mathrm{d}x
		\end{align*}
		Hierbei ist $\frac{x-x_0}{r}$ die äußere Einheitsnormalenableitung an $\partial B_k(x_0)$ im Punkt x. Wegen $u \in C^2(\Omega)$ gilt $ \nabla u \in C^1( \overline{B_k(x_0)})$ wobei $\overline{B_k(x_0)} \subseteq \Omega$. \\
		Es gilt außerdem $\div(  \nabla u) = \Delta u$ und somit
		\begin{equation}
			\varphi'(r) \stackrel{\text{Gauß}}{=} \frac{r}{S_n r^n} \int_{B_k(x_0)}^{} \Delta u \,\mathrm{d}x 
			\stackrel{\omega_n = S_n n}{=} \frac{r}{n} \fint_{B_k(x_0)}^{} \Delta u \,\mathrm{d}x
		\end{equation}
		\end{enumerate}
		
	\end{beweis}
\begin{korollar}[Mittelwerteigenschaft]
	Sei $\Omega$ eine offene Teilmenge in $\mathbb{R}^n$, $\overline{B_r(x_0)} \subseteq \Omega$ ($B_r(x_0) \subset \subset \Omega$), $u \in C^2(\Omega)$.
	\begin{enumerate}[(i)]
		\item Falls $ \Delta u = 0 $ in $\Omega$ gilt, so folgt
		\begin{equation}
			u(x_0) = \fint_{\partial B_r(x_0)}^{} u \,\mathrm{d}S = \fint_{B_r(x_0)}^{} u 	\,\mathrm{d}x
		\end{equation}
		\item Falls $\Delta \geq 0$ in $\Omega$ gilt, so folgt
		\begin{equation}
			u(x_0) \geq \fint_{\partial B_r(x_0)}^{} u \,\mathrm{d}S \qquad \text{,} \qquad u(x_0) \leq \fint_{B_r(x_0)}^{}u \,\mathrm{d}x
		\end{equation}
		\item Falls $\Delta u > 0$ in $\Omega$ gilt, so folgt
		\begin{equation}
			u(x_0) < \fint_{\partial B_r(x_0)}^{} u\,\mathrm{d}S \qquad \text{,} \qquad u(x_0) < \fint_{B_r(x_0)}^{} u \,\mathrm{d}x
		\end{equation}
	\end{enumerate}
	\end{korollar}
	\begin{beweis}
		\begin{enumerate}[(i)]
			\item folgt aus $(ii)$ angewandt auf $u$ und $-u$
			\item Aus $\Delta u \geq 0$ folgt, dass $\varphi$ monoton wachsend ist.
			\begin{equation}
				u(x_0) = \lim_{S \to 0^+} \varphi(s) \leq \varphi(\rho) = \fint_{\partial B_{\rho}(x_0)}^{}u \,\mathrm{d}S \qquad \forall\, \rho \in (0,r]
			\end{equation}
			Insbesondere folgt direkt
			\[
				u(x_0) \geq \fint_{\partial B_r(x_0)}^{}u \,\mathrm{d}S
			\]
			Die zweite Ungleichung in $(ii)$ folgt durch Integration.
			\begin{align*}
				u(x_0)\abs{B_r(x_0)} &= u(x_0) \omega_n r^n \\ &= u(x_0) \omega_n \int_{0}^{r}n \rho^{n-1} \,\mathrm{d}\rho \\ 
				&= n\omega_n \int_{0}^{r}u(x_0)\rho^{n-1} \,\mathrm{d}\rho  \\
				&\leq n \omega_n \int_{0}^{r}\fint_{\partial B_{\rho}(x_0)}^{} u \,\mathrm{d}S \rho^{n-1} \,\mathrm{d}\rho \\
				&= n \omega_n \int_{0}^{r} \frac{1}{S_n} \int_{B_{\rho}(x_0)}^{} u \,\mathrm{d}S \,\mathrm{d}\rho \\
				&= \int_{0}^{r} \int_{B_{\rho}(x_0)}^{}u \,\mathrm{d}S \,\mathrm{d}\rho \\
				&= \int_{B_{r}(x_0)}^{}u \,\mathrm{d}x
			\end{align*}
			wobei der letzte Schritt die Anwendung von Polarkoordinaten beinhaltet. $(iii)$ folgt analog. In diesem Fall ist $\varphi$ streng monoton wachsend.
			\end{enumerate}
	\end{beweis}

\begin{satz}
	Sei $\Omega$ eine offene Teilmenge in $\mathbb{R}^n$ und $u \in C^2(\Omega)$. Die folgenden Eigenschaften sind äquivalent:
	\begin{enumerate}[(i)]
		\item $u$ ist harmonisch, d.h. es gilt 
		\begin{equation}
			\Delta u = 0 \qquad \text{in } \Omega
		\end{equation}
		\item $u$ erfüllt die sphärische Mittelwerteigenschaft, d.h. es gilt
		\begin{equation}
			u(x_0) \fint_{\partial B_r(x_0)}^{}u \,\mathrm{d}S
		\end{equation}
		für alle $x_0 \in \Omega$ und $r>0$ mit $B_k(x_0) \subset \subset \Omega$
		\item $u$ erfüllt die Mittelwerteigenschaft auf Kugeln, d.h. es gilt
		\begin{equation}
			u(x_0) = \fint_{B_r(x_0)}^{}u \,\mathrm{d}x
		\end{equation}
		für alle $x_0 \in \Omega$ und $r > 0$ mit $B_r(x_0) \subset \subset \Omega$.
		\item $u$ erfüllt die Mittelwerteigenschaft auf \underline{kleinen} Kugeln, d.h. für jedes $x_0 \in \Omega$ existiert eine positive Zahl von $R(x_0) < \dist(x_0,\partial \Omega)$, so dass für alle $r < R(x_0)$ gilt
		\begin{equation}
			u(x_0) = \fint_{B_r(x_0)}^{}u \,\mathrm{d}x
		\end{equation}
	\end{enumerate}
\end{satz}
\begin{beweis}
	\begin{description}
		\item[\underline{$(i) \Rightarrow (ii)$}:] War gerade die Aussage $(i)$ von Korrolar $2.4$
		\item[\underline{$(ii) \Rightarrow (iii)$}:] folgt mittels der Polarkoordinaten
		\begin{align*}
			\int_{B_r(x_0)}^{}u \,\mathrm{d}x &= \int_{0}^{r} \int_{\partial B_r(x_0)}^{} u \,\mathrm{d}S \,\mathrm{d}\rho  \\
			&= \int_{0}^{r} \rho^{n-1}S_n \underset{=u(x_0)}{\underbrace{\fint_{\partial B_{\rho}(x_0)}^{} u \,\mathrm{d}S}} \,\mathrm{d} \rho \\
			&= \int_{0}^{r}S_n \rho^{n-1} u(x_0)\,\mathrm{d}\rho \\
			&= S_n \frac{r^n}{n} u(x_0)
		\end{align*}
		Es folgt \[
			u(x_0) = \fint_{B_r(x_0)}^{} u \,\mathrm{d}x
		\]
		\item[\underline{$(iii) \Rightarrow (iv)$}:] ist offensichtlich
		\item[\underline{$(iv) \Rightarrow (i)$}:] folgt durch ein Widerspruchsargument: \\
		Angenommen, es gilt $(iv)$ und $\Delta u(x_0) \neq 0$ für ein $x_0 \in \Omega$, zum Beispiel $ \Delta u(x_0) > 0$. Da $u \in C^2(\Omega)$ ist, findet man eine Kugel $B_r(x_0)$ mit $r < R(x_0)$, so dass
		\[
			\Delta u(x) > 0 \qquad \text{in } B_r(x_0)
		\]
		Außerdem folgt dann aus Korollar $2.4$, dass 
		\[
			u(x_0)< \fint_{\partial B_r(x_0)}^{} u \,\mathrm{d}S
		\]
		Dies ist ein Widerspruch.
	\end{description}
\end{beweis}

\subsection{Folgerung aus der Mittelwerteigenschaft} 
\label{sub:folgerung_aus_der_mittelwerteigenschaft}
\begin{satz}[Maximumsprinzipien]
	Sei $\Omega$ eine beschränkte, offene Teilmenge des $\mathbb{R}^n$ und $u \in C^0(\overline{\Omega}) \cap C^2(\Omega)$ eine subharmonische Funktion (d.h. $ \Delta u \geq 0$ in $\Omega$). Dann gilt
	\begin{enumerate}[(i)]
		\item Das schwache Maximumsprinzip:
		\[
			\max_{\overline{\Omega}} u = \max_{\partial \Omega}u
		\]
		\item Das starke Maximumsprinzip: \\
		Ist $\Omega$ zusammenhängend und existiert ein innerer Punkt $x_0 \in \Omega$ mit \[
			u(x_0) = \max_{ \overline{\Omega}} u,
		\]so ist $u$ konstant.
	\end{enumerate}
\end{satz}

\begin{definition*}
	Eine Teilmenge $\Omega$ in $\mathbb{R}^n$ heißt zusammenhängend,
	falls sie sich nicht als disjunkte Vereinigung zweier nicht leerer $\Omega$-offene (= relativ offen in $\Omega$) Mengen schreiben lässt. \\
	\\
	Wenn $\Omega = O_1 \cup O_2$ mit $O_1,O_2$ offen und $O_1 \cap O_2 = \emptyset$, so folgt entweder $O_1 = \emptyset$ oder $O_2 = \emptyset$.
\end{definition*}
\begin{bemerkung}
	Diese Definition ist äquivalent zu: \\
	\\
	$\Omega$ und $\emptyset$ sind die beiden einzigen Mengen, die zugleich $\Omega$-offen und $\Omega$-abgeschlossen.
\end{bemerkung}

\begin{beweis}
	\begin{description}
		\item[\underline{$(ii) \Rightarrow (i)$}:] Beweis durch ein Wiederspruchsargument: \\
		Angenommen, es gelte 
		\begin{equation}
			\max_{\overline{\Omega}}u > \max_{\partial \Omega} u
		\end{equation}
		($\geq$ ist trivial, da $\partial \Omega \subseteq \overline{\Omega}$). \\
		Dann gäbe es ein Punkt $x_0 \in \Omega$ mit $u(x_0) = \max_{\overline{\Omega}}u$ auf die Zusammenhangskomponente von $\Omega(x_0)$, wäre also
		\begin{align*}
			\max_{\overline{\Omega(x_0)}}u &= \max_{\overline{\Omega}}u \\
			&> \max_{\partial \Omega}u \\
			&\geq  \max_{ \partial \Omega(x_0)}u
		\end{align*}
		Dies gilt wegen $ \partial \Omega(x_0) \subseteq \partial \Omega$. \\
		Aus dem starken Maximumsprinzip folgt, dass $u$ Konstant in $\Omega(x_0)$ ist und dies ist ein Widerspruch zu der obigen Aussage
		\[
			\max_{\overline{\Omega(x_0)}}u > \max_{\partial \Omega(x_0)} u
		\]
		\item[\underline{$(ii)$}:] Sei $x_0 \in \Omega$ mit $u(x_0) = \max_{\overline{\Omega}}u =: M$. Dann gilt
		\begin{align*}
			M = u(x_0) \stackrel{(*)}{\leq } \fint_{B_r(x_0)}^{} u \,\mathrm{d}x \leq  \fint_{B_r(x_0)}^{} M \,\mathrm{d}x = M 
		\end{align*}
		mit $\overline{B_r(x_0)} \subseteq \Omega$. Hierbei folgt $(*)$ aus der Mittelwerteigenschaft mit $ \Delta u \geq 0$. $M \leq M$ ist allerdings nur möglich, 
		wenn $u \equiv M$ auf $B_r(x_0)$ ist. Sei
		\[
			\Omega_M := \set[x \in \Omega]{u(x)=M}
		\]
		Dann ist $\Omega_M$ offen also findet man für jedes $x_0 \in \Omega_M$ eine offene Kugel $B_r(x_0) \subseteq \Omega_M$ finden. \\
		Für $x_0 \in \Omega_M$ existiert $r>0$, so dass auf $B_r(x_0)$ $u(x) \equiv M$ gilt. 
		Daraus folgt $B_r(x_0) \subseteq  \Omega_M$ und somit ist $\Omega_M$ offen. \\
		Da $u$ stetig ist, ist $\Omega_M$ relativ abgeschlossen in $\Omega$. 
		Und weil $\Omega$ zusammenhängend ist, gilt somit entweder $\Omega_M = \Omega$ oder $\Omega_M = \emptyset$.
		Da aber $x_0 \in \Omega_M$ gilt somit $\Omega_M = \Omega$ und so $u \equiv M$ auf $\Omega$.
	\end{description}
\end{beweis}
\cleardoubleoddemptypage
\pagenumbering{Alph}
\setcounter{page}{1}

\end{document}