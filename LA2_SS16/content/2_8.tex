%!TEX root = ../LA2.tex
\section{Quadratische Polynome und Quadriken}
\label{sec:2.8}

Wir wollen die Diagonalisierbarkeit von selbstadjungierten Matrizen ausnutzen, um in diesem Abschnitt quadratische Polynome auf $\RR^n$ zu untersuchen.

\begin{definition}[Quadratisches Polynom]
	\label{def:8.1}
	Eine Abbildung $p \colon \RR^n \rightarrow \RR$ heißt \Index{quadratisches Polynom}, wenn $b_{ij}, \alpha_j, \alpha \in \RR, 1 \leq i,j \leq n$ existieren, sodass für alle $x = (x_1,\dots,x_n)^T \in \RR^n$ gilt:
	\[
		p((x_1,\dots,x_n)^T) = \sum_{i,j=1}^{n} b_{ij} x_i x_j + \sum_{j=1}^{n} \alpha_j x_j + \alpha.
	\]
\end{definition}

Die $x_ix_j = x_jx_i$, können wir in der Formel die $b_{ij}$ durch $a_{ij} := \frac{1}{2}(b_{ij} + b_{ji})$ ersetzen, und erhalten dann $a_{ij} = a_{ji}$.
Ist dann $A = (a_{ij})_{ij} \in M(n \times n, \RR)$, so ist $A = A^T$ und wir erhalten
\[
	p((x_1,\dots,x_n)^T) = \sum_{i,j=1}^{n} a_{ij} x_i x_j + \sum_{j=1}^{n} \alpha_j x_j + \alpha.
\]

\begin{lemma}
	\label{lemma:8.2}
	Ist $A = A^T$ wie oben und ist $a := \frac{1}{2} \cdot (\alpha_1,\dots,\alpha_n) \in \RR^n$, so gilt für alle $x \in \RR^n$:
	\[
		p(x) = \sk{Ax,x} + 2 \sk{a,x} + \alpha = x^T A x + 2 a^T x + \alpha.
	\]
\end{lemma}

\begin{beweis}
	Die erste Gleichung folgt direkt durch einsetzen.
	Die zweite Gleichung folgt aus $\sk{y,x} = y^Tx$ für alle $x,y \in \RR^n$ und $\sk{Ax,x} = \sk{x,A^Tx} = \sk{x,Ax} = x^T Ax$. \qedhere
\end{beweis}

Unser Ziel: Wir wollen durch geeignete Wahl von Koordinaten in $\RR^n$ und durch Verschiebung des Ursprungs erreichen, dass $p$ eine besonders schöne Gestalt annimmt (vgl. mit Blatt 6, Aufgabe 1).
Beachte: Der Faktor $2$ vor dem Term $\sk{a,x}$ bzw. $a^Tx$ vereinfacht einige Umformungen!

\begin{lemma}
	\label{lemma:8.3}
	Sei $A \in M(n \times n,\RR)$ mit $A = A^T$.
	Dann gilt $\Bild(A) = \Bild(A^2)$.
	Insbesondere existiert zu jedem $a \in \RR^n$ ein $c \in \RR^n$ mit $A^2c = -Aa$.
\end{lemma}

\begin{beweis}
	Sei $\{v_1,\dots,v_n\}$ eine Orthonormalbasis aus Eigenvektoren von $A$ mit zugehörigen Eigenwerten $\lambda_1,\dots,\lambda_n$.
	Die Reihenfolge der Eigenvektoren und Eigenwerte sei so gewählt, dass $\lambda_1,\dots,\lambda_k \neq 0$ und (falls $k \leq n$) $\lambda_{k+1}, \dots, \lambda_n = 0$.
	Dann folgt für ein $v = \sum_{i=1}^{n} \mu_i v_i$:
	\[
		Av = \sum_{i=1}^{n} \mu_i Av_i = \sum_{i=1}^{n} \mu_i \lambda_i v_i = \sum_{i=1}^{k} \mu_i \lambda_i v_i.
	\]
	Setzen wir dann $w := \sum_{i=1}^{k} \frac{\mu_i}{\lambda_i} v_i$, so folgt
	\[
		A^2w = A \enb{\sum_{k=1}^{n} \frac{\mu_i}{\lambda_i} Av_i} = A \enb{\sum_{i=1}^{k} \mu_i v_i} = \sum_{i=1}^{n} \mu_i \lambda_i v_i = Av.
	\]
	Damit folgt $\Bild(A) \subseteq \Bild(A^2)$.
	Die Umkehrung ist trivial. \qedhere
\end{beweis}

\begin{lemma}[Reduktion durch Translation]
	\label{lemma:8.4}
	Sei $p(x) = x^T A x + 2a^Tx + \alpha$ mit $A = A^T \in M(n \times n, \RR), a \in \RR^n$ und $\alpha \in \RR$.
	Ferner sei $c \in \RR^n$ mit $A^2c = -Aa$ (existiert nach \autoref{lemma:8.3}) und sei $b := Ac + a$.
	Dann gelten:
	\begin{enumerate}[(i)]
		\item $p(x+c) = x^T Ax + 2b^Tx + p(c)$ und $Ab = 0$.
		\item Ist $b = 0$, so folgt $p(x+c) = x^TAx + p(c)$ (kein linearer Term).
		\item Ist $b \neq 0$, so gilt für $d := c+\delta b$ mit $\delta := - \frac{p(c)}{2 \no{b}^2}$: $p(x+d) = x^TAx + 2b^Tx$ (kein konstanter Term).
	\end{enumerate}
\end{lemma}

\begin{beweis}
	Zunächst gilt für alle $x,y\in \RR^n$:
	\begin{align*}
		p(x+y) &\stack{}{=} (x+y)^T A (x+y) = 2a^T(x+y) + \alpha \\
		&\stack{}{=} x^T Ax + 2y^T Ax + y^T Ay + 2a^Tx + 2a^Ty + \alpha \\
		&\stack{A=A^T}{=} x^TAx + 2(Ay + a)^Tx + p(y).
	\end{align*}
	\begin{enumerate}[(i)]
		\item Die Gleichung $p(x+c) = x^TAx + 2b^Tx + p(c)$ folgt auch obiger Rechnung mit $b = Ac+a$.
		Da $A^2c = -Aa$, folgt $Ab = A(Ac + a) = A^2c + Aa = 0$.
		\item folgt sofort aus (i).
		\item Sei $d = c + \delta b$ wie in (iii).
		Dann gilt
		\begin{align*}
			p(x+d)& \stack{}{=} x^TAx + 2(Ad+a)^T x + p(d) \\
			&\stack{}{=} x^TAx + 2(A(c+\delta b) + a)^T x + p(d) \\
			&\stack{Ab=0}{=} x^TAx + 2\Underbrace{(Ac+a)}{=b}^Tx + p(d) = x^TAx + 2b^Tx + p(d).
		\end{align*}
	\end{enumerate}
	Zeige nun $p(d) = 0$.
	Es gilt
	\begin{align*}
		p(d) &= p(c+\delta b) = c^TAc + 2(\overbrace{A\delta b}^{\mathclap{=0}}+a)^T c + p(\delta b) \\
		&= p(c) - \alpha + p(\delta b) = p(c) - \alpha + \delta^2 b^TAb + \delta 2 a^T b + \alpha \\
		&= p(c) - p(c) \enb{\frac{1}{2 \no{b}^2}} 2a^Tb = 0,
	\end{align*}
	denn $\no{b}^2 = b^Tb = (Ac+a)^Tb = c^T\Underbrace{Ab}{=0} + a^Tb = a^Tb$. \qedhere
\end{beweis}

\newpage
