%!TEX root = ../LA2_SS16.tex
\subsection{Diagonalisierbarkeit normaler Endomorphismen}
\label{sec:2.7}

In diesem Abschnitt wollen wir die Eigenwerttheorie benutzen, um selbstadjungierte oder unitäre Endormorphismen genauer zu untersuchen.
Beide sind \enquote{normal} im folgenden Sinne:

\begin{definition}
	\label{def:7.1}
	Sei $(V,\sk{\cdot,\cdot}$ ein endlich dimensionaler unitärer bzw. euklidischer $\KK$-Vektorraum.
	Ein Endomorphismus $F \in \End(V)$ heißt \Index{normal}, falls $F^* \circ F = F \circ F^*$.
	
	Analog heißt eine Matrix $A \in M(n \times n,\KK)$ \Index{normal}, wenn $A^*A = AA^*$ gilt.
\end{definition}

Ist $F = F^*$, so ist $F$ auch normal, da $F^* \circ F = F \circ F = F \circ F^*$.
Ist $F$ unitär bzw. orthogonal, so ist $F$ ebenfalls normal, da dann $F^* \circ F = \id_V = F \circ F^*$ gilt (analog für Matrizen).

Wir wollen im Folgenden zeigen, dass jeder normale Endomorphismus über $\CC$ diagonalisierbar ist.
Es gibt dann sogar eine Orthonormalbasis aus Eigenvektoren!
Analoges folgt dann auch für selbstadjungierte Endomorphismen über $\RR$.
Ferner werden wir die Struktur von beliebigen orthogonalen Endomorphismen bzw. Matrizen verstehen.

\begin{lemma}
	\label{lemma:7.2}
	Sei $(V,\sk{\cdot,\cdot})$ ein endlich dimensionaler unitärer $\CC$-Vektorraum und sei $F \in \End(V)$ normal.
	Dann gelten:
	\begin{enumerate}[(i)]
		\item Ist $\lambda \in \CC$ ein Eigenwert von $F$, so ist $\ol{\lambda}$ ein Eigenwert von $F^*$ und es gilt $E_\lambda(F) = E_{\ol{\lambda}}(F^*)$.
		\item Sind $\lambda_1,\lambda_2 \in \CC$ Eigenwerte von $F$ mit $\lambda_1 \neq \lambda_2$, so gilt $E_{\lambda_1}(F) \perp E_{\lambda_2}(F)$.
	\end{enumerate}
\end{lemma}

\begin{beweis}
	\begin{enumerate}[(i)]
		\item Da $F$ normal ist, ist auch $\lambda \cdot \id -F$ normal, denn
		\[
			(\lambda \cdot \id - F)^* \circ (\lambda \cdot \id - F) = \lambda \ol{\lambda} \id - \ol{\lambda} F - \lambda F^* + F \circ F^* = \lambda \ol{\lambda} \id - \lambda F^* - \ol{\lambda} F + F^* \circ F = (\lambda \cdot \id - F) \circ (\lambda \cdot \id - F)^*.
		\]
		Nun gilt für alle $G \in \End(V)$ mit $G$ normal, dass $\Kern(G) = \Kern(G^*)$, denn:
		\begin{align*}
			v \in \Kern(G) \quad &\Leftrightarrow \quad 0 = \sk{Gv,Gv} = \sk{G^* \circ Gv,v} = \sk{G \circ G^*v,v} = \sk{G^*v,G^*v} \\
			&\Leftrightarrow \quad v \in \Kern(G^*).
		\end{align*} 
		Mit $G = \lambda \id - F$ folgt
		\[
			E_\lambda(F) = \Kern(\lambda \id - F) = \Kern(\lambda \id - F)^* = \Kern(\ol{\lambda} \id - F^*) = E_{\ol{\lambda}}(F^*).
		\]
		\item Sind $v \in E_{\lambda_1}(F), w \in E_{\lambda_2}(F) = E_{\ol{\lambda_2}}(F^*)$, so folgt
		\[
			\lambda_1 \sk{v,w} = \sk{\lambda_1v,w} = \sk{F(v),w} = \sk{v,F^*(w)} = \sk{v,\ol{\lambda_2} w} = \lambda_2 \sk{v,w},
		\]
		also $(\lambda_1 - \lambda_2)\sk{v,w} = 0$.
		Da $\lambda_1 \neq \lambda_2$, folgt $\sk{v,w} = 0$. 
	\end{enumerate}
\end{beweis}

\begin{satz}
	\label{satz:7.3}
	Sei $(V,\sk{\cdot,\cdot})$ ein endlich dimensionaler unitärer $\CC$-Vektorraum und sei $F \in \End(V)$ normal.
	Dann besitzt $V$ eine Orthonormalbasis $B = \{v_1,\dots,v_n\}$ aus Eigenvektoren von $F$.
	Insbesondere ist $F$ diagonalisierbar.
\end{satz}

\begin{beweis}
	Seien $\lambda_1,\dots,\lambda_k$ die paarweise verschiedenen Eigenwerte von $F$ (da $\KK = \CC$, existiert mindestens ein Eigenwert, da das charakteristische Polynom von $F$ mindestens eine Nullstelle besitzt).
	Seien $E_{\lambda_1}(F), \dots E_{\lambda_k}(F)$ die zugehörigen Eigenräume.
	Bestimme (etwa mit dem Schmidtschen Verfahren) eine Orthonormalbasis $\{v_{i_1},\dots v_{i_{n_i}}\}$ für $E_{\lambda_i}(F)$ für alle $1 \leq i \leq k$.
	Nach Lemma~\ref{lemma:7.2} gilt $E_{\lambda_i}(F) \perp E_{\lambda_j}(F)$ für $i \neq j$, und damit ist
	\[
		B = \{v_{11},\dots,v_{1,n_1},v_{21},\dots,v_{2,n_2},\dots,v_{k1},\dots,v_{k,n_k}\}
	\]
	ein Orthonormalsystem aus Eigenvektoren von $F$ in $V$.
	
	Zeige nun $V = \LH(B)$ (dann ist $B$ die gesuchte Orthonormalbasis).
	Sei dazu $W := \LH(B) = \bigoplus_{i=1}^k E_{\lambda_i}(V)$.
	Dann gilt $W \neq V \Leftrightarrow W^\perp \neq \setzero$.
	Wir wollen dies zu einem Widerspruch führen!
	
	Angenommen, $W^\perp \neq \setzero$.
	Dann gilt $F(W^\perp) \subseteq W^\perp$, das heißt $F \big|_{W^\perp} \in \End(W^\perp)$, denn:
	Da $E_{\lambda_i}(F) = E_{\ol{\lambda_i}}(F^*)$, folgt $F^*(E_{\lambda_i}(F)) \subseteq E_{\lambda_i}(F)$ und damit $F^*(W) \subseteq W$, da $W = \bigoplus_{i=1}^k E_{\lambda_i}(F)$.
	Dann folgt für alle $w \in W$ und $v \in W^\perp$:
	\[
		\sk{F(v),w} = \sk{v,F^*(w)} = 0,
	\]
	also $F(v) \in W^\perp$.
	
	Da $\KK= \CC$ besitzt $F \big|_{W^\perp}$ mindestens einen Eigenwert $\mu \in \CC$, und dann existiert ein Eigenvektor $v \in W^\perp \setminus \setzero$ für $\mu$.
	Da $W^\perp \subseteq V$, ist $\mu$ auch Eigenwert von $F$ und $v \in E_\mu(F)$, das heißt es existiert ein $i \in \{1,\dots,k\}$ mit $\mu = \lambda_i$ und dann $v \in E_{\lambda_i}(F) \subseteq W$.
	Damit ist $v \in W \cap W^\perp = \setzero$.
	Widerspruch zu $v \neq 0$. 
\end{beweis}

\begin{korollar}
	\label{kor:7.4}
	Sei $A \in M(n\times n,\CC)$ normal.
	Dann existiert eine unitäre Matrix $U \in \UU(n)$, sodass $U^*AU$ eine Diagonalmatrix ist.
\end{korollar}

\begin{beweis}
	Betrachte $\CC^n$ mit dem Standard-Skalarprodukt und $F_A \colon \CC^n \rightarrow \CC^n, x \mapsto Ax$.
	Dann ist $F_A$ normal und ist $B = \{v_1,\dots,v_n\}$ eine Orthonormalbasis von $\CC^n$ wie in Satz~\ref{satz:7.3}, so ist $U = (v_1,\dots,v_n)$ invertierbar und $U^{-1} A U$ ist eine Diagonalmatrix (vgl. Lineare Algebra I, \ref{sec:I.6}).
	Nach Satz~\ref{satz:6.9} ist $U$ unitär und $U^* = U^{-1}$. 
\end{beweis}

Wir wollen nun die reelle Situation untersuchen.
Sei also im Folgenden $(V,\sk{\cdot,\cdot})$ ein endlich dimensionaler euklidischer $\RR$-Vektorraum und sei $F \in \End(V)$ normal.
Dann ist $F$ nicht immer über $\RR$ diagonalisierbar.
Zum Beispiel ist
\[
	O(\alpha) = \begin{pmatrix}
		\cos(\alpha) & -\sin(\alpha) \\
		\sin(\alpha) & \cos(\alpha)
	\end{pmatrix}
\]
diagonalisierbar genau dann, wenn $\alpha = 0,\pi$, aber $O(\alpha)$ ist normal für alle $\alpha \in \RR$, da orthogonal.
Nach Satz~\ref{satz:7.3} bzw. Korollar~\ref{kor:7.4} ist $O(\alpha)$ aber über $\CC$ diagonalisierbar.
Wir wollen dieses nutzen, um auch die reelle Situation zu verstehen.

\begin{definition}[Komplexifizierung]
	\label{def:7.5}
	Sei $(V,\sk{\cdot,\cdot})$ ein euklidischer $\RR$-Vektorraum.
	Wir setzen dann
	\[
		V_{\CC} := \{ v+iw : v,w \in V\}
	\]
	versehen mit der Addition und skalaren Multiplikation
	\begin{align*}
		(v_1 + iw_1) + (v_2 + iw_2) &:= (v_1+v_2) + i(w_1 + w_2) \\
		(a+ib) \cdot (v + iw) &:= (av - bw) + i (bv + aw)
	\end{align*}
	und dem komplexwertigen Skalarprodukt
	\[
		\sk{v_1+iw_1,v_2+iw_2}_{\CC} := \sk{v_1,v_2} + \sk{w_1,w_2} + i(\sk{w_1,v_2}-\sk{v_1,w_2}).
	\]
	Dann gilt:
	$(V_{\CC},\sk{\cdot,\cdot}_{\CC})$ ist ein unitärer $\CC$-Vektorraum mit $\dim_{\CC}(V_{\CC}) = \dim_{\RR}(V)$.
	Ist $F \in \End(V)$, so wird durch
	\begin{align*}
		F_{\CC}\colon V_{\CC} &\longrightarrow V_{\CC} \\
		v+iw &\longmapsto F(v) + i F(w)
	\end{align*}
	ein Endomorphismus $F_{\CC} \in \End(V_{\CC})$ definiert.
	$(V_{\CC},\sk{\cdot,\cdot}_{\CC})$ heißt die \Index{Komplexifizierung} von $(V,\sk{\cdot,\cdot})$ und $F_{\CC}$ heißt die Komplexifizierung von $F$.
\end{definition}

\begin{beispiel}
	\label{bsp:7.6}
	Ist $V = \RR^n$ mit Standard-Skalarprodukt, so ist $V_{\CC} = \CC^n$ mit Standard-Skalarprodukt, denn jeder Vektor $z \in \CC^n$ besitzt eine eindeutige Zerlegung $z = x+ iy$ mit $x,y \in \RR^n$.
	
	Ist $A \in M(n \times n,\RR)$ und $F_A\colon \RR^n \rightarrow \RR^n, x \mapsto Ax$, so ist $(F_A)_{\CC}(z) = Az$, wenn wir $A$ als komplexe Matrix auffassen.
\end{beispiel}

\begin{lemma}
	\label{lemma:7.7}
	Sei $(V,\sk{\cdot,\cdot})$ ein endlich dimensionaler euklidischer $\RR$-Vektorraum und sei $F \in \End(V)$.
	Dann sind äquivalent:
	\begin{enumerate}[(i)]
		\item $F$ ist normal (bzw. orthogonal bzw. selbstadjungiert)
		\item $F_{\CC}$ ist normal (bzw. unitär bzw. selbstadjungiert)
	\end{enumerate}
	Ferner gilt $F_{\CC}^* = (F^*)_{\CC}$.
\end{lemma}

\begin{beweis}
	Wir zeigen nur $F_{\CC}^* = (F^*)_{\CC}$.
	Die Äquivalenz der beiden Aussagen folgt dann leicht.
	Für alle $v_1+iw_1, v_2+iw_2 \in V_{\CC}$ gilt:
	 
	\vspace*{-1.3cm}
	\begin{align*}
	\sk{F_{\CC}(v_1+iw_1),v_2+iw_2} &= \sk{F(v_1) + i F(w_1),v_2+iw_2} \\
	&= \sk{F(v_1),v_2} + \sk{F(w_1),w_2} + i( \sk{F(w_1),v_2} - \sk{F(v_1),w_2}) \\
	&= \sk{v_1,F^*(v_2)} + \sk{w_1,F^*(w_2)} + i(\sk{w_1,F^*(v_2)} - \sk{v_1,F^*(w_2)}) \\
	&= \sk{v_1 + iw_2,F^*(v_2) + iF^*(w_2)} \\
	&= \sk{v_1 + iw_1, (F^*)_{\CC}(v_2+iw_2)} 
	\end{align*}
\end{beweis}

\begin{bemerkung}
	\label{bem:7.8}
	Ist $(V,\sk{\cdot,\cdot})$ ein euklidischer $\RR$-Vektorraum und ist $B := \{v_1,\dots,v_n\}$ eine Orthonormalbasis von $V$, so ist $B$ auch eine Orthonormalbasis von $V_{\CC}$ (wir fassen $V$ als Teilmenge von $V_{\CC}$ auf via $v \mapsto v + i \cdot 0$).
	Denn es gilt $\sk{v_i,v_j}_{\CC} = \sk{v_i,v_j} = \delta_{ij}$ und $\dim(V_{\CC}) = \dim(V) = n$.
	
	Ist $F \in \End(V)$, so erhält man für die zugehörige Darstellungsmatrix $A^B_{F_{\CC}} = A^B_F$, denn ist $A^B_{F_{\CC}} = (\wt{a}_{ij}), A^B_F = (a_{ij})$, so gilt nach Korollar~\ref{kor:5.4}:
	\[
		\wt{a}_{ij} = \sk{F_{\CC}(v_j + i0),v_i + i0}_{\CC} = \sk{F(v_j)+i0,v_i+i0} = \sk{F(v_j),v_i} = a_{ij}.
	\]
\end{bemerkung}

Wir wollen aber nun umgekehrt aus einer Orthonormalbasis von $V_{\CC}$ aus Eigenvektoren von $F_{\CC}$ eine \enquote{schöne} Orthonormalbasis für $V$ basteln.
Dazu benutzen wir zunächst, dass jedes reelle Polynom $p(\lambda) = \sum_{k=0}^{n} a_k \lambda^k$ durch Einsetzen von $\lambda \in \CC$ auch als komplexes Polynom aufgefasst werden kann.

\begin{lemma}
	\label{lemma:7.9}
	Sei $(V,\sk{\cdot,\cdot})$ ein endlich dimensionaler euklidischer $\RR$-Vektorraum und sei $F \in \End(V)$.
	Dann gilt $\chi_{F_{\CC}}(\lambda) = \chi_F(\lambda)$ für alle $\lambda \in \CC$, das heißt das charakteristische Polynom von $F_{\CC}$ ist gleich dem charakteristischen Polynom von $F$ (fortgesetzt auf $\CC$).
\end{lemma}

\begin{beweis}
	Wähle eine Orthonormalbasis $B = \{v_1,\dots,v_n\}$ von $V$ wie in Lemma~\ref{lemma:7.7}.
	Dann folgt mit Bemerkung~\ref{bem:7.8}:
	$\chi_{F_{\CC}}(\lambda) = \det(\lambda E_n - A^B_{F_{\CC}}) = \det(\lambda E_n - A^B_F) = \chi_F(\lambda)$. 
\end{beweis}

\begin{lemma}
	\label{lemma:7.10}
	Sei $p(\lambda) = \sum_{k=0}^{n} a_k \lambda^k$ ein reelles Polynom.
	Dann gilt:
	Ist $\mu \in \CC$ eine $k$-fache komplexe Nullstelle von $p$, so ist auch $\ol{\mu} \in \CC$ eine $k$-fache komplexe Nullstelle von $p$.
	Damit besitzt $p$ eine Faktorisierung 
	\[
		p(\lambda) = a(\lambda-\lambda_1)^{n_1} \cdots (\lambda-\lambda_l)^{n_l}(\lambda-\mu_1)^{m_1} (\lambda-\ol{\mu}_1)^{m_1} \cdots (\lambda-\mu_k)^{m_k}(\lambda-\ol{\mu}_k)^{m_k},
	\]
	wobei $\lambda_1,\dots,\lambda_l$ die paarweise verschiedenen reellen Nullstellen mit Vielfachheiten $n_1,\dots,n_l$ und \linebreak $\mu_1,\ol{\mu}_1,\dots,\mu_l,\ol{\mu}_l$ die paarweise verschiedenen konjugiert komplexen Paare nicht-reller Nullstellen mit Vielfachheiten $m_1,\dots,m_k$ von $p$ sind.
\end{lemma}

\begin{beweis}
	Sei $\mu \in \CC$ Nullstelle von $p$.
	Dann gilt
	\[
		p(\ol{\mu}) = \sum_{k=0}^{n} a_k \ol{\mu}^k = \sum_{k=0}^{n} \ol{a_k} \ol{\mu}^k = \ol{\sum_{k=0}^{n} a_k \mu^k} = \ol{p(\mu)} = 0,
	\]
	also ist auch $\ol{\mu}$ Nullstelle von $p$.
	Ist $\mu \notin \RR$, so ist $\ol{\mu} \neq \mu$.
	Ferner gilt mit $\mu = \alpha + i \beta$:
	\[
		(\lambda-\mu)(\lambda-\ol{\mu}) = (\lambda - (\alpha-i\beta))(\lambda-(\alpha-i\beta)) = \lambda^2 + 2\alpha \lambda + (\alpha^2 + \beta^2),
	\]
	also ist $(\lambda-\mu)(\lambda-\ol{\mu})$ ein reelles Polynom.
	Damit existiert ein reelles Polynom mit $\grad(q) = \grad(p) - 2$ und $p(\lambda) = q(\lambda)(\lambda-\mu)(\lambda-\ol{\mu})$.
	Die Behauptung folgt dann per Induktion nach $\grad(p)$. 
\end{beweis}

\begin{korollar}
	\label{lemma:7.11}
	Ist $p(\lambda) = \sum_{k=0}^{n} a_k \lambda^k$ ein reelles Polynom, so besitzt $p$ eine Faktorzerlegung
	\[
		p(\lambda) = a(\lambda-\lambda_1)^{n_1} \cdots (a-\lambda_l)^{n_l} (\lambda^2 + a_1 \lambda + b_1)^{m_1} \cdots (\lambda^2 + a_k \lambda + b_k)^{m_k}
	\]
	mit $a,\lambda_1,\dots,\lambda_l \in \RR$ und $\lambda^2 + a_i \lambda + b_i$ ist quadratisches reelles Polynom, das keine reelle Nullstelle besitzt.
\end{korollar}

\begin{beweis}
	Ist $\mu_i, \ol{\mu}_i$ ein konjugiert komplexes Nullstellenpaar von $p$, so gilt mit $\mu_i = \alpha_i + i \beta_i$:
	\[
		(\lambda - \mu_i)(\lambda - \ol{\mu}_i) = \lambda^2 - 2 \alpha_i \lambda + (\alpha_i^2 + \beta_i^2),
	\]
	wobei die rechte Seite keine Nullstellen in $\RR$ hat, da beide Nullstellen in $\CC \setminus \RR$ sind.
	Damit ist $(\lambda - \mu_i)^{m_i} (\lambda - \ol{\mu}_i)^{m_i} = (\lambda^2 - 2 \alpha_i \lambda + (\alpha_i^2 + \beta_i^2))^{m_i}$, also $a_i = -2\alpha_i, b_i = (\alpha_i + \beta_i)^2$. 
\end{beweis}

\begin{lemma}
	\label{lemma:7.12}
	Seien $(V,\sk{\cdot,\cdot})$ ein endlich dimensionaler euklidischer $\RR$-Vektorraum und $F \in \End(V)$ normal.
	Für $u = v + iw \in V_{\CC}$ setze $\ol{u} = v - iw \in V_{\CC}$.
	Dann gelten:
	\begin{enumerate}[(i)]
		\item Ist $\lambda = \alpha + i \beta$ ein Eigenwert von $F_{\CC}$ und $u = v+iw \in V_{\CC}$ ein Eigenvektor zu $\lambda$, so gilt $F(v) = \alpha v - \beta w, F(w) = \beta v + \alpha w$ und $F_{\CC}(\ol{u}) = \ol{\lambda} \ol{u}$.
		\item Ist $\{u_1,\dots,u_l\}$ eine Orthonormalbasis für $E_\lambda(F_{\CC}) \subseteq V_{\CC}$, so ist $\{\ol{u_1},\dots,\ol{u_l}\}$ eine Orthonormalbasis für $E_{\ol{\lambda}}(F_{\CC})$.
		\item Ist $\lambda = \alpha + i\beta$ wie in (i) mit $\beta \neq 0$ und ist $\{u_1,\dots,u_l\}$ eine Orthonormalbasis für $E_\lambda(F_{\CC})$ mit $u_j = v_j + iw_j$, so ist
		\[
			\penb{\sqrt{2} v_1, \sqrt{2} w_2, \dots, \sqrt{2} v_l,\sqrt{2} w_l}
		\]
		eine Orthonormalbasis von $E_{\lambda}(F_{\CC}) \oplus E_{\ol{\lambda}}(F_{\CC})$.
	\end{enumerate}
\end{lemma}

\begin{beweis}
	\begin{enumerate}[(i)]
		\item Sei $\lambda = \alpha + i \beta$ ein Eigenwert von $F_{\CC}$ und $u = v+iw$ ein Eigenvektor zu $\lambda$.
		Dann gilt:
		\[
			F(v) + i F(w) = F_{\CC}(v+iw) = (\alpha + i\beta)(v+iw) = \alpha v - \beta w +i(\beta v + \alpha w).
		\]
		Damit folgt $F(v) = \alpha v - \beta w$ und $F(w) = \beta v + \alpha w$.
		 
		Dann folgt auch
		\[
			F_{\CC}(v-iw) = F(v) - iF(w) = (\alpha v - \beta w)-i(\beta v + \alpha w) = (\alpha - i\beta)(v-iw),
		\]
		also $F_{\CC}(\ol{u}) = \ol{\lambda} \ol{u}$.
		\item Ist $\{u_1,\dots,u_l\}$ ein Orthonormalsystem, so auch $\{\ol{u_1},\dots,\ol{u_l}\}$, denn man rechnet schnell nach, dass $\sk{\ol{u_i},\ol{u_j}} = \ol{\sk{u_i,u_j}} = \ol{\delta_{ij}} = \delta_{ij}$ gilt.
		Die Aussage folgt dann mit (i).
		\item Sei $u_j = v_j + iw_j$ für $1 \leq j \leq l$.
		Da $\lambda \neq \ol{\lambda}$ Eigenwert von $F_{\CC}$ und $F_{\CC}$ normal ist, gilt $E_{\lambda}(F_{\CC}) \perp E_{\ol{\lambda}}(F_{\CC})$ nach Lemma~\ref{lemma:7.2}.
		Damit folgt für $i \neq j$, dass $\sk{u_i,u_j} = 0 = \sk{u_i,\ol{u_j}}$ und $\sk{u_i,\ol{u_i}} = 0$.
		Wegen $v_j = \frac{1}{2}(u_j + \ol{u_j}), w_j = \frac{1}{2i}(u_j - \ol{u_j})$ folgt für $i \neq j$:
		\begin{align*}
			\sk{v_i,v_j} &= \sk{\frac{1}{2}(u_i+\ol{u_i}),\frac{1}{2}(u_j+\ol{u_j})} \\
			\sk{w_i,w_j} &= \sk{\frac{1}{2i}(u_i-\ol{u_i}),\frac{1}{2i}(u_j-\ol{u_j})} \\
			\sk{v_i,w_j} &= \sk{\frac{1}{2}(u_i+\ol{u_i}),\frac{1}{2i}(u_j-\ol{u_j})}.
		\end{align*}
		Ferner folgt wegen $E_{\lambda}(F_{\CC}) \perp E_{\ol{\lambda}}(F_{\CC})$ für alle $1 \leq j \leq l$:
		\begin{align*}
			0 = \sk{u_j,\ol{u_j}} &= \sk{v_j + iw_j,v_j -iw_j} \\
			&= \sk{v_j,v_j} - \sk{w_j,w_j} + i(\sk{w_j,v_j} + \sk{v_j,w_j}) \\
			&= \no{v_j}^2 - \no{w_j}^2 + 2i \sk{v_j,w_j}.
		\end{align*}
		Damit folgt $\no{w_j}^2 = \no{v_j}^2$ und $\sk{v_j,w_j} = 0$.
		Wegen $1 = \no{u_j}^2 = \no{v_j+iw_j}^2 \stackrel{\ref{satz:4.7}}{=} \no{v_j}^2 + \no{w_j}^2$ folgt dann $\no{v_j} = \no{w_j} \cdot \frac{1}{\sqrt{2}}$.
		Somit ist $\penb{\sqrt{2} v_1, \sqrt{2} w_2, \dots, \sqrt{2} v_l,\sqrt{2} w_l}$ ein Orthonormalsystem in $E_{\lambda}(F_{\CC}) \oplus E_{\ol{\lambda}}(F_{\CC})$ mit $2l$ Elementen, und damit eine Orthonormalbasis von $E_{\lambda}(F_{\CC}) \oplus E_{\ol{\lambda}}(F_{\CC})$. 
	\end{enumerate}
\end{beweis}

\begin{satz}[reelle normale Endomorphismen]
	\label{satz:7.13}
	Sei $(V,\sk{\cdot,\cdot})$ ein endlich dimensionaler euklidischer $\RR$-Vektorraum und sei $F \in \End(V)$ normal.
	Dann existiert eine Orthonormalbasis $B:= \{u_1,\dots,u_k,v_1,w_1,\dots,v_m,w_m\}$ von $V$ mit
	\[
		A^B_F = \enb{\begin{BMAT}(b){cccccccc}{cccccccc}
		\lambda_1 &  &  &  &  &  &  &  \\ 
		& \ddots &  &  &  &  &  &  \\ 
		&  & \lambda_k &  &  &  &  &  \\ 
		&  &  & \alpha_1 & \beta_1 &  &  &  \\ 
		&  &  & -\beta_1 & \alpha_1 &  &  &  \\ 
		&  &  &  &  & \ddots &  &  \\ 
		&  &  &  &  &  & \alpha_m & \beta_m \\ 
		&  &  &  &  &  & -\beta_m & \alpha_m
		\addpath{(6,0,|)uurrddll}
		\addpath{(3,3,|)uurrddll}
		\end{BMAT}}.
	\]
	Hierbei sind $\lambda_1,\dots,\lambda_k$ die reellen Eigenwerte von $F$ und $\mu_j = \alpha_j + i\beta_j$, $\ol{\mu_j} = \alpha_j - i\beta_j$ sind die komplexen Eigenwerte von $F_{\CC}$, $1\leq j \leq m$ (jeweils mit Vielfachheiten).
\end{satz}

\begin{beweis}
	Sei $F_{\CC} \in \End(V_{\CC})$ die Komplexifizierung von $F$.
	Wähle zu jedem Eigenwert $\lambda$ von $F_{\CC}$ eine Orthonormalbasis $\{u_1,\dots,u_l\}$ von $E_{\lambda}(F_{\CC})$ und wähle zu $E_{\ol{\lambda}}(F_{\CC})$ die Orthonormalbasis $\{\ol{u_1},\dots,\ol{u_l}\}$.
	\begin{description}
		\item[1. Fall:] Ist $\lambda \in \RR$, so können wir o.B.d.A. $u_i \in V$ annehmen, denn ist $u_i = v_i + iw_i$ ein komplexer Eigenvektor zu $\lambda$, so ist auch $\ol{u_i} = v_i - iw_i$ ein Eigenvektor zu $\ol{\lambda} = \lambda$, und dann sind auch $v_i = \frac{1}{2}(u_i + \ol{u_i})$ und $w_i = \frac{1}{2i}(u_i + \ol{u_i})$ Eigenvektoren zu $\lambda$.
		Da $u_i \in \LH\{v_i,w_i\}$, ist $\{v_1,w_1,\dots, v_l,w_l\}$ ein Erzeugendensystem von $E_{\lambda}(F_{\CC})$.
		Nach Basisauswahlsatz erhält dieses eine Basis $\{\wt{u_1},\dots,\wt{u_l}\}$ von $E_{\lambda}(F_{\CC})$.
		Wenden wir hierauf das Schmidtsche Orthonormalisierungsverfahren (Satz~\ref{satz:4.13}) an, erhalten wir eine Orthonormalbasis $\{u_1',\dots,u_l'\}$ von $E_\lambda(F_{\CC})$ mit $u_i'$ reell (also $u_i' \in V$) für alle $1 \leq i \leq l$.
		
		Beachte: Die Anwendung des Schmidtschen Orthonormalisierungsverfahrens auf reelle Vektoren führt wieder zu reellen Vektoren!
		\item[2. Fall:] Ist $\mu, \ol{\mu} \in \CC \setminus \RR$ ein Paar komplexer Eigenwerte von $F_{\CC}$, so gilt nach Lemma~\ref{lemma:7.12}(iii):
		Ist $u_i = \wt{v_i} + i \wt{w_i}$ und $v_i = \sqrt{2} \wt{v_i}$ sowie $w_i = \sqrt{2} \wt{w_i}$, so ist $\{v_1,w_1,\dots,v_l,w_l\}$ eine Orthonormalbasis von $E_\mu(F_{\CC}) \oplus E_{\ol{\mu}}(F_{\CC})$ und es gilt nach Lemma~\ref{lemma:7.12}(i): 
		\[
			F(v_i) = \alpha v_i - \beta w_i, \qquad F(w_i) = \beta v_i - \alpha w_i,
		\]
		wenn $\lambda = \alpha + i \beta$.
	\end{description}
	Ist $B = \{u_1,\dots,u_k,v_1,w_1,\dots,v_m,w_m\}$, wobei $\{u_1,\dots,u_k\}$ die Vereinigung der im Fall 1 konstruierten Orthonormalbasen für die reellen Eigenvektoren $\lambda_i$ und $\{v_1,w_1,\dots,v_m,w_m\}$ die Vereinigung der in Fall 2 konstruierten Orthonormalbasen für $E_{\mu_j}(F_{\CC}) \oplus E_{\ol{\mu_j}}(F_{\CC})$, so erhalten wir eine Orthonormalbasis von $V_{\CC}$ mit Elementen in $V$, und damit eine Orthonormalbasis von $V$ mit $A_F^B$ wie im Satz:
		
	Da $F(u_i) = \lambda_i u_i$, hat $A_F^B$ in der $i$-ten Spalte $\lambda_i$ als Diagonaleintrag und sonst nur Nullen.
	Da $F(v_j) = \alpha v_j - \beta w_j$ und $F(w_j) = \beta_j v_j + \alpha_j w_j$, erhalten wir (da $v_j$ an der $k+2j-1$-ten Stelle in $B$ und $w_j$ an $k+2j$-ter Stelle in $B$ steht) in der $k+2j-1$-ten Spalte $\alpha_j$ in Zeile $k+2j-1$ und $-\beta_j$ an Stelle $k+2j$, und in der $k+2j$-ten Spalte $\beta_j$ in Zeile $k+2j-1$ und $\alpha_j$ an Stelle $k+2j$ und sonst nur Nullen. 	
\end{beweis}

\begin{korollar}
	\label{kor:7.14}
	Sei $A \in M(n \times n,\RR)$ normal.
	Dann existiert eine orthogonale Matrix $O \in \oh(n)$ mit
	
	\[
		O^*AO = O^TAO = \enb{\begin{BMAT}(b){cccccccc}{cccccccc}
			\lambda_1 &  &  &  &  &  &  &  \\ 
			& \ddots &  &  &  &  &  &  \\ 
			&  & \lambda_k &  &  &  &  &  \\ 
			&  &  & \alpha_1 & \beta_1 &  &  &  \\ 
			&  &  & -\beta_1 & \alpha_1 &  &  &  \\ 
			&  &  &  &  & \ddots &  &  \\ 
			&  &  &  &  &  & \alpha_m & \beta_m \\ 
			&  &  &  &  &  & -\beta_m & \alpha_m
			\addpath{(6,0,|)uurrddll}
			\addpath{(3,3,|)uurrddll}
			\end{BMAT}}.
	\]
\end{korollar}

\begin{beweis}
	Wende Satz~\ref{satz:7.13} an auf $F_A \colon \RR^n \rightarrow \RR^n, x \mapsto Ax$ und setze $O = (u_1,\dots,u_k,v_1,w_1,\dots,v_m,w_m)$, wenn $B = \{u_1,\dots,u_k,v_1,w_1,\dots,v_m,w_m\}$ die in Satz~\ref{satz:7.13} konstruierte Orthonormalbasis von $\RR^n$ ist. 
\end{beweis}

Wir wollen dieses Resultat nutzen, um insbesondere orthogonale Endomorphismen bzw. Matrizen besser zu verstehen.
Wir benötigen:

\begin{lemma}
	\label{lemma:7.15}
	Sei $(V,\sk{\cdot,\cdot})$ ein endlich dimensionaler unitärer oder euklidischer $\KK$-Vektorraum und sei $F \in \End(V)$.
	\begin{enumerate}[(i)]
		\item Ist $F$ selbstadjungiert, also $F = F^*$, so ist jeder Eigenwert von $F$ reell.
		\item Ist $F$ unitär oder orthogonal, so gilt für jeden Eigenwert $\lambda$ von $F$:
		$\abs{\lambda}=1$.
	\end{enumerate}
\end{lemma}

\begin{beweis}
	\begin{enumerate}[(i)]
		\item Sei $\lambda$ ein Eigenwert von $F$ mit $F = F^*$ und sei $v \neq 0$ ein Eigenvektor zu $\lambda$.
		Nach Lemma~\ref{lemma:7.2} ist dann $v$ auch Eigenvektor zu $F^*$ zum Eigenwert $\ol{\lambda}$.
		Damit folgt $\lambda v = F(v) = F^*(v) = \ol{\lambda} v$, also $\lambda = \ol{\lambda}$.
		\item Da $F$ unitär bzw. orthogonal ist, gilt $\sk{F(v),F(v)} = \sk{v,v}$ für alle $v \in V$.
		Ist dann $v \in V \setminus \setzero$ ein Eigenvektor zum Eigenwert $\lambda$, so folgt $0 \neq \sk{v,v} = \sk{F(v),F(v)} = \sk{\lambda v,\lambda v} = \abs{\lambda}^2 \sk{v,v}$, also $\abs{\lambda}^2 = 1$, und dann auch $\abs{\lambda} = 1$. 
	\end{enumerate}
\end{beweis}

Wir kommen nun zu sehr wichtigen Folgerungen von Satz~\ref{satz:7.13}:

\begin{korollar}
	\label{kor:7.16}
	Sei $(V,\sk{\cdot,\cdot})$ ein endlich dimensionaler euklidischer $\RR$-Vektorraum und sei $F \in \End(V)$ selbstadjungiert, also $F = F^*$.
	Dann existiert eine Orthonormalbasis $B = \{v_1,\dots,v_n\}$ von $V$ mit
	\[
		A_B^F = \begin{pmatrix}
		\lambda_1 & & \\
		& \ddots & \\
		& & \lambda_n
		\end{pmatrix}.
	\]
	Insbesondere ist $F$ reell diagonalisierbar.
	
	Analog: Ist $A \in M(n \times n,\RR)$ mit $A = A^T$, so existiert eine orthogonale Matrix $O \in \oh(n)$ mit
	\[
		O^TAO = \begin{pmatrix}
		\lambda_1 & & \\
		& \ddots & \\
		& & \lambda_n
		\end{pmatrix}.
	\]
	Insbesondere ist $A$ reell diagonalisierbar.
\end{korollar}

\begin{beweis}
	Der Beweis folgt sofort aus Satz~\ref{satz:7.13}, Korollar~\ref{kor:7.14} und Lemma~\ref{lemma:7.15}. 
\end{beweis}

\begin{korollar}
	\label{kor:7.17}
	Sei $(V,\sk{\cdot,\cdot})$ ein endlich dimensionaler euklidischer $\RR$-Vektorraum und sei $F \in \oh(V)$ ein ortogonaler Endomorphismus.
	Dann existiert eine Orthonormalbasis $B = \{v_1,\dots,v_n\}$ von $V$ mit
	\[
		A_F^B = \enb{\begin{BMAT}(@)[6pt]{ccccccccccc}{ccccccccccc}
   	 -1 &        &    &   &        &   &               &                &        &               &  \\
		& \ddots &    &   &        &   &               &                &        &               &  \\
		&        & -1 &   &        &   &               &                &        &               &  \\
		&        &    & 1 &        &   &               &                &        &               &  \\
		&        &    &   & \ddots &   &               &                &        &               &  \\
		&        &    &   &        & 1 &               &                &        &               &  \\
		&        &    &   &        &   & \cos \alpha_1 & -\sin \alpha_1 &        &               &  \\
		&        &    &   &        &   & \sin \alpha_1 & \cos \alpha_1  &        &               &  \\
		&        &    &   &        &   &               &                & \ddots &               &  \\
		&        &    &   &        &   &               &                &        & \cos \alpha_m & -\sin \alpha_m \\
		&        &    &   &        &   &               &                &        & \sin \alpha_m & \cos \alpha_m
			\addpath{(9,0,|)uurrddll}
			\addpath{(6,3,|)uurrddll}
			\end{BMAT}} = (*).
	\]
	
	Analog: Ist $A \in \oh(n)$ orthogonal, so existiert eine orthogonale Matrix $O \in \oh(n)$ mit $O^TAO = (*)$.
\end{korollar}

\begin{beweis}
	Nach Lemma~\ref{lemma:7.15}(ii) ist jeder reelle Eigenwert von $F$ (bzw. $A$) $1$ oder $-1$, und jeder komplexe Eigenwert $\mu$ erfüllt $\abs{\mu}=1$.
	Dann gilt $\mu = \cos(\alpha) - i \sin(\alpha)$ für ein $\alpha \in [0,2\pi)$, und der Satz folgt sofort aus Satz~\ref{satz:7.13} und Korollar~\ref{kor:7.14}. 
\end{beweis}

Wir sehen also, dass jede orthogonale lineare Abbildung sich aus Drehungen geeigneter Teilebenen und Spiegelungen an geeigneten Achsen zusammensetzt (Beachte, dass die Matrix $-E_2$ eine Drehung um den Winkel $\pi$ ist).
 
\begin{anwendung}[Der Fall $n=3$]
	\label{anw:7.18}
	Ist $A \in M(3 \times 3,\RR)$ orthogonal, so existiert eine Orthonormalbasis $B = \{u,v,w\}$ von $\RR^3$, so dass die orthogonale Abbildung $F_A \colon \RR^3 \rightarrow \RR^3, x \mapsto Ax$ bezüglich $B$ eine Darstellungsmatrix der Form
	\[
		A_F^B = \begin{pmatrix}
			\pm 1 & 0 & 0 \\
		0 & \cos \alpha & -\sin \alpha \\
		0 & \sin \alpha & \cos \alpha
		\end{pmatrix} = O^T A O
	\]
	mit $\alpha \in [0,2\pi)$ und $O = (u,v,w)$ besitzt.
	
	Im Fall $+1$ ist $F_A$ eine Drehung mit dem Winkel $\alpha$ um die durch $u$ aufgespannte Drehachse.
	Im FAll $-1$ ist $F_A$ die Drehung mit Winkel $\alpha$ um die $u$-Achse, gefolgt von einer Spiegelung an der $v$-$w$-Ebene.
	Es gilt $\det(A) = 1$ genau dann, wenn $A$ eine reine Drehung ist, das heißt die Gruppe $\SO(3)$ besteht aus genau den Drehungen des $\RR^3$ durch geeignete Achsen durch den Nullpunkt.
\end{anwendung}

Wir schließen den Abschnitt mit einer geeigneten Anwendung von Satz~\ref{satz:7.3} und Satz~\ref{satz:7.13} für positiv definite Endomorphismen bzw. Matrizen.

\begin{satz}
	\label{satz:7.19}
	Sei $(V,\sk{\cdot,\cdot})$ ein endlich dimensionaler unitärer oder euklidischer $\KK$-Vektorraum und sei $F \in \End(V)$ selbstadjungiert.
	Dann gilt
	\[
		F \text{ ist positiv definit} \quad \iff \quad \text{Alle Eigenwerte von } F \text{ sind positiv}
	\]
	Analoges gilt für eine selbstadjungierte Matrix $A \in M(n \times n, \KK)$.
\end{satz}

\begin{beweis}
	Da $F = F^*$, sind alle Eigenwerte von $F$ reell und es existiert eine Orthonormalbasis $B = \{v_1,\dots,v_n\}$ von $V$ mit $A^B_F = \begin{pmatrix}
	\lambda_1 & & \\ & \ddots & \\ & & \lambda_n \end{pmatrix}$.
	\begin{description}
		\item[\bewrueck] Es gelte $\lambda_i > 0$ für alle $1 \leq i \leq n$.
		Ist dann $v \in V \setminus \setzero$ beliebig, so existiert  $x = (x_1,\dots,x_n) \in \KK^n$ mit $v = \sum_{i=1}^{n} x_i v_i$, und dann gilt
		\[
			\sk{F(v),v} = \sk{A_F^B x,x} = \sum_{i=1}^{n} \lambda_i x_i \ol{x_i} = \sum_{i=1}^{n} \lambda_i \abs{x_i}^2 > 0,
		\]
		da mindestens ein $x_i \neq 0$.
		Also ist $F$ positiv definit.
		\item[\bewhin] Sei $F$ positiv definit.
		Angenommen, es gäbe ein Eigenwert $\lambda$ von $F$ mit $\lambda \leq 0$.
		Ist dann $0 \in v \setminus \setzero$ ein Eigenvektor zu $\lambda$, so gilt
		\[
			\sk{F(v),v} = \sk{\lambda v,v} = \lambda \Underbrace{\sk{v,v}}{\geq 0} \leq 0.
		\]
		Widerspruch zu $F$ positiv definit. 
	\end{description}
\end{beweis}
\cleardoubleoddemptypage