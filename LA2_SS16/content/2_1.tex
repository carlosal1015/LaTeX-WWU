\chapter{Lineare Algebra II} % (fold)
\label{cha:2}
\setcounter{section}{0}
\section{Polynome und der Fundamentalsatz der Algebra}
\label{sec:2.1}

\begin{erinnerung}[Polynomring]
	\label{erinnerung:1.1}
	Sei $K$ ein Körper.
	Dann ist $K[T]$ gegeben als die Menge aller formalen Summen
	\[
		p(T) = \sum_{k=0}^{\infty} a_k T^k \text{ mit } a_k \in K \ \forall k \in \NN_0 \text{ und } \exists N \in \NN_0 \text{ mit } a_k = 0 \ \forall k > N.
	\]
	Wir schreiben dann auch $p(T) = \sum_{k=0}^{N} a_kT^k$.
	
	Ist $a_k=0$ für alle $k \in \NN_0$, so heißt $\sum_{k=0}^{\infty} a_kT^k$ das Nullpolynom, welches wir einfach mit $0$ oder $0(T)$ bezeichnen.
	Zwei Polynome $\sum_{k=0}^{\infty} a_kT^k$ und $\sum_{k=0}^{\infty} b_kT^k$ sind gleich genau dann, wenn $a_k = b_k$ für alle $k \in \NN_0$ gilt.
	Wir definieren Addition und Multipliation auf $K[T]$ wie folgt:
	\begin{align*}
		\sum_{k=0}^{\infty} a_kT^{k} + \sum_{k=0}^{\infty} b_kT^{k} &= \sum_{k=0}^{\infty} (a_k+b_k)T^{k} \\
		\enb{\sum_{k=0}^{\infty} a_kT^{k}} \cdot \enb{\sum_{k=0}^{\infty} b_kT^{k}} &= \sum_{k=0}^{\infty} \enb{\sum_{n=0}^{k} a_kb_{n-k}} T^{k}
	\end{align*}
	$(K[T],+,\cdot)$ ist ein kommutativer Ring mit Einselement -- der \Index{Polynomring} über $K$.
	
	$K[T]$ ist auch ein $K$-Vektorraum mit obiger Addition und mit der skalaren Multiplikation
	\[
		\lambda \cdot \enb{\sum_{k=0}^{\infty} a_kT^{k}} = \sum_{k=0}^{\infty} \lambda a_kT^{k}.
	\]
\end{erinnerung}

\begin{bemerkung}
	\label{bem:1.2}
	Ist $\mathcal{A}$ ein $K$-Vektorraum, der zusätzlich mit einer Multiplikation $*\colon \mathcal{A} \times \mathcal{A} \rightarrow \mathcal{A}$ ausgestattet ist, sodass $(\mathcal{A},+,*)$ ein Ring ist und für alle $\lambda \in K$ und $a,b\in \mathcal{A}$
	\[
		\lambda(a*b) = (\lambda a)*b = a * (\lambda b)
	\]
	gilt, so heißt $\mathcal{A}$ eine $\mathbf{K}$\textbf{-Algebra}. \index{K-Algebra}
\end{bemerkung}

Beispiele für $K$-Algebren sind $K[T]$ mit Addition und Multiplikation wie in \autoref{erinnerung:1.1} und $M(n\times n,K)$ mit kanonischer $K$-Vektorraum-Struktur und Matrixmultiplikation.

\begin{definition}[Grad]
	\label{def:1.3}
	Sei $p = p(T) = \sum_{k=0}^{\infty} a_k T^{k} \in K[T]$.
	Wir definieren den \Index{Grad} von $p$ wie folgt:
	
	Ist $p = 0$ das Nullpolynom, so setzen wir $\grad(p) = -\infty$.
	Ist $p \neq 0$ und ist $N \in \NN_0$ maximal mit $a_N \neq 0$, so setzen wir $\grad(p) = N$.
	$N$ ist dann minimal in $\NN_0$ mit $p(T) = \sum_{k=0}^{N} a_k T^{k}$.
\end{definition}

\begin{lemma}
	\label{lemma:1.4}
	Sind $p,q \in K[T]$, so gelten $\grad(p+q) \leq \max\{\grad(p),\grad(q)\}$ und $\grad(pq) = \grad(p)+\grad(q)$ mit $-\infty + N = N + (-\infty) := -\infty$ für alle $N \in \NN_0 \cup \{-\infty\}$.
\end{lemma}

\begin{beweis}
	Für $p,q \neq 0$ wurde dies in \textit{Lineare Algebra I} auf Blatt 5 gezeigt.
	Für $p = 0$ oder $q=0$ ist die Aussage klar. \qedhere
\end{beweis}

Wir kommen nun zur \Index{Polynomdivision}:
\begin{satz}[Division mit Rest]
	\label{satz:1.5}
	Seien $p,q \in K[T]$ mit $q=0$.
	Dann existieren eindeutig bestimmte Polynome $h,r \in K[T]$ mit $p=h\cdot q+r$ und $\grad(r) < \grad(q)$.
\end{satz}

\begin{beweis}
	Ist $\grad(q) > \grad(p)$, so setze $h=0$ und $r=p$.
	
	Sei also o.B.d.A. $n := \grad(p) \geq \grad(q) = m$, also $p(T) = \sum_{k=0}^{n} a_kT^{k}$ und $q(T) = \sum_{k=0}^{m} b_kT^{k}$ mit $a_n,b_m \neq 0$. Setze $h_1(T) := \frac{a_n}{b_m} T^{n-m}$ und
	\begin{equation}
		p_1 = p-h_1 \cdot q. \label{eq:1.5.1}
	\end{equation}
	Dann ist der $n$-te Koeffizient von $p_1$ gegeben durch $a_n - \frac{a_n}{b_m} \cdot b_m = 0$, denn $p_1 \cdot q(T) = \sum_{k=0}^{m} \frac{a_n}{b_m} b_k T^{k+(n-m)}$.
	Es folgt $\grad(p_1) < n = \grad(p)$.
	
	Falls $\grad(p_1) < m = \grad(q)$, setze $h = h_1, r= p_1$.
	Dann folgt $r = p-hq$ aus \eqref{eq:1.5.1}, also $p = hq+r$.
	Sonst vertauschen wir $p$ mit $p_1$ und schreiben wie oben
	\begin{align*}
		p_1 &= h_2 \cdot q + p_2 \text{ mit } \grad(p_2) < \grad(p_1) \\
		p_2 &= h_3 \cdot q + p_3 \text{ mit } \grad(p_3) < \grad(p_2) \\
		&\quad \vdots \\
		p_{l-1} &= h_l \cdot q + p_l \text{ mit } \grad(p_l) < \grad(p_{l-1})
	\end{align*}
	solange, bis $\grad(p_l) < m = \grad(q)$ gilt, was nach höchstens $n-(m+1)$ Schritten erreicht ist.
	
	Setze dann $h := h_1 + h_2 + \cdots + h_l$ und $r = p_l$.
	Dann folgt:
	\begin{align*}
		P &\stack{\eqref{eq:1.5.1}}{=} h_1 q + p_1 = h_1 q + h_2 q + p_2 =  (h_1+h_2+h_3)q + p_3 \\
		&\stack{}{=} \dots = (h_1 + \dots + h_l)q + p_l = hq + r
	\end{align*}
	mit $\grad(r) = \grad(p_l) < \grad(q)$.
	
	\textit{Zur Eindeutigkeit:} Sei $p = hq + r = h'q'+r'$ mit $\grad(r), \grad(r') < \grad(q) = m$.
	
	Angenommen, $h \neq h'$.
	Dann folgt $\grad(h-h') \geq 0$ und $0 = p-p = (h-h')q + (r-r')$, also $r-r' = (h-h') \cdot q$, und weiter:
	\begin{align*}
		m &> \grad(r-r') = \grad((h-h')\cdot q) \\
		&= \grad(h-h') + \grad(q) \geq m,
	\end{align*}
	was unmöglich ist.
	Es folgt $h = h'$ und dann auch $r = p-hq = p-h'q = r'$. \qedhere
\end{beweis}

\begin{satz}
	\label{satz:1.6}
	Sei $0 \neq p \in K[T]$ mit $\grad(p) = n$.
	Dann hat $p$ höchstens $n$ verschiedene Nullstellen in $K$.
	Sind $\lambda_1,\dots,\lambda_l$ die paarweise verschiedenen Nullstellen von $p$ in $K$ (mit $l \leq n$), so existieren eindeutig bestimmte natürliche Zahlen $n_1,\dots,n_l \in \NN$ mit $n_1+n_2+\dots+n_l = m \leq n$ und ein eindeutig bestimmtes Polynom $q \in K[T]$ mit $\grad(q) = n-m$ und
	\begin{equation}
		p(T) = q(T)(T-\lambda_1)^{n_1} \cdots (T-\lambda_l)^{n_l} \label{eq:1.6.1}
	\end{equation}
		
	und $q(\lambda) \neq 0$ für alle $\lambda \in K$.
\end{satz}

\begin{beweis}
	Induktion nach $n = \grad(p)$.
	\begin{description}
		\item[$n=0$:] Ist $\grad(p) = 0$, so ist $p(\lambda) = a_0 \in K$ für alle $\lambda$, und da $p \neq 0$, ist $a_0 \neq 0$, also besitzt $p$ keine Nullstelle in $K$.
		Es folgt $q = p$ (eindeutig!)
		\item[$n \rightsquigarrow n+1$:] Sei $\grad(p) = n+1$.
		Besitzt $p$ keine Nullstelle in $K$, so setze wieder $q = p$.
		Sonst sei $\lambda_0$ eine beliebige Nullstelle von $p$.
		Nach \autoref{satz:1.5} existieren eindeutig bestimmte Polynome $h,r \in K[T]$ mit 
		\[
			p(T) = h(T)(T-\lambda_0) + r(T) \text{ und } \grad(r) < 1.
		\]
		Dann folgt $r(T) = a$ für ein $a \in K$, und dann
		\[
			0 = p(\lambda_0) = h(\lambda_0)(\lambda_0-\lambda_0) + r(\lambda_0) = r(\lambda_0) = a.
		\]
		Also gilt $r = 0$ und $p(T) = h(T)(T-\lambda_0)$ mit $\grad(h) = n$.
		Nach Induktionsvoraussetzung gilt
		\[
			h(T) = q(T)\cdot (T-\lambda_1)^{\widetilde{n_1}} \cdots (T-\lambda_l)^{\widetilde{n_l}}
		\]
		mit $q(\lambda) \neq 0$ für alle $\lambda \in K$ und $\widetilde{n_1} + \dots + \widetilde{n_l} \leq n$.
		Ist dann $\lambda_0 \neq \lambda_1,\dots,\lambda_k$, so setze $n_0 = 1$ und $n_i = \widetilde{n_i}$ für $1 \leq i \leq l$.
		Dann folgt
		\[
			p(T) = q(T)(T-\lambda_0)^{n_0} \cdots (T-\lambda_l)^{n_l}.
		\]
		Ist $\lambda_0 = \lambda_i$ für ein $1 \leq i \leq l$, so setze $n_i = \widetilde{n_i} + 1$ und $n_j = \widetilde{n_j}$ für alle $j \neq i$.
		Dann folgt
		\[
			p(T) = q(T)(T-\lambda_1)^{n_1} \cdots (T-\lambda_l)^{n_l}.
		\]
		In beiden Fällen gilt $n_1+\dots+n_l \leq n+1$ bzw. $n_0 + \dots + n_l \leq n+1$.
		
		\textit{Zur Eindeutigkeit:}
		Ist $\lambda$ eine beliebige Nullstelle von $p$, so folgt $\lambda = \lambda_i$ für ein $1 \leq i \leq n$.
		Damit erhalten wir eine Zerlegung
		\begin{align*}
			p(T) &= (q(T) (T-\lambda_1)^{n_1} \cdots (T-\lambda_i)^{n_i-1} \cdots (T-\lambda_l)^{n_l})\cdot (T-\lambda_i) \\
			&= h(T) \cdot (T-\lambda_i).
		\end{align*}
		Nach \autoref{satz:1.5} ist diese Zerlegung eindeutig und nach Induktionsvoraussetzung ist die Zerlegung
		\[
			h(T) = q(T) (T-\lambda_1)^{n_1} \cdots (T-\lambda_i)^{n_i-1} \cdots (T-\lambda_i)^{n_l}
		\]
		von $h$ ebenfalls eindeutig.
		Dann ist auch die Zerlegung von $p$ eindeutig. \qedhere
	\end{description}
\end{beweis}

\begin{definition}[algebraische Vielfachheit]
	\label{def:1.7}
	Der Exponent $n_i, 1 \leq i \leq l$, in \eqref{eq:1.6.1} heißt die \Index{algebraische Vielfachheit} der Nullstelle $\lambda_i$ von $P$.
\end{definition}

\begin{bemerkung}
	\label{bem:1.8}
	Ein Polynom $p \in K[T]$ zerfällt genau dann in Linearfaktoren, wenn das Polynom $q$ in \autoref{satz:1.6} konstant ist.
	In diesem Fall gilt $q(\lambda) = a \neq 0$ für alle $\lambda \in K$ und
	\[
		p(T) = a(T-\lambda_1)^{n_1} \cdots (T-\lambda_l)^{n_l}
	\]
	mit $n_1+\cdots+n_l = n = \grad(p)$.
\end{bemerkung}

\begin{korollar}
	\label{kor:1.9}
	Sei $K$ ein Körper, sodass jedes nichtkonstante Polynom $p \in K[T]$ mindestens eine Nullstelle in $K$ besitzt.
	Dann zerfällt jedes (nichtkonstante) Polynom $p \in K[T]$ in Linearfaktoren.
\end{korollar}

\begin{beweis}
	Sei $p(T) = q(T)(T-\lambda_1)^{n_1} \cdots (T-\lambda_l)^{n_l}$ die Zerlegung wie in \autoref{satz:1.6}.
	Dann gilt $q(\lambda) \neq 0$ für alle $\lambda \in K$.
	Nach Voraussetzung folgt, dass $q$ konstant ist, das heißt es existiert ein $a \in K$ mit $q(\lambda) = a$ für alle $\lambda \in K$. \qedhere
\end{beweis}

\begin{bemerkung}
	\label{bem:1.10}
	Ein Körper $K$, der die Voraussetzung von \autoref{kor:1.9} erfüllt, heißt \Index{algebraisch abgeschlossen}.
	Die Körper $\RR$ und $\QQ$ sind nicht algebraisch abgeschlossen, denn $T^2+1$ besitzt keine Nullstelle in $\RR$ bzw. $\QQ$.
	Der Fundamentalsatz der Algebra besagt aber, dass $\CC$ algebraisch abgeschlossen ist:
\end{bemerkung}
\newpage
\begin{satz}[Fundamentalsatz der Algebra]
	\label{satz:1.11}
	Jedes nichtkonstante Polynom $p \in \CC[T]$ besitzt eine Nullstelle in $\CC$.
	Insbesondere folgt, dass jedes (nichtkonstante) komplexe Polynom in Linearfaktoren zerfällt. \index{Fundamentalsatz der Algebra}
\end{satz}

\begin{beweis}[nach Oswaldo Rio Branco \textsc{de Oliveira}]
	Der Beweis benutzt die folgenden Tatsachen:
	\begin{enumerate}[(i)]
		\item Ist $K \subseteq \CC$ beschränkt und abgeschlossen (also kompakt) und ist $f \colon K \rightarrow \RR$ stetig, so existiert $z_0 \in K$ mit $f(z_0) \leq f(z)$ für alle $z \in K$ ($f$ nimmt auf $K$ sein absolutes Minimum an).
		\item Für jedes $0 \leq \omega \in \CC$ besitzt die Gleichung $z^n = \omega$ eine Lösung $z \in \CC$ (sogar genau $n$ verschiedene Lösung).
		\item Für alle $a,b \in \CC$ gilt $|a+b|^2 - |a|^2 = 2 \cdot \Re(\ol{a},b)+|b|^2$.
	\end{enumerate}
	(i) wurde (oder wird) in der Analysis bewiesen, (ii) folgt mit Polarkoordinaten:
	Ist $\omega = |\omega|e^{i\varphi}$, so sind $z_0,\dots,z_{n-1}$ mit
	\[
		z_j := |\omega|^{\frac{1}{n}} e^{i \frac{\varphi+2\pi}{n}}, \quad j=0,1,\dots,n-1
	\]
	(die) Lösungen von $z^n = \omega$.
	
	(iii) folgt aus:
	\begin{align*}
		|a+b|^2 - |a|^2 &= (\ol{a+b})(a+b) - \ol{a}a \\
		&= \ol{a}a + \ol{a}b + \ol{b}a + \ol{b}b - \ol{a}a = \ol{a}b + \ol{\ol{a}b} + |b|^2 \\
		&=2 \cdot \Re(\ol{a}b) + |b|^2
	\end{align*}
	Wir kommen nun zum eigentlichen Beweis des Satzes:
	
	Da $p \in \CC[T]$ nicht konstant, gilt $n := \grad(p) \geq 1$ und $p(z) = \sum_{k=0}^{n} a_kz^k$ mit $a_n \neq 0$.
	Sei $a \in \CC$ beliebig gewählt und sei $S:= |p(a)|$.
	
	Es existiert ein $R > 0$ mit $|p(z)|> S$ für alle $z \in \CC$ mit $|z| \geq R$, denn:
	\begin{align*}
		|p(z)| &= \abs*{\sum_{k=0}^{n} a_k z^{k}} \stack{\Delta\text{Ungl.}}{\geq} |a_n| \cdot |z|^n - \sum_{k=0}^{n-1} |a_k| \cdot |z|^k \\
		&= \Bigg(\Underbrace{|a_n| - \sum_{k=0}^{n-1} |a_k| \cdot |z|^{k-n}}{\rightarrow |a_n| \text{ für } |z| \rightarrow \infty}\Bigg) |z|^n \xrightarrow{|z|\rightarrow \infty} \infty.
	\end{align*}
	Nach (i) existiert ein $z_0 \in B_R(0) = \{z \in \CC : |z| \leq R\}$ mit $|p(z_0)| \leq |p(z)|$ für alle $z \in B_r(z_0)$.
	Aber dann gilt auch $|p(z_0)| \leq |p(z)$ für alle $z \in \CC$, denn ist $z \notin z \in B_R(z_0)$, so gilt $|z|>R$ und dann $|p(z)| > S = |p(a)| \geq |p(z_0)|$ (wegen $|p(a)| = S$ folgt $a \in B_R(0)$).
	
	Durch übergang auf das Polynom $\tilde{p}(z) = p(z+z_0)$ wenn nötig, können wir o.B.d.A. $z_0 = 0$ annehmen.
	Es gilt dann
	\begin{equation}
		0 \leq |p(0)| \leq |p(z)| \text{ für alle } z \in \CC. \label{eq:1.11.1}
	\end{equation}
	Wir wollen zeigen, dass $p(0) = 0$ gilt.
	
	Setze $h(z) = p(z) - p(0)$.
	Dann ist $h(0) = 0$ und es existiert ein Polynom $q$ mit $q(0) \neq 0$ und ein $l \in \NN$ mit $h(z) = z^{l} q(z)$ für alle $z \in \CC$ (nutze \autoref{satz:1.6}).
	Damit folgt dann
	\[
		p(z) = p(0) + z^{l}q(z) \text{ für alle } z \in \CC \text{ mit } q(0) \neq 0.
	\]
	Sei $z = r \omega$ mit $r > 0, |\omega| = 1$.
	Dann folgt mit \eqref{eq:1.11.1}:
	\begin{align*}
		0 &\stack{}{\leq} |p(r \omega)|^2 - |p(0)|^2 = |p(0) + r^{l}\omega^{l}q(r\omega)|^2 - |p(0)|^2 \\
		&\stack{(iii)}{=} 2 \cdot \Re(\ol{p(0)} r^{l} \omega^{l} q(r\omega)) - |r^l \omega^{l} q(r\omega)|^2 \\
		&\stack{}{=} r^{l} (2\cdot \Re(\ol{p(0)} \omega^{l}q(r\omega))) - r^{2l} |q(rw)|^2
	\end{align*}
	Division mit $r^{l} > 0$ liefert
	\[
		0 \leq 2 \cdot \Re(\ol{p(0)}\omega^{l} q(r\omega)) - r^{l} |q(rw)|^2 =: f(r).
	\]
	Da alle Terme stetig in $r$ sind, folgt mit $r \rightarrow 0$
	\[
		0 \leq \lim\limits_{r \rightarrow 0} f(r) = f(0) = 2 \cdot \Re(\ol{p(0)} \omega^l q(0))
	\]
	und dann $0 \leq \Re(\ol{p(0)}q(0) \omega^l)$ für alle $\omega \in \CC$ mit $|\omega| = 1$.
	Mit $\omega = 1$ folgt $\Re(\ol{p(0)}q(0)) \geq 0$.
	Ist $\omega$ Lösung von $\omega^l = -1$, so folgt $\Re(\ol{p(0)}q(0)) \leq 0$.
	Zusammen folgt $\Re(\ol{p(0)}q(0))=0$.
	
	Ist nun $\omega$ Lösung von $\omega^l=i$, so folgt $0 \leq \Re(i \ol{p(0)} q(0)) = -\Im(\ol{p(0)}q(0))$, also $\Im(\ol{p(0)}q(0)) \leq 0$.
	
	Ist $\omega$ Lösung von $\omega^l = -i$, so folgt $0 \leq \Re(-i \ol{p(0)} q(0)) = \Im(\ol{p(0)}q(0))$.
	Zusammen folgt damit auch $\Im(\ol{p(0)}q(0)) = 0$.
	Also folgt $\ol{p(0)}q(0)=0$, und da $q(0) \neq 0$, folgt $\ol{p(0)} = 0$ und dann $p(0) = 0$. \qedhere
\end{beweis}
\newpage