%!TEX TS-program = xelatex
%!TEX TS-options = -shell-escape
%!TEX root = ../OpAlg2_SS16/operatoralgebren2.tex
\RequirePackage{fix-cm} 
\documentclass[a4paper, twoside, headsepline, index=totoc,toc=listof,toc=bibliography,toc=index, fontsize=10pt, cleardoublepage=empty, headinclude, DIV=12, BCOR=5mm, titlepage,draft]{scrartcl}
%!TEX root = ../AnaTopGeo_SS14/ana_top_geo.tex
\usepackage{scrtime} % KOMA, Uhrzeit ermoeglicht

%--Pakete zum "Programmieren"
% ======================================================================================
\usepackage{etoolbox}
\usepackage{letltxmacro}
\usepackage{ifthen}
% ======================================================================================

%--Farbdefinitionen und Grafiken (muss vor tikz geladen werden)
% ======================================================================================
\usepackage[usenames, table, x11names]{xcolor}
\definecolor{dark_gray}{gray}{0.45}
\definecolor{light_gray}{gray}{0.6}
\definecolor{fb10_blue}{cmyk}{0.8,0.4,0.13,0.07}
\usepackage[final]{graphicx}
\usepackage{adjustbox}
\newcommand{\cfbox}[2]{% coloured frame box
	\ifmmode
	\mathchoice{\adjustbox{cfbox=#1}{$\displaystyle#2$}}{\adjustbox{cfbox=#1}{$\textstyle#2$}}{\adjustbox{cfbox=#1}{$\scriptstyle#2$}}{\adjustbox{cfbox=#1}{$\scriptscriptstyle#2$}}
	\else
	\adjustbox{cfbox=#1}{#2}
	\fi
}
% ======================================================================================

%--Zum Zeichnen/ TikZ-Kram (vor polyglossia bzw. babel geladen werden)
% ======================================================================================
\usepackage{tikz}
\usepackage{tikz-cd}
\usetikzlibrary{external}
\tikzset{>=latex}
\usetikzlibrary{%
	shapes,
	arrows.meta,
	intersections,
	calc,
	3d,
	decorations.pathreplacing,decorations.markings,decorations.pathmorphing,
	angles,
	quotes,
}
\tikzexternalize[prefix=tikz/,up to date check=diff]
\pgfkeys{/pgf/images/include external/.code=\includegraphics{#1}}
\tikzset{external/system call={lualatex \tikzexternalcheckshellescape -halt-on-error -interaction=batchmode --shell-escape -jobname "\image" "\texsource"}}
\AtBeginEnvironment{tikzcd}{\tikzexternaldisable} % tikzexternalize fuer tikzcd deaktivieren, da inkompatibel
\AtEndEnvironment{tikzcd}{\tikzexternalenable}
\tikzset{% um Inkompatibilitaeten von quotes und polyglossia bzw. babel zu vermeiden
  every picture/.append style={
    execute at begin picture={\shorthandoff{"}},
    execute at end picture={\shorthandon{"}}
  }
}
\usepackage{pgfplots}
\usepgfplotslibrary{colormaps}
\newcommand*\circled[1]{\tikzexternaldisable\tikz[baseline=(char.base)]{\node[shape=circle,draw,inner sep=2pt] (char) {#1};}\tikzexternalenable}
% ======================================================================================



%-- Mathepakete etc.
% ======================================================================================
\usepackage[T1]{fontenc}
\renewcommand{\rmdefault}{zpltlf}
\usepackage{mathtools} % beinhaltet amsmath
\mathtoolsset{showonlyrefs,centercolon,showmanualtags}
\newtagform{brackets}[\textbf]{[}{]}
\usetagform{brackets}
\usepackage{fix-cm}
\usepackage[bbgreekl]{mathbbol}
\usepackage{amssymb,marvosym} 
\usepackage{nicefrac} % schräge Brüche
\usepackage{faktor}
\newcommand{\Faktor}[1]{\faktor[\textstyle]{#1}}
\usepackage{xfrac}
\usepackage{cancel}
\usepackage{mathdots} % Verbesserung von Punkten wie zB \ldots
\usepackage[bb=px]{mathalfa} % \mathbb als px font
\usepackage{centernot}
\usepackage{stackrel}
\DeclareSymbolFont{bbold}{U}{bbold}{m}{n}
\DeclareSymbolFontAlphabet{\mathbbold}{bbold}
\newcommand{\ind}{\mathbbold{1}} % charakteristische-Funktion-Eins
\def\mathul#1#2{\color{#1}\underline{{\color{black}#2}}\color{black}} %farbiges Untersteichen im Mathe-Modus
\renewcommand{\le}{\leqslant}
\renewcommand{\ge}{\geqslant}
% ======================================================================================


%-- Von xfrac erzeuge font warnings ignorieren
% ======================================================================================
\usepackage{silence}
\WarningFilter{latexfont}{Size substitutions with differences}
\WarningFilter{latexfont}{Font shape `U/bbold/m/n' in size}
% ======================================================================================


%-- Typographie/Polyglossia
% ======================================================================================
\usepackage[euler-digits]{eulervm} % vor fontspec laden!
\usepackage[no-math]{fontspec}
\usepackage{polyglossia} % moderner babel-ersatz
\setmainlanguage[spelling=new,babelshorthands=true]{german}
\shorthandoff{"}
\setotherlanguage{english}
\defaultfontfeatures{Mapping=tex-text, WordSpace={1.2}, Ligatures={Required,Common,Contextual},Extension=.otf} %


\setmainfont{TeXGyrePagellaX}[UprightFont=*-Regular,BoldFont=*-Bold,ItalicFont=*-Italic,BoldItalicFont=*-BoldItalic,ItalicFeatures={Style=Historic},Ligatures={Required,Common,Contextual,Historic}]
\setsansfont{texgyreadventor}[Scale=MatchUppercase, UprightFont=*-regular, BoldFont=*-bold, ItalicFont=*-italic, BoldItalicFont=*-bolditalic]
\setmonofont{SourceCodePro}[Scale=0.9,UprightFont=*-Regular, BoldFont=*-Semibold, ItalicFont=*-Light]
\usepackage{xltxtra}
\usepackage{fontawesome}
\usepackage[final]{microtype}
\usepackage[draft=false]{scrlayer-scrpage} 
\flushbottom
% ======================================================================================


%-- Aufzählungen
% ======================================================================================
\usepackage[shortlabels,inline]{enumitem}
\setlist[itemize,1]{label=\faCaretRight}
\setlist[enumerate]{font=\bfseries}
\setlist[description]{font=\normalfont\bfseries}
\usepackage{multicol}
% ======================================================================================


%-- Floats/Figures/Tabellen
% ======================================================================================
\usepackage{wrapfig}
\usepackage{float}
\usepackage[margin=10pt, font=small, labelfont={sf, bf}, format=plain, indention=1em]{caption}
\captionsetup[wrapfigure]{name=Abb. }
\usepackage{booktabs}
% ======================================================================================


%-- korrekte Anführungszeichen und Zitierbefehle
% ======================================================================================
\usepackage[autostyle,german=quotes,english=british]{csquotes}
% ======================================================================================


%--Indexverarbeitung
% ======================================================================================
\usepackage{makeidx}
\newcommand{\bet}[1]{\textbf{\emph{#1}}}
\newcommand{\Index}[1]{\bet{#1}\index{#1}}
\makeindex
\setindexpreamble{{\noindent\sffamily\small Die \emph{Seitenzahlen} sind mit Hyperlinks versehen und somit anklickbar} \par \bigskip}
\renewcommand{\indexpagestyle}{scrheadings}
% ======================================================================================


%-- Marginnotes/Todonotes/Footnotes
% ======================================================================================
\deffootnote[1.5em]{1.5em}{1.5em}{\textsuperscript{\thefootnotemark}\ }
\usepackage[fulladjust]{marginnote}
\renewcommand*{\marginfont}{\itshape\footnotesize}
\usepackage[textsize=small]{todonotes}
\usepackage{ragged2e}
\renewcommand*{\raggedleftmarginnote}{\RaggedLeft}
\renewcommand*{\raggedrightmarginnote}{\RaggedRight}
\LetLtxMacro{\oldtodo}{\todo}
\renewcommand{\todo}[2][]{\tikzexternaldisable\oldtodo[#1]{#2}\tikzexternalenable}
\LetLtxMacro{\oldmissingfigure}{\missingfigure}
\renewcommand{\missingfigure}[2][]{\tikzexternaldisable\oldmissingfigure[{#1}]{#2}\tikzexternalenable}
% ======================================================================================


% -- BibLaTeX
% ======================================================================================
\usepackage[%
	backend=biber,
	sortlocale=auto,
	natbib,
	hyperref,
	backref,
	style=alphabetic
	]%
{biblatex}
\renewcommand*{\mkbibnamelast}[1]{%
  \ifmknamesc{\textsc{#1}}{#1}}
\renewcommand*{\mkbibnameprefix}[1]{%
  \ifboolexpr{ test {\ifmknamesc} and test {\ifuseprefix} }
    {\textsc{#1}}
    {#1}}
\def\ifmknamesc{%
  \ifboolexpr{ test {\ifcurrentname{labelname}}
               or test {\ifcurrentname{author}}
               or ( test {\ifnameundef{author}} and test {\ifcurrentname{editor}} ) }}
\addbibresource{../!config/quellen.bib}
% ======================================================================================

%--Konfiguration von Hyperref und Cleveref
% ======================================================================================
\usepackage[hidelinks, pdfpagelabels,  bookmarksopen=true, bookmarksnumbered=true, linkcolor=black, urlcolor=SkyBlue2, plainpages=false,pagebackref, citecolor=black, hypertexnames=true, pdfauthor={Jannes Bantje}, pdfborderstyle={/S/U}, linkbordercolor=SkyBlue2, colorlinks=false,final,backref=false]{hyperref}
\usepackage[nameinlink,noabbrev]{cleveref}
\newcommand{\appendLink}[1]{#1\,\faExternalLink}
\newcommand{\hrefsym}[2]{\href{#1}{\texttt{\appendLink{#2}}}}
\newcommand{\hrefsymX}[2]{\href{#1}{\appendLink{#2}}}
\newcommand{\hrefsymmail}[2]{\href{#1}{\texttt{\faEnvelopeO\,#2}}}
\renewcommand{\url}[1]{\hrefsym{#1}{\nolinkurl{#1}}}
% ======================================================================================


% -- QR-Codes (hinter hyperref laden!)
% ======================================================================================
\usepackage{qrcode}
% ======================================================================================

%--Römische Zahlen
% ======================================================================================
\newcommand{\RM}[1]{\MakeUppercase{\romannumeral #1{}}}
% ======================================================================================

%-- Definition von diversen Mathe-Befehlen
% ======================================================================================
%!TEX root = mitschrift_main.tex

% -- Zum Finetuning von Befehlen
% ======================================================================================
\makeatletter
\newcommand{\raisemath}[1]{\mathpalette{\raisem@th{#1}}}
\newcommand{\raisem@th}[3]{\raisebox{#1}{$#2#3$}}
\makeatother
\makeatletter
\newcommand{\killDescendersM}[1]{\mathpalette{\killD@scendersM{#1}}}
\newcommand{\killD@scendersM}[2]{\raisebox{0pt}[\height][0pt]{$#2#1$}}
\makeatother
\DeclareRobustCommand{\minwidthbox}[2]{%
  \ifmmode
    \expandafter\mathmakebox
  \else
    \expandafter\makebox
  \fi
  [\ifdim#2<\width\width\else#2\fi]{#1}%
}
% ======================================================================================


%-- Klammerbefehle
% ======================================================================================
\DeclarePairedDelimiter{\abs}{\lvert}{\rvert}
\DeclarePairedDelimiter{\floor}{\lfloor}{\rfloor}
\DeclarePairedDelimiter{\ceil}{\lceil}{\rceil}
\DeclarePairedDelimiter\norm{\Vert}{\Vert}
\DeclarePairedDelimiter\enbrace{(}{)}
\DeclarePairedDelimiter\benbrace{[}{]}
\DeclarePairedDelimiter\bbenbrace{[\![}{]\!]}
\DeclarePairedDelimiter\lenbrace{<}{>}
\DeclarePairedDelimiter\angbrace{\langle}{\rangle}
\newcommand{\ssbrace}[1]{{\scriptscriptstyle\enbrace{#1}}}
\newcommand{\ssbbrace}[1]{{\scriptscriptstyle\benbrace{#1}}}
% ======================================================================================

%-- Mengen
% ======================================================================================
\newcommand\SetSymbol[1][]{\nonscript\:#1\vert\allowbreak\nonscript\:\mathopen{}}
\providecommand\given{} % to make it exist
\DeclarePairedDelimiterX\set[1]\{\}{\renewcommand\given{\SetSymbol[\delimsize]}#1}
% ======================================================================================

%-- Skalarprodukt (3 Varianten) 
% ======================================================================================
\DeclarePairedDelimiterX\sprod[2]{\langle}{\rangle}{#1\,\delimsize\vert\,#2}
\DeclarePairedDelimiterX\skal[2]{\langle}{\rangle}{#1\,,\,#2}
\makeatletter
\DeclareFontFamily{OMX}{MnSymbolE}{}
\DeclareSymbolFont{MnLargeSymbols}{OMX}{MnSymbolE}{m}{n}
\SetSymbolFont{MnLargeSymbols}{bold}{OMX}{MnSymbolE}{b}{n}
\DeclareFontShape{OMX}{MnSymbolE}{m}{n}{
    <-6>  MnSymbolE5
   <6-7>  MnSymbolE6
   <7-8>  MnSymbolE7
   <8-9>  MnSymbolE8
   <9-10> MnSymbolE9
  <10-12> MnSymbolE10
  <12->   MnSymbolE12
}{}
\DeclareFontShape{OMX}{MnSymbolE}{b}{n}{
    <-6>  MnSymbolE-Bold5
   <6-7>  MnSymbolE-Bold6
   <7-8>  MnSymbolE-Bold7
   <8-9>  MnSymbolE-Bold8
   <9-10> MnSymbolE-Bold9
  <10-12> MnSymbolE-Bold10
  <12->   MnSymbolE-Bold12
}{}
\let\llangle\@undefined
\let\rrangle\@undefined
\DeclareMathDelimiter{\llangle}{\mathopen}%
                     {MnLargeSymbols}{'164}{MnLargeSymbols}{'164}
\DeclareMathDelimiter{\rrangle}{\mathclose}%
                     {MnLargeSymbols}{'171}{MnLargeSymbols}{'171}
\makeatother
\DeclarePairedDelimiterX\sskal[2]{\llangle}{\rrangle}{#1\,,\,#2}
% ======================================================================================

%-- Abbildungsdefinition
% ======================================================================================
\newcommand{\mapdef}[5]{%
	\[
		\begin{array}{rcl}
			\textstyle #1 &\xrightarrow{\minwidthbox{#5}{2em}} & \textstyle #2 \\[0.5ex]
			\textstyle #3 &\xmapsto{\minwidthbox{\mbox{ }}{2em}} & \textstyle #4
		\end{array}
	\]
}
% ======================================================================================

%-- modifiziertes Stackrel 
% ======================================================================================
\newcommand{\StackText}[2]{\stackrel{\mbox{\scriptsize #1}}{#2}}
\newcommand{\StackTextClap}[2]{\stackrel{\mathclap{\mbox{\scriptsize #1}}}{#2}}
% ======================================================================================

%-- Blitz
% ======================================================================================
\newcommand{\light}{\text{\raisebox{-.3ex}{\Large\Lightning}}}
% ======================================================================================


%-- Underbrace u.Ä. als Befehl in LaTeX-Syntax (und ohne Spacingprobleme mit nachfolgenden Operatoren...)
% ======================================================================================
\newcommand{\Underbrace}[2]{{\underbrace{#1}_{#2}}}
\newcommand{\Underbracket}[2]{{\underbracket[0.7pt][2pt]{#1}_{#2}}}
\newcommand{\Overbracket}[2]{{\overbracket[0.7pt][2pt]{#1}^{#2}}}
% ======================================================================================


%-- Deklaration weiterer Operatoren (allgemein)
% ======================================================================================
\DeclareMathOperator{\re}{Re} % Realteil
\let\Re\relax
\DeclareMathOperator{\Re}{Re} % Realteil
\DeclareMathOperator{\im}{im} % Bild
\let\Im\relax
\DeclareMathOperator{\Im}{Im} % Bild
\DeclareMathOperator{\id}{id} % identische Abbildung
\DeclareMathOperator{\conj}{conj} % Konjugation
\DeclareMathOperator{\sgn}{sgn} % Signum
\DeclareMathOperator{\End}{End} % Endomorphismen
\DeclareMathOperator{\Hom}{Hom} % Homomorphismen
\DeclareMathOperator{\Iso}{Iso} % Isomorphismen
\DeclareMathOperator{\Aut}{Aut} % Automorphismen
\DeclareMathOperator{\Span}{span} % Span
\DeclareMathOperator{\coker}{coker} % Kokern
\DeclareMathOperator{\Tr}{Tr} % Spur,Trace
\DeclareMathOperator{\pr}{pr} % Projektion
\DeclareMathOperator{\diag}{diag} % Diagonalmatrix
\DeclareMathOperator{\Rg}{Rg} % Rang
\DeclareMathOperator{\const}{const} % konstante Abbildung
\DeclareMathOperator{\Spur}{Spur} % Spur
\DeclareMathOperator{\Arg}{Arg} % Argument
\DeclareMathOperator{\dist}{dist} % Distanz
\DeclareMathOperator{\supp}{supp} % Träger
\DeclareMathOperator{\Char}{char} % Charakteristik
% ======================================================================================


%-- Deklaration weiterer Operatoren (Differentiale etc.)
% ======================================================================================
\DeclareMathOperator{\grad}{grad} % Gradient
\DeclareMathOperator{\dive}{div} % Gradient
\DeclareMathOperator{\rot}{rot} % Rotation
\newcommand{\D}{\ensuremath{\mathrm{D}\mkern-1.0mu}} % Differential
\newcommand{\mathd}{\ensuremath{\mathrm{d}\mkern-1.0mu}} % äußere Ableitung
\newcommand{\Tmap}{\ensuremath{\mathrm{T}\mkern-0.85mu}} % Tangentialraum
\let\Tang\Tmap
\DeclareMathOperator{\Diff}{Diff}
\newcommand{\diff}[2]{\ensuremath{\frac{{\partial #1}}{{\partial #2}} }}
\newcommand{\diffd}[2]{\ensuremath{\frac{\mathd #1}{\mathd #2} }}
\DeclareMathOperator{\rank}{rank}
% ======================================================================================


%-- Deklaration weiterer Operatoren (Topologie)
% ======================================================================================
\newcommand*\interior[1]{\overset{\smash{\raisebox{-0.18ex}{$\scriptstyle\circ$}}}{#1}}
\newcommand{\sing}{{\raisemath{1.1pt}{\scriptscriptstyle\mathrm{sing}}}}
\newcommand{\pt}{\mathrm{pt}}
\DeclareMathOperator{\Zyl}{Zyl}
\newcommand{\rZyl}{\widetilde{\Zyl}}
\DeclareMathOperator{\Tel}{Tel}
\newcommand{\op}{\mathrm{op}}
\DeclareMathOperator{\Sp}{Sp}
\DeclareMathOperator{\Keg}{Keg}
\newcommand{\slashedi}{i\hspace{-3.5pt}/}
\newcommand{\cupp}{\smallsmile}
\newcommand{\capp}{\smallfrown}
\DeclareMathOperator*{\colim}{colim}
\DeclareMathOperator{\PD}{PD}
\newcommand{\lf}{\mathrm{lf}}
\DeclareMathOperator{\sig}{sig}
\DeclareMathOperator{\Tor}{Tor}
\DeclareMathOperator{\Ext}{Ext}
\DeclareMathOperator{\AW}{AW}
\DeclareMathOperator{\Proj}{Proj}
\DeclareMathOperator{\Gr}{Gr}
\DeclareMathOperator{\res}{res}
\DeclareMathOperator{\Spec}{Spec}
\DeclareMathOperator{\co}{co}
\DeclareMathOperator{\ch}{ch}
\DeclareMathOperator{\wOp}{w}
\DeclareMathOperator{\Ar}{Ar}
\newcommand{\actson}{\mathrel{\curvearrowright}}
\let\acts\actson
\let\action\actson
\DeclareMathSymbol{\bbDelta}{\mathord}{bbold}{"01}
\newcommand{\DDelta}{\bbDelta}
\DeclareMathOperator{\Star}{Star}
\DeclareMathOperator{\Link}{Link}
\DeclareMathOperator{\EPK}{EPK}
\DeclareMathOperator{\Vol}{Vol}
\newcommand{\cell}{{\raisemath{1.1pt}{\scriptscriptstyle\mathrm{cell}}}}
\DeclarePairedDelimiter{\homologieklasse}{\llbracket}{\rrbracket}
\newcommand{\rand}[1]{\ensuremath{\partial^{\scriptscriptstyle #1}}}
\DeclareMathOperator{\ab}{ab}
\DeclareMathOperator{\CW}{CW}
% ======================================================================================


%-- Deklaration von Operatoren (Liegruppen)
% ======================================================================================
\DeclareMathOperator{\GL}{GL}
\DeclareMathOperator{\SO}{SO}
\DeclareMathOperator{\Ad}{Ad}
\DeclareMathOperator{\ad}{ad}
\DeclareMathOperator{\On}{O}
\DeclareMathOperator{\Un}{U}
\DeclareMathOperator{\SU}{SU}
\DeclareMathOperator{\Mat}{Mat}
\DeclareRobustCommand{\Der}{\mathop{\mathfrak{der}}}
\DeclareMathOperator{\SL}{SL}
\DeclareMathOperator{\Graph}{Graph}
\DeclareMathOperator{\Int}{Int}
\DeclareRobustCommand{\intAlg}{\mathop{\mathfrak{int}}}
\DeclareMathOperator{\aut}{aut}
\DeclareMathOperator{\Rad}{Rad}
\DeclareMathOperator{\Nil}{Nil}
\DeclareMathOperator{\rad}{rad}
\DeclareMathOperator{\nil}{nil}
\DeclareMathOperator{\Ric}{Ric}
\DeclareMathOperator{\ric}{ric}
\newcommand{\bi}{\mathrm{bi}}
\DeclareMathOperator{\Isom}{Isom}
\DeclareMathOperator{\Sym}{Sym}
\newcommand{\opL}{\ensuremath{\mathrm{L}\mkern-0.6mu}}
% ======================================================================================

%-- Deklaration von Operatoren (Funktionalanalysis)
% ======================================================================================
\DeclareMathOperator{\tr}{tr}
\newcommand{\w}{\mkern1mu\mathrm{w}}
\newcommand{\sa}{\mathrm{sa}}
\newcommand{\vb}{\mathrm{v\mkern-2.5mu.b\mkern-1.5mu.}} % vollständig beschränkt
\newcommand{\so}{\mathrm{\mkern.3mu s\mkern-1.4mu.\mkern-.6mu o\mkern-1.7mu.}} % \newcommand{\so}{\mathrm{s.o.}}
\newcommand{\solim}{\so\text{-}\mkern-0.8mu\lim}
\newcommand{\wo}{\mathrm{w\mkern-3mu.\mkern-.4mu o\mkern-1.7mu.}}
\newcommand{\Top}[1]{\mathcal{T}_{\mkern-2.3mu #1}}
\newcommand{\weakT}[1]{\ensuremath{\mathcal{T}_{#1}^{\mkern+1.0mu\text{\raisebox{0.4ex}{$\mathrm{w}$}}}}}
\newcommand{\weakTstar}[1]{\ensuremath{\mathcal{T}_{#1}^{\mkern+1.0mu\text{\raisebox{0.4ex}{$\mathrm{w}$}}^*}}}
\newcommand{\TWeakStar}{\Top{\w^*}}
\newcommand{\TWeakOp}{\Top{\wo}}
\newcommand{\Tso}{\Top{\so}}
\newcommand{\finSub}{\subset\mkern-0.7mu \subset}
\DeclareMathOperator{\Inv}{Inv}
\newcommand{\simm}{{\hspace{-1.6pt}\raisemath{0.5pt}{\sim}}}
\newcommand{\plus}{{\hspace{-1.6pt}+}}
\DeclareMathOperator{\ev}{ev}
\DeclareMathOperator{\Alg}{Alg}
\DeclareMathOperator{\her}{her}
\newcommand{\subher}{\subset_{\her}}
\newcommand{\grenzw}[1]{\xrightarrow{\minwidthbox{#1}{1.4em}}}
\newcommand{\grenzwl}[1]{\xleftarrow{\minwidthbox{#1}{1.4em}}}
\newcommand{\grenzwIn}[1]{\grenzw{\raisemath{-2pt}{#1}}}
\newcommand{\MyTo}[1]{\tikzexternaldisable\mathbin{\tikz[baseline] \draw[-to,line width=.4pt] (0ex,0.94ex) -- (#1,0.94ex);}\tikzexternalenable}
\newcommand{\dlim}{%
    \mathchoice
      {\lim\limits_{\MyTo{4.2ex}}}% \displaystyle
      {\lim\limits_{\MyTo{2.8ex}}}% \textstyle
      {\lim\limits_{\MyTo{2.3ex}}}% \scriptstyle
      {\lim\limits_{\MyTo{2.3ex}}}% \scriptscriptstyle
}
\newcommand{\Dlim}{\killDescendersM{\dlim}}
\DeclareMathOperator{\sep}{sep}
\DeclareMathOperator{\diam}{diam}
\DeclareMathOperator{\conv}{conv}
\DeclareMathOperator{\Prim}{Prim}
\DeclareMathOperator{\hull}{hull}
\DeclareMathOperator{\red}{red}
\DeclarePairedDelimiterX\bra[1]{\langle}{\rvert}{#1\,}
\DeclarePairedDelimiterX\ket[1]{\lvert}{\rangle}{\,#1}
\DeclarePairedDelimiterX\bracket[2]{\langle}{\rangle}{#1\,\delimsize\vert\,#2}
\newcommand{\tensormax}{\mathbin{\otimes_{\max}}}
\newcommand{\tensormin}{\mathbin{\otimes_{\min}}}
\DeclareMathOperator{\Ped}{Ped}
\newcommand{\alg}{\mathrm{alg}}
\DeclareMathOperator{\CPC}{CPC}
\DeclareMathOperator{\CP}{CP}
\DeclareMathOperator{\UPC}{UPC}
\newcommand{\DeltaOp}{\mathbin{\Delta}}
\newcommand{\kernedP}{\mathcal{P}\mkern-2mu}
\newcommand{\Pinfty}{\kernedP_{\infty}}
\DeclareMathOperator{\Groth}{Groth}
\DeclareMathOperator{\rk}{rk}
\newcommand{\MvN}{\mathrm{MvN}}
% ======================================================================================

%-- Kategorien
% ======================================================================================
\DeclareMathOperator{\Mor}{Mor}
\DeclareMathOperator{\mor}{mor}
\DeclareMathOperator{\Obj}{Obj}
\DeclareMathOperator{\Ob}{Ob}
\newcommand{\TOP}{\textsc{Top}}
\newcommand{\HTOP}{\textsc{HTop}}
\newcommand{\VR}{\textsc{VR}}
\newcommand{\MOD}{\textsc{Mod}}
\newcommand{\Mod}[1]{#1\text{-}\MOD}
\newcommand{\MONOIDE}{\textsc{Monoide}}
\newcommand{\SET}{\textsc{Set}}
\newcommand{\MAN}{\textsc{Man}}
\newcommand{\GRUPPEN}{\textsc{Gruppen}}
\newcommand{\ABELGRUPPEN}{\textsc{Abel.Gruppen}}
\newcommand{\ABEL}{\textsc{Abel}}
\newcommand{\KAT}{\textsc{Kat}}
\newcommand{\FUN}{\textsc{Fun}}
\newcommand{\SIMP}{\textsc{Simp}}
\newcommand{\VEKT}{\textsc{Vekt}}
\newcommand{\CH}{\textsc{Ch}}
\newcommand{\CSTARUN}{C^*\text{-}\textsc{Alg}^{\raisemath{-2.5pt}{1}}}
\newcommand{\CSTAR}{C^*\text{-}\textsc{Alg}}
\newcommand{\AB}{\textsc{Ab}}
% ======================================================================================
% ======================================================================================



% -- theorem packages
% ======================================================================================
\usepackage{amsthm}
\usepackage{thmtools,thm-restate}
\usepackage{mdframed}
\renewcommand{\listtheoremname}{Übersicht aller Aussagen}
\usepackage{bookmark}
\bookmarksetup{open,numbered}
\makeatletter
\newcommand*{\theorembookmark}{%
  \bookmark[
    dest=\@currentHref,
    rellevel=1,
    keeplevel,
  ]{%
    \thmt@thmname\space\csname the\thmt@envname\endcsname
    \ifx\thmt@shortoptarg\@empty
    \else
      \space(\thmt@shortoptarg)%
    \fi
  }%
}   
\makeatother
% ======================================================================================

% -- Definition der einzelnen Theorem-Umgebungen
% ======================================================================================
\declaretheoremstyle[%
	headfont=\sffamily\bfseries,
	notefont=\normalfont\sffamily\scshape,
	bodyfont=\normalfont,
	headformat=\NUMBER\ \NAME\NOTE,
	headpunct=.,
	postheadspace=1em,
	spaceabove=15pt,spacebelow=10pt,
	shaded={bgcolor=gray!20},
	postheadhook=\theorembookmark]%
{mainstyle}
\declaretheoremstyle[%
	headfont=\sffamily\bfseries,
	notefont=\normalfont\sffamily\scshape,
	bodyfont=\normalfont,
	headformat=\NUMBER\ \NAME\NOTE,
	headpunct=.,
	postheadspace=1em,
	spaceabove=15pt,spacebelow=10pt,
	shaded={bgcolor=fb10_blue!20},
	postheadhook=\theorembookmark]%
{mainstyle_blue}
\declaretheoremstyle[%
	headfont=\sffamily\bfseries,
	notefont=\normalfont\sffamily\scshape,
	bodyfont=\normalfont,
	headformat=\NUMBER\ \NAME\NOTE,
	headpunct=.,
	postheadspace=1em,
	spaceabove=15pt,spacebelow=10pt,
	postheadhook=\theorembookmark]%
{mainstyle_unshaded}
\declaretheoremstyle[%
	headfont=\sffamily\bfseries,
	notefont=\normalfont\sffamily\scshape,
	bodyfont=\normalfont,
	headformat=\NUMBER\NAME\NOTE,
	headpunct=.,
	postheadspace=1em,
	spaceabove=15pt,spacebelow=10pt,
	% shaded={bgcolor=gray!20},
	postheadhook=\theorembookmark]%
{mainstyle_unnumbered}
\declaretheoremstyle[%
	headfont=\sffamily\bfseries,
	notefont=\normalfont\sffamily\scshape,
	bodyfont=\normalfont,
	headformat=swapnumber,
	headpunct=.,
	postheadspace=1em,
	spaceabove=15pt,spacebelow=10pt,
	shaded={bgcolor=gray!20},
	postheadhook=\theorembookmark,
	qed=\qedsymbol]%
{mainstyleB}
\declaretheoremstyle[%
	headfont=\bfseries\scshape,
	bodyfont=\normalfont,
	headpunct=:,
	postheadspace=1em,
	spacebelow=12pt,spaceabove=2pt,
	qed=\qedsymbol]%
{beweise}
\declaretheoremstyle[%
	headfont=\bfseries\scshape,
	bodyfont=\normalfont,
	headpunct=:,
	postheadspace=1em,
	spacebelow=12pt,spaceabove=2pt]%
{beweisskizze}
\declaretheoremstyle[%
	headfont=\sffamily\bfseries,
	bodyfont=\normalfont,
	headpunct=:,
	postheadspace=1em,
	spacebelow=10pt,spaceabove=10pt]%
{bemerkungen}
\declaretheorem[name=Definition,parent=section,style=mainstyle_blue]{definition}
\declaretheorem[name=Definition \& Proposition,refname=Proposition,sharenumber=definition,style=mainstyle_blue]{definitionP}
\declaretheorem[name=Definition,numbered=no,style=mainstyle_unnumbered]{definition*}
\declaretheorem[name=Theorem,sharenumber=definition,style=mainstyle]{theorem}
\declaretheorem[name=Theorem,numbered=no,style=mainstyle_unnumbered]{theorem*}
\declaretheorem[name=Proposition,sharenumber=definition,style=mainstyle,refname=Proposition]{proposition}
\declaretheorem[name=Lemma,sharenumber=definition,style=mainstyle]{lemma}
\declaretheorem[name=Satz,sharenumber=definition,style=mainstyle,refname=Satz]{satz}
\declaretheorem[name=Satz,sharenumber=definition,style=mainstyle_unshaded]{satzUnshaded}
\declaretheorem[name=Definition,sharenumber=definition,style=mainstyle_unshaded]{definitionUnshaded}
\declaretheorem[name=Satz,numbered=no,style=mainstyle_unnumbered]{satz*}
\declaretheorem[name=Korollar,sharenumber=definition,style=mainstyle,refname=Korollar]{korollar}
\declaretheorem[name=Korollar,sharenumber=definition,style=mainstyleB,refname=Korollar]{korollarB}
\declaretheorem[name=Frage,numbered=no,style=mainstyle_unnumbered]{frage}
\declaretheorem[name=Frage,sharenumber=definition,style=mainstyle_unshaded]{frageA}
\declaretheorem[name=Erinnerung,sharenumber=definition,style=mainstyle_unshaded]{erinnerungA}
\declaretheorem[name=Ausblick,sharenumber=definition,style=mainstyle_unshaded]{ausblick}
\declaretheorem[name=Konvention,sharenumber=definition,style=mainstyle]{konvention}
\declaretheorem[name=Notation,sharenumber=definition,style=mainstyle_unshaded]{notation}
\declaretheorem[name=Bemerkung,sharenumber=definition,style=mainstyle_unshaded,refname=Bemerkung]{bemerkung}
\declaretheorem[name=Bemerkung,numbered=no,style=mainstyle_unnumbered]{bemerkung*}
\declaretheorem[name=Beispiel,sharenumber=definition,style=mainstyle_unshaded,refname=Beispiel]{beispiel}
\declaretheorem[name=Beispiel,numbered=no,style=mainstyle_unnumbered]{beispiel*}
\declaretheorem[name=Exkurs,numbered=no,style=mainstyle_unnumbered]{exkurs*}
\declaretheorem[name=Beweis,numbered=no,style=beweise]{beweis}
\declaretheorem[name=Übung,numbered=no,style=bemerkungen]{uebung}
\declaretheorem[name=Erinnerung,numbered=no,style=bemerkungen]{erinnerung}

% english versions
\declaretheorem[name=Remark,sharenumber=definition,style=mainstyle_unshaded]{remark}
\declaretheorem[name=Remark,numbered=no,style=mainstyle_unnumbered]{remark*}
\declaretheorem[name=Example,sharenumber=definition,style=mainstyle_unshaded]{example}
\declaretheorem[name=Corollary,sharenumber=definition,style=mainstyle]{corollary}
\let\proof\relax
\declaretheorem[name=Proof,numbered=no,style=beweise]{proof}
\declaretheorem[name=Sketch of Proof,numbered=no,style=beweisskizze]{sketch}
% ======================================================================================

%--Inhaltsverzeichnis
% ======================================================================================
\usepackage[tocindentauto]{tocstyle}
\usetocstyle{KOMAlike}
% ======================================================================================

%-- Dinge, die erst am Ende getan werden dürfen
% ======================================================================================
\shorthandon{"}
\usepackage{ellipsis}
% ======================================================================================


\newcommand{\fach}{Operatoralgebren \RM{2}.}
\newcommand{\semester}{Sose 2016}
\newcommand{\homepage}{http://wwwmath.uni-muenster.de/u/wilhelm.winter/wwinter/operatoralgebren_II.html}

\newcommand{\prof}{Prof.\ Dr.\ Wilhelm Winter}
\publishers{\scalebox{10}{\Huge$^{\raisebox{2.2pt}{\text{\small\RM{2}.}}}\mkern-1.5mu C^*$}}
\input{../!config/mitschrift_headings.tex}

\begin{document}
\pagenumbering{Roman}
\maketitle
\begin{abstract}
\section*{Aktuelle Version verfügbar bei}
\newcommand{\dieBreite}{11cm}
\begin{minipage}{4cm}
	\qrcode[height=3.3cm, version=6]{https://gitlab.com/JaMeZ-B/LaTeX-WWU}
\end{minipage}
\hfill
\begin{minipage}{\dieBreite}
	% \includegraphics[height=0.6cm, keepaspectratio]{../!config/Bilder/wm_no_bg.pdf}
	\includegraphics[height=0.8cm, keepaspectratio]{../!config/Bilder/wm_no_bg.pdf}\\
	\url{https://gitlab.com/JaMeZ-B/LaTeX-WWU} \smallskip\\
	Das zentrale Repository des \enquote{\LaTeX-WWU}-Projekts befindet sich auf der Plattform GitLab.com.
	Neben der Koordination aller Beteiligten werden über diesen Dienst auch die PDFs gebaut, die in der Readme verlinkt sind.
\end{minipage}\\[1cm]
\begin{minipage}{4cm}
	\qrcode[height=3.3cm, version=6]{https://github.com/JaMeZ-B/latex-wwu}
\end{minipage}
\hfill
\begin{minipage}{\dieBreite}
	\includegraphics[height=0.6cm, keepaspectratio]{../!config/Bilder/github_octo.pdf}
	\includegraphics[height=0.6cm, keepaspectratio]{../!config/Bilder/GitHub_Logo.pdf}\\
	\url{https://github.com/JaMeZ-B/latex-wwu} \smallskip\\
	Die Entwicklung des \enquote{\LaTeX-WWU}-Projekts hat ursprünglich auf GitHub stattgefunden, ist mittlerweile aber zu GitLab gewechselt.
	Das GitHub-Repository wird stündlich automatisch aktualisiert, Merge-Requests werden aber nicht mehr entgegengenommen.
\end{minipage}\\[1cm]
% \begin{minipage}{4cm}
% 	\qrcode[height=3.3cm, version=6]{https://uni-muenster.sciebo.de/public.php?service=files&t=965ae79080a473eb5b6d927d7d8b0462}
% \end{minipage}
% \hfill
% \begin{minipage}{\dieBreite}
% 	\raisebox{-2pt}{\includegraphics[height=0.6cm, keepaspectratio]{../!config/Bilder/sciebo_logo.pdf}}
% 	\resizebox{!}{0.5cm}{\large \sffamily\textbf{sciebo}} {\sffamily\large die Campuscloud} \\
% 	\resizebox{\dieBreite}{!}{\footnotesize\url{https://uni-muenster.sciebo.de/public.php?service=files&t=965ae79080a473eb5b6d927d7d8b0462}}\smallskip\\
% 	Sciebo ist ein Dropbox-Ersatz der Hochschulen in NRW, der von der Uni Münster in leitender Position auf Basis der OpenSource-Software Owncloud aufgebaut wurde.
% \end{minipage}\\[1cm]
\hrule \mbox{ }\\[0.7cm]
\begin{minipage}{4cm}
	\qrcode[height=3.3cm, version=6]{\homepage}
\end{minipage}
\hfill
\begin{minipage}{\dieBreite}
	\resizebox{!}{0.5cm}{\large\sffamily\textbf{Vorlesungshomepage}}\\
	\resizebox{\dieBreite}{!}{\footnotesize\url{\homepage}}\smallskip\\
	Hier ist ein Link zur offiziellen Vorlesungshomepage.
\end{minipage}
\newpage
\section*{Vorwort --- Mitarbeit am Skript}
Dieses Dokument ist eine Mitschrift aus der Vorlesung \enquote{\fach, \semester}, gelesen von \prof. 
Der Inhalt entspricht weitestgehend dem Tafelanschrieb. 
Für die Korrektheit des Inhalts übernehme ich keinerlei Garantie! 
Für Bemerkungen und Korrekturen -- und seien es nur Rechtschreibfehler -- bin ich sehr dankbar. 
Korrekturen lassen sich prinzipiell auf drei Wegen einreichen: 
\begin{itemize}
	\item Persönliches Ansprechen in der Uni, Mails an \hrefsymmail{mailto:\mail}{\mail} (gerne auch mit annotieren PDFs) oder Kommentare auf \url{https://gitlab.com/JaMeZ-B/LaTeX-WWU}.
	\item \emph{Direktes} Mitarbeiten am Skript: Den Quellcode poste ich auf GitLab (siehe oben), also stehen vielfältige Möglichkeiten der Zusammenarbeit zur Verfügung:
	Zum Beispiel durch Kommentare am Code über die Website und die Kombination Fork und Merge-Request. 
	Wer sich verdient macht oder ein Skript zu einer Vorlesung, die ich nicht besuche, beisteuern will, dem gewähre ich gerne auch Schreibzugriff.
	
	Beachten sollte man dabei, dass dazu ein Account bei \url{gitlab.com} notwendig ist, der allerdings ohne Angabe von persönlichen Daten angelegt werden kann. 
	Wer bei GitLab (bzw. dem zugrunde liegenden Open-Source-Programm \enquote{\texttt{git}}) -- verständlicherweise -- Hilfe beim Einstieg braucht, dem helfe ich gerne weiter. 
	Es gibt aber auch zahlreiche empfehlenswerte Tutorials im Internet.\footnote{zB. \url{https://try.github.io/levels/1/challenges/1}, ist auf Englisch, aber dafür interaktiv}
	\item \emph{Indirektes} Mitarbeiten: \TeX-Dateien per Mail verschicken. 
	
	Dies ist nur dann sinnvoll, wenn man einen ganzen Abschnitt ändern möchte (zB. einen alternativen Beweis geben), da ich die Änderungen dann per Hand einbauen muss! Ich freue mich aber auch über solche Beiträge!
\end{itemize}
\section*{Anmerkung}
Innerhalb dieser Mitschrift wird man öfter den Ausdruck \enquote{Warum?} finden. Dies sind vom Dozenten bewusst weggelassene Details, die zu verstärktem Mitdenken beim Lesen 
animieren sollen. Oftmals sind dies schon aus vorherigen Semestern bekannte Sachverhalte. Nur an wenigen Stellen habe ich die fehlenden Details hinzugefügt.

\section*{Literatur}
\begin{itemize}[itemsep=0pt]
	\item G. J. \textsc{Murphy} \emph{C*-algebras and Operator Theory} \cite{Murphy}
	\item N. P. \textsc{Brown}, N. \textsc{Ozawa} \emph{C*-algebras and Finite-Dimensional Approximations} \cite{brown2008}
	\item M. \textsc{Rørdam}, F. \textsc{Larsen}, N. J. \textsc{Laustsen} \emph{An Introduction to K-theory for C*-Algebras} \cite{rordam2000}
	\item N. E. \textsc{Wegge-Olsen} \emph{K-theory and C*-algebras} \cite{wegge}
	\item B. \textsc{Blackadar} \emph{K-theory for Operator Algebras} \cite{blackadar1998}
	\item B. \textsc{Blackadar} \emph{Operator Algebras} \cite{blackadar2006}
	\item M. \textsc{Takesaki} \emph{Theory of Operator Algebras I.} \cite{takesaki2002}
	\item R.V. \textsc{Kadison}, J. R. \textsc{Ringrose} \emph{Fundamentals of the Theory of Operator Algebras Volumes I. \& II.} \cite{kadison1997-1} \& \cite{kadison1997-2}
	\item G.K. \textsc{Pedersen} \emph{C*-algebras and their Automorphism Groups} \cite{PedC*}
\end{itemize}
\end{abstract}

\tableofcontents
\cleardoubleoddemptypage

\pagenumbering{arabic}
\setcounter{page}{1}
\setcounter{footnote}{0}

\section{Tensorprodukte} % (fold)
\label{sec:1}

\begin{erinnerungA}[{name=[Tensorprodukt]}]
	Seien $V,W$ $\mathbb{K}$-Vektorräume.
	Sei $X$ ein weiterer $\mathbb{K}$-Vektorraum und $\phi \colon V\times W \to X$ bilinear.
	Dann heißt $(X,\varphi)$ \Index{Tensorprodukt} von $V$ und $W$, falls folgende universelle Eigenschaft gilt:
	\[
		\begin{tikzcd}[sep=large]
			V \times W \drar["\text{bil.}"'] \rar["\varphi"] & X \dar[dashed,"\exists ! \psi \text{ lin.} "] \\
			& U
		\end{tikzcd}
	\]
\end{erinnerungA}

\begin{definitionP}[label=propdef:12,{name=[{Konstruktion des algebraischen Tensorproduktes}]}]
	$(X,\varphi)$ wie oben existiert und ist eindeutig bis auf linearen Isomorphismus.
	Wir schreiben $V \odot W$ für $X$ und $v \otimes w$ für $\varphi(v,w)$.\marginnote{$\odot$ algebraisches Tensorprodukt}
\end{definitionP}
\begin{beweis}
	Sei $M$ der freie Vektorraum mit Basis $V \times W$, anders ausgedrückt
	\[
		M:= \set[\big]{f \colon V\times W \to \mathbb{K} \given \supp f \text{ endlich}}
	\]
	Sei nun 
	\[
		N := \Span \set*{
			\begin{array}{l}
				(v_1 +v_2,w) - \enbrace[\big]{(v_1,w) + (v_2,w)} \\
				(v,w_1+w_2) - \enbrace[\big]{(v,w_1) + (v,w_2)} \\
				(v,\lambda \cdot w) - \lambda \cdot (v,w), (\lambda \cdot v,w) - \lambda \cdot (v,w) 
			\end{array}
		\given 
		\begin{array}{l}
			v,v_1,v_2 \in V,\\
			w,w_1,w_2 \in W,\\
			\lambda \in \mathbb{K}
		\end{array}}
	\]
	Wir setzen $X := \sfrac{M}{N}$ und $\varphi(v,w) := (v,w)+N$.
	Der Rest ist eine einfache Übung.
\end{beweis}

\begin{proposition}[{name=[Tensorprodukt von Algebren]}]
	Seien $A,B$ $\mathbb{K}$-Algebren.
	Dann ist $A \odot B$ wieder eine $\mathbb{K}$-Algebra mit 
	\[
		(a \otimes b) (a' \otimes b') = aa' \otimes bb'
	\]
	Sind $A$ und $B$ $^*$-Algebren, dann ist $A \odot B$ wieder eine $^*$-Algebra mit 
	\[
		(a \otimes b)^* = a^* \otimes b^*
	\]
	Sind $A$ und $B$ kommutativ/unital, so auch $A \odot B$.
\end{proposition}
\begin{beweis}
	Betrachte $M$ wie in \autoref{propdef:12} und überprüfe, dass Multiplikation auf $\sfrac{M}{N}$ wohldefiniert und assoziativ ist.
	
	Alternativ: Die Elemente $(a,b) \in A \times B$ liefern eine bilineare Abbildung
	\mapdef{M_{(a,b)} \colon A \times B}{A \odot B}{(a',b')}{aa' \otimes bb'}{}
	Aus der universellen Eigenschaft erhält man $m_{(a,b)} \colon A \odot B \to A \odot B$ linear.
	Dies liefert wiederum eine bilineare Abbildung $m \colon A \times B \to \mathcal{L}(A \odot B, A \odot B)$ durch $(a,b) \mapsto m_{(a,b)}$.
	Nach der universellen Eigenschaft bekommt man nun eine lineare Abbildung $A \odot B \to \mathcal{L}(A \odot B, A\odot B)$.
	Daraus erhält man eine bilineare Abbildung
	\[
		\mu \colon A \odot B \times A \odot B \longrightarrow A \odot B
	\]
	Dann ist $(A \odot B,\mu)$ eine $\mathbb{K}$-Algebra.
\end{beweis}

\begin{bemerkung}[label=bem:14,{name=[{universelle Eigenschaft des Tensorproduktes von *-Algebren}]}]
	\begin{enumerate}[(i)]
		\item Seien $A,B$ $^*$-Algebren.
		Dann hat das Tensorprodukt $A \odot B$ als $^*$-Algebra die folgende universelle Eigenschaft:
		
		Falls $C$ eine $^*$-Algebra ist und $\pi_A \colon A \to C$, $\pi_B \colon B \to C$ $^*$-Homomorphismen sind, sodass für den Kommutator $\benbrace*{\pi_A(A), \pi_B(B)}=0$ gilt,\marginnote{$[A,B]$ besteht aus den Elementen der Form $ab -ba$} so existiert genau ein $^*$-Homomorphismus $\pi \colon A \odot B \to C$ mit $\pi(a \otimes b)= \pi_A(a) \cdot \pi_B(b)$ für alle $a \in A$, $b \in B$.
		\item Es seien $A,B,C,D$ $^*$-Algebren, $\pi_A \colon A \to C$ und $\pi_B \colon B \to D$ $^*$-Homomorphismen.
		Dann existiert genau ein $\pi \colon A \odot B \to C \odot D$ mit $\pi(a \otimes b)= \pi_A(a) \otimes \pi_B(b)$.
	\end{enumerate}
\end{bemerkung}

\begin{definitionP}[{name=[{Tensorprodukt von Hilberträumen}]}]
	Es seien $\mathcal{H}_1$, $\mathcal{H}_2$ Hilberträume.
	Dann ist $\mathcal{H}_1 \odot \mathcal{H}_2$ ein Prä-Hilbertraum mit
	\[
		\skal*{\xi \otimes \eta}{\xi' \otimes \eta'} := \skal*{\xi}{\xi'} \cdot \skal*{\eta}{\eta'} 
	\]
	Das \bet{Tensorprodukt von Hilberträumen}\index{Tensorprodukt!von Hilberträumen} ist nun definiert durch
	\[
		\mathcal{H}_1 \otimes \mathcal{H}_2 := \overline{\mathcal{H}_1 \odot \mathcal{H}_2}^{\skal*{\cdot}{\cdot}}
	\]
\end{definitionP}
\begin{beweis}
	Es seien $\xi \in \mathcal{H}_1$ und $\eta \in \mathcal{H}_2$.
	Nach der universellen Eigenschaft existiert eine lineare Abbildung $\tau_{(\xi,\eta)} \colon \mathcal{H}_1 \odot \mathcal{H}_2 \to \mathbb{C}$ mit $\tau_{{\xi,\eta}} \enbrace*{\xi' \otimes \eta'} = \skal*{\xi}{\xi'} \cdot \skal*{\eta}{\eta'}$.
	Die Abbildung $(\xi,\eta) \mapsto \overline{\tau_{{\xi,\eta}}}$ ist dann bilinear.
	Aus der universellen Eigenschaft erhalten wir eine antilineare Abbildung $\xi \otimes \eta \mapsto \tau_{(\xi,\eta)}$.
	Daraus erhält man eine Sesquilinearform 
	\[
		\skal*{\cdot}{\cdot } \colon \mathcal{H}_1 \odot \mathcal{H}_2 \longrightarrow \mathbb{C}
	\]
	Wir müssen noch zeigen, dass diese positiv definit ist:
	Sei $z = \sum_{i=1}^{n} \xi_i \otimes \eta_i \in \mathcal{H}_1 \odot \mathcal{H}_2$.
	Wähle eine Orthonormalbasis $\zeta_1, \ldots , \zeta_m$ für $\Span \set*{\eta_1, \ldots,\eta_n}$.
	Dann ist $z=\sum_{j=1}^{m} \xi_j' \otimes \zeta_j$ für $\xi_j' \in \mathcal{H}_1$ geeignet.
	Damit folgt 
	\[
		\skal*{z}{z} = \sum_{j=1}^{m} \norm*{\xi'_j}^2 \ge 0
	\]
	Insbesondere ist $z=0$ genau dann, wenn $\xi_1' = \ldots = \xi_m'=0$ ist, also $\skal*{z}{z}=0$ gilt.
\end{beweis}

\begin{bemerkung}[{name=[{Einbettung der Tensorprodukte beschränkter Operatoren}]},label=bem:16]
	Es existiert eine Einbettung $\iota \colon \mathcal{B}(\mathcal{H}_1) \odot \mathcal{B}(\mathcal{H}_2) \hookrightarrow \mathcal{B}(\mathcal{H}_1 \otimes \mathcal{H}_2)$ gegeben durch $(a \otimes b)(\xi \otimes \eta) = a(\xi) \otimes b (\eta)$. \emph{Siehe \cref{sub:injek_tensor_beschraenkte_operatoren}}
\end{bemerkung}

\begin{frageA}[{name=[{C*-Normen auf dem algebraischen Tensorprodukt}]}]
	Seien $A$ und $B$ $C^*$-Algebren.
	Die spannende Frage ist nun:
	\begin{center}
		\fbox{{\large¿}\,Was für $C^*$-Normen existieren auf $A \odot B$\,{\large?}}
	\end{center}
\end{frageA}

\begin{bemerkung}[{name=[Tensorprodukt aus C*-Norm auf dem algebraischen Tensorprodukt]}]
	Falls $\gamma \colon A \odot B \to \mathbb{R}^+$ eine $C^*$-Norm ist, so ist $A \otimes_\gamma B := \overline{A \odot B}^\gamma$ eine $C^*$-Algebra.\todo{RevChap 1}
\end{bemerkung}

\begin{proposition}[{name=[{minimale C*-Norm auf dem Tensorprodukt}]},label=prop:19]
	Es seien $A$ und $B$ $C^*$-Algebren.
	\begin{enumerate}[(i)]
		\item \label{enum:19:1}Für $^*$-Darstellungen $\pi_A \colon A \to \mathcal{B}(\mathcal{H}_1)$, $\pi_B \colon B \to \mathcal{B}(\mathcal{H}_2)$ definiert 
		\[
			(\pi_A \otimes \pi_B)(a \otimes b) := \iota\enbrace[\big]{\pi_A(a) \otimes \pi_B(b)}
		\]
		eine $^*$-Darstellung $\pi_A \otimes \pi_B \colon A \odot B \to \mathcal{B}(\mathcal{H}_1 \otimes \mathcal{H}_2)$.
		Falls $\pi_A$ und $\pi_B$ treu sind, so ist auch $\pi_A \otimes \pi_B$ treu.
		\item Die Abbildung $\norm*{\cdot }_{\min} \colon A \odot B \to \mathbb{R}^+$, gegeben durch
		\[
			\norm[\big]{\textstyle\sum\nolimits_{i=1}^{n} a_i \otimes b_i}_{\min} := \sup \set[\Big]{\norm[\big]{(\pi_A \otimes \pi_B)\enbrace[\big]{\textstyle\sum\nolimits_{i=1}^{n} a_i \otimes b_i}} \given \pi_A, \pi_B \text{ $^*$-Darstellungen} }
		\]
		ist eine $C^*$-Norm.
	\end{enumerate}
\end{proposition}
\begin{beweis}
	\begin{enumerate}[(i)]
		\item Nach \autoref{bem:14} und \autoref{bem:16} existiert genau eine $^*$-Darstellung 
		\[
			\pi_A \otimes \pi_B = \iota \circ (\pi_A \otimes \pi_B)
		\]
		Seien $\pi_A$ und $\pi_B$ treu und $0 \neq c \in \ker (\pi_A \otimes \pi_B)$.
		Es gilt $c= \sum_{i=1}^{n} a_i \otimes b_i$ für geeignete $a_i$, $b_i$ und $n$.
		Ohne Einschränkungen können wir annehmen, dass die $b_i$ linear unabhängig sind.
		Da $\pi_B$ treu ist, sind auch die $\pi_B(b_i)$ linear unabhängig.
		Für $\xi \in \mathcal{H}_1$ wählen eine Orthonormalbasis $\set*{\xi_1, \ldots ,\xi_m}$ für $\Span \set*{\pi_A(a_i) \xi \given i=1,\ldots ,n} \subset \mathcal{H}_1$.
		Es existieren also $\lambda_{ij} \in \mathbb{C}$ für $i=1,\ldots ,n$, $j=1,\ldots ,m$ mit 
		\(
			\pi_A(a_i) \xi = \sum_{j=1}^{m} \lambda_{ij} \cdot \xi_j
		\)
		für $i=1,\ldots ,n$.
		Für $\eta \in \mathcal{H}_2$ gilt
		\begin{align}
			0 = \enbrace*{\pi_A \otimes \pi_B}(c) \enbrace*{\xi \otimes \eta} = \sum_{i=1}^{n} \pi_A(a_i) \xi \otimes \pi_B(b_i) \eta &= \sum_{i=1}^{n} \enbrace*{\sum_{j=1}^{m} \lambda_{ij} \cdot \xi_j} \otimes \pi_B(b_i) \eta \\
			&= \sum_{j=1}^{m} \xi_j \otimes \enbrace*{\sum_{i=1}^{n} \lambda_{ij} \cdot \pi_B(b_i) \eta }
		\end{align}
		Es folgt, dass $\sum_{i=1}^{n} \lambda_{ij} \cdot \pi_B(b_i) \eta=0$ für $j=1,\ldots ,m$ und $\eta \in \mathcal{H}_2$.
		Damit sind die $\pi_B(b_i)$ linear abhängig oder $\lambda_{ij}=0$.
		Ersteres kann nicht gelten, also muss $\pi_A(a_i)\xi=0$ für $\xi \in \mathcal{H}_1$ gelten, also $\pi_A(a_i)=0$.
		Folglich ist $c=0$ im Widerspruch zur Annahme.
		\item $C^*$-Halbnorm: Übung!
		Wir zeigen, dass $\norm*{\cdot }_{\min}$ endlich ist: 
		\begin{align}
			\norm*{\enbrace*{\pi_A \otimes \pi_B} \enbrace*{\sum\nolimits_{i=1}^{n} a_i \otimes b_i}} &= \norm*{\sum_{i=1}^{n} \enbrace*{\pi_A(a_i) \otimes \ind_{\mathcal{H}_1} } \enbrace*{\ind_{\mathcal{H}_2} \otimes \pi_B(b_i)}}\\
			&\le \sum_{i=1}^{n} \norm*{\pi_A(a_i) \otimes \ind_{\mathcal{H}_1}} \norm*{\ind_{\mathcal{H}_2} \otimes \pi_B(b_i)} \\
			\intertext{$\pi_A(\cdot ) \otimes \ind_{\mathcal{H}_1}$ und $\ind_{\mathcal{H}_2} \otimes \pi_B(\cdot )$ sind als $^*$-Darstellungen kontraktiv, also}
			&\le \sum_{i=1}^{n} \norm*{a_i} \norm*{b_i} < \infty
		\end{align}
		$\norm*{\cdot }_{\min}$ ist eine Norm nach (i), denn $A$ und $B$ besitzen treue Darstellungen.\qedhere
	\end{enumerate}
\end{beweis}

\begin{definition}[{name=[{minimales Tensorprodukt}]}]
	\[
		\overline{A \odot B}^{\norm*{\cdot }_{\min}} =: A \otimes_{\min}B \enspace(= {A \otimes B})
	\]
	heißt \bet{minimales} (oder \bet{räumliches} oder \bet{spatiales}) \bet{Tensorprodukt}\index{minimales Tensorprodukt} von $A$ und $B$.
\end{definition}

\begin{satz}[{name=[{Zerlegung von Darstellungen des algebraischen Tensorproduktes}]},label=satz:111]
	Seien $A,B$ $C^*$-Algebren und $\pi \colon A \odot B \to \mathcal{B}(\mathcal{H})$ eine nichtdegenerierte Darstellung.
	Dann existieren eindeutig bestimmte nichtdegenerierte Darstellungen $\pi_A \colon A \to \mathcal{B}(\mathcal{H})$ und $\pi_B \colon B \to \mathcal{B}(\mathcal{H})$ mit 
	\[
		\pi(a \otimes b) = \pi_A(a) \cdot \pi_B(b)=\pi_B(b) \cdot \pi_A(a)
	\]
	Falls $\pi$ eine Faktordarstellung ist (das heißt $\overline{\pi(A \odot B)}^{\wo}$ ist ein Faktor), so auch $\pi_A$ und $\pi_B$.
\end{satz}
\begin{beweis}
	Falls $A$ und $B$ unital sind, so sind $\pi_A$, $\pi_B$ gegeben durch $\pi_A(a) = \pi(a \otimes \ind_B)$ bzw. $\pi_B(b) = \pi(\ind_A \otimes b)$.
	Im allgemeinen Fall seien $(h_\lambda), (k_\mu)$ approximative Einsen.
	Setze 
	\begin{align}
		\pi_A(a) &:= \solim \pi(a \otimes k_\mu) \\
		\pi_B(b) &:= \solim \pi(h_\lambda \otimes b)
	\end{align}
	Wir zeigen, dass diese Limiten existieren:
	Für $\xi \in \mathcal{H}$ und $a \in A_+$ definiert $b \mapsto \skal*{\xi}{\pi(a \otimes b) \xi}$ ein positives Funktional $f_a$ auf $B$ (beschränkt nach \cite[Satz 5.3]{OpAlg}).
	Damit existiert
	\[
		\lim_\mu f_a(k_\mu) = \lim_\mu \skal*{\xi}{\pi(a \otimes k_\mu)\xi}
	\]
	Für $a \in A_+$, $\mu \le \mu'$ gilt 
	\begin{align}
		\norm[\Big]{\enbrace[\big]{\pi(a \otimes k_{\mu'}) - \pi (a \otimes k_\mu)}\xi}^2 = \skal[\Big]{\xi}{\pi \enbrace[\big]{a^2 \otimes \enbrace*{k_{\mu'} - k_\mu}^2} \xi}
		&\le \skal[\Big]{\xi}{ \pi \enbrace[\big]{a \otimes \enbrace*{k_{\mu'}- k_\mu}} \xi} \\
		&= f_a(k_{\mu'}) - f_a(k_\mu) < \varepsilon
	\end{align}
	für $\mu, \mu'$ groß genug.
	Also existiert $\pi_A(a)$ für $a \in A_+$. 
	Mit Linearität folgt, dass $\pi_A$ auch insgesamt wohldefiniert ist.
	Dass $\pi_A, \pi_B$ $^*$-Darstellungen sind mit den gewünschten Eigenschaften, verbleibt als Übung!
	
	Es gilt nach dem Bikommutantensatz
	\[
		\begin{tikzcd}
			\pi_A(A)\dar[phantom,"\subset", sloped] \rar[phantom,"\subset", sloped] & \pi_B(B)' \dar[equals] \\
			\pi_A(A)'' \rar[phantom,"\subset", sloped] & \pi_B(B)''
		\end{tikzcd}
	\]
	Damit folgt $\pi_A(A)' \cap \pi_A(A)'' \subset \pi_A(A)' \cap \pi_B(B)' \subset \pi(A \odot B)'$.
	Weiter haben wir $\pi_A(A) \subset \overline{\pi(A \odot B)}^{\so} = \pi(A \odot B)''$, also gilt $\pi_A(A)'' \subset \pi(A \odot B)''$, womit weiter folgt
	\[
		\mathbb{C} \cdot \ind_{\mathcal{H}} \subset \pi_A(A)' \cap \pi_A(A)'' \subset \pi(A \odot B)' \cap \pi(A \odot B)'' = \mathbb{C} \cdot \ind_{\mathcal{H}}\qedhere
	\]
\end{beweis}

\begin{korollar}[{name=[{Submultiplikativität bezüglich des Tensorproduktes}]},label=kor:112]
	Seien $A,B$ $C^*$-Algebren, $\gamma$ eine $C^*$-Halbnorm auf $A \odot B$.
	Dann gilt $\gamma(a \otimes b) \le \norm*{a} \cdot \norm*{b}$ für $a \in A$, $b	\in B$.
	$\gamma$ besitzt eine eindeutige Fortsetzung zu einer $C^*$-Halbnorm auf $A^\simm \odot B^\simm$.
\end{korollar}
\begin{beweis}
	Setze $N := \set*{x \in A \odot B \given \gamma(x)=0} \lhd A \odot B$.
	$\gamma$ induziert eine $C^*$-Norm $\overline{\gamma}$ auf $\sfrac{A \odot B}{N}$.
	Damit ist $C := \overline{\sfrac{A \odot B}{N}}^{\overline{\gamma}}$ eine $C^*$-Algebra.
	$C$ besitzt eine nichtdegenerierte treue Darstellung.
	Wir erhalten $\pi \colon A \odot B \to \mathcal{B}(\mathcal{H})$ mit $\norm*{\pi(X)}= \overline{\gamma}(x +N) =\gamma(x)$ für $x \in A \odot B$.
	Weiter gilt
	\[
		\gamma(a \otimes b) = \norm*{\pi(a \otimes b)} \le \norm*{\pi_A(a)} \cdot \norm*{\pi_B(b)} \le \norm*{a} \cdot \norm*{b}
	\]
	Seien $\tilde{\pi}_A \colon A^\simm \to \mathcal{B}(\mathcal{H})$, $\tilde{\pi}_B \colon B^\simm \to \mathcal{B}(\mathcal{H})$ die Fortsetzungen auf den Unitalisierungen.
	Dann definiert $\tilde{\pi}(a \otimes b) := \tilde{\pi}_A(a) \cdot \tilde{\pi}_B(b)$ eine Fortsetzung $\tilde{\pi} \colon A^\simm \odot B^\simm \to \mathcal{B}(\mathcal{H})$ von $\pi$.
	($\tilde{\pi}$ existiert nach \autoref{bem:14}).
	$\tilde{\gamma}(\cdot ) := \norm*{\tilde{\pi}(\cdot)}$ ist eine Fortsetzung von $\gamma$.
	Wir müssen die Eindeutigkeit zeigen:
	\begin{align}
		\tilde{\gamma}(x) = \norm*{\tilde{\pi}(x)} = \norm*{\ind_{\mathcal{H}} \cdot \tilde{\pi}(x)} &= \norm*{\solim \pi\enbrace[\big]{h_\lambda^{\sfrac{1}{2}} \otimes k_\mu^{\sfrac{1}{2}}} \cdot \tilde{\pi}(x)} \\
		&= \lim_{\lambda,\mu} \norm[\big]{\tilde{\pi}(x^*) \pi \enbrace*{h_\lambda \otimes k_\mu} \tilde{\pi}(x)}^{\sfrac{1}{2}} \\
		&= \lim_{\lambda,\mu} \norm[\big]{\pi \enbrace*{x^* (h_\lambda \otimes k_\mu) x}}^{\sfrac{1}{2}} \\
		&= \lim_{\lambda,\mu} \gamma \enbrace[\big]{x^*(h_\lambda \otimes k_\mu)x}
	\end{align}
	Der letzte Ausdruck ist unabhängig von $\pi$.
\end{beweis}

\begin{definitionP}[{name=[{Max-Norm}]}]
	Die Abbildung $\norm*{\cdot }_{\max} \colon A \odot B \to \mathbb{R}^+$ gegeben durch
	\[
		\norm*{x}_{\max} := \sup \set[\big]{\norm*{\pi(x)} \given \pi \text{ ist $^*$-Darstellung von } A \odot B}
	\] 
	ist eine $C^*$-Norm auf $A \odot B$.
	Für jede weitere $C^*$-Norm $\gamma$ gilt $\gamma(x) \le \norm*{x}_{\max}$ für $x \in A \odot B$.
\end{definitionP}
\begin{beweis}
	$\norm*{\cdot}_{\max}$ ist submultiplikativ, also ist $\norm*{x}_{\max}$ endlich für alle $x \in A \odot B$.
	Damit ist dann auch klar, dass $\norm*{\cdot}_{\max}$ eine $C^*$-Halbnorm ist.
	Es seien nun $\pi_A, \pi_B$ treue Darstellungen von $A$ beziehungsweise $B$.
	Dann ist $\pi_A \otimes \pi_B$ eine treue Darstellung von $A \odot B$ nach \autoref{prop:19} \ref{enum:19:1}.
	Damit ist $\norm*{\cdot}_{\max}$ eine $C^*$-Norm und auch die Maximalität folgt.
\end{beweis}

\begin{definition}[{name=[{maximales Tensorprodukt}]}]
	$A \tensormax B := \overline{A \odot B}^{\norm*{\cdot }_{\max}}$ heißt \Index{maximales Tensorprodukt} von $A$ und $B$. 
\end{definition}

\begin{bemerkung}
	\begin{enumerate}[(i)]
		\item Sei $\gamma$ eine $C^*$-Norm auf $A \odot B$.
		Dann erhalten wir eine Abbildung
		\[
			\begin{tikzcd}
				A \tensormax B \rar[two heads] & A \otimes_\gamma B
			\end{tikzcd}
		\]
		denn $\id \colon A \odot B \to A \odot B$ ist normvermindernd, setzt sich also fort auf $A \tensormax B$.
		Das Bild ist eine dichte $C^*$-Unteralgebra, also alles.
		\item Sind $A,B$ unital, so ist $A \otimes_\gamma B$ unital und es gibt Inklusionen
		\begin{align}
			A &\xhookrightarrow{\hspace{1.4em}} A \mathop{\otimes_\gamma} B , \qquad a \longmapsto a \otimes \ind_B \\
			B &\xhookrightarrow{\hspace{1.4em}} A \mathop{\otimes_\gamma} B , \qquad b \longmapsto \ind_A \otimes b
		\end{align}
		\item Sind $A$, $B$ unital, dann ist $A \tensormax B \cong C^* \enbrace*{A,B \mid \benbrace*{A,B}=0}$ (universelle $C^*$-Algebra, Übung!)
		\item $\tensormax $ und $\tensormin$ sind assoziativ und kommutativ (warum?).
	\end{enumerate}
\end{bemerkung}

\begin{definition}[{name=[{Zustände auf dem algebraischen Tensorprodukt}]},label=def:116]
	Für $C^*$-Algebren $A$, $B$ setzen wir 
	\[
		(A \odot B)^* := \set[\big]{f \colon A \odot B \to \mathbb{C} \text{ linear}}
	\]
	$f \in (A \odot B)^*$ heißt positiv, falls $f(x^* x)\ge 0$ für $x \in A \odot B$.
	Wir setzen
	\[
		\norm*{f} := \sup \set*{f(a \otimes b) \given a \in A^1_+, b \in B_+^1}
	\]
	Wir definieren die \bet{Zustände}\index{Zustände auf dem algebraischen Tensorprodukt} auf $A \odot B$ durch $S(A \odot B) := \set[\big]{f \in (A \odot B)^* \given \norm*{f}=1}$.
\end{definition}

\begin{proposition}
	Jedes positive $f \in (A \odot B)^*$ besitzt eine positive Fortsetzung $f^\sim \in \enbrace*{A^\simm \odot B^\simm}^*$ mit 
	\[
		\norm*{f}= \norm*{f^\sim} = \tilde{f}(\ind_{A^\simm} \otimes \ind_{B^\simm}) < \infty
	\]
\end{proposition}
\begin{beweis}
	Sei $(h_\lambda)_{\Lambda} \subset A$ eine approximative Eins. 
	Für $b \in B_+$ setze $f_b(a) := f (a \otimes b)$.
	Dann ist $f_b \in A^*$ positiv und insbesondere beschränkt.\todo{Ref}
	Wir setzen
	\[
		\hat{f} \enbrace*{\ind_{A^\simm} \otimes b} := \Underbracket{\lim_\lambda f(h_\lambda \otimes b)}{f^\sim_b (\ind_{A^\simm})} = \sup_\lambda f(h_\lambda \otimes b) < \infty
	\]
	Wir setzen $\hat{f}$ wie folgt fort, sodass $\hat{f} \in (A^\simm \odot B)^*$ gilt:
	\[
		\hat{f}(x) = \lim_\lambda f \enbrace[\big]{\enbrace*{h_\lambda \otimes  \ind_{B^\simm}} x} = \lim_\lambda f\enbrace*{\enbrace*{h_\lambda \otimes  \ind_{B^\simm}}^{\sfrac{1}{2}} x \enbrace*{h_\lambda \otimes  \ind_{B^\simm}}^{\sfrac{1}{2}}}
	\]
	Man sieht damit, dass $\hat{f}$ linear, wohldefiniert und positiv ist. 
	$\hat{f}$ lässt sich fortsetzen zu $f^\sim \in (A^\simm \odot B^\simm)^*$ positiv.
	Es gilt trivialerweise $\norm*{f} \le \norm*{f^\sim}$ und $f^\sim (a \otimes b) \le f^\sim (\ind_{A^\simm} \otimes \ind_{B^\simm})$ für $a \in (A_+^\simm)^1$ und $b \in (B_+^\simm)^1$.
	Damit folgt $\norm*{f^\sim} \le f^\sim (\ind_{A^\simm} \otimes \ind_{B^\simm}) \le \norm*{f^\sim}$.
	Weiter gilt
	\[
		f^\sim \enbrace*{\ind_{A^\simm} \otimes \ind_{B^\simm}} = \lim_{\lambda,\mu} f \enbrace*{h_\lambda \otimes k_\mu} \le \norm*{f}
	\]
	Damit ist $\norm*{f} = \norm*{f^\sim} = f^\sim \enbrace*{\ind_{A^\simm} \otimes \ind_{B^\simm}}$ wie behauptet.
\end{beweis}

\begin{lemma}[label=lem:118]
	Seien $A$, $B$ unitale $C^*$-Algebren und $x=x^* \in A \odot B$.
	Dann existieren $\lambda > 0$, $n \in \mathbb{N}$, $z_1, \ldots ,z_n \in A \odot B$ mit 
	\[
		\lambda \cdot \enbrace*{\ind_{A^\simm} \otimes \ind_{B^\simm}} -x = \sum_{i=1}^{n} z_i^* \, z_i
	\]
	Wir schreiben auch $\lambda \cdot \enbrace*{\ind_{A^\simm} \otimes \ind_{B^\simm}} \ge_{\alg} x$.
\end{lemma}
\begin{beweis}
	Für $a \in A_+$, $b \in B_+$ gilt
	\[
		0 \le_\alg a \otimes b \le_\alg \norm*{a} \cdot \norm*{b} \cdot \enbrace*{\ind_A \otimes \ind_B}
	\]
	Weiter gilt $(A \odot B)_\so = A_\so \odot B_\so$ \emph{nicht}.
	Falls $y=y* = \sum_{k=1}^{m} a_k \otimes b_k$, so gilt
	\[
		y= \frac{1}{4} \cdot \sum_{k=1}^{m} (a_k +a_k^*) \otimes (b_k + b_k^*) - i (a_k - a_k^*) \otimes i \cdot (b_k -b_k^*)
	\]
	Schreibe $x= \sum_{k=1}^{n} c_k \otimes d_k$ mit $c_k \in A_\sa$, $d_k \in B_\sa$.
	Dann ist 
	\[
		x \le_\alg \sum_{k=1}^{n} \enbrace[\big]{(c_k)_+ + (c_k)_-} \otimes  \enbrace[\big]{(d_k)_+ + (d_k)_-} \le_\alg \enbrace[\bigg]{\Underbracket{\sum \norm*{c_k} \cdot \norm*{d_k}}{=: \lambda} } \cdot \enbrace*{\ind_A \otimes \ind_B} \qedhere
	\]
\end{beweis}

Sei $f \in (A \odot B)^*$ positiv. 
Die GNS-Konstruktion liefert einen Hilbertraum 
\[
	\mathcal{H}_f := \overline{\sfrac{A \odot B}{N_f}}^{\skal*{\cdot}{\cdot }_f}
\]
Mit Vektoren $\xi_f = \benbrace*{\ind_A \otimes \ind_B} \in \mathcal{H}_f$.
Falls $A$ oder $B$ nicht unital sind, benutze $\tilde{f} \in (A^\simm \odot B^\simm)^*$.
Wir definieren einen $^*$-Homomorphismus $\pi_f \colon A \odot B \to \mathcal{B}(\mathcal{H}_f)$ wie folgt:
Für $x,y \in A \odot B$ setze $\pi_f(x)[y] := [xy] \in \sfrac{A \odot B}{N_f} \subset \mathcal{H}_f$.
$\pi_f(x)$ setzt sich auf $\mathcal{H}_f$ fort: 
Mit \autoref{lem:118} folgt die Existenz von $\lambda>0$ mit $x^*x \le_\alg \lambda \cdot (\ind_{A} \otimes \ind_B)$.
Für $y \in A \odot B$ gilt 
\begin{align}
	\norm*{\pi_f(x)[y]}^2 = \norm*{[xy]}^2 = f \enbrace*{y^* x^*x y} &= f \enbrace*{y^* \lambda  \cdot (\ind_{A} \otimes \ind_B) y} - \sum_{k=1}^{n} f \enbrace*{y^* z_k^* z_k y}\\
	&\le \lambda \cdot 	f(y^* y) = \lambda \cdot \norm*{[y]}^2
\end{align}
Also ist $\norm*{\pi_f(x)} \le \lambda^{\sfrac{1}{2}}$ und damit ist $\pi_f(x)$ beschränkt und setzt sich fort auf $\mathcal{H}_f$.
$\pi_f$ induziert einen $^*$-Homomorphismus $\overline{\pi}_f \colon A \tensormax B \to \mathcal{B}(\mathcal{H}_f)$.
Dieser wiederum induziert eine eindeutige stetige Fortsetzung 
\[
	\overline{f}( .) := \skal*{\xi_f}{\overline{\pi}_f(.) \xi_f}
\]
von $f$ auf $A \tensormax B$.
Wir haben nun einen Homöomorphismus $S(A \odot B) \xrightarrow{\approx} S(A \tensormax B)$ konstruiert.
Wir erhalten:

\begin{satz}[label=satz:119]
	Für $x \in A \odot B$ gilt
	\[
		\norm*{x}_{\max}^2 = \sup \set*{\frac{f(y^* x^* xy)}{f(y^* y)} \given f \in S(A \odot B) , y \in A \odot B, f(y^* y)\neq 0 }
	\]
\end{satz}
\begin{beweis}
	Es ist
	\begin{align}
		&\hspace{1.14em}\sup \set*{\frac{f(y^* x^* xy)}{f(y^* y)} \given f \in S(A \odot B) , y \in A \odot B, f(y^* y)\neq 0 }\\ 
		&=\sup \set*{\frac{f(y^* x^* xy)}{f(y^* y)} \given f \in S(A \tensormax B) , y \in A \odot B, f(y^* y)\neq 0 } \\
		&= \sup \set*{\norm*{\pi_f(x)}^2 \given f \in S(A \tensormax B)} \\
		&= \norm*{x}_{\max}^2 \qedhere
	\end{align}
\end{beweis}

Wir wollen nun ein ähnliches Ergebnis für die Minimumsnorm beweisen:
Seien $f \in A^*$, $g \in B^*$ positiv, dann ist $f \otimes g \in (A \odot B)^*$ positiv, wo $(f \otimes g) \enbrace*{\sum_{i=1}^{n} a_i \otimes b_i} = \sum_{i=1}^n f(a_i) g(b_i)$.
Seien $(\pi_f,\xi_f)$, $(\pi_g,\xi_g)$ die GNS-Darstellungen von $A$ beziehungsweise $B$.
Dann ist
\[
	(f \otimes g)(x) = \skal[\big]{ \enbrace*{\xi_f \otimes \xi_g} }{(\pi_f \otimes \pi_g)(x) \enbrace*{\xi_f \otimes \xi_g}}
\]
für $x \in A \odot B$.
$f \otimes g$ ist also ein Vektorzustand und damit positiv.
Wir erhalten:

\begin{satz}[label=satz:120]
	Es gilt für $x \in A \odot B$
	\[
		\norm*{x}^2_{\min} = \sup \set*{\frac{(f \otimes g) (y^*x^*xy)}{(f \otimes g)(y^*y)} \given f \in S(A), g \in S(B), y \in A \odot B, (f \otimes g)(y^* y)\neq 0 }
	\]
	(Tatsächlich genügt es reine Zustände zu betrachten.)
\end{satz}
\begin{beweis}
	Es gilt für $x \in A \odot B$\marginnote{nach Zorn ist jede Darstellung Summe zyklischer Darstellungen}
	\begin{align}
		\norm*{x}_{\min}^2 &= \sup \set*{\norm*{(\pi_A \otimes \pi_B)(x)}^2 \given \pi_A, \pi_B \text{ Darstell.} } \\
		&= \sup \set*{ \norm*{(\pi_A \otimes \pi_B)(x)}^2 \given (\pi_A,\xi_A), (\pi_B,\xi_B) \text{ zykl. Darstell.}} \\
		&= \sup \set*{ \frac{\skal[\Big]{(\pi_A \otimes \pi_B)(y) \enbrace*{\xi_A \otimes \xi_B}}{(\pi_A \otimes \pi_B)(x^*x) \enbrace[\big]{ (\pi_A \otimes \pi_B)(y) \enbrace*{\xi_A \otimes \xi_B}}}}{\norm[\big]{(\pi_A \otimes \pi_B)(y) \enbrace*{\xi_A \otimes \xi_B}}^2}  \given\hspace{-3pt} \begin{array}{l}
	(\pi_A,\xi_A), \\
	(\pi_B, \xi_B)
\end{array} \hspace{-3pt}} \\
		&= \sup \set*{ \frac{\enbrace*{f_{\xi_A} \otimes f_{\xi_B}} \enbrace*{y^*x^*xy}}{\enbrace*{f_{\xi_A} \otimes f_{\xi_B} }(y^*y)} \given (\pi_A,\xi_A), (\pi_B, \xi_B)} \\
		&\le \sup \set*{ \frac{(f \otimes g)(y^*x^* xy)}{(f \otimes g)(y^*y)} \given f \in S(A) , g \in S(B), y \in A \odot B, (f \otimes g)(y^* y)\neq 0 } \\
		&\StackTextClap{GNS}{\le} \norm*{x}_{\min}^2 \qedhere
	\end{align}
\end{beweis}

\begin{satz}[{name={Takesaki}},label=satz:121]
	Seien $A =C_0(X)$ und $B$ $C^*$-Algebren.
	Dann existiert nur eine $C^*$-Norm auf $A \odot B$ und $A \otimes B \cong C_0(X,B)$ (mit $a \otimes b(x) = a(x) \cdot b$).
	Falls $B=C_0(Y)$, so haben wir $A \otimes B \cong C_0(X \times Y)$ mit $(a \otimes b)(x,y)=a(x) \cdot b(y)$.
\end{satz}
\begin{beweis}
	In mehreren Schritten
	\begin{enumerate}[(i)]
		\item Wir dürfen annehmen, dass $A$ und $B$ unital sind: 
		\[
			\norm*{\cdot }_{\min, A^\simm \odot B^\simm}\big|_{A \odot B} = \norm*{\cdot }_{\min, A \odot B}
		\]
		denn $\pi_A^\simm \otimes \pi_B^\simm\big|_{A \odot B} = \pi_A \otimes \pi_B$.
		Ebenso für $\norm*{\cdot }_\gamma$, denn $\gamma$ setzt sich fort zu $\gamma^\simm$ auf $A^\simm \otimes B^\simm$ (nach \autoref{satz:111} bzw. \autoref{kor:112}).
		\item Es gilt $P(A \tensormax B) \approx P(A) \times P(B)$:
		
		Sei $f \in P(A \tensormax B)$.
		Die GNS-Darstellung $\pi_f$ ist dann irreduzibel.\todo{Ref letztes Semester}
		Nach \autoref{satz:111} erhalten wir Darstellungen $\pi_A$, $\pi_B$ von $A$ bzw. $B$.
		$\pi_f$ ist eine Faktordarstellung, also ist $\pi_A$ nach \autoref{satz:111} auch eine Faktordarstellung.
		Nun gilt, da $A$ kommutativ ist
		\[
			\pi_A(A) \subset \pi_A(A)'' \cap \pi_A(A)'= C \cdot \ind_{\mathcal{H}_f} \implies \pi_A \in \hat{A}
 		\]
		also ist $\pi_A(\cdot ) = \ev_x(\cdot) \cdot \ind_{\mathcal{H}_f}$ für ein $x \in X$.
		Damit ist 
		\[
			f(a \otimes b) = \skal[\big]{\xi_f}{\pi_f(a \otimes b) \xi_f} = \ev_x(a) \cdot \Underbracket{\skal[\big]{\xi_f}{\pi_f(\ind_A \otimes b) \xi_f}}{=: g(b)}
		\]
		$\pi_B$ ist irreduzibel und $g$ ist Vektorstand und damit rein, also $g \in P(B)$ und damit $f= \ev_x \otimes g$.
		Für die Umkehrung: $\ev_x \otimes h \in P(A \tensormax B)$ (Warum?)
		\item Wir zeigen $\norm*{\cdot }_{\max} = \norm*{\cdot }_{\min}$:
		\[
			\hspace{-1em}\norm*{z}^2_{\max} = \sup \set[\big]{f(z^*z) \given f \in P(A \tensormax B)} \le \sup \set[\big]{(\ev_x \otimes g)(z^*z) \given x \in X, g \in P(B)} \StackTextClap{\ref{satz:120}}{\le} \norm*{z}_{\min}^2 \le \norm*{z}_{\max}^2
		\]
		für $z \in A \odot B$.
		\item Es gilt $\Prim(A \tensormax B) \approx X \times \Prim B$:\todo{hübsch machen}
		\[
			\begin{tikzcd}
				& {\ev_x} \otimes g & (x,g) \lar[mapsto] \\
				{\ev_x} \otimes g \dar[mapsto] & P(A \tensormax B) \dar[two heads]& X\times P(B) \lar \dar[two heads] & (x,g) \dar[mapsto] \\
				\ker(\pi_{\ev_x \otimes g}) = \ker(\pi_{\ev_x} \otimes \pi_g) & \Prim(A \tensormax B) & X \times \Prim B \lar[dotted,"\Theta"] & (x, \ker \pi_g) \\
				& \ker (\pi_{\ev_x} \otimes \pi_g) & (x, \ker \pi_g) \lar[mapsto]
			\end{tikzcd}
		\]
		$\Theta$ ist wohldefiniert (dann ist $\Theta$ automatisch ein Homöomorphismus):
		\[
			\begin{tikzcd}[column sep=0pt,row sep=2.5em]
				& & A \tensormax B \dlar["\delta_x"] \ar[ddrr,"\ev_x \otimes \pi_g = \pi_{\ev_x \otimes g}"]\\
				& B \dlar["\pi_g"] \\
				\mathcal{B}(\mathcal{H}_g) \ar[rrrr,"\cong"] & & & & \mathcal{B} \enbrace*{C \otimes \mathcal{H}_g}
			\end{tikzcd}
		\]
		Damit ist $\ker \pi_{\ev_x \otimes g} = \delta^{-1}_x (\ker \pi_g)$.
		Dies hängt nur von $\ker \pi_g$ ab und damit ist $\Theta$ wohldefiniert.
		\item $\norm*{\cdot}_{\max}=\norm*{\cdot}_{\min}$ ist die einzige $C^*$-Norm auf $A \odot B$:
        
        Sei $\gamma$ eine weitere $C^*$-Norm auf $A \odot B$ und $q \colon A \tensormax B \twoheadrightarrow  A \otimes_\gamma B$ die Quotientenabbildung.
        Dann ist 
        \[
            Y \coloneqq \set*{y \in \Prim(A \tensormax B) \given \ker q \subset y} \subset_{\mathrm{abg.}} \Prim \enbrace*{A \tensormax B}
        \]
        Erinnerung an die Hülle-Kern-Topologie:\todo{hübsch machen} 
        \[
            \begin{tikzcd}
                \set*{\text{abg. Ideale von }C} \rar["\approx"] & \set*{\text{abg. Teilmengen von }\Prim C } \\
                J \rar[mapsto] & \set*{y \in \Prim C \given J \subset y} \\
                \bigcap_{y \in W} y & W \lar[mapsto]
            \end{tikzcd}
        \]
        Angenommen $\set*{0} \neq \ker q$.
        Dann ist $Y \neq \Prim (A \tensormax B) \approx X \times \Prim B$.
        Folglich existiert $\emptyset \neq U \subset \Prim A =X$ offen und $\emptyset \neq V \subset \Prim B$ offen, sodass $U \times V \cap Y = \emptyset$.
        Dann sind $X \setminus U \subsetneq X$ und $\Prim B \setminus V \subsetneq \Prim B$ abgeschlossen, sowie $I \coloneqq \bigcap_{x \in X \setminus U} \ker \ev_x \neq \set*{0}$.
        Weiter ist $I \lhd A$ ein Ideal.
        Auch
        \(
            J \coloneqq \bigcap_{z \in \Prim B \setminus V} z \neq \set*{0}
        \)
        ist ein Ideal in $B$.
        Wir haben nun\marginnote{nach (iv) kann man $y=(y_A,y_B)$ schreiben}
        \[
            \ker q = \bigcap_{y \in Y} y \supset \bigcap_{y \notin U \times V} y = \enbrace*{\bigcap_{y_A \notin U} y} \cap \enbrace*{\bigcap_{y_B \notin V} y} \supset I \odot B \cap A \odot J \supset I \odot J \neq \set*{0}
        \]
        Damit folgt $\gamma(I \odot J)=0$.
        Aber $\gamma|_{I \odot J}$ ist eine $C^*$-Norm, darf also nicht verschwinden. Widerspruch!\qedhere
	\end{enumerate}
\end{beweis}

\begin{satz}[{name=[Die minimale Norm ist minimal]}]
    Seien $A,B$ $C^*$-Algebren und $\gamma$ eine $C^*$-Norm auf $A \odot B$.
    Dann gilt $\norm*{x}_{\min} \le \gamma(x)$ für alle $x \in A \odot B$.
\end{satz}
\begin{beweis}
    Wir beweisen den Satz in mehreren Schritten:
    \begin{itemize}
        \item Wir dürfen $A$ und $B$ als unital annehmen:
        
        Nach \autoref{kor:112} setzt sich $\gamma$ auf $A^\simm \odot B^\simm$ fort.
        Weiter gilt $\norm*{\cdot}_{\min,A^\simm \odot B^\simm}\big|_{A \odot B} = \norm*{\cdot}_{\min, A \odot B}$. (warum?)\marginnote{Darstellungen fortsetzen}
        \item Für reine Zustände $f \in P(A)$ und $g \in P(B)$ ist $f \otimes g$ stetig bezüglich $\gamma$ und setzt zu einem Zustand auf $A \otimes_\gamma B$ fort:
        
        Setze $S_\gamma \coloneqq \set[\big]{(f,g) \in P(A) \times P(B) \given f \otimes g \text{ hat Fortsetzung zu Zustand auf } A \otimes_\gamma B}$.
        Dann ist $S_\gamma \subset P(A) \times P(B)$ abgeschlossen in der Produkttopologie der $\w^*$-Topologien (Übung).
        Für $b \in B_+$ setze $C \coloneqq C^*(b) \cong C_0 \enbrace*{\sigma(b) \setminus \set*{0}}$.
        Wähle $\omega \in P(C)$ mit $\omega(b)=\norm*{b}$.\todo{Ref letztes Semester}
        Falls $f' \in P(A)$, dann ist $f' \otimes \omega \in P(A \tensormax C)= P (A \otimes_\gamma C)$ wie im Beweis von \autoref{satz:121} gesehen.
        Nach Krein-Milmann existiert eine Fortsetzung $\Theta \in P(A \otimes_\gamma B)$.\todo{ref  7.14 letztes Semester}
        
        Es gilt $\Theta = f' \otimes g'$, wobei $g' \coloneqq \Theta|_{\ind_A \otimes_\gamma B} \in P(B)$:
        Sei $y \in B_+^1$.
        Dann definieren wir $f_1 \in A^*$ durch $f_1(x) := \Theta(x \otimes y)$ und $f_2 \in  A^*$ durch $f_2(x) \coloneqq \Theta(x \otimes \ind_B -y)$.
        Wir haben dann $f_1 + f_2 =f'$ und da $f'$ rein ist, folgt $f_1 = \lambda \cdot f'$ mit $\lambda = f_1(\ind_A) =g'(y)$.
        Also ist
        \[
            \Theta(x \otimes y) =f_1(x) = f'(x) g'(y) = \enbrace*{f' \otimes g'}(x \otimes y) \implies \Theta = f' \otimes g'
        \]
        $g'$ ist ein reiner Zustand (warum?).
        Weiter gilt 
        \[
            \norm*{b} =\omega(b) =f' \otimes \omega(\ind_A \otimes b) = \Theta(\ind _A \otimes b) = (f' \otimes g')(\ind_A \otimes b) = g'(b)
        \]
        Insgesamt: $Y_{f'} \coloneqq \set*{g' \in P(B) \given (f',g') \in S_\gamma}$ normiert $B$ für jedes $f' \in P(A)$.
        Damit ist $Y_{f'} \subset P(B)$ $\w^*$-dicht (Übung).
        Aber $\set*{f'} \times Y_{f'} \subset S_\gamma \subset_{\mathrm{abg.}} P(A) \times P(B)$, also 
        \[
            \set*{f'} \times P(B) = \set*{f'} \times \overline{Y_{f'}}^{P(B)} = \overline{\set*{f'} \times Y_{f'}}^{P(A) \times P(B)} \subset S_\gamma
        \]
        Es folgt also $P(A) \times P(B) \subset S_\gamma \subset P(A) \times P(B)$.
        \item Jedes $f \otimes g$ mit $f \in S(A)$, $g \in S(B)$ setzt sich fort zu einem Zustand auf $A \otimes_\gamma B$ (Übung).
        \item Für $x \in A \odot B$ gilt
        \begin{align}
            \norm*{x}_{\min}^2 &\StackTextClap{\ref{satz:120}}{=} \sup \set*{ \frac{f \otimes g(y^* x^* xy)}{f \otimes g(y* y)} \given f \in S(A), g \in S(B), y \in A \odot B, \text{ Nenner } \neq 0} \\
            &\StackTextClap{(iii)}{\le} \sup \set*{ \frac{\gamma(x^* x) f \otimes g(y^*y)}{f \otimes g(y* y)} \given f \in S(A), g \in S(B), y \in A \odot B, \text{ Nenner } \neq 0}\\
            &= \gamma(x^*x) = \gamma(x)^2 \qedhere
        \end{align}
    \end{itemize}
\end{beweis}

\begin{bemerkung}
    \begin{enumerate}[(i)]
        \item Für jede $C^*$-Norm $\gamma$ auf $A \odot B$ gilt 
        \[
            \norm*{\cdot}_{\min} \le \gamma(\cdot ) \le \norm*{\cdot}_{\max}
        \]
        und $\gamma(a \otimes b) = \norm*{a}_A \cdot \norm*{b}_B$ für $a \in A$ und $b \in B$, denn
        \begin{align}
            \norm*{a}^2 \cdot \norm*{b}^2 \ge \gamma(a \otimes b)^2 \ge \norm*{a \otimes b}_{\min}^2 &\StackTextClap{\ref{satz:120}}{\ge} \sup \set[\big]{(f \otimes g) \enbrace*{a^* a \otimes b^*b} \given f \in P(A), g \in P(B)} \\
            &= \sup \set[\big]{f (a^*a) g(b^*b) \given f \in P(A), g \in P(B)} \\
            &= \norm*{a}^2 \cdot \norm*{b}^2
        \end{align}
        \item Wir haben kanonische $^*$-Homomorphismen 
        $A \tensormax B \twoheadrightarrow A \otimes_\gamma B \twoheadrightarrow A \tensormin B$
    \end{enumerate}
\end{bemerkung}

\begin{korollar}[{name=[minimales Tensorprodukt erhält Einfachheit]}]
    Sind $A$ und $B$ einfach, so ist $A \tensormin B$ auch einfach.
\end{korollar}
\begin{beweis}
	Es genügt zu zeigen, dass jede irreduzible Darstellung $\pi \colon A \tensormin B \to \mathcal{B}(\mathcal{H})$ isometrisch ist, denn wenn man für $J \lhd A \tensormin B$ die Komposition
	\[
		A \tensormin B \longrightarrow \sfrac{A \tensormin B}{J} \xrightarrow{\text{irred. Darst.}} \mathcal{B}(\mathcal{H})
	\]
	betrachtet, folgt dann $J=0$.
	Es genügt weiter zu zeigen, dass für $\pi \colon A \tensormin B \to \mathcal{B}(\mathcal{H})$ irreduzibel $\pi|_{A \odot B}$ treu ist, denn dann ist $\gamma(\cdot ) = \norm*{\pi(\cdot )}$ eine $C^*$-Norm auf $A \odot B$ und es gilt $\norm*{\cdot}_{\min} \le \gamma(\cdot) \le \norm*{\cdot }_{\min} \implies \pi$ isometrisch.
	
	Sei also $\pi$ irreduzibel gegeben und seien $\pi_A$, $\pi_B$ wie in \autoref{satz:111}.
	Sei $c = \sum_{i=1}^{n} a_i \otimes b_i \in A \odot B$ mit $\pi(c) = \sum_{i=1}^{n} \pi_A(a_i) \pi_B(b_i)=0$.
	Wir zeigen, dass $c=0$ ist.
	Setze $\tilde{\mathcal{H}} \coloneqq \mathcal{H} \otimes \mathbb{C}^n$ und
	\begin{align}
		X &\coloneqq \set*{\sum\nolimits_{i=1}^{n} \pi_B(bb_i) \xi \otimes e_i \given b \in B, \xi \in \mathcal{H}} \stackrel[\text{abg.}]{}{\subset} \tilde{\mathcal{H}} \\
		Y &\coloneqq \set*{\sum\nolimits_{i=1}^{n} \pi_A(a_i^*) \eta \otimes e_i \given \eta \in \mathcal{H} }\stackrel[\text{abg.}]{}{\subset} \tilde{\mathcal{H}}
	\end{align}
	Da $\pi(c) =0$ und $\benbrace*{\pi_A(A), \pi_B(B)}=0$ ist, gilt $X \perp Y$.
	Sei $\mathcal{B}(\tilde{\mathcal{H}}) \ni p \colon \tilde{\mathcal{H}} \to X \subset \tilde{\mathcal{H}}$ die orthogonale Projektion.
	Mit $\mathcal{B} (\tilde{\mathcal{H}}) \cong M_n \enbrace*{\mathcal{B}(\mathcal{H})}$ können wir $p=(p_{ij})$ mit $p_{ij} \in \mathcal{B}(\mathcal{H})$ schreiben.
	Es gilt $\enbrace*{\pi_B(B) \otimes \ind_{\mathbb{C}^n}} X \subset X$ und $\enbrace*{\pi_A(A) \otimes \ind_{\mathbb{C}^n}}X \subset X$, also ist
	\[
		p \in \enbrace[\big]{\pi_B(B) \otimes \ind_{\mathbb{C}^n}}' \cap \enbrace[\big]{\pi_A(A) \otimes \ind_{\mathbb{C}^n}}' \subset \mathcal{B}(\tilde{\mathcal{H}})
	\]
	Damit gilt dann auch $p_{ij} \in \pi_B(B)' \cap \pi_A(A)' \subset \mathcal{B}(\mathcal{H})$.
	Es folgt nun, dass $p_{ij} \in \pi \enbrace*{A \tensormin B}'$ gilt und weiter mit der Irreduzibilität $p_{ij} \in \mathbb{C}\cdot \ind_{\mathcal{H}}$, also $p \in \ind_{\mathcal{H}} \otimes M_n$.
	
	Für $b \in B$ und $\xi \in \mathcal{H}$ haben wir nun
	\begin{align}
		X \ni \sum_{i=1}^{n} \enbrace[\big]{\pi_B(bb_i) \xi} \otimes e_i = p \enbrace*{\sum_{i=1}^{n} \pi_B(bb_i) \xi \otimes e_i} &= \sum_{i,j=1}^n p_{ij} \pi_B(bb_j) \otimes e_i \\
		&= \sum_{i=1}^{n} \sum_{j=1}^{n} p_{ij} \pi_B(bb_j) \xi \otimes e_i
	\end{align}
	Also ist $\pi_B(bb_i)\xi = \sum_{i=1}^{n} p_{ij} \pi_B(bb_j) \xi$ für alle $b \in B$ und $\xi \in \mathcal{H}$.
	Da $b \in B$ beliebig war, folgt $\pi_B(b_i) = \sum_{j=1}^{n}  p_{ij} \pi_B(b_j)$ für $i=1,\ldots ,n$.
	Weiter gilt für $\eta \in \mathcal{H}$
	\[
		0 = p \Underbracket{\sum_{j=1}^{n} \pi_A(a_i^*) \eta \otimes e_i}{\in Y} = \sum_{i,j=1}^{n} p_{ij} \pi_B(a_j^*) \eta \otimes e_i
	\]
	Damit ist $0 = \sum_{j=1}^{n} p_{ij} \pi_A(a_j^*)$ für $i=1,\ldots ,n$, also auch $0 = \sum_{j=1}^{n} \overline{p_{ij}} \pi_A(a_j)$ für $i=1,\ldots ,n$.
	Wir erhalten also insgesamt\todo{Indizes noch falsch anscheinend}
	\[
		\pi_A \otimes \pi_B(c) = \sum_{i=1}^{n} \pi_A(a_i) \otimes \pi_B(b_i) = \sum_{i=1}^{n} \pi_A(a_i) \otimes  \sum_{j=1}^{n} p_{ij} \pi_B(b_j) = \sum_{i,j=1}^{n} p_{ij} \pi_A(a_i) \otimes \pi_B(b_j) =0
	\]
	$\pi_A$ und $\pi_B$ sind treu, also ist auch $\pi_A \otimes \pi_B$ treu auf $A \odot B$ und es folgt $c=0$.
\end{beweis}

\begin{proposition}[label=prop:125]
	Es seien $A_1$, $A_2$, $B_1$ und $B_2$ $C^*$-Algebren, $\varphi \colon A_1 \to B_1$ und $\psi \colon A_2 \to B_2$ $^*$-Homomorphismen.
	Dann existieren kanonische $^*$-Homomorphismen $\varphi \otimes \psi$
	\[
		\begin{tikzcd}[row sep=0pt]
			\varphi \otimes \psi \colon & A_1 \odot A_2 \rar & B_1 \odot B_2 \\
			\varphi \tensormax  \psi \colon & A_1 \tensormax  A_2 \rar & B_1 \tensormax  B_2 \\
			\varphi \tensormin  \psi \colon & A_1 \tensormin  A_2 \rar & B_1 \tensormin  B_2
		\end{tikzcd}
	\]
	Sind $\varphi$ und $\psi$ injektiv, so ist auch $\varphi \tensormin \psi$ injektiv -- im Allgemeinen ist dies nicht richtig für $\tensormax$.
\end{proposition}
\begin{beweis}
	\emph{Übung!}
\end{beweis}
% section 1 (end)
\newpage

\section{Nuklearität und Exaktheit} % (fold)
\label{sec:2}

\emph{\textenglish{Nucular. It's pronounced, \enquote{nucular}.}} -- Homer \textsc{Simpson}

\begin{definition}[{name=[{nukleare C*-Algebra}]}]
	Eine $C^*$-Algebra $A$ heißt \Index{nuklear}, falls für jede $C^*$-Algebra $B$ nur eine $C^*$-Norm auf $A \odot B$ existiert, das heißt $A \tensormax B = A \tensormin B$. 
\end{definition}

\begin{proposition}[{name=[{Nuklearität und Summen, Tensorprodukte, direkte Limiten und Quotienten}]},label=prop:22]
	\leavevmode
	\begin{enumerate}[(i)]
		\item Sind $A,B$ nuklear, so sind auch $A \oplus B$ und $A \otimes B$ nuklear.
		\item Ist $A = \Dlim\, A_i$ mit $A_i$ nuklear, so ist auch $A$ nuklear.
		\item Für \(
			\begin{tikzcd}[cramped,sep=small]
				0 \rar & J \rar & A \rar & \sfrac{A}{J} \rar & 0 
			\end{tikzcd}
		\) ist $A$ nuklear genau dann, wenn $J$ und $\sfrac{A}{J}$ nuklear sind.
	\end{enumerate}
\end{proposition}
\begin{beweis}[{name={mit \autoref{prop:24}}}]
	\leavevmode
	\begin{enumerate}[(i)]
		\item Für direkte Summen, ist die Aussage klar.
		Für Tensorprodukte auch wegen der Assoziativität von $\tensormax$ und $\tensormin$.\todo{RevChap 2}
		\item \enquote{klar} für injektive Verbindungsabbildungen. 
		Für den allgemeinen Fall benutzte (iii).
		\item Für die erste Implikation zeigen wir, dass $J$ nuklear ist.
		Folgendes Diagramm kommutiert
		\[
			\begin{tikzcd}
				J \tensormax  B \rar["\rho",two heads] & \overline{J \odot B}^{\norm*{\cdot}_{\max, A \odot B}} \dar[hook,two heads] \rar[phantom,sloped,"\lhd"] & A \tensormax \dar[equals] \\
				& J \tensormin B \rar[hook,"\ref{prop:125}"] & A \tensormin B
			\end{tikzcd}
		\]
		Damit ist $\overline{J \odot B}^{\norm*{\cdot}_{\max,A \odot B}} = J \tensormin B$.  
		Falls $\pi \colon J \odot B \to \mathcal{B}(\mathcal{H})$ eine nichtdegenerierte Darstellung ist und $(h_\lambda)$, $(g_\nu)$ approximative Einsen von $J$ bzw. $B$, so definieren wir $\sigma \colon A \odot B \to \mathcal{B}(\mathcal{H})$ durch
		\[
			\sigma(a \otimes b) \coloneqq \lim_{\lambda,\nu} \pi \enbrace[\big]{a h_\lambda \otimes b g_\nu}
		\]
		$\sigma$ existiert und setzt $\pi$ fort.
		Also gilt für $x \in J \odot B \subset J \tensormax B$
		\begin{align}
			\norm*{x}_{\max, J \odot B} &= \sup \set[\big]{\norm*{\pi(x)} \given \pi \text{ nicht deg. Darst. von } J \odot B} \\
			&\le \sup \set[\big]{\norm*{\sigma(x)} \given \sigma \text{ nicht deg. Darst. von } A \odot B} \\
			&\le \norm*{\rho(x)} \\
			&\le \norm*{x}_{\max, J \odot B} 
		\end{align} 
		Damit ist $\rho$ isometrisch und $J \tensormax B = J \tensormin B$ für $B$ beliebig.
		Also ist $J$ nuklear.
		$\sfrac{A}{J}$ zeigen wir später.\todo{referenzieren bzw. hier einfügen}
		
		Es seien nun $J$ und $\sfrac{A}{J}$ nuklear. 
		Wir müssen zeigen, dass auch $A$ nuklear ist.
		Betrachte dazu das folgende kommutative Diagramm
		\[
			\begin{tikzcd}
				0 \rar & B \tensormax J \rar \dar["\varphi_J","\cong"'] & B \tensormax A \rar \dar["\varphi_A"] & B \tensormax \sfrac{A}{J} \dar["\varphi_{\sfrac{A}{J}}","\cong"'] \rar & 0 \\
				0 \rar & B \tensormin J \rar & B \tensormin A \rar & B \tensormin \sfrac{A}{J} \rar & 0
			\end{tikzcd}
		\]
		Die obere Zeile ist exakt nach \autoref{prop:24}.
		Wir wollen zeigen, dass auch die untere Zeile exakt ist.
		Wir haben $B \odot A \cap B \tensormin J = B \odot J$ und damit folgt
		\[
			B \odot \faktor{A}{J} = \faktor{B \odot A}{B \odot J} = \faktor{B \odot A}{B \odot A \cap B \tensormin J} \,\,\stackrel[\text{dicht}]{}{\subset}\,\, \faktor{B \tensormin A}{B \tensormin J}
		\]
		Damit induziert $\norm*{\cdot}_{\min}$ eine $C^*$-Norm $\gamma$ auf $B \odot \sfrac{A}{J}$ mit $B \mathop{\otimes_\gamma} \sfrac{A}{J} = \sfrac{B \tensormin A}{B \tensormin J}$.
		Da $\sfrac{A}{J}$ nuklear ist, muss also $\gamma(\cdot) = \norm*{\cdot}_{\min, B \odot \sfrac{A}{J}}$ sein und damit $B \otimes_\gamma \sfrac{A}{J} = B \tensormin \sfrac{A}{J}$.
		Damit ist auch die untere Zeile exakt.
		Nach dem 5er-Lemma ist $\varphi_A$ dann auch ein Isomorphismus.
		\qedhere
	\end{enumerate}
\end{beweis}

\begin{beispiel}
	\begin{enumerate}[(i)]
		\item Ist $A$ endlichdimensional, so ist $A$ auch nuklear.
		\item $A=C_0(X)$ ist nuklear nach \autoref{satz:121}.
		\item Ist $A$ AF\todo{was sollt das sein?}\footnote{z.B. $M_{2^\infty} = M_2 \otimes M_2 \otimes \ldots$} ist nach (i) und \autoref{prop:22}(ii) $A$ nuklear.
		\item Ist $A$ AH, so ist $A$ nuklear. Zum Beispiel
		\[
			A_\theta = C^* \enbrace*{u,v \mid u,v \text{ unitär}, uv=e^{2 \pi i \theta}vu}
		\]
		\item Die Teoplitzalgebra ist nuklear: Nach \autoref{prop:22}(iii) und (iii)
		\[
			\begin{tikzcd}
				0 \rar & \mathcal{K} \rar & \mathcal{T} \rar & C(S^1) \rar & 0
			\end{tikzcd}
		\]
		\item {\large¿} $C_r^*(G)$ (reduzierte Gruppen-$C^*$-Algebra) {\large?} \quad $\leadsto$ \quad {\large¡} Amenabilität {\large!}
		\item {\large¿} $A \rtimes_{\alpha(r)} G$ {\large?}
	\end{enumerate}
\end{beispiel}
\begin{beweis}
	\begin{enumerate}[(i)]
		\item Wegen \autoref{prop:22} (i) können wir $A=M_r$ annehmen.
		Sei $B$ eine $C^*$-Algebra.
		Dann ist $A \odot B$ bereits vollständig bezüglich $\norm*{\cdot}_{\min}$: 
		Sei dazu $(d_\lambda) \subset A \odot B$ ein Cauchy-Netz, wo $d_\lambda = \sum_{i,j=1}^r e_{ij} \otimes d_{\lambda,ij}$ mit $d_{\lambda,ij} \in B$ für $\lambda, i,j$.
		Dann ist
		\[
			e_{ij} \otimes d_{\lambda,ij} = \enbrace*{e_{ii} \otimes \ind_{B^\simm}} d_\lambda \enbrace*{e_{jj} \otimes \ind_{B^\simm}}
		\]
		ebenfalls Cauchy.\footnote{$\varepsilon \leadsto \lambda_0$, für $\lambda_1, \lambda_2 \ge \lambda_0$  $\norm*{d_{\lambda_1,ij} - d_{\lambda_2,ij}} = \norm*{e_{ij} \otimes \enbrace*{d_{\lambda_1,ij}- d_{\lambda_2, ij}}} = \norm*{e_{ij} \otimes d_{\lambda_1,ij}- e_{ij} \otimes d_{\lambda_2, ij}}_{\min} < \varepsilon$}
		Aber $\norm*{e_{ij} \otimes b}_{\min} = \norm*{e_{ij}}_A \norm*{b}_B = \norm*{b}_B$.
		Also ist $(d_{\lambda,ij})_{\Lambda} \subset B$ Cauchy und konvergiert in $B$ für jedes $i,j$.
		Damit konvergiert auch $(d_\lambda)$ und somit ist $\enbrace*{A \odot B, \norm*{\cdot}_{\min}}$ eine $C^*$-Algebra.
		Daraus folgt $\norm*{\cdot}_{\min}=\norm{\cdot}_{\max}$ mit der Eindeutigkeit einer \emph{vollständigen} $C^*$-Norm.\qedhere
	\end{enumerate}
\end{beweis}

\begin{proposition}[label=prop:24,{name=[{das maximale Tensorprodukt ist exakt}]}]
	$A \tensormax -$ ist ein exakter Funktor für jede $C^*$-Algebra $A$
\end{proposition}
\begin{beweis}
	Für Funktorialität benutze \autoref{prop:125}.
	$\pi \colon B \to C$ induziert also $\id_A \tensormax \pi \colon A \tensormax B \to A \tensormax C$.
	Sei nun \(
		\begin{tikzcd}[cramped,sep=small]
			0 \rar & J \rar["\iota"] & B \rar["\pi"] & \sfrac{B}{J} \rar & 0
		\end{tikzcd}
	\) exakt.
	Zu zeigen ist, dass folgende Seqenz exakt ist
	\[
		\begin{tikzcd}[column sep=3.8em]
			0 \rar & A \tensormax J \rar["\id_A \tensormax \iota"] & A \tensormax B \rar["\id_A \tensormax \pi"] & A \tensormax \sfrac{B}{J} \rar & 0
		\end{tikzcd}
	\]
	$\id_A \tensormax \pi$ hat dichtes Bild und vollständiges Bild und ist damit surjektiv, womit die Exaktheit an der dritten Stelle gezeigt ist.
	
	Für die Injektivität von $\id_A \tensormax \iota$ sei $f \in S(A \tensormax J) \StackText{\ref{satz:119}}{\approx} S(A \odot J)$ und $(u_\lambda)$ eine approximative Eins für $J$.
	Dann definiert $a \otimes b \mapsto \lim_\lambda f \enbrace*{a \otimes u_\lambda b u_\lambda}$ eine Fortsetzung von $f$ zu $\tilde{f} \in S(A \odot B)\approx S(A \tensormax B)$ (warum?).
	Nach \autoref{satz:119} ist nun $\norm*{\cdot}_{A \tensormax J} \big|_{A \odot J} = \norm*{\cdot}_{A \tensormax B}\big|_{A \odot J}$, also ist ${\id_{A}} \tensormax \iota \big|_{A \odot J}$ isometrisch.
	Wegen Stetigkeit ist dann auch ${\id_A} \tensormax \iota$ isometrisch.
	
	Zur Exaktheit an der zweiten Stelle: 
	Wir haben $A \tensormax J \lhd A \tensormax B$ (warum?) und außerdem
	\[
		A \tensormax J \subset \ker \enbrace*{{\id_A} \tensormax \pi}
	\]
	Damit erhalten wir eine induzierte Abbildung $\sigma\colon\sfrac{A \tensormax B}{A \tensormax J} \twoheadrightarrow \sfrac{A \tensormax B}{\ker \enbrace*{{\id_A} \tensormax \pi}} = A \tensormax \sfrac{B}{J}$.
	Weiter haben wir $A \odot B \cap A \tensormax J = A \odot J$.
	Damit gilt
	\[
		A \odot \faktor{B}{J} = \faktor{A \odot B}{A \odot J} = \faktor{A \odot B}{A \odot B \cap A \tensormax J} \,\,\stackrel[\text{dicht}]{}{\subset}\, \, \faktor{A \tensormax B}{A \tensormax J}
	\]
	Also erhalten wir $\rho \colon A \tensormax \sfrac{B}{J} \twoheadrightarrow \sfrac{A \tensormax B}{A \tensormax J}$ mit $\sigma \colon \circ \rho \big|_{A \odot \sfrac{B}{J}} = \id_{A \odot \sfrac{B}{J}}$.
	$\sigma \circ \rho$ ist also isometrisch auf $A \odot \sfrac{B}{J}$ und wegen Stetigkeit damit überall.
	Da $\rho$ surjektiv ist, muss $\sigma$ insbesondere injektiv sein.
	Also gilt $A \tensormax J = \ker \enbrace*{{\id_A} \otimes \pi}$ und damit Exaktheit.
\end{beweis}

\begin{definition}[{name=[exakte C^*-Algebra]}]
	Eine $C^*$-Algebra $A$ heißt \bet{exakt}\index{exakte $C^*$-Algebra}, falls $A \tensormin -$ ein exakter Funktor ist, das heißt falls \(
		\begin{tikzcd}[cramped,sep=small]
			0 \rar & J \rar["\iota"] & B \rar["\pi"] & \sfrac{B}{J} \rar & 0
		\end{tikzcd}
	\)
	exakt ist, so ist auch die folgende Sequenz exakt
	\[
		\begin{tikzcd}
			0 \rar & A \tensormin J \rar["{\id_A} \otimes \iota"] & A \tensormin B \rar["{\id_A} \otimes \pi"] & A \tensormin \sfrac{B}{J} \rar & 0
		\end{tikzcd}
	\]
\end{definition}

\begin{bemerkung}[{name=[{nukleare $C^*$-Algebren sind exakt}]}]
	Nach \autoref{prop:24} ist jede nukleare $C^*$-Algebra exakt.
\end{bemerkung}

\begin{satz}[{name={Kirchberg, Wassermann}}]
	Eine $C^*$-Unteralgebra einer exakten $C^*$-Algebra ist exakt.
	Insbesondere sind $C^*$-Unteralgebren von nuklearen $C^*$-Algebren exakt.
\end{satz}
\begin{beweis}[Idee]
	Es seien $A \subset C$ $C^*$-Algebren und $C$ exakt.
	Sei \(
		\begin{tikzcd}[cramped,sep=small]
			0 \rar & J \rar["\iota"] & B \rar["\pi"] & \sfrac{B}{J} \rar & 0
		\end{tikzcd}
	\) eine beliebige exakte Sequenz.
	Es genügt zu zeigen: 
	\[
		A \tensormin J = \ker \enbrace*{{\id_A} \tensormin \pi}
	\]
	Dabei ist die Inklusion \enquote{$\subseteq$} klar.
	Die andere zeigt man in mehereren Schritten:
	\begin{enumerate}[1.]
		\item Es gilt $C \tensormin J \cap A \tensormin B = A \tensormin J$.
		
		Dabei ist \enquote{$\supseteq$} klar. 
		Die andere Inklusion zeigt man mit Hilfe einer approximativen Eins.
		\item Sei $\varphi \in S(A)$. 
		Man benutzt die \emph{slice map}
		\mapdef{\varphi \otimes {\id_B} \colon A \odot B}{\mathbb{C} \odot B = B}{a \otimes b}{\varphi(a) \cdot b}{}
		$\varphi \otimes {\id_B}$ ist stetig bezüglich $\norm*{\cdot}_{{\min}, A \odot B}$ und wir erhalten eine stetige Fortsetzung $\varphi \tensormin  {\id_B} \colon A \tensormin B \to B$. (Übung)\todo{\TeX{}en?}
		\item Betrachte 
		\[
			K_J \coloneqq \set[\big]{x \in A \tensormin B \given (\varphi \otimes {\id_B})(x^* x) \in J \, \forall \varphi \in S(A)}
		\]
		und zeige $K_J = \ker \enbrace*{{\id_A} \tensormin \pi_n}$ (benutze \autoref{satz:120}).
		\item Es gilt $K_J \subset A \tensormin J$:
		Sei $x_0 \in K_J$.
		Falls $\psi \in S(C)$, so ist $\varphi \coloneqq \psi|_A$ bis auf Normierung ein Zustand auf $A$ und es gilt $(\psi \otimes {\id_B})(x_0^*x_0) = (\varphi \otimes {\id_B})(x_0^* x_0) \in J$.
		Also gilt
		\[
			x_0 \in \set*{x \in C \tensormin B \given (\rho \otimes {\id_B})(x^*x) \in J \, \forall \rho \in S(C)} \StackText{3.}{=} \ker \enbrace*{{\id_C} \tensormin \pi_n} = C \tensormin J
		\]
		Damit ist $x_0 \in \enbrace*{A \tensormin B} \cap \enbrace*{C \tensormin J} \StackText{1.}{=} A \tensormin J$.\qedhere
	\end{enumerate}
\end{beweis}
% section 2 (end)
\newpage

\section{Vollständig positive Abbildungen} % (fold)
\label{sec:3}

\begin{erinnerungA}[{name=[{induzierte Abbildung auf Matrizen}]}]
	Eine Abbildung $\varphi\colon A \to B$ induziert eine Abbildung $\varphi^\ssbrace{n} \colon M_n(A) \to M_n(B)$ durch $(a_{ij})_{ij} \mapsto \enbrace*{\varphi(a_{ij})}_{ij}$.
	Dabei müssen $A$ und $B$ keine $C^*$-Algebren sein.
	Sind $A$ und $B$ Vektorräume, so ist $\varphi^\ssbrace{n} = \varphi \otimes \id_{M_n}$ und aus Linearität von $\varphi$ folgt Linearität von $\varphi^\ssbrace{n}$.
	Ist $\varphi$ ein $^*$-Homomorphismus, so ist auch $\varphi^\ssbrace{n}$ ein $^*$-Homomorphismus.
	Wir wir in \autoref{bsp:35} sehen werden, muss $\varphi^\ssbrace{n}$ aber nicht positiv sein, wenn $\varphi$ positiv ist!
	Dies führt uns zu folgender Definition:
\end{erinnerungA}

\begin{definition}[{name=[positiv, vollständig positiv und beschränkt]},label=def:32]
	Seien $A,B$ $C^*$-Algebren und $\varphi \colon A \to B$ linear
	\begin{enumerate}[(i),itemsep=1pt]
		\item $\varphi$ heißt \bet{positiv}, falls $\varphi(A_+) \subset B_+$.\index{positive Abbildung}
		\item $\varphi$ heißt $\mathbold n$\bet{-positiv}, falls $\varphi^\ssbrace{n}$ positiv ist.\index{n-positive Abbildung@$n$-positive Abbildung}
		\item $\varphi$ heißt \bet{vollständig positiv}, falls $\varphi$ $n$-positiv ist für alle $n \in \mathbb{N}$.\index{vollständig positive Abbildung}
		\item $\varphi$ heißt \bet{vollständig beschränkt}, falls 
		\[
			\norm*{\varphi}_\vb \coloneqq \sup_{n \in \mathbb{N}} \norm*{\varphi^\ssbrace{n}} < \infty
		\]
 	\end{enumerate}
\end{definition}

\begin{definition}[{name=[Operatorsystem]}]
	Sei $A$ eine unitale $C^*$-Algebra. 
	Ein unitaler selbstadjungierter Untervektorraum von $X$ heißt \Index{Operatorsystem}. 
\end{definition}

\begin{bemerkung}[{name=[grundlegende Eigenschaften von Operatorsystemen]}]
	\leavevmode
	\begin{enumerate}[(i)]
		\item Sei $X$ ein Operatorsystem.
		Dann gilt $X = \Span X_+$:
		
		Ist $x \in X$, so sind $x+x^*$ sowie $ i(x-x^*)$ in $X_\sa$ enthalten und es folgt
		\[
			x = \sfrac{1}{2}\enbrace[\big]{x + x^* - i \cdot i (x-x^*)} \implies x \in \Span X_\sa
		\]
		Ist $y \in X_\sa$, so sind $\norm*{y} \cdot \ind -y, \norm*{y} \cdot \ind \in X_+$ und es folgt
		\[
			y = \norm*{y} \cdot \ind - \enbrace[\big]{\norm*{y} \cdot \ind -y} \implies X_\sa = \Span X_+
		\]
		\item \autoref{def:32} ist auch sinnvoll für $\varphi \colon X \to Y$ linear, wobei $X$ und $Y$ Operatorsysteme sind.
	\end{enumerate}
\end{bemerkung}

\begin{beispiel}[label=bsp:35,{name=[{n-positive und vollständig positive Abbildungen}]}]
	\leavevmode
	\begin{enumerate}[(i)]
		\item Ist $\varphi \colon A \to B$ ein $^*$-Homomorphismus, so ist $\varphi$ vollständig positiv.
		Die Positivität folgt mit
		\[
			\varphi(x^*x) = \varphi(x)^* \varphi(x) \ge 0
		\]
		Da $\varphi^\ssbrace{n}$ auch ein $^*$-Homorphismus ist, ist auch $\varphi^\ssbrace{n}$ positiv für alle $n \in \mathbb{N}$.
		\item Sei $\varphi \colon A \to B$ $n$-positiv und $v \in B$ fest gewählt.
		Dann ist die \Index{Kompression}
		\mapdef{\varphi_v \colon A}{B}{a}{v^* \varphi(a) v}{}
		ebenfalls $n$-positiv:
		Es ist $(\varphi_v)^\ssbrace{n} = \enbrace*{\varphi^\ssbrace{n}}_{v \otimes \ind_{M_n}}$ und nach \cite[Satz~3.7 (v)]{OpAlg} gilt
		\[
			(\varphi_v)^\ssbrace{n}(x^* x) = \Underbracket{\enbrace*{v^* \otimes \ind_{M_n}} \Underbracket{\varphi^\ssbrace{n}(x^*x)}{\ge 0} \enbrace*{v \otimes \ind_{M_n}}}{\ge 0}
		\]
		Insbesondere gilt: Ist $\varphi$ vollständig positiv, so ist $\varphi_v$ vollständig positiv. Ebenso wenn $\varphi$ ein $^*$-Homomorphismus ist.
		Wir werden weiter sehen: \emph{Jede vollständig positive Abbildung ist von dieser Form!}\todo{referenzieren!}
		\item $\varphi \colon \mathbb{C}^k \to B$ ist positiv genau dann, wenn $b_i \coloneqq \varphi(e_i) \ge 0$ für $i=1,\ldots ,k$.
		
		$\varphi \colon \mathbb{C}^k \to C(X)$ ist unital und positiv genau dann, wenn $(b_i)_{i=1,\ldots ,k}$ Zerlegung der Eins ist.
		\item Die Involution $\tau \colon M_2 \to M_2$, $\tau(a)=a^*$ ist positiv, aber nicht $2$-positiv:
		Die Positivität folgt sofort mit $\tau(a^*a)= a^*a \ge 0$. Betrachtet man die Matrix von Matrizen
		\[
			x \coloneqq \begin{pmatrix}
				1 & 0 & 0 & 1 \\
				0 & 0 & 0 & 0 \\
				0 & 0 & 0 & 0 \\
				1 & 0 & 0 & 1
			\end{pmatrix} \in M_2(M_2)
		\]
		so gilt $x \ge 0$, da die Eigenwerte positiv sind. Es gilt aber 
		\[
			\tau^\ssbrace{2}(x) = \begin{pmatrix}
				1 & 0 & 0 & 0 \\
				0 & 0 & 1 & 0 \\
				0 & 1 & 0 & 0 \\
				0 & 0 & 0 & 1
			\end{pmatrix} \not\ge 0
		\]
	\end{enumerate}
\end{beispiel}

\begin{proposition}[label=prop:36,{name=[positive Abbildungen sind beschränkt]}]
	Es seien $A,B$ $C^*$-Algebren und $\varphi \colon A \to B$ positiv.
	Dann ist $\varphi$ beschränkt.
\end{proposition}
\begin{beweis}
	Falls $\varphi$ unbeschränkt auf $A$ ist, so auch auf $A_+$.
	Folglich existiert $(a_n)_{\mathbb{N}} \subset A_+^1$ mit $\norm*{\varphi(a_n)} \ge n^3$ für $n \in \mathbb{N}$.
	Dann ist $a \coloneqq \sum_{n=0}^{\infty} \sfrac{1}{n^2} \cdot a_n \in A$ und $\sfrac{1}{n^2} \cdot a_n \le a$ für $n \in \mathbb{N}$.
	Damit haben wir für alle $n \in \mathbb{N}$
	\[
		n \le \sfrac{1}{n^2} \cdot \norm*{\varphi(a_n)} = \norm[\big]{\varphi \enbrace*{\sfrac{1}{n^2} \cdot a_n}} \le \norm*{\varphi(a)} 
	\]
	Dies ist ein Widerspruch.
\end{beweis}

\begin{bemerkung}[label=bem:37,{name=[{positive Abbildungen auf Operatorsystemen}]}]
	\leavevmode
	\begin{enumerate}[(i),itemsep=1pt]
		\item Sei $X$ ein Operatorsystem und $\varphi \colon X \to B$ positiv.
		Dann gilt $\norm*{\varphi} \le 2 \cdot \norm*{\varphi(\ind)}$ (warum?)\todo{Beweis in den Anhang}
		\item In (i) kann Gleichheit auftreten: 
		Betrachte dazu $X \coloneqq \Span \set[\big]{\ind,z,\overline{z}} \subset C(\mathbb{T})$ und $\varphi \colon X \to M_2$ definiert durch
		\[
			\varphi \enbrace*{a \cdot \ind + b \cdot z + c \cdot \overline{z}} = \begin{pmatrix}
				a & 2 b \\
				2 c & a
			\end{pmatrix}
		\]
		$\varphi$ ist positiv, wie man leicht nachrechnet (generisches positives Element hinschreiben).
		Es gilt dann nach (i) \todo{RevChap 3}
		\[
			2 \cdot \norm*{\varphi(\ind)} = 2 = \norm*{\varphi(z)} \le \norm*{\varphi} \le 2 \cdot \norm*{\varphi(\ind)}
		\]
	\end{enumerate}
\end{bemerkung}

\begin{lemma}[label=lem:38]
	Sei $A$ eine $C^*$-Algebra und $a,h \in A$ ($h \in A_\sa$).
	Dann gilt
	\begin{enumerate}[(i)]
		\item $\begin{psmallmatrix}
			\ind_{A^\simm} & a \\
			a^* & \ind_{A^\simm}
		\end{psmallmatrix} \ge 0 \iff \norm*{a} \le 1$
		\item $\begin{psmallmatrix}
			h & a \\
			a^* & h
		\end{psmallmatrix} \ge 0 \implies a^* \le \norm*{h} \cdot h$, also $\norm*{a} \le \norm*{h}$\marginnote{Die Umkehrung gilt nicht!}
		\item $\begin{psmallmatrix}
			\ind_{A^\simm} & a \\
			a^* & h
		\end{psmallmatrix} \ge 0 \iff a^*a \le h$. 
	\end{enumerate}
\end{lemma}
\begin{beweis}
	Ohne Einschränkungen können wir $A \subset \mathcal{B}(\mathcal{H})$ und $M_2(A) \subset M_2(\mathcal{B}(\mathcal{H})) = \mathcal{B}(\mathcal{H} \oplus \mathcal{H})$ annehmen.
	\begin{enumerate}[(i)]
		\item Es gilt
		\begin{align}
			\skal*{\binom{\xi}{\eta}}{ \begin{pmatrix}
				\ind_{A^\simm} & a \\
				a^* & \ind_{A^\simm}
			\end{pmatrix} \binom{\xi}{\eta}} &= \skal*{\xi}{\xi} + \skal*{\xi}{a \eta} + \skal*{\eta}{a^* \xi} + \skal*{\eta}{\eta} \label{eq:38:1} \tag{$*$}\\
			&\ge \skal*{\xi}{\xi} - 2 \norm*{a} \cdot \norm*{\xi} \cdot \norm*{\eta} + \skal*{\eta}{\eta} \label{eq:38:2} \tag{$**$}
		\end{align}
		Falls $\norm*{a} \le 1$ ist also \eqref{eq:38:2} $\ge 0$ und somit $\begin{psmallmatrix}
			1 & a \\ a^* & 1
		\end{psmallmatrix} \ge 0$.
		Falls $\norm*{a} > 1$, so existieren $\xi,\eta \in \mathcal{H}$ mit $\norm*{\xi} = \norm*{\eta}=1$ und $\skal*{\xi}{a \eta} < -1$
		Damit folgt \eqref{eq:38:2} $<0$ und somit $\begin{psmallmatrix}
			1 & a \\ a^* & 1
		\end{psmallmatrix} \not\ge 0$.
		\item Angenommen $h \neq 0$. Für $\eta \in \mathcal{H}$ setzen wir $\xi \coloneqq - \sfrac{1}{\norm*{h}} \cdot a \cdot \eta$.
		Dann gilt
		\[
			\skal*{\binom{\xi}{\eta}}{\begin{pmatrix}
				h & a \\ a^* & h
			\end{pmatrix} \binom{\xi}{\eta}} = \ldots \le \sfrac{1}{\norm*{h}} \skal*{\eta}{a^* a \eta} - \sfrac{2}{\norm*{h}} \skal*{\eta}{a^* a \eta} + \skal*{\eta}{h \eta}
		\]
		und daraus folgt $\skal*{\eta}{ \enbrace*{h - \sfrac{1}{\norm*{h}} a^*a}\eta} \ge 0$ für $\eta \in \mathcal{H}$, also $h \ge \sfrac{1}{\norm*{h}} a^* a$.\marginnote{für $h=0$ den Bruch jeweils weglassen}
	\end{enumerate}
\end{beweis}

\begin{proposition}[label=prop:39]
	Sei $X$ ein Operatorsystem, $B$ eine unitale $C^*$-Algebra und $\varphi \colon X \to B$ unital und $2$-positiv.
	Dann gilt $\norm*{\varphi}=1$.
\end{proposition}
\begin{beweis}
	Sei $a \in X$ mit $\norm*{a} \le 1$. Mit \autoref{lem:38} (i) folgt 
	\[
		\begin{pmatrix}
			\ind_{X} & a \\ a^* & \ind_X
		\end{pmatrix} \ge 0 \implies \begin{pmatrix}
			\ind_B & \varphi(a) \\ \varphi(a)^* & \ind_B
		\end{pmatrix} \ge 0
	\]
	woraus mit \autoref{lem:38} (i) wieder $\norm*{\varphi(a)} \le 1$ folgt.
\end{beweis}

\begin{proposition}[{name={Cauchy-Schwarz}},label=prop:310]
	Es seien $A$ und $B$ unitale $C^*$-Algebren und $\varphi \colon A \to B$ unital und $2$-positiv.
	Dann gilt für $a \in A$
	\[
		\varphi(a)^* \varphi(a) \le \varphi(a^* a)
	\]
\end{proposition}
\begin{beweis}
	Es gilt
	\[
		\begin{pmatrix}
			\ind_A & a \\ a^* & a^* a
		\end{pmatrix} 
		=
		\begin{pmatrix}
			\ind_A & 0  \\ a^* & 0
		\end{pmatrix} \cdot 
		\begin{pmatrix}
			\ind_A & a \\ 0 & 0
		\end{pmatrix} \ge 0
	\]
	Da $\varphi$ $2$-positiv ist, folgt
	\[
		\begin{pmatrix}
			\ind_B & \varphi(a) \\ \varphi(a)^* & \varphi(a^* a) 
		\end{pmatrix} \ge 0
	\]
	woraus wir mit \autoref{lem:38} (ii) $\varphi(a)^* \varphi(a) \le \varphi(a^* a)$ erhalten.
\end{beweis}

\begin{definitionP}
	Sei $A$ eine $C^*$-Algebra.
	Dann ist
	\[
		\mathcal{F}(A) \coloneqq \set[\big]{a \in A \given \exists e \in A : \norm*{e}\le 1, a e = e a = a}
	\]
	dicht in $A$.
	Weiter ist $\mathcal{F}(A)_+ \subset A_+$ dicht und $\mathcal{F}(A)_\sa \subset A_\sa$ dicht.
\end{definitionP}
\begin{beweis}
	Sei $(u_\lambda)_{\Lambda}$ eine approximative Eins.
	Für $\varepsilon>0$ definiere $f_\varepsilon $ und $g_\varepsilon$ durch
	\missingfigure{Hier fehlen noch zwei Bilder von Funktionsgraphen}
	Es gilt $\lim_{\lambda, \varepsilon} f_\varepsilon(u_\lambda) a f_\varepsilon(u_\lambda) = a$ für alle $a \in A$ und $f_\varepsilon(u_\lambda) a f_\varepsilon(u_\lambda) \in \mathcal{F}(A)_{(+, \sa)}$, da $g_\varepsilon(u_\lambda) f_\varepsilon(u_\lambda) = f_\varepsilon(u_\lambda)$.
\end{beweis}

\begin{definition}[{name=[{Pedersen-Ideal}]}]
	\[
		\Ped(A) \coloneqq A \, \mathcal{F}(A)\, A \stackrel[\text{dicht}]{}{\subset} A
	\]
	ist das (eindeutig bestimmte) minimale dichte Ideal von $A$.\marginnote{$A$ unital $\Rightarrow$ $\Ped(A)=A$}
	$\Ped(A)$ heißt das \Index{Pedersen-Ideal}. 
\end{definition}

\begin{proposition}[label=prop:313]
	Seien $A,B$ beides $C^*$-Algebren, $\varphi \colon A \to B$ vollständig positiv und $\mu \ge \norm*{\varphi}$.
	Sei $\varphi^+ \colon A^+ \to B^+$ (bzw. $B^\simm$) die durch $\varphi^*(\ind_{A^+}) \coloneqq \mu \cdot \ind_{B^+}$ definierte lineare Fortsetzung.
	Dann ist $\varphi^+$ vollständig positiv und es gilt $\norm*{\varphi^+}= \mu$.
\end{proposition}
\begin{beweis}
	Wir müssen zeigen, dass $(\varphi^+)^\ssbrace{n}$ positiv ist: 
	Sei $\pi \colon A^+ \to \mathbb{C}$ der kanonische $^*$"=Homomorphismus.
	Dann ist auch $\pi^\ssbrace{n} \colon M_n(A^+) \to M_n$ ein $^*$-Homomorphismus, also insbesondere positiv.
	Sei $0 \le a + x  \otimes \ind_{A^+} \in M_n(A^+) = M_n \otimes A^+$ mit $a=a^* \in M_n(A)$, $x \ge 0 \in M_n$.\footnote{$\sum y_i \otimes a_i= \sum y_i \otimes (b_i + \lambda \cdot \ind_{A^+})$}
	Zu zeigen: 
	\[
		0 \le (\varphi^+)^\ssbrace{n} \enbrace*{a + x \otimes \ind_{A^+}} = \varphi^\ssbrace{n}(a) + \mu x \otimes \ind_{B^+}
	\]
	Es genügt dies für $a \in \mathcal{F}(M_n(A))_\sa$ zu zeigen, da $\mathcal{F}(M_n(A))_\sa \subset M_n(A)_\sa$ dicht, $\varphi^\ssbrace{n}$ beschränkt und $M_n(B^+)_+ \subset M_n(B_+)$ abgeschlossen ist (Warum genau?).
	Wir dürfen sogar annehmen, dass $e \in A_+$ existiert mit $\norm*{e}\le 1$ und $\overline{e}a = a \overline{e} =a$, wo $\overline{e}= \diag(e,\ldots ,e)=\ind_n \otimes e$.
	Dann ist
	\begin{align}
		(\varphi^+)^\ssbrace{n} (a + x \otimes \ind_{A^+}) &= (\varphi^+)^\ssbrace{n} \enbrace[\big]{\overline{e} \enbrace*{a + x \otimes \ind_{A^+}}\overline{e} + (1- \overline{e}^2) \enbrace*{ a + x \otimes \ind_{A^+}}} \\
		&= (\varphi^+)^\ssbrace{n} \enbrace[\Big]{\Underbracket{\overline{e} \enbrace*{a + x \otimes \ind_{A^+}} \overline{e}}{\ge 0 \in M_n(A)}} + (\varphi^+)^\ssbrace{n} \enbrace*{\Underbracket{x \otimes (1 - e^2)}{\ge 0 }} \\
		&\ge 0
	\end{align}
\end{beweis}

\begin{proposition}[label=prop:314]
	Seien $A,B$ $C^*$-Algebren und $\varphi \colon A \to B$ vollständig positiv.
	Dann ist $\varphi$ vollständig beschränkt und es gilt für eine approximative Eins $(u_\lambda)_\Lambda \subset A$
	\[
		\norm*{\varphi}_\vb = \norm*{\varphi} = \sup_\lambda \norm*{\varphi(u_\lambda)}
	\]
\end{proposition}
\begin{beweis}
	Es gilt offenbar $\norm*{\varphi(u_\lambda)} \le \norm*{\varphi} \le \norm*{\varphi}_\vb$.
	Zu zeigen bleibt $\norm*{\varphi}_\vb \le \sup_\lambda \norm*{\varphi(u_\lambda)}$.
	Sei also $a \in M_n(A)$ mit $\norm*{a} \le 1$ und $\overline{u}_\lambda = \diag (u_\lambda,\ldots ,u_\lambda)$.
	Dann ist
	\[
		0 \StackText{\ref{lem:38}(i)}{\le} \begin{pmatrix}
			\ind_{M_n(A)^\simm} & a \\ a^* & \ind_{M_n(A)^\simm}
		\end{pmatrix}
		\implies
		0 \le \begin{pmatrix}
			\overline{u}_\lambda^{\sfrac{1}{2}} & 0 \\ 0 & \overline{u}_\lambda^{\sfrac{1}{2}}
		\end{pmatrix}
		\begin{pmatrix}
			\ind & a \\ a^* & \ind
		\end{pmatrix}
		\begin{pmatrix}
			\overline{u}_\lambda^{\sfrac{1}{2}} & 0 \\ 0 & \overline{u}_\lambda^{\sfrac{1}{2}}
		\end{pmatrix}
	\]
	also gilt da $\varphi$ vollständig positiv ist
	\[
		0 \le \begin{pmatrix}
			\varphi^\ssbrace{n}(\overline{u}_\lambda) & \varphi^\ssbrace{n} \enbrace*{\overline{u}_\lambda^{\sfrac{1}{2}} a \overline{u}_\lambda^{\sfrac{1}{2}}} \\
			\varphi^\ssbrace{n}\enbrace*{\overline{u}_\lambda^{\sfrac{1}{2}} a^* \overline{u}_\lambda^{\sfrac{1}{2}}} & \varphi^\ssbrace{n}(\overline{u}_\lambda)
		\end{pmatrix}
	\]
	Mit \autoref{lem:38} (iii) folgt nun 
	\[
		\Underbracket{\norm*{\varphi^\ssbrace{n} \enbrace*{\overline{u}_\lambda^{\sfrac{1}{2}} a \overline{u}_\lambda^{\sfrac{1}{2}}}}}{\grenzw{} \norm*{\varphi^\ssbrace{n} (a)}} \le \norm*{\varphi^\ssbrace{n} (\overline{u}_\lambda)} = \norm*{\varphi(u_\lambda)} \grenzw{\lambda} \sup_\lambda \norm*{\varphi(u_\lambda)} \qedhere
	\]
\end{beweis}

\begin{bemerkung}[label=bem:315]
	Die Aussage gilt entsprechend, falls $A$ ein Operatorsystem ist.
\end{bemerkung}

\begin{proposition}[label=prop:316]
	Sei $A$ eine $C^*$-Algebra und $\varphi \colon A \to \mathbb{C}$ positiv.
	Dann ist $\varphi$ vollständig positiv und es gilt.
	Ebenso für $\varphi \colon X \to \mathbb{C}$ positiv, falls $X$ ein Operatorsystem ist.
\end{proposition}
\begin{beweis}
	Betrachte $0 \le (a_{ij})_{ij} \in M_n(A)$ und $\xi= (\xi_1, \ldots ,\xi_n)^T \in \mathbb{C}^n$.
	Dann gilt
	\begin{align}
		\skal[\big]{\xi}{\varphi^\ssbrace{n} \enbrace*{(a_{ij})_{ij}}} = \sum_{i,j=1}^n \varphi(a_{ij}) \overline{\xi}_i \xi_j &= \varphi \enbrace*{\sum_{i,j=1}^n \overline{\xi}_i \xi_j a_{ij}} \ge 0 \\
		&= 
	\end{align}
	Dann ist 
	\[
		M_n(A) \ni \begin{pmatrix}
					\overline{\xi}_1 \ind_{A^\simm} & \cdots & \overline{\xi}_n \ind_{A^\simm} \\
					& &  \\
					& 0 & 
				\end{pmatrix}\enbrace[\big]{a_{ij}}_{ij} 
				\begin{pmatrix}
					\xi_1 \ind_{A^\simm} & & \\
					\vdots & & 0 \\
					\xi_n \ind_{A^\simm}  & & 
				\end{pmatrix} \ge 0
	\]
\end{beweis}

\begin{korollar}[label=kor:317]
	Sei $\varphi \colon A \to C_0(\Omega)$ positiv, wobei $\Omega$ lokal kompakt und Hausdorff.
	Dann ist $\varphi$ vollständig positiv.
	Ebenso für $\varphi \colon X \to C_0(\Omega)$, wobei $X$ ein Operatorsystem ist.
\end{korollar}
\begin{beweis}
	Es gilt
	\[
		\varphi^\ssbrace{n} \enbrace[\big]{(a_{ij})_{ij}} \ge 0 \iff \varphi^\ssbrace{n} \enbrace[\big]{(a_{ij})_{ij}}(t) \ge 0 , \enspace t \in \Omega
	\]
	Aber $\varphi^\ssbrace{n}(\cdot )(t) = \enbrace*{\varphi(\cdot )(t)}^\ssbrace{n}$ ist vollständig positiv nach \autoref{prop:316}.
\end{beweis}

\begin{proposition}[label=prop:318]
	Sei $\Omega$ ein lokalkompakter Hausdorffraum und $B$ eine $C^*$-Algebra.
	Falls $\varphi \colon C_0(\Omega) \to B$ positiv ist, so ist $\varphi$ vollständig positiv.
\end{proposition}
\begin{beweis}
	Wähle $0 \le a \in M_n(C_0(\Omega))\cong M_n \otimes C_0(\Omega) \cong C_0(\Omega,M_n)$.
	Zu zeigen ist nun, dass $M_n(B) \ni \varphi^\ssbrace{n}(a) \ge 0$.
	Zu $\varepsilon>0$ existieren $0 \le f_1, \ldots ,f_k \in  C_0(\Omega)$ und $0 \le a_1, \ldots ,a_k \in M_n(A)$ mit
	\[
		\norm*{a - \sum\nolimits_{i=1}^{k} a_i \otimes f_i} < \varepsilon
	\]
	(warum?)
	Dann gilt 
	\begin{align}
		\varphi^\ssbrace{n}(a) \ge \varphi^\ssbrace{n} \enbrace*{\sum_{i=1}^k a_i \otimes f_i} - \varepsilon \cdot \norm*{\varphi} \cdot \ind_{M_n \otimes B^\simm} &= \sum_{i=1}^{k} \Underbracket{a_i}{\ge 0} \otimes  \Underbracket{\varphi(f_i)}{\ge 0} - \varepsilon \cdot \norm*{\varphi^\ssbrace{n}} \cdot \ind_{M_n \otimes B^\simm} \\
		&\ge 0 - \varepsilon \cdot \norm*{\varphi^\ssbrace{n}} \cdot \ind
	\end{align}
	Damit ist $\varphi^\ssbrace{n}(a) \ge 0$, da $\varepsilon$ beliebig war und $M_n(B)_+ \subset M_n(B)$ abgeschlossen.
\end{beweis}

\begin{satz}[label=satz:319,name={Stinespring}]
	Sei $A$ eine $C^*$-Algebra, $\varphi \colon A \to \mathcal{B}(\mathcal{H})$ vollständig positiv.
	Dann existieren ein Hilbertraum $\tilde{\mathcal{H}}$, ein $^*$-Homomorphismus $\pi \colon A \to \mathcal{B}(\tilde{\mathcal{H}})$ und ein $v \in \mathcal{B}(\mathcal{H},\tilde{\mathcal{H}})$ mit $\norm*{\varphi} = \norm*{v}^2$ und für $a \in A$ gilt \marginnote{$\begin{psmallmatrix}
		0 & 0 \\ v & 0
	\end{psmallmatrix} \in \mathcal{B}(\mathcal{H} \oplus \tilde{\mathcal{H}})$}
	\[
		\varphi(a) = v^* \pi(a) v
	\]
	Das heißt $v$ ist \Index{Kompression} eines $^*$-Homomorphismus.
	Falls $A,\varphi$ unital sind, so ist $v$ eine Isometrie.
\end{satz}
\begin{beweis}
	Ohne Beschränkung der Allgemeinheit können wir $\norm*{\varphi} =1$ annehmen.
	Nach \autoref{prop:313} können wir $\varphi$ zu $\varphi^+ \colon A^+ \to \mathcal{B}(\mathcal{H})$ unital und vollständig positiv fortsetzen.
	Wir dürfen also auch $A$ und $\varphi$ als unital annehmen.
	
	Definiere $\mathcal{H}' \coloneqq A \odot \mathcal{H}$ und eine symmetrische Bilinearform auf $\mathcal{H}'$ durch\marginnote{man beachte die Ähnlichkeiten zur GNS-Konstruktion}
	\[
		\skal*{a \otimes \xi}{b \otimes \eta}_{\mathcal{H}'} \coloneqq \skal*{\xi}{\varphi(a^* b) \eta}_{\mathcal{H}} \qquad a,b \in A, \enspace \xi, \eta \in \mathcal{H}
	\]
	Wir rechnen nach, dass diese Bilinearform positiv definit ist, da $\varphi$ vollständig positiv ist:
	\[
		\skal*{\sum_{i=1}^{n} a_i \otimes \xi_i}{\sum_{i=1}^{n} a_i \otimes \xi_i}_{\mathcal{H}'}\hspace{-1em} = \skal*{\begin{pmatrix} \xi_1 \\ \vdots \\ \xi_n \end{pmatrix}}{\varphi^\ssbrace{n} \enbrace[\big]{(a_i^*a_j)_{ij}}\begin{pmatrix} \xi_1 \\ \vdots \\ \xi_n \end{pmatrix}}_{\mathcal{H}^{\oplus n}} \hspace{-1em}\ge 0
	\]
	Setze nun $N \coloneqq \set*{x \in \mathcal{H}' \given \skal*{x}{x}_{\mathcal{H}'} =0}$.
	Wegen $\abs*{\skal*{x}{y}_{\mathcal{H}'}}^2 \le \skal*{x}{x}_{\mathcal{H}'} \cdot \skal*{y}{y}_{\mathcal{H}'} $ gilt 
	\[
		N = \set*{x \in \mathcal{H}' \given \skal*{x}{y}_{\mathcal{H}'}=0 \text{ für } y \in \mathcal{H}'}
	\]
	Daher ist $N \subset \mathcal{H}'$ ein abgeschlossener Unterraum und wir setzen $\tilde{\mathcal{H}} \coloneqq \overline{\sfrac{\mathcal{H}'}{N}}^{\skal*{\cdot}{\cdot}_{\mathcal{H}'}}$.
	Dies ist offenbar ein Hilbertraum.
	Definiere nun $\pi' \colon A \to \mathcal{L}(\tilde{\mathcal{H}})$ durch
	\[
		\pi'(a) \enbrace*{\sum_{i=1}^{n} a_i \otimes \xi_i} \coloneqq \sum_{i=1}^{n} a a_i \otimes \xi_i
	\]
	Es gilt dann
	\begin{align}
		\enbrace*{a_i^* a^* aa_j}_{ij} &= \begin{pmatrix}
			a_1^* & & \\
			\vdots & & 0 \\
			a_n^* & & 
		\end{pmatrix} \cdot \begin{pmatrix}
			a^*a & & 0 \\
			& \ddots & \\
			0 & & a^*a
		\end{pmatrix} \cdot \begin{pmatrix}
			a_1 & \cdots & a_n \\
			& \rule{0cm}{1.6em} & \\
			& 0 & 
		\end{pmatrix} \\
		&\le \norm*{a^*a} \cdot \begin{pmatrix}
			a_1^* & & \\
			\vdots & & 0 \\
			a_n^* & & 
		\end{pmatrix} \cdot \begin{pmatrix}
			a_1 & \cdots & a_n \\
			& \rule{0cm}{1.6em} & \\
			& 0 & 
		\end{pmatrix} = \norm*{a}^2 \cdot (a_i^*a_i)_{ij}
	\end{align}
	Damit können wir zeigen, dass $\pi'$ kontraktiv ist:
	\begin{align}
		\skal*{\pi'(a) \enbrace*{\sum_{i=1}^{n} a_i \otimes \xi_i}}{\pi'(a) \enbrace*{\sum_{i=1}^{n} a_i \otimes \xi_i}}_{\mathcal{H}'} &= \skal*{\begin{pmatrix} \xi_1 \\ \vdots \\ \xi_n \end{pmatrix}}{\varphi^\ssbrace{n} \enbrace[\big]{(a_i^*a^*a a_j)_{ij}}\begin{pmatrix} \xi_1 \\ \vdots \\ \xi_n \end{pmatrix}}_{\mathcal{H}^{\oplus n}} \\
		&\le \norm*{a}^2_A \cdot \skal*{\sum_{i=1}^{n} a_i \otimes \xi_i}{\sum_{i=1}^{n} a_i \otimes \xi_i}
	\end{align}
	Damit folgt $\norm*{\pi'(a)}_{\mathcal{H}'} \le \norm*{a}_A$ und $\pi'$ induziert einen unitalen $^*$-Homomorphismus $\pi \colon A \to \mathcal{B}(\tilde{\mathcal{H}})$.
	Definiere $v \in \mathcal{B}(\mathcal{H},\tilde{\mathcal{H}})$ durch $v(\xi) \coloneqq \ind_A \otimes \xi + N$.
	Dann sind $\tilde{\mathcal{H}}$, $\pi$ und $v$ wie gewünscht.
\end{beweis}

\begin{lemma}[label=lem:320]
	Seien $A,B$ beide $C^*$-Algebren und $\varphi \colon A \to B$ vollständig positiv und kontraktiv.
	Dann gilt:
	\begin{enumerate}[a)]
		\item Für $x,y \in A$ gilt
		\[
			\norm[\big]{\varphi(xy) -\varphi(x) \varphi(y)}_B \le \norm[\big]{\varphi(xx^*) -\varphi(x) \varphi(x^*)}_B^{\sfrac{1}{2}} \cdot \norm*{y}_A
		\]
		\item Für den \emph{\enquote{multiplikativen Bereich}} 
		\[
			M \coloneqq \set[\big]{a \in A \given \varphi(a^*a) = \varphi(a)^* \varphi(a), \varphi(aa^*)= \varphi(a)\varphi(a)^*} \subseteq A
		\]
		von $\varphi$ gilt
		\begin{enumerate}[i)]
			\item $M= \set[\big]{a \in A \given \varphi(ax)=\varphi(a)\varphi(x), \varphi(xa)=\varphi(x) \varphi(a) , x \in A}$
			\item $M \subseteq A$ ist eine $C^*$-Unteralgebra
			\item $\varphi\big|_M$ ist ein $^*$-Homomorphismus
		\end{enumerate}
	\end{enumerate}
\end{lemma}
\begin{beweis}
	\begin{enumerate}[a)]
		\item Ohne Beschränkung der Allgemeinheit können wir $A,B,\varphi$ als unital annehmen.
		Nach \hyperref[satz:319]{\textsc{Stinespring}} dürfen wir annehmen, dass $\varphi$ von der Form $\varphi(.) = v^* \pi(.)v$ für einen $^*$-Homomorphismus $\pi$ und eine Isometrie $v$ ist.
		Dann gilt
		\begin{align}
			\norm[\big]{\varphi(xy) -\varphi(x) \varphi(y)}_B = \norm[\big]{v^*\pi(xy)v - v^*\pi(x)vv^*\pi(y)v} &\le \norm*{v^* \pi(x) - v^*\pi(x)vv^*} \cdot \norm*{\pi(y)v} \\
			&\le \norm*{v^* \pi(x)(\ind -vv^*)} \cdot \norm*{y} \marginnote{$\ind- vv^*$ ist Projektion, $C^*$-Gleichung} \\
			&= \norm*{v^* \pi(x) (\ind -vv^*) \pi(x^*) v}^{\sfrac{1}{2}} \cdot \norm*{y} \\
			&= \norm*{\varphi(xx^*) - \varphi(x)\varphi(x)^*}^{\sfrac{1}{2}} \cdot \norm*{y}
		\end{align}
		\item Folgt direkt aus a). \qedhere
	\end{enumerate}
\end{beweis}

\begin{lemma}
	Es seien $A,B$ $C^*$-Algebren und $\begin{tikzcd}[cramped,sep=small] A \rar["\psi"] & B \rar["\varphi"] & A \end{tikzcd}$ vollständig positiv und kontraktiv.
	Dann gilt für $a \in A^1_+$ und $b \in B$
	\[
		\norm[\big]{\varphi \enbrace*{\psi(a) \cdot b} - \varphi \enbrace*{\psi(a)} \varphi(b)} \le 3^{\sfrac{1}{2}} \cdot  \norm*{b} \cdot \Underbracket{\max \set[\Big]{\norm[\big]{\varphi \enbrace*{\psi(a)}-a}, \norm*{\varphi \enbrace*{\psi(a^2)} -a^2}}}{=: \eta}^{\sfrac{1}{2}}
	\]
\end{lemma}
\begin{beweis}
	Nach \autoref{prop:310} gilt
	\[
		0 \le \varphi \enbrace*{\psi(a)^2} - \varphi \enbrace[\big]{\psi(a)}^2 \le \varphi \enbrace*{\psi(a^2)} - \varphi \enbrace[\big]{\psi(a)}^2 \le (\eta + \eta +\eta) \cdot \ind
	\]
	Damit erhalten wir
	\[
		\norm[\big]{\varphi \enbrace*{\psi(a)b} - \varphi \enbrace*{\psi(a)} \varphi(b)} \StackText{\ref{lem:320}}{\le} \norm*{\varphi \enbrace*{\psi(a)^2} - \varphi \enbrace[\big]{\psi(a)}^2}^{\sfrac{1}{2}} \cdot \norm*{b} \le (3 \eta)^{\sfrac{1}{2}} \cdot \norm*{b} \qedhere
	\]
\end{beweis}

\begin{proposition}
	Sei $A$ eine $C^*$-Algebra.
	Für $(a_1,\ldots ,a_n)$ mit $a_i \in A$ ist $(a_i^* a_j)_{i,j} \in M_n(A)_+$.
	Jedes $h \in M_n(A)_+$ lässt sich als Summe von $n$ solchen Elementen schreiben.
\end{proposition}
\begin{beweis}
	Sei $h=x^*x$ mit $x \in M_n(A)$.
	Wir schreiben $x = x_1 + \ldots + x_n$, wobei $x_i$ die Matrix ist, die nur aus der $i$-ten Zeile von $x$ (und sonst Nullen) besteht.
	Dann gilt
	\[
		h = x^* x = \sum_{i,j} x_i^* x_j = \sum_{k=1}^{n} x_k^* x_k = \sum_{k=1}^{n} \enbrace*{x_{ki}^* x_{kj}}_{ij} \qedhere
	\]
\end{beweis}

\begin{satz}[label=satz:323]
	Sei $B$ eine $C^*$-Algebra und $\varphi \colon M_n \to B$ linear.
	Dann sind äquivalent:
	\begin{enumerate}[(i)]
		\item $\varphi$ ist vollständig positiv
		\item $\varphi$ ist $n$-positiv
		\item $\enbrace*{\varphi(e_{ij})}_{ij} \in M_n(B)$ ist positiv
	\end{enumerate}
\end{satz}
\begin{beweis}
	Die Implikation (i) $\Rightarrow $ (ii) ist trivial.
	Für die zweite Implikation stellen wir zunächst fest, dass\marginnote{$e_{ik} e_{lj} = e_{ij}$}
	\[
		0 \neq (e_{ij})_{ij} = \begin{pmatrix}
			e_{11} & \hphantom{\cdots} & \hphantom{e_{1n}}\\
			\vdots & & 0 \\
			e_{n1} & &
		\end{pmatrix} \begin{pmatrix}
			e_{11} & \cdots & e_{1n} \\
			\rule{0cm}{1.6em} & & \\
			& 0 &
		\end{pmatrix} \in M_n(M_n)
	\]
	gilt.
	Da $\varphi$ $n$-positiv ist, folgt
	\[
		0 \le \varphi^\ssbrace{n} \enbrace[\big]{(e_{ij})_{ij}} = \enbrace[\big]{\varphi(e_{ij})}_{ij} \in M_n(B)
	\]
	\textit{Die verbleibe Implikation ist die erste Aufgabe von Übungsblatt 5.}\todo{eventuell \TeX{}en}
\end{beweis}

\begin{bemerkung}[label=bem:324]
	Sei $M$ ein Operatorraum, das heißt ein linearer Unterraum einer $C^*$-Algebra.
	Zu $\varphi \colon M \to M_n$ lineare assoziiere $s_\varphi \colon M_n(M) \to \mathbb{C}$ linear durch
	\[
		s_\varphi\enbrace[\big]{(a_{ij})_{ij}} = \frac{1}{n} \sum_{i,j} \varphi(a_{ij})_{ij}
	\]
	Zu $s \colon M_n(M) \to \mathbb{C}$ linear assoziiere $\varphi_s \colon M \to M_n$ linear durch
	\(
		\varphi_s (a)_{ij} = n \cdot s \enbrace*{a \otimes e_{ij}}
	\).
	Die Abbildungen 
	\[
		\begin{tikzcd}[sep=large]
			\mathcal{L}(M,M_n) \rar["\varphi \mapsto s_\varphi",yshift=.7ex] & \mathcal{L} \enbrace*{M_n(M),\mathbb{C}} \lar["\varphi_s \reflectbox{$\mapsto$}s",yshift=-.7ex]
		\end{tikzcd}
	\]
	sind invers zueinander (warum?).
	Wenn $\varphi$ unital ist, so ist auch $s_\varphi$ unital, die Umkehrung gilt allerdings nicht (warum?).
\end{bemerkung}

\begin{satz}[label=satz:325]
	Sei $A$ eine $C^*$-Algebra und $\varphi \colon A \to M_n$ linear.
	Dann sind äquivalent:
	\begin{enumerate}[(i)]
		\item $\varphi$ ist vollständig positiv
		\item $\varphi$ ist $n$-positiv
		\item $s_\varphi$ ist positiv
	\end{enumerate}
\end{satz}
\begin{beweis}
	Die Implikation (i) $\Rightarrow $ (ii) ist trivial. Für die nächste Implikation betrachte $\eta = \eta_1 \oplus \ldots \oplus \eta_n \in \mathbb{C}^n \oplus \ldots \oplus \mathbb{C}^n \cong \mathbb{C}^{n^2}$, wobei $\eta_i$ der $i$-te Einheitsvektor ist.
	Es gilt
	\[
		\sfrac{1}{n} \cdot \skal[\big]{\eta}{\varphi^\ssbrace{n} \enbrace*{(a_{ij})_{ij}} \,\eta} = s_\varphi \enbrace[\big]{(a_{ij})_{ij}}
	\]
	Also gilt weiter
	\[
		M_n(A) \ni (a_{ij})_{ij} \ge 0 \implies \varphi^\ssbrace{n} \enbrace*{(a_{ij})_{ij}} \ge 0 \implies s_\varphi \enbrace*{(a_{ih})_{ij}} \ge 0
	\] 
	Für die verbleibende Implikation (iii) $\Rightarrow $ (i) ist zu zeigen, dass $\varphi^\ssbrace{m} \colon M_m(A) \to M_m(M_n)$ positiv ist für $n \in \mathbb{N}$.
	Es ist $M_m(M_n) = M_m \otimes M_n = \mathcal{B} \enbrace*{\bigoplus_{i=1}^m \mathbb{C}^n}$.
	Wegen \autoref{satz:323} genügt es, 
	\[
		\varphi^\ssbrace{m} \enbrace[\big]{(a_i^* a_j)_{i,j=1, \ldots ,m}} \ge 0 
	\]
	zu zeigen.
	Betrachte $\xi = \xi_1 \oplus \ldots \oplus \xi_m \in \mathbb{C}^n \oplus  \ldots \oplus \mathbb{C}^n$ mit $\xi_j = \sum_{k=1}^n \lambda_{jk} \eta_k$ (wobei die $\eta_i$ wieder die Standardbasis von $\mathbb{C}^n$ sind).
	Es gilt
	\begin{align}
		\skal*{\xi}{\varphi^\ssbrace{m} \enbrace[\big]{ \enbrace*{a_i^* a_j}_{ij}}} = \sum_{i,j=1}^{m} \skal*{\xi_i}{\varphi \enbrace*{a_i^* a_j}_{ij} \xi_j} 
		&= \sum_{i,j=1}^{m} \sum_{k,l=1}^{n} \overline{\lambda_{ik}}\cdot \lambda_{jl} \cdot \Underbracket{\skal[\big]{\eta_k}{\varphi(a_i^* a_j) \eta_l}}{\text{$kl$-Eintrag von $\varphi(a_i^*a_j)$}} \\
		&= \sum_{i,j=1}^{m} \sum_{k,l=1}^{n} \overline{\lambda_{ik}} \cdot \lambda_{jl} \cdot n \cdot s_\varphi \enbrace*{a_i^* a_j \otimes e_{kl}} \\
		&= n \cdot \sum_{i,j=1}^{m} s_\varphi \enbrace*{\sum_{k,l=1}^{n} \overline{\lambda_{ij}} \lambda_{jl} \cdot a_i^* a_j \otimes e_{kl}} \\
		\shortintertext{mit $A_j = \begin{psmallmatrix}
			\lambda_{j1} & \cdots & \lambda_{jn} \\
			& & \\
			 & 0 & 
		\end{psmallmatrix}$ sodass $A_i^*A_j = \sum_{k,l=1}^{n} \overline{\lambda_{ik}} \lambda_{jl} e_{kl}$}
		&= n \cdot \sum_{i,j=1}^{m}  s_\varphi \enbrace*{a_i^*a_j \otimes A_i^* A_j} \ge 0 \qedhere
	\end{align}
\end{beweis}

\begin{satz}[label=satz:326]
	Sei $B \subset A $ eine $C^*$-Unteralgebra und $\varphi \colon B \to M_n$ vollständig positiv und kontraktiv.
	Dann besitzt $\varphi$ eine vollständig positive, kontraktive Fortsetzung $\overline{\varphi} \colon A \to M_n$.
	
	Ebenso für ein Operatorsystem $X \subset A$ und $\varphi \colon X \to M_n$.
\end{satz}
\begin{beweis}
	Ohne Einschränkungen können wir annehmen, dass $\ind \in B \subset A$ und $\varphi$ unital ist.
	Dann ist $s_\varphi \in S(M_n(B))$ und $s_\varphi$ besitzt eine Fortsetzung $\overline{s} \in S(M_n(A))$.
	$\overline{\varphi} \coloneqq \varphi_{\overline{s}} \colon A \to M_n$ setzt nun $\varphi$ fort und die Behauptung ist gezeigt.
\end{beweis}

\begin{satz}[label=satz:327,name={Arveson}]
	Sei $B \subset A$ ein $C^*$-Unteralgebra und $\varphi \colon B \to \mathcal{B}(\mathcal{H})$ vollständig positiv und kontraktiv.
	Dann besitzt $\varphi$ eine vollständig positive, kontraktive Fortsetzung $\overline{\varphi} \colon A \to \mathcal{B}(\mathcal{H})$.
	
	Ebenso für ein Operatorsystem $X \subset A$ und $\varphi \colon X \to \mathcal{B}(\mathcal{H})$.
\end{satz}
\begin{beweis}[name=Idee]
	Für zwei Banachräume $E,F$ betrachtet man $\varepsilon \colon E \odot F \to \mathcal{B} \enbrace*{E,F^*}^*$ definiert durch $e \otimes f \mapsto  \enbrace*{T \mapsto  T(e)(f)}$.
	Setze $Z \coloneqq \varepsilon(E \odot F)$.
	Man überlegt sich nun, dass $\mathcal{B} \enbrace*{E,F^*} \to Z^*$, $T \mapsto \ev_T$ ein isometrischer Isomorphismus ist.
	Weiter ist
	\[
		\mathcal{B}(\mathcal{H}) = \mathcal{B}(\mathcal{H},\mathcal{H}) \cong \mathcal{B}(\mathcal{H},\mathcal{H}^*) \cong Y^*
	\]
	also ein Dualraum; ebenso wie $\mathcal{B} \enbrace*{A, \mathcal{B}(\mathcal{H})} \cong \mathcal{B}\enbrace*{A,Y^*}\cong V^*$.
	Folglich lässt sich $\mathcal{B} \enbrace*{A, \mathcal{B}(\mathcal{H})}$ mit einer $\w^*$-Topologie austatten, BW-Topologie\todo{?}.
	Man zeigt nun
	\[
		\mathrm{CPC} \enbrace*{A, \mathcal{B}(\mathcal{H})} \coloneqq \set*{\psi \colon A \to \mathcal{B}(\mathcal{H}) \given \text{vollst. pos. \& kontraktiv}} \stackrel[\text{BW abg.}]{}{\subset} \mathcal{B} \enbrace*{A,\mathcal{B}(\mathcal{H})}^1
	\]
	Also ist $\mathrm{CPC}(A,\mathcal{B}(\mathcal{H}))$ BW-kompakt nach Banach-Alaoglu.
	Sei nun $\mathbb{C}^n \cong \mathcal{F} \subset \mathcal{H}$ ein endlich-dimensionaler Unterraum.
	Setze $\varphi_{\mathcal{F}} \coloneqq \pr_{\mathcal{F}} \circ \varphi \colon B \to \pr_{\mathcal{F}} \mathcal{B}(\mathcal{H}) \pr_{\mathcal{F}} = \mathcal{B}(\mathcal{F}) = M_n$.
	Nach \autoref{satz:326} besitzt $\varphi_{\mathcal{F}}$ eine vollständig positive, kontraktive Fortsetzung $\overline{\varphi_{\mathcal{F}}} \colon A \to \mathcal{B}(\mathcal{F}) \hookrightarrow \mathcal{B}(\mathcal{H})$.
	\[
		\enbrace*{\overline{\varphi_{\mathcal{F}}}}_{\mathcal{F} \subset \mathcal{H} \text{ endl.dim.}} \subset \mathrm{CPC} \enbrace*{A,\mathcal{B}(\mathcal{H})} 
	\]
	ist ein durch Inklusion gerichtetes Netz.
	Mit der eben gezeigten Kompaktheit folgt, dass $\enbrace*{\overline{\varphi_{\mathcal{F}}}}_{\mathcal{F}}$ ein konvergentes Teilnetz besitzt mit Limes $\overline{\varphi}$.
	$\overline{\varphi}$ setzt dann $\varphi$ fort (warum?).
\end{beweis}

\begin{definition}[{name=[injektive C*-Algebra]}]
	Eine unitale $C^*$-Algebra $C$ heißt \Index{injektiv}, falls folgendes gilt:
	Für ein Operatorsystem $X \subset A$ und eine vollständig positive, kontraktive Abbildung $\varphi \colon X \to C$ existiert eine vollständig positive, kontraktive Fortsetzung $\overline{\varphi} \colon A \to C$.
\end{definition}
% section 3 (end)
\newpage

\section{Die Sätze von Choi-Effros} % (fold)
\label{sec:4}

\todo[inline]{RevChap 4}

\begin{proposition}[label=prop:41]
	Sei $B$ eine $C^*$-Algebra und $J \lhd B$ ein Ideal.
	Sei $\varphi \colon M_n \to \sfrac{B}{J}$ vollständig positiv  unital (bzw. kontraktiv).
	Dann hat $\varphi$ einen vollständig positiv unitalen (bzw. kontraktiven) Lift $\overline{\varphi} \colon M_n \to B$, das heißt $\pi \circ \overline{\varphi} = \varphi$, wobei $\pi$ die Quotientenabbildung ist.
\end{proposition}
\begin{beweis}
	Es ist 
	\[
		0 \le \varphi^\ssbrace{n}\enbrace[\bigg]{\Underbracket{\sum_{i,j=1}^{n} e_{ij} \otimes e_{ij}}{\ge 0}} = \sum_{i,j} \varphi(e_{ij}) \otimes e_{ij} \in \sfrac{B}{J} \otimes M_n = \sfrac{B \otimes M_n}{J \otimes M_n}
	\]
	und somit existiert $0 \le \sum_{i,j=1}^{n} b_{ij} \otimes e_{ij} \in B \otimes M_n$ mit $\sum_{i,j} \varphi(e_{ij}) \otimes e_{ij} = \sum_{i,j} \pi(b_{ij}) \otimes e_{ij}$.
	Dann muss bereits $\varphi(e_{ij})=b_{ij}$ für alle $i,j=1,\ldots,n$ gelten.
	Definiere nun $\varphi' \colon M_n \to B$ linear durch $(\alpha_{ij})_{ij} \mapsto \sum_{i,j=1}^{n} \alpha_{ij} \cdot b_{ij}$.
	Dann gilt $\pi \circ \varphi' = \varphi$, das heißt $\varphi'$ ist ein Lift.
	Mit 
	\[
		\enbrace[\big]{\varphi'(e_{ij})}_{ij} = \sum_{i,j} \varphi'(e_{ij}) \otimes e_{ij} = \sum_{ij} b_{ij} \otimes e_{ij} \ge 0
	\]
	folgt aus \autoref{satz:323} (iii) $\Rightarrow$ (i), dass $\varphi'$ ein vollständig positiver Lift ist.
	$\varphi'$ ist aber nicht notwendig unital.
	
	Seien nun $B$ und $\varphi$ unital.
	Setze $h \coloneqq \varphi' (\ind_n) - \ind_B$, dann ist $h \in J = \ker \pi$.
	Wähle $\rho \in S(M_n)$ beliebig und definiere $\overline{\varphi} \colon M_n \to B$ durch
	\[
		\overline{\varphi}(x) \coloneqq (\ind_B + h_+)^{-\sfrac{1}{2}}\enbrace[\big]{\varphi'(x)+ \rho(x) \cdot h_-} \enbrace*{\ind_B + h_+}^{-\sfrac{1}{2}} \in \varphi'(x) + J
	\]
	Dann ist $\overline{\varphi}$ ein vollständig positiver Lift und es gilt $\overline{\varphi}(\ind_n) = \ind_B$, denn
	\[
		\varphi(\ind_n) + \ind \cdot h_- = \ind_B + h_+ \qedhere
	\]
\end{beweis}

Sei $A$ eine separable $C^*$-Algebra und $\set*{a_1, a_2, \ldots } \subset A^1$ dicht.
Für $\varphi,\psi \colon A \to B$ vollständig positiv und kontraktiv setze 
\[
	d_B(\varphi,\psi) \coloneqq \sum_{k=1}^{\infty} \frac{1}{2^k} \norm[\big]{\varphi(a_k) - \psi(a_k)}
\]
Dann ist $d_B$ eine Metrik auf $\CPC(A,B)$ (bzw. $\CP(A,B)$ und $\UPC(A,B)$).
Es gilt 
\[
	d_B(\varphi_\lambda,\varphi) \grenzw{\lambda} 0 \iff \norm*{\varphi_\lambda(a) - \varphi(a)} \grenzw{\lambda} 0, \qquad a \in A
\]
$(\varphi_\lambda)_\Lambda$ ist also genau dann Cauchy bezüglich $d_B$, wenn $(\varphi_\lambda(a))_\Lambda$ Cauchy ist für $a \in A$ für beschränkte Netze $(\varphi_\lambda)_\Lambda$ (warum?) 

\begin{proposition}[label=prop:42]
	Für $J \lhd B$ und $\pi \colon B \to \sfrac{B}{J}$ gilt
	\[
		d_{\sfrac{B}{J}} \enbrace*{\pi \varphi, \pi \psi} \le d_B(\varphi,\psi)
	\]
	und $d_{\sfrac{B}{J}} \enbrace*{\pi \varphi, \pi \psi} = \inf \set[\big]{d_B(\varphi,\psi') \given \psi \colon A \to B \text{ v.p. kontraktiv und } \pi \psi' = \pi \psi}$
\end{proposition}
\begin{beweis}
	Wir müssen nur die zweite Gleichung beweisen.
	Dazu zeigen wir zunächst \enquote{$\le$}: 
	\[
		d_{\sfrac{B}{J}} \enbrace*{\pi \varphi, \pi \psi} = \sum_{k} 2^{-k} \norm[\big]{\pi \varphi(a_k) - \pi \psi(a_k)} \le \sum_k 2^{-k} \norm[\big]{\varphi(a_k) - \psi(a_k)} = d_B(\varphi,\psi)
	\]
	Für \enquote{$\ge$} sei $(h_\lambda)_\Lambda \subset J$ eine \Index{quasizentrale approximative Eins}, das heißt $(h_\lambda)_\Lambda$ ist approximative Eins für $J$ und $\benbrace*{h_\lambda,b} \grenzwIn{\lambda} 0$ für alle $b \in B$.\footnote{Sei $(e_\gamma)_\Gamma$ eine approximative Eins für $J$, dann existiert $(h_\lambda)_\Lambda$ quasizentral in $\conv\set*{e_\gamma \given \gamma \in  \Gamma}$ im Wesentlichen nach Hahn-Banach.}
	Für $b \in B$ gilt 
	\begin{equation}
		\norm*{\pi(b)} = \lim_\lambda \norm*{b(\ind_B - h_\lambda)} = \lim_\lambda \norm*{\enbrace*{\ind_B -h_\lambda}^{\sfrac{1}{2}} b \enbrace*{\ind_B - h_\lambda}^{\sfrac{1}{2}}} \label{eq:42:1} \tag{\#}
	\end{equation}
	Wir setzen $\varphi_\lambda(.) = \enbrace*{\ind_B - h_\lambda}^{\sfrac{1}{2}}\varphi(.) \enbrace*{\ind_B - h_\lambda}^{\sfrac{1}{2}}$, $\psi_\lambda$ analog und
	\[
		\overline{\varphi}_\lambda(.)  \coloneqq \varphi_\lambda(.) + h_\lambda^{\sfrac{1}{2}} \varphi(.) h_\lambda^{\sfrac{1}{2}} \qquad \overline{\psi}_\lambda \coloneqq \psi_\lambda(.) + h_\lambda^{\sfrac{1}{2}} \varphi(.) h_\lambda^{\sfrac{1}{2}}
	\]
	Dann sind $\varphi_\lambda, \psi_\lambda, \overline{\varphi}_\lambda, \overline{\psi}_\lambda$ sind vollständig positiv und kontraktiv.
	Weiter gilt $\pi \varphi_\lambda= \pi \overline{\varphi}_\lambda = \pi \varphi$, $\pi \psi_\lambda = \pi \overline{\psi}_\lambda = \pi \psi$ sowie $d_B(\overline{\varphi}_\lambda,\varphi) \grenzw{\lambda} 0$.
	Es gilt nun
	\begin{align}
		d_{\sfrac{B}{J}} \enbrace*{\pi \varphi, \pi \psi} = \lim_\lambda d_B(\varphi_\lambda,\psi_\lambda) &= \lim_\lambda d_B \enbrace*{\overline{\varphi}_\lambda, \overline{\psi}_\lambda} \\
		&\le \inf \set[\big]{d_B(\varphi,\psi') \given \psi' \colon A \to B \text{v.p. kontraktiv, } \pi \psi' = \pi \psi}
	\end{align}
	wegen \eqref{eq:42:1}.
\end{beweis}

\begin{proposition}[label=prop:43]
	Seien $A$ und $J \lhd B$ wie oben.
	$\varphi_n \colon A \to \sfrac{B}{J}$ vollständig positiv und kontraktiv (unital) für $n \in \mathbb{N}$ mit vollständig positiven kontraktiven (unitalen) Lifts $\Theta_n \colon A \to B$ für $n \in \mathbb{N}$.
	Falls $\varphi \colon A \to \sfrac{B}{J}$ vollständig positiv kontraktiv (unital) ist mit $d_{\sfrac{B}{J}}(\varphi_n,\varphi) \grenzwIn{n \to \infty} 0$, so existiert ein vollständig positiver kontraktiver (unitaler) Lift $\overline{\varphi} \colon A \to B$ für $\varphi$.
	
	Anders ausgedrückt:
	\[
		\set[\big]{\psi \in \CPC(A,\sfrac{B}{J}) \given \psi \text{ hat v.p. kontraktiven Lift}} \stackrel[\text{$d_{\sfrac{B}{J}}$-abg.}]{}{\subset} \CPC(A,\sfrac{B}{J})
	\]
\end{proposition}
\begin{beweis}
	Wir dürfen $d_{\sfrac{B}{J}}(\varphi_n, \varphi) < \frac{1}{2^{n+1}}$ annehmen.
	Setze $\psi_0 \coloneqq 0$ und $\psi_1 \coloneqq \Theta_1$.
	Es seien nun $\psi_0, \ldots , \psi_n \colon A \to B$ vollständig positiv kontraktiv bereits konstruiert mit $\pi \psi_k = \varphi_k$ und $d(\psi_{k-1}, \psi_k) \le \frac{1}{2^{k-1}}$ für $k =1,\ldots ,n$.
	Es gilt
	\begin{align}
		\inf \set[\big]{d_B(\psi_n,\Theta) \given \Theta \colon A \to B \text{ v.p. kontraktiv, } \pi \circ \Theta = \varphi_{n+1}} &\StackTextClap{\ref{prop:42}}{=}d_{\sfrac{B}{J}} \enbrace*{\pi \circ \psi_n, \pi \circ \Theta_{n+1}} \\
		&= d_{\sfrac{B}{J}} \enbrace*{\varphi_n,\varphi_{n+1}} < \frac{1}{2^n} 
	\end{align}
	Es existiert also ein $\psi \colon A \to B$ vollständig positiv und kontraktiv, mit $\pi \psi = \varphi_{n+1}$ und $d_{B} \enbrace*{\psi_n,\psi} < \sfrac{1}{2^n}$.
	Setze $\psi_{n+1} \coloneqq \psi$.
	Induktiv erhalten wir eine Cauchy-Folge $(\psi_n)_\mathbb{N} \subset \CPC(A,B)$.
	Dann ist weiter auch $\enbrace*{\psi_n(a)}_\mathbb{N}$ Cauchy für jedes $a \in A$ und somit definiert $\overline{\varphi}(a) \coloneqq \lim_n \psi_n(a)$ ein vollständig positive kontraktive Abbildung $\overline{\varphi} \colon A \to B$ mit $\pi \overline{\varphi} = \varphi$ (warum?).
\end{beweis}

\begin{definition}
	Seien $A,B$ $C^*$-Algebren und $\varphi \colon A \to B$ vollständig positiv und kontraktiv.
	Dann heißt $\varphi$ \Index{nuklear}, falls gilt: 
	$\forall \varepsilon >0, \mathcal{F} \finSub A$ endlich existiert eine endlichdimensionale $C^*$-Algebra $F$ und 
	\[
		\begin{tikzcd}
			A \rar["\rho"] & F \rar["\sigma"] & B
		\end{tikzcd}
	\]
	vollständig positiv und kontraktiv mit $\norm*{\sigma \rho (a )- \varphi(a) } < \varepsilon$ für $a \in \mathcal{F}$.
	\begin{itemize}
		\item Mit anderen Worten: Es existiert ein Netz
		\[
			\enbrace*{A \xrightarrow{\rho_\lambda} F_\lambda \xrightarrow{\sigma_\lambda} B}_{\lambda \in \Lambda} \text{ mit } \sigma_\lambda \circ \rho_\lambda \grenzw{\lambda} \varphi \text{ punktweise}  
		\]
		\item Man kann stets $F_\lambda$ durch $M_{r_\lambda}$ ersetzen.
		\item Für $A$ separabel kann man $\Lambda = \mathbb{N}$ wählen.
	\end{itemize}
\end{definition}

\begin{satz}[{name={Choi-Effros}}]
	Sei $A$ eine separable $C^*$-Algebra, $J \lhd B$ ein Ideal und $\varphi\colon A \to \sfrac{B}{J}$ vollständig positiv kontraktiv.
	Falls $\varphi$ nuklear ist, so existiert ein vollständig positiver kontraktiver Lift $\overline{\varphi} \colon A \to B$ für $\varphi$.
	
	Ebenso für $\varphi \colon X \to \sfrac{B}{J}$, wobei $X$ ein Operatorsystem ist.
\end{satz}
\begin{beweis}
	Wähle $\big(\begin{tikzcd}[cramped,sep=small] A \rar["\rho_n"] & M_{r_n} \rar["\sigma_n"] & \sfrac{B}{J}\end{tikzcd}\big)_\mathbb{N}$ ein System von vollständig positiven kontraktiven Approximationen mit $\sigma_n \rho_n \grenzwIn{n \to \infty} \varphi$ punktweise.
	Dann gilt $d_{\sfrac{B}{J}}(\sigma_n \rho_n,\varphi) \grenzwIn{n \to \infty} 0$.
	Nach \autoref{prop:41} existiert für jedes $\sigma_n$ vollständig positiv kontraktiv ein Lift $\overline{\sigma}_n \colon M_{r_n} \to B$ für $n \in \mathbb{N}$.
	Damit besitzt auch $\sigma_n \rho_n$ einen vollständig positiven kontraktiven Lift und mit \autoref{prop:43} folgt, dass auch $\varphi$ einen vollständig positiven Lift $\overline{\varphi}$ besitzt.
\end{beweis}

\begin{definition}[{name=[{vollständig positive Approximationseigenschaft}]}]
	Eine $C^*$-Algebra $A$ hat die \Index{vollständig positive Approximationseigenschaft} (CPAP), falls $\id_A$ nuklear ist.
\end{definition}

\begin{satz}[{name={Choi-Effros; Kirchberg}}]
	Eine $C^*$-Algebra $A$ ist nuklear genau dann, wenn sie die vollständig positive Approximationseigenschaft besitzt.
\end{satz}
\begin{beweis}
	Angenommen $A$ hat die CPAP.
	Sei $B$ eine weitere $C^*$-Algebra und 
	\[
		\big(\begin{tikzcd}
			A \rar["\psi_\lambda"] & F_\lambda \rar["\varphi_\lambda"] & B
		\end{tikzcd}\big)_{\lambda \in \Lambda}
	\]
	ein System von vollständig positiven kontraktiven Approximationen mit $F_\lambda$ endlichdimensional.
	Die Abbildungen $\psi_\lambda \otimes {\id_B} \colon A \odot B \to F_\lambda \odot B$ und $\varphi_\lambda \otimes {\id_B} \colon F_\lambda \odot B \to A \odot B$ haben vollständig positive kontraktive Fortsetzungen $\psi_\lambda \tensormin {\id_B} \colon A \tensormin B \to F_\lambda \tensormin B$ und $\varphi_\lambda \tensormax  {\id_B} \colon F_\lambda \tensormax B \to A \tensormax B$ (Übung).
	
	Wir erhalten eine vollständig positive kontraktive Abbildung
	\[
		\rho_\lambda \coloneqq \enbrace*{\varphi_\lambda \tensormax {\id_B}} \enbrace*{\psi_\lambda \tensormin {\id_B}} \colon A \tensormin B \longrightarrow F_\lambda \tensormin B = F_\lambda \tensormax B \longrightarrow A \tensormax B
	\]
	Für $x = \sum_{i=1}^{n} a_i \otimes b_i \in  A \odot B$ gilt
	\[
		\norm*{\rho_\lambda(x) -x}_{\max} = \norm*{\sum_{i=1}^{n} \enbrace[\big]{\varphi_\lambda \psi_\lambda(a_i) - a_i} \otimes b_i}_{\max} \le \sum_{i=1}^{n} \norm*{\varphi_\lambda \psi_\lambda(a_i) - a_i}_A \cdot \norm*{b_i}_B \grenzw{\lambda} 0 
	\]
	$\rho_\lambda$ ist kontraktiv und damit ist $\norm*{\rho_\lambda(x)}_{\max} \le \norm*{x}_{\min}$. 
	Weiter gilt
	\[
		\norm*{x}_{\max} \le \norm*{x - \rho_\lambda(x)}_{\max} + \norm*{\rho_\lambda(x)}_{\max} \le \norm*{x}_{\min} + \varepsilon
	\]
	also ist $\norm*{x}_{\max} = \norm*{x}_{\min}$.
	
	Für die andere Implikation geben wir nur die Idee sowie die nötigen Schritte an: 
	\begin{enumerate}[1.]
		\item $(A \odot B)^*_+ \coloneqq \set*{f \colon A \odot B \to \mathbb{C} \text{ linear } \given f(x^*x)\ge 0, x \in A \odot B}$ (vergleiche \autoref{def:116}).
		Weiter definiere:
		\[
			(A \odot B)^*_{++} \coloneqq \set*{f \colon A \odot B \to \mathbb{C} \text{ linear }  \given f =  \rho \otimes \sigma, \rho \colon A \to \mathbb{C}, \sigma\colon B \to \mathbb{C}}
		\]
		Nach \autoref{satz:120} gilt $(A \odot B)^*_{++} \subset (A \odot B)^*_{+}$.
		Zusammen mit \autoref{satz:119} folgt weiter
		\begin{align}
			\norm*{x}_{\max}^2 &= \sup \set*{\frac{f(y^*x^*xy)}{f(y^*y)} \given f \in (A \odot B)^*_+, y \in A \odot B, f(y^*y)\neq 0} \\
			\norm*{x}_{\min}^2 &= \sup \set*{\frac{f(y^*x^*xy)}{f(y^*y)} \given f \in (A \odot B)^*_{++}, y \in A \odot B, f(y^*y)\neq 0} \\
		\end{align}
		Zeige nun, dass aus der Nuklearität von $A$ folgt, dass $\conv (A \odot B)_{++}^* \subset (A \odot B)_+^*$ $\w^*$-dicht ist für alle $B$.
		\item Ist $\sigma \colon B \to \mathbb{C}$ positiv, so ist auch $\sigma^\ssbrace{n} \colon M_n(B)\to M_n$ positiv und man kann folgern, dass $B^*$ ein Matrix-geordneter Raum ist und man kann von vollständig positiven Abbildungen $A \to B^*$ sprechen.
		Zeige 
		\mapdef{(A \odot B)^*_+}{\CP(A,B^*)}{f}{T_f, \quad T_f(a)(b) = f(a \otimes b)}{\approx}
		Man stellt nun fest, dass $\conv(A \odot B)^*_{++} \approx \set*{T \in \CP(A,B^*) \given T \text{ hat endl. Rang}} =: \CP_f(A,B^*)$.
		Dies überlegt man sich, indem man aus $\rho \otimes \sigma$ normiert einen Vektorzustand bezüglich $\xi \coloneqq \xi_{\pi_\rho} \otimes \xi_{\pi_\sigma} \in \mathcal{H}_{\pi_\rho} \otimes \mathcal{H}_{\pi_\sigma}$ konstruiert.
		
		Aus $A$ nuklear folgt also $\CP_f(A,B^*) \subset \CP(A,B^*)$ dicht ist bezüglich der punktweisen $w^*$-Topologie.
		\item Sei $\enbrace*{\pi \colon A \to \mathcal{B}(\mathcal{H}),f}$ eine zyklische Darstellung.
		Betrachte nun $\phi \colon \pi(A)'' \to \enbrace*{\pi(A)'}^*$ gegeben durch $\phi(x)(y) \coloneqq \skal*{\xi}{xy \xi}$.
		$\phi$ ist vollständig positiv und aus 2. folgt, dass\marginnote{$B = \pi(A)'$}
		\[
			\begin{tikzcd}
				A \rar["\pi"] & \pi(A) \rar[hook] & \pi(A)'' \rar["\phi"] & \enbrace*{\pi(A)'}^*
			\end{tikzcd}
		\]
		nuklear ist.
		Zeige nun, dass dann auch die Komposition der ersten beiden Abbildungen nuklear ist.
		Damit kann man nun folgern, dass für die universelle Darstellung $\pi_u$ auch
		\[
			\begin{tikzcd}
				A \rar["\cong"] & \pi_u(A) \rar[hook] & \pi_u(A)'' \cong A^{**}
			\end{tikzcd}
		\]
		nuklear ist.
		Man zeigt dann, dass auch der Isomorphismus alleine nuklear ist (Konvexität und Kaplanskys Dichtesatz).\qedhere
	\end{enumerate}
\end{beweis}
% section 4 (end)
\newpage

\section{Amenabilität} % (fold)
\label{sec:5}

\begin{definition}[{name=[amenabel]{von Neumann}}]
	Eine diskrete Gruppe $G$ heißt \Index{amenabel} (oder mittelbar), falls ein Zustand $\mu \in S \enbrace*{\ell^\infty(G)}$ existiert mit $\mu(f) = \mu(s \cdot f)$ wo $s \cdot f(.) = f(s^{-1} .)$ für $s \in G$, $f \in \ell^\infty(G)$.
	Ein solches $\mu$ heißt \Index{linksinvariantes Mittel}.
\end{definition}

\begin{definition}[{name=[Følner-Bedingung]}]
	Eine diskrete Gruppe $G$ erfüllt die \Index{Følner-Bedingung}, falls für jede endliche Teilmenge $E \finSub G$ und $\varepsilon>0$ eine Teilmenge $F \finSub G$ existiert mit 
	\[
		\max_{s \in E} \frac{\abs*{F \DeltaOp sF} }{\abs*{F}} < \varepsilon  
	\]
	wo $sF = \set*{st \given t \in F} \subset G$ und $F \DeltaOp sF = (F \cup sF) \setminus F \cap sF$.
	Eine solche Menge $F$ heißt \Index{Følnermenge} zu $E,\varepsilon$.
\end{definition}

\begin{satz}[label=satz:53,{name=[{Zusammenhang von amenabel und nuklear}]}]
	Für eine diskrete Gruppe $G$ sind äquivalent:
	\begin{enumerate}[(i)]
		\item $G$ ist amenabel
		\item $G$ erfüllt die Følner-Bedingung
		\item $C^*_r(G)$ ist nuklear
	\end{enumerate}
\end{satz}

\todo[inline]{RevChap 5}

\begin{erinnerung}
	Es ist $C^*_r(G) = C^*(\lambda(G)) \subset \mathcal{B}(\ell^2(G))$, wo $\lambda \colon G \to \mathcal{U}(\mathcal{B}(\ell^2(G)))$ gegeben ist durch $\lambda_g(\eta_h) \coloneqq \eta_{gh}$ (linksreguläre Darstellung; $\eta_g(t)= \delta_{gt}$ Basisvektoren).
	Wir haben $\ell^\infty(G) \hookrightarrow \mathcal{B} \enbrace*{\ell^2(G)}$ mittels $f \mapsto m_f$.
	Weiter ist $m_{s \cdot f} = \lambda_s m_f \lambda_{s^{-1}}$ (warum?).
\end{erinnerung}

\begin{beweis}[name={von \autoref{satz:53}}]
	Zunächst die Idee für (i) $\Rightarrow $ (ii):
	Sei $E \finSub G$ endlich und $\varepsilon>0$.
	Nach Voraussetzung existiert ein linksinvariantes Mittel $\mu \in S(\ell^\infty(G))$.
	Es gilt $\ell^\infty(G) = \ell^1(G)^*$ und $\ell^1(G) \subset \ell^1(G)^{**}=\ell^\infty(G)^*$ $\w^*$-dicht.
	Also existiert ein Netz $(\mu_\gamma)_{\Gamma} \subset \ell^1(G)$ mit $\mu_\gamma \grenzwIn{\w^*} \mu$.
	Man kann sogar $(\mu_\gamma)_\Gamma \subset \ell^1(G)^1_+ =: \mathrm{Prob}(G)$ annehmen.\marginnote{Wahrscheinlichkeitsmaße auf $G$}
	Für jedes $s \in E$ gilt 
	\[
		s \cdot \mu_\gamma - \mu_g \grenzw{\w^*} 0 \implies s \cdot \mu_ga - \mu_g \grenzw{\w} 0 \marginnote{warum?}
	\]
	Damit ist 
	\begin{align}
		0 \in \bigcap_{s \in E} \overline{\set[\big]{s \cdot \mu' -\mu' \given \mu' \in \mathrm{Prob}(G)}}^{\w} &= \Underbracket{\overline{\bigcap_{s \in E} \Underbracket{\set[\big]{s \cdot \mu' -\mu' \given \mu' \in \mathrm{Prob}(G)}}{\text{konvex}}}^{\w}}{\text{konvex}} \\
		&\stackrel[\mathclap{\text{Banach}}]{\mathclap{\text{Hahn}}}{= }\enspace \,\overline{\bigcap_{s \in E} \set[\big]{s \cdot \mu' -\mu' \given \mu' \in \mathrm{Prob}(G)}}^{\norm*{\cdot}_1}
	\end{align}
	Damit existiert $\overline{\mu} \in  \mathrm{Prob}(G)$ mit $\sum_{s \in E} \norm*{s \cdot \overline{\mu}- \overline{\mu}}_1 < \varepsilon$.
	Setze $F_r \coloneqq \set*{g \in G \given \overline{\mu}(g) >r}$ für $r >0$.
	Für genügend kleines $r$ ist $F_r$ Følnermenge zu $E,\varepsilon$.
	
	Kommen wir nun zu (ii) $\Rightarrow $ (iii):
	Sei $E \finSub G$ endlich, $\varepsilon>0$ und $F \finSub G$ die Følnermenge zu $E,\varepsilon$.
	Sei $p_F \in \mathcal{B} \enbrace*{\ell^2(G)}$ die Projektion auf $\Span \set*{\eta_g \given g \in F} \subset \ell^2(G)$.
	Es ist nun
	\[
		p_F \mathcal{B} \enbrace*{\ell^2(G)} p_F \cong M_F \cong C^* \enbrace*{e_{g,h} \mid g,h \in F, e_{g,h}^* = e_{h,g}, e_{g,h} e_{k,l} = \delta_{h,k} \cdot e_{g,l}}
	\]
	Für $s \in G$ gilt $e_{g,g} \lambda_s e_{h,h} = \delta_{g,sh} e_{g,h}$.
	Wegen $p_F = \sum_{g \in F}e_{g,g}$ gilt
	\[
		p_F \lambda_s p_F ) = \sum_{g,h \in F} e_{g,g} \lambda_s e_{h,h} = \sum_{g,h \in F} \delta_{g,sh} \cdot e_{g,h} = \sum_{g \in F \cap sF} e_{g,s^{-1}g}
	\]
	Definiere $\psi_F \colon C^*_r(G) \to M_F$ als die Kompression $x \mapsto p_F x p_F$ und $\varphi_F \colon M_f \to C^*_r(G)$ durch $e_{g,h} \mapsto \sfrac{1}{\abs{F}}\cdot \lambda_g \lambda_{h^{-1}}$.
	Dann sind $\psi_F$ und $\varphi_F$ unital (klar) und vollständig positiv: Für $\psi_F$ ist dies als Kompression nach \autoref{bsp:35} klar.
	Für $\varphi_F$ betrachte:
	\[
		\varphi_F^\ssbrace{F} \enbrace*{(e_8{g,h})_{g,h \in F}} = \frac{1}{\abs{F}} \begin{pmatrix}
			\vdots & \hphantom{\lambda_g^*}& \\
			\lambda_g & & 0 \\
			\vdots & & 
		\end{pmatrix}
		\begin{pmatrix}
			\cdots & \lambda_g^* & \cdots \\
			& & \vphantom{\vdots}\\
			& 0 & \vphantom{\vdots}
		\end{pmatrix} \ge 0 \StackText{\ref{satz:323}}{\implies} \varphi_F \text{ v.p.} 
	\]
	Es gilt für $s \in E$
	\[
		1 \ge \frac{\abs*{F \cap sF}}{\abs*{F}} = \frac{\abs*{F \cup sF}}{\abs*{F} } - \frac{\abs*{F \DeltaOp sF}}{\abs*{F}} \ge 1 - \varepsilon
	\]
	Für $s \in E$ gilt nun 
	\[
		\varphi_F \psi_F (\lambda_s) = \varphi_F \enbrace*{\sum_{g \in F \cap sF} e_{g,s^{-1}g}} = \sum_{g \in F\cap sF} \frac{1}{\abs*{F}} \lambda_g \lambda_{(s^{-1}g)^{-1}} = \sum_{g \in F \cap sF} \frac{1}{\abs*{F}} \lambda_s 
		= \frac{\abs*{F \cap sF}}{\abs*{F} }  \cdot \lambda_s \approx_\varepsilon \lambda_s
	\]
	Da $\Span \set*{\lambda_s}$ dicht in $C^*_r(G)$ ist, folgt die Behauptung.
	
	Zu guter Letzt (iii) $\Rightarrow$ (i):
	Betrachte
	\[
		\Big(  
		\begin{tikzcd}
			C_r^*(G) \rar["\psi_\gamma"] & F_\gamma \rar["\varphi_\gamma"] & C_r^*(G)
		\end{tikzcd}
		\Big)_{\Gamma}
	\]
	mit $\varphi_r$, $\psi_r$ vollständig positiv kontraktiv und $\varphi_\gamma \psi_\gamma \to \id_{C^*_r(G)}$ punktweise.
	Nach Averson erhalten wir Abbildungen $C_r^*(G) \subset \mathcal{B} \enbrace*{\ell^2(G)} \to F_\gamma$, die wir mit $\overline{\psi}_\gamma$ bezeichnen.
	Damit folgt $\varphi_\gamma \overline{\psi}_\gamma \in \mathcal{B} \enbrace[\big]{ \mathcal{B}(\ell^2(G)), C_r^*(G)}^1 \subset \mathcal{B} \enbrace*{\mathcal{B}(\ell^2(G)), W^*(G)}^1$.
	Letzteres können wir als Dualraum auffassen (vgl. \autoref{satz:327}).
	Damit besitzt $(\varphi_\gamma \overline{\psi}_\gamma)_\Gamma$ besitzt einen $\w^*$-Häufungspunkt und wir erhalten $\Phi \colon \mathcal{B} \enbrace*{\ell^2(G)} \to W^*(G)$.
	$W^*(G)$ besitzt einen kanonischen Spurzustand $\tau$ mit $\tau(\sum \alpha_g \cdot \lambda_g) = \alpha_e$ und es gilt $\Phi|_{C^*_r(G)} = \id_{C_r^*(G)}$.
	Also ist $C^*_r(G)$ im multiplikativen Bereich von $\Phi$.
	Damit erhalten wir für $f \in \ell^\infty(G)$ und $s \in G$ 
	\[
		\tau \circ \Phi(m_{s \cdot f}) = \tau \circ \Phi(\lambda_s m_f \lambda_{s^{-1}}) = \tau \enbrace*{\lambda_s \Phi(m_f) \lambda_{s^{-1}}} =\tau \circ \Phi(m_f)
	\]
	Also ist die Zuordnung $f \mapsto \tau \circ \Phi(m_f)$ ein linksinvariantes Mittel auf $\ell^\infty(G)$. 
\end{beweis}
\todo[inline]{Hier fehlt noch eine komplette Vorlesung!}
\newpage

% section 5 (end)

\section{Die $K_0$-Gruppe einer unitalen $C^*$-Algebra} % (fold)
\label{sec:6}

\begin{definition}[{name=[{Projektionen in Matrizen über einer $C^*$-Algebra}]}]
	Für eine $C^*$-Algebra $A$ und $n \in \mathbb{N}$ setzen wir
	\[
		\kernedP_n(A) \coloneqq \mathcal{P}(M_n(A)) = \set[\big]{\text{Projektionen in }M_n(A) }
	\]
	und $\Pinfty(A) = \bigcup_{n=1}^\infty \kernedP_n(A)$.
	Für $p,q \in \Pinfty(A)$ mit $p \in \kernedP_n(A)$ und $q \in \kernedP_m(A)$ schreiben wir $p \sim_0 q$, falls $v = (v_{ij})_{ij} \in M_{m,n}(A)$ existiert mit $p=v^*v$ und $q = vv^*$.\marginnote{dabei ist $v^* = (v_{ji}^*)_{ij} \in M_{n,m}$}
	Wir definieren eine Verknüpfung $\oplus  \colon \Pinfty(A) \times \Pinfty(A) \to \Pinfty(A)$ durch
	\[
		p \oplus q \coloneqq \diag(p,q) = \begin{pmatrix}
			p & 0 \\
			0 & q
		\end{pmatrix} \in M_{n+m}(A)
	\]
\end{definition}

\begin{proposition}[{name=[{grundlegende Eigenschaften der Relation $\sim_0$}]},label=prop:62]
	\leavevmode
	\begin{enumerate}[a)]
		\item $\sim_0$ ist eine Äquivalenzrelation.
		\item Für $p,q,v,p',q' \in \Pinfty(A)$ gilt
		\begin{enumerate}[(i)]
			\item $p \sim p \oplus 0_n$ für $n \in \mathbb{N}$
			\item Falls $p \sim_0 p'$ und $q \sim_0 q'$ ist, so gilt $p \oplus q \sim_0 p' \oplus q'$
			\item $p \oplus q \sim_0 q \oplus p$
			\item Falls $p,q \in \kernedP_n(A)$ \emph{orthogonal} sind, also $pq=0$ gilt, so ist $p +q \in \kernedP_n(A)$ und es gilt $p + q \sim_0 p \oplus q$
			\item $p \oplus (q \oplus v) \sim_0 (p \oplus q) \oplus v$
		\end{enumerate}
	\end{enumerate}
\end{proposition}
\begin{beweis}
	\todo{kommt noch}
\end{beweis}

\begin{definitionP}
	Für eine $C^*$-Algebra $A$ setzen wir $\mathcal{D}(A) \coloneqq \sfrac{\Pinfty(A)}{\sim_0}$ und schreiben $\benbrace*{p}_\mathcal{D}$ für die Äquivalenzklasse von $p \in \Pinfty(A)$.
	Wir definieren eine Addition \enquote{$+$} auf $\mathcal{D}(A)$ durch
	\[
		\benbrace*{p}_\mathcal{D} + \benbrace*{q}_\mathcal{D} \coloneqq \benbrace*{p \oplus q}_\mathcal{D}
	\]
	Die Operation ist wohldefiniert und $(\mathcal{D}(A),+)$ ist eine abelsche Halbgruppe.
\end{definitionP}
\begin{beweis}
	Folgt aus \autoref{prop:62}.
\end{beweis}

\begin{erinnerungA}[label=err:64]
	Sei $(S,+)$ eine abelsche Halbgruppe.
	Wir definieren eine Äquivalenzrelation $\sim$ auf $S \times S$ durch
	\[
		(x_1 , y_1 ) \sim (x_2,y_2) :\iff \exists z \in S : x_1 +y_2 + z  = x_2 + y_1 + z
	\]
	Wir schreiben $\skal*{x}{y}$ für die Klasse von $(x,y)$ in $\Groth(S) = \sfrac{S \times S}{\sim}$.
	Definiere eine Addition  auf $\Groth(S)$ durch
	\[
		\skal*{x_1}{y_1} + \skal*{x_2}{y_2} :\iff \skal*{x_1 +x_2}{y_1 +y_2}
	\]
	$\Groth(S)$ ist dann eine abelsche Gruppe, die \Index{Grothendieck-Gruppe}\footnote{nach Alexander \textsc{Grothendieck}, * 28. März 1928; † 13. November 2014, deutschstämmiger französischer Mathematiker}.
	Zu $\overline{y} \in S$ beliebig betrachten wir die Abbildung 
	\mapdef{\gamma_S \colon S}{\Groth(S)}{x}{\skal*{x +\overline{y}}{\overline{y}}}{}
	Dann hängt $\gamma_S$ nicht von der Wahl von $\overline{y}$ ab (warum?) und wir können die Grothendiekgruppe auch schreiben als $\Groth(S) = \set[\big]{\gamma_S(x) - \gamma_S(y) \given x,y \in S}$.
	$\gamma_S$ ist injektiv genau dann, wenn $S$ die Kürzungseigenschaft besitzt, das heißt, falls $x+z = y+ z \implies x=y$.
	$\Groth(\cdot)$ ist ein Funktor bezüglich additiver Abbildungen:
	\[
		\begin{tikzcd}[sep =large]
			S \rar["\varphi"] \dar["\gamma_S"] & T \dar["\gamma_T"] \\
			\Groth(S) \rar["\Groth(\varphi)"] & \Groth(T)
		\end{tikzcd}
	\]
	Dabei ist $\Groth(\varphi)$ ein Gruppenhomomorphismus.
	Es gilt folgende universelle Eigenschaft: Für eine abelsche Gruppe $G$  und $\varphi$ additiv existiert der folgende Gruppenhomomorphismus
	\[
		\begin{tikzcd}
			S \dar["\gamma_S"] \rar["\varphi"] & G \\
			\Groth(S) \urar[dashed,"\exists!"']
		\end{tikzcd}
	\]
\end{erinnerungA}

\begin{definition}[{name=[{$K_0$-Gruppe einer unitalen $C^*$-Algebra}]}]
	Sei $A$ eine unitale $C^*$-Algebra.
	Wir definieren
	\[
		K_0(A) \coloneqq \Groth \enbrace[\big]{\mathcal{D}(A)}
	\]
	Für $p \in \Pinfty(A)$ schreiben wir $\benbrace*{p}_0 \coloneqq \gamma_{\mathcal{D}(A)} \enbrace*{\benbrace*{p}_{\mathcal{D}}}$.
\end{definition}

\todo[inline]{RevChap 6}

\begin{proposition}[label=prop:66]
	Sei $A$ eine unitale $C^*$-Algebra.
	Dann gilt
	\begin{enumerate}[(i)]
		\item $K_0(A) = \set*{[p]_0 - [q]_0 \given p,q \in \Pinfty(A)}$
		\item $\benbrace*{p \oplus q}_0 = [p]_0 + [q]_0$ für $p,q \in \Pinfty(A)$
		\item $\benbrace*{0_A}_0 = 0_{K_0(A)}$
		\item Falls $p \perp q \in \Pinfty(A)$, so gilt $\benbrace*{p + q}_0 = \benbrace*{p}_0 + \benbrace*{q}_0$
	\end{enumerate}
	Außerdem erfüllt $K_0(A)$ die folgende universelle Eigenschaft: Sei $G$ eine abelsche Grupe und $\nu \colon \Pinfty(A) \to G$ eine Abbildung mit
	\begin{enumerate}[a)]
		\item $\nu(p \oplus q) = \nu(p) + \nu(q)$
		\item $\nu(0_A) = 0_G$
		\item $p \sim_0 q \implies \nu(p)=\nu(q)$
	\end{enumerate}
	Dann existiert genau ein Gruppenhomomorphismus $\alpha \colon K_0(A) \to G$, sodass folgendes Diagramm kommutiert
	\[
		\begin{tikzcd}
			\Pinfty(A) \dar["{[\cdot]_0}"] \drar["\nu"] &\\
			K_0(A) \rar[dashed,"\exists! \alpha"] & G
		\end{tikzcd}
	\]
\end{proposition}
\begin{beweis}
	\emph{Folgt aus den bisherigen Erkenntnissen dieses Kapitels.}
\end{beweis}

\begin{definition}[label=def67,{name=[{stabil äquivalent, homotop, unitär äquivalent}]}]
	Sei $A$ eine $C^*$-Algebra.
	\begin{enumerate}[(i)]
		\item $p,q \in \Pinfty(A)$ heißen \Index{stabil äquivalent}, $p \sim_s q$, falls $r \in \Pinfty(A)$ existiert mit $p \oplus r \sim_0 q \oplus r$.
		\item $p,q \in \kernedP_n(A)$ heißen \Index{homotop}, $p \sim_h q$, falls $\Phi \colon [0,1] \to \mathcal{P}_n(A)$ stetig existiert mit $\Phi(0)=p$ und $\Phi(1)=q$. 
		\item $p,q \in \kernedP_n(A)$ heißen \Index{unitär äquivalent}, $p \sim_u q$, falls $u \in \mathcal{U}(M_n(A)^\sim)$ existiert mit $u^* p u = q$.
	\end{enumerate}
	$\sim_s$, $\sim_n$ und $\sim_u$ sind Äquivalenzrelationen (warum?).
\end{definition}

\begin{proposition}[label=prop:68]
	Sei $A$ eine unitale $C^*$-Algebra.
	\begin{enumerate}[(i)]
		\item Für $p,q \in \Pinfty(A)$ gilt: $p \sim_s q \iff p \oplus \ind_m \sim_0 q \oplus \ind_m \iff \benbrace*{p}_0 = \benbrace*{q}_0$ für ein $m \in \mathbb{N}$.
		\item Für $p,q \in \kernedP_n(A)$ gilt: $p \sim_h q \implies \benbrace*{p}_0 = \benbrace*{q}_0$.
		\item Für $p,q \in \kernedP_n(A)$ gilt: $p \sim_0 q \implies \begin{psmallmatrix} p &0 \\ 0 & 0 \end{psmallmatrix} \sim_h \begin{psmallmatrix} q &0 \\ 0 & 0 \end{psmallmatrix}$ in $M_{2n}(A)$.
		\item Für $p,q \in \kernedP_n(A)$ gilt: $p \sim_u q \implies \begin{psmallmatrix} p &0 \\ 0 & 0 \end{psmallmatrix} \sim_h \begin{psmallmatrix} q &0 \\ 0 & 0 \end{psmallmatrix}$ in $M_{2n}(A)$
	\end{enumerate}
\end{proposition}
\begin{beweis}
	\begin{enumerate}[(i)]
		\item Nach Definition folgt aus $p \sim_s q$ die Existenz von $r \in \kernedP_m(A)$ mit $p \oplus r \sim_0 q \oplus r$.
		Damit erhalten nach \autoref{err:64}
		\[
			p \oplus \ind_m \StackText{\ref{prop:66}}{\sim_0} p \oplus r \oplus (\ind_m -r) \sim_0 q \oplus r \oplus (\ind_m -r) \StackText{\ref{prop:66}}{\sim_0} q \oplus \ind_m
		\]
		Wieder nach \autoref{err:64} folgt $\benbrace*{p}_0 = \benbrace*{q}_0$ und auch $\benbrace*{p}_D + \benbrace*{e}_D = \benbrace*{q}_D + \benbrace*{e}_D$ für ein $e \in \kernedP_k(A)$.
		Dann gilt auch $p \oplus e \sim_0 q \oplus e$ und wir haben $p \sim_s q$.
		\item Sind $p$ und $q$ homotop, so finden wir $p=p_0, p_1, \ldots , p_l=q \in \kernedP_n(A)$ mit $\norm*{p_i - p_{i+1}} < \sfrac{1}{2}$.
		Wir dürfen also ohne Einschränkungen annehmen, dass $\norm*{p-q} < \sfrac{1}{2}$ ist.
		Setze $z \coloneqq pq + (\ind_n -p)(\ind_n -q)$, dann gilt $pz = pq = zq$ und
		\begin{align}
			\norm*{z- \ind_n} = \norm*{p(q-p) + (\ind_n -p) \enbrace[\big]{(\ind_n -q) -(\ind_n -p)}} &= \norm*{p(q-p) + (\ind_n-p)(p-q)} \\
			&\le \norm*{p(q-p)} +\norm*{(\ind_n-p)(p-q)} \\
			&< 2 \norm*{p-q} < 1
		\end{align}
		Mit der von-Neumann-Reihe folgt, dass $z$ invertierbar ist und somit $z=u \cdot \abs*{z}$ für ein $u \in \mathcal{U}(M_n(A))$.
		Also ist 
		\[
			q= z^{-1}zq = z^{-1}pz = \abs*{z}^{-1} u* p u \abs*{z}  \stackrel{q=q^*}{=\joinrel=\joinrel=} \abs*{z} u^* p u \abs*{z}^{-1}  
		\]
		Es folgt $\abs*{z}^2 u^* p u \abs*{z}^{-2} = u^+ p u$ und somit $\abs*{z}^2 u^* p u = u^* p u \abs*{z}^2$.
		Daraus erhalten wir nun weiter $\benbrace[\big]{\abs*{z}^2, u^*p u }=0$ und somit auch $\benbrace*{\abs*{z},u^*pu} =0$ (warum?).
		Also git $q = u^* p u$, also $q \sim_u p \implies q \sim_0 p$.
		\item Sei $v \in M_n(A)$ mit $p=v^*v$, $q=vv^*$.
		Wir setzen 
		\[
			u \coloneqq \begin{pmatrix}
				v & \ind - q \\
				\ind -p & v
			\end{pmatrix} \quad, \quad 
			w \coloneqq \begin{pmatrix}
				q & \ind -q \\
				\ind -q & q
			\end{pmatrix} \in \mathcal{U} \enbrace*{M_{2n}(A)}
		\]
		Die Eigenschaft unitär zu sein, rechnet man leicht nach.
		Es gilt
		\begin{align}
			w u \begin{pmatrix} p & 0 \\ 0 & 0 \end{pmatrix} u^* w^* = w \begin{pmatrix} q & 0 \\ 0 & 0 \end{pmatrix}w^* = \begin{pmatrix} q & 0 \\ 0 & 0 \end{pmatrix}
		\end{align}
		Also gilt $\begin{psmallmatrix} p & 0 \\ 0 & 0 \end{psmallmatrix} \sim_u \begin{psmallmatrix} q & 0 \\ 0 & 0 \end{psmallmatrix}$.
		\item Aus $u^*pu=q$ folgt 
		\[
			\Underbracket{\begin{pmatrix}
				u^* & 0 \\ 0 & \ind
			\end{pmatrix} \begin{pmatrix}
				p & 0 \\ 0 & 0
			\end{pmatrix} \begin{pmatrix}
				u & 0 \\ 0 & \ind
			\end{pmatrix}}{\sim_h \begin{psmallmatrix}
				\ind & 0 \\ 0 & u^*
			\end{psmallmatrix} \begin{psmallmatrix}
				p & 0 \\ 0 & 0
			\end{psmallmatrix} \begin{psmallmatrix}
				\ind & 0 \\ 0 & u
			\end{psmallmatrix}} = \begin{pmatrix}
				q & 0 \\0 & 0
			\end{pmatrix} \in M_{2n}(A) \qedhere
		\]
	\end{enumerate}
\end{beweis}

\begin{bemerkung}[label=bem:69]
	\begin{enumerate}[a)]
		\item Der Beweis von (ii) funktioniert auch mit $A^\simm$, falls $A$ nicht unital ist.
		\item In (ii) kann man auch zeigen, dass $\sigma(p qp) \subset [0,\sfrac{1}{4}] \cup [\sfrac{3}{4},1]$, falls $\norm*{p-q} < \sfrac{1}{4}$ (?).
		Es gilt: 
		\[
			\forall \varepsilon>0 : \exists \delta>0 : p,q \in \mathcal{P}(A), \norm*{p-q}<\delta \implies \exists u \in \mathcal{U}(A^\simm) : u^* pu = q , \norm*{u-\ind} < \varepsilon
		\]
		Die Abhängigkeit $\delta(\varepsilon)$ lässt sich explizit angeben.
	\end{enumerate}
\end{bemerkung}

\begin{korollar}[label=kor:610]
	In der universellen Eigenschaft von \autoref{prop:66} kann man c) ersetzen durch
	\begin{description}
		\item[c')] $p,q \in \kernedP_n(A)$, $p \sim_h q \implies \nu(p) = \nu(q)$
	\end{description}
\end{korollar}
\begin{beweis}
	Falls $p,q \in \kernedP_\infty(A)$ mit $p \sim_0 q$, so existieren $k_1,k_2,k_3 \in \mathbb{N}$ mit 
	\[
		p' \coloneqq p \oplus 0_{k_1} \sim_0 p \sim_0 q \sim q \oplus 0_{k_2} =: q'
	\]
	und $p',q' \in \kernedP_{k_3}(A)$.
	Dann ist $p' \oplus 0_{3 \cdot k_3} \sim_h q' \oplus 0_{3 \cdot k_3}$.
	Mit der Bedingung c') folgt nun
	\[
		\nu(p) = \nu \enbrace*{p' \oplus 0_{3 \cdot k_3}} \StackText{c')}{=} \nu \enbrace*{q' \oplus 0_{3 \cdot k_3}} = \nu(q) \qedhere
	\]
\end{beweis}

Seien $A,B$ unitale $C^*$-Algebren und $\varphi \colon A \to B$ ein $^*$-Homomorphismus.
Dann induziert $\varphi$ eine Abbildung $\varphi^\ssbrace{\infty} \colon \Pinfty(A) \to \Pinfty(B)$ mit $\varphi^\ssbrace{\infty}|_{\kernedP_n(A) } = \varphi^\ssbrace{n}|_{\kernedP_n(A)}$.
Für $p,q \in \kernedP_n(A)$ gilt $p \sim_h q \implies \varphi^\ssbrace{\infty}(p) \sim_h \varphi^\ssbrace{\infty}(q)$.

Definiere $\nu \colon \Pinfty(A) \to K_0(B)$ durch $p \mapsto \benbrace*{\varphi^\ssbrace{\infty}(p)}_0$.
Dann erfüllt $\nu$ die Punkte a) und b) aus \autoref{prop:66}, sowie c') aus \autoref{kor:610}.
Folglich existiert genau ein Gruppenhomomorphismus $K_0(\varphi) \colon K_0(A) \to K_0(B)$, sodass
\[
	\begin{tikzcd}[column sep=3em]
		\Pinfty(A) \dar["\benbrace*{\cdot}_0"] \rar["\varphi^\ssbrace{\infty}"] & \Pinfty(B) \dar["\benbrace*{\cdot}_0"] \\
		K_0(A) \rar[dashed,"K_0(\varphi)"] & K_0(B)
	\end{tikzcd}
\]
kommutiert.
Man schreibt oft auch $\varphi_*$ oder $\varphi_0$ für $K_0(\varphi)$.

\begin{proposition}[label=prop:611,{name=[{Funktorialität von K0}]}]
	Seien $A,B,C$ unitale $C^*$-Algebren.
	Seien $\varphi \colon A\to B$, $\psi \colon B \to C$  $^*$-Homomorphismen.
	Dann gilt
	\begin{enumerate}[(i)]
		\item $K_0(\id_A) = \id_{K_0(A)}$
		\item $K_0(\psi \circ \varphi) = K_0(\psi) \circ K_0(\varphi)$
		\item $K_0(0_{A \to B}) = 0_{K_0(A) \to K_0(B)}$
	\end{enumerate}
	Insbesondere ist $K_0$ ein kovarianter Funktor $(\CSTARUN, {^*\Hom}) \to \enbrace*{\AB, {\Hom}}$.
	Lässt man $\set*{0}$ in $\CSTARUN$ zu, so gilt $K_0(\set*{0}) = \set*{0}$.
\end{proposition}
\begin{beweis}
	Wegen $K_0(\varphi) \enbrace*{\benbrace*{p}_0} = \benbrace*{\varphi^\ssbrace{\infty}(p)}_0$ für $p \in \Pinfty(A)$ gilt $K_0(\id_A) \enbrace*{\benbrace*{p}_0} = \benbrace*{p}_0 =\id_{K_0(A)}$ für alle $p \in \Pinfty(A)$.
	Ebenso folgt 
	\[
		K_0(\psi \circ \varphi) \enbrace*{\benbrace*{p}_0} = \benbrace*{(\psi \circ \varphi)(p)}_0 = K_0(\psi) \circ K_0(\varphi) \enbrace*{\benbrace*{p}_0}
	\]
	Wegen $K_0(A) = \set[\big]{\benbrace*{p}_0 - \benbrace*{q}_0 \given p,q \in \Pinfty(A)}$ gilt dann (i) und (ii).
	(iii) folgt mit $\benbrace*{0_B}_0 = 0_{K_0(B)}$.
	
	Es gilt außerdem $\kernedP_n( \set*{0}) = \set*{0_n}$, wobei $0_n \coloneqq 0_{M_n(\set*{0})}$.
	Aber $0_n \sim_0 0_m$ und es folgt $\mathcal{D}(\set*{0}) = \set{\benbrace{0_1}_{\mathcal{D}}} = \set*{0}$ als abelsche Halbgruppe.
	Mit $\Groth(\set*{0}) = \set*{0}$ folgt die Behauptung.
\end{beweis}

\begin{definition}[label=def:612,{name=[{homotope *-Homomorphismen}]}]
	Es seien $A,B$ zwei $C^*$-Algebren. Zwei $^*$-Homomorphismen $\varphi,\psi \colon A \to B$ heißen \Index{homotop}, $\varphi \sim_h \psi$, falls es $^*$-Homomorphismen $\varphi_t \colon A \to B$ für $t \in [0,1]$ gibt mit $\varphi_0=\varphi$, $\varphi_1 = \psi$ und sodass $t \mapsto \varphi_t(a)$ stetig ist für jedes $a \in A$.
	Äquivalent: Es existiert ein $^*$-Homomorphismus $\gamma \colon A \to C \enbrace*{[0,1],B}$ mit $\ev_0 \circ \gamma = \varphi$ und $\ev_1 \circ \gamma = \psi$.
	
	$A$ und $B$ heißen \Index{homotopieäquivalent}, fals $^*$-Homomorphismen $\rho \colon A \to B$ und $\sigma \colon B \to A$ existieren mit $\sigma \circ \rho \sim_h \id_A$ und $\rho \circ \sigma \sim_h \id_B$.
	$A$ und $B$ heißen dann auch \Index{homotopieäquivalent} via $\begin{tikzcd}[cramped,sep=small] A \rar["\rho"] & B \rar["\sigma"] & A \end{tikzcd}$.
\end{definition}

\begin{proposition}[label=prop:613]
	Seien $A,B$ unitale $C^*$-Algebren.
	\begin{enumerate}[(i)]
		\item Für $\varphi, \psi \colon A \to B$ $^*$-Homomorphismen gilt: $\varphi \sim_h \psi \implies K_0(\varphi) = K_0(\psi)$.
		\item Falls $A,B$ homotopieäquivalent via $\begin{tikzcd}[cramped,sep=small] A \rar["\rho"] & B \rar["\sigma"] & A \end{tikzcd}$ sind, so sind $K_0(\rho)$ und $K_0(\sigma)$ zueinander inverse Isomorphismen.
	\end{enumerate}
\end{proposition}
\begin{beweis}
	\begin{enumerate}[(i)]
		\item Sei $\varphi_t \colon A \to B$, $t \in [0,1]$ ein punktweise stetiger Pfad von $^*$-Homomorphismen mit $\varphi_0 = \varphi$ und $\varphi_1 = \psi$.
		Dann ist auch $\varphi_t^\ssbrace{n} \colon M_n(A) \to M_n(B)$, $t \in [0,1]$ punktweise stetig für $n \in \mathbb{N}$.
		Für $p \in \kernedP_n(A)$ ist dann $\varphi_t^\ssbrace{n}(p) \in \kernedP_n(B)$ stetig und es folgt
		\[
			\varphi^\ssbrace{n}(p) = \varphi^\ssbrace{n}(p) \sim_h \varphi^\ssbrace{n}(p) = \psi^\ssbrace{n}(p)
		\]
		Damit ist $K_0(\varphi) \enbrace*{\benbrace*{p}_0} = K_0(\psi) \enbrace*{\benbrace*{p}_0}$.
		Mit \autoref{prop:66} folgt $K_0(\varphi) = K_0(\psi)$.
		\item Folgt aus (i) mit \autoref{prop:611} (i) und (ii). \qedhere
	\end{enumerate}
\end{beweis}

\begin{proposition}[label=prop:614]
	Seien $A,B$ unitale $C^*$-Algebren und $\varphi,\psi \colon A \to B$ $^*$-Homomorphismen.
	Falls $\varphi \perp \psi$, das heißt $\varphi(a) \psi(a') = 0$ für $a,a' \in A$ (äquivalent: $\varphi(\ind_A) \psi(\ind_A)=0$), so ist $\varphi +\psi \colon A \to B$ ein $^*$-Homomorphismus und es gilt $K_0(\varphi + \psi) = K_0(\varphi) + K_0(\varphi)$.
\end{proposition}
\begin{beweis}
	Dass $\varphi +\psi$ unter den Voraussetzungen ein $^*$-Homomorphismus ist, ist klar, ebenso für $\varphi^\ssbrace{n} + \psi^\ssbrace{n}$.
	Es folgt für $p \in \kernedP_n(A)$
	\begin{align}
		K_0(\varphi + \psi) \enbrace*{\benbrace*{p}_0} = \benbrace*{(\varphi + \psi)^\ssbrace{n} (p)}_0 = \benbrace*{\varphi^\ssbrace{n}(p) + \psi^\ssbrace{n}(p)}_0 &\StackTextClap{\ref{prop:66}}{=} \benbrace*{\varphi^\ssbrace{n}(p)}_0 + \benbrace*{\psi^\ssbrace{n}(p)}_0 \\
		&= \benbrace*{K_0(\varphi) + K_0(\psi)} \enbrace*{\benbrace*{p}}_0 \qedhere
	\end{align}
\end{beweis}

\begin{proposition}[label=prop:615]
	Sei $A$ eine unitale $C^*$-Algebra.
	Dann induziert die zerfallende exakte Sequenz\marginnote{wo $\sigma(\ind_\mathbb{C}) = \ind_{A^+}$}
	\[
		\begin{tikzcd}
			0 \rar & A \rar["\iota"] & A^+ \rar["\pi"] & \mathbb{C} \rar \lar["\sigma"', bend right] & 0
		\end{tikzcd}
	\]
	eine zerfallende exakte Sequenz
	\[
		\begin{tikzcd}
			0 \rar & K_0(A) \rar["K_0(\iota)"] & K_0(A^+) \rar["K_0(\pi)"] & K_0(\mathbb{C}) \rar  \lar["K_0(\sigma)"', bend right] & 0
		\end{tikzcd}
	\]
\end{proposition}
\begin{beweis}
	Betrachte die $^*$-Homomorphismem $\mu \colon A^+ \to A$ gegeben durch $x \mapsto \ind_A x \ind_A$ und $\lambda \colon \mathbb{C} \to A^+$ mit $\ind_\mathbb{C} \mapsto \ind_{A^+} - \ind_A$.
	Dann gilt $\mu \circ \iota = \id_A$ und $\iota \circ \mu + \lambda \circ \pi = \id_{A^+}$, sowie $\pi \circ \iota = 0_{A\to \mathbb{C}}$ und $\pi \circ \sigma = \id_\mathbb{C}$.
	Mit der Funktorialität folgt
	\[
		0_{K_0(A) \to K_0(\mathbb{C})} = K_0(0_{A \to \mathbb{C}}) = K_0(\pi_n) \circ K_0(\iota) \implies \im K_0(\iota) \subset \ker K_0(\pi)
	\]
	und weiter $\id_{K_0(\mathbb{C})} = K_0(\id_\mathbb{C}) = K_0(\pi) K_0(\sigma) \implies K_0(\pi)$ surjektiv und die neue Sequenz zerfällt.
	Aus $\id_{K_0(A)} = K_0(\mu) K_0(\iota)$ folgt die Injektivität von $K_0(\iota)$. Weiter gilt
	\[
		\id_{K_0(A^+)} = K_0(\id_{A^+}) = K_0(\iota \circ \mu) + K_0(\lambda \circ \pi) = K_0(\iota) K_0(\mu) + K_0(\lambda) K_0(\pi)
	\]
	Woraus $\ker K_0(\pi) \subset \im K_0(\iota)$ folgt.
\end{beweis}

\begin{proposition}[label=prop:616]
	Sei $A$ eine unitale $C^*$-Algebra und $\tau$ ein positives Spurfunktional, das heißt $\tau \colon A \to \mathbb{C}$ ist linear, positiv mit $\tau(ab) =\tau(ba)$.
	Dann definiert 
	\mapdef{\tau_A \colon K_0(A)}{\mathbb{R}}{\benbrace*{p}_0}{\enbrace*{\Tr_{M_n} \otimes \tau}(p)}{}
	einen Gruppenhomomorphismus mit $\tau_*\enbrace*{\gamma_D \enbrace*{\mathcal{D}(A)}} \subset \mathbb{R}_+$ und $\tau_* \enbrace*{\benbrace*{\ind_A}_0} = \norm*{\tau}$.
\end{proposition}
\begin{beweis}
	${\Tr_{M_n}} \otimes \tau$ ist ein positives Spurfunktional mit Norm $n \cdot \norm*{\tau}$ auf $M_n \otimes A$.
	Nach \autoref{prop:68} (ii) gilt $p \sim_h q \implies p \sim_u q$ und damit
	\[
		\enbrace*{{\Tr_{M_n}} \otimes \pi}(p) = \enbrace*{{\Tr_{M_n}} \otimes \tau} (q)
	\]
	Betrachte nun $\nu \colon \Pinfty(A) \to \mathbb{R}$, $p \mapsto \enbrace*{{\Tr_{M_n}} \otimes \tau}(p)$ für $p \in M_n \otimes A$.
	Diese Abbildung erfüllt \autoref{kor:610} c') und auch \autoref{prop:66} a), b).
	Also kommutiert
	\[
		\begin{tikzcd}
			\Pinfty(A) \dar["\benbrace*{\cdot}"] \drar["\nu"] \\
			K_0(A) \rar[dashed,"\exists! \tau_A"] & \mathbb{R}
		\end{tikzcd}
	\]
	Der Rest ist klar.
\end{beweis}

\begin{beispiel}[label=bsp:617]
	Es gilt $K_0(M_n) \cong \mathbb{Z}$:
	\begin{itemize}
		\item Für $p,q \in \Pinfty(M_n)$ gilt $p \sim_0 q \iff \rk p = \rk q$.
		\item $\forall r \in \mathbb{N} : \exists k \in \mathbb{N}, p \in M_k(M_n) : \rk p = r$. 
	\end{itemize}
	Definiere $\mathbb{N} \to \mathcal{D} \enbrace*{M_n}$ durch $0 \mapsto \benbrace*{0_{M_n}}_{\mathcal{D}}$, $1 \mapsto \benbrace*{e_{11}}_{\mathcal{D}}$ und setze dies entsprechend fort.
	Dies ist ein Isomorphismus von Halbgruppen.
	Mit $\Groth(\mathbb{N}) = \mathbb{Z}$ folgt dann die Behauptung.
	
	Es gilt außerdem
	\[
		\begin{tikzcd}
			\mathbb{Z} \rar \ar[rr,bend left,"\id_\mathbb{Z}"] & K_0(M_n) \rar["\Tr_*"] & \mathbb{Z} \\[-2em]
			1 \rar[mapsto] & \benbrace*{e_{11}}_0 & 
		\end{tikzcd}
	\]
\end{beispiel}

\begin{beispiel}[label=618]
	Sei $\mathcal{H}$ ein unendlichdimensionaler Hilbertraum. 
	Dann gilt $K_0 \enbrace*{\mathcal{B}(\mathcal{H})}=0$.
\end{beispiel}
\begin{beweis}
	Für $\mathcal{H}$ separabel gilt $M_n(\mathcal{B}(\mathcal{H})) \cong \mathcal{B} \enbrace*{\mathcal{H}^{\oplus n}}$ und wir definieren $\gamma \colon \Pinfty(\mathcal{B}(\mathcal{H})) \to \mathbb{N} \cup \set*{\infty}$ durch $p \mapsto \dim p \enbrace*{\mathcal{H}^{\oplus n}}$.
	$\mathbb{N} \cup \set*{\infty}$ ist eine abelsche Halbgruppe.
	Für $p,q \in M_n(\mathcal{B}(\mathcal{H}))$ gilt: $\gamma(p) = \gamma(q) \iff p \sim_{\MvN} q$.\footnote{$p,q$ Projektionen in $C^*$-Algebra $B$, $p\sim_{\MvN} q : \Leftrightarrow \exists v \in B : v^*vp, vv^*=q$ (Murray-von-Neumann äquivalent)}
	Für $p \in M_n(\mathcal{B}(\mathcal{H}))$, $q \in M_m(\mathcal{B}(\mathcal{H}))$ mit $m \le n$ gilt: 
	\[
		p \sim_0 q \iff p \sim_{\MvN} q \oplus 0_{m-n}
	\]
	Weiter gilt $\gamma(p \oplus q) = \gamma(p) + \gamma(q)$ (Dimension ist additiv) und $\gamma(0)=0$.
	Damit ist $\gamma(p) = \gamma(q)$ genau dann, wenn $p\sim_0 q$ für $p,q \in \Pinfty(\mathcal{B}(\mathcal{H}))$.
	Folglich induziert $\gamma$ einen Halbgruppenisomorphismus $\overline{\gamma} \colon \mathcal{D} \enbrace*{\mathcal{B}(\mathcal{H})} \to \mathbb{N} \cup \set*{\infty}$, $\benbrace*{p}_\mathcal{D} \mapsto \gamma(p)$.
	Mit $\Groth(\mathbb{N} \cup \set*{\infty}) =0$ folgt nun die Behauptung für separabeles $\mathcal{H}$.
	
	Für ein beliebiges $\mathcal{H}$ benutzt man die Kardinalzahlen $\le \dim \mathcal{H}$.
\end{beweis}

\begin{beispiel}
	Sei $X$ ein kontrahierbarer kompakter Hausdorffraum, das heißt es gibt $x_0 \in X$ und $h \colon [0,1] \times X \to X$ stetig mit $h(1,x)=x$ und $h(0,x)=x_0$ für $x \in X$.
	Dann gilt $K_0 \enbrace*{C(X)} \cong \mathbb{Z}$:
	
	Weiter sind $C(X)$ und $\mathbb{C}$ homotopieäquivalent via \(
		\begin{tikzcd}[cramped,sep=large]
			C(X) \rar["\ev_{x_0}","\rho"'] & \mathbb{C} \rar["\ind_\mathbb{C} \mapsto \ind_{C(X)}","\sigma"'] & C(X)
		\end{tikzcd}
	\), denn $\rho \circ \sigma = \id_\mathbb{C}$ und $\sigma \circ \rho \sim_h \id_{C(X)}$ mittels $\varphi_t \colon C(X) \to C(X)$, $f(\cdot) \mapsto f(h(t,\cdot))$ für $t \in [0,1]$.
	Mit \autoref{prop:613} folgt $K_0(C(X)) \cong K_0(\mathbb{C}) \cong \mathbb{Z}$.
\end{beispiel}
% section 6 (end)
\newpage

\section{Der Funktor $K_0$} % (fold)
\label{sec:7}

\begin{definition}
	Sei $A$ eine nichtunitale $C^*$-Algebra.
	Betrachte die zerfallende exakte Sequenz
	\[
		\begin{tikzcd}
			0 \rar & A \rar["\iota"] & A^+ \rar["\pi"] & \mathbb{C} \rar & 0
		\end{tikzcd}
	\]
	und die induzierte Abbildung $K_0(\pi) \colon K_0(A^+) \to K_0(\mathbb{C})$.
	Definiere $K_0(A) \coloneqq \ker \enbrace*{K_0(\pi)} \subset K_0(A^+)$.\marginnote{$K_0(A)$ ist abelsche Gruppe}
\end{definition}

\begin{bemerkung}
	\begin{enumerate}[(i)]
		\item Für $p \in \Pinfty(A) \stackrel{\iota}{\subset} \Pinfty(A^+)$ gilt für $\benbrace*{p}_0 \in K_0(A^+)$
		\[
			K_0(\pi) \enbrace*{\benbrace*{p}_0} = \benbrace[\big]{\pi^\ssbrace{\infty}(p)}_0 =0
		\]
		Also ist $\benbrace*{p}_0 \in K_0(A)$ und wir definieren $\benbrace*{\cdot}_0 \colon \Pinfty(A) \to K_0(A)$.
		\item Sei $A$ eine beliebige $C^*$-Algebra (unital oder nicht).
		Wir erhalten folgende exakte Sequenz
		\[
			\begin{tikzcd}
				0 \rar & K_0(A) \rar["\kappa"] & K_0(A^+) \rar["K_0(\pi)"] & K_0(\mathbb{C})  \rar & 0
			\end{tikzcd}
		\]
		wobei $\kappa = K_0(\iota)$, falls $A$ unital ist (vgl. \autoref{prop:615}).
		Wenn $A$ unital ist, gilt $K_0(A) \cong K_0(\iota) \enbrace*{K_0(A)} \subset K_0(A^+)$, wobei der Isomorphismus $\benbrace*{p}_0 \in K_0(A)$ auf $ \benbrace*{\iota(p)}_0 \in K_0(A^+)$ abbildet.
		Da $\pi \circ \iota =0$ ist, folgt $K_0(A) = \ker K_0(\pi)$ auch im unitalen Fall.
	\end{enumerate}
\end{bemerkung}

Seien $A,B$ $C^*$-Algebren, $\varphi \colon A \to B$ ein $^*$-Homomorphismus mit Unitaltisierung $\varphi^+ \colon A^+ \to B^+$.
Wir erhalten folgendes Diagramm
\begin{equation}
	\begin{tikzcd}
		0 \rar & A \rar["\iota_A"] \dar["\varphi"] & A^+ \rar["\pi_A"]  \dar["\varphi^+"] & \mathbb{C} \rar \dar[equal] & 0 \\
		0 \rar & B \rar["\iota_B"] & B^+ \rar["\pi_B"] & \mathbb{C} \rar & 0
	\end{tikzcd} \label{eq:73:1} \tag{*}
\end{equation}
In $K_0$ erhalten wir
\begin{equation}
	\begin{tikzcd}
		0 \rar & K_0(A) \dar["\exists K_0(\varphi)"] \rar & K_0(A^+) \dar["K_0(\varphi^+)"] \rar & K_0(\mathbb{C}) \rar \dar[equal] & 0 \\
		0 \rar & K_0(B) \rar & K_0(B^+) \rar & K_0(\mathbb{C}) \rar & 0
	\end{tikzcd} \label{eq:73:2} \tag{**}
\end{equation}
Es gilt $K_0(\varphi) = K_0(\varphi^+)|_{K_0(A)}$.
Falls $A,B$ unital sind, so gilt $\varphi^+ \circ \iota_A = \iota_B \circ \varphi \colon A \to B^+$ und diese Definition von $K_0(\varphi)$ stimmt mit der aus \autoref{prop:611} überein.
Für $p \in \Pinfty(A)$ gilt $K_0(\varphi) \enbrace*{\benbrace*{p}_0} = \benbrace*{\varphi^\ssbrace{\infty}(p)}_0$

\begin{proposition}[label=prop:73]
	Seien $A,B,C$ $C^*$-Algebren, $\varphi \colon A \to B$, $\psi \colon B \to C$ $^*$-Homomorphismen.
	Dann gilt
	\begin{enumerate}[(i)]
		\item $K_0(\id_A) = \id_{K_0(A)}$
		\item $K_0(\psi \circ \varphi) = K_0(\psi) \circ K_0(\varphi)$
		\item $K_0 \enbrace*{0_{A \to B}} = 0_{K_0(A) \to K_0(B)}$
		\item $K_0(\set*{0}) = \set*{0}$.
	\end{enumerate}
	Insbesondere ist $K_0$ ein kovarianter Funktor 
	\[
		\enbrace*{C^*\text{-}\CSTAR, {^*\text{-}\Hom}} \longrightarrow \enbrace*{\AB, \Hom}
	\]
\end{proposition}
\begin{beweis}
	\begin{enumerate}[(i)]
		\item Es gilt $(\id_A)^+ = \id_{A^+}$, also
		\[
			K_0(\id_A) = K_0 \enbrace*{(\id_A)^+} \big|_{K_0(A)} = K_0(\id_{A^+})\big|_{K_0(A)} = \id_{K_0(A^+)} \big|_{K_0(A)}  =\id_{K_0(A)}
		\]
		\item Ebenso mit $(\psi \circ \varphi)^+ = \psi^+ \circ \varphi^+$
		\item Setze man $\varphi=0_{A\to B}$ in \eqref{eq:73:1}, so faktorisiert $\varphi^+$ durch $\pi_A$.
		Damit faktorisiert $K_0(\varphi^+)$ auch durch $K_0(\pi_A)$ und es gilt 
		\[
			K_0(\varphi) = K_0(\varphi^+)|_{\ker K_0(\pi_A)} = K_0(\sigma_B) \circ K_0(\pi_A)|_{\ker K_0(\pi_A)} = 0
		\]
		\item Es ist $\set*{0}^+ = \mathbb{C}$ und somit ist $\pi$ in diesem Fall die Identität auf $\mathbb{C}$.
		In $K_0$ erhalten wir
		\[
			\begin{tikzcd}
				K_0(\set*{0}) \rar & K_0(\mathbb{C}) \rar["K_0(\id_\mathbb{C})","\id"'] & K_0(\mathbb{C})
			\end{tikzcd}
		\]
		Es folgt $K_0(\set*{0}) = ker \id = \set*{0}$.\qedhere
	\end{enumerate}
\end{beweis}

\begin{proposition}
	Seien $A,B$ $C^*$-Algebren.
	\begin{enumerate}[(i)]
		\item Falls $\varphi \sim_h \psi \colon A \to B$ homotope $^*$-Homomorphismen sind, so gilt $K_0(\varphi) = K_0(\psi)$.
		\item Falls $A$ und $B$ homotopieäquivalent sind via \(
			\begin{tikzcd}[cramped,sep=small]
				A \rar["\rho"] & B \rar["\sigma"] & A
			\end{tikzcd}
		\), so sind $K_0(\rho)$ und $K_0(\sigma)$ zueinander inverse Isomorphismen.
	\end{enumerate}
\end{proposition}
\begin{beweis}
	\begin{enumerate}[(i)]
		\item Es seien $\varphi \sim_h \psi$ via $(\varphi_t)_{t \in [0,1]}$.
		Dann gilt auch $\varphi^+ \sim_h \psi^+$ via $(\varphi^+_t)_{t \in [0,1]}$.
		Mit \autoref{prop:613} folgt $K_0(\varphi^+) = K_0(\psi^+)$.
		Damit erhalten wir $K_0(\varphi) = K_0(\varphi^+)|_{K_0(A)} = K_0(\psi^+)|_{K_0(A)} = K_0(\psi)$.
		\item Folgt aus (i) mit \autoref{prop:73} (i) und (ii).\qedhere
	\end{enumerate}
\end{beweis}

\begin{beispiel}
	Sei $A$ eine $C^*$-Algebra.
	Dann ist der \Index{Kegel} 
	\[
		CA \colon \set[\big]{f \in C([0,1],A) \given  f(0)=0} = C_0 \enbrace[\big]{(0,1],A}
	\]
	homotop zu $0$ via $CA \to 0 \to CA$:
	Für $\varphi_t \colon CA \to CA$ $\varphi_t(f)(s) = f(t \cdot s)$ für $f \in CA$, $t,s \in [0,1]$.
	Die Zuordnung $t \mapsto  \enbrace[\big]{s \mapsto \varphi_t(f)(s)}$ ist stetig (warum?) und es gilt $\varphi_0=0$, sowie $\varphi_1 = \id_{CA}$.
	Damit ist $K_0(CA) =0$.
	
	Wir werden sehen, dass die \Index{Einhängung} 
	\[
		SA \coloneqq C_0 \enbrace*{(0,1),A}
	\]
	nichttriviale $K$-Theorie hat (im Allgemeinen).
\end{beispiel}

\begin{definitionP}[label=prop:76]
	Betrachte \(
		\begin{tikzcd}[cramped,sep=small]
			0 \rar & A \rar & A^+ \rar["\pi"] & \mathbb{C} \lar[bend right,"\sigma"'] \rar & 0
		\end{tikzcd}
	\) und definiere $s_A \colon A^+ \to A^+$ durch $s_A \coloneqq \sigma \circ \pi$.
	Dann gilt $\pi \circ s_A = \pi \circ \sigma \circ \pi= \pi$ und $a \in s_A(a) \in \ker \pi$ für alle $a \in A^+$.
	$s_A$ induziert $s_A^\ssbrace{n} \colon M_n(A^+) \to M_n(A^+)$ mit $\im s_A^\ssbrace{n} = M_n(\mathbb{C} \cdot \ind_{A^+})$.
	Falls $\varphi \colon A \to B$ ein $^*$-Homomorphismus ist, so kommutiert 
	\[
		\begin{tikzcd}
			A^+ \rar["s_A"] \dar["\varphi^+"] & A^+ \dar["\varphi^+"] \\
			B^+ \rar["s_B"] &  B^+
		\end{tikzcd}
	\]
	Ebenso für $s_A^\ssbrace{n}$.
	Wir schreiben oft auch $s$ für $s_A^\ssbrace{n}$.
	Für jede $C^*$-Algebra $A$ gilt 
	\begin{enumerate}[(i)]
		\item $K_0(A) = \set[\big]{\benbrace*{p}_0 - \benbrace*{s(p)}_0 \given p \in \Pinfty(A^+)}$
		\item Für $p,q \in \Pinfty(A^+)$ gilt 
		\begin{enumerate}[a)]
			\item $\benbrace*{p}_0 - \benbrace*{s(p)}_0 = \benbrace*{q}_0 - \benbrace*{s(q)}_0$
			\item $\exists k,l \in \mathbb{N} : p \oplus \ind_{k} \sim_0 q \oplus \ind_l$
			\item $\exists v_1,v_2 \in \kernedP_n(\mathbb{C} \cdot \ind_{A^+}) : p \oplus v_1 \sim_0 q \oplus v_2$ in $\Pinfty(A^+)$ 
		\end{enumerate}
		\item Für jedes $p \in \Pinfty(A^+)$ mit $\benbrace*{p}_0 = \benbrace*{s(p)}_0$ existiert $m \in \mathbb{N}$ mit $p \oplus \ind_m \sim_0 s(p) \oplus \ind_m$.
	\end{enumerate}
\end{definitionP}
\begin{beweis}
	\begin{enumerate}[(i)]
		\item Zunächst zeigen wir \enquote{$\supset$}: Für $p \in \Pinfty(A^+)$ gilt 
		\[
			K_0(\pi)\enbrace*{\benbrace*{p}_0 - \benbrace*{s(p)}_0} = \benbrace*{\pi(p)}_0 - \benbrace*{\pi s(p)}_0 =0
		\]
		Damit folgt \enquote{$\supset$}.
		Für die umgekehrte Inklusion sei $x \in K_0(A)$.
		Dann existieren $n \in \mathbb{N}$, $e,f \in \kernedP_n(A^+)$ mit $x = \benbrace*{e}_0 - \benbrace*{f}_0$.
		Setze $p \coloneqq e \oplus (\ind_n -f)$ und $q \coloneqq 0_n \oplus \ind_n \in \kernedP(A^+)$.
		Dann gilt
		\[
			\benbrace*{p}_0 - \benbrace*{q}_0 = \benbrace*{e}_0 + \benbrace*{\ind_n -f} - \benbrace*{\ind_n} = \benbrace*{e}_0 -\benbrace*{f}_0 = x
		\]
		Es gilt $s(q)=q$ und $K_0(\pi)(a) =0$.
		Deshalb gilt
		\[
			\benbrace*{s(p)}_0 - \benbrace*{q}_0 = \benbrace*{s(p)}_0 - \benbrace*{s(q)}_0 = K_0(s)(x) = K_0(\sigma) K_0(\pi_n)(x) =0
		\]
		Damit ist $x = \benbrace*{p}_0 - \benbrace*{q}_0 = \benbrace*{p}_0 - \benbrace*{s(p)}_0$ und die Gleichheit der Mengen folgt.
	\end{enumerate}
\end{beweis}

\begin{proposition}[label=prop:77]
	
\end{proposition}

\todo[inline]{Hier fehlt noch eine halbe Vorlesung}

\begin{proposition}[label=prop:78]
	Sei 
	\begin{equation}
		\begin{tikzcd}[cramped,sep=small]
			0 \rar & J \rar["\iota"] & A \rar["\pi"]  & B \rar & 0
		\end{tikzcd} \label{eq:78:1} \tag{\#}
	\end{equation}
	eine kurze exakte Sequenz von $C^*$-Algebren.
	Dann ist 
	\begin{equation}
		\begin{tikzcd}
			K_0(J) \rar["K_0(\iota)"] & K_0(A) \rar["K_0(\pi_n)"] & K_0(B)
		\end{tikzcd} \label{eq:78:2} \tag{\#\#}
	\end{equation}
	exakt.
	Falls \eqref{eq:78:2} zerfällt mit $\sigma \colon B \to A$, $\pi \circ \sigma = {\id_B}$, so ist 
	\begin{equation}
		\begin{tikzcd}
			0 \rar & K_0(J) \rar & K_0(A) \rar & K_0(B) \rar & 0
		\end{tikzcd} \label{eq:78:3} \tag{\#\#\#}
	\end{equation}
	exakt und zerfällt.
\end{proposition}
\begin{beweis}
	\begin{enumerate}[(i)]
		\item Es gilt $K_0(\pi) \circ K_0(\iota) = K_0(\pi \circ \iota) = K_0(0) =0$.
		Folglich ist $\im K_0(\iota) \subset \ker K_0(\pi)$.
		Sei nun $x \in \ker K_0(\pi)$.
		Mit \autoref{prop:77} folgt die Existenz von $n \in \mathbb{N}$, $p \in \kernedP_n(A^+)$ mit $x = \benbrace*{p}_0 - \benbrace*{s_A(p)}_0$ und ${\pi^+}^\ssbrace{n}(p) = s_B^\ssbrace{n} {\pi^+}^\ssbrace{n}(p)$.
		Es folgt
		\[
			\Underbracket{{\pi^+}^\ssbrace{n}(p)}{\in \sigma^\ssbrace{n}_B(M_n)} \in s_B^\ssbrace{n} \enbrace*{M_n(B^+)} 
		\]
		Damit ist $p \in \sigma_A^\ssbrace{n}(M_n) + \iota^\ssbrace{n}(M_n(J)) = {\iota^+}^\ssbrace{n} \enbrace*{M_n(J^+)}$.
		Wegen Injektivität finden wir nun ein $e \in \kernedP_n(J^+)$ mit ${\iota^+}^\ssbrace{n}(e)=p$ und somit gilt 
		\[
			x = \benbrace*{{\iota^+}^\ssbrace{n}(e)}_0 - \benbrace*{s_A {\iota^+}^\ssbrace{n}(e)}_0 = \benbrace*{{\iota^+}^\ssbrace{n}(e)}_0 - \benbrace*{{\iota^+}^\ssbrace{n} s_J(e)}_0 = K_0(\iota) \enbrace*{\benbrace*{e}_0 - \benbrace*{s_J(e)}_0}
		\]
		Dies ist im Bild von $K_0(\iota)$ und es folgt, dass \eqref{eq:78:2} exakt ist.
		\item \eqref{eq:78:3} ist exakt in der Mitte. Es gilt
		\[
			K_0(\pi)K_0(\sigma) = K_0(\pi \sigma) = K_0(\id_B) = {\id_{K_0(N)}}
		\]
		und damit ist $K_0(\pi)$ surjektiv. 
		\eqref{eq:78:3} ist also exakt bei $K_0(B)$ und zerfällt.
		Sei nun $x \in \ker K_0(\iota) \subset K_0(J)$.
		Nach \autoref{prop:77} (ii) existieren $n \in \mathbb{N}$, $p \in \kernedP_n(J^+)$ und $u \in \mathcal{U} \enbrace*{M_n(A^+)}$ mit $x = \benbrace*{p}_0 - \benbrace*{s_J^\ssbrace{n}(p)}_0$ und $u{\iota^+}^\ssbrace{n}(p) u^* = s_A{\iota^+}^\ssbrace{n}(p)$.
		Setze nun $v \coloneqq {\sigma^+}^\ssbrace{n} {\pi^+}^\ssbrace{n}(u^*)u \in \mathcal{U} \enbrace*{M_n(A^+)}$.
		Dann gilt ${\pi^+}^\ssbrace{n}(v) = \ind_{M_n(B^+)} = s_B^\ssbrace{n} {\pi^+}^\ssbrace{n}(v)$.
		Damit existiert wie in (i) ein $w \in \mathcal{U} \enbrace*{M_n(J^+)}$ mit ${\iota^+}^\ssbrace{n}(w)=v$.
		Wir haben nun
		\begin{align}
			{\iota^+}^\ssbrace{n} (w pw^*) = v {\iota^+}^\ssbrace{n}(p) v^* = \sigma\pi(u^*) u \iota(p) u^* \sigma \pi(u) &= \sigma \pi (u^*) s_A\iota(p) \sigma \pi u \\
			&= \sigma \pi \enbrace*{u^* s_A \iota(p) u} \\
			&= \sigma \pi\enbrace*{\iota(p)} \\
			&= s_A \iota(p) \\
			&= {\iota^+}^\ssbrace{n} s_J^\ssbrace{n} (p)
		\end{align}
		Damit ist $p \sim_u s_J(p)$ und somit auch $\benbrace*{p}_0 = \benbrace*{s_J(p)} $, also $x=0$ und $K_0(\iota)$ ist injektiv.\qedhere
	\end{enumerate}
\end{beweis}

\begin{korollar}[{name=[{$K_0$ von der direkten Summe}]}]
	Für $C^*$-Algebren $A,B$ gilt $K_0(A \oplus B) = K_0(A) \oplus K_0(B)$.
\end{korollar}
\begin{beweis}
	Betrachte \(
		\begin{tikzcd}[cramped,sep=2em]
			0 \rar & A \rar["\iota_A"] & A \oplus B \rar["\pi_B"] & B \lar[bend right,"\iota_B"'] \rar & 0
		\end{tikzcd}
	\).
	Wir erhalten folgendes Diagramm in $K$-Theorie
	\[
		\begin{tikzcd}
			0 \rar & K_0(A)\dar[equal]  \rar & K_0(A \oplus B) \rar & K_0(B) \dar[equal] \rar & 0 \\
			0 \rar & K_0(A) \rar & K_0(A) \oplus K_0(B) \uar["K_0(\iota_A) + K_0(\iota_B)","\cong"'] \rar & K_0(B) \rar & 0
		\end{tikzcd}
	\]
	Den Isomorphismus erhalten wir durch das Fünfer-Lemma.
\end{beweis}

\begin{beispiel}
	\begin{enumerate}[(i)]
		\item Betrachte \(
			\begin{tikzcd}
				0 \rar & A \rar & A^+ \rar & \mathbb{C} \lar[bend right] \rar & 0
			\end{tikzcd}
		\). Dann erhalten wir auch
		\[
			\begin{tikzcd}
				0 \rar & K_0(A) \rar & K_0(A^+) \rar & K_0(\mathbb{C}) \rar \lar[bend right] & 0
			\end{tikzcd}
		\]
		Damit ist $K_0(A^+) \cong K_0(A) \oplus K_0(\mathbb{C})$.
		\item 
		\[
			\begin{tikzcd}[column sep=large,row sep =small]
				K_0\enbrace*{C_0 \enbrace*{(0,1)}} \rar & K_0 \enbrace*{C \enbrace*{[0,1]}} \rar["K_0(\ev_0 \oplus \ev_1)"] & K_0(\mathbb{C} \oplus \mathbb{C}) \\
				& K_0(\mathbb{C}) \uar[phantom,sloped,"\cong"'] & K_0(\mathbb{C}) \oplus K_0(\mathbb{C}) \uar[phantom,sloped,"\cong"']\\
				& \mathbb{Z}  \uar[phantom,sloped,"\cong"']& \mathbb{Z}\oplus \mathbb{Z} \uar[phantom,sloped,"\cong"']
			\end{tikzcd}
		\]
		\item Betrachte den Hilbertraum $\mathcal{H} = \ell^2(\mathbb{N})$.
		\[
			\begin{tikzcd}[row sep=small]
				K_0(\mathcal{K}) \rar["K_0(\iota)"] & K_0(\mathcal{B}(\mathcal{H})) \rar & K_0 \enbrace*{\sfrac{\mathcal{B}(\mathcal{H})}{\mathcal{K}}} \\
				\mathbb{Z} \uar[phantom,sloped,"\cong"'] & 0 \uar[phantom,sloped,"="']
			\end{tikzcd}
		\]
		Also ist $K_0(\iota)$ im Allgemeinen nicht injektiv.
	\end{enumerate}
\end{beispiel}

\begin{proposition}[label=prop:711]
	Sei $A$ eine $C^*$-Algebra und $\iota_{nA} \colon A \to M_n(A)$ die Einbettung, die $a \in A$ auf die Matrix mit $a_{11}= a$ und $a_{ij}=0$ sonst abbildet.
	Dann ist $K_0(\iota_{nA}) \colon K_0(A) \to K_0(M_n(A))$ ein Isomorphismus.
\end{proposition}
\begin{beweis}
	Betrachte 
	\[
		\begin{tikzcd}
			0 \rar & K_0(A) \rar \dar["K_0(\iota_{nA})"] & K_0(A^+) \rar \dar["K_0(\tilde{\iota}_{nA})"] & K_0(\mathbb{C}) \rar \dar["K_0(\iota_{n \mathbb{C}})"] & 0 \\
			0 \rar & K_0(M_n(A)) \rar & K_0(M_n(A^+)) \rar & K_0(M_n) \rar & 0
		\end{tikzcd}
	\]
	Nach dem Fünfer-Lemma genügt es also die Proposition für unitale $C^*$-Algebren zu beweisen.
	Für $k \in \mathbb{N}$ wähle $\gamma_{n,k} \colon M_k(M_n(A)) \xrightarrow{\cong} M_{kn}(A)$.
	Dies liefert uns $\gamma_{n,\infty} \colon \Pinfty(M_n(A)) \to \Pinfty(A)$, wobei $\gamma_{n,\infty} \big|_{\kernedP_k(M_n(A))} \coloneqq \gamma_{n,k}\big|_{\kernedP(M_n(A))}$.
	Wir erhalten
	\mapdef{\gamma_n \colon \Pinfty(M_n(A))}{K_0(A)}{p}{\benbrace[\big]{\gamma_{n,\infty}(p)}}{}
	$\leadsto$ universelle Eigenschaft für $K_0(\cdot)$. 
	Damit erhält man $\gamma \colon K_0(M_n(A)) \to K_0(A)$.
	Überprüfe nun, dass $\gamma$ und $K_0(\iota_{nA})$ zueinander invers sind.
\end{beweis}

Wir haben in \autoref{prop:68} gesehen: Für $p,q \in \mathcal{P}(A)$ gilt $\norm*{p-q} < \sfrac{1}{2} \implies p \sim_h q$.\marginnote{tatsächlich genügt $\norm*{p-q}<1$}
Wir wollen nun einige ähnliche Aussagen beweisen:

\begin{lemma}[label=lem:712]
	Sei $A$ eine $C^*$-Algebra.
	\begin{enumerate}[(i)]
		\item Falls $a \in A_\sa$ mit $\delta \coloneqq \norm*{a-a^2}< \sfrac{1}{4}$, so existiert $p \in \mathcal{P}(A)$ mit $\norm*{a-p} \le 2 \cdot \delta$.
		\item Falls $p,q \in \mathcal{P}(A)$ und $x \in A$ mit $\norm*{x^*x -p}, \norm*{xx^* -q} < \sfrac{1}{4}$, so gilt $p \sim_{\MvN} q$.\marginnote{tatsächlich reicht $< \sfrac{1}{2}$}
	\end{enumerate}
\end{lemma}
\begin{beweis}
	\begin{enumerate}[(i)]
		\item Nach dem Spektralsatz ist $\sigma(a) \subset \set*{t \in \mathbb{R} \given \abs*{t-t^2} \le \delta } \subset \benbrace*{-2 \delta, 2 \delta} \cup \benbrace*{1- 2 \delta, 1+ 2 \delta}$.
		Da $\delta < \sfrac{1}{4}$  ist, ist $\sfrac{1}{2}$ nicht im Spektrum enthalten. Also definiert 
		\[
			f(t) \coloneqq \begin{cases}
				0 &\text{ falls }t \le 2 \delta\\
				1 &\text{ falls } t \ge 1- 2 \delta
			\end{cases}
		\]
		eine stetige Funktion auf $\sigma(a)$.
		Dann ist $p \coloneqq f(a)$ eine Projektion, für die wie gewünscht $\norm*{p-a} = \norm*{{\id_{\sigma(a)}} -f}_{\sigma(a)} \le 2 \delta$ gilt.
		\item Setze $\delta \coloneqq \sfrac{1}{2} \max \set*{\norm*{x^*x -p}, \norm*{xx^* -q}} < \sfrac{1}{8}$.
		Überprüfe, dass $\sigma(x^*), \sigma(xx^*) \subset \benbrace*{-2 \delta, 2 \delta} \cup \benbrace*{1- 2 \delta, 1+2\delta}$.
		Sei $f$ wie in (i).
		Definiere Projektionen $p_0 \coloneqq f(x^*x)$, $q_0 \coloneqq f(xx^*)$.
		Dann gilt $\norm*{p-p_0} \le 3 \delta< \sfrac{3}{8} < \sfrac{1}{2}$ und somit $p \sim_{\MvN} p_0$ und ebenso $q \sim_{\MvN} q_0$.
		Definiere nun 
		\[
			g(t) \coloneqq \begin{cases}
				0 &\text{ falls }t \le 2 \delta\\
				t^{-\sfrac{1}{2}} &\text{ falls } t \ge 1 - 2 \delta
			\end{cases}
		\]
		Dann ist $g$ stetig auf $\sigma(x^*x)$, $\sigma(xx^*)$ und es gilt $t g(t)^2 = f(t)$.
		Setze $v \coloneqq x g(x^*x)$.
		Dann gilt 
		\[
			v^* v = g(x^*x) x^*x g(x^*x) = f(x^*x) = p_0 \quad \text{und} \quad  vv^* = x g(x^*x)^2 x^* = g(xx^*)^2 xx^* = f(xx^*) = q_0
		\]
		Damit ist $p \sim_{\MvN} p_0 \sim_{\MvN} q_0 \sim_{\MvN} f$.\qedhere
	\end{enumerate}
\end{beweis}

\begin{satz}[label=satz:713]
	Sei 
	\[
		\begin{tikzcd}[sep=3em]
			A_1 \rar & \ldots \rar & A_n \rar["\varphi_{n,n+1}"'] \ar[rrr,bend left=20,"\varphi_{n,\infty}"] & A_{n+1} \rar & \ldots  \rar & A \coloneqq \Dlim (A_n,\varphi_{n,n+1}) 
		\end{tikzcd}
	\]
	ein induktives System von $C^*$-Algebren mit Limes $A$.
	Dann gilt
	\[
		K_0(A) = K_0(\Dlim A_n) \cong \Dlim K_0(A_n)
	\]
	Mit anderen Worten: $K_0(\cdot)$ ist stetig.
\end{satz}
\begin{beweis}
	\begin{enumerate}[(i)]
		\item Es gilt $K_0(A) = \bigcup_{n=1}^\infty K_0(\varphi_{n,\infty}) \enbrace*{K_0(A_n)}$:
		Die Inklusion \enquote{$\supset$} ist trivial.
		Für die andere sei $x \in K_0(A)$.
		Dann existieren $k \in \mathbb{N}$, $p \in \kernedP_k(A^+)$ mit $x = \benbrace*{p}_0 - \benbrace*{s(p)}_0$.
		Das induktive System induziert
		\[
			\begin{tikzcd}
				M_k(A_1^+) \rar & \ldots \rar & M_k(A_n^+) \rar & \ldots  \rar & M_k{A^+}
			\end{tikzcd}
		\]
		Es gibt also $n_0 \in \mathbb{N}$, $a \in M_k(A^+_{n_0})_\sa$ mit $\norm*{{\varphi_{n_0, \infty}^+}^\ssbrace{k}(a)  -p } < \sfrac{1}{8}$ und $\norm*{{\varphi_{n_0, \infty}^+}^\ssbrace{k}(a -a^2)} < \sfrac{1}{8}$.
		Daraus erhalten wir die Existenz von $n_1 \in \mathbb{N}$, $\overline{a} \in M_k \enbrace*{A_{n_1}^+}_\sa^1$ mit 
		\[
			\norm*{\varphi_{n_1,\infty}(\overline{a}) -p} < \sfrac{1}{8} \qquad \norm*{\overline{a} - \overline{a}^2} < \sfrac{1}{8}
		\]
		Mit \autoref{lem:712} folgt, dass $p \in $ existiert mit.
		Also gilt $\norm*{p - \varphi_{n_1,\infty}(q)} < \sfrac{1}{2}$ und somit $p \sim_0 \varphi_{n_1,\infty}(q)$.
		Damit können wir $x$ auch schreiben als
		\begin{align}
			x = \benbrace*{p}_0 - \benbrace*{s(p)}_0 &= \benbrace*{\varphi_{n_1,\infty}(q)}_0 - \benbrace*{s_A \enbrace*{ \varphi_{n_1,\infty}(q)}}_0 \\
			&= K_0 \enbrace*{\varphi_{n_1,\infty}} \enbrace*{\benbrace*{q}_0 - \benbrace[\big]{s_{A_{n_1}}(q)}_0} \in K_0(\varphi_{n_1,\infty}) \enbrace*{K_0(A_{n_1})}
		\end{align}
		\item Es gilt $\ker \enbrace*{K_0(\varphi_{n,\infty})} = \bigcup_{mm=n+1}^\infty \ker \enbrace*{K_0(\varphi_{n,m})}$ für alle $n \in \mathbb{N}$.
		
		\enquote{$\supset$} ist klar. Sei umgekehrt $x \in \ker \enbrace*{K_0(\varphi_{n,\infty})} \in K_0(A_n)$.
		Mit \autoref{prop:76} (i),(iii) folgt, dass ein $k \in \mathbb{N}$ und ein $p \in \kernedP_k(A^+_n)$ existieren mit $x=\benbrace*{p}_0 - \benbrace*{s(p)}_0$ und 
		\[
			{\varphi_{n,\infty}^+}^\ssbrace{k}(p) \sim {\varphi_{n,\infty}^+}^\ssbrace{k} \enbrace*{s(p)} 
		\]
		in $M_k(A^+)$.
		Das heißt es existiert $v \in M_k(A^+)$ mit $v^*v = \varphi_{n,\infty}$ und $vv^* = \varphi_{n,\infty}(s(p))$.
		Damit gilt für ein $m \ge n$, $w \in M_k(A_m^+)$ mit
		\[
			\norm*{w^*w - {\varphi_{n,m}^+}^\ssbrace{k}(p)} , \quad \norm*{ww^* - {\varphi_{n,m}^+}^\ssbrace{k}(s(p))} < \frac{1}{4} 
		\]
		mit \autoref{lem:712} (ii) folgt ${\varphi_{n,m}^+}^\ssbrace{k}(p) \sim_{\MvN} {\varphi_{n,m}^+}^\ssbrace{k}(s(p))$.
		Folglich gilt dann auch
		\[
			K_0(\varphi_{n,m})(x) = \benbrace*{{\varphi_{n,m}^+}^\ssbrace{k}(p)}_0 - \benbrace*{{\varphi_{n,m}^+}^\ssbrace{k}(s(p))}_0 =0
		\]
		Also ist $x \in \ker K_0(\varphi_{n,m})$.
		\item Betrachte
		\[
			\begin{tikzcd}
				K_0(A_1) \rar & \ldots \rar & K_0(A_n)\ar[drr,"K_0(\varphi_{n,\infty})"]  \rar & \ldots \rar & \Dlim K_0(A_n) \dar["\mu"] \\
				& & & & K_0(A)
			\end{tikzcd}
		\]
		mit (i) folgt die Surjektivität und mit (ii) die Injektivität von $\mu$.\qedhere
	\end{enumerate}
\end{beweis}

\begin{korollar}
	Für jede $C^*$-Algebra $A$ induziert ${\id_A} \otimes e_{11} \colon A \to A \otimes \mathcal{K}$ einen Isomorphismus $K_0(A) = K_0(A \otimes \mathcal{K})$.
\end{korollar}
\begin{beweis}
	Betrachte
	\[
		\begin{tikzcd}
			A \otimes M_1 \rar & A \otimes M_2 \rar & \ldots \rar & A \otimes \mathcal{K} = \Dlim A \otimes M_n \\
			A \uar["{\id} \otimes e_{11}"] \urar["{\id} \otimes e_{11}"'] \ar[urrr,"{\id} \otimes e_{11}"']
		\end{tikzcd}
	\]
	Damit erhalten wir 
	\[
		\begin{tikzcd}
			K_0(A \otimes M_1) \rar["\cong"] & K_0(A \otimes M_2) \rar & \ldots \rar & K_0(A \otimes \mathcal{K})  \\
			K_0(A) \uar["\cong"',"\text{\ref{prop:711}}"] \urar["\cong"]  \ar[urrr,"\cong"] & & & \Dlim K_0(A \otimes M_n) \uar[equal,"\text{\ref{satz:713}}"]
		\end{tikzcd}
	\]
\end{beweis}
% section 7 (end)
\newpage

\section{Der Funktor $K_1$} % (fold)
\label{sec:8}

\begin{definitionP}[label=def:81]
	Für eine unitale $C^*$-Algebra $A$ setzen wir $U_m(A) = \mathcal{U}(M_m(A))$ und
	\[
		U_\infty(A) \coloneqq \bigcup_{n=1}^\infty U_n(A)
	\]
	Dabei $U_n(A) \subseteq U_{n+1}(A)$ via $u \hat{=} \begin{psmallmatrix}
		u & \\ & \ind 
	\end{psmallmatrix} = u \oplus \ind$.
	Für $u,v \in U_\infty(A)$ mit $u \in U_n(A)$ und $v \in U_m(A)$ setzen wir 
	\[
		u \oplus v = \begin{pmatrix}
			u & 0 \\
			0 & v
		\end{pmatrix} \in U_{n+m}(A) \subseteq U_\infty(A)
	\]
	Für solche $u,v$ definieren wir
	\[
		u \sim_1 v :\Leftrightarrow \exists k \ge \max(n,m) : u \oplus \ind_{k-n} \sim_h v \oplus \ind_{k-m} \text{ in } U_k(A) 
	\]
	Dann ist $\sim_1$ eine Äquivalenzrelation auf $U_\infty(A)$ und es gilt 
	\begin{itemize}
		\item $u \sim_1 u \oplus \ind_\ell$ $\forall u \in U_\infty(A), \ell \in \mathbb{N}$
		\item $u \oplus v \sim_1 v \oplus u$ $\forall u,v \in U_\infty(A)$ via
		\[
			\begin{pmatrix}
				u & \\ & v
			\end{pmatrix} = \Underbracket{\begin{pmatrix}
				0 & \ind \\
				\ind & 0
			\end{pmatrix}}{\sim_h \ind}
			\begin{pmatrix}
				v & \\ & u
			\end{pmatrix}
			\begin{pmatrix}
				0 & \ind \\ \ind & 0
			\end{pmatrix}
			\sim_h \begin{pmatrix}
				v & \\ & u
			\end{pmatrix}
		\]
		\item $u \sim_1 u' , v \sim_1 v' \implies u \oplus v \sim_1 u' \oplus v'$
		\item $uv \sim_1 u \oplus v$ via
		\[
			\begin{pmatrix}
				u & \\ & v
			\end{pmatrix} = \begin{pmatrix}
				u & \\ & \ind 
			\end{pmatrix} \begin{pmatrix}
				0 & \ind \\ \ind & 0
			\end{pmatrix}
			\begin{pmatrix}
				v & \\ & \ind
			\end{pmatrix}
			\begin{pmatrix}
				0 & \ind \\ \ind & 0
			\end{pmatrix}
			\sim_h \begin{pmatrix}
				uv & \\ & \ind
			\end{pmatrix}
		\]
		\item $(u \oplus v) \oplus w = u \oplus  (v \oplus w)$.
	\end{itemize}
\end{definitionP}

\begin{definitionP}
	Für eine $C^*$-Algebra $A$ definieren wir 
	\[
		K_1(A) \coloneqq \sfrac{\mathcal{U}_\infty(A^+)}{\sim_1}
	\]
	und schreiben $\benbrace*{u}_1$ für die Äquivalenzklasse von $u$.
	$K_1(A)$ ist eine abelsche Gruppe via $\benbrace*{u}_1 + \benbrace*{v}_1 = \benbrace*{u \oplus v}_1$ für $u,v \in \mathcal{U}_\infty(A)$.
	Man hat $- \benbrace*{u}_1 = \benbrace*{u^*}_1$ und $0_{K_1(A)} = \benbrace*{\ind}_1$
\end{definitionP}

\begin{proposition}[label=prop:83]
	Sei $A$ eine $C^*$-Algebra, $G$ eine abelsche Gruppe und $\nu \colon \mathcal{U}_\infty(A^+) \to G$ eine Abbildung mit 
	\begin{enumerate}[a)]
		\item $\nu(u \oplus v) = \nu(u) + \nu(v)$
		\item $\nu(\ind)=0$
		\item $u,v \in \mathcal{U}_n(A^+)$, $u \sim_h v \implies \nu(u) = \nu(v)$
	\end{enumerate}
	Dann existiert genau ein Homomorphismus $\alpha \colon K_1(A) \to G$, sodass das folgende Diagramm kommutiert:
	\[
		\begin{tikzcd}
			U_\infty(A^+) \dar["\benbrace*{-}_1"] \drar["\nu"] \\
			K_1(A) \rar[dashed,"\exists ! \alpha"] & G
		\end{tikzcd}
	\]
\end{proposition}

\begin{bemerkung}[label=bem:84]
	\begin{enumerate}[(i)]
		\item Es gilt $K_1(A) = \set*{\benbrace*{u}_1 \given u \in \mathcal{U}_k(A^+), \pi^{+\ssbrace{k}}(u) =\ind_k}$
		\item Falls $A$ unital ist, so ist $K_1(A) \cong \sfrac{\mathcal{U}_\infty(A)}{\sim_1}$
	\end{enumerate}
\end{bemerkung}
\begin{beweis}
	\emph{Übung!}
\end{beweis}

Seien $A,B$ $C^*$-Algebren und $\varphi \colon A \to B$ ein $^*$-Homomorphismus.
Wir erhalten eine induzierte Abbildung $\varphi^{+\ssbrace{\infty}} \colon \mathcal{U}_\infty(A^+) \to \mathcal{U}_\infty(B^+)$.
Man überprüft, dass $\nu \colon U_\infty(A^+) \to K_1(B)$, $u \mapsto \benbrace*{\varphi^{+\ssbrace{\infty}}(u)}_1$ die Bedingungen aus \autoref{prop:83} erfüllt.
Dann erhält man $K_1(\varphi) \colon K_1(A) \to K_1(B)$.

\begin{proposition}[label=prop:85]
	Seien $A,B,C$ $C^*$-Algebren, $\varphi \colon A \to B$, $\psi \colon B \to C$ $^*$-Homomorphismen.
	Dann gilt
	\begin{enumerate}[(i)]
		\item $K_1({\id_A}) = {\id_{K_1(A)}}$
		\item $K_1(\psi \circ \varphi) = K_1(\psi) \circ K_1(\varphi)$
	\end{enumerate}
	Insbesondere ist auf diese Weise $K_1(-)$ ein kovarianter Funktor
	\[
		\enbrace*{\CSTAR, {^*}\text{-}\Hom} \longrightarrow \enbrace*{\AB, \Hom}
	\]
	Es gelten außerdem:\todo[inline]{Beginn ändern}
	\begin{enumerate}[(i)]
		\item $K_1(\set*{0}) = \set*{0}$
		\item $K_1(0_{A\to B}) = 0_{K_1(A) \to K_1(B)}$
		\item $\varphi_0 \sim \varphi_1 \colon A \to B \implies K_1(\varphi_0) = K_1(\varphi_1)$
		\item Falls $A,B$ homotopieäquivalent via \(
			\begin{tikzcd}[cramped,sep=small]
				A \rar["\rho"] & B \rar["\sigma"] & A
			\end{tikzcd}
		\) sind, so sind $K_1(\rho)$, $K_1(\sigma)$ zueinander inverse Isomorphismen.
		\item Falls $x \in \ker K_1(\varphi)$, so existiert $u \in \mathcal{U}_\infty(A^+)$ mit $x= \benbrace*{u}_1$ und $\varphi^{+\ssbrace{\infty}}(u) \sim_h \ind$.
		Falls $\varphi$ surjektiv ist, so kann man $\varphi^{+\ssbrace{\infty}}(u)= \ind$ annehmen.
	\end{enumerate}
\end{proposition}
\begin{beweis}
	(i) -(iv) Im Wesentlichen wie für $K_0$: Klar mit $(\psi \circ \varphi)^{+\ssbrace{\infty}} = \psi^{+\ssbrace{\infty}} \circ \varphi^{+\ssbrace{\infty}}$ und $\varphi_0 \sim_h \varphi_1 \implies \varphi_0^{+\ssbrace{\infty}} \sim \varphi_1^{+\ssbrace{\infty}}$
	\begin{enumerate}[(i)]
		\item[vii] $x=\benbrace*{u}_1$ mit $u \in \mathcal{U}_m(A^+)$ und $\varphi^{+\ssbrace{\infty}}(u) \sim_1 \ind_m$.
		Nach \autoref{def:81} existiert ein $k$ mit
		\[
			\Underbracket{\varphi^{+\ssbrace{m}}(n) \oplus \ind_k}{= \varphi^{+\ssbrace{m+k}}(u \oplus \ind_k)} \sim_h \ind_{m+k}
		\]
		Ersetze un $u$ durch $u \oplus \ind_k$.
		Ist $\varphi$ surjektiv, so $\varphi^{+\ssbrace{n}}(u) = e^{ih_1} \cdots e^{i h_k}$ für $h_1, \ldots ,h_k \in  M_m(A^+)_\sa$.\footnote{Für $v \in \mathcal{u}(A^+)$ mit $v \sim_h \ind$ finde $v_1, \ldots ,v_n \in \mathcal{u}(A^+)$ mit $v_1 = \ind$, $v_n =v$, $2 > \norm*{v_i - v_{i+1}}$, Funktionalkalkül, $-1 \notin \Sp(v_i^* v_{i+1}) \implies v_i^* v_{i+1}= e^{ih}$ mit $h \in A_\sa^+$}
		Lifte jedes $h_i$ zu $g_i \in M_m(A^+)_\sa$.
		Dann ist $v= e^{ig_1} \cdots e^{i g_k} \in \mathcal{U}_m(A^+)$ mit $\varphi^{+\ssbrace{m}}(v) = \varphi^{+\ssbrace{m}}(u)$.
		Dann ersetze $u$ durch $u v^*$ ($\varphi^{+ \ssbrace{m}}(uv^*)=\ind_m$ und $u \sim_h u v^*$, weil $v \sim_h \ind$)
	\end{enumerate}
\end{beweis}

\begin{proposition}
	$K_1(-)$ ist halbexakt, das heißt für eine kurze exakte Sequenz
	\[
		\begin{tikzcd}
			0 \rar & J \rar["\iota"] & A \rar["\pi_n"] & B \rar & 0
		\end{tikzcd}
	\]
	ist 
	\[
		\begin{tikzcd}
			K_1(J) \rar["K_1(\iota)"] & K_1(A) \rar["K_1(\pi)"] & K_1(B)
		\end{tikzcd}
	\]
	exakt.
\end{proposition}
\begin{beweis}
	$\im K_1(\iota) \subseteq \ker K_1(\pi)$ ist klar mit \autoref{prop:85} (ii), (iv).
	Sei $x \in \ker K_1(\pi)$.
	Mit \autoref{prop:85} (vii) folgt $x=\benbrace*{u}_1$ mit $u \in \mathcal{U}_m(A^+)$, $\pi^{+ \ssbrace{m}}(u)=\ind_m$.
	Also ist $\pi^{+\ssbrace{m}(u-1)}=0 \implies u - \ind_m \in \iota^{+\ssbrace{m}(M_m(J^+))}$.
	Mit der Injektivität von $\iota$ folgt die Existenz eines $v \in \mathcal{U}_m(J^+)$ mit $\iota^{+\ssbrace{m}}(v)=u$.
	Damit folgt $K_1(\iota)(v) = \benbrace*{u}_1= x$, also $x \in \im K_1(\iota)$.
\end{beweis}

Im Wesentlichen wie bei $K_0$:

\begin{proposition}
	$K_1(-)$ ist split-exakt, das heißt für \(
		\begin{tikzcd}[cramped,sep=small]
			0 \rar & J \rar & A \rar & B \lar[bend right] \rar & 0
		\end{tikzcd}
	\)
	exakt und zerfallend ist auch
	\[
		\begin{tikzcd}
			0 \rar & K_1(J) \rar & K_1(A) \rar & K_1(B) \lar[bend right] \rar & 0
		\end{tikzcd}
	\]
	exakt und zerfallend.
	Insbesondere gilt für jede zwei $C^*$-Algebren $A$ und $B$
	\[
		K_1(A \oplus B) \cong K_1(A) \oplus K_1(B)
	\]
\end{proposition}

\begin{proposition}
	$K_1(-)$ ist stetig, das heißt
	\[
		K_1 \enbrace*{\Dlim A_i} \cong \Dlim K_1(A_i)
	\]
	für jedes induktive System $A_1 \to A_2 \to A_2 \to \ldots $.
\end{proposition}
\begin{beweis}
	Wie für $K_1$ mit Blatt 11, Aufgabe 1 anstelle von \autoref{lem:712}.
\end{beweis}

\begin{proposition}
	$K_1(-)$ ist stabil, das heißt $e_{11} \otimes {\id_A}$ induziert einen Isomorphismus
	\[
		K_1(A) \cong K_1(M_n(A))
	\]
	für $n \ge 2$ und $K_1(A) \cong K_1(\mathcal{K} \otimes A)$.
\end{proposition}
% section 8 (end)
\newpage

\section{Die 6-Term Sequenz} % (fold)
\label{sec:9}

\begin{satz}[label=satz:91]
	Sei \(
		\begin{tikzcd}[cramped,sep=small]
			0 \rar & J \rar["\iota"] & A \rar["\pi"] & B \rar & 0
		\end{tikzcd}
	\) eine kurze exakte Sequenz von $C^*$-Algebren.
	Dann existiert ein Homomorphismus $\delta_1$, sodass
	\begin{equation}
		\begin{tikzcd}
			K_0(J) \rar["K_0(\iota)"] & K_0(A) \rar["K_0(\pi)"] & K_0(B)  \\
			K_1(B) \uar["\delta_1"] & K_1(A) \lar["K_0(\pi)"'] & K_1(J) \lar["K_1(\iota)"']
		\end{tikzcd} \label{eq:91}\tag{\#}
	\end{equation}
	exakt ist.
	Für $u \in \mathcal{U}_n(B^+)$ ist $\delta_1 \enbrace*{\benbrace*{u}_1} = \benbrace*{p}_0 - \benbrace*{s(p)}_0$ für $p \in \kernedP_{2n}(J^+)$ mit $\iota^{+\ssbbrace{2n}}(p) = v \begin{psmallmatrix}
		\ind_n & \\
		& 0
	\end{psmallmatrix} v^*$ und $v \in \mathcal{U}_{2n}(A^+)$ mit $\pi^{+\ssbrace{2n}}(v) = \begin{psmallmatrix}
		u & \\
		& u^*
	\end{psmallmatrix}$.
	$\delta_1$ ist \Index{natürlich}, das heißt für 
	\[
		\begin{tikzcd}
			0 \rar & J \rar \dar["\gamma"] & A \rar\dar["\alpha"] & B \rar \dar["\beta"] & 0 \\
			0 \rar & J' \rar & A' \rar & B' \rar & 0
		\end{tikzcd}
	\]
	kommutiert auch
	\[
		\begin{tikzcd}
			K_1(B) \dar["K_1(\beta)"] \rar["\delta_1"] & K_0(J)  \dar["K_0(\gamma)"] \\
			K_1(B') \rar["\delta_1"] & K_0(J')
		\end{tikzcd}
	\]
\end{satz}
\begin{beweis}[Idee]
	Es ist $\begin{psmallmatrix}u & \\ & u^* \end{psmallmatrix} \sim_h \ind_{2n}$ in $U_{2n}(B^+)$.
	Wie letztes Mal existiert ein $v \in \mathcal{U}_{2n}(A^+)$ mit $\pi^+(v)= \begin{psmallmatrix} u & \\ & u^* \end{psmallmatrix}$.
	Es gilt 
	\[
		\pi^+ \enbrace*{v \begin{pmatrix}
			\ind_{n} & \\ & 0
		\end{pmatrix} v^*} = \begin{pmatrix}
			\ind_n 6 \\ & 0
		\end{pmatrix}
	\]
	also erhält man $v \begin{psmallmatrix}
			\ind_{n} & \\ & 0
		\end{psmallmatrix} v^* - \begin{psmallmatrix}
			\ind_{n} & \\ & 0
		\end{psmallmatrix} \in \ker (\pi^+) = M_{2n}(J)$.
	Damit folgt
	\(
		p = v \begin{psmallmatrix}
			\ind_{n} & \\ & 0
		\end{psmallmatrix} v^* \in \kernedP_{2n}(J^+)
	\).
	Setze dann 
	\[
		\delta_1 \enbrace*{[u]_1} = \benbrace*{p}_0 - \benbrace*{\begin{pmatrix}
			1 & \\ & 0
		\end{pmatrix}}_0 \in K_0(J)
	\]
	Zur Wohldefiniertheit von $\delta_1$: 
	\begin{description}
		\item[Unabhängigkeit von $n$:] Falls wir $u$ durch $u \oplus \ind$ ersetzen und $v= \begin{psmallmatrix} v_1 & v_2 \\ v_3 & v_4 \end{psmallmatrix} \in \mathcal{U}_{2n}(A^+)$ schreiben, so gilt für 
		\(
			v' = \begin{psmallmatrix}
				v_1 \oplus \ind & v_2 \oplus 0\\
				v_3 \oplus 0 & v_4 \oplus \ind
			\end{psmallmatrix} \in \mathcal{U}_{2(n+1)}(A)
		\)
		\[
			\pi^+(v') = \begin{pmatrix}
				u \oplus \ind & \\
				& u^* \oplus \ind
			\end{pmatrix} \quad \text{ und } \quad 
			p' = v' \begin{pmatrix}
				\ind_{n+1} & \\ & 0
			\end{pmatrix} v'^* = p \oplus \ind
		\]
		Dann gilt $\benbrace*{p'}_0 - \benbrace*{s(p')}_0 = \benbrace*{p}_0 - \benbrace*{s(p)}_0$.
		\item[Unabhängigkeit von $v$:] Ist $v' \in \mathcal{U}_{2n}(A^+)$ ein anderen Lift von $\begin{psmallmatrix} u & // & u^* \end{psmallmatrix} \in \mathcal{U}_{2n}(B^+)$, so gilt $\pi^+(v'v^*) = \ind_{2n}$, also $v'v^* = \iota^+(w)$ für ein $w \in \mathcal{U}_{2n}(J^+)$.
		Also ist
		\[
			\iota^+(p') = v' \begin{pmatrix}
				\ind_n & \\ & 0
			\end{pmatrix} v'^* = \iota^+(w) v \begin{pmatrix}
				\ind_n & \\ & 0
			\end{pmatrix} v^* \iota^+(w) = \iota^+(w p w^*)
		\]
		Mit der Injektivität von $\iota$ folgt $p \sim_u p'$, also $\benbrace*{p}_0 - \benbrace*{s(p)}_0 = \benbrace*{p'}_0 - \benbrace*{s(p')}_0$.
		\item[label] Ist $u \sim_h u_1$ in $\mathcal{U}_n(B^+)$, so ist $u^* u_1 \sim_h \ind \sim u u_1^*$.
		Folglich existieren $w_1, w_2 \in \mathcal{U}_n(A^+)$ mit $\pi^+(w_1) = u^* u_1$ und $\pi^+(w_2)= uu_1^*$.
		Dann erfüllt $v_1 \coloneqq v \begin{psmallmatrix} w_1 & \\ & w_2 \end{psmallmatrix} \in \mathcal{U}_{2n}(A^+)$ 
		\[
			\pi^+(v_1) = \begin{pmatrix}
				u_1 & \\ & u_1^*
			\end{pmatrix}
		\]
		und $v_1 \begin{psmallmatrix}
			\ind_n & \\ & 0
		\end{psmallmatrix} v_1^* = v \begin{psmallmatrix}
			\ind_n & \\ & 0
		\end{psmallmatrix} v^*$.
	\end{description}
	$\delta_1$ ist ein Homomorphismus: Seien $u_1,u_2 \in \mathcal{U}_n(B^+)$.
	Wähle $v_1, v_2 \in \mathcal{U}_{2n}(A^+)$ mit $\pi^+(v_i) = \begin{psmallmatrix} u_i & \\ & u_i^* \end{psmallmatrix}$ für $i=1,2$.
	Falls $v_i = \begin{psmallmatrix}
		v_1^\ssbrace{i} & v_2^\ssbrace{i} \\ v_3^\ssbrace{i} & v_4^\ssbrace{i}
	\end{psmallmatrix}$, so folgt für
	\[
		v \coloneqq \begin{pmatrix}
			v_1^\ssbrace{0} \oplus v_1^\ssbrace{1} & v_2^\ssbrace{0} \oplus v_2^\ssbrace{1} \\
			v_3^\ssbrace{0} \oplus v_3^\ssbrace{1} & v_4^\ssbrace{0} \oplus v_4^\ssbrace{1}
		\end{pmatrix} \in \mathcal{U}_{2(2n)}(A^+)
	\]
	Dann gilt $\pi^+ = \begin{psmallmatrix}
		u_1 \oplus u_2 & \\ & u_1^* \oplus u_2^*
	\end{psmallmatrix}$.
	Es gilt $\ind_{2n} = \ind_n \oplus \ind_n$, also
	\[
		p = v \begin{pmatrix}
			\ind_{2n} & \\ & 0
		\end{pmatrix} v^* \sim_{\MvN} \Underbracket{v_1 \begin{pmatrix}
			\ind_n & \\ & 0
		\end{pmatrix} v_1^*}{=p_1} \oplus \Underbracket{v_2 \begin{pmatrix}
			\ind_n & \\ & 0
		\end{pmatrix}v_2^*}{= p_2}
	\]
	Also ist 
	\[
		\delta_1 \enbrace*{\benbrace*{u_1}_1 + \benbrace*{u_2}_1} = \delta_1 \enbrace*{\benbrace*{u_1 \oplus u_2}_1} = \benbrace*{p}_0 - \benbrace*{s(p)}_0 = \benbrace*{p_1}_0 - \benbrace*{s(p_1)}_0 + \benbrace*{p_2}_0 - \benbrace*{s(p_2)}_0 = \delta_1 \enbrace*{\benbrace*{u_1}_1} + \delta_1 \enbrace*{\benbrace*{u_2}_1}
	\]
	Natürlichkeit von $\delta_1$:
	Für $v \in \mathcal{U}_{2n}(A^+)$ Lift für $\begin{psmallmatrix} u & \\ & u^* \end{psmallmatrix}$, $u \in \mathcal{U}_n(B^+)$ wie oben, dann ist $\alpha(v)$ Lift für $\begin{psmallmatrix} \beta(u) & \\ & \beta(u^*) \end{psmallmatrix}$.
	$x= \benbrace*{u}_1 \in K_1(B)$
	\begin{align}
		\delta_1 \circ K_1(\beta)(x) = \delta_1 \enbrace*{\benbrace*{\beta(u)}_1} = \benbrace*{\alpha(v) \begin{pmatrix}
			\ind_n & \\ & 0
		\end{pmatrix} \alpha(v)^*}_0 - \benbrace*{s(\cdots)}_0 &= K_0(\alpha) \enbrace*{\benbrace*{v \begin{pmatrix}
			\ind_n & \\ & 0
		\end{pmatrix} v^*}_0 - \benbrace*{s(\cdots)}_0} \\
		&= K_0(\alpha) \circ \delta_1(x).
	\end{align}
	Überprüfe nun noch, dass \eqref{eq:91} exakt ist.
\end{beweis}

Wir schreiben $SA$ für die \Index{Einhängung}
\[
	SA = C_0(0,1) \otimes A
\]
und $S^n A = S \enbrace*{S^{n-1}A}$ für $n \ge 2$.
Wir wissen aus \cref{sec:2}, dass $S(-)$ ein exakter Funktor ist ($C_0(0,1)$ ist nuklear nach \autoref{satz:121} und $C_0(0,1) \otimes -$ ist exakt nach \autoref{prop:24}). 

\begin{satz}[label=satz:92]
	Sei $A$ eine $C^*$-Algebra.
	Dann gibt es einen natürlichen Isomorphismus $\vartheta_A \colon K_1(A) \to K_0(SA)$.
	Für $u \in \mathcal{U}_n(A^+)$ mit $s_A(u)= \ind_n$ (vergleiche \autoref{bem:84} (i)) ist $\vartheta_A \enbrace*{\benbrace*{u}_1} = \benbrace*{p}_0 - \benbrace*{s_A(p)}_0$, wobei $p \in \kernedP_{2n} \enbrace*{(SA)^+}$ mit $p = v\begin{psmallmatrix} \ind_n & \\ & 0 \end{psmallmatrix} v^*$ und $v \in \mathcal{U}_{2n} \enbrace*{C[0,1], A^+}$, $v(0)= \ind_{2n}$, $v(1) = \begin{psmallmatrix} u & \\ & u^* \end{psmallmatrix}$ und $s_A(v(t)) = \ind_{2n}$ für $t \in [0,1]$.
\end{satz}
\begin{beweis}
	Betrachte die kurze exakte Sequenz mit dem Kegel
	\[
		\begin{tikzcd}
			0 \rar & SA \rar & CA \rar["\ev_1"] & A \rar & 0
		\end{tikzcd}
	\]
	Mit \autoref{satz:91} erhält man nun
	\[
		\begin{tikzcd}
			K_0(SA) \rar & K_0(CA) \rar & K_0(A) \\
			K_1(A) \uar["\delta_1","\cong"'] & K_1(CA) \lar & K_1(SA) \lar
		\end{tikzcd}
	\]
	Setze $\vartheta_A = \delta_1$. Rest: Übung!
\end{beweis}

\begin{definitionP}[label=def:93]
	Für $n \ge 2$ definieren wir induktiv $K_n(_) \coloneqq K_{n-1}(S_)$.
	Dann ist für jedes $n \in \mathbb{N}$ $K_n$ ein halbexakter Funktor
	\[
		\enbrace*{\CSTAR, {^*}\text{-}\Hom} \longrightarrow \enbrace*{\AB, \Hom}
	\]
\end{definitionP}
\begin{beweis}
	$S-$ ist ein exakter Funktor und somit ist $K_n(-)$ halbexakt, falls $K_{n-1}(-)$ halbexakt.
	Dies stimmt für $n=0$.
\end{beweis}

Für \(
	\begin{tikzcd}[cramped,sep=small]
		0 \rar & J \rar["\iota"] & A \rar["\pi"] & B \rar & 0
	\end{tikzcd}
\) betrachte 
\[
	\begin{tikzcd}
		0 \rar & S^n J \rar["S^n \iota"] & S^n A \rar["S^n \pi"] & S^n B \rar & 0
	\end{tikzcd}
\]
mit Indexabbildung $\delta_1 \colon K_1(S^n B) \to K_0(S^n J)$.
Mit \autoref{satz:92} folgt die Existenz eines Isomorphismus $\vartheta_{S^{n-1}J} \colon K_n(J) = K_1(S^{n-1}J) \to K_0(S^n J)$.
Dies liefert
\[
	\begin{tikzcd}
		K_{n+1}(B) \rar[dashed,"\exists! \delta_{n+1}"] & K_n(J) \dar["\cong"',"\vartheta_{S^{n-1}J}"] \\
		K_1(S^n B) \uar[equal] \rar["\delta_1"] & K_0(S^n J)
	\end{tikzcd}
\]
Die Indexabbildungen $\delta_n$, $n \ge 1$ sind natürlich, denn die Abbildungen aus \autoref{satz:91} und \autoref{satz:92} sind natürlich.
Wier erhalten mit \autoref{satz:91} und \autoref{def:93}:

\begin{proposition}
	Eine kurze exakte Sequenz \(
		\begin{tikzcd}[cramped,sep=small]
			0 \rar & J \rar & A \rar & B \rar & 0
		\end{tikzcd}
	\)
	von $C^*$-Algebren induziert eine lange exakte Sequenz von $K$-Gruppen
	\[
		\begin{tikzcd}
			\ldots \rar & K_{n+1}(B) \rar["\delta_{n+1}"] & K_n(J) \rar & \ldots \rar & K_1(B) \rar["\delta_1"] & K_0(J) \rar & K_0(A) \rar & K_0(B)
		\end{tikzcd}
	\]
\end{proposition}

Sei $\mathcal{T}$ die Toeplitz-Algebra\footnote{oder von Isometrie, also $S^*S=\ind$}, mit $\mathcal{T} = C^*(S) \subset \mathcal{B}(\ell^2(\mathbb{N}))$ und 
\[
	\begin{tikzcd}
		0 \rar & \mathcal{K} \rar & \mathcal{T} \rar \drar["\tau"] & C(S^1) \dar["\ev_1"] \rar & 0 \\
		& & & \mathbb{C}
	\end{tikzcd}\marginnote{mit $\tau(S)=1$ und $\tau$ Spurzustand}
\]
Damit erhalten wir 
\[
	\begin{tikzcd}
		0 \rar & \mathcal{T}_0 \rar & \mathcal{T} \rar["\tau"] & \mathbb{C} \lar[bend right,"\sigma"'] \rar & 0
	\end{tikzcd}
\]

\begin{satz}[name={Cuntz},label=satz:95]
	Für jede unitale $C^*$-Algebra $D$ ist 
	\[
		K_0(\tau \otimes {\id}) \colon  K_0 \enbrace*{\mathcal{T} \otimes D} \longrightarrow K_0(\mathbb{C} \otimes D) = K_0(D)
	\]
	ein Isomorphismus. Insbesondere gilt $K_0(\mathcal{T})=K_0(\mathbb{C}) = \mathbb{Z}$.
\end{satz}
\begin{beweis}
	Wir betrachten den Fall $D=\mathbb{C}$, das heißt es ist zu zeigen, dass $K_0(\tau)$ ein Isomorphismus ist.
	Für beliebiges $D$ tensoriere \enquote{den Beweis} mit $D$ bzw. $\id_D$.
	
	Es gilt $K_0(\tau) K_0(\sigma) = K_0({\id_\mathbb{C}}) = {\id_{K_0(\mathbb{C})}}$.
	Wir müssen also $K_0(\sigma) K_0(\tau) = {\id_{K_0(\mathcal{T})}}$ zeigen.
	Sei $\varepsilon \colon \mathcal{T} \to \mathcal{K} \otimes \mathcal{T}$ gegeben durch $\varepsilon = e_{00} \otimes {\id_{\mathcal{T}}}$.
	Dann ist $K_0(\varepsilon)$ ein Isomorphismus.
	Setze $B \coloneqq \mathcal{T} \otimes \ind_{\mathcal{T}} + \mathcal{K} \otimes \mathcal{T}$.
	Dann ist $B \subset \mathcal{T} \otimes \mathcal{T}$ eine $C^*$-Unteralgebra und $\mathcal{K} \otimes \mathcal{T} \lhd B$ ist ein Ideal.
	Sei nun $\pi \colon B \to \sfrac{B}{\mathcal{K} \otimes \mathcal{T}}$ die Quotientenabbildungen und $\Theta \colon \mathcal{T} \to B$ gegeben durch $\Theta(a) \coloneqq a \otimes \ind_{\mathcal{T}}$ die Inklusion.
	Betrachte den \emph{Pullback}
	\[
		\begin{tikzcd}
			C \dar \rar & \mathcal{T} \dar["\pi \Theta"] \\
			B \rar["\pi"] & \sfrac{B}{\mathcal{K} \otimes \mathcal{T}}
		\end{tikzcd}
	\]
	mit $C= \set[\big]{(a,b) \in \mathcal{T} \oplus B \given \pi \Theta(a)= \pi(b)}$ sowie die $^*$-Homomorphismen 
	\[
		\begin{array}{rcl}
			\textstyle \gamma \colon \mathcal{K} \otimes \mathcal{T} &\xhookrightarrow{\minwidthbox{}{2em}} & \textstyle C \\[0.5ex]
			\textstyle b &\xmapsto{\minwidthbox{\mbox{ }}{2em}} & \textstyle (0,b)
		\end{array}
	\]
	$\rho \colon C \twoheadrightarrow \mathcal{T}$, $(a,b) \mapsto a$ und $\kappa \colon \mathcal{T} \hookrightarrow C$, $a \mapsto (a, a \otimes \ind_\mathcal{T})$.
	Damit haben wir die folgende kurze split-exakte Sequenz
	\[
		\begin{tikzcd}
			0 \rar & \mathcal{K} \otimes \mathcal{T} \rar["\gamma"] & C \rar["\rho"] & \mathcal{T} \lar[bend right,"\kappa"'] \rar & 0
		\end{tikzcd}
	\]
	Damit ist $K_0(\gamma)$ injektiv.
	Setze $\lambda \coloneqq \gamma \varepsilon \colon \mathcal{T} \to C$.
	Dann ist $K_0(\lambda) = K_0(\gamma) K_0(\varepsilon)$ injektiv.
	Es genügt also $K_0(\lambda) K_0(\sigma) K_0(\tau) = K_0(\lambda)$ zu zeigen.
	Setze 
	\begin{align}
		z_0 &\coloneqq S^2 \enbrace{S^*}^2 \otimes \ind + e_{00}S^* \otimes S + Se_{00} \otimes S^* + e_{00} \otimes e_{00} \\
		z_1 &\coloneqq S^2 \enbrace*{S^*}^2 \otimes \ind + e_{00} S^* \otimes \ind + S e_{00} \otimes \ind
	\end{align}
	\missingfigure{riiiiieeeeesige Matrizen}
	Dann sind $z_0,z_1$ Symmetrien in $B$ und für $t \in (0,1)$ definiert
	\[
		z_t \coloneqq -i \cdot e^{\sfrac{1}{2} i \pi (1-t) z_0} \cdot e^{ \sfrac{1}{2} i \pi t z_1}
	\]
	eine Homotopie $z \colon [0,1] \to \mathcal{U}(B)$.
	Es gilt $\pi(z_0) = \pi(z_1) = \ind_{\sfrac{B}{\mathcal{K} \otimes \mathcal{T}}}$, also auch $\pi(z_t) = \ind_{\sfrac{B}{\mathcal{K} \otimes \mathcal{T}}}$ für $t \in (0,1)$.
	$\mathcal{T}$ ist (isomorph zur) universelle $C^*$-Algebra, die von einer Isometrie erzeugt wird.
	Damit dürfen wir $\varphi_t \colon \mathcal{T} \to C$ für $t \in [0,1]$ definieren durch 
	\[
		\varphi_t(S) \coloneqq \enbrace*{S,z_t \enbrace*{S \otimes \ind_\mathcal{T}}} \in C \marginnote{denn $\pi(z_t)=\ind$}
	\]
	Ebenso existiert ein $^*$-Homomorphismus 
	\[
		\mu \colon \mathcal{T} \longrightarrow \set[\Big]{(a,b) \in \mathcal{T} \oplus \Theta \enbrace[\big]{(\ind -e_{00}) \mathcal{T}(\ind - e_{00})} \given \pi \Theta(a)= \pi(b)}  \subset C 
	\]
	mit $\mu(S) = \enbrace*{S, S^2 S^* \otimes \ind} \in C$. 
	$\lambda$ und $\mu$ haben orthogonale Bilder in $C$, $\lambda(x) \mu(y)=0$ für $x,y \in \mathcal{T}$.
	Damit haben auch $\lambda \circ \sigma \circ \tau$ und $\mu$ orthogonale Bilder.
	Weiter gilt $\varphi_0 = \mu + \lambda$ sowie $\varphi_1 = \mu + \lambda \circ \sigma \circ \tau$ (warum?).
	Damit erhalten wir
	\[
		K_0(\mu) + K_0(\lambda) = K_0(\mu + \lambda) = K_0(\varphi_0) = K_0(\varphi_1) = K_0(\mu + \lambda \circ \sigma \circ \tau) = K_0(\mu) + K_0(\lambda \circ \sigma \circ \tau)
	\]
	Kürzen ergibt $K_0(\lambda) = K_0(\lambda) \circ K_0(\sigma) \circ K_0(\tau)$.
\end{beweis}

\begin{korollar}
	Für jede $C^*$-Algebra $D$ gilt $K_0(\mathcal{T}_0 \otimes D) = K_1(\mathcal{T}_0 \otimes D) = 0$.
\end{korollar}
\begin{beweis}
	Zunächst für $D$ unital.
	Die folgende Sequenz zerfällt
	\[
		\begin{tikzcd}
			0 \rar & \mathcal{T}_0 \otimes D \rar & \mathcal{T} \otimes D \rar["\tau \otimes {\id}"] & \mathbb{C} \otimes D \rar \lar[bend right,"\sigma \otimes {\id_D}"'] & 0
		\end{tikzcd}
	\]
	und somit zerfällt sich auch in $K_0$
	\[
		\begin{tikzcd}
			0 \rar & K_0(\mathcal{T}_0 \otimes D) \rar & K_0(\mathcal{T} \otimes D) \rar["K_0(\tau \otimes {\id_D})"] & K_0(\mathbb{C} \otimes D) \lar[bend right] \rar & 0
		\end{tikzcd}
	\]
	Mit \autoref{satz:95} folgt $K_0(\mathcal{T}_0 \otimes D)=0$.
	Ist $D$ nicht unital, so betrachte
	\[
		\begin{tikzcd}[row sep=small]
			0 \rar & K_0(\mathcal{T}_0 \otimes D) \rar & K_0 \enbrace*{\mathcal{T}_0 \otimes D^+} \rar & K_0(\mathcal{T}_0 \otimes \mathbb{C}) \rar & 0 \\[-1em]
			& 0 \uar[equal] & \uar[equal]0  & \uar[equal] 0
		\end{tikzcd}
	\]
	Für $K_1$ stellen wir fest, dass $K_1(\mathcal{T}_0 \otimes D) \cong K_0 \enbrace*{\mathcal{T}_0 \otimes D \otimes C_0(0,1)} = 0$, wobei der Isomorphismus nach \autoref{satz:92} existiert.
\end{beweis}
% section 9 (end)


\cleardoubleoddemptypage
\pagenumbering{Alph}
\setcounter{page}{1}
\cleardoubleoddemptypage
\appendix

\section{Anhang} % (fold)
\label{sec:anhang}
%!TEX root = ana_top_geo.tex

\subsection{Ausführlicher Beweis zu \cref{lem:kpt-schnitte}} % (fold)
\label{sub:kpt-schnitte}
Sei $X$ ein Hausdorffraum. Dann ist $X$ genau dann kompakt, wenn gilt: Hat eine Familie $\mathcal{A}$ von abgeschlossenen Teilmengen von $X$ die endliche 
Durchschnittseigenschaft, so gilt 
\[
	\bigcap_{A \in \mathcal{A}} A \not= \emptyset.
\]
\begin{beweis}
	Für die erste Implikation sei $X$ kompakt und $\mathcal{A}$ eine Familie von abgeschlossenen Mengen mit der endlichen Durchschnittseigenschaft.
	Angenommen $\bigcap_{A \in \mathcal{A}} A = \emptyset$.
	Dann gilt
	\[
		X = X \setminus \bigcap_{A \in \mathcal{A}} A = \bigcup_{A \in \mathcal{A}} X \setminus A.
	\]
	Nun ist $\mathcal{U} \coloneqq \set*{X \setminus A \given A \in \mathcal{A}}$ eine offene Überdeckung von $X$ und da $X$ kompakt ist, existiert $\mathcal{A}_0 \subset \mathcal{A}$ endlich, sodass
	\[
		X = \bigcup_{A \in \mathcal{A}_0} X \setminus A = X \setminus \underbrace{\bigcap_{A \in \mathcal{A}_0 } A }_{\neq \emptyset} \quad \light
	\]
	Für die umgekehrte Implikation sei nun $\mathcal{U} = \set{U_i}_{i \in I}$ eine offene Überdeckung von $X$.
	Angenommen für jede endliche Teilmenge $J \subseteq I$ gilt $X \neq \bigcup_{i \in J} U_i$.
	Betrachte nun $\mathcal{A} =  \set{X \setminus U_i}_{i \in I}$. Dann gilt nach Annahme
	\[
		\bigcap_{i \in J} X \setminus U_i = X \setminus \bigcup_{i \in J} U_i \neq \emptyset.
	\]
	Also hat $\mathcal{A}$ die endliche Durchschnittseigenschaft. Nach Vorraussetzung gilt dann
	\[
		\emptyset \not= \bigcap_{i \in I} X \setminus U_i = X \setminus \underbrace{\bigcup_{i \in I} U_i}_{= X} \quad \light \qedhere
	\]
\end{beweis}


\subsection[Blatt3, Aufgabe 4: Hilfssatz für den Hauptsatz der Algebra]{Blatt 3, Aufgabe 4} % (fold)
\label{sub:B3A4}
\emph{Diese Übungsaufgabe ist zentral für den Beweis des Hauptsatzes der Algebra, \cref{satz:hauptsatz-algebra}.} 

Sei $p(x)= x^n + a_{n-1} x^{n-1} + \ldots + a_1 x + a_0$ mit $n \in \mathbb{N}_0$ ein Polynom mit Koeffizienten $a_i \in \mathbb{C}$, dass \emph{keine} Nullstelle in $\mathbb{C}$ besitzt. 
Sei $S^1= \set*{z \in \mathbb{C} \given \abs*{z}=1}$.
\begin{enumerate}[(a)]
	\item $f \colon S^1 \to S^1$ gegeben durch $f(z) = \frac{p(z)}{\abs*{p(z)} } $ ist wohldefiniert und homotop zu einer konstanten Abbildung.
	\item $f$ ist homotop zur Abbildung $g_n \colon S^1 \to S^1$ mit $g_n(z)= z^n$.
\end{enumerate}
\minisec{Beweis}
\begin{enumerate}[(a)]
	\item \begin{description}
		\item[Wohldefiniertheit:] Sei $z \in S^1$ beliebig. Dann gilt
		\[
			\abs*{\frac{p(z)}{\abs*{p(z)} } } = \frac{1}{\abs*{p(z)} } \cdot \abs*{p(z)} =1,
		\]
		also ist $f(z) \in S^1$.
		\item[Homotop zu einer konstanten Abbildung:] Definiere $f_t \colon S^1 \to S^1$ für $t \in [0,1]$ durch 
		\[
			f_t(z) = \frac{p(t \cdot z)}{\abs*{p(t \cdot z)} } 
		\]
		Dies ist mit der gleichen Begründung wie oben wohldefiniert. 
		Außerdem ist $f_0(z)= \frac{a_0}{\abs*{a_0} } \in S^1 $ konstant und $f_1(z)= \frac{p(z)}{\abs*{p(z)} }=f(z)$. 
		Definiere nun $H \colon S^1 \times [0,1] \to S^1$ durch $H(x,t) \coloneqq f_t(x)$. 
		Dann ist $H$ stetig, da Polynome und $\abs*{.} $, sowie Multiplikation stetig sind. 
		$H$ ist die gesuchte Homotopie.
	\end{description}
	\item Sei $h \colon S^1 \times [0,1] \to \mathbb{C}$ gegeben durch $h(z,t) = z^n + \sum_{k=0}^{n-1} a_k z^k t^{n-k}$. 
	Dann gilt $h(z,0)=z^n \not= 0$, da $z \in S^1$.
	Für $t \neq 0$ gilt nun
	\begin{align*}
		h(z,t) = 0 \iff \frac{h(z,t)}{t^n} = 0 \iff \frac{z^n}{t^n} + \sum_{k=0}^{n-1} a_k \frac{z^k}{t^k} = 0 \iff p \enbrace*{\frac{z}{t}} = 0
	\end{align*}
	Aber nach Vorraussetzung gilt $p \enbrace*{\frac{z}{t}} \neq 0$. 
	Also $h(z,t) \neq 0$ für alle $t \in [0,1]$. 
	Definiere nun $H \colon S^1 \times [0,1]\to S^1$ durch $H(z,t) = \frac{h(z,t)}{\abs*{h(z,t)}}$. 
	Wie eben gezeigt, ist dies wohldefiniert und offensichtlich stetig. Da
	\[
		H(z,0) = \frac{z^n}{\abs*{z^n} } = z^n \quad \text{ und } \quad H(z,1) = \frac{h(z,1)}{\abs*{h(z,1)} } = \frac{p(z)}{\abs*{p(z)} } =f(z)
	\]
	ist $H$ die gesuchte Homotopie. \qedhere
\end{enumerate}

\subsection{Blatt 10, Aufgabe 3} % (fold)
\label{sub:B10A3}
\emph{Diese Übungsaufgabe lieferte den Beweis zu \cref{prop:iso-covering}.} \smallskip \\
Sei $p \colon \overline{X} \to X$ eine Überlagerung. 
Seien $\overline{x}_0  \in \overline{X}$ und $x_0= p(\overline{x}_0 )$ Basispunkte. 
Dann ist die induzierte Abbildung $\pi_n (p) \colon \pi_n(\overline{X}, \overline{x}_0) \to \pi_n(X,x_0)$ ein Isomorphismus für alle $n \ge 2$.
\minisec{Beweis}
Als Überlagerung ist $p$ stetig, also ist $\pi_n(p)$ ein Gruppenhomomorphismus nach \hyperref[prop:eig-hom-gruppen:enum:4]{ \cref*{prop:eig-hom-gruppen} \ref*{prop:eig-hom-gruppen:enum:4}}.
\begin{description}
	\item[Surjektivität:] Sei $[\omega] \in \pi_n(X,x_0)$, also $\omega \colon I^n \to X$ mit $\omega(\partial I^n) = \set{x_0}$. Betrachte $\omega$ nun als Abbildung $I^{n-1} \times [0,1] \to X$:
	\[
		\begin{tikzcd}[column sep=4em]
			I^{n-1} \times \set{0} \dar[hook] \rar["\mathrm{const}_{\overline{x}_0}"] & \overline{X} \dar["p"]\\
			I^{n-1} \times I \rar["\omega"] & X  
		\end{tikzcd}
	\]
	$\mathrm{const}_{\overline{x}_0} \colon I^{n-1} \times \set{0}$ ist eine Hebung von $\omega\big|_{I^{n-1} \times \set{0}} \equiv x_0$. 
	Nach dem Homotopiehebungssatz (\ref{satz:hebung-homotopie}) existiert eine Hebung $\overline{\omega} \colon I^{n-1} \times I \to \overline{X}$ von $\omega$ mit $\overline{\omega}\big|_{I^{n-1} \times \set{0}} \equiv \overline{x}_0 $. 
	Also gilt
	\[
		p \circ \overline{\omega} \big|_{\partial I^n} = \omega \big|_{\partial I^n} \equiv x_0 \enspace \Longrightarrow \enspace \overline{\omega} \big|_{\partial I^n} 
		\in p ^{-1}( \set{x_0} ) .
	\]
	Da $p^{-1}(\set{x_0})$ diskret und $\partial I^n$ für $n \ge 2$ zusammenhängend ist, muss $\overline{\omega} \big|_{\partial I^n}$ konstant sein. 
	Da $\overline{\omega}\big|_{I^{n-1} \times \set{0}} \equiv \overline{x}_0 $ gilt, folgt somit $\overline{\omega}(\partial I^n) = \set{\overline{x}_0}$. 
	Also ist $[\overline{\omega}] \in \pi_n(\overline{X},\overline{x}_0)$ und weiter gilt
	\[
		\pi_n(p) \enbrace*{[\overline{\omega}]} = [p \circ \overline{\omega} ] = [\omega] \in \pi_n(X,x_0). 
	\]
	\item[Injektivität:] Sei $[\omega] \in \ker \pi_n(p)$, also $[p \circ \omega] = [c_{x_0}]$. 
	Es existiert also eine Homotopie $H$ relativ $\partial I^n$ zwischen $p \circ \omega$ und $c_{x_0}$. 
	Offensichtlich ist $\omega$ eine Hebung von $p \circ \omega$. 
	Mit dem Homotopiehebungssatz erhalten wir eine Hebung $\overline{H}$ von $H$ mit $\overline{H}(-,0) = \omega$. 
	Weiter wissen wir, dass
	\[
		\overline{H} \big|_{\partial I^n \times [0,1]} \in p ^{-1}(\set{x_0} ) \quad \text{ und }\quad  \overline{H} \big|_{ I^n \times \set{1}} \in p ^{-1}(\set{x_0} )
	\]
	gelten muss, da $H = p \circ \overline{H}$ und $H(-,1)= c_{x_0} \equiv x_0$. 
	Mit dem gleichen Argument wie oben folgt, dass $\overline{H} \big|_{\partial I^n \times [0,1]}$ und $\overline{H} \big|_{ I^n \times \set{1}}$ konstant sind. 
	Für $z \in \partial I^n$ gilt nun
	\[
		\overline{H}(z,0) = \omega(z) = \overline{x}_0
	\]
	Da $\partial I^n \times [0,1] \cap I^n \times \set{1} \not= \emptyset$, muss also auch $\overline{H}(-,1) \equiv \overline{x}_0$ gelten. 
	Damit folgt $[\omega] = [c_{x_0}]$.\qedhere
\end{description}
\printindex
\printbibliography
\listoffigures
\todototoc
\listoftodos[To-do's und andere Baustellen]
\end{document}
