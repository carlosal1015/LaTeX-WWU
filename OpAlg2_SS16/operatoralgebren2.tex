%!TEX TS-program = xelatex
%!TEX TS-options = -shell-escape
%!TEX root = ../OpAlg2_SS16/operatoralgebren2.tex
\RequirePackage{fix-cm} 
\documentclass[a4paper, twoside, headsepline, index=totoc,toc=listof,toc=bibliography,toc=index, fontsize=10pt, cleardoublepage=empty, headinclude, DIV=12, BCOR=5mm, titlepage,draft]{scrartcl}
%!TEX root = ../AnaTopGeo_SS14/ana_top_geo.tex
\usepackage{scrtime} % KOMA, Uhrzeit ermoeglicht

%--Pakete zum "Programmieren"
% ======================================================================================
\usepackage{etoolbox}
\usepackage{letltxmacro}
\usepackage{ifthen}
% ======================================================================================

%--Farbdefinitionen und Grafiken (muss vor tikz geladen werden)
% ======================================================================================
\usepackage[usenames, table, x11names]{xcolor}
\definecolor{dark_gray}{gray}{0.45}
\definecolor{light_gray}{gray}{0.6}
\definecolor{fb10_blue}{cmyk}{0.8,0.4,0.13,0.07}
\usepackage[final]{graphicx}
\usepackage{adjustbox}
\newcommand{\cfbox}[2]{% coloured frame box
	\ifmmode
	\mathchoice{\adjustbox{cfbox=#1}{$\displaystyle#2$}}{\adjustbox{cfbox=#1}{$\textstyle#2$}}{\adjustbox{cfbox=#1}{$\scriptstyle#2$}}{\adjustbox{cfbox=#1}{$\scriptscriptstyle#2$}}
	\else
	\adjustbox{cfbox=#1}{#2}
	\fi
}
% ======================================================================================

%--Zum Zeichnen/ TikZ-Kram (vor polyglossia bzw. babel geladen werden)
% ======================================================================================
\usepackage{tikz}
\usepackage{tikz-cd}
\usetikzlibrary{external}
\tikzset{>=latex}
\usetikzlibrary{%
	shapes,
	arrows.meta,
	intersections,
	calc,
	3d,
	decorations.pathreplacing,decorations.markings,decorations.pathmorphing,
	angles,
	quotes,
}
\tikzexternalize[prefix=tikz/,up to date check=diff]
\pgfkeys{/pgf/images/include external/.code=\includegraphics{#1}}
\tikzset{external/system call={lualatex \tikzexternalcheckshellescape -halt-on-error -interaction=batchmode --shell-escape -jobname "\image" "\texsource"}}
\AtBeginEnvironment{tikzcd}{\tikzexternaldisable} % tikzexternalize fuer tikzcd deaktivieren, da inkompatibel
\AtEndEnvironment{tikzcd}{\tikzexternalenable}
\tikzset{% um Inkompatibilitaeten von quotes und polyglossia bzw. babel zu vermeiden
  every picture/.append style={
    execute at begin picture={\shorthandoff{"}},
    execute at end picture={\shorthandon{"}}
  }
}
\usepackage{pgfplots}
\usepgfplotslibrary{colormaps}
\newcommand*\circled[1]{\tikzexternaldisable\tikz[baseline=(char.base)]{\node[shape=circle,draw,inner sep=2pt] (char) {#1};}\tikzexternalenable}
% ======================================================================================



%-- Mathepakete etc.
% ======================================================================================
\usepackage[T1]{fontenc}
\renewcommand{\rmdefault}{zpltlf}
\usepackage{mathtools} % beinhaltet amsmath
\mathtoolsset{showonlyrefs,centercolon,showmanualtags}
\newtagform{brackets}[\textbf]{[}{]}
\usetagform{brackets}
\usepackage{fix-cm}
\usepackage[bbgreekl]{mathbbol}
\usepackage{amssymb,marvosym} 
\usepackage{nicefrac} % schräge Brüche
\usepackage{faktor}
\newcommand{\Faktor}[1]{\faktor[\textstyle]{#1}}
\usepackage{xfrac}
\usepackage{cancel}
\usepackage{mathdots} % Verbesserung von Punkten wie zB \ldots
\usepackage[bb=px]{mathalfa} % \mathbb als px font
\usepackage{centernot}
\usepackage{stackrel}
\DeclareSymbolFont{bbold}{U}{bbold}{m}{n}
\DeclareSymbolFontAlphabet{\mathbbold}{bbold}
\newcommand{\ind}{\mathbbold{1}} % charakteristische-Funktion-Eins
\def\mathul#1#2{\color{#1}\underline{{\color{black}#2}}\color{black}} %farbiges Untersteichen im Mathe-Modus
\renewcommand{\le}{\leqslant}
\renewcommand{\ge}{\geqslant}
% ======================================================================================


%-- Von xfrac erzeuge font warnings ignorieren
% ======================================================================================
\usepackage{silence}
\WarningFilter{latexfont}{Size substitutions with differences}
\WarningFilter{latexfont}{Font shape `U/bbold/m/n' in size}
% ======================================================================================


%-- Typographie/Polyglossia
% ======================================================================================
\usepackage[euler-digits]{eulervm} % vor fontspec laden!
\usepackage[no-math]{fontspec}
\usepackage{polyglossia} % moderner babel-ersatz
\setmainlanguage[spelling=new,babelshorthands=true]{german}
\shorthandoff{"}
\setotherlanguage{english}
\defaultfontfeatures{Mapping=tex-text, WordSpace={1.2}, Ligatures={Required,Common,Contextual},Extension=.otf} %


\setmainfont{TeXGyrePagellaX}[UprightFont=*-Regular,BoldFont=*-Bold,ItalicFont=*-Italic,BoldItalicFont=*-BoldItalic,ItalicFeatures={Style=Historic},Ligatures={Required,Common,Contextual,Historic}]
\setsansfont{texgyreadventor}[Scale=MatchUppercase, UprightFont=*-regular, BoldFont=*-bold, ItalicFont=*-italic, BoldItalicFont=*-bolditalic]
\setmonofont{SourceCodePro}[Scale=0.9,UprightFont=*-Regular, BoldFont=*-Semibold, ItalicFont=*-Light]
\usepackage{xltxtra}
\usepackage{fontawesome}
\usepackage[final]{microtype}
\usepackage[draft=false]{scrlayer-scrpage} 
\flushbottom
% ======================================================================================


%-- Aufzählungen
% ======================================================================================
\usepackage[shortlabels,inline]{enumitem}
\setlist[itemize,1]{label=\faCaretRight}
\setlist[enumerate]{font=\bfseries}
\setlist[description]{font=\normalfont\bfseries}
\usepackage{multicol}
% ======================================================================================


%-- Floats/Figures/Tabellen
% ======================================================================================
\usepackage{wrapfig}
\usepackage{float}
\usepackage[margin=10pt, font=small, labelfont={sf, bf}, format=plain, indention=1em]{caption}
\captionsetup[wrapfigure]{name=Abb. }
\usepackage{booktabs}
% ======================================================================================


%-- korrekte Anführungszeichen und Zitierbefehle
% ======================================================================================
\usepackage[autostyle,german=quotes,english=british]{csquotes}
% ======================================================================================


%--Indexverarbeitung
% ======================================================================================
\usepackage{makeidx}
\newcommand{\bet}[1]{\textbf{\emph{#1}}}
\newcommand{\Index}[1]{\bet{#1}\index{#1}}
\makeindex
\setindexpreamble{{\noindent\sffamily\small Die \emph{Seitenzahlen} sind mit Hyperlinks versehen und somit anklickbar} \par \bigskip}
\renewcommand{\indexpagestyle}{scrheadings}
% ======================================================================================


%-- Marginnotes/Todonotes/Footnotes
% ======================================================================================
\deffootnote[1.5em]{1.5em}{1.5em}{\textsuperscript{\thefootnotemark}\ }
\usepackage[fulladjust]{marginnote}
\renewcommand*{\marginfont}{\itshape\footnotesize}
\usepackage[textsize=small]{todonotes}
\usepackage{ragged2e}
\renewcommand*{\raggedleftmarginnote}{\RaggedLeft}
\renewcommand*{\raggedrightmarginnote}{\RaggedRight}
\LetLtxMacro{\oldtodo}{\todo}
\renewcommand{\todo}[2][]{\tikzexternaldisable\oldtodo[#1]{#2}\tikzexternalenable}
\LetLtxMacro{\oldmissingfigure}{\missingfigure}
\renewcommand{\missingfigure}[2][]{\tikzexternaldisable\oldmissingfigure[{#1}]{#2}\tikzexternalenable}
% ======================================================================================


% -- BibLaTeX
% ======================================================================================
\usepackage[%
	backend=biber,
	sortlocale=auto,
	natbib,
	hyperref,
	backref,
	style=alphabetic
	]%
{biblatex}
\renewcommand*{\mkbibnamelast}[1]{%
  \ifmknamesc{\textsc{#1}}{#1}}
\renewcommand*{\mkbibnameprefix}[1]{%
  \ifboolexpr{ test {\ifmknamesc} and test {\ifuseprefix} }
    {\textsc{#1}}
    {#1}}
\def\ifmknamesc{%
  \ifboolexpr{ test {\ifcurrentname{labelname}}
               or test {\ifcurrentname{author}}
               or ( test {\ifnameundef{author}} and test {\ifcurrentname{editor}} ) }}
\addbibresource{../!config/quellen.bib}
% ======================================================================================

%--Konfiguration von Hyperref und Cleveref
% ======================================================================================
\usepackage[hidelinks, pdfpagelabels,  bookmarksopen=true, bookmarksnumbered=true, linkcolor=black, urlcolor=SkyBlue2, plainpages=false,pagebackref, citecolor=black, hypertexnames=true, pdfauthor={Jannes Bantje}, pdfborderstyle={/S/U}, linkbordercolor=SkyBlue2, colorlinks=false,final,backref=false]{hyperref}
\usepackage[nameinlink,noabbrev]{cleveref}
\newcommand{\appendLink}[1]{#1\,\faExternalLink}
\newcommand{\hrefsym}[2]{\href{#1}{\texttt{\appendLink{#2}}}}
\newcommand{\hrefsymX}[2]{\href{#1}{\appendLink{#2}}}
\newcommand{\hrefsymmail}[2]{\href{#1}{\texttt{\faEnvelopeO\,#2}}}
\renewcommand{\url}[1]{\hrefsym{#1}{\nolinkurl{#1}}}
% ======================================================================================


% -- QR-Codes (hinter hyperref laden!)
% ======================================================================================
\usepackage{qrcode}
% ======================================================================================

%--Römische Zahlen
% ======================================================================================
\newcommand{\RM}[1]{\MakeUppercase{\romannumeral #1{}}}
% ======================================================================================

%-- Definition von diversen Mathe-Befehlen
% ======================================================================================
%!TEX root = mitschrift_main.tex

% -- Zum Finetuning von Befehlen
% ======================================================================================
\makeatletter
\newcommand{\raisemath}[1]{\mathpalette{\raisem@th{#1}}}
\newcommand{\raisem@th}[3]{\raisebox{#1}{$#2#3$}}
\makeatother
\makeatletter
\newcommand{\killDescendersM}[1]{\mathpalette{\killD@scendersM{#1}}}
\newcommand{\killD@scendersM}[2]{\raisebox{0pt}[\height][0pt]{$#2#1$}}
\makeatother
\DeclareRobustCommand{\minwidthbox}[2]{%
  \ifmmode
    \expandafter\mathmakebox
  \else
    \expandafter\makebox
  \fi
  [\ifdim#2<\width\width\else#2\fi]{#1}%
}
% ======================================================================================


%-- Klammerbefehle
% ======================================================================================
\DeclarePairedDelimiter{\abs}{\lvert}{\rvert}
\DeclarePairedDelimiter{\floor}{\lfloor}{\rfloor}
\DeclarePairedDelimiter{\ceil}{\lceil}{\rceil}
\DeclarePairedDelimiter\norm{\Vert}{\Vert}
\DeclarePairedDelimiter\enbrace{(}{)}
\DeclarePairedDelimiter\benbrace{[}{]}
\DeclarePairedDelimiter\bbenbrace{[\![}{]\!]}
\DeclarePairedDelimiter\lenbrace{<}{>}
\DeclarePairedDelimiter\angbrace{\langle}{\rangle}
\newcommand{\ssbrace}[1]{{\scriptscriptstyle\enbrace{#1}}}
\newcommand{\ssbbrace}[1]{{\scriptscriptstyle\benbrace{#1}}}
% ======================================================================================

%-- Mengen
% ======================================================================================
\newcommand\SetSymbol[1][]{\nonscript\:#1\vert\allowbreak\nonscript\:\mathopen{}}
\providecommand\given{} % to make it exist
\DeclarePairedDelimiterX\set[1]\{\}{\renewcommand\given{\SetSymbol[\delimsize]}#1}
% ======================================================================================

%-- Skalarprodukt (3 Varianten) 
% ======================================================================================
\DeclarePairedDelimiterX\sprod[2]{\langle}{\rangle}{#1\,\delimsize\vert\,#2}
\DeclarePairedDelimiterX\skal[2]{\langle}{\rangle}{#1\,,\,#2}
\makeatletter
\DeclareFontFamily{OMX}{MnSymbolE}{}
\DeclareSymbolFont{MnLargeSymbols}{OMX}{MnSymbolE}{m}{n}
\SetSymbolFont{MnLargeSymbols}{bold}{OMX}{MnSymbolE}{b}{n}
\DeclareFontShape{OMX}{MnSymbolE}{m}{n}{
    <-6>  MnSymbolE5
   <6-7>  MnSymbolE6
   <7-8>  MnSymbolE7
   <8-9>  MnSymbolE8
   <9-10> MnSymbolE9
  <10-12> MnSymbolE10
  <12->   MnSymbolE12
}{}
\DeclareFontShape{OMX}{MnSymbolE}{b}{n}{
    <-6>  MnSymbolE-Bold5
   <6-7>  MnSymbolE-Bold6
   <7-8>  MnSymbolE-Bold7
   <8-9>  MnSymbolE-Bold8
   <9-10> MnSymbolE-Bold9
  <10-12> MnSymbolE-Bold10
  <12->   MnSymbolE-Bold12
}{}
\let\llangle\@undefined
\let\rrangle\@undefined
\DeclareMathDelimiter{\llangle}{\mathopen}%
                     {MnLargeSymbols}{'164}{MnLargeSymbols}{'164}
\DeclareMathDelimiter{\rrangle}{\mathclose}%
                     {MnLargeSymbols}{'171}{MnLargeSymbols}{'171}
\makeatother
\DeclarePairedDelimiterX\sskal[2]{\llangle}{\rrangle}{#1\,,\,#2}
% ======================================================================================

%-- Abbildungsdefinition
% ======================================================================================
\newcommand{\mapdef}[5]{%
	\[
		\begin{array}{rcl}
			\textstyle #1 &\xrightarrow{\minwidthbox{#5}{2em}} & \textstyle #2 \\[0.5ex]
			\textstyle #3 &\xmapsto{\minwidthbox{\mbox{ }}{2em}} & \textstyle #4
		\end{array}
	\]
}
% ======================================================================================

%-- modifiziertes Stackrel 
% ======================================================================================
\newcommand{\StackText}[2]{\stackrel{\mbox{\scriptsize #1}}{#2}}
\newcommand{\StackTextClap}[2]{\stackrel{\mathclap{\mbox{\scriptsize #1}}}{#2}}
% ======================================================================================

%-- Blitz
% ======================================================================================
\newcommand{\light}{\text{\raisebox{-.3ex}{\Large\Lightning}}}
% ======================================================================================


%-- Underbrace u.Ä. als Befehl in LaTeX-Syntax (und ohne Spacingprobleme mit nachfolgenden Operatoren...)
% ======================================================================================
\newcommand{\Underbrace}[2]{{\underbrace{#1}_{#2}}}
\newcommand{\Underbracket}[2]{{\underbracket[0.7pt][2pt]{#1}_{#2}}}
\newcommand{\Overbracket}[2]{{\overbracket[0.7pt][2pt]{#1}^{#2}}}
% ======================================================================================


%-- Deklaration weiterer Operatoren (allgemein)
% ======================================================================================
\DeclareMathOperator{\re}{Re} % Realteil
\let\Re\relax
\DeclareMathOperator{\Re}{Re} % Realteil
\DeclareMathOperator{\im}{im} % Bild
\let\Im\relax
\DeclareMathOperator{\Im}{Im} % Bild
\DeclareMathOperator{\id}{id} % identische Abbildung
\DeclareMathOperator{\conj}{conj} % Konjugation
\DeclareMathOperator{\sgn}{sgn} % Signum
\DeclareMathOperator{\End}{End} % Endomorphismen
\DeclareMathOperator{\Hom}{Hom} % Homomorphismen
\DeclareMathOperator{\Iso}{Iso} % Isomorphismen
\DeclareMathOperator{\Aut}{Aut} % Automorphismen
\DeclareMathOperator{\Span}{span} % Span
\DeclareMathOperator{\coker}{coker} % Kokern
\DeclareMathOperator{\Tr}{Tr} % Spur,Trace
\DeclareMathOperator{\pr}{pr} % Projektion
\DeclareMathOperator{\diag}{diag} % Diagonalmatrix
\DeclareMathOperator{\Rg}{Rg} % Rang
\DeclareMathOperator{\const}{const} % konstante Abbildung
\DeclareMathOperator{\Spur}{Spur} % Spur
\DeclareMathOperator{\Arg}{Arg} % Argument
\DeclareMathOperator{\dist}{dist} % Distanz
\DeclareMathOperator{\supp}{supp} % Träger
\DeclareMathOperator{\Char}{char} % Charakteristik
% ======================================================================================


%-- Deklaration weiterer Operatoren (Differentiale etc.)
% ======================================================================================
\DeclareMathOperator{\grad}{grad} % Gradient
\DeclareMathOperator{\dive}{div} % Gradient
\DeclareMathOperator{\rot}{rot} % Rotation
\newcommand{\D}{\ensuremath{\mathrm{D}\mkern-1.0mu}} % Differential
\newcommand{\mathd}{\ensuremath{\mathrm{d}\mkern-1.0mu}} % äußere Ableitung
\newcommand{\Tmap}{\ensuremath{\mathrm{T}\mkern-0.85mu}} % Tangentialraum
\let\Tang\Tmap
\DeclareMathOperator{\Diff}{Diff}
\newcommand{\diff}[2]{\ensuremath{\frac{{\partial #1}}{{\partial #2}} }}
\newcommand{\diffd}[2]{\ensuremath{\frac{\mathd #1}{\mathd #2} }}
\DeclareMathOperator{\rank}{rank}
% ======================================================================================


%-- Deklaration weiterer Operatoren (Topologie)
% ======================================================================================
\newcommand*\interior[1]{\overset{\smash{\raisebox{-0.18ex}{$\scriptstyle\circ$}}}{#1}}
\newcommand{\sing}{{\raisemath{1.1pt}{\scriptscriptstyle\mathrm{sing}}}}
\newcommand{\pt}{\mathrm{pt}}
\DeclareMathOperator{\Zyl}{Zyl}
\newcommand{\rZyl}{\widetilde{\Zyl}}
\DeclareMathOperator{\Tel}{Tel}
\newcommand{\op}{\mathrm{op}}
\DeclareMathOperator{\Sp}{Sp}
\DeclareMathOperator{\Keg}{Keg}
\newcommand{\slashedi}{i\hspace{-3.5pt}/}
\newcommand{\cupp}{\smallsmile}
\newcommand{\capp}{\smallfrown}
\DeclareMathOperator*{\colim}{colim}
\DeclareMathOperator{\PD}{PD}
\newcommand{\lf}{\mathrm{lf}}
\DeclareMathOperator{\sig}{sig}
\DeclareMathOperator{\Tor}{Tor}
\DeclareMathOperator{\Ext}{Ext}
\DeclareMathOperator{\AW}{AW}
\DeclareMathOperator{\Proj}{Proj}
\DeclareMathOperator{\Gr}{Gr}
\DeclareMathOperator{\res}{res}
\DeclareMathOperator{\Spec}{Spec}
\DeclareMathOperator{\co}{co}
\DeclareMathOperator{\ch}{ch}
\DeclareMathOperator{\wOp}{w}
\DeclareMathOperator{\Ar}{Ar}
\newcommand{\actson}{\mathrel{\curvearrowright}}
\let\acts\actson
\let\action\actson
\DeclareMathSymbol{\bbDelta}{\mathord}{bbold}{"01}
\newcommand{\DDelta}{\bbDelta}
\DeclareMathOperator{\Star}{Star}
\DeclareMathOperator{\Link}{Link}
\DeclareMathOperator{\EPK}{EPK}
\DeclareMathOperator{\Vol}{Vol}
\newcommand{\cell}{{\raisemath{1.1pt}{\scriptscriptstyle\mathrm{cell}}}}
\DeclarePairedDelimiter{\homologieklasse}{\llbracket}{\rrbracket}
\newcommand{\rand}[1]{\ensuremath{\partial^{\scriptscriptstyle #1}}}
\DeclareMathOperator{\ab}{ab}
\DeclareMathOperator{\CW}{CW}
% ======================================================================================


%-- Deklaration von Operatoren (Liegruppen)
% ======================================================================================
\DeclareMathOperator{\GL}{GL}
\DeclareMathOperator{\SO}{SO}
\DeclareMathOperator{\Ad}{Ad}
\DeclareMathOperator{\ad}{ad}
\DeclareMathOperator{\On}{O}
\DeclareMathOperator{\Un}{U}
\DeclareMathOperator{\SU}{SU}
\DeclareMathOperator{\Mat}{Mat}
\DeclareRobustCommand{\Der}{\mathop{\mathfrak{der}}}
\DeclareMathOperator{\SL}{SL}
\DeclareMathOperator{\Graph}{Graph}
\DeclareMathOperator{\Int}{Int}
\DeclareRobustCommand{\intAlg}{\mathop{\mathfrak{int}}}
\DeclareMathOperator{\aut}{aut}
\DeclareMathOperator{\Rad}{Rad}
\DeclareMathOperator{\Nil}{Nil}
\DeclareMathOperator{\rad}{rad}
\DeclareMathOperator{\nil}{nil}
\DeclareMathOperator{\Ric}{Ric}
\DeclareMathOperator{\ric}{ric}
\newcommand{\bi}{\mathrm{bi}}
\DeclareMathOperator{\Isom}{Isom}
\DeclareMathOperator{\Sym}{Sym}
\newcommand{\opL}{\ensuremath{\mathrm{L}\mkern-0.6mu}}
% ======================================================================================

%-- Deklaration von Operatoren (Funktionalanalysis)
% ======================================================================================
\DeclareMathOperator{\tr}{tr}
\newcommand{\w}{\mkern1mu\mathrm{w}}
\newcommand{\sa}{\mathrm{sa}}
\newcommand{\vb}{\mathrm{v\mkern-2.5mu.b\mkern-1.5mu.}} % vollständig beschränkt
\newcommand{\so}{\mathrm{\mkern.3mu s\mkern-1.4mu.\mkern-.6mu o\mkern-1.7mu.}} % \newcommand{\so}{\mathrm{s.o.}}
\newcommand{\solim}{\so\text{-}\mkern-0.8mu\lim}
\newcommand{\wo}{\mathrm{w\mkern-3mu.\mkern-.4mu o\mkern-1.7mu.}}
\newcommand{\Top}[1]{\mathcal{T}_{\mkern-2.3mu #1}}
\newcommand{\weakT}[1]{\ensuremath{\mathcal{T}_{#1}^{\mkern+1.0mu\text{\raisebox{0.4ex}{$\mathrm{w}$}}}}}
\newcommand{\weakTstar}[1]{\ensuremath{\mathcal{T}_{#1}^{\mkern+1.0mu\text{\raisebox{0.4ex}{$\mathrm{w}$}}^*}}}
\newcommand{\TWeakStar}{\Top{\w^*}}
\newcommand{\TWeakOp}{\Top{\wo}}
\newcommand{\Tso}{\Top{\so}}
\newcommand{\finSub}{\subset\mkern-0.7mu \subset}
\DeclareMathOperator{\Inv}{Inv}
\newcommand{\simm}{{\hspace{-1.6pt}\raisemath{0.5pt}{\sim}}}
\newcommand{\plus}{{\hspace{-1.6pt}+}}
\DeclareMathOperator{\ev}{ev}
\DeclareMathOperator{\Alg}{Alg}
\DeclareMathOperator{\her}{her}
\newcommand{\subher}{\subset_{\her}}
\newcommand{\grenzw}[1]{\xrightarrow{\minwidthbox{#1}{1.4em}}}
\newcommand{\grenzwl}[1]{\xleftarrow{\minwidthbox{#1}{1.4em}}}
\newcommand{\grenzwIn}[1]{\grenzw{\raisemath{-2pt}{#1}}}
\newcommand{\MyTo}[1]{\tikzexternaldisable\mathbin{\tikz[baseline] \draw[-to,line width=.4pt] (0ex,0.94ex) -- (#1,0.94ex);}\tikzexternalenable}
\newcommand{\dlim}{%
    \mathchoice
      {\lim\limits_{\MyTo{4.2ex}}}% \displaystyle
      {\lim\limits_{\MyTo{2.8ex}}}% \textstyle
      {\lim\limits_{\MyTo{2.3ex}}}% \scriptstyle
      {\lim\limits_{\MyTo{2.3ex}}}% \scriptscriptstyle
}
\newcommand{\Dlim}{\killDescendersM{\dlim}}
\DeclareMathOperator{\sep}{sep}
\DeclareMathOperator{\diam}{diam}
\DeclareMathOperator{\conv}{conv}
\DeclareMathOperator{\Prim}{Prim}
\DeclareMathOperator{\hull}{hull}
\DeclareMathOperator{\red}{red}
\DeclarePairedDelimiterX\bra[1]{\langle}{\rvert}{#1\,}
\DeclarePairedDelimiterX\ket[1]{\lvert}{\rangle}{\,#1}
\DeclarePairedDelimiterX\bracket[2]{\langle}{\rangle}{#1\,\delimsize\vert\,#2}
\newcommand{\tensormax}{\mathbin{\otimes_{\max}}}
\newcommand{\tensormin}{\mathbin{\otimes_{\min}}}
\DeclareMathOperator{\Ped}{Ped}
\newcommand{\alg}{\mathrm{alg}}
\DeclareMathOperator{\CPC}{CPC}
\DeclareMathOperator{\CP}{CP}
\DeclareMathOperator{\UPC}{UPC}
\newcommand{\DeltaOp}{\mathbin{\Delta}}
\newcommand{\kernedP}{\mathcal{P}\mkern-2mu}
\newcommand{\Pinfty}{\kernedP_{\infty}}
\DeclareMathOperator{\Groth}{Groth}
\DeclareMathOperator{\rk}{rk}
\newcommand{\MvN}{\mathrm{MvN}}
% ======================================================================================

%-- Kategorien
% ======================================================================================
\DeclareMathOperator{\Mor}{Mor}
\DeclareMathOperator{\mor}{mor}
\DeclareMathOperator{\Obj}{Obj}
\DeclareMathOperator{\Ob}{Ob}
\newcommand{\TOP}{\textsc{Top}}
\newcommand{\HTOP}{\textsc{HTop}}
\newcommand{\VR}{\textsc{VR}}
\newcommand{\MOD}{\textsc{Mod}}
\newcommand{\Mod}[1]{#1\text{-}\MOD}
\newcommand{\MONOIDE}{\textsc{Monoide}}
\newcommand{\SET}{\textsc{Set}}
\newcommand{\MAN}{\textsc{Man}}
\newcommand{\GRUPPEN}{\textsc{Gruppen}}
\newcommand{\ABELGRUPPEN}{\textsc{Abel.Gruppen}}
\newcommand{\ABEL}{\textsc{Abel}}
\newcommand{\KAT}{\textsc{Kat}}
\newcommand{\FUN}{\textsc{Fun}}
\newcommand{\SIMP}{\textsc{Simp}}
\newcommand{\VEKT}{\textsc{Vekt}}
\newcommand{\CH}{\textsc{Ch}}
\newcommand{\CSTARUN}{C^*\text{-}\textsc{Alg}^{\raisemath{-2.5pt}{1}}}
\newcommand{\CSTAR}{C^*\text{-}\textsc{Alg}}
\newcommand{\AB}{\textsc{Ab}}
% ======================================================================================
% ======================================================================================



% -- theorem packages
% ======================================================================================
\usepackage{amsthm}
\usepackage{thmtools,thm-restate}
\usepackage{mdframed}
\renewcommand{\listtheoremname}{Übersicht aller Aussagen}
\usepackage{bookmark}
\bookmarksetup{open,numbered}
\makeatletter
\newcommand*{\theorembookmark}{%
  \bookmark[
    dest=\@currentHref,
    rellevel=1,
    keeplevel,
  ]{%
    \thmt@thmname\space\csname the\thmt@envname\endcsname
    \ifx\thmt@shortoptarg\@empty
    \else
      \space(\thmt@shortoptarg)%
    \fi
  }%
}   
\makeatother
% ======================================================================================

% -- Definition der einzelnen Theorem-Umgebungen
% ======================================================================================
\declaretheoremstyle[%
	headfont=\sffamily\bfseries,
	notefont=\normalfont\sffamily\scshape,
	bodyfont=\normalfont,
	headformat=\NUMBER\ \NAME\NOTE,
	headpunct=.,
	postheadspace=1em,
	spaceabove=15pt,spacebelow=10pt,
	shaded={bgcolor=gray!20},
	postheadhook=\theorembookmark]%
{mainstyle}
\declaretheoremstyle[%
	headfont=\sffamily\bfseries,
	notefont=\normalfont\sffamily\scshape,
	bodyfont=\normalfont,
	headformat=\NUMBER\ \NAME\NOTE,
	headpunct=.,
	postheadspace=1em,
	spaceabove=15pt,spacebelow=10pt,
	shaded={bgcolor=fb10_blue!20},
	postheadhook=\theorembookmark]%
{mainstyle_blue}
\declaretheoremstyle[%
	headfont=\sffamily\bfseries,
	notefont=\normalfont\sffamily\scshape,
	bodyfont=\normalfont,
	headformat=\NUMBER\ \NAME\NOTE,
	headpunct=.,
	postheadspace=1em,
	spaceabove=15pt,spacebelow=10pt,
	postheadhook=\theorembookmark]%
{mainstyle_unshaded}
\declaretheoremstyle[%
	headfont=\sffamily\bfseries,
	notefont=\normalfont\sffamily\scshape,
	bodyfont=\normalfont,
	headformat=\NUMBER\NAME\NOTE,
	headpunct=.,
	postheadspace=1em,
	spaceabove=15pt,spacebelow=10pt,
	% shaded={bgcolor=gray!20},
	postheadhook=\theorembookmark]%
{mainstyle_unnumbered}
\declaretheoremstyle[%
	headfont=\sffamily\bfseries,
	notefont=\normalfont\sffamily\scshape,
	bodyfont=\normalfont,
	headformat=swapnumber,
	headpunct=.,
	postheadspace=1em,
	spaceabove=15pt,spacebelow=10pt,
	shaded={bgcolor=gray!20},
	postheadhook=\theorembookmark,
	qed=\qedsymbol]%
{mainstyleB}
\declaretheoremstyle[%
	headfont=\bfseries\scshape,
	bodyfont=\normalfont,
	headpunct=:,
	postheadspace=1em,
	spacebelow=12pt,spaceabove=2pt,
	qed=\qedsymbol]%
{beweise}
\declaretheoremstyle[%
	headfont=\bfseries\scshape,
	bodyfont=\normalfont,
	headpunct=:,
	postheadspace=1em,
	spacebelow=12pt,spaceabove=2pt]%
{beweisskizze}
\declaretheoremstyle[%
	headfont=\sffamily\bfseries,
	bodyfont=\normalfont,
	headpunct=:,
	postheadspace=1em,
	spacebelow=10pt,spaceabove=10pt]%
{bemerkungen}
\declaretheorem[name=Definition,parent=section,style=mainstyle_blue]{definition}
\declaretheorem[name=Definition \& Proposition,refname=Proposition,sharenumber=definition,style=mainstyle_blue]{definitionP}
\declaretheorem[name=Definition,numbered=no,style=mainstyle_unnumbered]{definition*}
\declaretheorem[name=Theorem,sharenumber=definition,style=mainstyle]{theorem}
\declaretheorem[name=Theorem,numbered=no,style=mainstyle_unnumbered]{theorem*}
\declaretheorem[name=Proposition,sharenumber=definition,style=mainstyle,refname=Proposition]{proposition}
\declaretheorem[name=Lemma,sharenumber=definition,style=mainstyle]{lemma}
\declaretheorem[name=Satz,sharenumber=definition,style=mainstyle,refname=Satz]{satz}
\declaretheorem[name=Satz,sharenumber=definition,style=mainstyle_unshaded]{satzUnshaded}
\declaretheorem[name=Definition,sharenumber=definition,style=mainstyle_unshaded]{definitionUnshaded}
\declaretheorem[name=Satz,numbered=no,style=mainstyle_unnumbered]{satz*}
\declaretheorem[name=Korollar,sharenumber=definition,style=mainstyle,refname=Korollar]{korollar}
\declaretheorem[name=Korollar,sharenumber=definition,style=mainstyleB,refname=Korollar]{korollarB}
\declaretheorem[name=Frage,numbered=no,style=mainstyle_unnumbered]{frage}
\declaretheorem[name=Frage,sharenumber=definition,style=mainstyle_unshaded]{frageA}
\declaretheorem[name=Erinnerung,sharenumber=definition,style=mainstyle_unshaded]{erinnerungA}
\declaretheorem[name=Ausblick,sharenumber=definition,style=mainstyle_unshaded]{ausblick}
\declaretheorem[name=Konvention,sharenumber=definition,style=mainstyle]{konvention}
\declaretheorem[name=Notation,sharenumber=definition,style=mainstyle_unshaded]{notation}
\declaretheorem[name=Bemerkung,sharenumber=definition,style=mainstyle_unshaded,refname=Bemerkung]{bemerkung}
\declaretheorem[name=Bemerkung,numbered=no,style=mainstyle_unnumbered]{bemerkung*}
\declaretheorem[name=Beispiel,sharenumber=definition,style=mainstyle_unshaded,refname=Beispiel]{beispiel}
\declaretheorem[name=Beispiel,numbered=no,style=mainstyle_unnumbered]{beispiel*}
\declaretheorem[name=Exkurs,numbered=no,style=mainstyle_unnumbered]{exkurs*}
\declaretheorem[name=Beweis,numbered=no,style=beweise]{beweis}
\declaretheorem[name=Übung,numbered=no,style=bemerkungen]{uebung}
\declaretheorem[name=Erinnerung,numbered=no,style=bemerkungen]{erinnerung}

% english versions
\declaretheorem[name=Remark,sharenumber=definition,style=mainstyle_unshaded]{remark}
\declaretheorem[name=Remark,numbered=no,style=mainstyle_unnumbered]{remark*}
\declaretheorem[name=Example,sharenumber=definition,style=mainstyle_unshaded]{example}
\declaretheorem[name=Corollary,sharenumber=definition,style=mainstyle]{corollary}
\let\proof\relax
\declaretheorem[name=Proof,numbered=no,style=beweise]{proof}
\declaretheorem[name=Sketch of Proof,numbered=no,style=beweisskizze]{sketch}
% ======================================================================================

%--Inhaltsverzeichnis
% ======================================================================================
\usepackage[tocindentauto]{tocstyle}
\usetocstyle{KOMAlike}
% ======================================================================================

%-- Dinge, die erst am Ende getan werden dürfen
% ======================================================================================
\shorthandon{"}
\usepackage{ellipsis}
% ======================================================================================


\newcommand{\fach}{Operatoralgebren \RM{2}.}
\newcommand{\semester}{Sose 2016}
\newcommand{\homepage}{http://wwwmath.uni-muenster.de/u/wilhelm.winter/wwinter/operatoralgebren_II.html}

\newcommand{\prof}{Prof.\ Dr.\ Wilhelm Winter}
\publishers{\scalebox{10}{\Huge$C^*$\Large\RM{2}.}}
\input{../!config/mitschrift_headings.tex}

\begin{document}
\pagenumbering{Roman}
\maketitle
\begin{abstract}
\section*{Aktuelle Version verfügbar bei}
\newcommand{\dieBreite}{11cm}
\begin{minipage}{4cm}
	\qrcode[height=3.3cm, version=6]{https://gitlab.com/JaMeZ-B/LaTeX-WWU}
\end{minipage}
\hfill
\begin{minipage}{\dieBreite}
	% \includegraphics[height=0.6cm, keepaspectratio]{../!config/Bilder/wm_no_bg.pdf}
	\includegraphics[height=0.8cm, keepaspectratio]{../!config/Bilder/wm_no_bg.pdf}\\
	\url{https://gitlab.com/JaMeZ-B/LaTeX-WWU} \smallskip\\
	Das zentrale Repository des \enquote{\LaTeX-WWU}-Projekts befindet sich auf der Plattform GitLab.com.
	Neben der Koordination aller Beteiligten werden über diesen Dienst auch die PDFs gebaut, die in der Readme verlinkt sind.
\end{minipage}\\[1cm]
\begin{minipage}{4cm}
	\qrcode[height=3.3cm, version=6]{https://github.com/JaMeZ-B/latex-wwu}
\end{minipage}
\hfill
\begin{minipage}{\dieBreite}
	\includegraphics[height=0.6cm, keepaspectratio]{../!config/Bilder/github_octo.pdf}
	\includegraphics[height=0.6cm, keepaspectratio]{../!config/Bilder/GitHub_Logo.pdf}\\
	\url{https://github.com/JaMeZ-B/latex-wwu} \smallskip\\
	Die Entwicklung des \enquote{\LaTeX-WWU}-Projekts hat ursprünglich auf GitHub stattgefunden, ist mittlerweile aber zu GitLab gewechselt.
	Das GitHub-Repository wird stündlich automatisch aktualisiert, Merge-Requests werden aber nicht mehr entgegengenommen.
\end{minipage}\\[1cm]
% \begin{minipage}{4cm}
% 	\qrcode[height=3.3cm, version=6]{https://uni-muenster.sciebo.de/public.php?service=files&t=965ae79080a473eb5b6d927d7d8b0462}
% \end{minipage}
% \hfill
% \begin{minipage}{\dieBreite}
% 	\raisebox{-2pt}{\includegraphics[height=0.6cm, keepaspectratio]{../!config/Bilder/sciebo_logo.pdf}}
% 	\resizebox{!}{0.5cm}{\large \sffamily\textbf{sciebo}} {\sffamily\large die Campuscloud} \\
% 	\resizebox{\dieBreite}{!}{\footnotesize\url{https://uni-muenster.sciebo.de/public.php?service=files&t=965ae79080a473eb5b6d927d7d8b0462}}\smallskip\\
% 	Sciebo ist ein Dropbox-Ersatz der Hochschulen in NRW, der von der Uni Münster in leitender Position auf Basis der OpenSource-Software Owncloud aufgebaut wurde.
% \end{minipage}\\[1cm]
\hrule \mbox{ }\\[0.7cm]
\begin{minipage}{4cm}
	\qrcode[height=3.3cm, version=6]{\homepage}
\end{minipage}
\hfill
\begin{minipage}{\dieBreite}
	\resizebox{!}{0.5cm}{\large\sffamily\textbf{Vorlesungshomepage}}\\
	\resizebox{\dieBreite}{!}{\footnotesize\url{\homepage}}\smallskip\\
	Hier ist ein Link zur offiziellen Vorlesungshomepage.
\end{minipage}
\newpage
\section*{Vorwort --- Mitarbeit am Skript}
Dieses Dokument ist eine Mitschrift aus der Vorlesung \enquote{\fach, \semester}, gelesen von \prof. 
Der Inhalt entspricht weitestgehend dem Tafelanschrieb. 
Für die Korrektheit des Inhalts übernehme ich keinerlei Garantie! 
Für Bemerkungen und Korrekturen -- und seien es nur Rechtschreibfehler -- bin ich sehr dankbar. 
Korrekturen lassen sich prinzipiell auf drei Wegen einreichen: 
\begin{itemize}
	\item Persönliches Ansprechen in der Uni, Mails an \hrefsymmail{mailto:\mail}{\mail} (gerne auch mit annotieren PDFs) oder Kommentare auf \url{https://gitlab.com/JaMeZ-B/LaTeX-WWU}.
	\item \emph{Direktes} Mitarbeiten am Skript: Den Quellcode poste ich auf GitLab (siehe oben), also stehen vielfältige Möglichkeiten der Zusammenarbeit zur Verfügung:
	Zum Beispiel durch Kommentare am Code über die Website und die Kombination Fork und Merge-Request. 
	Wer sich verdient macht oder ein Skript zu einer Vorlesung, die ich nicht besuche, beisteuern will, dem gewähre ich gerne auch Schreibzugriff.
	
	Beachten sollte man dabei, dass dazu ein Account bei \url{gitlab.com} notwendig ist, der allerdings ohne Angabe von persönlichen Daten angelegt werden kann. 
	Wer bei GitLab (bzw. dem zugrunde liegenden Open-Source-Programm \enquote{\texttt{git}}) -- verständlicherweise -- Hilfe beim Einstieg braucht, dem helfe ich gerne weiter. 
	Es gibt aber auch zahlreiche empfehlenswerte Tutorials im Internet.\footnote{zB. \url{https://try.github.io/levels/1/challenges/1}, ist auf Englisch, aber dafür interaktiv}
	\item \emph{Indirektes} Mitarbeiten: \TeX-Dateien per Mail verschicken. 
	
	Dies ist nur dann sinnvoll, wenn man einen ganzen Abschnitt ändern möchte (zB. einen alternativen Beweis geben), da ich die Änderungen dann per Hand einbauen muss! Ich freue mich aber auch über solche Beiträge!
\end{itemize}
\end{abstract}

\tableofcontents
\cleardoubleoddemptypage

\pagenumbering{arabic}
\setcounter{page}{1}
\setcounter{footnote}{0}

\section{Tensorprodukte} % (fold)
\label{sec:1}

\begin{erinnerungA}[{name=[Tensorprodukt]}]
	Seien $V,W$ $\mathbb{K}$-Vektorräume.
	Sei $X$ ein weiterer $\mathbb{K}$-Vektorraum und $\phi \colon V\times W \to X$ bilinear.
	Dann heißt $(X,\varphi)$ \Index{Tensorprodukt} von $V$ und $W$, falls folgende universelle Eigenschaft gilt:
	\[
		\begin{tikzcd}[sep=large]
			V \times W \drar["\text{bil.}"'] \rar["\varphi"] & X \dar[dashed,"\exists ! \psi \text{ lin.} "] \\
			& U
		\end{tikzcd}
	\]
\end{erinnerungA}

\begin{definitionP}[label=propdef:12]
	$(X,\varphi)$ wie oben existiert und ist eindeutig bis auf linearen Isomorphismus.
	Wir schreiben $V \odot W$ für $X$ und $v \otimes w$ für $\varphi(v,w)$.
\end{definitionP}
\begin{beweis}
	Sei $M$ der freie Vektorraum mit Basis $V \times W$, anders ausgedrückt
	\[
		M:= \set[\big]{f \colon V\times W \to \mathbb{K} \given \supp f \text{ endlich}}
	\]
	Sei nun 
	\[
		N := \Span \set*{
			\begin{array}{l}
				(v_1 +v_2,w) - \enbrace[\big]{(v_1,w) + (v_2,w)} \\
				(v,w_1+w_2) - \enbrace[\big]{(v,w_1) + (v,w_2)} \\
				(v,\lambda \cdot w) - \lambda \cdot (v,w), (\lambda \cdot v,w) - \lambda \cdot (v,w) 
			\end{array}
		\given 
		\begin{array}{l}
			v,v_1,v_2 \in V,\\
			w,w_1,w_2 \in W,\\
			\lambda \in \mathbb{K}
		\end{array}}
	\]
	Wir setzen $X := \sfrac{M}{N}$ und $\varphi(v,w) := (v,w)+N$.
	Der Rest ist eine einfache Übung.
\end{beweis}

\begin{proposition}[{name=[Tensorprodukt von Algebren]}]
	Seien $A,B$ $\mathbb{K}$-Algebren.
	Dann ist $A \odot B$ wieder eine $\mathbb{K}$-Algebra mit 
	\[
		(a \otimes b) (a' \otimes b') = aa' \otimes bb'
	\]
	Sind $A$ und $B$ $^*$-Algebren, dann ist $A \odot B$ wieder eine $^*$-Algebra mit 
	\[
		(a \otimes b)^* = a^* \otimes b^*
	\]
	Sind $A$ und $B$ kommutativ/unital, so auch $A \odot B$.
\end{proposition}
\begin{beweis}
	Betrachte $M$ wie in \autoref{propdef:12} und überprüfe, dass Multiplikation auf $\sfrac{M}{N}$ wohldefiniert und assoziativ ist.
	
	Alternativ: $(a,b) \in A \times B$ liefern eine bilineare Abbildung
	\mapdef{M_{(a,b)} \colon A \times B}{A \odot B}{(a',b')}{aa' \otimes bb'}{}
	Aus der universellen Eigenschaft erhält man $m_{(a,b)} \colon A \odot B \to A \odot B$ linear.
	Dies liefert wiederum eine bilineare Abbildung $m \colon A \times B \to \mathrm{L}(A \odot B, A \odot B)$ durch $(a,b) \mapsto m_{(a,b)}$.
	Nach der universellen Eigenschaft bekommt man nun eine lineare Abbildung $A \odot B \to \mathrm{L}(A \odot B, A\odot B)$.
	Daraus erhält man eine bilineare Abbildung
	\[
		\mu \colon A \odot B \times A \odot B \longrightarrow A \odot B
	\]
	Dann ist $(A \odot B,\mu)$ eine $\mathbb{K}$-Algebra.
\end{beweis}

\begin{bemerkung}[label=bem:14]
	\begin{enumerate}[(i)]
		\item Seien $A,B$ $^*$-Algebren.
		Dann hat das Tensorprodukt $A \odot B$ als $^*$-Algebra die folgende universelle Eigenschaft:
		
		Falls $C$ eine $^*$-Algebra ist und $\pi_A \colon A \to C$, $\pi_B \colon B \to C$ $^*$-Homomorphismen sind, sodass für den Kommutator $\benbrace*{\pi_A(A), \pi_B(B)}=0$ gilt, so existiert genau ein $^*$-Homomorphismus $\pi \colon A \odot B \to C$ mit $\pi(a \otimes b)= \pi_A(a) \pi_B(b)$ für alle $a \in A$, $b \in B$.
		\item Es seien $A,B,C,D$ $^*$-Algebren, $\pi_A \colon A \to C$, $\pi_B \colon B \to D$ $^*$-Homomorphismen.
		Dann existiert genau ein $\pi \colon A \odot B \to C \odot D$ mit $\pi(a \otimes b)= \pi_A(a) \otimes \pi_B(b)$.
	\end{enumerate}
\end{bemerkung}

\begin{definitionP}[{name=[{Tensorprodukt von Hilberträumen}]}]
	Es seien $\mathcal{H}_1$, $\mathcal{H}_2$ Hilberträume.
	Dann ist $\mathcal{H}_1 \odot \mathcal{H}_2$ ein Prä-Hilbertraum mit
	\[
		\skal*{\xi \otimes \eta}{\xi' \otimes \eta'} := \skal*{\xi}{\xi'} \cdot \skal*{\eta}{\eta'} 
	\]
	Dann ist das \bet{Tensorprodukt von Hilberträumen}\index{Tensorprodukt!von Hilberträumen} definiert durch
	\[
		\mathcal{H}_1 \otimes \mathcal{H}_2 := \overline{\mathcal{H}_1 \odot \mathcal{H}_2}^{\skal*{\cdot}{\cdot}}
	\]
\end{definitionP}
\begin{beweis}
	Es seien $\xi \in \mathcal{H}_1$ und $\eta \in \mathcal{H}_2$.
	Nach der universellen Eigenschaft existiert eine lineare Abbildung $\tau_{(\xi,\eta)} \colon \mathcal{H}_1 \odot \mathcal{H}_2 \to \mathbb{C}$ mit $\tau_{{\xi,\eta}} \enbrace*{\xi' \otimes \eta'} = \skal*{\xi}{\xi'} \cdot \skal*{\eta}{\eta'}$.
	Die Abbildung $(\xi,\eta) \mapsto \overline{\tau_{{\xi,\eta}}}$ ist dann bilinear.
	Aus der universellen Eigenschaft erhalten wir eine antilineare Abbildung $\xi \otimes \eta \mapsto \tau_{(\xi,\eta)}$.
	Daraus erhält man eine Sesquilinearform 
	\[
		\skal*{\cdot}{\cdot } \colon \mathcal{H}_1 \odot \mathcal{H}_2 \longrightarrow \mathbb{C}
	\]
	Wir müssen noch zeigen, dass diese positiv definit ist:
	Sei $z = \sum_{i=1}^{n} \xi_i \otimes \eta_i \in \mathcal{H}_1 \odot \mathcal{H}_2$.
	Wähle eine Orthonormalbasis $\zeta_1, \ldots , \zeta_m$ für $\Span \set*{\eta_1, \ldots,\eta_n}$.
	Dann ist $z=\sum_{j=1}^{m} \xi_j' \otimes \zeta_j$ für $\xi_j' \in \mathcal{H}_1$ geeignet.
	Damit folgt 
	\[
		\skal*{z}{z} = \sum_{j=1}^{m} \norm*{\xi'_j}^2 \ge 0
	\]
	Insbesondere ist $z=0$ genau dann, wenn $\xi_1' = \ldots = \xi_m'=0$, also $\skal*{z}{z}=0$ gilt.
\end{beweis}

\begin{bemerkung}[{name=[{Einbettung mit beschränkten Operatoren}]},label=bem:16]
	Es existiert eine Einbettung $\iota \colon \mathcal{B}(\mathcal{H}_1) \odot \mathcal{B}(\mathcal{H}_2) \hookrightarrow \mathcal{B}(\mathcal{H}_1 \otimes \mathcal{H}_2)$ gegeben durch $(a \otimes b)(\xi \otimes \eta) = a(\xi) \otimes b (\eta)$. (Warum?)
\end{bemerkung}

\begin{frageA}[{name=[{C*-Norm auf dem algebraischen Tensorprodukt}]}]
	Seien $A$, $B$ $C^*$-Algebren.
	\begin{center}
		¿ Was für $C^*$-Normen existieren auf $A \odot B$?
	\end{center}
\end{frageA}

\begin{bemerkung}
	Falls $\gamma \colon A \odot B \to \mathbb{R}^+$ eine $C^*$-Norm ist, so ist $A \otimes_\gamma B := \overline{A \odot B}^\gamma$ eine $C^*$-Algebra.
\end{bemerkung}

\begin{proposition}
	Es seien $A$, $B$ $C^*$-Algebren.
	\begin{enumerate}[(i)]
		\item Für $^*$-Darstellungen $\pi_A \colon A \to \mathcal{B}(\mathcal{H}_1)$, $\pi_B \colon B \to \mathcal{B}(\mathcal{H}_2)$ definiert 
		\[
			(\pi_A \otimes \pi_B)(a \otimes b) := \iota\enbrace[\big]{\pi_A(a) \otimes \pi_B(b)}
		\]
		eine $^*$-Darstellung $\pi_A \otimes \pi_B \colon A \odot B \to \mathcal{B}(\mathcal{H}_1 \otimes \mathcal{H}_2)$.
		Falls $\pi_A$ und $\pi_B$ treu sind, so ist auch $\pi_A \otimes \pi_B$ treu.
		\item Die Abbildung $\norm*{\cdot }_{\min} \colon A \odot B \to \mathbb{R}^+$, gegeben durch
		\[
			\norm[\big]{\textstyle\sum\nolimits_{i=1}^{n} a_i \otimes b_i}_{\min} := \sup \set[\Big]{\norm[\big]{(\pi_A \otimes \pi_B)\enbrace*{\textstyle\sum\nolimits_{i=1}^{n} a_i \otimes b_i}} \given \pi_A, \pi_B \text{ $^*$-Darstellungen} }
		\]
		ist eine $C^*$-Norm.
	\end{enumerate}
\end{proposition}
\begin{beweis}
	\begin{enumerate}[(i)]
		\item Nach \autoref{bem:14} und \autoref{bem:16} existiert genau eine $^*$-Darstellung 
		\[
			\pi_A \otimes \pi_B = \iota \circ (\pi_A \otimes \pi_B)
		\]
		Seien $\pi_A$ und $\pi_B$ treu und $0 \neq c \in \ker (\pi_A \otimes \pi_B)$.
		Es gilt $c= \sum_{i=1}^{n} a_i \otimes b_i$ für geeignete $a_i$, $b_i$ und $n$.
		Ohne Einschränkungen können wir annehmen, dass die $b_i$ linear unabhängig sind.
		Da $\pi_B$ treu ist, sind auch die $\pi_B(b_i)$ linear unabhängig.
	\end{enumerate}
\end{beweis}
% section 1 (end)

\cleardoubleoddemptypage
\pagenumbering{Alph}
\setcounter{page}{1}
\cleardoubleoddemptypage
\appendix

\section{Anhang} % (fold)
\label{sec:anhang}
%!TEX root = ana_top_geo.tex

\subsection{Ausführlicher Beweis zu \cref{lem:kpt-schnitte}} % (fold)
\label{sub:kpt-schnitte}
Sei $X$ ein Hausdorffraum. Dann ist $X$ genau dann kompakt, wenn gilt: Hat eine Familie $\mathcal{A}$ von abgeschlossenen Teilmengen von $X$ die endliche 
Durchschnittseigenschaft, so gilt 
\[
	\bigcap_{A \in \mathcal{A}} A \not= \emptyset.
\]
\begin{beweis}
	Für die erste Implikation sei $X$ kompakt und $\mathcal{A}$ eine Familie von abgeschlossenen Mengen mit der endlichen Durchschnittseigenschaft.
	Angenommen $\bigcap_{A \in \mathcal{A}} A = \emptyset$.
	Dann gilt
	\[
		X = X \setminus \bigcap_{A \in \mathcal{A}} A = \bigcup_{A \in \mathcal{A}} X \setminus A.
	\]
	Nun ist $\mathcal{U} \coloneqq \set*{X \setminus A \given A \in \mathcal{A}}$ eine offene Überdeckung von $X$ und da $X$ kompakt ist, existiert $\mathcal{A}_0 \subset \mathcal{A}$ endlich, sodass
	\[
		X = \bigcup_{A \in \mathcal{A}_0} X \setminus A = X \setminus \underbrace{\bigcap_{A \in \mathcal{A}_0 } A }_{\neq \emptyset} \quad \light
	\]
	Für die umgekehrte Implikation sei nun $\mathcal{U} = \set{U_i}_{i \in I}$ eine offene Überdeckung von $X$.
	Angenommen für jede endliche Teilmenge $J \subseteq I$ gilt $X \neq \bigcup_{i \in J} U_i$.
	Betrachte nun $\mathcal{A} =  \set{X \setminus U_i}_{i \in I}$. Dann gilt nach Annahme
	\[
		\bigcap_{i \in J} X \setminus U_i = X \setminus \bigcup_{i \in J} U_i \neq \emptyset.
	\]
	Also hat $\mathcal{A}$ die endliche Durchschnittseigenschaft. Nach Vorraussetzung gilt dann
	\[
		\emptyset \not= \bigcap_{i \in I} X \setminus U_i = X \setminus \underbrace{\bigcup_{i \in I} U_i}_{= X} \quad \light \qedhere
	\]
\end{beweis}


\subsection[Blatt3, Aufgabe 4: Hilfssatz für den Hauptsatz der Algebra]{Blatt 3, Aufgabe 4} % (fold)
\label{sub:B3A4}
\emph{Diese Übungsaufgabe ist zentral für den Beweis des Hauptsatzes der Algebra, \cref{satz:hauptsatz-algebra}.} 

Sei $p(x)= x^n + a_{n-1} x^{n-1} + \ldots + a_1 x + a_0$ mit $n \in \mathbb{N}_0$ ein Polynom mit Koeffizienten $a_i \in \mathbb{C}$, dass \emph{keine} Nullstelle in $\mathbb{C}$ besitzt. 
Sei $S^1= \set*{z \in \mathbb{C} \given \abs*{z}=1}$.
\begin{enumerate}[(a)]
	\item $f \colon S^1 \to S^1$ gegeben durch $f(z) = \frac{p(z)}{\abs*{p(z)} } $ ist wohldefiniert und homotop zu einer konstanten Abbildung.
	\item $f$ ist homotop zur Abbildung $g_n \colon S^1 \to S^1$ mit $g_n(z)= z^n$.
\end{enumerate}
\minisec{Beweis}
\begin{enumerate}[(a)]
	\item \begin{description}
		\item[Wohldefiniertheit:] Sei $z \in S^1$ beliebig. Dann gilt
		\[
			\abs*{\frac{p(z)}{\abs*{p(z)} } } = \frac{1}{\abs*{p(z)} } \cdot \abs*{p(z)} =1,
		\]
		also ist $f(z) \in S^1$.
		\item[Homotop zu einer konstanten Abbildung:] Definiere $f_t \colon S^1 \to S^1$ für $t \in [0,1]$ durch 
		\[
			f_t(z) = \frac{p(t \cdot z)}{\abs*{p(t \cdot z)} } 
		\]
		Dies ist mit der gleichen Begründung wie oben wohldefiniert. 
		Außerdem ist $f_0(z)= \frac{a_0}{\abs*{a_0} } \in S^1 $ konstant und $f_1(z)= \frac{p(z)}{\abs*{p(z)} }=f(z)$. 
		Definiere nun $H \colon S^1 \times [0,1] \to S^1$ durch $H(x,t) \coloneqq f_t(x)$. 
		Dann ist $H$ stetig, da Polynome und $\abs*{.} $, sowie Multiplikation stetig sind. 
		$H$ ist die gesuchte Homotopie.
	\end{description}
	\item Sei $h \colon S^1 \times [0,1] \to \mathbb{C}$ gegeben durch $h(z,t) = z^n + \sum_{k=0}^{n-1} a_k z^k t^{n-k}$. 
	Dann gilt $h(z,0)=z^n \not= 0$, da $z \in S^1$.
	Für $t \neq 0$ gilt nun
	\begin{align*}
		h(z,t) = 0 \iff \frac{h(z,t)}{t^n} = 0 \iff \frac{z^n}{t^n} + \sum_{k=0}^{n-1} a_k \frac{z^k}{t^k} = 0 \iff p \enbrace*{\frac{z}{t}} = 0
	\end{align*}
	Aber nach Vorraussetzung gilt $p \enbrace*{\frac{z}{t}} \neq 0$. 
	Also $h(z,t) \neq 0$ für alle $t \in [0,1]$. 
	Definiere nun $H \colon S^1 \times [0,1]\to S^1$ durch $H(z,t) = \frac{h(z,t)}{\abs*{h(z,t)}}$. 
	Wie eben gezeigt, ist dies wohldefiniert und offensichtlich stetig. Da
	\[
		H(z,0) = \frac{z^n}{\abs*{z^n} } = z^n \quad \text{ und } \quad H(z,1) = \frac{h(z,1)}{\abs*{h(z,1)} } = \frac{p(z)}{\abs*{p(z)} } =f(z)
	\]
	ist $H$ die gesuchte Homotopie. \qedhere
\end{enumerate}

\subsection{Blatt 10, Aufgabe 3} % (fold)
\label{sub:B10A3}
\emph{Diese Übungsaufgabe lieferte den Beweis zu \cref{prop:iso-covering}.} \smallskip \\
Sei $p \colon \overline{X} \to X$ eine Überlagerung. 
Seien $\overline{x}_0  \in \overline{X}$ und $x_0= p(\overline{x}_0 )$ Basispunkte. 
Dann ist die induzierte Abbildung $\pi_n (p) \colon \pi_n(\overline{X}, \overline{x}_0) \to \pi_n(X,x_0)$ ein Isomorphismus für alle $n \ge 2$.
\minisec{Beweis}
Als Überlagerung ist $p$ stetig, also ist $\pi_n(p)$ ein Gruppenhomomorphismus nach \hyperref[prop:eig-hom-gruppen:enum:4]{ \cref*{prop:eig-hom-gruppen} \ref*{prop:eig-hom-gruppen:enum:4}}.
\begin{description}
	\item[Surjektivität:] Sei $[\omega] \in \pi_n(X,x_0)$, also $\omega \colon I^n \to X$ mit $\omega(\partial I^n) = \set{x_0}$. Betrachte $\omega$ nun als Abbildung $I^{n-1} \times [0,1] \to X$:
	\[
		\begin{tikzcd}[column sep=4em]
			I^{n-1} \times \set{0} \dar[hook] \rar["\mathrm{const}_{\overline{x}_0}"] & \overline{X} \dar["p"]\\
			I^{n-1} \times I \rar["\omega"] & X  
		\end{tikzcd}
	\]
	$\mathrm{const}_{\overline{x}_0} \colon I^{n-1} \times \set{0}$ ist eine Hebung von $\omega\big|_{I^{n-1} \times \set{0}} \equiv x_0$. 
	Nach dem Homotopiehebungssatz (\ref{satz:hebung-homotopie}) existiert eine Hebung $\overline{\omega} \colon I^{n-1} \times I \to \overline{X}$ von $\omega$ mit $\overline{\omega}\big|_{I^{n-1} \times \set{0}} \equiv \overline{x}_0 $. 
	Also gilt
	\[
		p \circ \overline{\omega} \big|_{\partial I^n} = \omega \big|_{\partial I^n} \equiv x_0 \enspace \Longrightarrow \enspace \overline{\omega} \big|_{\partial I^n} 
		\in p ^{-1}( \set{x_0} ) .
	\]
	Da $p^{-1}(\set{x_0})$ diskret und $\partial I^n$ für $n \ge 2$ zusammenhängend ist, muss $\overline{\omega} \big|_{\partial I^n}$ konstant sein. 
	Da $\overline{\omega}\big|_{I^{n-1} \times \set{0}} \equiv \overline{x}_0 $ gilt, folgt somit $\overline{\omega}(\partial I^n) = \set{\overline{x}_0}$. 
	Also ist $[\overline{\omega}] \in \pi_n(\overline{X},\overline{x}_0)$ und weiter gilt
	\[
		\pi_n(p) \enbrace*{[\overline{\omega}]} = [p \circ \overline{\omega} ] = [\omega] \in \pi_n(X,x_0). 
	\]
	\item[Injektivität:] Sei $[\omega] \in \ker \pi_n(p)$, also $[p \circ \omega] = [c_{x_0}]$. 
	Es existiert also eine Homotopie $H$ relativ $\partial I^n$ zwischen $p \circ \omega$ und $c_{x_0}$. 
	Offensichtlich ist $\omega$ eine Hebung von $p \circ \omega$. 
	Mit dem Homotopiehebungssatz erhalten wir eine Hebung $\overline{H}$ von $H$ mit $\overline{H}(-,0) = \omega$. 
	Weiter wissen wir, dass
	\[
		\overline{H} \big|_{\partial I^n \times [0,1]} \in p ^{-1}(\set{x_0} ) \quad \text{ und }\quad  \overline{H} \big|_{ I^n \times \set{1}} \in p ^{-1}(\set{x_0} )
	\]
	gelten muss, da $H = p \circ \overline{H}$ und $H(-,1)= c_{x_0} \equiv x_0$. 
	Mit dem gleichen Argument wie oben folgt, dass $\overline{H} \big|_{\partial I^n \times [0,1]}$ und $\overline{H} \big|_{ I^n \times \set{1}}$ konstant sind. 
	Für $z \in \partial I^n$ gilt nun
	\[
		\overline{H}(z,0) = \omega(z) = \overline{x}_0
	\]
	Da $\partial I^n \times [0,1] \cap I^n \times \set{1} \not= \emptyset$, muss also auch $\overline{H}(-,1) \equiv \overline{x}_0$ gelten. 
	Damit folgt $[\omega] = [c_{x_0}]$.\qedhere
\end{description}
\printindex
\printbibliography
\listoffigures
\todototoc
\listoftodos[To-do's und andere Baustellen]
\end{document}
